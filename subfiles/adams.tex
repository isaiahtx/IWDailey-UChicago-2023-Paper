\documentclass[../main.tex]{subfiles}
\begin{document}

\subsection{Construction of the spectral sequence}

In the sections that follow, let $(E,\mu,e)$ be a monoid object and $X$ and $Y$ be objects in $\cSH$.

\begin{definition}[{\cite[Definition 11.3.1]{Rognes_SSeq}}]\label{defn:E-Adams_resolution}
	An \emph{$E$-Adams resolution of $Y$} $(Y_s,W_s;i,j,k)$ is a diagram of the form
	% https://q.uiver.app/#q=WzAsMTEsWzAsMCwiXFxjZG90cyJdLFsxLDAsIllfe3MrMX0iXSxbMiwwLCJZX3MiXSxbMywwLCJcXGNkb3RzIl0sWzQsMCwiWV8yIl0sWzUsMCwiWV8xIl0sWzYsMCwiWV8wIl0sWzYsMSwiV18wIl0sWzUsMSwiWV8xIl0sWzIsMSwiV19zIl0sWzEsMSwiV197cysxfSJdLFswLDFdLFsxLDIsImlfcyJdLFsyLDNdLFszLDRdLFs0LDUsImlfMSJdLFs1LDYsImlfMCJdLFs2LDcsImpfMCJdLFs1LDgsImpfMSJdLFsyLDksImpfcyJdLFsxLDEwLCJqX3tzKzF9IiwyXSxbNyw1LCJrXzAiLDEseyJzdHlsZSI6eyJib2R5Ijp7Im5hbWUiOiJkYXNoZWQifX19XSxbOCw0LCJrXzEiLDEseyJzdHlsZSI6eyJib2R5Ijp7Im5hbWUiOiJkYXNoZWQifX19XSxbOSwxLCJrX3MiLDEseyJzdHlsZSI6eyJib2R5Ijp7Im5hbWUiOiJkYXNoZWQifX19XV0=
	\[\begin{tikzcd}
		\cdots & {Y_{s+1}} & {Y_s} & \cdots & {Y_2} & {Y_1} & {Y_0} \\
		& {W_{s+1}} & {W_s} &&& {Y_1} & {W_0}
		\arrow[from=1-1, to=1-2]
		\arrow["{i_s}", from=1-2, to=1-3]
		\arrow[from=1-3, to=1-4]
		\arrow[from=1-4, to=1-5]
		\arrow["{i_1}", from=1-5, to=1-6]
		\arrow["{i_0}", from=1-6, to=1-7]
		\arrow["{j_0}", from=1-7, to=2-7]
		\arrow["{j_1}", from=1-6, to=2-6]
		\arrow["{j_s}", from=1-3, to=2-3]
		\arrow["{j_{s+1}}"', from=1-2, to=2-2]
		\arrow["{k_0}"{description}, dashed, from=2-7, to=1-6]
		\arrow["{k_1}"{description}, dashed, from=2-6, to=1-5]
		\arrow["{k_s}"{description}, dashed, from=2-3, to=1-2]
	\end{tikzcd}\]
	such that the dashed arrows really stand for (degree $-\1$) maps $k_s:W_s\to \Sigma Y_{s+1}$, and
	\begin{enumerate}
		\item There is an isomorphism $Y_0\cong Y$;
		\item for each $s$, the sequence
		\[Y_{s+1}\xr{i_s}Y_s\xr{j_s}W_s\xr{k_s}\Sigma Y_{s+1}\]
		is a distinguished triangle;
		\item $W_{s}$ is isomorphic to $E\otimes T_{s}$ for some object $T_{s}$ in $\cSH$;
		\item $E_*(i_s):E_*(Y_{s+1})\to E_*(Y_s)$ is zero.
	\end{enumerate}
\end{definition}

It turns out that every object $Y$ in $\cSH$ admits a \emph{canonical} $E$-Adams resolution:

\begin{definition}\label{defn:canonical_E-Adams_resolution}
	Let $\ol E$ be the fiber of the unit map $e:S\to E$ (\autoref{fiber}). Let $Y_0:=Y$ and $W_0:=E\otimes Y$. For $s>0$, define
	\[Y_s:=\ol E^s\otimes Y,\qquad W_s := E\otimes Y_s=E\otimes\ol E^s\otimes Y,\]
	where $\ol E^s$ denotes the $s$-fold tensor product $\ol E\otimes\cdots\otimes\ol E$. Then we get fiber sequences
	\[Y_{s+1}\xrightarrow{i_s}Y_s\xrightarrow{j_s} W_s\xrightarrow{k_s}\Sigma Y_{s+1}\]
	obtained by applying $-\otimes Y_s$ to the fiber sequence
	\[\ol E\to S\xrightarrow eE\to\Sigma\ol E.\]
	We can splice these sequences together to get the \emph{canonical Adams-resolution of $Y$}:
	% https://q.uiver.app/#q=WzAsOSxbMSwwLCJZXzMiXSxbMiwwLCJZXzIiXSxbMywwLCJZXzEiXSxbNCwwLCJZXzA9WSJdLFswLDAsIlxcY2RvdHMiXSxbMSwxLCJXXzMiXSxbMiwxLCJXXzIiXSxbMywxLCJXXzEiXSxbNCwxLCJXXzAiXSxbMCwxLCJpXzIiXSxbMSwyLCJpXzEiXSxbMiwzLCJpXzAiXSxbNCwwXSxbMCw1LCJqXzMiXSxbMSw2LCJqXzIiXSxbMiw3LCJqXzEiXSxbMyw4LCJqXzAiXSxbNiwwLCJrXzIiLDAseyJsYWJlbF9wb3NpdGlvbiI6MzAsInN0eWxlIjp7ImJvZHkiOnsibmFtZSI6ImRhc2hlZCJ9fX1dLFs3LDEsImtfMSIsMCx7ImxhYmVsX3Bvc2l0aW9uIjozMCwic3R5bGUiOnsiYm9keSI6eyJuYW1lIjoiZGFzaGVkIn19fV0sWzgsMiwia18wIiwwLHsibGFiZWxfcG9zaXRpb24iOjMwLCJzdHlsZSI6eyJib2R5Ijp7Im5hbWUiOiJkYXNoZWQifX19XV0=
	\[\begin{tikzcd}
		\cdots & {Y_3} & {Y_2} & {Y_1} & {Y_0=Y} \\
		& {W_3} & {W_2} & {W_1} & {W_0}
		\arrow["{i_2}", from=1-2, to=1-3]
		\arrow["{i_1}", from=1-3, to=1-4]
		\arrow["{i_0}", from=1-4, to=1-5]
		\arrow[from=1-1, to=1-2]
		\arrow["{j_3}", from=1-2, to=2-2]
		\arrow["{j_2}", from=1-3, to=2-3]
		\arrow["{j_1}", from=1-4, to=2-4]
		\arrow["{j_0}", from=1-5, to=2-5]
		\arrow["{k_2}"{pos=0.3}, dashed, from=2-3, to=1-2]
		\arrow["{k_1}"{pos=0.3}, dashed, from=2-4, to=1-3]
		\arrow["{k_0}"{pos=0.3}, dashed, from=2-5, to=1-4]
	\end{tikzcd}\]
\end{definition}

\begin{proposition}\label{canonical_E-Adams_resolution_is_an_Adams_resolution}
	The ``canonical $E$-Adams resolution of $Y$'' from \autoref{defn:canonical_E-Adams_resolution} is in fact an $E$-Adams resolution of $Y$, in the sense of \autoref{defn:E-Adams_resolution}
\end{proposition}
\begin{proof}
	By construction, the only thing we need to check is that $E_*(i_s):E_*(Y_{s+1})\to E_*(Y_s)$ is zero. First, note that since 
	\[Y_{s+1}\xr{i_s}Y_s\xr{j_s} W_s\xr{k_s}\Sigma Y_{s+1}\]
	is a distinguished triangle and $\cSH$ is tensor triangulated, there is a distinguished triangle of the form
	\[E\otimes Y_{s+1}\xr{E\otimes i_s}E\otimes Y_s\xr{E\otimes j_s}E\otimes W_s\to\Sigma(E\otimes Y_{s+1}).\]
	Thus, applying $\pi_*(-)\cong{[S,-]}_*$ to the triangle yields that the following sequence is exact (see \autoref{distinguished_tri_is_exact} for details):
	\[E_*(Y_{s+1})\xr{E_*(i_s)}E_*(Y_s)\xr{E_*(j_s)}E_*(W_s).\]
	Now, it is straightforward to verify by how it is constructed that $j_s$ is the map $e\otimes Y_s:Y_s\to E\otimes Y_s=W_s$. Thus, by unitality of $\mu$, we have that $E\otimes j_s:E\otimes Y_s\to E\otimes W_s$ is a split monomorphism, with right inverse $\mu\otimes Y_s:E\otimes W_s=E\otimes E\otimes Y_s\to E\otimes Y_s$. Then since any functor preserves split monomorphisms, it follows that $E_*(j_s)=\pi_*(E\otimes j_s)$ is likewise a split monomorphism, so that in particular $E_*(j_s)$ is injective. Thus $\imm E_*(i_s)=\ker E_*(j_s)=0$, so that $i_s$ is indeed the zero map, as desired.
\end{proof}

Now, by applying ${[X,-]}_*$ to an $E$-Adams resolution of $Y$, we get an associated unrolled exact couple, and thus a spectral sequence:

\begin{definition}\label{ASS}
	Suppose we have an $E$-Adams resolution of $Y$ (\autoref{defn:E-Adams_resolution}):
	% https://q.uiver.app/#q=WzAsOSxbMSwwLCJZXzMiXSxbMiwwLCJZXzIiXSxbMywwLCJZXzEiXSxbNCwwLCJZXzA9WSJdLFswLDAsIlxcY2RvdHMiXSxbMSwxLCJXXzMiXSxbMiwxLCJXXzIiXSxbMywxLCJXXzEiXSxbNCwxLCJXXzAiXSxbMCwxLCJpXzIiXSxbMSwyLCJpXzEiXSxbMiwzLCJpXzAiXSxbNCwwXSxbMCw1LCJqXzMiXSxbMSw2LCJqXzIiXSxbMiw3LCJqXzEiXSxbMyw4LCJqXzAiXSxbNiwwLCJrXzIiLDAseyJsYWJlbF9wb3NpdGlvbiI6MzAsInN0eWxlIjp7ImJvZHkiOnsibmFtZSI6ImRhc2hlZCJ9fX1dLFs3LDEsImtfMSIsMCx7ImxhYmVsX3Bvc2l0aW9uIjozMCwic3R5bGUiOnsiYm9keSI6eyJuYW1lIjoiZGFzaGVkIn19fV0sWzgsMiwia18wIiwwLHsibGFiZWxfcG9zaXRpb24iOjMwLCJzdHlsZSI6eyJib2R5Ijp7Im5hbWUiOiJkYXNoZWQifX19XV0=
	\[\begin{tikzcd}
		\cdots & {Y_3} & {Y_2} & {Y_1} & {Y_0=Y} \\
		& {W_3} & {W_2} & {W_1} & {W_0}
		\arrow["{i_2}", from=1-2, to=1-3]
		\arrow["{i_1}", from=1-3, to=1-4]
		\arrow["{i_0}", from=1-4, to=1-5]
		\arrow[from=1-1, to=1-2]
		\arrow["{j_3}", from=1-2, to=2-2]
		\arrow["{j_2}", from=1-3, to=2-3]
		\arrow["{j_1}", from=1-4, to=2-4]
		\arrow["{j_0}", from=1-5, to=2-5]
		\arrow["{k_2}"{pos=0.3}, dashed, from=2-3, to=1-2]
		\arrow["{k_1}"{pos=0.3}, dashed, from=2-4, to=1-3]
		\arrow["{k_0}"{pos=0.3}, dashed, from=2-5, to=1-4]
	\end{tikzcd}\]
	We can extend this diagram to the right by setting $Y_s=Y$, $W_s=0$, and $i_s=\id_Y$ for $s<0$. Then we may apply the functor ${[X,-]}_\ast$, and by \autoref{X,Y*_LES_compact}, we obtain the following $A$-graded unrolled exact couple (\autoref{unrolled_exact_couple}):
	% https://q.uiver.app/#q=WzAsMTAsWzAsMCwiXFxjZG90cyJdLFsxLDAsIntbWCxZX3tzKzJ9XX1fKiJdLFsyLDAsIntbWCxZX3tzKzF9XX1fKiJdLFszLDAsIntbWCxZX3tzfV19XyoiXSxbNCwwLCJ7W1gsWV97cy0xfV19XyoiXSxbNSwwLCJcXGNkb3RzIl0sWzIsMSwie1tYLFdfe3MrMX1dfV8qIl0sWzEsMSwie1tYLFdfe3MrMn1dfV8qIl0sWzQsMSwie1tYLFdfe3MtMX1dfV8qIl0sWzMsMSwie1tYLFdfe3N9XX1fKiJdLFswLDFdLFsxLDIsImlfe3MrMX0iXSxbMiwzLCJpX3MiXSxbMyw0LCJpX3tzLTF9Il0sWzQsNV0sWzIsNiwial97cysxfSJdLFs2LDEsIlxccGFydGlhbF97cysxfSIsMV0sWzEsNywial97cysyfSJdLFs0LDgsImpfe3MtMX0iXSxbOCwzLCJcXHBhcnRpYWxfe3MtMX0iLDFdLFszLDksImpfe3N9Il0sWzksMiwiXFxwYXJ0aWFsX3MiLDFdXQ==
	\[\begin{tikzcd}
		\cdots & {{[X,Y_{s+2}]}_*} & {{[X,Y_{s+1}]}_*} & {{[X,Y_{s}]}_*} & {{[X,Y_{s-1}]}_*} & \cdots \\
		& {{[X,W_{s+2}]}_*} & {{[X,W_{s+1}]}_*} & {{[X,W_{s}]}_*} & {{[X,W_{s-1}]}_*}
		\arrow[from=1-1, to=1-2]
		\arrow["{i_{s+1}}", from=1-2, to=1-3]
		\arrow["{i_s}", from=1-3, to=1-4]
		\arrow["{i_{s-1}}", from=1-4, to=1-5]
		\arrow[from=1-5, to=1-6]
		\arrow["{j_{s+1}}", from=1-3, to=2-3]
		\arrow["{\partial_{s+1}}"{description}, from=2-3, to=1-2]
		\arrow["{j_{s+2}}", from=1-2, to=2-2]
		\arrow["{j_{s-1}}", from=1-5, to=2-5]
		\arrow["{\partial_{s-1}}"{description}, from=2-5, to=1-4]
		\arrow["{j_{s}}", from=1-4, to=2-4]
		\arrow["{\partial_s}"{description}, from=2-4, to=1-3]
	\end{tikzcd}\]
	where here we are being abusive and writing $i_s:{[X,Y_{s+1}]}_*\to{[X,Y_{s}]}_*$ and $j_s:{[X,Y_s]}_*\to{[X,W_s]}_*$ to denote the pushforward maps induced by $i_s:Y_{s+1}\to Y_s$ and $j_s:Y_s\to W_s$, respectively. Each $i_s$, $j_s$, and $\partial_s$ are $A$-graded homomorphisms of degrees $0$, $0$, and $-\1$, respectively. 

	By \autoref{SSeq_assoc_to_unrolled_EC}, we may associate a $\bZ\times A$-graded spectral sequence $r\mapsto(E_r^{*,*}(X,Y),d_r)$ to the above $A$-graded unrolled exact couple, where $d_r$ has $\bZ\times A$-degree $(r,-\1)$. We call this spectral sequence the \emph{$E$-Adams spectral sequence for the computation of ${[X,Y]}_*$}.
\end{definition}

For those who would rather not lose themselves in the appendix, we give a brief unravelling of how \autoref{SSeq_assoc_to_unrolled_EC} applies to the present situation. Given some $s\in\bZ$ and some $r\geq1$, we may define the following $A$-graded subgroups of ${[X,W_s]}_*$:
\[Z_r^s:=\partial_s^{-1}(\imm[i^{(r-1)}:{[X,Y_{s+r}]}_*\to{[X,Y_{s+1}]}_*])\]
and
\[B_r^s:=j_s(\ker[i^{(r-1)}:{[X,Y_s]}_*\to{[X,Y_{s-r+1}]}_*]),\]
where we adopt the convention that $i^{(0)}$ is simply the identity. This yields an infinite sequence of inclusions
\[0=B_1^s\sseq B_2^s\sseq B_3^s\sseq\cdots\sseq\imm j_s=\ker\partial_s\sseq\cdots\sseq Z_3^s\sseq Z_2^s\sseq Z_1^s={[X,W_s]}_*.\]
Then for $r\geq1$, we define $E_r^s$ to be the $A$-graded quotient group
\[E_r^s:=Z_r^s/B_r^s.\]
Thus taking the direct sum of all the $E_r^s$'s yields the $r^\text{th}$ page of the spectral sequence
\[E_r:=\bigoplus_{s\in \bZ}E_r^s,\]
which is a $\bZ\times A$-graded abelian group. 

The differential $d_r:E_r\to E_r$ is a map of $\bZ\times A$-degree $(r,\1)$, and is constructed as follows: an element of $E_r^s=Z_r^s/B_r^s$ is a coset represented by some $x\in Z_r^s$, so that $\partial_s(x)=i^{(r-1)}(y)$ for some $y\in{[X,Y_{s+r}]}_*$. Then we define $d_r([x])$ to be the coset $[j_{s+r}(y)]$ in $Z_r^{s+r}/B_r^{s+r}$. 

In the case $r=1$, since $B_1^s=0$ and $Z_1^s={[X,W_s]}_*$, we have that $E_1^s={[X,W_s]}_*$, and given some $x\in E_1^s=[X,W_s]_*$, the differential $d_1$ is given by $d_1(x)=j_{s+1}(\partial_s(x))$, so that $d_1=j\circ\partial$. Furthermore, since the unrolled exact couple which yields the spectral sequence vanishes on its negative terms, we hav that $E_r^{s,a}(X,Y)=0$ for $s<0$.

In \Cref{subsection:unrolled_exact_couples}, it is shown in explicit detail that all of these definitions make sense and are well-defined. In particular, it is shown that the differentials are well-defined $A$-graded homomorphisms, that $d_r\circ d_r=0$, and that
\[\ker d_r^s/\imm d_r^s=\frac{Z_{r+1}^s/B_r^s}{B_{r+1}^s/B_r^s}\cong Z_{r+1}^s/B_{r+1}^s=E_{r+1}^s.\]

Note, we have called the above spectral sequence \emph{the} $E$-Adams spectral sequence for the computation of ${[X,Y]}_*$, even though it was constructed in terms of an $E$-Adams resolution for $Y$. We would like to show that, from the $E_2$-page onwards, that this spectral sequence is independent (up to isomorphism) of the chosen $E$-Adams resolution of $Y$. To start, we prove the following proposition:

\begin{proposition}{\cite[Proposition 11.4.1]{Rognes_SSeq}}\label{arrow_induces_homo_of_Adams_resolutions}
	Suppose we have $E$-Adams resolutions (\autoref{defn:E-Adams_resolution}) $(Y_s,W_s;i,j,k)$ and $(Y_s',W_s';i',j',k')$ of objects $Y$ and $Y'$ in $\cSH$, respectively. Then any arrow $f:Y\to Y'$ in $\cSH$ induces a homomorphism of unrolled exact couples $(Y_s,W_s;i,j,k)\to(Y_s',W_s';i',j',k')$ (\autoref{defn:sseq_homomorphism}), thus, an induced homomorphism of associated spectral sequences by \autoref{UEC_homo_induces_SSeq_homo}.
\end{proposition}
\begin{proof}
	First we need maps $f_s:Y_s\to Y_s'$ and $g_s:W_s\to W_s'$. To start with, define $f_0$ to be the composition 
	\[f_0:Y_0\cong Y\xr fY'\cong Y_0'.\] 
	Now, by induction, supposing $f_0,g_0,f_1,g_1,\ldots,f_{s-2},g_{s-2},f_{s-1}$ have been defined for some $s>0$, consider the following diagram:
	% https://q.uiver.app/#q=WzAsOSxbMCwwLCJZX3tzKzF9Il0sWzEsMSwiWV9zIl0sWzIsMSwiV19zIl0sWzMsMSwiXFxTaWdtYSBZX3tzKzF9Il0sWzQsMSwiXFxTaWdtYSBZX3MiXSxbMSwyLCJZX3MnIl0sWzIsMiwiV19zJyJdLFszLDIsIlxcU2lnbWEgWV97cysxfSciXSxbNCwyLCJcXFNpZ21hIFlfcyciXSxbMCwxLCJpX3MiXSxbMSwyLCJqX3MiXSxbMiwzLCJrX3MiXSxbMyw0LCItXFxTaWdtYSBpX3MiXSxbMSw1LCJmX3MiLDJdLFs1LDYsImpfcyciXSxbNiw3LCJrX3MnIl0sWzcsOCwiLVxcU2lnbWEgaV9zJyJdLFs0LDgsIlxcU2lnbWEgZl9zIl0sWzIsNiwiZ19zIiwwLHsic3R5bGUiOnsiYm9keSI6eyJuYW1lIjoiZGFzaGVkIn19fV0sWzMsNywiXFxTaWdtYSBmX3tzKzF9IiwwLHsic3R5bGUiOnsiYm9keSI6eyJuYW1lIjoiZGFzaGVkIn19fV1d
	\[\begin{tikzcd}
		{Y_{s+1}} \\
		& {Y_s} & {W_s} & {\Sigma Y_{s+1}} & {\Sigma Y_s} \\
		& {Y_s'} & {W_s'} & {\Sigma Y_{s+1}'} & {\Sigma Y_s'}
		\arrow["{i_s}", from=1-1, to=2-2]
		\arrow["{j_s}", from=2-2, to=2-3]
		\arrow["{k_s}", from=2-3, to=2-4]
		\arrow["{-\Sigma i_s}", from=2-4, to=2-5]
		\arrow["{f_s}"', from=2-2, to=3-2]
		\arrow["{j_s'}", from=3-2, to=3-3]
		\arrow["{k_s'}", from=3-3, to=3-4]
		\arrow["{-\Sigma i_s'}", from=3-4, to=3-5]
		\arrow["{\Sigma f_s}", from=2-5, to=3-5]
		\arrow["{g_s}", dashed, from=2-3, to=3-3]
		\arrow["{\Sigma f_{s+1}}", dashed, from=2-4, to=3-4]
	\end{tikzcd}\]
	Our goal is to construct the dashed arrows so that the diagram commutes.
\end{proof}

\begin{proposition}\label{E-iso_induces_iso_of_Adams}
	Suppose we have $E$-Adams resolutions (\autoref{defn:E-Adams_resolution}) $(Y_s,W_s;i,j,k)$ and $(Y_s',W_s';i',j',k')$ of objects $Y$ and $Y'$ in $\cSH$, respectively. Then given an arrow $f:Y\to Y'$ in $\cSH$ such that $E_*(f):E_*(Y)\to E_*(Y')$ is an isomorphism of $A$-graded abelian groups, the induced homomorphism of spectral sequences $(E_r(X,Y),d_r)\to(E_r(X,Y'),d_r)$ from \autoref{arrow_induces_homo_of_Adams_resolutions} is an isomorphism from the $E_2$-page onwards.

	In particular, the $E$-Adams spectral sequence for ${[X,Y]}_*$, from the $E_2$-page onwards, does not depend on the choice of $E$-Adams resolution for $Y$.
\end{proposition}
\begin{proof}
	\todo{todo}
\end{proof}

As a result of this proposition, we will simply say ``the $E$-Adams spectral sequence for ${[X,Y]}_*$'' to generally refer to any such spectral sequence induced by any $E$-Adams resolution of $Y$.

\subsection{The \texorpdfstring{$E_2$}{E2} page}

Now, we would like to characterize the $E_2$ page of the spectral sequence in terms of something more concrete. Namely, we will characterize the $E_2$ page in terms of $\Ext$ of comodules over the dual $E$-Steenrod algebra. For a quick review of $\Ext$ in an abelian category and derived functors, see \autoref{appendix:coalgebra}. The goal of this subsection will be to prove the following theorem:

\begin{theorem}
	Let $(E,\mu,e)$ be a commutative monoid object, and $X$ and $Y$ objects in $\cSH$. Suppose further that:\begin{itemize}
		\item $E$ is flat (\autoref{flat}) and cellular (\autoref{cellular}),
		\item $X$ is cellular and $E_*(X)$ is a graded projective left $\pi_*(E)$-module (via \autoref{module}),
		\item $Y$ is cellular.
	\end{itemize}
	Then the non-vanishing entries of the second page of the $E$-Adams spectral sequence for the computation of ${[X,Y]}_*$ (\autoref{ASS}) are the $\Ext$ groups of $A$-graded left comodules over the anticommutative Hopf algebroid structure on the dual $E$-Steenrod algebra (\autoref{dual_E-Steenrod_algebra_is_a_Hopf_algebroid_main}), i.e., we have the following isomorphisms for all $s\in\bN$ and $a\in A$:
	\[E_2^{s,a}(X,Y)\cong\Ext_{E_*(E)}^{s,a+\mbf s}(E_*(X),E_*(Y)):=\Ext_{E_*(E)}^s(E_{*}(X),E_{*+a+\mbf s}(Y)).\]
\end{theorem}
\begin{proof}
	As we have shown above in \autoref{E-iso_induces_iso_of_Adams}, from the $E_2$-page onwards, the $E$-Adams spectral sequence is indepdent of choice of $E$-Adams resolution of $Y$. Thus, in order to characterize the $E_2$ page as desired, we may assume we are working with the canonical $E$-Adams resolution $(Y_s,W_s;i,j,k)$ of $Y$ from \autoref{defn:canonical_E-Adams_resolution}.
	
	By \autoref{E_1_page_line_resolution_identification} below, for each $s\in\bN$ and $a\in A$, $E_2^{s,a}(X,Y)$ is isomorphic to the $s^\text{th}$ cohomology group of the cochain complex obtained by applying $F:=\Hom_{E_*(E)}^{a+\mbf s}(E_*(X),-)$ to the complex
	\[0\xr{\phantom{E_*(\delta_3)}}E_*(W_0)\xr{E_*(\delta_0)}E_*(\Sigma W_1)\xr{E_*(\delta_1)}E_*(\Sigma^2W_2)\xr{E_*(\delta_2)}E_*(\Sigma^3W_3)\xr{\phantom{E_*(\delta_3)}}\cdots.\]
	Furthermore, by \autoref{E_*W_s's_acyclic_resolution_of_E_*Y}, this complex is an $F$-acyclic resolution of $E_*(Y)$ (\autoref{defn:acyclic_resolution}). Thus, since the category of $E_*(E)$-comodules is an abelian category with enough injectives (\autoref{G-CoMod^A_is_abelian_if_eta_R_flat_and_has_enough_injectives}), we have by \autoref{acyclic_resolution_computes_R^nF} that
	\[E_2^{s,a}(X,Y)\cong R^s\Hom^{a+\mbf s}_{E_*(E)}(E_*(X),-)(E_*(Y))=\Ext^{s,a+\mbf s}(E_*(X),E_*(Y)),\]
	as desired.
\end{proof}

We leave it to the reader to unravel what the differential $d_2$ corresponds to under this identification.

\begin{definition}\label{nu^n_isos}
	Given some (nonnegative integer) $n\in\bN$, define natural isomorphisms $\nu^n_X:\Sigma^\n X\to\Sigma^nX$ inductively, by setting $\nu^0_X:=\lambda_X$, $\nu^1_X:=\nu_X^{-1}$, and supposing $\nu^{n-1}_X$ has been defined for some $n>1$, define $\nu^n_X$ to be the composition
	\[\nu^n_X:\Sigma^\n X=S^\n\otimes X\xr{\phi_{\n-\1,\1}\otimes X}S^{\n-\1}\otimes S^\1\otimes X\xr{S^{\n-\1}\otimes\nu_X^{-1}}S^{\n-\1}\Sigma X\xr{\nu_{\Sigma X}^{\n-\1}}\Sigma^{n}X.\]
	By induction, naturality of $\nu$, and functoriality of $-\otimes-$, these isomorphisms are clearly natural in $X$. 
\end{definition}

\begin{lemma}\label{W_s_cellular_if_E_and_Y_are}
	Let $(E,\mu,e)$ be a monoid object and $X$ and $Y$ objects in $\cSH$. Further suppose $E$ and $Y$ are cellular. Then for all $s\in\bZ$, the objects $Y_s$ and $W_s$ from the canonical $E$-Adams resolution of $Y$ (\autoref{defn:canonical_E-Adams_resolution}) are cellular.
\end{lemma}
\begin{proof}
	Unravelling definitions, for $s<0$, $W_s=0$ and $Y_s=Y$, which are both cellular.\footnote{$0$ is cellular because it is the cofiber of the identity on $S$ by axiom TR1 for a triangulated category (\autoref{triangulated_defn}), i.e., there is a distinguished triangle $S\to S\to 0\to\Sigma S$.} For $s\geq0$, we have $W_s=E\otimes Y_s$, so that by cellularity of $E$ and \autoref{cellular_closed_under_tensor}, it suffices to show that $Y_s$ is cellular for $s\geq0$. We know $Y_0=Y$ is cellular by definition. For $s>0$, $Y_s$ is the tensor product $\ol E^s\otimes Y$, where $\ol E$ fits into the distinguished triangle
	\[\ol E\to S\xr eE\to\Sigma\ol E.\]
	By the definition of cellularity, $\ol E$ is cellular since $S$ and $E$ are. Thus, by the aforementioned lemma, $\ol E^s\otimes Y$ is cellular by \autoref{cellular_closed_under_tensor}, as it is a tensor product of cellular objects in $\cSH$.
\end{proof}

\begin{lemma}\label{E_*W_s's_acyclic_resolution_of_E_*Y}
	Let $(E,\mu,e)$ be a flat (\autoref{flat}) and cellular (\autoref{cellular}) commutative monoid object and $X$ and $Y$ cellular objects in $\cSH$, and define $Y_s$, $W_s$ as in \autoref{defn:canonical_E-Adams_resolution}. In particular, for each $s\in\bZ$, we have distinguished triangles
	\[Y_{s+1}\xr{i_s}Y_s\xr{j_s}W_s\xr{k_s}\Sigma Y_{s+1}.\]
	Then if $E_*(X)$ is a graded projective (\autoref{graded_projective_module}) left $\pi_*(E)$-module (via \autoref{module}) then the sequence
	\[0\to E_*(Y)\xr{E_*(j_0)}E_*(W_0)\xr{E_*(\delta_0)}E_*(\Sigma W_1)\xr{E_*(\delta_1)}E_*(\Sigma^2W_2)\xr{E_*(\delta_2)}E_*(\Sigma^3W_3)\to\cdots\]
	is an $F$-acyclic resolution (\autoref{defn:acyclic_resolution}) of $E_*(Y)$ in $E_*(E)\text-\CoMod^A$ for 
	\[F=\Hom_{E_*(E)}^a(E_*(X),-)\]
	for all $a\in A$, where $\delta_s$ is the composition
	\[\Sigma^sW_s\xr{\Sigma^sk_s}\Sigma^{s+1}Y_{s+1}\xr{\Sigma^{s+1}j_{s+1}}\Sigma^{s+1}W_{s+1}.\]
\end{lemma}
\begin{proof}
	By \autoref{W_s_cellular_if_E_and_Y_are}, each $W_s$ is cellular, so that furthermore $\Sigma^sW_s\cong S^{\mbf s}\otimes W_s$ is cellular for each $s\geq0$, by \autoref{cellular_closed_under_tensor}. Thus, the sequence does indeed live in $E_*(E)\text-\CoMod^A$ by \autoref{E_*_functor_from_SH_to_E*E-comodules}, as desired. Next, we claim that $E_*(\Sigma^sW_s)$ is an $F$-acyclic object for each $s\geq0$, i.e., that 
	\[\Ext_{E_*(E)}^{n,a}(E_{*}(X),E_*(\Sigma^sW_s))=\Ext_{E_*(E)}^n(E_*(X),E_{*+a}(\Sigma^sW_s))=0\] 
	for all $n>0$, $s\geq0$, and $a\in A$. Note that we have an $A$-graded isomorphism of left $E_*(E)$-comodules:
	% https://q.uiver.app/#q=WzAsNyxbMCwwLCJFXyooRSlcXG90aW1lc197XFxwaV8qKEUpfUVfeyorYX0oXFxTaWdtYV5zWV9zKSJdLFsxLDAsIkVfKihFKVxcb3RpbWVzX3tcXHBpXyooRSl9RV97KithfShcXFNpZ21hXnNZX3MpIl0sWzEsMSwiRV8qKEVcXG90aW1lcyBcXFNpZ21hXnNZX3MpIl0sWzEsMiwiRV8qKEVcXG90aW1lcyBTXlxcbWF0aGJmIHNcXG90aW1lcyBZX3MpIl0sWzEsMywiRV8qKFNee1xcbWF0aGJmIHN9XFxvdGltZXMgRVxcb3RpbWVzIFlfcykiXSxbMSw0LCJFXyooXFxTaWdtYV5zKEVcXG90aW1lcyBZX3MpKSJdLFsyLDQsIkVfKihcXFNpZ21hXnNXX3MpIl0sWzAsMSwiIiwwLHsibGV2ZWwiOjIsInN0eWxlIjp7ImhlYWQiOnsibmFtZSI6Im5vbmUifX19XSxbMSwyLCJcXFBoaV97RSxcXFNpZ21hXnNZX3N9Il0sWzIsMywiRV8qKEVcXG90aW1lc3soXFxudV5zX3tZX3N9KX1eey0xfSkiXSxbMyw0LCJFXyooXFx0YXVcXG90aW1lcyBZX3MpIl0sWzQsNSwiRV8qKFxcbnVec197RVxcb3RpbWVzIFlfc30pIl0sWzUsNiwiIiwwLHsibGV2ZWwiOjIsInN0eWxlIjp7ImhlYWQiOnsibmFtZSI6Im5vbmUifX19XV0=
	\[\begin{tikzcd}
		{E_*(E)\otimes_{\pi_*(E)}E_{*+a}(\Sigma^sY_s)} & {E_*(E)\otimes_{\pi_*(E)}E_{*+a}(\Sigma^sY_s)} \\
		& {E_*(E\otimes \Sigma^sY_s)} \\
		& {E_*(E\otimes S^\mathbf s\otimes Y_s)} \\
		& {E_*(S^{\mathbf s}\otimes E\otimes Y_s)} \\
		& {E_*(\Sigma^s(E\otimes Y_s))} & {E_*(\Sigma^sW_s)}
		\arrow[Rightarrow, no head, from=1-1, to=1-2]
		\arrow["{\Phi_{E,\Sigma^sY_s}}", from=1-2, to=2-2]
		\arrow["{E_*(E\otimes{(\nu^s_{Y_s})}^{-1})}", from=2-2, to=3-2]
		\arrow["{E_*(\tau\otimes Y_s)}", from=3-2, to=4-2]
		\arrow["{E_*(\nu^s_{E\otimes Y_s})}", from=4-2, to=5-2]
		\arrow[Rightarrow, no head, from=5-2, to=5-3]
	\end{tikzcd}\]
	where $\Phi_{E,\Sigma^sY}$ is an $A$-graded isomorphism of abelian groups by \autoref{Kunneth_map_iso}, and furthermore an isomorphism of $E_*(E)$-comodules by \autoref{Phi_E,X_is_comodule_homo}. Every other arrow is an isomorphism of $E_*(E)$-comodules by functoriality of $E_*(-):\cSH\text-\Cell\to E_*(E)\text-\CoMod^A$. Thus, since $E_*(\Sigma^sW_s)$ is isomorphic to $E_*(E)\otimes_{\pi_*(E)}E_{*+a}(\Sigma^sY_s)$ in $E_*(E)\text-\CoMod^A$, and in particular since $\Ext^{n}_{E_*(E)}(E_*(X),-)$ is a functor, we have
	\[\Ext^{n}_{E_*(E)}(E_*(X),E_{*+a}(\Sigma^sW_s))\cong\Ext^{n}_{E_*(E)}(E_*(X),E_*(E)\otimes_{\pi_*(E)}E_{*+a}(\Sigma^sY_s)).\]
	Yet, $E_*(E)\otimes_{\pi_*(E)}E_{*+a}(\Sigma^sY_s)$ is a co-free $E_*(E)$-comodule, in which case since $E_*(X)$ is graded projective as an object in $\pi_*(E)\text-\Mod^A$, we have that 
	\[\Ext^{n,a}_{E_*(E)}(E_*(X),E_*(E)\otimes_{\pi_*(E)}E_{*+a}(\Sigma^sY_s))=0,\]
	by \autoref{co-free_comodules_are_hom(P,-)-acyclic}.

	Finally, it remains to show that the sequence is exact. To that end, first note that by induction on axiom TR4 for a triangulated category and the fact that distinguished triangles are exact (\autoref{distinguished_tri_is_exact}), the following sequence in $\cSH$ is exact (since a sequence clearly remains exact even after changing the signs of its maps):
	\[\Sigma^sY_s\xr{\Sigma^sj_s}\Sigma^sW_s\xr{\Sigma^sk_s}\Sigma^{s+1}Y_{s+1}\xr{\Sigma^{s+1}i_s}\Sigma^{s+1}Y_s\xr{\Sigma^{s+1}j_s}\Sigma^{s+1}W_s\]
	(see \autoref{defn_exact} for the definition of an exact triangle in an additive category). Furthermore, since $\cSH$ is tensor triangulated, the sequence remains exact after applying $E\otimes-$ (see \autoref{LES_remains_exact_after_tensor} for details), so that taking $E$-homology yields the following exact sequence of homology groups:
	\[E_*(\Sigma^sY_{s+1})\xr{E_*(\Sigma^si_s)}E_*(\Sigma^sY_s)\xr{E_*(\Sigma^sj_s)}E_*(\Sigma^sW_s)\xr{E_*(\Sigma^sk_s)}E_*(\Sigma^{s+1}Y_{s+1})\xr{E_*(\Sigma^{s+1}i_s)}E_*(\Sigma^{s+1}Y_s).\]
	Then since $E_*(i_s):E_*(Y_{s+1})\to E_*(Y_s)$ is the zero map (by \autoref{canonical_E-Adams_resolution_is_an_Adams_resolution}) and we have natural isomorphisms
	\[E_*(\Sigma^tX)\xr{\nu^t_X}E_*(\Sigma^\mbf tX)\xr{t^\mbf t_X}E_{*-\mbf t}(X)\]
	(the first from \autoref{nu^n_isos} and the latter from \autoref{E_homology_suspension_iso_t^a's}), we have that $E_*(\Sigma^ti_s):E_*(\Sigma^tY_{s+1})\to E_*(\Sigma^tY_s)$ is the zero map for all $t\in\bZ$, so that in particular the above exact sequence splits to yield the short exact sequence
	\[0\to E_*(\Sigma^sY_s)\xr{E_*(\Sigma^sj_s)}E_*(\Sigma^sW_s)\xr{E_*(\Sigma^sk_s)}E_*(\Sigma^{s+1}Y_{s+1})\to0.\]
	Then we may splice these sequences together for $s\geq0$ to yield the following diagram:
	% https://q.uiver.app/#q=WzAsOCxbMCwwLCIwIl0sWzEsMCwiRV8qKFkpIl0sWzIsMCwiRV8qKFdfMCkiXSxbNCwwLCJFXyooXFxTaWdtYSBXXzEpIl0sWzYsMCwiRV8qKFxcU2lnbWFeMldfMikiXSxbNywwLCJcXGNkb3RzIl0sWzMsMSwiRV8qKFxcU2lnbWEgWV8xKSJdLFs1LDEsIkVfKihcXFNpZ21hXjIgWV8yKSJdLFswLDFdLFsxLDIsIkVfKihqXzApIl0sWzIsMywiRV8qKFxcZGVsdGFfMCkiXSxbMyw0LCJFXyooXFxkZWx0YV8xKSJdLFs0LDVdLFsyLDYsIkVfKihrXzApIiwxXSxbNiwzLCJFXyooXFxTaWdtYSBqXzEpIiwxXSxbMyw3LCJFXyooXFxTaWdtYSBrXzEpIiwxXSxbNyw0LCJFXyooXFxTaWdtYV4yal8yKSIsMV1d
	\[\begin{tikzcd}
		0 & {E_*(Y)} & {E_*(W_0)} && {E_*(\Sigma W_1)} && {E_*(\Sigma^2W_2)} & \cdots \\
		&&& {E_*(\Sigma Y_1)} && {E_*(\Sigma^2 Y_2)}
		\arrow[from=1-1, to=1-2]
		\arrow["{E_*(j_0)}", from=1-2, to=1-3]
		\arrow["{E_*(\delta_0)}", from=1-3, to=1-5]
		\arrow["{E_*(\delta_1)}", from=1-5, to=1-7]
		\arrow[from=1-7, to=1-8]
		\arrow["{E_*(k_0)}"{description}, from=1-3, to=2-4]
		\arrow["{E_*(\Sigma j_1)}"{description}, from=2-4, to=1-5]
		\arrow["{E_*(\Sigma k_1)}"{description}, from=1-5, to=2-6]
		\arrow["{E_*(\Sigma^2j_2)}"{description}, from=2-6, to=1-7]
	\end{tikzcd}\]
	It follows the top row is exact, as desired.
\end{proof}

\begin{lemma}\label{X,EY_iso_Hom(E_*X,E_*EY)}
	Let $(E,\mu,e)$ be a commutative monoid object, and $X$ and $Y$ objects in $\cSH$. Suppose further that:\begin{itemize}
		\item $E$ is flat (\autoref{flat}) and cellular (\autoref{cellular}),
		\item $X$ is cellular and $E_*(X)$ is a graded projective left $\pi_*(E)$-module (via \autoref{module}), and
		\item $Y$ is cellular.
	\end{itemize}
	Then the assignment
	\[E_*(-):{[X,E\otimes Y]}\to\Hom_{E_*(E)}(E_*(X),E_*(E\otimes Y)),\qquad f\mapsto E_*(f)\]
	induced by the functor $E_*(-):\cSH\text-\Cell\to E_*(E)\text-\CoMod^A$ is an isomorphism of abelian groups.
\end{lemma}
\begin{proof}
	Since $X$ is cellular, by \autoref{E_*_functor_from_SH_to_E*E-comodules} we have that $E_*(X)$ is canonically an $A$-graded left $E_*(E)$-comodule. Similarly, since $E$ and $Y$ are cellular, we know that $E\otimes Y$ is cellular, so that $E_*(E\otimes Y)$ is also canonically an $E_*(E)$-comodule. Thus, we have a well-defined assignment
	\[[X,E\otimes Y]\xr{E_*(-)}\Hom_{E_*(E)}(E_*(X),E_*(E\otimes Y)).\]
	To see this arrow is an isomorphism, consider the following diagram:
	% https://q.uiver.app/#q=WzAsNCxbMCwwLCJbWCxFXFxvdGltZXMgWV0iXSxbMiwwLCJcXG1hdGhybXtIb219X3tFXyooRSl9KEVfKihYKSxFXyooRVxcb3RpbWVzIFkpKSJdLFsyLDIsIlxcbWF0aHJte0hvbX1fe0VfKihFKX0oRV8qKFgpLEVfKihFKVxcb3RpbWVzX3tcXHBpXyooRSl9IEVfKihZKSkiXSxbMCwyLCJcXG1hdGhybXtIb219X3tcXHBpXyooRSl9KEVfKihYKSxFXyooWSkpIl0sWzAsMSwiRV8qKC0pIl0sWzIsMSwieyhcXFBoaV97RSxZfSl9XyoiLDJdLFswLDMsIlxccGlfKihcXG11XFxvdGltZXMgWSlcXGNpcmMgRV8qKC0pIiwyXSxbMSwzLCJcXHBpXyooXFxtdVxcb3RpbWVzIFkpXFxjaXJjKC0pIiwxXSxbMiwzLCJcXHRleHR7YWRqfSIsMl1d
	\[\begin{tikzcd}
		{[X,E\otimes Y]} && {\mathrm{Hom}_{E_*(E)}(E_*(X),E_*(E\otimes Y))} \\
		\\
		{\mathrm{Hom}_{\pi_*(E)}(E_*(X),E_*(Y))} && {\mathrm{Hom}_{E_*(E)}(E_*(X),E_*(E)\otimes_{\pi_*(E)} E_*(Y))}
		\arrow["{E_*(-)}", from=1-1, to=1-3]
		\arrow["{{(\Phi_{E,Y})}_*}"', from=3-3, to=1-3]
		\arrow["{\pi_*(\mu\otimes Y)\circ E_*(-)}"', from=1-1, to=3-1]
		\arrow["{\pi_*(\mu\otimes Y)\circ(-)}"{description}, from=1-3, to=3-1]
		\arrow["{\text{adj}}"', from=3-3, to=3-1]
	\end{tikzcd}\]
	We know the left vertical map is an isomorphism by \autoref{theorem:UCT}, and the bottom horizontal isomorphism is the forgetful-cofree adjunction (\autoref{comodule_co-free_adjunction}) for $A$-graded left comodules over the dual $E$-Steenrod algebra. The right vertical arrow is a well-defined isomorphism, as $\Phi_{E,Y}$ is a homomorphism of $A$-graded left $E_*(E)$-comodules (\autoref{Phi_E,X_is_comodule_homo}), and in fact it is an isomorphism by \autoref{Kunneth_map_iso}, since $E_*(E)$ is flat and $Y$ is cellular. Thus in order to see the top arrow is an isomorphism, it suffices to show that the diagram commutes. The left triangle clearly commutes; to see the right triangle commutes, recall that by how the how forgetful-cofree adjunction for left comodules over a Hopf algebroid is defined, that the bottom vertical arrow sends an $A$-graded homomorphism of left $E_*(E)$-comodules $\psi:E_{*}(X)\to E_*(E)\otimes_{\pi_*(E)}E_*(Y)$ to the composition
	\[E_{*}(X)\xr\psi E_*(E)\otimes_{\pi_*(E)}E_*(Y)\xr{\pi_*(\mu)\otimes E_*(Y)}\pi_*(E)\otimes_{\pi_*(E)}E_*(Y)\xr\cong E_*(Y).\]
	Thus, in order to show that this composition equals $\pi_*(\mu\otimes Y)\circ\Phi_{E,Y}\circ\psi$, it suffices to show the following diagram commutes:
	% https://q.uiver.app/#q=WzAsNCxbMCwwLCJFXyooRSlcXG90aW1lc197XFxwaV8qKEUpfUVfKihZKSJdLFsyLDAsIlxccGlfKihFKVxcb3RpbWVzX3tcXHBpXyooRSl9RV8qKFkpIl0sWzIsMiwiRV8qKFkpIl0sWzAsMiwiRV8qKEVcXG90aW1lcyBZKSJdLFswLDEsIlxccGlfKihcXG11KVxcb3RpbWVzIEVfKihZKSJdLFsxLDIsIlxcY29uZyJdLFswLDMsIlxcUGhpX3tFLFl9IiwyXSxbMywyLCJcXHBpXyooXFxtdVxcb3RpbWVzIFkpIl1d
	\[\begin{tikzcd}
		{E_*(E)\otimes_{\pi_*(E)}E_*(Y)} && {\pi_*(E)\otimes_{\pi_*(E)}E_*(Y)} \\
		\\
		{E_*(E\otimes Y)} && {E_*(Y)}
		\arrow["{\pi_*(\mu)\otimes E_*(Y)}", from=1-1, to=1-3]
		\arrow["\cong", from=1-3, to=3-3]
		\arrow["{\Phi_{E,Y}}"', from=1-1, to=3-1]
		\arrow["{\pi_*(\mu\otimes Y)}", from=3-1, to=3-3]
	\end{tikzcd}\]
	Since all the arrows here are homomorphisms of abelian groups, in order to show the diagram commutes, it suffices to chase pure homogeneous tensors around. To that end, let $x:S^a\to E\otimes E$ and $y:S^b\to E\otimes Y$, and consider the following diagram exhibiting the two ways to chase $x\otimes y$ around:
	% https://q.uiver.app/#q=WzAsNixbMCwwLCJTXnthK2J9Il0sWzEsMCwiU15hXFxvdGltZXMgU15iIl0sWzIsMCwiRVxcb3RpbWVzIEVcXG90aW1lcyBFXFxvdGltZXMgWSJdLFszLDAsIkVcXG90aW1lcyBFXFxvdGltZXMgWSJdLFszLDEsIkVcXG90aW1lcyBZIl0sWzIsMSwiRVxcb3RpbWVzIEVcXG90aW1lcyBZIl0sWzAsMSwiXFxwaGlfe2EsYn0iXSxbMSwyLCJ4XFxvdGltZXMgeSJdLFsyLDMsIlxcbXVcXG90aW1lcyBFXFxvdGltZXMgWSJdLFszLDQsIlxcbXVcXG90aW1lcyBZIl0sWzIsNSwiRVxcb3RpbWVzIFxcbXVcXG90aW1lcyBZIiwyXSxbNSw0LCJcXG11XFxvdGltZXMgWSJdXQ==
	\[\begin{tikzcd}
		{S^{a+b}} & {S^a\otimes S^b} & {E\otimes E\otimes E\otimes Y} & {E\otimes E\otimes Y} \\
		&& {E\otimes E\otimes Y} & {E\otimes Y}
		\arrow["{\phi_{a,b}}", from=1-1, to=1-2]
		\arrow["{x\otimes y}", from=1-2, to=1-3]
		\arrow["{\mu\otimes E\otimes Y}", from=1-3, to=1-4]
		\arrow["{\mu\otimes Y}", from=1-4, to=2-4]
		\arrow["{E\otimes \mu\otimes Y}"', from=1-3, to=2-3]
		\arrow["{\mu\otimes Y}", from=2-3, to=2-4]
	\end{tikzcd}\]
	The diagram commutes by associtiavity of $\mu$. Thus, we have indeed show that
	\[E_*(-):[X,E\otimes Y]\to\Hom_{E_*(E)}(E_*(X),E_*(Y))\]
	is an isomorphism of abelian groups.
\end{proof}

\begin{proposition}\label{E_1_page_line_resolution_identification}
	Let $(E,\mu,e)$ be a commutative monoid object, and $X$ and $Y$ objects in $\cSH$. Suppose further that:\begin{itemize}
		\item $E$ is flat (\autoref{flat}) and cellular (\autoref{cellular}),
		\item $X$ is cellular, and $E_*(X)$ is a graded projective left $\pi_*(E)$-module (via \autoref{module}), and
		\item $Y$ is cellular.
	\end{itemize}
	Then for all $s\in\bZ$ and $a\in A$, the line in the first page of the $E$-Adams spectral sequence for the computation of ${[X,Y]}_*$ associated to the canonical $E$-Adams resolution of $Y$ (\autoref{defn:canonical_E-Adams_resolution})
	\[0\to E_1^{0,a+\mbf s}(X,Y)\xr{d_1}E_1^{1,a+\mbf s-\1}(X,Y)\xr{d_1}E_1^{2,a+\mbf s-\mbf 2}(X,Y)\to\cdots\to E_1^{s,a}(X,Y)\to\cdots\]
	is isomorphic to the complex obtained by applying $\Hom_{E_*(E)}^{a+\mbf s}(E_*(X),-)$ to the complex of $A$-graded left $E_*(E)$-comodules
	\[0\to E_*(W_0)\xr{E_*(\delta_0)}E_*(\Sigma W_1)\xr{E_*(\delta_1)}E_*(\Sigma^2W_2)\to\cdots\to E_*(\Sigma^sW_s)\to\cdots\]
	from \autoref{E_*W_s's_acyclic_resolution_of_E_*Y}.
\end{proposition}
\begin{proof}
	By \autoref{W_s_cellular_if_E_and_Y_are}, since $E$ and $Y$ are cellular, $W_t$ is as well for each $t\in\bN$. Furthermore, for $t>0$, we have isomorphisms
	\[S^{\mbf t}\otimes W_t\xr{\nu^t_{W_t}}\Sigma^tW_t,\]
	and by \autoref{cellular_closed_under_tensor}, the object $S^{\mbf t}\otimes W_t$ is cellular since $S^{\mbf t}$ and $W_t$ are. Hence, by \autoref{E_*_functor_from_SH_to_E*E-comodules}, the complex
	\[0\to E_*(W_0)\xr{E_*(\delta_0)}E_*(\Sigma W_1)\xr{E_*(\delta_1)}E_*(\Sigma^2W_2)\to\cdots\to E_*(\Sigma^sW_s)\to\cdots\]
	actually lives in $E_*(E)\text-\CoMod^A$, as desired. Now, let $t\in\bN$, and consider the following diagram:
	% https://q.uiver.app/#q=WzAsMTMsWzAsMCwie1tYLFdfdF19X3thK1xcbWJmIHMtXFxtYmYgdH0iXSxbMCw0LCJ7W1gsV197dCsxfV19X3thK1xcbWJmIHMtXFxtYmYgdC1cXDF9Il0sWzEsMCwie1tYLFxcU2lnbWFee1xcbWJmIHR9V190XX1fe2ErXFxtYmYgc30iXSxbMSw0LCJ7W1gsXFxTaWdtYV57XFxtYmYgdCtcXDF9V197dCsxfV19X3thK1xcbWJmIHN9Il0sWzIsMCwie1tYLFxcU2lnbWFedFdfdF19X3thK1xcbWJmIHN9Il0sWzIsNCwie1tYLFxcU2lnbWFee3QrMX1XX3t0KzF9XX1fe2ErXFxtYmYgc30iXSxbMCwxLCJ7W1gsXFxTaWdtYSBZX3t0KzF9XX1fe2ErXFxtYmYgcy1cXG1iZiB0fSJdLFswLDIsIntbWCxcXFNpZ21hXlxcMSBZX3t0KzF9XX1fe2ErXFxtYmYgcy1cXG1iZiB0fSJdLFswLDMsIntbWCxZX3t0KzF9XX1fe2ErXFxtYmYgcy1cXG1iZiB0LVxcMX0iXSxbMSwxLCJ7W1gsXFxTaWdtYV57XFxtYmYgdH1cXFNpZ21hIFlfe3QrMX1dfV97YStcXG1iZiBzfSJdLFsxLDIsIntbWCxcXFNpZ21hXntcXG1iZiB0fVxcU2lnbWFeXFwxIFlfe3QrMX1dfV97YStcXG1iZiBzfSJdLFsxLDMsIntbWCxcXFNpZ21hXntcXG1iZiB0K1xcMX0gWV97dCsxfV19X3thK1xcbWJmIHN9Il0sWzIsMiwie1tYLFxcU2lnbWFee3QrMX1ZX3t0KzF9XX1fe2ErXFxtYmYgc30iXSxbMiwwLCJzXntcXG1iZiB0fV97WCxXX3R9IiwyXSxbMywxLCJzXntcXG1iZiB0K1xcMX1fe1gsV197dCsxfX0iLDJdLFsyLDQsInsoXFxudV50X3tXX3R9KX1fKiJdLFszLDUsInsoXFxudV57dCsxfV97V197dCsxfX0pfV8qIl0sWzAsNiwieyhrX3QpfV8qIiwyXSxbNiw3LCJ7KFxcbnVfe1lfe3QrMX19KX1fKiIsMl0sWzcsOCwic15cXDFfe1gsWV97dCsxfX0iLDJdLFs4LDEsInsoal97dCsxfSl9XyoiLDJdLFsyLDksInsoXFxTaWdtYV57XFxtYmYgdH1rX3QpfV8qIiwxXSxbOSwxMCwieyhcXFNpZ21hXntcXG1iZiB0fVxcbnVfe1lfe3QrMX19KX1fKiIsMV0sWzEwLDcsInNee1xcbWJmIHR9X3tYLFxcU2lnbWFeXFwxWV97dCsxfX0iLDJdLFs5LDYsInNee1xcbWJmIHR9X3tYLFxcU2lnbWEgWV97dCsxfX0iLDJdLFsxMSwxMCwieyhcXHBoaV97XFxtYmYgdCxcXDF9XFxvdGltZXMgWV97dCsxfSl9XyoiLDFdLFsxMSw4LCJzXntcXG1iZiB0KzF9X3tYLFlfe3QrMX19IiwyXSxbMTEsMywieyhcXFNpZ21hXntcXG1iZiB0K1xcMX1qX3t0KzF9KX1fKiIsMV0sWzQsMTIsInsoXFxTaWdtYV50IGtfdCl9XyoiLDFdLFsxMiw1LCJ7KFxcU2lnbWFee3QrMX0gal97dCsxfSl9XyoiLDFdLFsxMSwxMiwieyhcXG51Xnt0KzF9X3tZX3t0KzF9fSl9XyoiLDJdLFs5LDEyLCJ7KFxcbnVee3R9X3tcXFNpZ21hIFlfe3QrMX19KX1fKiJdLFs0LDUsInsoXFxkZWx0YV90KX1fKiIsMCx7Im9mZnNldCI6LTUsImN1cnZlIjotNX1dXQ==
	\[\begin{tikzcd}
		{{[X,W_t]}_{a+\mbf s-\mbf t}} & {{[X,\Sigma^{\mbf t}W_t]}_{a+\mbf s}} & {{[X,\Sigma^tW_t]}_{a+\mbf s}} \\
		{{[X,\Sigma Y_{t+1}]}_{a+\mbf s-\mbf t}} & {{[X,\Sigma^{\mbf t}\Sigma Y_{t+1}]}_{a+\mbf s}} \\
		{{[X,\Sigma^\1 Y_{t+1}]}_{a+\mbf s-\mbf t}} & {{[X,\Sigma^{\mbf t}\Sigma^\1 Y_{t+1}]}_{a+\mbf s}} & {{[X,\Sigma^{t+1}Y_{t+1}]}_{a+\mbf s}} \\
		{{[X,Y_{t+1}]}_{a+\mbf s-\mbf t-\1}} & {{[X,\Sigma^{\mbf t+\1} Y_{t+1}]}_{a+\mbf s}} \\
		{{[X,W_{t+1}]}_{a+\mbf s-\mbf t-\1}} & {{[X,\Sigma^{\mbf t+\1}W_{t+1}]}_{a+\mbf s}} & {{[X,\Sigma^{t+1}W_{t+1}]}_{a+\mbf s}}
		\arrow["{s^{\mbf t}_{X,W_t}}"', from=1-2, to=1-1]
		\arrow["{s^{\mbf t+\1}_{X,W_{t+1}}}"', from=5-2, to=5-1]
		\arrow["{{(\nu^t_{W_t})}_*}", from=1-2, to=1-3]
		\arrow["{{(\nu^{t+1}_{W_{t+1}})}_*}", from=5-2, to=5-3]
		\arrow["{{(k_t)}_*}"', from=1-1, to=2-1]
		\arrow["{{(\nu_{Y_{t+1}})}_*}"', from=2-1, to=3-1]
		\arrow["{s^\1_{X,Y_{t+1}}}"', from=3-1, to=4-1]
		\arrow["{{(j_{t+1})}_*}"', from=4-1, to=5-1]
		\arrow["{{(\Sigma^{\mbf t}k_t)}_*}"{description}, from=1-2, to=2-2]
		\arrow["{{(\Sigma^{\mbf t}\nu_{Y_{t+1}})}_*}"{description}, from=2-2, to=3-2]
		\arrow["{s^{\mbf t}_{X,\Sigma^\1Y_{t+1}}}"', from=3-2, to=3-1]
		\arrow["{s^{\mbf t}_{X,\Sigma Y_{t+1}}}"', from=2-2, to=2-1]
		\arrow["{{(\phi_{\mbf t,\1}\otimes Y_{t+1})}_*}"{description}, from=4-2, to=3-2]
		\arrow["{s^{\mbf t+1}_{X,Y_{t+1}}}"', from=4-2, to=4-1]
		\arrow["{{(\Sigma^{\mbf t+\1}j_{t+1})}_*}"{description}, from=4-2, to=5-2]
		\arrow["{{(\Sigma^t k_t)}_*}"{description}, from=1-3, to=3-3]
		\arrow["{{(\Sigma^{t+1} j_{t+1})}_*}"{description}, from=3-3, to=5-3]
		\arrow["{{(\nu^{t+1}_{Y_{t+1}})}_*}"', from=4-2, to=3-3]
		\arrow["{{(\nu^{t}_{\Sigma Y_{t+1}})}_*}", from=2-2, to=3-3]
		\arrow["{{(\delta_t)}_*}", shift left=5, curve={height=-40pt}, from=1-3, to=5-3]
	\end{tikzcd}\]
	where here the $s^a_{X,Y}:{[X,\Sigma^aY]}_*\cong{[X,Y]}_{*-a}$'s are the natural isomorphisms from \autoref{s^a_isos}. By unravelling definitions, we have the top left object is $E_1^{t,a+\mbf s-\mbf t}(X,Y)$ and the bottom left object is $E_1^{t+1,a+\mbf s-\mbf t-\1}$, and the vertical left composition in the above diagram is the differential $d_1$ between them. The first, second, and fourth rectangles from the top on the left rectangle commute by naturality of the $s^a$'s. Furthermore, a simple diagram chase and coherence of the $\phi$'s (\autoref{unique_comp_Sas}) yields that the third rectangle on the left commutes. The trapezoids on the right commute by naturality of $\nu^t$ and $\nu^{t+1}$. Finally, the middle right triangle commutes by how we defined $\nu^{t+1}$ in terms of $\nu^t$. 
	
	Now, consider the following diagram:
	% https://q.uiver.app/#q=WzAsMTAsWzAsMCwiRV8xXnt0LGErXFxtYmYgcy1cXG1iZiB0fShYLFkpIl0sWzIsMCwiRV8xXnt0KzEsYStcXG1iZiBzLVxcbWJmIHQtXFwxfShYLFkpIl0sWzAsMSwie1tYLFxcU2lnbWFee1xcbWJmIHR9V190XX1fe2ErXFxtYmYgc30iXSxbMiwxLCJ7W1gsXFxTaWdtYV57XFxtYmYgdCtcXDF9V197dCsxfV19X3thK1xcbWJmIHN9Il0sWzAsMiwie1tYLFxcU2lnbWFedFdfdF19X3thK1xcbWJmIHN9Il0sWzIsMiwie1tYLFxcU2lnbWFee3QrMX1XX3t0KzF9XX1fe2ErXFxtYmYgc30iXSxbMCwzLCJcXEhvbV97RV8qKEUpfShFXyooXFxTaWdtYV57YStcXG1iZiBzfVgpLEVfKihcXFNpZ21hXnRXX3QpKSJdLFsyLDMsIlxcSG9tX3tFXyooRSl9KEVfKihcXFNpZ21hXnthK1xcbWJmIHN9WCksRV8qKFxcU2lnbWFee3QrMX1XX3t0KzF9KSkiXSxbMCw0LCJcXEhvbV97RV8qKEUpfV57YStcXG1iZiBzfShFXyooWCksRV8qKFxcU2lnbWFedFdfdCkpIl0sWzIsNCwiXFxIb21fe0VfKihFKX1ee2ErXFxtYmYgc30oRV8qKFgpLEVfKihcXFNpZ21hXnt0KzF9V197dCsxfSkpIl0sWzAsMiwieyhzXntcXG1iZiB0fV97WCxXX3R9KX1eey0xfSIsMl0sWzEsMywieyhzXntcXG1iZiB0K1xcMX1fe1gsV197dCsxfX0pfV57LTF9Il0sWzIsNCwieyhcXG51XnRfe1dfdH0pfV8qIiwyXSxbMyw1LCJ7KFxcbnVee3QrMX1fe1dfe3QrMX19KX1fKiJdLFs0LDUsInsoXFxkZWx0YV90KX1fKiJdLFswLDEsImRfMSJdLFs0LDYsIkVfKigtKSIsMl0sWzYsNywiRV8qKFxcZGVsdGFfdCkiXSxbNSw3LCJFXyooLSkiXSxbNiw4LCJ7KHsodF57YStcXG1iZiBzfV9YKX1eey0xfSl9XioiLDJdLFs4LDksIkVfKihcXGRlbHRhX3QpIl0sWzcsOSwieyh7KHRee2ErXFxtYmYgc31fWCl9XnstMX0pfV4qIl1d
	\[\begin{tikzcd}
		{E_1^{t,a+\mbf s-\mbf t}(X,Y)} && {E_1^{t+1,a+\mbf s-\mbf t-\1}(X,Y)} \\
		{{[X,\Sigma^{\mbf t}W_t]}_{a+\mbf s}} && {{[X,\Sigma^{\mbf t+\1}W_{t+1}]}_{a+\mbf s}} \\
		{{[X,\Sigma^tW_t]}_{a+\mbf s}} && {{[X,\Sigma^{t+1}W_{t+1}]}_{a+\mbf s}} \\
		{\Hom_{E_*(E)}(E_*(\Sigma^{a+\mbf s}X),E_*(\Sigma^tW_t))} && {\Hom_{E_*(E)}(E_*(\Sigma^{a+\mbf s}X),E_*(\Sigma^{t+1}W_{t+1}))} \\
		{\Hom_{E_*(E)}^{a+\mbf s}(E_*(X),E_*(\Sigma^tW_t))} && {\Hom_{E_*(E)}^{a+\mbf s}(E_*(X),E_*(\Sigma^{t+1}W_{t+1}))}
		\arrow["{{(s^{\mbf t}_{X,W_t})}^{-1}}"', from=1-1, to=2-1]
		\arrow["{{(s^{\mbf t+\1}_{X,W_{t+1}})}^{-1}}", from=1-3, to=2-3]
		\arrow["{{(\nu^t_{W_t})}_*}"', from=2-1, to=3-1]
		\arrow["{{(\nu^{t+1}_{W_{t+1}})}_*}", from=2-3, to=3-3]
		\arrow["{{(\delta_t)}_*}", from=3-1, to=3-3]
		\arrow["{d_1}", from=1-1, to=1-3]
		\arrow["{E_*(-)}"', from=3-1, to=4-1]
		\arrow["{E_*(\delta_t)}", from=4-1, to=4-3]
		\arrow["{E_*(-)}", from=3-3, to=4-3]
		\arrow["{{({(t^{a+\mbf s}_X)}^{-1})}^*}"', from=4-1, to=5-1]
		\arrow["{E_*(\delta_t)}", from=5-1, to=5-3]
		\arrow["{{({(t^{a+\mbf s}_X)}^{-1})}^*}", from=4-3, to=5-3]
	\end{tikzcd}\]
	where here the maps $t^{a+\mbf s}_X:E_*(\Sigma^a)\to E_{*-a}(X)$ are the $E_*(E)$-comodule isomorphisms from \autoref{t^a_isos_are_E_*E-comodule_isos}. We have just shown the top region commutes. Furthermore, since $X$ and $\Sigma^tW_t$ are cellular for all $t\in\bN$, the arrows labelled $E_*(-)$ are well-defined, and they clearly make the middle rectangle commute (a simple diagram chase suffices). The bottom rectangle also clearly commutes, Thus, it suffices to show that the maps labelled $E_*(-)$ are isomorphisms. To that end, consider the following diagram:
	% https://q.uiver.app/#q=WzAsNCxbMSwwLCJcXEhvbV97RV8qKEUpfShFXyooXFxTaWdtYV57YStcXG1iZiBzfVgpLEVfKihcXFNpZ21hXnRXX3QpKSJdLFswLDIsIntbWCxFXFxvdGltZXNcXFNpZ21hXntcXG1iZiB0fVlfdF19X3thK1xcbWJmIHN9Il0sWzEsMiwiXFxIb21fe0VfKihFKX0oRV8qKFxcU2lnbWFee2ErXFxtYmYgc31YKSxFXyooRVxcb3RpbWVzIFxcU2lnbWFee1xcbWJmIHR9WV90KSkiXSxbMCwwLCJ7W1gsXFxTaWdtYV57dH1XX3RdfV97YStcXG1iZiBzfSJdLFsxLDIsIkVfKigtKSJdLFszLDEsImZfKiIsMl0sWzAsMiwie0VfKihmKX1fKiJdLFszLDAsIkVfKigtKSJdXQ==
	\[\begin{tikzcd}
		{{[X,\Sigma^{t}W_t]}_{a+\mbf s}} & {\Hom_{E_*(E)}(E_*(\Sigma^{a+\mbf s}X),E_*(\Sigma^tW_t))} \\
		\\
		{{[X,E\otimes\Sigma^{\mbf t}Y_t]}_{a+\mbf s}} & {\Hom_{E_*(E)}(E_*(\Sigma^{a+\mbf s}X),E_*(E\otimes \Sigma^{\mbf t}Y_t))}
		\arrow["{E_*(-)}", from=3-1, to=3-2]
		\arrow["{f_*}"', from=1-1, to=3-1]
		\arrow["{{E_*(f)}_*}", from=1-2, to=3-2]
		\arrow["{E_*(-)}", from=1-1, to=1-2]
	\end{tikzcd}\]
	where here $f:\Sigma^tW_t\to E\otimes \Sigma^{\mbf t}Y_t$ is the isomorphism
	\[\Sigma^tW_t\xr{\nu^t_W}\Sigma^{\mbf t}W_t=S^{\mbf t}\otimes E\otimes Y_t\xr{\tau\otimes Y_t}E\otimes S^{\mbf t}\otimes Y_t=E\otimes\Sigma^{\mbf t}Y_t.\]
	The bottom horizontal arrow is an isomorphism by \autoref{X,EY_iso_Hom(E_*X,E_*EY)}. Thus, the top horizontal arrow is an isomorphism, as desired. Showing
	\[E_*(-):{[X,\Sigma^{t+1}W_{t+1}]}_{a+\mbf s}\to\Hom_{E_*(E)}(E_*(\Sigma^{a+\mbf s}X),E_*(\Sigma^{t+1}W_{t+1}))\]
	is an isomorphism is entirely analagous. Thus, for each $t\in\bN$, we have constructed isomorphisms
	\[E^{t,a+\mbf s-\mbf t}(X,Y)\xr\cong\Hom_{E_*(E)}^{a+\mbf s}(E_*(X),E_*(\Sigma^tW_t))\]
	such that the following diagram commutes:
	% https://q.uiver.app/#q=WzAsNCxbMCwwLCJFXnt0LGErXFxtYmYgcy1cXG1iZiB0fShYLFkpIl0sWzQsMCwiRV57dCsxLGErXFxtYmYgcy1cXG1iZiB0LVxcMX0oWCxZKSJdLFswLDIsIlxcSG9tX3tFXyooRSl9XnthK1xcbWJmIHN9KEVfKihYKSxFXyooXFxTaWdtYV50V190KSkiXSxbNCwyLCJcXEhvbV97RV8qKEUpfV57YStcXG1iZiBzfShFXyooWCksRV8qKFxcU2lnbWFee3QrMX1XX3t0KzF9KSkiXSxbMCwxLCJkXzEiXSxbMCwyLCJcXGNvbmciLDJdLFsyLDMsIlxcSG9tX3tFXyooRSl9XnthK1xcbWJmIHN9KEVfKihYKSxFXyooXFxkZWx0YV90KSkiXSxbMSwzLCJcXGNvbmciXV0=
	\[\begin{tikzcd}
		{E^{t,a+\mbf s-\mbf t}(X,Y)} &&&& {E^{t+1,a+\mbf s-\mbf t-\1}(X,Y)} \\
		\\
		{\Hom_{E_*(E)}^{a+\mbf s}(E_*(X),E_*(\Sigma^tW_t))} &&&& {\Hom_{E_*(E)}^{a+\mbf s}(E_*(X),E_*(\Sigma^{t+1}W_{t+1}))}
		\arrow["{d_1}", from=1-1, to=1-5]
		\arrow["\cong"', from=1-1, to=3-1]
		\arrow["{\Hom_{E_*(E)}^{a+\mbf s}(E_*(X),E_*(\delta_t))}", from=3-1, to=3-5]
		\arrow["\cong", from=1-5, to=3-5]
	\end{tikzcd}\]
	Hence, we have proven the desired result.
\end{proof}

\subsection{Convergence of the spectral sequence}

Before we can state and prove some convergence results for the spectral sequence we have constructed above, we outline a bit of the theory of \emph{nilpotent completion} of objects in $\cSH$. Namely, we will outline suitable conditions under which the $E$-Adams spectral sequence for ${[X,Y]}_*$ converges to the homotopy groups ${[X,Y^\wedge_E]}_*$, where $Y^\wedge_E$ is an \emph{$E$-nilpotent completion of $Y$}. The main reference for this section and the next will be \S 5--6 in the paper \cite{Bousfield_79} by Bousfield. First, we state some definitions.

\begin{definition}[{\cite{RGB}}]
	Given an object $Y$ in $\cSH$ and a monoid objedct $(E,\mu,e)$ an \emph{$E$-completion} $\widehat Y$ of $Y$ is an object in $\cSH$ such that:\begin{enumerate}[label=(\alph*)]
		\item There is a map $Y\to\wh Y$ inducing an isomorphism in $E_*$-homology.
		\item $\wh Y$ has an $E$-Adams resolution $(\wh Y_s,\wh W_s;i,j,k)$ (\autoref{defn:E-Adams_resolution}) with $\holim\wh Y_s=0$ (see \autoref{homotopy_limit_defn} for the definition of homotopy limits in a triangulated category with products).
	\end{enumerate}
\end{definition}

\begin{definition}[{\cite[pgs.\ 272--273]{Bousfield_79}}]\label{defn:nilpotent_completion}
	Let $(E,\mu,e)$ be a monoid object in $\cSH$, and $Y$ any object. Write $\ol E$ for the homotopy fiber (\autoref{fiber}) of the unit $S\xr eE$, so we have a distinguished triangle
	\[\ol E\to S\xr eE\to\Sigma\ol E.\]
	Set $Y_0:=Y$ and $W_0:=Y\otimes E$, and for $s>0$ define $Y_s:= Y\otimes\ol E^s$ and $W_s:=Y_s\otimes E$. Then since $\cSH$ is tensor triangulated, for each $s\geq0$ we may tensor the above sequence with $Y_s$ on the right, which yields the following distinguished triangle
	\[Y_{s+1}\xr iY_s\xr jW_s\xr k\Sigma Y_{s+1}.\]
	Then for $s\in\bN$, define $Y/Y^s$ to be the cofiber of $i^{s}:Y_s\to Y_0=Y$ (so in particular we may take $Y/Y_1=E\otimes Y$ and $Y/Y_0=0$), so we have a distinguished triangle
	\[Y_s\xr{i^s}Y\xr bY/Y_s\xr c\Sigma Y_s.\]
	Then for each $s\geq0$, by the octahedral axiom (axiom TR5) for a triangulated category applied to the triangles
	\begin{gather*}
		Y_{s+1}\xr i Y_s\xr jW_s\xr k\Sigma Y_{s+1} \\
		Y_s\xr{i^s}Y\xr bY/Y_s\xr c\Sigma Y_s \\
		Y_{s+1}\xr{i^{s+1}}Y\xr bY/Y_{s+1}\xr c\Sigma Y_{s+1},
	\end{gather*}
	there exists a distinguished triangle
	\begin{equation}\label{quotient_triangle_Y's}W_s\xr p Y/Y_{s+1}\xr qY/Y_s\xr r\Sigma W_s.\end{equation}
	is distinguished and the following diagram commutes:
	% https://q.uiver.app/#q=WzAsOSxbMCwwLCJZX3tzKzF9Il0sWzIsMCwiWSJdLFs0LDAsIlkvWV9zIl0sWzYsMCwiXFxTaWdtYSBXX3MiXSxbMSwxLCJZX3MiXSxbMywxLCJZL1lfe3MrMX0iXSxbNSwxLCJcXFNpZ21hIFlfcyJdLFsyLDIsIldfcyJdLFs0LDIsIlxcU2lnbWEgWV97cysxfSJdLFswLDEsImlee3MrMX0iXSxbMSwyLCJiIl0sWzIsMywiciJdLFswLDQsImkiLDJdLFs0LDcsImoiLDJdLFs3LDgsImsiXSxbOCw2LCJcXFNpZ21hIGkiLDJdLFs2LDMsIlxcU2lnbWEgaiIsMl0sWzEsNSwiYiIsMl0sWzUsMiwicSIsMl0sWzcsNSwicCJdLFs1LDgsImMiXSxbNCwxLCJpXnMiLDJdLFsyLDYsImMiLDJdXQ==
	\begin{equation}\label{gross_octa_diagram_Y's}\begin{tikzcd}
		{Y_{s+1}} && Y && {Y/Y_s} && {\Sigma W_s} \\
		& {Y_s} && {Y/Y_{s+1}} && {\Sigma Y_s} \\
		&& {W_s} && {\Sigma Y_{s+1}}
		\arrow["{i^{s+1}}", from=1-1, to=1-3]
		\arrow["b", from=1-3, to=1-5]
		\arrow["r", from=1-5, to=1-7]
		\arrow["i"', from=1-1, to=2-2]
		\arrow["j"', from=2-2, to=3-3]
		\arrow["k", from=3-3, to=3-5]
		\arrow["{\Sigma i}"', from=3-5, to=2-6]
		\arrow["{\Sigma j}"', from=2-6, to=1-7]
		\arrow["b"', from=1-3, to=2-4]
		\arrow["q"', from=2-4, to=1-5]
		\arrow["p", from=3-3, to=2-4]
		\arrow["c", from=2-4, to=3-5]
		\arrow["{i^s}"', from=2-2, to=1-3]
		\arrow["c"', from=1-5, to=2-6]
	\end{tikzcd}\end{equation}
	The triangles from (\ref{quotient_triangle_Y's}) may be spliced together to yield a tower $\{Y/Y_{s}\}_s$ under $Y$:
	% https://q.uiver.app/#q=WzAsMTEsWzAsMCwiWSJdLFsxLDAsIlxcY2RvdHMiXSxbMiwwLCJZL1lfMyJdLFszLDAsIlkvWV8yIl0sWzQsMCwiWS9ZXzEiXSxbNSwwLCJZL1lfMCJdLFs2LDAsIjAiXSxbMywxLCJXXzIiXSxbNSwxLCJXXzAiXSxbNCwxLCJXXzEiXSxbMiwxLCJXXzMiXSxbMCwxXSxbMSwyXSxbMiwzLCJxIl0sWzMsNCwicSJdLFs0LDUsInEiXSxbNSw2LCIiLDAseyJsZXZlbCI6Miwic3R5bGUiOnsiaGVhZCI6eyJuYW1lIjoibm9uZSJ9fX1dLFszLDcsInIiLDAseyJzdHlsZSI6eyJib2R5Ijp7Im5hbWUiOiJkYXNoZWQifX19XSxbNywyLCJwIiwxXSxbNSw4LCJyIiwwLHsic3R5bGUiOnsiYm9keSI6eyJuYW1lIjoiZGFzaGVkIn19fV0sWzgsNCwicCIsMV0sWzQsOSwiciIsMCx7InN0eWxlIjp7ImJvZHkiOnsibmFtZSI6ImRhc2hlZCJ9fX1dLFs5LDMsInAiLDFdLFsyLDEwLCJyIiwwLHsic3R5bGUiOnsiYm9keSI6eyJuYW1lIjoiZGFzaGVkIn19fV1d
	\[\begin{tikzcd}
		Y & \cdots & {Y/Y_3} & {Y/Y_2} & {Y/Y_1} & {Y/Y_0} & 0 \\
		&& {W_3} & {W_2} & {W_1} & {W_0}
		\arrow[from=1-1, to=1-2]
		\arrow[from=1-2, to=1-3]
		\arrow["q", from=1-3, to=1-4]
		\arrow["q", from=1-4, to=1-5]
		\arrow["q", from=1-5, to=1-6]
		\arrow[Rightarrow, no head, from=1-6, to=1-7]
		\arrow["r", dashed, from=1-4, to=2-4]
		\arrow["p"{description}, from=2-4, to=1-3]
		\arrow["r", dashed, from=1-6, to=2-6]
		\arrow["p"{description}, from=2-6, to=1-5]
		\arrow["r", dashed, from=1-5, to=2-5]
		\arrow["p"{description}, from=2-5, to=1-4]
		\arrow["r", dashed, from=1-3, to=2-3]
	\end{tikzcd}\]
	where here the dashed arrows are really (degree $-\1$) maps $Y/Y_s\to\Sigma W_s$. The fact that this is a tower under $Y$ follows from diagram (\ref{gross_octa_Y's}), which tells us that $Y\xr b Y/Y_s$ factors as $Y\xr bY/Y_{s+1}\xr qY/Y_s$. We define the \emph{$E$-nilpotent completion of $Y$} to be the object $Y^\wedge_E$ (defined up to non-canonical isomorphism) obtained as the homotopy limit of this tower (\autoref{homotopy_limit_defn}):
	\[Y^\wedge_E:=\holim_sY_s/Y.\]
	This comes equipped with a map $\alpha:Y\to Y^\wedge_E$. 
\end{definition}

\begin{proposition}
	Consider the tower under $Y$ constructed in \autoref{defn:nilpotent_completion}:
	% https://q.uiver.app/#q=WzAsMTEsWzAsMCwiWSJdLFsxLDAsIlxcY2RvdHMiXSxbMiwwLCJZL1lfMyJdLFszLDAsIlkvWV8yIl0sWzQsMCwiWS9ZXzEiXSxbNSwwLCJZL1lfMCJdLFs2LDAsIjAiXSxbMywxLCJXXzIiXSxbNSwxLCJXXzAiXSxbNCwxLCJXXzEiXSxbMiwxLCJXXzMiXSxbMCwxXSxbMSwyXSxbMiwzLCJxIl0sWzMsNCwicSJdLFs0LDUsInEiXSxbNSw2LCIiLDAseyJsZXZlbCI6Miwic3R5bGUiOnsiaGVhZCI6eyJuYW1lIjoibm9uZSJ9fX1dLFszLDcsInIiLDAseyJzdHlsZSI6eyJib2R5Ijp7Im5hbWUiOiJkYXNoZWQifX19XSxbNywyLCJwIiwxXSxbNSw4LCJyIiwwLHsic3R5bGUiOnsiYm9keSI6eyJuYW1lIjoiZGFzaGVkIn19fV0sWzgsNCwicCIsMV0sWzQsOSwiciIsMCx7InN0eWxlIjp7ImJvZHkiOnsibmFtZSI6ImRhc2hlZCJ9fX1dLFs5LDMsInAiLDFdLFsyLDEwLCJyIiwwLHsic3R5bGUiOnsiYm9keSI6eyJuYW1lIjoiZGFzaGVkIn19fV1d
	\[\begin{tikzcd}
		Y & \cdots & {Y/Y_3} & {Y/Y_2} & {Y/Y_1} & {Y/Y_0} & 0 \\
		&& {W_3} & {W_2} & {W_1} & {W_0}
		\arrow[from=1-1, to=1-2]
		\arrow[from=1-2, to=1-3]
		\arrow["q", from=1-3, to=1-4]
		\arrow["q", from=1-4, to=1-5]
		\arrow["q", from=1-5, to=1-6]
		\arrow[Rightarrow, no head, from=1-6, to=1-7]
		\arrow["r", dashed, from=1-4, to=2-4]
		\arrow["p"{description}, from=2-4, to=1-3]
		\arrow["r", dashed, from=1-6, to=2-6]
		\arrow["p"{description}, from=2-6, to=1-5]
		\arrow["r", dashed, from=1-5, to=2-5]
		\arrow["p"{description}, from=2-5, to=1-4]
		\arrow["r", dashed, from=1-3, to=2-3]
	\end{tikzcd}\]
	We may extend it to the right by defining $Y/Y_s=W_s=0$ for $s<0$. Then by \autoref{X,Y*_LES_compact}, we may apply the functor ${[X,-]}_*$ which yields the following $A$-graded unrolled exact couple (\autoref{unrolled_exact_couple}):
	% https://q.uiver.app/#q=WzAsMTAsWzAsMCwiXFxjZG90cyJdLFsxLDAsIntbWCxZL1lfe3MrMn1dfV8qIl0sWzIsMCwie1tYLFkvWV97cysxfV19XyoiXSxbMywwLCJ7W1gsWS9ZX3tzfV19XyoiXSxbNCwwLCJ7W1gsWS9ZX3tzLTF9XX1fKiJdLFs1LDAsIlxcY2RvdHMiXSxbNCwxLCJ7W1gsV197cy0xfV19XyoiXSxbMywxLCJ7W1gsV197c31dfV8qIl0sWzIsMSwie1tYLFdfe3MrMX1dfV8qIl0sWzEsMSwie1tYLFdfe3MrMn1dfV8qIl0sWzAsMV0sWzEsMiwicSJdLFsyLDMsInEiXSxbMyw0LCJxIl0sWzQsNV0sWzQsNiwiXFxkZWx0YSJdLFs2LDMsInAiLDFdLFszLDcsIlxcZGVsdGEiXSxbNywyLCJwIiwxXSxbMiw4LCJcXGRlbHRhIl0sWzgsMSwicCIsMV0sWzEsOSwiXFxkZWx0YSJdXQ==
	\[\begin{tikzcd}
		\cdots & {{[X,Y/Y_{s+2}]}_*} & {{[X,Y/Y_{s+1}]}_*} & {{[X,Y/Y_{s}]}_*} & {{[X,Y/Y_{s-1}]}_*} & \cdots \\
		& {{[X,W_{s+2}]}_*} & {{[X,W_{s+1}]}_*} & {{[X,W_{s}]}_*} & {{[X,W_{s-1}]}_*}
		\arrow[from=1-1, to=1-2]
		\arrow["q", from=1-2, to=1-3]
		\arrow["q", from=1-3, to=1-4]
		\arrow["q", from=1-4, to=1-5]
		\arrow[from=1-5, to=1-6]
		\arrow["\delta", from=1-5, to=2-5]
		\arrow["p"{description}, from=2-5, to=1-4]
		\arrow["\delta", from=1-4, to=2-4]
		\arrow["p"{description}, from=2-4, to=1-3]
		\arrow["\delta", from=1-3, to=2-3]
		\arrow["p"{description}, from=2-3, to=1-2]
		\arrow["\delta", from=1-2, to=2-2]
	\end{tikzcd}\]
	Thus by \autoref{SSeq_assoc_to_unrolled_EC}, there is an induced spectral sequence. This spectral sequence is precisely the $E$-Adams spectral sequence for ${[X,Y]}_*$ (\autoref{ASS}) determined by the canonical $E$-Adams resolution of $Y$ (\autoref{defn:canonical_E-Adams_resolution}).
\end{proposition}
\begin{proof}
	For $s\geq0$, define
	\[f_s:{[X,Y/Y_s]}_*\xr{c_*}{[X,\Sigma Y_s]}_*\xr{{(\nu_{Y_s})}_*}{[X,\Sigma^\1Y_s]}_*\xr{s^\1_{X,Y_s}}{[X,Y_s]}_{*-\1},\]
	and for $s<0$ let it be the unique map
	\[f_s:{[X,Y/Y_s]}_*=0\to{[X,Y_s]}_{*-\1}={[X,Y]}_{*-1}.\]
	For $s\in\bZ$, let
	\[g_s:=\id_{W_s}:{[X,W_s]}_*\to{[X,W_s]}_*.\]
	We claim these maps ${(f_s,g_s)}_s$ define a homomorphism of $A$-graded unrolled exact couples (\autoref{defn:unrolled_exact_couple_homo}) between the unrolled exact couple given above and that obtained by applying ${[X,-]}_*$ to the canonical $E$-Adams resolution. To that end, it suffices to show that the following diagram commutes for all $s\in\bZ$: 
	% https://q.uiver.app/#q=WzAsOCxbMCwwLCJ7W1gsWS9ZX3NdfV8qIl0sWzEsMCwie1tYLFkvWV97cy0xfV19XyoiXSxbMiwwLCJ7W1gsV197cy0xfV19X3sqLVxcbWF0aGJmMX0iXSxbMywwLCJ7W1gsWS9ZX3NdfV97Ki1cXG1hdGhiZjF9Il0sWzAsMSwie1tYLFlfc119X3sqLVxcMX0iXSxbMSwxLCJ7W1gsWV97cy0xfV19X3sqLVxcMX0iXSxbMiwxLCJ7W1gsV197cy0xfV19X3sqLVxcMX0iXSxbMywxLCJ7W1gsWV9zXX1feyotXFxtYXRoYmYgMn0iXSxbMCwxXSxbMSwyXSxbMiwzXSxbMCw0LCJmX3MiLDJdLFs0LDVdLFs1LDZdLFs2LDddLFsxLDUsImZfe3MtMX0iLDFdLFsyLDYsIiIsMSx7ImxldmVsIjoyLCJzdHlsZSI6eyJoZWFkIjp7Im5hbWUiOiJub25lIn19fV0sWzMsNywiZl9zIl1d
	\[\begin{tikzcd}
		{{[X,Y/Y_s]}_*} & {{[X,Y/Y_{s-1}]}_*} & {{[X,W_{s-1}]}_{*-\mathbf1}} & {{[X,Y/Y_s]}_{*-\mathbf1}} \\
		{{[X,Y_s]}_{*-\1}} & {{[X,Y_{s-1}]}_{*-\1}} & {{[X,W_{s-1}]}_{*-\1}} & {{[X,Y_s]}_{*-\mathbf 2}}
		\arrow[from=1-1, to=1-2]
		\arrow[from=1-2, to=1-3]
		\arrow[from=1-3, to=1-4]
		\arrow["{f_s}"', from=1-1, to=2-1]
		\arrow[from=2-1, to=2-2]
		\arrow[from=2-2, to=2-3]
		\arrow[from=2-3, to=2-4]
		\arrow["{f_{s-1}}"{description}, from=1-2, to=2-2]
		\arrow[Rightarrow, no head, from=1-3, to=2-3]
		\arrow["{f_s}", from=1-4, to=2-4]
	\end{tikzcd}\]
	In the case $s\leq0$, we know $Y/Y_s=Y/Y_{s-1}=W_{s-1}=0$, so that the top row is entirely $0$, and thus the diagram must commute. In the case $s>0$, by unravelling definitions we have that the diagram becomes
	% https://q.uiver.app/#q=WzAsMTYsWzAsMCwie1tYLFkvWV9zXX1fKiJdLFsyLDAsIntbWCxZL1lfe3MtMX1dfV8qIl0sWzQsMCwie1tYLFdfe3MtMX1dfV97Ki1cXG1hdGhiZjF9Il0sWzYsMCwie1tYLFkvWV9zXX1feyotXFxtYXRoYmYxfSJdLFswLDEsIntbWCxcXFNpZ21hIFlfc119XyoiXSxbMCwyLCJ7W1gsXFxTaWdtYV5cXDEgWV9zXX1fKiJdLFswLDMsIntbWCxZX3NdfV97Ki1cXG1hdGhiZjF9Il0sWzIsMywie1tYLFlfe3MtMX1dfV97Ki1cXG1hdGhiZjF9Il0sWzIsMSwie1tYLFxcU2lnbWEgWV97cy0xfV19XyoiXSxbMiwyLCJ7W1gsXFxTaWdtYV5cXDFZX3tzLTF9XX1fKiJdLFs0LDMsIntbWCxXX3tzLTF9XX1feyotXFxtYXRoYmYxfSJdLFs2LDMsIntbWCxZX3NdfV97Ki1cXG1hdGhiZjJ9Il0sWzYsMSwie1tYLFxcU2lnbWEgWV9zXX1feyotXFxtYXRoYmYxfSJdLFs2LDIsIntbWCxcXFNpZ21hXlxcbWF0aGJmMVlfc119X3sqLVxcbWF0aGJmMX0iXSxbMywxLCJ7W1gsXFxTaWdtYSBXX3tzLTF9XX1fKiJdLFszLDIsIntbWCxcXFNpZ21hXlxcMVdfe3MtMX1dfV8qIl0sWzAsMSwicV8qIl0sWzEsMiwiXFxkZWx0YSJdLFsyLDMsInBfKiJdLFswLDQsImNfKiIsMl0sWzQsNSwieyhcXG51X3tZX3N9KX1fKiIsMl0sWzUsNiwic15cXG1hdGhiZjFfe1gsWV9zfSIsMl0sWzYsNywiaV8qIl0sWzEsOCwiY18qIl0sWzgsOSwieyhcXG51X3tZX3tzLTF9fSl9XyoiXSxbOSw3LCJzXlxcbWF0aGJmMV97WCxZX3tzLTF9fSJdLFs3LDEwLCJqXyoiXSxbMTAsMTEsIlxccGFydGlhbF8qIl0sWzMsMTIsImNfKiJdLFsxMiwxMywieyhcXG51X3tZX3N9KX1fKiJdLFsxMywxMSwic15cXDFfe1gsWV9zfSJdLFs0LDgsIlxcU2lnbWEgaV8qIl0sWzUsOSwiXFxTaWdtYV5cXDFpXyoiXSxbMTAsMTIsImtfKiJdLFsxLDE0LCJyXyoiXSxbMTQsMTUsInsoXFxudV97V197cy0xfX0pfV8qIl0sWzgsMTQsIlxcU2lnbWEgal8qIl0sWzksMTUsIlxcU2lnbWFeXFwxal8qIl0sWzE1LDEwLCJzXlxcMV97WCxXX3tzLTF9fSJdLFsyLDEwLCIiLDAseyJsZXZlbCI6Miwic3R5bGUiOnsiaGVhZCI6eyJuYW1lIjoibm9uZSJ9fX1dXQ==
	\[\begin{tikzcd}[column sep=small]
		{{[X,Y/Y_s]}_*} && {{[X,Y/Y_{s-1}]}_*} && {{[X,W_{s-1}]}_{*-\mathbf1}} && {{[X,Y/Y_s]}_{*-\mathbf1}} \\
		{{[X,\Sigma Y_s]}_*} && {{[X,\Sigma Y_{s-1}]}_*} & {{[X,\Sigma W_{s-1}]}_*} &&& {{[X,\Sigma Y_s]}_{*-\mathbf1}} \\
		{{[X,\Sigma^\1 Y_s]}_*} && {{[X,\Sigma^\1Y_{s-1}]}_*} & {{[X,\Sigma^\1W_{s-1}]}_*} &&& {{[X,\Sigma^\mathbf1Y_s]}_{*-\mathbf1}} \\
		{{[X,Y_s]}_{*-\mathbf1}} && {{[X,Y_{s-1}]}_{*-\mathbf1}} && {{[X,W_{s-1}]}_{*-\mathbf1}} && {{[X,Y_s]}_{*-\mathbf2}}
		\arrow["{q_*}", from=1-1, to=1-3]
		\arrow["\delta", from=1-3, to=1-5]
		\arrow["{p_*}", from=1-5, to=1-7]
		\arrow["{c_*}"', from=1-1, to=2-1]
		\arrow["{{(\nu_{Y_s})}_*}"', from=2-1, to=3-1]
		\arrow["{s^\mathbf1_{X,Y_s}}"', from=3-1, to=4-1]
		\arrow["{i_*}", from=4-1, to=4-3]
		\arrow["{c_*}", from=1-3, to=2-3]
		\arrow["{{(\nu_{Y_{s-1}})}_*}", from=2-3, to=3-3]
		\arrow["{s^\mathbf1_{X,Y_{s-1}}}", from=3-3, to=4-3]
		\arrow["{j_*}", from=4-3, to=4-5]
		\arrow["{\partial_*}", from=4-5, to=4-7]
		\arrow["{c_*}", from=1-7, to=2-7]
		\arrow["{{(\nu_{Y_s})}_*}", from=2-7, to=3-7]
		\arrow["{s^\1_{X,Y_s}}", from=3-7, to=4-7]
		\arrow["{\Sigma i_*}", from=2-1, to=2-3]
		\arrow["{\Sigma^\1i_*}", from=3-1, to=3-3]
		\arrow["{k_*}", from=4-5, to=2-7]
		\arrow["{r_*}", from=1-3, to=2-4]
		\arrow["{{(\nu_{W_{s-1}})}_*}", from=2-4, to=3-4]
		\arrow["{\Sigma j_*}", from=2-3, to=2-4]
		\arrow["{\Sigma^\1j_*}", from=3-3, to=3-4]
		\arrow["{s^\1_{X,W_{s-1}}}", from=3-4, to=4-5]
		\arrow[Rightarrow, no head, from=1-5, to=4-5]
	\end{tikzcd}\]
	Clearly commutativity of this diagram yields that the given collection of maps define a homomorphism of $A$-graded unrolled exact couples. Each rectangular region commutes by naturality, as does the middle bottom trapezoidal region. The two regions involving $\delta$ and $\partial$ commute by unravelling how the differential is defined in \autoref{X,Y*_LES_compact}. Finally, the remaining two regions commute by commutativity of \autoref{gross_octa_diagram_Y's}.

	Thus, we have defined a homomorphism of $A$-graded unrolled exact couples, so that by \autoref{UEC_homo_induces_SSeq_homo} it induces a homomorphism of the associated spectral sequences $\wt g$. Further unravelling how this homomorphism of spectral sequences is defined, since the homomorphism of unrolled exact couples is the identity on the ${[X,W_s]}_*$ terms, it follows that the two spectral sequences are strictly equal.
\end{proof}

\begin{remark}
	In \cite{Bousfield_79}, the $E$-nilpotent completion of $Y$ (\autoref{defn:nilpotent_completion}) is denoted ``$E^\wedge Y$'', while the notation ``$Y^\wedge_E$'' we use here is standard in the modern literature.
\end{remark}

\begin{definition}
	Let $(E,\mu,e)$ be a monoid object and $X$ and $Y$ two objects in $\cSH$. Then we have an associated $E$-Adams spectral sequence $(E^{*,*}_r(X,Y),d_r)$ (\autoref{ASS}) and $E$-nilpotent completion $Y^\wedge_E$ (\autoref{defn:nilpotent_completion}). Then we may define a decreasing $A$-graded filtration of ${[X,Y^\wedge_E]}_*$ by defining
	\[F^s{[X,Y^\wedge_E]}_*:=\ker\({(\alpha_s)}_*:{[X,Y^\wedge_E]}_*\to{[X,\ol E_{s-1}\otimes Y]}_*\),\]
	for $s>0$, where $\alpha_s$ is the composition
	\[Y_E^\wedge\to\prod_{i=0}^\infty(\ol E_{i}\otimes Y)\onto \ol E_{s-1}\otimes Y\]
\end{definition}

Note that $F^1{[X,Y^\wedge_E]}_*={[X,E\otimes Y]}_*$. To see this, it suffices to show that $\alpha_1$ is the zero map. To see this, note that by how homotopy limits are constructed in \autoref{homotopy_limit_defn}, we have that the following diagram commutes:

\begin{definition}
	Let $(E,\mu,e)$ be a monoid object in $\cSH$, and $X$ and $Y$ any objects. Then for all
\end{definition}

\end{document}
