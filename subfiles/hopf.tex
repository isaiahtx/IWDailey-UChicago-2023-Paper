\documentclass[../main.tex]{subfiles}
\begin{document}

In this appendix, we will define the notion of \emph{$A$-graded anticommutative Hopf algebroids} (\autoref{hopf_algebroid_defn}) over an $A$-graded anticommutative ring $R$ (\autoref{A-graded_anticommutative_ring_defn}), and left comodules over them (\autoref{left_comodule_defn}).

\subsection{\texorpdfstring{$A$}{A}-graded anticommutative Hopf algebroids over \texorpdfstring{$R$}{R}}

Given an $A$-graded anticommutative ring $R$, we will define an $A$-graded anticommutative Hopf algebroid over $R$ to be a co-groupoid object in $R\text-\GCA^{A}$, i.e., a groupoid object in $(R\text-\GCA^{A})^\op$. First, recall the definition of a \emph{groupoid object} in a category with pullbacks:

\begin{definition}
	Let $\cC$ be a category with pullbacks. A \emph{groupoid object} in $\cC$ consists of a pair of objects $(M,O)$ together with five morphisms
	\begin{enumerate}
		\item \emph{Source and target}: $s,t:M\to O$,
		\item \emph{Identity}: $e:O\to M$,
		\item \emph{Composition}: $c:M\times_{O}M\to M$,
		\item \emph{Inverse}: $i:M\to M$
	\end{enumerate}
	Where $M\times_OM$ will always refer to the object which into the following pullback diagram in $\cC$:
	\[\begin{tikzcd}
		{M\times_OM} & M \\
		M & O
		\arrow["{p_2}", from=1-1, to=1-2]
		\arrow["{p_1}"', from=1-1, to=2-1]
		\arrow["s"', from=2-1, to=2-2]
		\arrow["t", from=1-2, to=2-2]
		\arrow["\lrcorner"{anchor=center, pos=0.125}, draw=none, from=1-1, to=2-2]
	\end{tikzcd}\]
	For example, if we're working in $\cC=\Set$, we should think of $M$ as a set of morphisms, and $O$ as a set of objects. The functions $s$ and $t$ take a morphism to their domain and codomain, respectively, and $M\times_OM$ is the collection of pairs of morphisms $(g,f)\in M\times M$ such that $t(f)=s(g)$, and the composition map $c:M\times_OM\to M$ takes such a pair to the element $g\circ f\in M$. We think of the identity $e:O\to M$ as taking some object $x\in O$ to the identity morphism $e(x)=\id_x\in M$ on $x$, and the inverse map $i:M\to M$ takes a morphism $f$ to its inverse $f^{-1}$. These data are required to make the following diagrams commute:
	\begin{enumerate}
		\item Composition works correctly:
		\[\begin{tikzcd}
			{M\times_OM} & M & M & O & M & {M\times_OM} & M \\
			M & O && O && M & O
			\arrow["e"', from=1-4, to=1-3]
			\arrow["s"', from=1-3, to=2-4]
			\arrow["e", from=1-4, to=1-5]
			\arrow["t", from=1-5, to=2-4]
			\arrow[Rightarrow, no head, from=1-4, to=2-4]
			\arrow["c", from=1-1, to=1-2]
			\arrow["t", from=1-2, to=2-2]
			\arrow["{p_1}"', from=1-1, to=2-1]
			\arrow["t", from=2-1, to=2-2]
			\arrow["c"', from=1-6, to=2-6]
			\arrow["s", from=2-6, to=2-7]
			\arrow["{p_2}", from=1-6, to=1-7]
			\arrow["s", from=1-7, to=2-7]
		\end{tikzcd}\]
		Expressed in terms of sets, the first diagram says that the target of $g\circ f$ is the target of $g$. The second diagram says that the domain and codomain of the identity on some object $x$ is $x$. The third diagram says that the domain of $g\circ f$ is the domain of $f$.
		\item Associativity of composition: Write $M\times_O(M\times_OM)$ and $(M\times_OM)\times_OM$ for the pullbacks of $(s,t\circ c)$ and $(s\circ c,t)$, respectively, so we have commuting diagrams
		% https://q.uiver.app/#q=WzAsMTQsWzQsMCwiTVxcdGltZXNfTyhNXFx0aW1lc19PTSkiXSxbNiwwLCJNXFx0aW1lc19PTSJdLFs1LDEsIk1cXHRpbWVzX09NIl0sWzYsMSwiTSJdLFs2LDIsIk8iXSxbNSwyLCJNIl0sWzQsMiwiTSJdLFswLDAsIihNXFx0aW1lc19PTSlcXHRpbWVzX09NIl0sWzEsMSwiTVxcdGltZXNfT00iXSxbMCwyLCJNXFx0aW1lc19PTSJdLFsxLDIsIk0iXSxbMiwyLCJPIl0sWzIsMSwiTSJdLFsyLDAsIk0iXSxbMCwxLCJwXzInJyJdLFswLDIsIk1cXHRpbWVzIGMiLDIseyJzdHlsZSI6eyJib2R5Ijp7Im5hbWUiOiJkYXNoZWQifX19XSxbMyw0LCJ0Il0sWzIsNSwicF8xIiwyXSxbNSw0LCJzIl0sWzIsMywicF8yIl0sWzEsMywiYyJdLFswLDYsInBfMScnIiwyXSxbNiw1LCIiLDAseyJsZXZlbCI6Miwic3R5bGUiOnsiaGVhZCI6eyJuYW1lIjoibm9uZSJ9fX1dLFs3LDgsImNcXHRpbWVzIE0iLDIseyJzdHlsZSI6eyJib2R5Ijp7Im5hbWUiOiJkYXNoZWQifX19XSxbNyw5LCJwXzEnIiwyXSxbOSwxMCwiYyJdLFsxMCwxMSwicyJdLFs4LDEwLCJwXzEiLDJdLFs4LDEyLCJwXzIiXSxbMTIsMTEsInQiXSxbNywxMywicF8yJyJdLFsxMywxMiwiIiwyLHsibGV2ZWwiOjIsInN0eWxlIjp7ImhlYWQiOnsibmFtZSI6Im5vbmUifX19XV0=
		\[\begin{tikzcd}[column sep=small]
			{(M\times_OM)\times_OM} && M && {M\times_O(M\times_OM)} && {M\times_OM} \\
			& {M\times_OM} & M &&& {M\times_OM} & M \\
			{M\times_OM} & M & O && M & M & O
			\arrow["{p_2''}", from=1-5, to=1-7]
			\arrow["{M\times c}"', dashed, from=1-5, to=2-6]
			\arrow["t", from=2-7, to=3-7]
			\arrow["{p_1}"', from=2-6, to=3-6]
			\arrow["s", from=3-6, to=3-7]
			\arrow["{p_2}", from=2-6, to=2-7]
			\arrow["c", from=1-7, to=2-7]
			\arrow["{p_1''}"', from=1-5, to=3-5]
			\arrow[Rightarrow, no head, from=3-5, to=3-6]
			\arrow["{c\times M}"', dashed, from=1-1, to=2-2]
			\arrow["{p_1'}"', from=1-1, to=3-1]
			\arrow["c", from=3-1, to=3-2]
			\arrow["s", from=3-2, to=3-3]
			\arrow["{p_1}"', from=2-2, to=3-2]
			\arrow["{p_2}", from=2-2, to=2-3]
			\arrow["t", from=2-3, to=3-3]
			\arrow["{p_2'}", from=1-1, to=1-3]
			\arrow[Rightarrow, no head, from=1-3, to=2-3]
		\end{tikzcd}\]
		where the inner and outer squares in both diagrams are pullback squares. Furthermore, assuming the diagrams in condition (1) above are satisfied, we have that $t\circ p_1\circ p_2''=t\circ c\circ p_2''=s\circ p_1''$, so that by the universal property of the pullback we have a map $M\times p_1:M\times_O(M\times_OM)\to M\times_OM$ like so:
		% https://q.uiver.app/#q=WzAsNSxbMCwwLCJNXFx0aW1lc19PKE1cXHRpbWVzX09NKSJdLFsxLDEsIk1cXHRpbWVzX09NIl0sWzIsMSwiTSJdLFsyLDIsIk8iXSxbMSwyLCJNIl0sWzAsMSwiTVxcdGltZXMgcF8xIiwxLHsic3R5bGUiOnsiYm9keSI6eyJuYW1lIjoiZGFzaGVkIn19fV0sWzEsMiwicF8yIl0sWzIsMywidCJdLFsxLDQsInBfMSIsMl0sWzQsMywicyJdLFswLDQsInBfMScnIiwyLHsiY3VydmUiOjR9XSxbMCwyLCJwXzFcXGNpcmMgcF8yJyciLDAseyJjdXJ2ZSI6LTR9XV0=
		\[\begin{tikzcd}
			{M\times_O(M\times_OM)} \\
			& {M\times_OM} & M \\
			& M & O
			\arrow["{M\times p_1}"{description}, dashed, from=1-1, to=2-2]
			\arrow["{p_2}", from=2-2, to=2-3]
			\arrow["t", from=2-3, to=3-3]
			\arrow["{p_1}"', from=2-2, to=3-2]
			\arrow["s", from=3-2, to=3-3]
			\arrow["{p_1''}"', curve={height=24pt}, from=1-1, to=3-2]
			\arrow["{p_1\circ p_2''}", curve={height=-24pt}, from=1-1, to=2-3]
		\end{tikzcd}\]
		Now note that again assuming the diagrams above in $(1)$ commute, we have $s\circ c=s\circ p_2$, so that
		\[s\circ c\circ (M\times p_1)=s\circ p_2\circ(M\times p_1)=s\circ p_1\circ p_2''=t\circ p_2\circ p_2''.\] 
		Then by the unviersal property of the pullback we get a map $a:M\times_O(M\times_OM)\to (M\times_OM)\times_OM$ like so:
		% https://q.uiver.app/#q=WzAsNyxbMCwwLCJNXFx0aW1lc19PKE1cXHRpbWVzX09NKSJdLFsxLDEsIihNXFx0aW1lc19PTSlcXHRpbWVzX09NIl0sWzEsMywiTVxcdGltZXNfT00iXSxbMiwzLCJNIl0sWzMsMywiTyJdLFszLDIsIk0iXSxbMywxLCJNIl0sWzEsMiwicF8xJyIsMl0sWzIsMywiYyJdLFswLDEsImEiLDAseyJzdHlsZSI6eyJib2R5Ijp7Im5hbWUiOiJkYXNoZWQifX19XSxbMCwyLCJNXFx0aW1lcyBwXzEiLDIseyJjdXJ2ZSI6NH1dLFszLDQsInMiXSxbNSw0LCJ0Il0sWzEsNiwicF8yJyJdLFs2LDUsIiIsMCx7ImxldmVsIjoyLCJzdHlsZSI6eyJoZWFkIjp7Im5hbWUiOiJub25lIn19fV0sWzAsNiwicF8yXFxjaXJjIHBfMicnIiwwLHsiY3VydmUiOi00fV1d
		\[\begin{tikzcd}
			{M\times_O(M\times_OM)} \\
			& {(M\times_OM)\times_OM} && M \\
			&&& M \\
			& {M\times_OM} & M & O
			\arrow["{p_1'}"', from=2-2, to=4-2]
			\arrow["c", from=4-2, to=4-3]
			\arrow["a", dashed, from=1-1, to=2-2]
			\arrow["{M\times p_1}"', curve={height=24pt}, from=1-1, to=4-2]
			\arrow["s", from=4-3, to=4-4]
			\arrow["t", from=3-4, to=4-4]
			\arrow["{p_2'}", from=2-2, to=2-4]
			\arrow[Rightarrow, no head, from=2-4, to=3-4]
			\arrow["{p_2\circ p_2''}", curve={height=-24pt}, from=1-1, to=2-4]
		\end{tikzcd}\]
		Exercise: Show that this map $a$ is an isomorphism. Then we require that the following diagram commutes:
		% https://q.uiver.app/#q=WzAsNSxbMCwwLCJNXFx0aW1lc19PKE1cXHRpbWVzX09NKSJdLFsyLDAsIihNXFx0aW1lc19PTSlcXHRpbWVzX09NIl0sWzAsMSwiTVxcdGltZXNfT00iXSxbMSwxLCJNIl0sWzIsMSwiTVxcdGltZXNfT00iXSxbMCwxLCJhIl0sWzAsMiwiTVxcdGltZXMgYyIsMl0sWzIsMywiYyJdLFsxLDQsImNcXHRpbWVzIE0iXSxbNCwzLCJjIiwyXV0=
		\[\begin{tikzcd}
			{M\times_O(M\times_OM)} && {(M\times_OM)\times_OM} \\
			{M\times_OM} & M & {M\times_OM}
			\arrow["a", from=1-1, to=1-3]
			\arrow["{M\times c}"', from=1-1, to=2-1]
			\arrow["c", from=2-1, to=2-2]
			\arrow["{c\times M}", from=1-3, to=2-3]
			\arrow["c"', from=2-3, to=2-2]
		\end{tikzcd}\]
		Expressed in terms of sets, this diagram says $h\circ(g\circ f)=(h\circ g)\circ f$.
		\item Unitality of composition: Given the maps $(\id_M,e\circ t),(e\circ s,\id_M):M\to M\times_OM$ defined by the universal property of $M\times_OM$:
		\[\begin{tikzcd}
			M &&& M \\
			& {M\times_OM} & M && {M\times_OM} & M \\
			& M & O && M & O
			\arrow["{(\id_M,e\circ s)}", dashed, from=1-1, to=2-2]
			\arrow["{p_2}", from=2-2, to=2-3]
			\arrow["t", from=2-3, to=3-3]
			\arrow["{p_1}"', from=2-2, to=3-2]
			\arrow["s", from=3-2, to=3-3]
			\arrow["{e\circ s}", curve={height=-18pt}, from=1-1, to=2-3]
			\arrow[curve={height=18pt}, Rightarrow, no head, from=1-1, to=3-2]
			\arrow["{(e\circ t,\id_M)}", dashed, from=1-4, to=2-5]
			\arrow["{p_2}", from=2-5, to=2-6]
			\arrow["t", from=2-6, to=3-6]
			\arrow["{p_1}"', from=2-5, to=3-5]
			\arrow["s", from=3-5, to=3-6]
			\arrow[curve={height=-18pt}, Rightarrow, no head, from=1-4, to=2-6]
			\arrow["{e\circ t}"', curve={height=18pt}, from=1-4, to=3-5]
			\arrow["\lrcorner"{anchor=center, pos=0.125}, draw=none, from=2-2, to=3-3]
			\arrow["\lrcorner"{anchor=center, pos=0.125}, draw=none, from=2-5, to=3-6]
		\end{tikzcd}\]
		the following diagram commutes:
		% https://q.uiver.app/#q=WzAsNCxbMCwwLCJNIl0sWzIsMiwiTSJdLFsyLDAsIk1cXHRpbWVzX09NIl0sWzAsMiwiTVxcdGltZXNfT00iXSxbMCwxLCIiLDAseyJsZXZlbCI6Miwic3R5bGUiOnsiaGVhZCI6eyJuYW1lIjoibm9uZSJ9fX1dLFswLDIsIihlXFxjaXJjIHQsXFxpZF9NKSJdLFsyLDEsImMiXSxbMCwzLCIoXFxpZF9NLGVcXGNpcmMgcykiLDJdLFszLDEsImMiLDJdXQ==
		\[\begin{tikzcd}
			M && {M\times_OM} \\
			\\
			{M\times_OM} && M
			\arrow[Rightarrow, no head, from=1-1, to=3-3]
			\arrow["{(e\circ t,\id_M)}", from=1-1, to=1-3]
			\arrow["c", from=1-3, to=3-3]
			\arrow["{(\id_M,e\circ s)}"', from=1-1, to=3-1]
			\arrow["c"', from=3-1, to=3-3]
		\end{tikzcd}\]
		Expressed in terms of sets, this diagram says that given $f\in M$ with $s(f)=x$ and $t(f)=y$, that $f\circ\id_x=f$ and $\id_y\circ f=f$.
		\item Inverse: The following diagrams must commute:
		\[\begin{tikzcd}
			& M & M & {M\times_OM} & M && M \\
			M & M & O & M & O & O & M & O
			\arrow["{(\id_M,i)}", from=1-3, to=1-4]
			\arrow["t"', from=1-3, to=2-3]
			\arrow["e", from=2-3, to=2-4]
			\arrow["c", from=1-4, to=2-4]
			\arrow["{(i,\id_M)}"', from=1-5, to=1-4]
			\arrow["s", from=1-5, to=2-5]
			\arrow["e"', from=2-5, to=2-4]
			\arrow["i", from=1-2, to=2-2]
			\arrow["i", from=1-7, to=2-7]
			\arrow["s"', from=1-7, to=2-6]
			\arrow["t"', from=2-7, to=2-6]
			\arrow["t", from=1-7, to=2-8]
			\arrow["s", from=2-7, to=2-8]
			\arrow["i"', from=2-2, to=2-1]
			\arrow[Rightarrow, no head, from=1-2, to=2-1]
		\end{tikzcd}\]
		where the arrows $(\id_M,i)$ and $(i,\id_M)$ are determined by the universal property of $M\times_OM$ like so:
		\[\begin{tikzcd}
			M &&& M \\
			& {M\times_OM} & M && {M\times_OM} & M \\
			& M & O && M & O
			\arrow["{(i,\id_M)}", dashed, from=1-4, to=2-5]
			\arrow["{p_2}", from=2-5, to=2-6]
			\arrow["t", from=2-6, to=3-6]
			\arrow["{p_1}"', from=2-5, to=3-5]
			\arrow["s", from=3-5, to=3-6]
			\arrow[curve={height=-18pt}, Rightarrow, no head, from=1-4, to=2-6]
			\arrow["i"', curve={height=18pt}, from=1-4, to=3-5]
			\arrow["{(\id_M,i)}", dashed, from=1-1, to=2-2]
			\arrow["{p_2}", from=2-2, to=2-3]
			\arrow["t", from=2-3, to=3-3]
			\arrow["{p_1}"', from=2-2, to=3-2]
			\arrow["s", from=3-2, to=3-3]
			\arrow["i", curve={height=-18pt}, from=1-1, to=2-3]
			\arrow[curve={height=18pt}, Rightarrow, no head, from=1-1, to=3-2]
		\end{tikzcd}\]
		Expressed in terms of sets, given $f\in M$ with $s(f)=x$ and $t(f)=y$, the first diagram says that ${(f^{-1})}^{-1}=f$, the second says that $f\circ f^{-1}=\id_y$ and $f^{-1}\circ f=\id_x$, and the last diagram says that the domain and codomain of $f^{-1}$ are $y$ and $x$, respectively.
	\end{enumerate}
\end{definition}

It can be seen that groupoid objects in $\cC=\Set$ are precisely (small) groupoids. Now, we can state and unravel the definition of a Hopf algebroid:

\begin{definition}\label{hopf_algebroid_defn}
	Given an $A$-graded anticommutative ring $R$ (\autoref{A-graded_anticommutative_ring_defn}), an \emph{$A$-graded anticommutative Hopf algebroid over $R$} is a co-groupoid object in $R\text-\GCA^{A}$, i.e., a groupoid object in ${(R\text-A\GrCAlg)}^\op$. Explicitly, an $A$-graded anticommutative Hopf algebroid over $E$ is a pair $(\Gamma,B)$ of objects in $R\text-A\GrCAlg$ along with morphisms
	\begin{enumerate}
		\item \emph{left unit}: $\eta_L:B\to\Gamma$ (corresponding to $t$),
		\item \emph{right unit}: $\eta_R:B\to\Gamma$ (corresponding to $s$),
		\item \emph{comultiplication}: $\Psi:\Gamma\to\Gamma\otimes_B\Gamma$ (corresponding to $c$),
		\item \emph{counit}: $\epsilon:\Gamma\to B$ (corresponding to $e$),
		\item \emph{conjugation}: $c:\Gamma\to\Gamma$ (corresponding to $i$),
	\end{enumerate}
	where here $\Gamma$ may be viewed as a $B$-bimodule with left $B$-module structure induced by $\eta_L$ and right $B$-module structure induced by $\eta_R$, so we may form the tensor product of bimodules $\Gamma\otimes_B\Gamma$, which further may be given the structure of an $A$-graded anticommutative $R$-algebra (by \autoref{B-tensor_product_in_R-GrCAlg}), and fits into the following pushout diagram in $R\text-\GCA^{A}g$ (\autoref{R-GrCAlg_has_pushouts_and_binary_coproducts}):
	% https://q.uiver.app/#q=WzAsNCxbMCwwLCJCIl0sWzAsMSwiXFxHYW1tYSJdLFsxLDEsIlxcR2FtbWFcXG90aW1lc19CXFxHYW1tYSJdLFsxLDAsIlxcR2FtbWEiXSxbMCwxLCJcXGV0YV9SIiwyXSxbMSwyLCJnXFxtYXBzdG8gZ1xcb3RpbWVzIDEiLDJdLFswLDMsIlxcZXRhX0wiXSxbMywyLCJnXFxtYXBzdG8gMVxcb3RpbWVzIGciXV0=
	\[\begin{tikzcd}
		B & \Gamma \\
		\Gamma & {\Gamma\otimes_B\Gamma}
		\arrow["{\eta_R}"', from=1-1, to=2-1]
		\arrow["{g\mapsto g\otimes 1}"', from=2-1, to=2-2]
		\arrow["{\eta_L}", from=1-1, to=1-2]
		\arrow["{g\mapsto 1\otimes g}", from=1-2, to=2-2]
	\end{tikzcd}\]
	These data must make the following diagrams commute:
	\begin{enumerate}
		\item (Composition works correctly)
		% https://q.uiver.app/#q=WzAsMTIsWzgsMSwiXFxHYW1tYVxcb3RpbWVzX0JcXEdhbW1hIl0sWzcsMCwiQiJdLFs4LDAsIlxcR2FtbWEiXSxbNywxLCJcXEdhbW1hIl0sWzQsMSwiQiJdLFs0LDAsIkIiXSxbNSwxLCJcXEdhbW1hIl0sWzMsMSwiXFxHYW1tYSJdLFswLDAsIkIiXSxbMCwxLCJcXEdhbW1hIl0sWzEsMSwiXFxHYW1tYVxcb3RpbWVzX0JcXEdhbW1hIl0sWzEsMCwiXFxHYW1tYSJdLFs0LDUsIiIsMCx7ImxldmVsIjoyLCJzdHlsZSI6eyJoZWFkIjp7Im5hbWUiOiJub25lIn19fV0sWzgsOSwiXFxldGFfTCIsMl0sWzksMTAsImdcXG1hcHN0byBnXFxvdGltZXMgMSIsMl0sWzgsMTEsIlxcZXRhX0wiXSxbMTEsMTAsIlxcUHNpIl0sWzUsNywiXFxldGFfUiIsMl0sWzcsNCwiXFxlcHNpbG9uIiwyXSxbNSw2LCJcXGV0YV9MIl0sWzYsNCwiXFxlcHNpbG9uIl0sWzEsMiwiXFxldGFfUiJdLFsyLDAsImdcXG1hcHN0bzFcXG90aW1lcyBnIl0sWzEsMywiXFxldGFfUiIsMl0sWzMsMCwiXFxQc2kiXV0=
		\[\begin{tikzcd}
			B & \Gamma &&& B &&& B & \Gamma \\
			\Gamma & {\Gamma\otimes_B\Gamma} && \Gamma & B & \Gamma && \Gamma & {\Gamma\otimes_B\Gamma}
			\arrow[Rightarrow, no head, from=2-5, to=1-5]
			\arrow["{\eta_L}"', from=1-1, to=2-1]
			\arrow["{g\mapsto g\otimes 1}"', from=2-1, to=2-2]
			\arrow["{\eta_L}", from=1-1, to=1-2]
			\arrow["\Psi", from=1-2, to=2-2]
			\arrow["{\eta_R}"', from=1-5, to=2-4]
			\arrow["\epsilon"', from=2-4, to=2-5]
			\arrow["{\eta_L}", from=1-5, to=2-6]
			\arrow["\epsilon", from=2-6, to=2-5]
			\arrow["{\eta_R}", from=1-8, to=1-9]
			\arrow["{g\mapsto1\otimes g}", from=1-9, to=2-9]
			\arrow["{\eta_R}"', from=1-8, to=2-8]
			\arrow["\Psi", from=2-8, to=2-9]
		\end{tikzcd}\]
		\item (Coassociativity) The following diagram must commute
		% https://q.uiver.app/#q=WzAsNSxbMiwwLCJcXEdhbW1hXFxvdGltZXNfQlxcR2FtbWEiXSxbMSwwLCJcXEdhbW1hIl0sWzAsMCwiXFxHYW1tYVxcb3RpbWVzX0JcXEdhbW1hIl0sWzAsMSwiKFxcR2FtbWFcXG90aW1lc19CXFxHYW1tYSlcXG90aW1lc19CXFxHYW1tYSJdLFsyLDEsIlxcR2FtbWFcXG90aW1lc19CKFxcR2FtbWFcXG90aW1lc19CXFxHYW1tYSkiXSxbMSwyLCJcXFBzaSIsMl0sWzIsMywiXFxQc2lcXG90aW1lc19CXFxHYW1tYSIsMl0sWzAsNCwiXFxHYW1tYVxcb3RpbWVzX0JcXFBzaSJdLFsxLDAsIlxcUHNpIl0sWzMsNCwiXFxjb25nIl1d
		\[\begin{tikzcd}
			{\Gamma\otimes_B\Gamma} & \Gamma & {\Gamma\otimes_B\Gamma} \\
			{(\Gamma\otimes_B\Gamma)\otimes_B\Gamma} && {\Gamma\otimes_B(\Gamma\otimes_B\Gamma)}
			\arrow["\Psi"', from=1-2, to=1-1]
			\arrow["{\Psi\otimes_B\Gamma}"', from=1-1, to=2-1]
			\arrow["{\Gamma\otimes_B\Psi}", from=1-3, to=2-3]
			\arrow["\Psi", from=1-2, to=1-3]
			\arrow["\cong", from=2-1, to=2-3]
		\end{tikzcd}\]
		where $(\Gamma\otimes_B\Gamma)\otimes_B\Gamma$ and $\Gamma\otimes_B(\Gamma\otimes_B\Gamma)$ denote the rings which fit into the following pushout diagrams in $R\text-\GCA^{A}$:
		% https://q.uiver.app/#q=WzAsMTAsWzIsMiwiKFxcR2FtbWFcXG90aW1lc19CXFxHYW1tYSlcXG90aW1lc19CXFxHYW1tYSJdLFswLDAsIkIiXSxbMiwwLCJcXEdhbW1hIl0sWzAsMSwiXFxHYW1tYSJdLFswLDIsIlxcR2FtbWFcXG90aW1lc19CXFxHYW1tYSJdLFs0LDAsIkIiXSxbNCwyLCJcXEdhbW1hIl0sWzUsMCwiXFxHYW1tYSJdLFs2LDAsIlxcR2FtbWFcXG90aW1lc19CXFxHYW1tYSJdLFs2LDIsIlxcR2FtbWFcXG90aW1lc19CKFxcR2FtbWFcXG90aW1lc19CXFxHYW1tYSkiXSxbMiwwLCJnXFxtYXBzdG8oMVxcb3RpbWVzIDEpXFxvdGltZXMgZyJdLFsxLDMsIlxcZXRhX1IiLDJdLFszLDQsIlxcUHNpIiwyXSxbNCwwLCJnXFxvdGltZXMgZydcXG1hcHN0byhnXFxvdGltZXMgZycpXFxvdGltZXMgMSIsMl0sWzEsMiwiXFxldGFfTCJdLFs1LDYsIlxcZXRhX1IiLDJdLFs1LDcsIlxcZXRhX0wiXSxbNyw4LCJcXFBzaSJdLFs4LDksIihnXFxvdGltZXMgZycpXFxtYXBzdG8gMVxcb3RpbWVzKGdcXG90aW1lcyBnJykiXSxbNiw5LCJnXFxtYXBzdG8gZ1xcb3RpbWVzKDFcXG90aW1lcyAxKSIsMl1d
		\[\begin{tikzcd}[column sep=scriptsize]
			B && \Gamma && B & \Gamma & {\Gamma\otimes_B\Gamma} \\
			\Gamma \\
			{\Gamma\otimes_B\Gamma} && {(\Gamma\otimes_B\Gamma)\otimes_B\Gamma} && \Gamma && {\Gamma\otimes_B(\Gamma\otimes_B\Gamma)}
			\arrow["{g\mapsto(1\otimes 1)\otimes g}", from=1-3, to=3-3]
			\arrow["{\eta_R}"', from=1-1, to=2-1]
			\arrow["\Psi"', from=2-1, to=3-1]
			\arrow["{g\otimes g'\mapsto(g\otimes g')\otimes 1}"', from=3-1, to=3-3]
			\arrow["{\eta_L}", from=1-1, to=1-3]
			\arrow["{\eta_R}"', from=1-5, to=3-5]
			\arrow["{\eta_L}", from=1-5, to=1-6]
			\arrow["\Psi", from=1-6, to=1-7]
			\arrow["{(g\otimes g')\mapsto 1\otimes(g\otimes g')}", from=1-7, to=3-7]
			\arrow["{g\mapsto g\otimes(1\otimes 1)}"', from=3-5, to=3-7]
		\end{tikzcd}\]
		and the isomorphism $(\Gamma\otimes_B\Gamma)\otimes_B\Gamma\to\Gamma\otimes_B(\Gamma\otimes_B\Gamma)$ sends $(g\otimes g')\otimes g''$ to $g\otimes(g'\otimes g'')$, the left vertical arrow $\Psi\otimes\Gamma$ sends $g\otimes g'$ to $\Psi(g)\otimes g$, and the right vertical arrow $\Gamma\otimes\Psi$ sends $g\otimes g'$ to $g\otimes\Psi(g')$ .
		\item (Co-unitality):
		% https://q.uiver.app/#q=WzAsNCxbMiwyLCJcXEdhbW1hIl0sWzAsMCwiXFxHYW1tYSJdLFsyLDAsIlxcR2FtbWFcXG90aW1lc19CXFxHYW1tYSJdLFswLDIsIlxcR2FtbWFcXG90aW1lc19CXFxHYW1tYSJdLFswLDEsIiIsMCx7ImxldmVsIjoyLCJzdHlsZSI6eyJoZWFkIjp7Im5hbWUiOiJub25lIn19fV0sWzIsMCwiKFxcZXRhX0xcXGNpcmNcXGVwc2lsb24pXFxjZG90XFxpZF9cXEdhbW1hIl0sWzEsMiwiXFxQc2kiXSxbMywwLCJcXGlkX1xcR2FtbWFcXGNkb3QoXFxldGFfUlxcY2lyY1xcZXBzaWxvbikiLDJdLFsxLDMsIlxcUHNpIiwyXV0=
		\[\begin{tikzcd}
			\Gamma && {\Gamma\otimes_B\Gamma} \\
			\\
			{\Gamma\otimes_B\Gamma} && \Gamma
			\arrow[Rightarrow, no head, from=3-3, to=1-1]
			\arrow["{(\eta_L\circ\epsilon)\cdot\id_\Gamma}", from=1-3, to=3-3]
			\arrow["\Psi", from=1-1, to=1-3]
			\arrow["{\id_\Gamma\cdot(\eta_R\circ\epsilon)}"', from=3-1, to=3-3]
			\arrow["\Psi"', from=1-1, to=3-1]
		\end{tikzcd}\]
		where the right vertical arrow sends $g\otimes g'$ to $\eta_L(\epsilon(g))g'$ and the bottom horizontal arrow sends $g\otimes g'$ to $g\eta_R(\epsilon(g'))$.
		\item (Convolution):
		% https://q.uiver.app/#q=WzAsMTMsWzEsMCwiXFxHYW1tYSJdLFsxLDEsIlxcR2FtbWEiXSxbMCwxLCJcXEdhbW1hIl0sWzMsMSwiXFxHYW1tYSJdLFs0LDEsIlxcR2FtbWFcXG90aW1lc19CXFxHYW1tYSJdLFs0LDAsIlxcR2FtbWEiXSxbMywwLCJCIl0sWzUsMCwiQiJdLFs1LDEsIlxcR2FtbWEiXSxbOCwxLCJcXEdhbW1hIl0sWzgsMCwiXFxHYW1tYSJdLFs3LDAsIkIiXSxbOSwwLCJCIl0sWzAsMSwiYyJdLFsxLDIsImMiXSxbMCwyLCIiLDIseyJsZXZlbCI6Miwic3R5bGUiOnsiaGVhZCI6eyJuYW1lIjoibm9uZSJ9fX1dLFs0LDMsIlxcaWRfXFxHYW1tYVxcY2RvdCBjIl0sWzUsNiwiXFxlcHNpbG9uIiwyXSxbNiwzLCJcXGV0YV9MIiwyXSxbNSw0LCJpIiwyXSxbNSw3LCJcXGVwc2lsb24iXSxbNyw4LCJcXGV0YV9SIl0sWzQsOCwiY1xcY2RvdFxcaWRfXFxHYW1tYSIsMl0sWzExLDEwLCJcXGV0YV9MIl0sWzEyLDEwLCJcXGV0YV9SIiwyXSxbMTAsOSwiYyIsMl0sWzExLDksIlxcZXRhX1IiLDJdLFsxMiw5LCJcXGV0YV9MIl1d
		\[\begin{tikzcd}
			& \Gamma && B & \Gamma & B && B & \Gamma & B \\
			\Gamma & \Gamma && \Gamma & {\Gamma\otimes_B\Gamma} & \Gamma &&& \Gamma
			\arrow["c", from=1-2, to=2-2]
			\arrow["c", from=2-2, to=2-1]
			\arrow[Rightarrow, no head, from=1-2, to=2-1]
			\arrow["{\id_\Gamma\cdot c}", from=2-5, to=2-4]
			\arrow["\epsilon"', from=1-5, to=1-4]
			\arrow["{\eta_L}"', from=1-4, to=2-4]
			\arrow["i"', from=1-5, to=2-5]
			\arrow["\epsilon", from=1-5, to=1-6]
			\arrow["{\eta_R}", from=1-6, to=2-6]
			\arrow["{c\cdot\id_\Gamma}"', from=2-5, to=2-6]
			\arrow["{\eta_L}", from=1-8, to=1-9]
			\arrow["{\eta_R}"', from=1-10, to=1-9]
			\arrow["c"', from=1-9, to=2-9]
			\arrow["{\eta_R}"', from=1-8, to=2-9]
			\arrow["{\eta_L}", from=1-10, to=2-9]
		\end{tikzcd}\]
		where the bottom left arrow in the middle diagram sends $g\otimes g'$ to $gc(g')$ and the bottom right arrow in the middle diagram sends $g\otimes g'$ to $c(g)g'$.
	\end{enumerate}
\end{definition}

The remainder of this subsection is devoted to proving some technical lemmas about $A$-graded anticommutative Hopf algebroids.

\begin{proposition}\label{G_ox_B_G_has_one_interpretation_as_B-bimodule}
	Suppose we have an $A$-graded anticommutative Hopf algebroid $(\Gamma,B)$ over $(R,\theta)$ with structure maps $\eta_L$, $\eta_R$, $\Psi$, $\epsilon$, and $c$ (\autoref{hopf_algebroid_defn}). Recall in the definition, we considered $\Gamma\otimes_B\Gamma$ to be the $A$-graded $R$-commutative ring whose underlying abelian group was given by the tensor product of $B$-bimodules, where $\Gamma$ has left $B$-module structure induced by $\eta_L$ and right $B$-module structure induced by $\eta_R$. Thus $\Gamma\otimes_B\Gamma$ is canonically a $B$-bimodule, as it is a tensor product of $B$-bimodules. Then the canonical left (resp.\ right) $B$-module structure on $\Gamma\otimes_B\Gamma$ coincides with that induced by the ring homomorphism $\Psi\circ\eta_L$ (resp.\ $\Psi\circ\eta_R$).
\end{proposition}
\begin{proof}
	First we show the left module structures coincide. By additivity, in order to show the module structures coincide, it suffices to show that given a homogeneous pure tensor $g\otimes g'$ in $\Gamma\otimes_B\Gamma$ and some $b\in B$ that $\Psi(\eta_L(b))\cdot(g\otimes g')=(\eta_L(b)\cdot g)\otimes g'$, where $\cdot$ on the left denotes the product in $\Gamma\otimes_B\Gamma$ and the $\cdot$ on the right denotes the product in $\Gamma$. By the axioms for a Hopf algebroid, we have that $\Psi(\eta_L(b))=\eta_L(b)\otimes 1$. Thus by how the product in $\Gamma\otimes_B\Gamma$ is defined (\autoref{B-tensor_product_in_R-GrCAlg}), we have that
	\[\Psi(\eta_L(b))\cdot(g\otimes g')=(\eta_L(b)\otimes 1)\cdot(g\otimes g')=(\varphi_\Gamma(\theta_{0,|g|})\cdot \eta_L(b)\cdot g)\otimes (g'\cdot 1)=(\eta_L(b)\cdot g)\otimes g',\]
	where $\varphi_\Gamma:R\to \Gamma$ is the structure map, and the last equality follows by the fact that $\theta_{0,|g|}=1$. An entirely analagous argument yields that the canonical right module structure on $\Gamma\otimes_B\Gamma$ coincides with that induced by $\Psi\circ\eta_R$, since $\Psi\circ\eta_R=1\otimes\eta_R$.
\end{proof}

\begin{remark}
	By the above proposition, given an $A$-graded commutative Hopf algebroid $(\Gamma,B)$ over $R$, there is no ambiguity when discussing the objects $\Gamma\otimes_B(\Gamma\otimes_B\Gamma)$ and $(\Gamma\otimes_B\Gamma)\otimes_B\Gamma$ --- they may both be considered as the threefold tensor product of the $B$-bimodule $\Gamma$ with itself. In particular, we have a canonical isomorphism of $B$-bimodules
	\[(\Gamma\otimes_B\Gamma)\otimes_B\Gamma\to\Gamma\otimes_B(\Gamma\otimes_B\Gamma)\]
	sending $(g\otimes g')\otimes g''$ to $g\otimes(g'\otimes g'')$, and this is precisely the isomorphism in the coassociativity diagram in the definition of a Hopf algebroid (\autoref{hopf_algebroid_defn}).
\end{remark}

\begin{proposition}\label{hopf_algebroid_structure_maps_are_module_homos}
	Suppose we have an $A$-graded commutative Hopf algebroid $(\Gamma,B)$ over $R$ with structure maps $\eta_L$, $\eta_R$, $\Psi$, $\epsilon$, and $c$. Then $\eta_L:B\to\Gamma$ is a homomorphism of left $B$-modules, $\eta_R:B\to\Gamma$ is a homomorphism of right $B$-modules, and $\Psi:\Gamma\to\Gamma\otimes_B\Gamma$ and $\epsilon:\Gamma\to B$ are homomorphisms of $B$-bimodules.
\end{proposition}
\begin{proof}
	Since the left (resp.\ right) $B$-module structure on $\Gamma$ is induced by $\eta_L$ (resp.\ $\eta_R$), the map $\eta_L$ (resp.\ $\eta_R$) is a homomorphism of left (resp.\ right) $B$-modules by definition.

	Next, we want to show $\Psi$ is a homomorphism of $B$-bimodules. The left (resp.\ right) $B$-module structure on $\Gamma$ is that induced by $\eta_L$ (resp.\ $\eta_R$), and in \autoref{G_ox_B_G_has_one_interpretation_as_B-bimodule}, we showed that the left (resp.\ right) $B$-module structure on $\Gamma\otimes_B\Gamma$ is that induced by $\Psi\circ\eta_L$ (resp.\ $\Psi\circ\eta_R$), so that by definition $\Psi:\Gamma\to\Gamma\otimes_B\Gamma$ is a homomorphism of left (resp.\ right) $B$-modules.

	Lastly, we claim that $\epsilon:\Gamma\to B$ is a homomorphism of $B$-bimodules. We need to show that given $g\in\Gamma$ and $b,b'\in B$ that $\epsilon(\eta_L(b)g\eta_R(g'))=b\epsilon(g)b'$. This follows from the fact that $\epsilon$ is a ring homomorphism satisfying $\epsilon\circ\eta_L=\epsilon\circ\eta_R=\id_B$. 
\end{proof}

\subsection{Comodules over a Hopf algebroid}

In what follows, fix an $A$-graded anticommutative ring $(R,\theta)$ and an $A$-graded anticommutative Hopf algebroid $(\Gamma,B)$ over $R$ with structure maps $\eta_L$, $\eta_R$, $\Psi$, $\epsilon$, and $c$. We will always view $\Gamma$ with its \emph{canonical} $B$-bimodule structure, with left $B$-module structure induced by $\eta_L$, and right $B$-module structure induced by $\eta_R$. In particular, any tensor product over $B$ involving $\Gamma$ will always refer to $\Gamma$ with this bimodule structure.

\begin{definition}\label{left_comodule_defn}
	A \emph{left comodule over $\Gamma$} is a pair $(N,\Psi_N)$, where $N$ is a left $A$-graded $B$-module and $\Psi_N:N\to\Gamma\otimes_BN$ is an $A$-graded homomorphism of left $A$-graded $B$-modules. These data are required to make the following diagrams commute
	% https://q.uiver.app/#q=WzAsOCxbMCwwLCJOIl0sWzEsMCwiXFxHYW1tYVxcb3RpbWVzX0JOIl0sWzEsMSwiQlxcb3RpbWVzX0JOIl0sWzQsMCwiTiJdLFszLDAsIlxcR2FtbWFcXG90aW1lc19CTiJdLFszLDEsIihcXEdhbW1hXFxvdGltZXNfQlxcR2FtbWEpXFxvdGltZXNfQk4iXSxbNSwwLCJcXEdhbW1hXFxvdGltZXNfQk4iXSxbNSwxLCJcXEdhbW1hXFxvdGltZXNfQihcXEdhbW1hXFxvdGltZXNfQk4pIl0sWzAsMSwiXFxQc2lfTiJdLFsxLDIsIlxcZXBzaWxvblxcb3RpbWVzIE4iXSxbMCwyLCJcXGNvbmciLDJdLFszLDQsIlxcUHNpX04iLDJdLFs0LDUsIlxcUHNpXFxvdGltZXMgTiIsMl0sWzMsNiwiXFxQc2lfTiJdLFs2LDcsIlxcR2FtbWFcXG90aW1lc1xcUHNpX04iXSxbNSw3LCJcXGNvbmciXV0=
	\[\begin{tikzcd}
		N & {\Gamma\otimes_BN} && {\Gamma\otimes_BN} & N & {\Gamma\otimes_BN} \\
		& {B\otimes_BN} && {(\Gamma\otimes_B\Gamma)\otimes_BN} && {\Gamma\otimes_B(\Gamma\otimes_BN)}
		\arrow["{\Psi_N}", from=1-1, to=1-2]
		\arrow["{\epsilon\otimes N}", from=1-2, to=2-2]
		\arrow["\cong"', from=1-1, to=2-2]
		\arrow["{\Psi_N}"', from=1-5, to=1-4]
		\arrow["{\Psi\otimes N}"', from=1-4, to=2-4]
		\arrow["{\Psi_N}", from=1-5, to=1-6]
		\arrow["{\Gamma\otimes\Psi_N}", from=1-6, to=2-6]
		\arrow["\cong", from=2-4, to=2-6]
	\end{tikzcd}\]
	The maps $\epsilon\otimes N$ and $\Psi\otimes N$ are well-defined by \autoref{hopf_algebroid_structure_maps_are_module_homos}, and the bottom isomorphism in the right diagram is the canonical one sending $(g\otimes g')\otimes n\mapsto g\otimes(g'\otimes n)$.

	Given two left $A$-graded $\Gamma$-comodules $(N_1,\Psi_{N_1})$ and $(N_2,\Psi_{N_2})$, a homomorphism of left $A$-graded comodules $f:N_1\to N_2$ is an $A$-graded homomorphism of the underlying left $B$-modules such that the following diagram commutes:
	% https://q.uiver.app/#q=WzAsNCxbMCwwLCJOXzEiXSxbMSwwLCJOXzIiXSxbMSwxLCJcXEdhbW1hXFxvdGltZXNfQk5fMiJdLFswLDEsIlxcR2FtbWFcXG90aW1lc19CTl8xIl0sWzAsMSwiZiJdLFsxLDIsIlxcUHNpX3tOXzJ9Il0sWzAsMywiXFxQc2lfe05fMX0iLDJdLFszLDIsIlxcR2FtbWFcXG90aW1lcyBmIl1d
	\[\begin{tikzcd}
		{N_1} & {N_2} \\
		{\Gamma\otimes_BN_1} & {\Gamma\otimes_BN_2}
		\arrow["f", from=1-1, to=1-2]
		\arrow["{\Psi_{N_2}}", from=1-2, to=2-2]
		\arrow["{\Psi_{N_1}}"', from=1-1, to=2-1]
		\arrow["{\Gamma\otimes f}", from=2-1, to=2-2]
	\end{tikzcd}\]

	We write $\Gamma\text-\CoMod^A$ for the resulting category of left $A$-graded comodules over $\Gamma$. In the above definition, we required $A$-graded left $\Gamma$-comodule homomorphisms to strictly preserve the grading, but we could have instead considered left $\Gamma$-comodule homomorphisms which are of degree $d$ for some $d\in A$, or equivalently, the set of degree zero $A$-graded $\Gamma$-comodule homomorphisms from $N_1$ to the shifted comodule ${(N_2)}_{*+d}$. We denote the hom-set of degree-$d$ $A$-graded left $\Gamma$-comodule homomorphisms from $(N_1,\Psi_{N_1})$ to $(N_2,\Psi_{N_2})$ by
	\[\Hom_{\Gamma\text-\CoMod^A}^d(N_1,N_2)\qquad\text{or usually just}\qquad\Hom_{\Gamma}^d(N_1,N_2).\]
	In particular, we simply write $\Hom_{\Gamma\text-\CoMod^A}(N_1,N_2)$ or $\Hom_{\Gamma}(N_1,N_2)$ for the set of strictly degree preserving (degree $0$) $A$-graded left $\Gamma$-comodule homomorphisms from $(N_1,\Psi_{N_1})$ to $(N_2,\Psi_{N_2})$.
\end{definition}

\begin{proposition}\label{G-CoMod^A_is_additive}
	The category $\Gamma\text-\CoMod^A$ is an additive category.
\end{proposition}
\begin{proof}
	First, we show the category is $\Ab$-enriched. Since the forgetful functor $\Gamma\text-\CoMod^A\to B\text-\Mod^A$ is clearly faithful, we may view hom-sets in $\Gamma\text-\CoMod^A$ as subsets of hom-groups in $B\text-\Mod^A$, so that in order to show $\Gamma\text-\CoMod^A$ is $\Ab$-enriched, it suffices to show that hom-sets in $\Gamma\text-\CoMod^A$ are closed under addition of module homomorphisms and taking inverses. To that end, suppose we have two $A$-graded left $\Gamma$-comodule homomorphisms $f,g:(N_1,\Psi_{N_1})\to(N_2,\Psi_{N_2})$, then we have
	\begin{align*}
		\Psi_{N_2}\circ(f+g)&=(\Psi_{N_2}\circ f)+(\Psi_{N_2}\circ g) \\
		&=((\Gamma\otimes_Bf)\circ\Psi_{N_1})+((\Gamma\otimes_Bg)\circ\Psi_{N_1}) \\
		&=((\Gamma\otimes_Bf)+(\Gamma\otimes_Bg))\circ\Psi_{N_1} \\
		&=(\Gamma\otimes_B(f+g))\circ\Psi_{N_1},
	\end{align*}
	where the first equality follows since $\Psi_{N_2}$ is a homomorphism, the second follows since $f$ and $g$ are left $\Gamma$-comodule homomorphisms, the third follows since $\Psi_{N_1}$ is a homomorphism, and the last equality follows by definition of the tensor product of modules. Hence $f+g$ is indeed an $A$-graded left $\Gamma$-comodule homomorphism, as desired. Now, we also claim $-f$ is an $A$-graded left $\Gamma$-comodule homomorphism. To that end, note that
	\[\Psi_{N_2}\circ(-f)=-\Psi_{N_2}\circ f=-(\Gamma\otimes_Bf)\circ\Psi_{N_1}=(\Gamma\otimes_B(-f))\circ\Psi_{N_1},\]
	where the first equality follows since $\Psi_{N_2}$ is a homomorphism, the second follows since $f$ is an $A$-graded left $\Gamma$-comodule homomorphism, and the third equality follows by definition of the tensor product.

	Thus, we've shown that the hom-sets in $\Gamma\text-\CoMod^A$ are abelian groups, and composition is clearly bilinear, so that $\Gamma\text-\CoMod^A$ is indeed $\Ab$-enriched.

	Now, in order to show $\Gamma\text-\CoMod^A$ is additive, it suffices to show that it contains a zero object and has binary coproducts. First of all, it is straightforward to check that the zero left $B$-module is clearly an $A$-graded left $\Gamma$-comodule with structure map the unique map $0\to\Gamma\otimes_B0\cong 0$, and that given any other $A$-graded left $\Gamma$-comodule $(N,\Psi_N)$, the unique homomorphisms of left $B$-modules $0\to N$ and $N\to 0$ are left comodule homomorphisms.

	Now, suppose we have two $A$-graded left $\Gamma$-comodules $(N_1,\Psi_{N_1})$ and $(N_2,\Psi_{N_2})$. First, we claim their direct sum as left $B$-modules $N_1\oplus N_2$ is canonically an $A$-graded left $\Gamma$-comodule. We know that $N_1\oplus N_2$ is an $A$-graded left $B$-module by \autoref{product_of_A_graded}, and we can define the structure map
	\[\Psi_{N_1\oplus N_2}:N_1\oplus N_2\xr{\Psi_{N_1}\oplus\Psi_{N_2}}(\Gamma\otimes_BN_1)\oplus(\Gamma\otimes_BN_2)\cong\Gamma\otimes_B(N_1\oplus N_2),\]
	where the final isomorphism is the canonical one sending $(g_1\otimes n_1)\oplus(g_2\otimes n_2)$ to $(g_1\otimes n_1)+(g_2\otimes n_2)$. Then
	\todo{finish}
\end{proof}

\begin{proposition}\label{comodule_co-free_adjunction}
	The forgetful functor $\Gamma\text-\CoMod^A\to B\text-\Mod^A$ (where here $B\text-\Mod^A$ is the category of $A$-graded left $B$-modules and degree-preserving module homomorphisms between them) has a right adjoint $\Gamma\otimes_B-:B\text-\Mod^A\to\Gamma\text-\CoMod^A$ called the \emph{co-free construction}, where the co-free left $A$-graded $\Gamma$-comodule on a left $A$-graded $B$-module $M$ is the $B$-module $\Gamma\otimes_BM$ equipped with the coaction
	\[\Psi_{\Gamma\otimes_BM}:\Gamma\otimes_BM\xr{\Psi\otimes_BM}(\Gamma\otimes_B\Gamma)\otimes_BM\xr\cong\Gamma\otimes_B(\Gamma\otimes_BM).\]

%	In particular, for all objects $(N,\Psi_N)$ in $\Gamma\text-\CoMod^A$ and $M$ in $B\text-\Mod^A$, we have isomorphisms of $A$-graded abelian groups 
%	\[\Hom_{B\text-\Mod^A}^*(N,M)\cong\Hom_{\Gamma\text-\CoMod^A}^*(N,\Gamma\otimes_BM)\]
%	via the identifications
%	\[\Hom_{\Gamma\text-\CoMod^A}^a(N_*,N_*')=\Hom_{\Gamma\text-\CoMod^A}(N_{*-a},N_{*}')\]
%	and
%	\[\Hom_{B\text-\Mod^A}^a(M_*,M_*')=\Hom_{B\text-\Mod^A}(M_{*-a},M_{*}').\]
%
	Explicitly, given some $(N,\Psi_N)$ in $\Gamma\text-\CoMod$ and some $M$ in $B\text-\Mod^A$, the counit and unit of this adjunction are given by
	\[\eta_{(N,\Psi_N)}:N\xr{\Psi_N}\Gamma\otimes_BN\]
	and
	\[\vare_M:\Gamma\otimes_BM\xr{\epsilon\otimes_BM}B\otimes_BM\xr\cong M.\]
\end{proposition}
\begin{proof}
	\todo{todo}
\end{proof}

\begin{proposition}\label{G-CoMod^A_is_abelian_if_eta_R_flat}
	Suppose that $\Gamma$ is flat as a right $B$-module, i.e., suppose $\eta_R:B\to\Gamma$ is a flat ring homomorphism. Then the category $\Gamma\text-\CoMod^A$ is an abelian category. 
\end{proposition}
\begin{proof}
	\todo{finish}
\end{proof}

\end{document}
