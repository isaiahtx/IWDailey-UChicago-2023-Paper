\documentclass[../main.tex]{subfiles}
\begin{document}

\subsection{Triangulated categories and their basic properties}

\begin{definition}\label{triangulated_defn}
    A \emph{triangulated category} $(\cC,\Sigma,\cD)$ is the data of:\begin{enumerate}
        \item An additive category $\cC$.
        \item An additive auto-equivalence $\Sigma:\cC\to\cC$ called the \emph{shift functor}.
        %\item An adjoint pair of additive functors $\Sigma,\Omega:\cC\to\cC$ such that the unit $\eta:\Id_\cC\Rightarrow\Omega\Sigma$ and counit $\vare:\Sigma\Omega\Rightarrow\Id_\cC$ are natural isomorphisms, i.e., the tuple $(\Sigma,\Omega,\eta,\vare)$ forms an adjoint autoequivalence of $\cC$. Usually $\Sigma$ is called the \emph{shift functor}.
        \item A collection $\cD$ of \emph{distinguished} triangles in $\cC$, where a \emph{triangle} is a sequence of arrows of the form
        \[X\to Y\to Z\to\Sigma X.\]
        Distinguished triangles are also sometimes called \emph{cofiber sequences} or \emph{fiber sequences}.
    \end{enumerate}
    These data must satisfy the following axioms:
    \begin{enumerate}[label={\textbf{TR\arabic*}}]
        \setcounter{enumi}{-1}
        \item Given a commutative diagram
        % https://q.uiver.app/#q=WzAsOCxbMCwwLCJYIl0sWzEsMCwiWSJdLFsyLDAsIloiXSxbMywwLCJcXFNpZ21hIFgiXSxbMCwxLCJYJyJdLFsxLDEsIlknIl0sWzIsMSwiWiciXSxbMywxLCJcXFNpZ21hIFgnIl0sWzAsMV0sWzEsMl0sWzIsM10sWzAsNCwiXFxjb25nIiwyXSxbNCw1XSxbNSw2XSxbNiw3XSxbMSw1LCJcXGNvbmciLDJdLFsyLDYsIlxcY29uZyIsMl0sWzMsNywiXFxjb25nIiwyXV0=
        \[\begin{tikzcd}
            X & Y & Z & {\Sigma X} \\
            {X'} & {Y'} & {Z'} & {\Sigma X'}
            \arrow[from=1-1, to=1-2]
            \arrow[from=1-2, to=1-3]
            \arrow[from=1-3, to=1-4]
            \arrow["\cong"', from=1-1, to=2-1]
            \arrow[from=2-1, to=2-2]
            \arrow[from=2-2, to=2-3]
            \arrow[from=2-3, to=2-4]
            \arrow["\cong"', from=1-2, to=2-2]
            \arrow["\cong"', from=1-3, to=2-3]
            \arrow["\cong"', from=1-4, to=2-4]
        \end{tikzcd}\]
        where the vertical arrows are isomorphisms, if the top row is distinguished then so is the bottom.
        \item For any object $X$ in $\cC$, the diagram
        \[X\xrightarrow{\id_X}X\to0\to\Sigma X\]
        is a distinguished triangle.
        \item For all $f:X\to Y$ there exists an object $C_f$ (also sometimes denoted $Y/X$) called the \emph{cofiber of $f$} and a distinguished triangle
        \[X\xrightarrow fY\to C_f\to\Sigma X.\]
        \item Given a solid diagram
        % https://q.uiver.app/#q=WzAsOCxbMCwwLCJYIl0sWzEsMCwiWSJdLFsyLDAsIloiXSxbMywwLCJcXFNpZ21hIFgiXSxbMCwxLCJYJyJdLFsxLDEsIlknIl0sWzIsMSwiWiciXSxbMywxLCJcXFNpZ21hIFgnIl0sWzAsMV0sWzEsMl0sWzIsM10sWzAsNCwiZiIsMl0sWzQsNV0sWzUsNl0sWzYsN10sWzEsNV0sWzIsNiwiIiwxLHsic3R5bGUiOnsiYm9keSI6eyJuYW1lIjoiZGFzaGVkIn19fV0sWzMsNywiXFxTaWdtYSBmIl1d
        \[\begin{tikzcd}
            X & Y & Z & {\Sigma X} \\
            {X'} & {Y'} & {Z'} & {\Sigma X'}
            \arrow[from=1-1, to=1-2]
            \arrow[from=1-2, to=1-3]
            \arrow[from=1-3, to=1-4]
            \arrow["f"', from=1-1, to=2-1]
            \arrow[from=2-1, to=2-2]
            \arrow[from=2-2, to=2-3]
            \arrow[from=2-3, to=2-4]
            \arrow[from=1-2, to=2-2]
            \arrow[dashed, from=1-3, to=2-3]
            \arrow["{\Sigma f}", from=1-4, to=2-4]
        \end{tikzcd}\]
        such that the leftmost square commmutes and both rows are distinguished, there exists a dashed arrow $Z\to Z'$ which makes the remaining two squares commute.
        \item A triangle
        \[X\xrightarrow fY\xrightarrow gZ\xrightarrow h\Sigma X\]
        is distinguished if and only if
        \[Y\xrightarrow gZ\xrightarrow h\Sigma X\xrightarrow{-\Sigma f}\Sigma Y\]
        is distinguished.
        \item (Octahedral axiom) Given three distinguished triangles
        \begin{gather*}
            X\xrightarrow fY\xrightarrow h Y/X\to\Sigma X \\
            Y\xrightarrow gZ\xrightarrow k Z/Y\to\Sigma Y \\
            X\xrightarrow{g\circ f}Z\xrightarrow lZ/X\to\Sigma X
        \end{gather*}
        there exists a distinguished triangle
        \[Y/X\xrightarrow uZ/X\xrightarrow vZ/Y\xrightarrow w\Sigma(Y/X)\]
        such that the following diagram commutes
        % https://q.uiver.app/#q=WzAsOSxbMCwwLCJYIl0sWzIsMCwiWiJdLFs0LDAsIlovWSJdLFs2LDAsIlxcU2lnbWEoWS9YKSJdLFsxLDEsIlkiXSxbMywxLCJaL1giXSxbNSwxLCJcXFNpZ21hIFkiXSxbMiwyLCJZL1giXSxbNCwyLCJcXFNpZ21hIFgiXSxbMCwxLCJnXFxjaXJjIGYiXSxbMSwyLCJrIl0sWzIsMywidyJdLFswLDQsImYiLDJdLFs0LDcsImgiLDJdLFs3LDhdLFs4LDYsIlxcU2lnbWEgZiIsMl0sWzYsMywiXFxTaWdtYSBoIiwyXSxbMSw1LCJsIiwyXSxbNSwyLCJ2IiwyXSxbNyw1LCJ1IiwyXSxbNSw4XSxbNCwxLCJnIiwyXSxbMiw2XV0=
        \[\begin{tikzcd}
            X && Z && {Z/Y} && {\Sigma(Y/X)} \\
            & Y && {Z/X} && {\Sigma Y} \\
            && {Y/X} && {\Sigma X}
            \arrow["{g\circ f}", from=1-1, to=1-3]
            \arrow["k", from=1-3, to=1-5]
            \arrow["w", from=1-5, to=1-7]
            \arrow["f"', from=1-1, to=2-2]
            \arrow["h"', from=2-2, to=3-3]
            \arrow[from=3-3, to=3-5]
            \arrow["{\Sigma f}"', from=3-5, to=2-6]
            \arrow["{\Sigma h}"', from=2-6, to=1-7]
            \arrow["l"', from=1-3, to=2-4]
            \arrow["v"', from=2-4, to=1-5]
            \arrow["u"', from=3-3, to=2-4]
            \arrow[from=2-4, to=3-5]
            \arrow["g"', from=2-2, to=1-3]
            \arrow[from=1-5, to=2-6]
        \end{tikzcd}\]
    \end{enumerate}
\end{definition}

It turns out that the above definition is actually redundant; TR3 and TR4 follow from the remaining axioms (see Lemmas 2.2 and 2.4 in \cite{MayTri}). From now on, we fix a triangulated category $(\cC,\Sigma,\cD)$. We will denote the hom-group $\cC(X,Y)$ by $[X,Y]$. To start, recall the following definition:

\begin{definition}\label{defn_exact}
    A sequence
    \[X_1\to X_2\to\cdots\to X_n\]
    of arrows in $\cC$ is \emph{exact} if, for any object $A$ in $\cC$, the induced sequences
    \[[A,X_1]\to[A,X_2]\to\cdots\to[A,X_{n-1}]\to[A,X_n]\]
    and
    \[[X_n,A]\to[X_{n-1},A]\to\cdots\to[X_2,A]\to[X_1,A]\]
    are exact sequences of abelian groups.
\end{definition}

It is straightforward to verify that if we have an exact sequence in $\cC$ 
\[X_1\xr{f_1}X_2\xr{f_2}\to\cdots\to X_n,\]
then the sequence remains exact if we change the signs of any of the maps involved. We will use this fact often without comment.

\begin{proposition}\label{distinguished_tri_is_exact}
    Every distinguished triangle is an exact sequence (in the sense of \autoref{defn_exact}).
\end{proposition}
\begin{proof}
    Suppose we have some distinguished triangle
    \[X\xrightarrow fY\xrightarrow gZ\xrightarrow h\Sigma X.\]
    Then first we would like to show that given any object $A$ in $\cC$, the sequence
    \[[A,X]\xrightarrow{f_*}[A,Y]\xrightarrow{g_*}[A,Z]\xrightarrow{h_*}[A,\Sigma X]\]
    is exact. First we show exactness at $[A,Y]$. To see $\imm f_*\sseq\ker g_*$, note it suffices to show that $g\circ f=0$. Indeed, consider the commuting diagram
    % https://q.uiver.app/#q=WzAsOCxbMCwxLCJYIl0sWzEsMSwiWSJdLFsyLDEsIloiXSxbMywxLCJcXFNpZ21hIFgiXSxbMCwwLCJYIl0sWzEsMCwiWCJdLFsyLDAsIjAiXSxbMywwLCJcXFNpZ21hIFgiXSxbMCwxLCJmIl0sWzEsMiwiZyJdLFsyLDMsImgiXSxbNCw1LCIiLDAseyJsZXZlbCI6Miwic3R5bGUiOnsiaGVhZCI6eyJuYW1lIjoibm9uZSJ9fX1dLFs1LDEsImYiXSxbNSw2XSxbNiw3XSxbNCwwLCIiLDEseyJsZXZlbCI6Miwic3R5bGUiOnsiaGVhZCI6eyJuYW1lIjoibm9uZSJ9fX1dXQ==
    \[\begin{tikzcd}
        X & X & 0 & {\Sigma X} \\
        X & Y & Z & {\Sigma X}
        \arrow["f", from=2-1, to=2-2]
        \arrow["g", from=2-2, to=2-3]
        \arrow["h", from=2-3, to=2-4]
        \arrow[Rightarrow, no head, from=1-1, to=1-2]
        \arrow["f", from=1-2, to=2-2]
        \arrow[from=1-2, to=1-3]
        \arrow[from=1-3, to=1-4]
        \arrow[Rightarrow, no head, from=1-1, to=2-1]
    \end{tikzcd}\]
    The top row is distinguished by axiom TR1. Thus by TR3, the following diagram commutes:
    % https://q.uiver.app/#q=WzAsOCxbMCwxLCJYIl0sWzEsMSwiWSJdLFsyLDEsIloiXSxbMywxLCJcXFNpZ21hIFgiXSxbMCwwLCJYIl0sWzEsMCwiWCJdLFsyLDAsIjAiXSxbMywwLCJcXFNpZ21hIFgiXSxbMCwxLCJmIl0sWzEsMiwiZyJdLFsyLDMsImgiXSxbNCw1LCIiLDAseyJsZXZlbCI6Miwic3R5bGUiOnsiaGVhZCI6eyJuYW1lIjoibm9uZSJ9fX1dLFs1LDEsImYiXSxbNSw2XSxbNiw3XSxbNCwwLCIiLDEseyJsZXZlbCI6Miwic3R5bGUiOnsiaGVhZCI6eyJuYW1lIjoibm9uZSJ9fX1dLFs2LDJdLFs3LDMsIiIsMSx7ImxldmVsIjoyLCJzdHlsZSI6eyJoZWFkIjp7Im5hbWUiOiJub25lIn19fV1d
    \[\begin{tikzcd}
        X & X & 0 & {\Sigma X} \\
        X & Y & Z & {\Sigma X}
        \arrow["f", from=2-1, to=2-2]
        \arrow["g", from=2-2, to=2-3]
        \arrow["h", from=2-3, to=2-4]
        \arrow[Rightarrow, no head, from=1-1, to=1-2]
        \arrow["f", from=1-2, to=2-2]
        \arrow[from=1-2, to=1-3]
        \arrow[from=1-3, to=1-4]
        \arrow[Rightarrow, no head, from=1-1, to=2-1]
        \arrow[from=1-3, to=2-3]
        \arrow[Rightarrow, no head, from=1-4, to=2-4]
    \end{tikzcd}\]
    In particular, commutativity of the second square tells us that $g\circ f=0$, as desired. Conversely, we'd like to show that $\ker g_*\sseq\imm f_*$. Let $\psi: A\to Y$ be in the kernel of $g_*$, so that $g\circ\psi=0$. Consider the following commutative diagram:
    % https://q.uiver.app/#q=WzAsOCxbMCwwLCJBIl0sWzEsMCwiMCJdLFsyLDAsIlxcU2lnbWEgQSJdLFszLDAsIlxcU2lnbWEgQSJdLFswLDEsIlkiXSxbMSwxLCJaIl0sWzIsMSwiXFxTaWdtYSBYIl0sWzMsMSwiXFxTaWdtYSBZIl0sWzAsMV0sWzEsMl0sWzIsMywiLVxcU2lnbWFcXGlkX0EiXSxbNCw1LCJnIl0sWzUsNiwiaCJdLFs2LDcsIi1cXFNpZ21hIGYiXSxbMCw0LCJcXHBzaSIsMl0sWzEsNV1d
    \[\begin{tikzcd}
        A & 0 & {\Sigma A} & {\Sigma A} \\
        Y & Z & {\Sigma X} & {\Sigma Y}
        \arrow[from=1-1, to=1-2]
        \arrow[from=1-2, to=1-3]
        \arrow["{-\Sigma\id_A}", from=1-3, to=1-4]
        \arrow["g", from=2-1, to=2-2]
        \arrow["h", from=2-2, to=2-3]
        \arrow["{-\Sigma f}", from=2-3, to=2-4]
        \arrow["\psi"', from=1-1, to=2-1]
        \arrow[from=1-2, to=2-2]
    \end{tikzcd}\]
    The top row is distinguished by axioms TR1 and TR4. The bottom row is distinguished by axiom TR4. Thus by axiom TR3 there exists a map $\wt\phi:\Sigma A\to \Sigma X$ such that the following diagram commutes:
    % https://q.uiver.app/#q=WzAsOCxbMCwwLCJBIl0sWzEsMCwiMCJdLFsyLDAsIlxcU2lnbWEgQSJdLFszLDAsIlxcU2lnbWEgQSJdLFswLDEsIlkiXSxbMSwxLCJaIl0sWzIsMSwiXFxTaWdtYSBYIl0sWzMsMSwiXFxTaWdtYSBZIl0sWzAsMV0sWzEsMl0sWzIsMywiLVxcU2lnbWFcXGlkX0EiXSxbNCw1LCJnIl0sWzUsNiwiaCJdLFs2LDcsIi1cXFNpZ21hIGYiXSxbMCw0LCJcXHBzaSIsMl0sWzEsNV0sWzIsNiwiXFx3dFxccGhpIiwyXSxbMyw3LCJcXFNpZ21hXFxwc2kiLDJdXQ==
    \[\begin{tikzcd}
        A & 0 & {\Sigma A} & {\Sigma A} \\
        Y & Z & {\Sigma X} & {\Sigma Y}
        \arrow[from=1-1, to=1-2]
        \arrow[from=1-2, to=1-3]
        \arrow["{-\Sigma\id_A}", from=1-3, to=1-4]
        \arrow["g", from=2-1, to=2-2]
        \arrow["h", from=2-2, to=2-3]
        \arrow["{-\Sigma f}", from=2-3, to=2-4]
        \arrow["\psi"', from=1-1, to=2-1]
        \arrow[from=1-2, to=2-2]
        \arrow["\wt\phi"', from=1-3, to=2-3]
        \arrow["\Sigma\psi"', from=1-4, to=2-4]
    \end{tikzcd}\]
    Now, since $\Sigma$ is an equivalence, it is a full functor, so that in particular there exists some $\phi:A\to X$ such that $\wt\phi=\Sigma\phi$. Then by faithfullness, we may pull back the right square to get a commuting diagram 
    % https://q.uiver.app/#q=WzAsNCxbMCwwLCJBIl0sWzEsMCwiQSJdLFsxLDEsIlkiXSxbMCwxLCJYIl0sWzAsMSwiLVxcaWRfQSJdLFsxLDIsIlxccHNpIl0sWzAsMywiXFxwaGkiLDJdLFszLDIsIi1mIiwyXV0=
    \[\begin{tikzcd}
        A & A \\
        X & Y
        \arrow["{-\id_A}", from=1-1, to=1-2]
        \arrow["\psi", from=1-2, to=2-2]
        \arrow["\phi"', from=1-1, to=2-1]
        \arrow["{-f}"', from=2-1, to=2-2]
    \end{tikzcd}\]
    Hence, 
    \[f_*(\phi)=f\circ\phi\overset{(*)}=-((-f)\circ\phi)=-(\psi\circ(-\id_A))\overset{(*)}=\psi\circ\id_A=\psi,\]
    where the equalities marked $(*)$ follow by bilinearity of composition in an additive category. Thus $\psi\in\imm f_*$, as desired, meaning $\ker g_*\sseq\imm f_*$.

    Now, we have shown that
    \[[A,X]\xrightarrow{f_*}[A,Y]\xrightarrow{g_*}[A,Z]\xrightarrow{h_*}[A,\Sigma X]\]
    is exact at $[A,Y]$. It remains to show exactness at $[A,Z]$. Yet this follows by the exact same argument given above applied to the sequence obtained from the shifted triangle (TR4)
    \[Y\xrightarrow g Z\xrightarrow h\Sigma X\xrightarrow{-\Sigma f}\Sigma Y\]

    On the other hand, we would like to show that
    \[[\Sigma X,A]\xr{h^*}[Z,A]\xr{g^*}[Y,A]\xr{f^*}[X,A]\]
    is exact. As above, since we can shift the triangle, it suffices to show exactness at $[Z,A]$. First, since we have shown $g\circ f=0$, we have $f^*\circ g^*=(g\circ f)^*=0$, so that $\imm g^*\sseq\ker f^*$, as desired. Conversely, in order to see $\ker f^*\sseq\imm g^*$, suppose $\psi:Y\to A$ is in the kernel of $f^*$, so that $\psi\circ f=0$. Consider the following commuting diagram:
    % https://q.uiver.app/#q=WzAsOCxbMCwwLCJYIl0sWzEsMCwiWSJdLFsyLDAsIloiXSxbMywwLCJcXFNpZ21hIFgiXSxbMCwxLCIwIl0sWzEsMSwiQSJdLFsyLDEsIkEiXSxbMywxLCIwIl0sWzAsMSwiZiJdLFsxLDIsImciXSxbMiwzLCJoIl0sWzAsNF0sWzQsNV0sWzUsNiwiIiwyLHsibGV2ZWwiOjIsInN0eWxlIjp7ImhlYWQiOnsibmFtZSI6Im5vbmUifX19XSxbNiw3XSxbMSw1LCJcXHBzaSJdXQ==
    \[\begin{tikzcd}
        X & Y & Z & {\Sigma X} \\
        0 & A & A & 0
        \arrow["f", from=1-1, to=1-2]
        \arrow["g", from=1-2, to=1-3]
        \arrow["h", from=1-3, to=1-4]
        \arrow[from=1-1, to=2-1]
        \arrow[from=2-1, to=2-2]
        \arrow[Rightarrow, no head, from=2-2, to=2-3]
        \arrow[from=2-3, to=2-4]
        \arrow["\psi", from=1-2, to=2-2]
    \end{tikzcd}\]
    The top row is a distinguished triangle by assumption, and the bottom row is distinguished by axioms TR1 and TR4 for a triangulated category, along with the fact that $\Sigma0=0$ since $\Sigma $ is additive. Thus by axiom TR3 there exists a map $\phi:Z\to A$ such that $\phi\circ g=\psi$, i.e., $g^*(\phi)=\psi$, so that $\phi\in\imm g^*$ as desired. 
\end{proof}

\begin{lemma}\label{2-of-3-dist_tri-lemma}
    Suppose we have a commutative diagram
    % https://q.uiver.app/#q=WzAsOCxbMCwwLCJYIl0sWzEsMCwiWSJdLFsyLDAsIloiXSxbMywwLCJcXFNpZ21hIFgiXSxbMCwxLCJYJyJdLFsxLDEsIlknIl0sWzIsMSwiWiciXSxbMywxLCJcXFNpZ21hIFgnIl0sWzAsMSwiZiJdLFsxLDIsImciXSxbMiwzLCJoIl0sWzQsNSwiZiciXSxbNSw2LCJnJyJdLFs2LDcsImgnIl0sWzAsNCwiaiJdLFsxLDUsImsiXSxbMiw2LCJcXGVsbCJdLFszLDcsIlxcU2lnbWEgaiJdXQ==
    \[\begin{tikzcd}
        X & Y & Z & {\Sigma X} \\
        {X'} & {Y'} & {Z'} & {\Sigma X'}
        \arrow["f", from=1-1, to=1-2]
        \arrow["g", from=1-2, to=1-3]
        \arrow["h", from=1-3, to=1-4]
        \arrow["{f'}", from=2-1, to=2-2]
        \arrow["{g'}", from=2-2, to=2-3]
        \arrow["{h'}", from=2-3, to=2-4]
        \arrow["j", from=1-1, to=2-1]
        \arrow["k", from=1-2, to=2-2]
        \arrow["\ell", from=1-3, to=2-3]
        \arrow["{\Sigma j}", from=1-4, to=2-4]
    \end{tikzcd}\]
    with both rows distinguished. Then if any two of the maps $j$, $k$, and $\ell$ are isomorphisms, then so is the third.
\end{lemma}
\begin{proof}
    Suppose we are given any object $W$ in $\cC$, and consider the commutative diagram
    % https://q.uiver.app/#q=WzAsMTQsWzAsMCwiW1csWF0iXSxbMSwwLCJbVyxZXSJdLFsyLDAsIltXLFpdIl0sWzMsMCwiW1csXFxTaWdtYSBYXSJdLFs0LDAsIltXLFxcU2lnbWEgWV0iXSxbMCwxLCJbVyxYJ10iXSxbMSwxLCJbVyxZJ10iXSxbMiwxLCJbVyxaJ10iXSxbMywxLCJbVyxcXFNpZ21hIFgnXSJdLFs0LDEsIltXLFxcU2lnbWEgWSddIl0sWzUsMCwiW1csXFxTaWdtYSBaXSJdLFs1LDEsIltXLFxcU2lnbWEgWiddIl0sWzYsMSwiW1csXFxTaWdtYV4yWCddIl0sWzYsMCwiW1csXFxTaWdtYV4yWF0iXSxbMCwxLCJmXyoiXSxbMSwyLCJnXyoiXSxbMiwzLCJrXyoiXSxbMyw0LCItXFxTaWdtYSBmXyoiXSxbMCw1LCJqXyoiXSxbNSw2LCJmXyonIl0sWzYsNywiZ18qJyJdLFs3LDgsImhfKiciXSxbOCw5LCItXFxTaWdtYSBmXyonIl0sWzEsNiwia18qIl0sWzMsOCwiXFxTaWdtYSBqXyoiXSxbNCw5LCJcXFNpZ21hIGtfKiJdLFs0LDEwLCItXFxTaWdtYSBnXyoiXSxbOSwxMSwiLVxcU2lnbWEgZ18qJyJdLFsxMSwxMiwiLVxcU2lnbWEgaF8qJyJdLFsxMCwxMywiLVxcU2lnbWEgaF8qIl0sWzEwLDExLCJcXFNpZ21hXFxlbGxfKiJdLFsxMywxMiwiXFxTaWdtYV4yal8qIl0sWzIsNywiXFxlbGxfKiJdXQ==
    \[\begin{tikzcd}[column sep=tiny]
        {[W,X]} & {[W,Y]} & {[W,Z]} & {[W,\Sigma X]} & {[W,\Sigma Y]} & {[W,\Sigma Z]} & {[W,\Sigma^2X]} \\
        {[W,X']} & {[W,Y']} & {[W,Z']} & {[W,\Sigma X']} & {[W,\Sigma Y']} & {[W,\Sigma Z']} & {[W,\Sigma^2X']}
        \arrow["{f_*}", from=1-1, to=1-2]
        \arrow["{g_*}", from=1-2, to=1-3]
        \arrow["{k_*}", from=1-3, to=1-4]
        \arrow["{-\Sigma f_*}", from=1-4, to=1-5]
        \arrow["{j_*}", from=1-1, to=2-1]
        \arrow["{f_*'}", from=2-1, to=2-2]
        \arrow["{g_*'}", from=2-2, to=2-3]
        \arrow["{h_*'}", from=2-3, to=2-4]
        \arrow["{-\Sigma f_*'}", from=2-4, to=2-5]
        \arrow["{k_*}", from=1-2, to=2-2]
        \arrow["{\Sigma j_*}", from=1-4, to=2-4]
        \arrow["{\Sigma k_*}", from=1-5, to=2-5]
        \arrow["{-\Sigma g_*}", from=1-5, to=1-6]
        \arrow["{-\Sigma g_*'}", from=2-5, to=2-6]
        \arrow["{-\Sigma h_*'}", from=2-6, to=2-7]
        \arrow["{-\Sigma h_*}", from=1-6, to=1-7]
        \arrow["{\Sigma\ell_*}", from=1-6, to=2-6]
        \arrow["{\Sigma^2j_*}", from=1-7, to=2-7]
        \arrow["{\ell_*}", from=1-3, to=2-3]
    \end{tikzcd}\]
    The rows are exact by \autoref{distinguished_tri_is_exact} and repeated applications of axiom TR4. It follows by the five lemma and faithfulness of $\Sigma$ that if $j$ and $k$ are isomorphisms, then $\ell_*$ is an isomorphism. Similarly, if $k$ and $\ell$ are isomorphisms then $\Sigma j_*$ is an isomorphism. Finally, if $\ell$ and $j$ are isomorphisms, then $\Sigma k_*$ is an isomorphism. The desired result follows by faithfullness of $\Sigma$ and the Yoneda embedding.
\end{proof}

%\begin{proposition}\label{cofiber_unique_up_to_isomorphism}
%    Given a map $f:X\to Y$ in a triangulated category $(\cC,\Sigma,\Omega,\cD)$, the cofiber sequence of $f$ is unique up to isomorphism, in the sense that given any two distinguished triangles
%    \[X\xrightarrow fY\to Z\to\Sigma X\qquad\text{and}\qquad X\xrightarrow fY\to Z'\to\Sigma X,\]
%    there exists an isomorphism $Z\to Z'$ which makes the following diagram commute:
%    % https://q.uiver.app/#q=WzAsOCxbMCwwLCJYIl0sWzEsMCwiWSJdLFsyLDAsIloiXSxbMywwLCJcXFNpZ21hIFgiXSxbMywxLCJcXFNpZ21hIFgiXSxbMCwxLCJYIl0sWzEsMSwiWSJdLFsyLDEsIlonIl0sWzAsMSwiZiJdLFsxLDJdLFsyLDNdLFszLDQsIiIsMCx7ImxldmVsIjoyLCJzdHlsZSI6eyJoZWFkIjp7Im5hbWUiOiJub25lIn19fV0sWzAsNSwiIiwyLHsibGV2ZWwiOjIsInN0eWxlIjp7ImhlYWQiOnsibmFtZSI6Im5vbmUifX19XSxbNSw2LCJmIl0sWzYsN10sWzEsNiwiIiwxLHsibGV2ZWwiOjIsInN0eWxlIjp7ImhlYWQiOnsibmFtZSI6Im5vbmUifX19XSxbMiw3LCJrIiwwLHsic3R5bGUiOnsiYm9keSI6eyJuYW1lIjoiZGFzaGVkIn19fV0sWzcsNF1d
%    \[\begin{tikzcd}
%        X & Y & Z & {\Sigma X} \\
%        X & Y & {Z'} & {\Sigma X}
%        \arrow["f", from=1-1, to=1-2]
%        \arrow[from=1-2, to=1-3]
%        \arrow[from=1-3, to=1-4]
%        \arrow[Rightarrow, no head, from=1-4, to=2-4]
%        \arrow[Rightarrow, no head, from=1-1, to=2-1]
%        \arrow["f", from=2-1, to=2-2]
%        \arrow[from=2-2, to=2-3]
%        \arrow[Rightarrow, no head, from=1-2, to=2-2]
%        \arrow["k", dashed, from=1-3, to=2-3]
%        \arrow[from=2-3, to=2-4]
%    \end{tikzcd}\]
%\end{proposition}
%\begin{proof}
%    Suppose we have two distinguished triangles
%    \[X\xrightarrow fY\xrightarrow{g} Z\xrightarrow{h}\Sigma X\qquad\text{and}\qquad X\xrightarrow fY\xrightarrow{g'} Z'\xrightarrow{h'}\Sigma X,\]
%    and consider the following commutative diagram
%    % https://q.uiver.app/#q=WzAsOCxbMCwwLCJYIl0sWzEsMCwiWSJdLFsyLDAsIloiXSxbMywwLCJcXFNpZ21hIFgiXSxbMCwxLCJYIl0sWzEsMSwiWSJdLFsyLDEsIlonIl0sWzMsMSwiXFxTaWdtYSBYIl0sWzAsMSwiZiJdLFsxLDIsImciXSxbMiwzLCJoIl0sWzAsNCwiIiwyLHsibGV2ZWwiOjIsInN0eWxlIjp7ImhlYWQiOnsibmFtZSI6Im5vbmUifX19XSxbNCw1LCJmIl0sWzUsNiwiZyciXSxbNiw3LCJoJyJdLFsxLDUsIiIsMSx7ImxldmVsIjoyLCJzdHlsZSI6eyJoZWFkIjp7Im5hbWUiOiJub25lIn19fV1d
%    \[\begin{tikzcd}
%        X & Y & Z & {\Sigma X} \\
%        X & Y & {Z'} & {\Sigma X}
%        \arrow["f", from=1-1, to=1-2]
%        \arrow["g", from=1-2, to=1-3]
%        \arrow["h", from=1-3, to=1-4]
%        \arrow[Rightarrow, no head, from=1-1, to=2-1]
%        \arrow["f", from=2-1, to=2-2]
%        \arrow["{g'}", from=2-2, to=2-3]
%        \arrow["{h'}", from=2-3, to=2-4]
%        \arrow[Rightarrow, no head, from=1-2, to=2-2]
%    \end{tikzcd}\]
%    By axiom TR3, there exists some map $k:Z\to Z'$ which makes the following diagram commute:
%    % https://q.uiver.app/#q=WzAsOCxbMCwwLCJYIl0sWzEsMCwiWSJdLFsyLDAsIloiXSxbMywwLCJcXFNpZ21hIFgiXSxbMCwxLCJYIl0sWzEsMSwiWSJdLFsyLDEsIlonIl0sWzMsMSwiXFxTaWdtYSBYIl0sWzAsMSwiZiJdLFsxLDIsImciXSxbMiwzLCJoIl0sWzAsNCwiIiwyLHsibGV2ZWwiOjIsInN0eWxlIjp7ImhlYWQiOnsibmFtZSI6Im5vbmUifX19XSxbNCw1LCJmIl0sWzUsNiwiZyciXSxbNiw3LCJoJyJdLFsxLDUsIiIsMSx7ImxldmVsIjoyLCJzdHlsZSI6eyJoZWFkIjp7Im5hbWUiOiJub25lIn19fV0sWzIsNiwiayJdLFszLDcsIiIsMSx7ImxldmVsIjoyLCJzdHlsZSI6eyJoZWFkIjp7Im5hbWUiOiJub25lIn19fV1d
%    \[\begin{tikzcd}
%        X & Y & Z & {\Sigma X} \\
%        X & Y & {Z'} & {\Sigma X}
%        \arrow["f", from=1-1, to=1-2]
%        \arrow["g", from=1-2, to=1-3]
%        \arrow["h", from=1-3, to=1-4]
%        \arrow[Rightarrow, no head, from=1-1, to=2-1]
%        \arrow["f", from=2-1, to=2-2]
%        \arrow["{g'}", from=2-2, to=2-3]
%        \arrow["{h'}", from=2-3, to=2-4]
%        \arrow[Rightarrow, no head, from=1-2, to=2-2]
%        \arrow["k", from=1-3, to=2-3]
%        \arrow[Rightarrow, no head, from=1-4, to=2-4]
%    \end{tikzcd}\]
%    Now, by \autoref{2-of-3-dist_tri-lemma}, $k$ is an isomorphism.
%\end{proof}

\begin{proposition}\label{fiber}
    Given an arrow $f:X\to Y$ in $\cC$, there exists an object $F_f$ called the \emph{fiber} of $f$, and a distinguished triangle
    \[F_f\to X\xrightarrow fY\to\Sigma F_f(\cong C_f).\]
\end{proposition}
\begin{proof}
    Since $\Sigma$ is an equivalence, there exists some functor $\Omega:\cC\to\cC$ and natural isomorphisms $\vare:\Omega\Sigma\Rightarrow\Id_\cC$ and $\eta:\Id_\cC\Rightarrow\Sigma\Omega$. By axiom TR2, we have a distinguished triangle
    \[X\xrightarrow fY\xrightarrow gC_f\xrightarrow h\Sigma X.\]
    Now, consider the commutative diagram
    % https://q.uiver.app/#q=WzAsOCxbMCwwLCJYIl0sWzEsMCwiWSJdLFsyLDAsIkNfZiJdLFszLDAsIlxcU2lnbWEgWCJdLFswLDEsIlgiXSxbMSwxLCJZIl0sWzIsMSwiXFxTaWdtYVxcT21lZ2EgQ19mIl0sWzMsMSwiXFxTaWdtYSBYIl0sWzEsMiwiZyJdLFsyLDMsImgiXSxbMCw0LCIiLDIseyJsZXZlbCI6Miwic3R5bGUiOnsiaGVhZCI6eyJuYW1lIjoibm9uZSJ9fX1dLFs0LDUsImYiXSxbNSw2LCJcXHd0IGciXSxbNiw3LCJcXHd0IGgiXSxbMiw2LCJcXGV0YV97Q19mfSJdLFszLDcsIiIsMCx7ImxldmVsIjoyLCJzdHlsZSI6eyJoZWFkIjp7Im5hbWUiOiJub25lIn19fV0sWzEsNSwiIiwxLHsibGV2ZWwiOjIsInN0eWxlIjp7ImhlYWQiOnsibmFtZSI6Im5vbmUifX19XSxbMCwxLCJmIl1d
    \[\begin{tikzcd}
        X & Y & {C_f} & {\Sigma X} \\
        X & Y & {\Sigma\Omega C_f} & {\Sigma X}
        \arrow["g", from=1-2, to=1-3]
        \arrow["h", from=1-3, to=1-4]
        \arrow[Rightarrow, no head, from=1-1, to=2-1]
        \arrow["f", from=2-1, to=2-2]
        \arrow["{\wt g}", from=2-2, to=2-3]
        \arrow["{\wt h}", from=2-3, to=2-4]
        \arrow["{\eta_{C_f}}", from=1-3, to=2-3]
        \arrow[Rightarrow, no head, from=1-4, to=2-4]
        \arrow[Rightarrow, no head, from=1-2, to=2-2]
        \arrow["f", from=1-1, to=1-2]
    \end{tikzcd}\]
    where $\wt g=\eta_{C_f}\circ g$, and $\wt h=h\circ\eta_{C_f}^{-1}$. Since each vertical map is an isomorphism and the top row is distinguished, the bottom row is also distinguished by axiom TR0. Now, since $\Sigma$ is an equivalence of categories, it is faithful, so that in particular there exists some map $k:\Omega C_f\to X$ such that $\Sigma k=-\wt h\implies-\Sigma k=\wt h$. Thus, we have a distinguished triangle of the form
    \[X\xrightarrow fY\xrightarrow{\wt g}\Sigma\Omega C_f\xrightarrow{-\Sigma k}\Sigma X.\]
    Finally, by axiom TR4, we get a distinguished triangle
    \[\Omega C_f\xrightarrow kX\xrightarrow fY\xrightarrow{\wt g}\Sigma\Omega C_f,\]
    so we may define the fiber of $f$ to be $\Omega C_f$.
\end{proof}

\subsection{Homotopy (co)limits in a triangulated category}

In this subsection, we will assume $\cC$ has countable products and coproducts.

\begin{definition}[{\cite[Definition 1.6.4]{Neeman_2001}}]\label{homotopy_colimit_defn}
    Let
    \[X_0\xr{j_1}X_1\xr{j_2} X_2\xr{j_3}X_3\to\cdots\]
    be a sequence of objects and morphisms in $\cC$. The \emph{homotopy colimit} of the sequence, denoted $\hocolim X_i$, is given (up to non-canonical isomorphism) as the cofiber of the map
    \[\bigoplus_{i=0}^\infty X_i\xr{1-\text{shift}}\bigoplus_{i=0}^\infty X_i,\]
    where the shift map $\bigoplus_{i=0}^\infty X_i\xr{\text{shift}}\bigoplus_{i=0}^\infty X_i$ is understood to be the direct sum of $j_{i+1}:X_i\to X_{i+1}$, i.e., by the universal property of the coproduct, it is induced by the maps
    \[X_s\xr{j_{s+1}}X_{s+1}\into\bigoplus_{i=0}^\infty X_i.\]
\end{definition}

Homotopy colimits in general are not colimits, but they do share some properties with the colimit. First of all, note that we have canonical $X_i\to\hocolim X_i$ given as the composition
\[X_i\into\bigoplus_{i=0}^\infty X_i\to\hocolim X_i\]
It is straightforward to see 

\begin{proposition}
    Let
    \[X_0\xr{j_1}X_1\xr{j_2}X_2\xr{j_3}X_3\to\cdots\]
    be a sequence of objects and morphisms in $\cC$, and suppose we have a cone under this diagram, i.e., an object $Y$ in $\cC$ along with maps $\eta_s:X_s\to Y$ such that $\eta_{s}\circ j_s=\eta_{s-1}$. Then this cone factors through the homotopy colimit, i.e., there exists a map $\ell:\hocolim X_i\to Y$ such that for all $s\geq0$, the following diagram commutes:
    % https://q.uiver.app/#q=WzAsNSxbMCwwLCJYX3MiXSxbMiwwLCJYX3tzKzF9Il0sWzEsMSwiXFxiaWdvcGx1c197aT0wfV5cXGluZnR5IFhfaSJdLFsxLDIsIlxcaG9jb2xpbSBYX2kiXSxbMSwzLCJZIl0sWzAsMSwial97cysxfSJdLFswLDIsIiIsMix7InN0eWxlIjp7InRhaWwiOnsibmFtZSI6Imhvb2siLCJzaWRlIjoidG9wIn19fV0sWzEsMiwiIiwwLHsic3R5bGUiOnsidGFpbCI6eyJuYW1lIjoiaG9vayIsInNpZGUiOiJib3R0b20ifX19XSxbMiwzXSxbMyw0LCJcXGVsbCJdLFswLDQsIlxcZXRhX3MiLDIseyJjdXJ2ZSI6M31dLFsxLDQsIlxcZXRhX3tzKzF9IiwwLHsiY3VydmUiOi0zfV1d
    \[\begin{tikzcd}
        {X_s} && {X_{s+1}} \\
        & {\bigoplus_{i=0}^\infty X_i} \\
        & {\hocolim X_i} \\
        & Y
        \arrow["{j_{s+1}}", from=1-1, to=1-3]
        \arrow[hook, from=1-1, to=2-2]
        \arrow[hook', from=1-3, to=2-2]
        \arrow[from=2-2, to=3-2]
        \arrow["\ell", from=3-2, to=4-2]
        \arrow["{\eta_s}"', curve={height=18pt}, from=1-1, to=4-2]
        \arrow["{\eta_{s+1}}", curve={height=-18pt}, from=1-3, to=4-2]
    \end{tikzcd}\]
\end{proposition}
\begin{proof}[Proof sketch]
    By direction inspection, the leftmost square in the following diagram commutes:
    % https://q.uiver.app/#q=WzAsOCxbMCwwLCJcXGJpZ29wbHVzX3tpPTB9XlxcaW5mdHkgWF9pIl0sWzEsMCwiXFxiaWdvcGx1c197aT0wfV5cXGluZnR5IFhfaSJdLFsyLDAsIlxcaG9jb2xpbSBYX2kiXSxbMSwxLCJZIl0sWzMsMCwiXFxTaWdtYVxcbGVmdChcXGJpZ29wbHVzX3tpPTB9XlxcaW5mdHkgWF9pXFxyaWdodCkiXSxbMCwxLCIwIl0sWzIsMSwiWSJdLFszLDEsIjAiXSxbMCwxLCIxLVxcdGV4dHtzaGlmdH0iXSxbMSwyXSxbMSwzLCJcXGV0YSJdLFsyLDRdLFswLDVdLFs1LDNdLFszLDYsIiIsMCx7ImxldmVsIjoyLCJzdHlsZSI6eyJoZWFkIjp7Im5hbWUiOiJub25lIn19fV0sWzYsN10sWzQsN10sWzIsNiwiXFxlbGwiLDAseyJzdHlsZSI6eyJib2R5Ijp7Im5hbWUiOiJkYXNoZWQifX19XV0=
    \[\begin{tikzcd}
        {\bigoplus_{i=0}^\infty X_i} & {\bigoplus_{i=0}^\infty X_i} & {\hocolim X_i} & {\Sigma\left(\bigoplus_{i=0}^\infty X_i\right)} \\
        0 & Y & Y & 0
        \arrow["{1-\text{shift}}", from=1-1, to=1-2]
        \arrow[from=1-2, to=1-3]
        \arrow["\eta", from=1-2, to=2-2]
        \arrow[from=1-3, to=1-4]
        \arrow[from=1-1, to=2-1]
        \arrow[from=2-1, to=2-2]
        \arrow[Rightarrow, no head, from=2-2, to=2-3]
        \arrow[from=2-3, to=2-4]
        \arrow[from=1-4, to=2-4]
        \arrow["\ell", dashed, from=1-3, to=2-3]
    \end{tikzcd}\]
    The top row is distinguished by definition. The bottom row is distinguished, since it may be obtained by shifting the triangle $(\id_Y,0,0)$. Thus we get an induced dashed arrow $\ell:\hocolim X_i\to Y$ which makes the diagram commute, by axiom TR3. By direct inspection it clearly makes the above diagram commute.
\end{proof}

\begin{definition}\label{homotopy_limit_defn}
    Assume that $\cC$ has countable products, and let
    \[\cdots\to X_3\xr{j_3}X_2\xr{j_2}X_1\xr{j_1}X_0\]
    be a sequence of objects and morphisms in $\cC$. The \emph{homotopy limit} of the sequence, denoted $\holim X_i$, is given (up to non-canonical isomorphism) as the fiber (\autoref{fiber}) of the map
    \[\prod_{i=0}^\infty X_i\xr{1-\text{shift}}\prod_{i=0}^\infty,\]
    where the shift map $\prod_{i=0}^\infty X_i\xr{\text{shift}}\prod_{i=0}^\infty X_i$ is understood to be the product of $j_i:X_i\to X_{i-1}$, i.e., by the universal property of the product, it is induced by the maps
    \[\prod_{i=0}^\infty X_i\onto X_{s+1}\xr{j_{s+1}}X_s.\]
\end{definition}

An analagous argument to the one given above yields that $\holim X_i$ satisfies the same properties as the limit of the $X_i$'s, minus uniqueness.

\begin{proposition}\label{idempotent_splits_in_tri_cat_with_countable_coproducts}
    Suppose $\cC$ has countable coproducts, and suppose $e:X\to X$ is an idempotent in $\cC$, so that $e\circ e=e$. Then $e$ splits in $\cC$, i.e., $e$ factors as
    \[X\xr rY\xr \iota X\]
    with $r\circ\iota=\id_Y$ and $\iota\circ r=e$. In particular, we may take $Y$ to be the colimit of
    \[X\xr eX\xr eX\xr eX\xr e\cdots.\]
\end{proposition}
\begin{proof}
    See~\cite[Proposition 1.6.8]{Neeman_2001}.
\end{proof}

\subsection{Adjointly triangulated categories}

For our purposes, we will always be dealing with triangulated categories with a bit of extra structure, in the following sense:

\begin{definition}\label{adjointly_triangulated_defn}
    An \emph{adjointly triangulated category} $(\cC,\Omega,\Sigma,\eta,\vare,\cD)$ is the data of a triangulated category $(\cC,\Sigma,\cD)$ along with an inverse shift functor $\Omega:\cC\to\cC$ and natural isomorphisms $\eta:\Id_\cC\Rightarrow\Sigma\Omega$ and $\vare:\Omega\Sigma\Rightarrow\Id_\cC$ such that $(\Omega,\Sigma,\eta,\vare)$ forms an adjoint equivalence of $\cC$. In other words, $\eta$ and $\vare$ are natural isomorphisms which also are the unit and counit of an adjunction $\Omega\dashv\Sigma$, so they satisfy either of the following ``zig-zag identities'':
    % https://q.uiver.app/#q=WzAsNixbMCwwLCJcXE9tZWdhIl0sWzEsMCwiXFxPbWVnYVxcU2lnbWFcXE9tZWdhIl0sWzEsMSwiXFxPbWVnYSJdLFsyLDAsIlxcU2lnbWFcXE9tZWdhXFxTaWdtYSJdLFszLDAsIlxcU2lnbWEiXSxbMiwxLCJcXFNpZ21hIl0sWzAsMSwiXFxPbWVnYVxcZXRhIl0sWzEsMiwiXFx2YXJlcHNpbG9uXFxPbWVnYSJdLFs0LDMsIlxcZXRhXFxTaWdtYSIsMl0sWzMsNSwiXFxTaWdtYVxcdmFyZXBzaWxvbiIsMl0sWzQsNSwiIiwyLHsibGV2ZWwiOjIsInN0eWxlIjp7ImhlYWQiOnsibmFtZSI6Im5vbmUifX19XSxbMCwyLCIiLDIseyJsZXZlbCI6Miwic3R5bGUiOnsiaGVhZCI6eyJuYW1lIjoibm9uZSJ9fX1dXQ==
    \[\begin{tikzcd}
        \Omega & \Omega\Sigma\Omega & \Sigma\Omega\Sigma & \Sigma \\
        & \Omega & \Sigma
        \arrow["\Omega\eta", from=1-1, to=1-2]
        \arrow["\varepsilon\Omega", from=1-2, to=2-2]
        \arrow["\eta\Sigma"', from=1-4, to=1-3]
        \arrow["\Sigma\varepsilon"', from=1-3, to=2-3]
        \arrow[Rightarrow, no head, from=1-4, to=2-3]
        \arrow[Rightarrow, no head, from=1-1, to=2-2]
    \end{tikzcd}\]
    (Satisfying one implies the other is automatically satisfied, see \cite[Lemma 3.2]{nlab:adjoint_equivalence}).
\end{definition}

From now on, we will assume that $\cC$ is an \emph{adjointly} triangulated category with inverse shift $\Omega$, unit $\eta:\Id_\cC\Rightarrow\Sigma\Omega$, and counit $\vare:\Omega\Sigma\Rightarrow\Id_\cC$.

\begin{lemma}\label{shift_dt_left}
    Given a triangle
    \[X\xrightarrow fY\xrightarrow gZ\xrightarrow h\Sigma X,\]
    it can be shifted to the left to obtain a distinguished triangle
    \[\Omega Z\xrightarrow{-\wt h}X\xrightarrow fY\xrightarrow{\wt{\Omega g}}\Sigma\Omega Z,\]
    where $\wt h:\Omega Z\to X$ is the adjoint of $h:Z\to\Sigma X$ and $\wt{\Omega g}:Y\to\Sigma\Omega Z$ is the adjoint of $\Omega g:\Omega Y\to\Omega Z$.
\end{lemma}
\begin{proof}
    Note that unravelling definitions, $\wt h$ and $\wt g$ are the compositions 
    \[\wt h:\Omega Z\xrightarrow{\Omega h}\Omega\Sigma X\xrightarrow{\vare_X}X\qquad\text{and}\qquad \wt{\Omega g}:Y\xrightarrow{\eta_Y}\Sigma\Omega Y\xrightarrow{\Sigma\Omega g}\Sigma\Omega Z.\] 
    Now consider the following diagram:
    % https://q.uiver.app/#q=WzAsOCxbMCwwLCJYIl0sWzEsMCwiWSJdLFsyLDAsIloiXSxbMywwLCJcXFNpZ21hIFgiXSxbMCwxLCJYIl0sWzEsMSwiWSJdLFsyLDEsIlxcU2lnbWFcXE9tZWdhIFoiXSxbMywxLCJcXFNpZ21hIFgiXSxbMCwxLCJmIl0sWzEsMiwiZyJdLFsyLDMsImgiXSxbNCw1LCJmIl0sWzUsNiwiXFx3dHtcXE9tZWdhIGd9Il0sWzYsNywiXFxTaWdtYVxcd3QgaCJdLFs0LDAsIiIsMSx7ImxldmVsIjoyLCJzdHlsZSI6eyJoZWFkIjp7Im5hbWUiOiJub25lIn19fV0sWzUsMSwiIiwxLHsibGV2ZWwiOjIsInN0eWxlIjp7ImhlYWQiOnsibmFtZSI6Im5vbmUifX19XSxbMiw2LCJcXGV0YV9aIiwyXSxbNywzLCIiLDEseyJsZXZlbCI6Miwic3R5bGUiOnsiaGVhZCI6eyJuYW1lIjoibm9uZSJ9fX1dXQ==
    \begin{equation}\label{shift_left_diagram}\begin{tikzcd}
        X & Y & Z & {\Sigma X} \\
        X & Y & {\Sigma\Omega Z} & {\Sigma X}
        \arrow["f", from=1-1, to=1-2]
        \arrow["g", from=1-2, to=1-3]
        \arrow["h", from=1-3, to=1-4]
        \arrow["f", from=2-1, to=2-2]
        \arrow["{\wt{\Omega g}}", from=2-2, to=2-3]
        \arrow["{\Sigma\wt h}", from=2-3, to=2-4]
        \arrow[Rightarrow, no head, from=2-1, to=1-1]
        \arrow[Rightarrow, no head, from=2-2, to=1-2]
        \arrow["{\eta_Z}"', from=1-3, to=2-3]
        \arrow[Rightarrow, no head, from=2-4, to=1-4]
    \end{tikzcd}\end{equation}
    The left square commutes by definition. To see that the middle square commutes, expanding definitions, note it is given by the following diagram:
    % https://q.uiver.app/#q=WzAsNSxbMCwxLCJZIl0sWzAsMCwiWSJdLFsyLDAsIloiXSxbMSwxLCJcXFNpZ21hXFxPbWVnYSBZIl0sWzIsMSwiXFxTaWdtYVxcT21lZ2EgWiJdLFswLDEsIiIsMCx7ImxldmVsIjoyLCJzdHlsZSI6eyJoZWFkIjp7Im5hbWUiOiJub25lIn19fV0sWzEsMiwiZyJdLFswLDMsIlxcZXRhX1kiXSxbMyw0LCJcXFNpZ21hXFxPbWVnYSBnIl0sWzIsNCwiXFxldGFfWSJdXQ==
    \[\begin{tikzcd}
        Y && Z \\
        Y & {\Sigma\Omega Y} & {\Sigma\Omega Z}
        \arrow[Rightarrow, no head, from=2-1, to=1-1]
        \arrow["g", from=1-1, to=1-3]
        \arrow["{\eta_Y}", from=2-1, to=2-2]
        \arrow["{\Sigma\Omega g}", from=2-2, to=2-3]
        \arrow["{\eta_Y}", from=1-3, to=2-3]
    \end{tikzcd}\]
    and this commutes by naturality of $\eta$. To see that the right square commutes, consider the following diagram:
    % https://q.uiver.app/#q=WzAsNSxbMCwxLCJcXFNpZ21hXFxPbWVnYSBaIl0sWzEsMSwiXFxTaWdtYVxcT21lZ2FcXFNpZ21hIFgiXSxbMiwwLCJcXFNpZ21hIFgiXSxbMCwwLCJaIl0sWzIsMSwiXFxTaWdtYSBYIl0sWzAsMSwiXFxTaWdtYVxcT21lZ2EgaCIsMl0sWzMsMCwiXFxldGFfWiIsMl0sWzMsMiwiaCJdLFsxLDQsIlxcU2lnbWFcXHZhcmVfWCIsMl0sWzIsNCwiIiwyLHsibGV2ZWwiOjIsInN0eWxlIjp7ImhlYWQiOnsibmFtZSI6Im5vbmUifX19XSxbMiwxLCJcXGV0YV97XFxTaWdtYSBYfSIsMV1d
    \[\begin{tikzcd}
        Z && {\Sigma X} \\
        {\Sigma\Omega Z} & {\Sigma\Omega\Sigma X} & {\Sigma X}
        \arrow["{\Sigma\Omega h}"', from=2-1, to=2-2]
        \arrow["{\eta_Z}"', from=1-1, to=2-1]
        \arrow["h", from=1-1, to=1-3]
        \arrow["{\Sigma\vare_X}"', from=2-2, to=2-3]
        \arrow[Rightarrow, no head, from=1-3, to=2-3]
        \arrow["{\eta_{\Sigma X}}"{description}, from=1-3, to=2-2]
    \end{tikzcd}\]
    By functoriality of $\Sigma$, the bottom composition is $\Sigma\wt h$. The left region commutes by naturality of $\eta$. Commutativity of the right region is precisely one of the the zig-zag identities. Hence, since diagram (\ref{shift_left_diagram}) commutes, the vertical arrows are isomorphisms, and the top row is distinguished, we have that the bottom row is distinguished as well by axiom TR0. Then by axiom TR4, since $(f,\wt{\Omega g},\Sigma\wt h)$ is distinguished, so is the triangle
    \[\Omega Z\xrightarrow{-\wt h}X\xrightarrow fY\xrightarrow{\wt{\Omega g}}\Sigma\Omega Z.\qedhere\]
\end{proof}

\begin{lemma}\label{big_shift_left_dt}
    Given a distinguished triangle
    \[X\xrightarrow fY\xrightarrow gZ\xrightarrow h\Sigma X,\]
    for any $n>0$, the triangle
    \[\Omega^n X\xrightarrow{(-1)^n\Omega^n f} \Omega^n Y\xrightarrow{(-1)^n\Omega^n g}\Omega^n Z\xrightarrow{(-1)^n\Omega^n h}\Omega^n\Sigma X\cong\Sigma\Omega^n X,\]
    is distinguished, where the final isomorphism is given by the composition
    \[\Omega^n\Sigma X=\Omega^{n-1}\Omega\Sigma X\xrightarrow{\Omega^{n-1}\vare_X}\Omega^{n-1}X\xrightarrow{\eta_{\Omega^{n-1}X}}\Sigma\Omega\Omega^{n-1}X=\Sigma\Omega^nX.\]
\end{lemma}
\begin{proof}
    We give a proof by induction. First we show the case $n=1$. Note by \autoref{shift_dt_left}, we have that given a distinguished triangle
    \[X\xrightarrow fY\xrightarrow gZ\xrightarrow h\Sigma X,\]
    we can shift it to the left to obtain a distinguished triangle
    \[\Omega Z\xrightarrow{-\wt h}X\xrightarrow fY\xrightarrow{\wt{\Omega g}}\Sigma\Omega Z,\]
    where $\wt h$ is the adjoint of $h:Z\to\Sigma X$ and $\wt{\Omega g}$ is the adjoint of $\Omega g:\Omega Y\to\Omega Z$. If we apply this shifting operation again, we get the distinguished triangle
    \[\Omega Y\xrightarrow{-\wt{\wt {\Omega g}}}\Omega Z\xrightarrow {-\wt h}X\xrightarrow{\wt{\Omega f}}\Sigma\Omega Y,\]
    where unravelling definitions, $\wt{\Omega f}$ is the right adjoint of $\Omega f:\Omega X\to\Omega Y$ and $\wt{\wt{\Omega g}}$ is the right adjoint of $\wt{\Omega g}$, which itself is the left adjoint of $\Omega g$, so $\wt{\wt{\Omega g}}=\Omega g$.
    Hence we have a distinguished triangle
    \[\Omega Y\xrightarrow{-\Omega g}\Omega Z\xrightarrow{-\wt h}X\xrightarrow{\wt{\Omega f}}\Sigma\Omega Y.\]
    We may again shift this triangle again and the above arguments yield the distinguished triangle 
    \[\Omega X\xrightarrow{-\Omega f}\Omega Y\xrightarrow{-\Omega g}\Omega Z\xrightarrow{\wt{\Omega(-\wt h)}}\Sigma\Omega X,\]
    where $\wt{\Omega (-\wt h)}$ is the right adjoint of $\Omega(-\wt h)=-\Omega\wt h:\Omega\Omega Z\to\Omega X$. Explicitly unravelling definitions, $\wt{\Omega(-\wt h)}=\wt{-\Omega\wt h}$ is the composition
    \begin{align*}
        [\Omega Z\xrightarrow{\eta_{\Omega Z}}\Sigma\Omega\Omega Z\xrightarrow{\Sigma(-\Omega\wt h)}\Sigma\Omega X]&=-[\Omega Z\xrightarrow{\eta_{\Omega Z}}\Sigma\Omega\Omega Z\xrightarrow{\Sigma\Omega\wt h}\Sigma\Omega X] \\
        &=-[\Omega Z\xrightarrow{\eta_{\Omega Z}}\Sigma\Omega\Omega Z\xrightarrow{\Sigma\Omega\Omega h}\Sigma\Omega\Omega\Sigma X\xrightarrow{\Sigma\Omega\vare_X}\Sigma\Omega X] \\
        &=-[\Omega Z\xrightarrow{\Omega h}\Omega\Sigma X\xrightarrow{\vare_X}X\xrightarrow{\eta_X}\Sigma\Omega X],
    \end{align*}
    where the first equality follows by additivity of $\Sigma$ and additivity of composition, the second follows by further unravelling how $\wt h$ is defined, and the third follows by naturality of $\eta$, which tells us the following diagram commutes:
    % https://q.uiver.app/#q=WzAsNixbMCwwLCJcXE9tZWdhIFoiXSxbMCwxLCJcXFNpZ21hXFxPbWVnYVxcT21lZ2EgWiAiXSxbMSwxLCJcXFNpZ21hXFxPbWVnYVxcT21lZ2FcXFNpZ21hIFgiXSxbMiwxLCJcXFNpZ21hXFxPbWVnYSBYIl0sWzEsMCwiXFxPbWVnYVxcU2lnbWEgWCJdLFsyLDAsIlgiXSxbMCwxLCJcXGV0YV97XFxPbWVnYSBafSJdLFsxLDIsIlxcU2lnbWFcXE9tZWdhXFxPbWVnYSBoIiwyXSxbMiwzLCJcXFNpZ21hXFxPbWVnYVxcdmFyZV9YIiwyXSxbMCw0LCJcXE9tZWdhIGgiXSxbNCwyLCJcXGV0YV97XFxPbWVnYVxcU2lnbWEgWH0iXSxbNCw1LCJcXHZhcmVfWCJdLFs1LDMsIlxcZXRhX1giXV0=
    \[\begin{tikzcd}
        {\Omega Z} & {\Omega\Sigma X} & X \\
        {\Sigma\Omega\Omega Z } & {\Sigma\Omega\Omega\Sigma X} & {\Sigma\Omega X}
        \arrow["{\eta_{\Omega Z}}", from=1-1, to=2-1]
        \arrow["{\Sigma\Omega\Omega h}"', from=2-1, to=2-2]
        \arrow["{\Sigma\Omega\vare_X}"', from=2-2, to=2-3]
        \arrow["{\Omega h}", from=1-1, to=1-2]
        \arrow["{\eta_{\Omega\Sigma X}}", from=1-2, to=2-2]
        \arrow["{\vare_X}", from=1-2, to=1-3]
        \arrow["{\eta_X}", from=1-3, to=2-3]
    \end{tikzcd}\]
    Thus indeed we have a distinguished triangle
    \[\Omega X\xrightarrow{-\Omega f}\Omega Y\xrightarrow{-\Omega g}\Omega Z\xrightarrow{-\Omega h}\Omega\Sigma X\cong\Sigma\Omega X,\]
    where the last isomorphism is $\eta_X\circ\vare_X$, as desired.
    
    Now, we show the inductive step. Suppose we know that given a distinguished triangle
    \[X\xrightarrow fY\xrightarrow gZ\xrightarrow h\Sigma X,\]
    that for some $n>0$ the triangle
    \[\Omega^n X\xrightarrow{(-1)^n\Omega^n f} \Omega^n Y\xrightarrow{(-1)^n\Omega^n g}\Omega^n Z\xrightarrow{(-1)^n h^n}\Sigma\Omega^n X,\]
    is distinguished, where $h^n:\Omega^nZ\to\Sigma\Omega^nX$ is the composition
    \[\Omega^nZ\xrightarrow{\Omega^nh}\Omega^n\Sigma X\xrightarrow{\Omega^{n-1}\vare_X}\Omega^{n-1}X\xrightarrow{\eta_{\Omega^{n-1}X}}\Sigma\Omega^nX.\]
    Then by applying the $n=1$ case to this triangle, we get that the following triangle is distinguished
    \[\Omega^{n+1}X\xrightarrow{-\Omega((-1)^n\Omega^nf)}\Omega^{n+1}Y\xrightarrow{-\Omega((-1)^n\Omega^ng)}\Omega^{n+1}Z\xrightarrow{-\Omega((-1)^nh^n)}\Omega\Sigma \Omega^nX\cong\Sigma\Omega^{n+1}X,\]
    where the final isomorphism is the composition
    \[\Omega\Sigma\Omega^nX\xrightarrow{\vare_{\Omega^nX}}\Omega^nX\xrightarrow{\eta_{\Omega^nX}}\Sigma\Omega\Omega^nX=\Sigma\Omega^{n+1}X.\]
    We claim that this is precisely the distinguished triangle given in the statement of the lemma for $n+1$.
    First of all, note that $-\Omega((-1)^n\Omega^nf)=(-1)^{n+1}\Omega^{n+1}f$, $-\Omega((-1)^n\Omega^ng)=(-1)^{n+1}\Omega^{n+1}g$, and $-\Omega((-1)^nh^n)=(-1)^{n+1}\Omega h^n$ by additivity of $\Omega$, so that the triangle becomes
    \begin{equation}\label{2nd_to_last_omega_tri}\Omega^{n+1}X\xrightarrow{(-1)^{n+1}\Omega^{n+1}f}\Omega^{n+1}Y\xrightarrow{(-1)^{n+1}\Omega^{n+1}g}\Omega^{n+1}Z\xrightarrow{(-1)^{n+1}\Omega h^n}\Omega\Sigma\Omega^nX\cong\Sigma\Omega^{n+1}X.\end{equation}
    Thus, in order to prove the desired characterization, it remains to show this diagram commutes:
    % https://q.uiver.app/#q=WzAsNixbMCwwLCJcXE9tZWdhXntuKzF9WiJdLFsyLDAsIlxcT21lZ2FcXFNpZ21hXFxPbWVnYV5uWCJdLFs0LDEsIlxcU2lnbWFcXE9tZWdhXntuKzF9WCJdLFswLDEsIlxcT21lZ2Fee24rMX1cXFNpZ21hIFgiXSxbMiwxLCJcXE9tZWdhXm5YIl0sWzQsMCwiXFxPbWVnYV5uWCJdLFswLDEsIigtMSlee24rMX1cXE9tZWdhIGhebiJdLFswLDMsIigtMSlee24rMX1cXE9tZWdhXntuKzF9aCIsMl0sWzMsNCwiXFxPbWVnYV5uXFx2YXJlX1giLDJdLFs0LDIsIlxcZXRhX3tcXE9tZWdhXm5YfSIsMl0sWzEsNSwiXFx2YXJlX3tcXE9tZWdhXm5YfSJdLFs1LDIsIlxcZXRhX3tcXE9tZWdhXm5YfSJdXQ==
    \[\begin{tikzcd}
        {\Omega^{n+1}Z} && {\Omega\Sigma\Omega^nX} && {\Omega^nX} \\
        {\Omega^{n+1}\Sigma X} && {\Omega^nX} && {\Sigma\Omega^{n+1}X}
        \arrow["{(-1)^{n+1}\Omega h^n}", from=1-1, to=1-3]
        \arrow["{(-1)^{n+1}\Omega^{n+1}h}"', from=1-1, to=2-1]
        \arrow["{\Omega^n\vare_X}"', from=2-1, to=2-3]
        \arrow["{\eta_{\Omega^nX}}"', from=2-3, to=2-5]
        \arrow["{\vare_{\Omega^nX}}", from=1-3, to=1-5]
        \arrow["{\eta_{\Omega^nX}}", from=1-5, to=2-5]
    \end{tikzcd}\]
    (The top composition is the last two arrows in diagram (\ref{2nd_to_last_omega_tri}), and the bottom composition is the last two arrows in the diagram in the statement of the lemma). Unravelling how $h^n$ is constructed, by additivity of $\Omega$ it further suffices to show the outside of the following diagram commutes:
    % https://q.uiver.app/#q=WzAsOCxbMCwwLCJcXE9tZWdhXntuKzF9WiJdLFs2LDAsIlxcT21lZ2FcXFNpZ21hXFxPbWVnYV5uWCJdLFs2LDIsIlxcU2lnbWFcXE9tZWdhXntuKzF9WCJdLFswLDIsIlxcT21lZ2Fee24rMX1cXFNpZ21hIFgiXSxbNCwyLCJcXE9tZWdhXm5YIl0sWzYsMSwiXFxPbWVnYV5uWCJdLFsyLDAsIlxcT21lZ2Fee24rMX1cXFNpZ21hIFgiXSxbNCwwLCJcXE9tZWdhXm5YIl0sWzAsMywiKC0xKV57bisxfVxcT21lZ2Fee24rMX1oIiwyXSxbMyw0LCJcXE9tZWdhXm5cXHZhcmVfWCIsMl0sWzQsMiwiXFxldGFfe1xcT21lZ2Feblh9IiwyXSxbMSw1LCJcXHZhcmVfe1xcT21lZ2Feblh9Il0sWzUsMiwiXFxldGFfe1xcT21lZ2Feblh9Il0sWzAsNiwiKC0xKV57bisxfVxcT21lZ2Fee24rMX1oIl0sWzYsNywiXFxPbWVnYV5uXFx2YXJlX3tYfSJdLFs3LDEsIlxcT21lZ2FcXGV0YV97XFxPbWVnYV57bi0xfVh9Il0sWzcsNCwiIiwxLHsibGV2ZWwiOjIsInN0eWxlIjp7ImhlYWQiOnsibmFtZSI6Im5vbmUifX19XSxbNCw1LCIiLDEseyJsZXZlbCI6Miwic3R5bGUiOnsiaGVhZCI6eyJuYW1lIjoibm9uZSJ9fX1dXQ==
    \[\begin{tikzcd}
        {\Omega^{n+1}Z} && {\Omega^{n+1}\Sigma X} && {\Omega^nX} && {\Omega\Sigma\Omega^nX} \\
        &&&&&& {\Omega^nX} \\
        {\Omega^{n+1}\Sigma X} &&&& {\Omega^nX} && {\Sigma\Omega^{n+1}X}
        \arrow["{(-1)^{n+1}\Omega^{n+1}h}"', from=1-1, to=3-1]
        \arrow["{\Omega^n\vare_X}"', from=3-1, to=3-5]
        \arrow["{\eta_{\Omega^nX}}"', from=3-5, to=3-7]
        \arrow["{\vare_{\Omega^nX}}", from=1-7, to=2-7]
        \arrow["{\eta_{\Omega^nX}}", from=2-7, to=3-7]
        \arrow["{(-1)^{n+1}\Omega^{n+1}h}", from=1-1, to=1-3]
        \arrow["{\Omega^n\vare_{X}}", from=1-3, to=1-5]
        \arrow["{\Omega\eta_{\Omega^{n-1}X}}", from=1-5, to=1-7]
        \arrow[Rightarrow, no head, from=1-5, to=3-5]
        \arrow[Rightarrow, no head, from=3-5, to=2-7]
    \end{tikzcd}\]
    The left rectangle and bottom right triangle commute by definition. Finally, commutativity of the top right trapezoid is precisely one of the zig-zag identities applied to $\Omega^{n-1}X$. Hence, we have shown the desired result.
\end{proof}

%\begin{lemma}\label{shifted_triangle_exact}
    %Suppose we have a distinguished triangle
    %\[X\xrightarrow fY\xrightarrow gZ\xrightarrow h\Sigma X.\]
    %Then for any integer $n\geq0$, the sequence
    %\[\Sigma^nX\xrightarrow{\Sigma^nf}\Sigma^nY\xrightarrow{\Sigma^ng}\Sigma^nZ\xrightarrow{\Sigma^nh}\Sigma^{n+1}X\]
    %is exact. Similarly, for any integer $n>0$, the sequence
    %\[\Omega^nX\xrightarrow{\Omega^nf}\Omega^nY\xrightarrow{\Omega^ng}\Omega^nZ\xrightarrow{\Omega^nh}\Omega^n\Sigma X\cong\Omega^{n-1}X\]
    %are exact (\autoref{defn_exact}), where the isomorphism is given by $\Omega^{n-1}\eta_{X}^{-1}$, where $\eta:\mathrm{Id}_\cC\to\Omega\Sigma$ is the unit of the adjunction $\Sigma\dashv\Omega$. 
%\end{lemma}
%\begin{proof}
    %By \autoref{distinguished_tri_is_exact}, the first statement holds when $n=0$. Now suppose we are given some $n>0$. Using axiom TR4, by induction we have that the triangle
    %\[\Sigma^nX\xrightarrow{(-1)^n\Sigma^nf}\Sigma^nY\xrightarrow{(-1)^n\Sigma^ng}\Sigma^nZ\xrightarrow{(-1)^n\Sigma^nh}\Sigma^{n+1}X\]
    %is also distinguished.
    %Thus, again by \autoref{distinguished_tri_is_exact}, given any object $A$ in $\cC$, the sequence of abelian groups
    %\[[A,\Sigma^nX]\xrightarrow{(-1)^n\Sigma^nf_*}[A,\Sigma^nY]\xrightarrow{(-1)^n\Sigma^ng_*}[A,\Sigma^nZ]\xrightarrow{(-1)^n\Sigma^nh_*}[A,\Sigma^{n+1}X]\]
    %is exact. A simple diagram chase yields that we can remove the signs and the sequence is still exact, so we have shown the desired statement for $\Sigma$. Now, we would like to show for $n>0$ that the sequence
    %\[\Omega^nX\xrightarrow{\Omega^nf}\Omega^nY\xrightarrow{\Omega^ng}\Omega^nZ\xrightarrow{\Omega^nh}\Omega^{n}\Sigma X\cong\Omega^{n-1}X\]
    %is exact, where the final isomorphism is $\Omega^{n-1}\eta_{X}^{-1}$. Now, consider the following diagram:
    %% https://q.uiver.app/#q=WzAsOSxbMCwwLCJcXE9tZWdhXm5YIl0sWzIsMCwiXFxPbWVnYV5uWSJdLFs0LDAsIlxcT21lZ2FebloiXSxbNiwwLCJcXFNpZ21hXFxPbWVnYV5uWCJdLFs2LDIsIlxcT21lZ2Fee24tMX1YIl0sWzQsMiwiXFxPbWVnYV5uWiJdLFs1LDEsIlxcT21lZ2FeblxcU2lnbWEgWCJdLFswLDIsIlxcT21lZ2FeblgiXSxbMiwyLCJcXE9tZWdhXm5ZIl0sWzAsMSwiKC0xKV5uXFxPbWVnYV5uZiJdLFsxLDIsIigtMSleblxcT21lZ2FebmciXSxbMiwzLCIiLDAseyJzdHlsZSI6eyJib2R5Ijp7Im5hbWUiOiJkYXNoZWQifX19XSxbNSw2LCIoLTEpXm5cXE9tZWdhXm5oIiwxXSxbNiw0LCJcXE9tZWdhXntuLTF9XFxldGFfe1h9XnstMX0iLDFdLFs3LDgsIigtMSleblxcT21lZ2FebmYiXSxbOCw1LCIoLTEpXm5cXE9tZWdhXm5nIl0sWzUsNCwiIiwwLHsic3R5bGUiOnsiYm9keSI6eyJuYW1lIjoiZGFzaGVkIn19fV0sWzIsNSwiIiwwLHsibGV2ZWwiOjIsInN0eWxlIjp7ImhlYWQiOnsibmFtZSI6Im5vbmUifX19XSxbMSw4LCIiLDEseyJsZXZlbCI6Miwic3R5bGUiOnsiaGVhZCI6eyJuYW1lIjoibm9uZSJ9fX1dLFswLDcsIiIsMSx7ImxldmVsIjoyLCJzdHlsZSI6eyJoZWFkIjp7Im5hbWUiOiJub25lIn19fV0sWzIsNiwiKC0xKV5uXFxPbWVnYV5uaCIsMV0sWzQsMywiXFx2YXJlXnstMX1fe1xcT21lZ2Fee24tMX1YfSIsMl1d
    %\[\begin{tikzcd}
        %{\Omega^nX} && {\Omega^nY} && {\Omega^nZ} && {\Sigma\Omega^nX} \\
        %&&&&& {\Omega^n\Sigma X} \\
        %{\Omega^nX} && {\Omega^nY} && {\Omega^nZ} && {\Omega^{n-1}X}
        %\arrow["{(-1)^n\Omega^nf}", from=1-1, to=1-3]
        %\arrow["{(-1)^n\Omega^ng}", from=1-3, to=1-5]
        %\arrow[dashed, from=1-5, to=1-7]
        %\arrow["{(-1)^n\Omega^nh}"{description}, from=3-5, to=2-6]
        %\arrow["{\Omega^{n-1}\eta_{X}^{-1}}"{description}, from=2-6, to=3-7]
        %\arrow["{(-1)^n\Omega^nf}", from=3-1, to=3-3]
        %\arrow["{(-1)^n\Omega^ng}", from=3-3, to=3-5]
        %\arrow[dashed, from=3-5, to=3-7]
        %\arrow[Rightarrow, no head, from=1-5, to=3-5]
        %\arrow[Rightarrow, no head, from=1-3, to=3-3]
        %\arrow[Rightarrow, no head, from=1-1, to=3-1]
        %\arrow["{(-1)^n\Omega^nh}"{description}, from=1-5, to=2-6]
        %\arrow["{\vare^{-1}_{\Omega^{n-1}X}}"', from=3-7, to=1-7]
    %\end{tikzcd}\]
    %where the dashed arrows are the necessary compositions which makes this diagram commute. By \autoref{big_shift_left_dt}, the top row is distinguished, thus exact (by \autoref{distinguished_tri_is_exact}). Thus, since the diagram commutes and the vertical arrows are isomorphisms, it follows that the bottom row is exact. Again, a simple diagram chase yields that we can forget the signs and the sequence is still exact, so we get precisely the desired result.
%\end{proof}

\begin{proposition}\label{dist_tri_LES}
    Given a distinguished triangle
    \[X\xrightarrow fY\xrightarrow gZ\xrightarrow h\Sigma X,\]
    let $\wt h:\Omega Z\to X$ be the left adjoint of $h$. Then the following infinite sequence is exact:
    % https://q.uiver.app/#q=WzAsMTksWzQsMCwiXFxjZG90cyJdLFswLDEsIlxcT21lZ2Fee24rMX1aIl0sWzIsMSwiXFxPbWVnYV5uWCJdLFs0LDEsIlxcT21lZ2FeblkiXSxbNiwxLCJcXE9tZWdhXm5aIl0sWzgsMSwiXFxPbWVnYV57bi0xfVgiXSxbMCwzLCJcXE9tZWdhIFoiXSxbMiwzLCJYIl0sWzQsMywiWSJdLFs2LDMsIlogIl0sWzgsMywiXFxTaWdtYSBYIl0sWzQsMiwiXFxjZG90cyJdLFs0LDQsIlxcY2RvdHMiXSxbMCw1LCJcXFNpZ21hXntuLTF9WiJdLFsyLDUsIlxcU2lnbWFeblgiXSxbNCw1LCJcXFNpZ21hXm5ZIl0sWzYsNSwiXFxTaWdtYV5uWiJdLFs4LDUsIlxcU2lnbWFee24rMX1YIl0sWzQsNiwiXFxjZG90cyJdLFswLDFdLFsxLDIsIigtMSlee24rMX1cXE9tZWdhXm5cXHd0IGgiLDJdLFsyLDMsIigtMSleblxcT21lZ2FebmYiLDJdLFs0LDUsIigtMSleblxcT21lZ2Fee24tMX1cXHd0IGgiXSxbMyw0LCIoLTEpXm5cXE9tZWdhXm5nIl0sWzYsNywiLVxcd3QgaCIsMl0sWzcsOCwiZiIsMl0sWzgsOSwiZyJdLFs5LDEwLCJoIl0sWzUsMTFdLFsxMSw2XSxbMTAsMTJdLFsxMiwxM10sWzEzLDE0LCIoLTEpXntuLTF9XFxTaWdtYV57bn0gaCIsMl0sWzE0LDE1LCIoLTEpXm5cXFNpZ21hXm5mIiwyXSxbMTUsMTYsIigtMSleblxcU2lnbWFebiBnIl0sWzE2LDE3LCIoLTEpXm5cXFNpZ21hIF5uaCJdLFsxNywxOF1d
    \[\begin{tikzcd}[column sep=scriptsize]
        &&&& \cdots \\
        {\Omega^{n+1}Z} && {\Omega^nX} && {\Omega^nY} && {\Omega^nZ} && {\Omega^{n-1}X} \\
        &&&& \cdots \\
        {\Omega Z} && X && Y && {Z } && {\Sigma X} \\
        &&&& \cdots \\
        {\Sigma^{n-1}Z} && {\Sigma^nX} && {\Sigma^nY} && {\Sigma^nZ} && {\Sigma^{n+1}X} \\
        &&&& \cdots
        \arrow[from=1-5, to=2-1]
        \arrow["{(-1)^{n+1}\Omega^n\wt h}"', from=2-1, to=2-3]
        \arrow["{(-1)^n\Omega^nf}"', from=2-3, to=2-5]
        \arrow["{(-1)^n\Omega^{n-1}\wt h}", from=2-7, to=2-9]
        \arrow["{(-1)^n\Omega^ng}", from=2-5, to=2-7]
        \arrow["{-\wt h}"', from=4-1, to=4-3]
        \arrow["f"', from=4-3, to=4-5]
        \arrow["g", from=4-5, to=4-7]
        \arrow["h", from=4-7, to=4-9]
        \arrow[from=2-9, to=3-5]
        \arrow[from=3-5, to=4-1]
        \arrow[from=4-9, to=5-5]
        \arrow[from=5-5, to=6-1]
        \arrow["{(-1)^{n-1}\Sigma^{n} h}"', from=6-1, to=6-3]
        \arrow["{(-1)^n\Sigma^nf}"', from=6-3, to=6-5]
        \arrow["{(-1)^n\Sigma^n g}", from=6-5, to=6-7]
        \arrow["{(-1)^n\Sigma ^nh}", from=6-7, to=6-9]
        \arrow[from=6-9, to=7-5]
    \end{tikzcd}\]
    In particular, it remains exact even if we remove the signs.
\end{proposition}
\begin{proof}
    Exactness of
    \[X\xrightarrow fY\xrightarrow gZ\xrightarrow h\Sigma X\xrightarrow{-\Sigma f}\Sigma Y\]
    is \autoref{distinguished_tri_is_exact} and axiom TR4. By induction using axiom TR4, for $n>0$ we get that each contiguous composition of three maps below is a distinguished triangle:
    \[\Sigma^n X\xrightarrow{(-1)^n\Sigma^nf}\Sigma^n Y\xrightarrow{(-1)^n\Sigma^n g}\Sigma^n Z\xrightarrow{(-1)^n\Sigma^nh}\Sigma^{n+1} X\xrightarrow{(-1)^{n+1}\Sigma^{n+1}f}\Sigma^{n+1}Y,\]
    thus the sequence is exact by \autoref{distinguished_tri_is_exact}.
    It remains to show exactness of the LES to the left of $Y$. It suffices to show that the row in the following diagram is exact for all $n>0$:
    % https://q.uiver.app/#q=WzAsNixbMCwwLCJcXE9tZWdhXm5YIl0sWzIsMCwiXFxPbWVnYV5uWSJdLFs0LDAsIlxcT21lZ2FebloiXSxbNiwwLCJcXE9tZWdhXntuLTF9WCJdLFs4LDAsIlxcT21lZ2Fee24tMX1ZIl0sWzUsMSwiXFxPbWVnYV5uXFxTaWdtYSBYIl0sWzAsMSwiKC0xKV5uXFxPbWVnYV5uZiJdLFsxLDIsIigtMSleblxcT21lZ2FebmciXSxbMyw0LCIoLTEpXntuLTF9XFxPbWVnYV57bi0xfWYiXSxbMiwzLCIoLTEpXm5cXE9tZWdhXntuLTF9KFxcdmFyZXBzaWxvbl9YXFxjaXJjIFxcT21lZ2EgaCkiXSxbMiw1LCIoLTEpXm5cXE9tZWdhXm5oIiwyXSxbNSwzLCJcXE9tZWdhXntuLTF9XFx2YXJlcHNpbG9uX3tYfSIsMl1d
    \begin{equation}\label{diag_asdjsaj}\begin{tikzcd}
        {\Omega^nX} && {\Omega^nY} && {\Omega^nZ} && {\Omega^{n-1}X} && {\Omega^{n-1}Y} \\
        &&&&& {\Omega^n\Sigma X}
        \arrow["{(-1)^n\Omega^nf}", from=1-1, to=1-3]
        \arrow["{(-1)^n\Omega^ng}", from=1-3, to=1-5]
        \arrow["{(-1)^{n-1}\Omega^{n-1}f}", from=1-7, to=1-9]
        \arrow["{(-1)^n\Omega^{n-1}(\varepsilon_X\circ \Omega h)}", from=1-5, to=1-7]
        \arrow["{(-1)^n\Omega^nh}"', from=1-5, to=2-6]
        \arrow["{\Omega^{n-1}\varepsilon_{X}}"', from=2-6, to=1-7]
    \end{tikzcd}\end{equation}
    First of all, to see exactness at $\Omega^nY$ and $\Omega^nZ$, consider the following commutative diagram:
    % https://q.uiver.app/#q=WzAsOSxbMCwwLCJcXE9tZWdhXm5YIl0sWzIsMCwiXFxPbWVnYV5uWSJdLFs0LDAsIlxcT21lZ2FebloiXSxbNiwwLCJcXE9tZWdhXntuLTF9WCJdLFs1LDEsIlxcT21lZ2FeblxcU2lnbWEgWCJdLFswLDIsIlxcT21lZ2FeblgiXSxbMiwyLCJcXE9tZWdhXm5ZIl0sWzQsMiwiXFxPbWVnYV5uWiJdLFs2LDIsIlxcU2lnbWFcXE9tZWdhXm5YIl0sWzAsMSwiKC0xKV5uXFxPbWVnYV5uZiJdLFsxLDIsIigtMSleblxcT21lZ2FebmciXSxbMiwzLCIoLTEpXm5cXE9tZWdhXntuLTF9KFxcdmFyZXBzaWxvbl9YXFxjaXJjIFxcT21lZ2EgaCkiXSxbMiw0LCIoLTEpXm5cXE9tZWdhXm5oIiwxXSxbNCwzLCJcXE9tZWdhXntuLTF9XFx2YXJlcHNpbG9uX3tYfSIsMV0sWzAsNSwiIiwyLHsibGV2ZWwiOjIsInN0eWxlIjp7ImhlYWQiOnsibmFtZSI6Im5vbmUifX19XSxbNSw2LCIoLTEpXm5cXE9tZWdhXm5mIl0sWzYsNywiKC0xKV5uXFxPbWVnYV5uZyJdLFs3LDgsIiIsMCx7InN0eWxlIjp7ImJvZHkiOnsibmFtZSI6ImRhc2hlZCJ9fX1dLFsxLDYsIiIsMSx7ImxldmVsIjoyLCJzdHlsZSI6eyJoZWFkIjp7Im5hbWUiOiJub25lIn19fV0sWzIsNywiIiwxLHsibGV2ZWwiOjIsInN0eWxlIjp7ImhlYWQiOnsibmFtZSI6Im5vbmUifX19XSxbMyw4LCJcXGV0YV97XFxPbWVnYV57bi0xfVh9Il0sWzcsNCwiKC0xKV5uXFxPbWVnYV5uaCIsMV1d
    \[\begin{tikzcd}
        {\Omega^nX} && {\Omega^nY} && {\Omega^nZ} && {\Omega^{n-1}X} \\
        &&&&& {\Omega^n\Sigma X} \\
        {\Omega^nX} && {\Omega^nY} && {\Omega^nZ} && {\Sigma\Omega^nX}
        \arrow["{(-1)^n\Omega^nf}", from=1-1, to=1-3]
        \arrow["{(-1)^n\Omega^ng}", from=1-3, to=1-5]
        \arrow["{(-1)^n\Omega^{n-1}(\varepsilon_X\circ \Omega h)}", from=1-5, to=1-7]
        \arrow["{(-1)^n\Omega^nh}"{description}, from=1-5, to=2-6]
        \arrow["{\Omega^{n-1}\varepsilon_{X}}"{description}, from=2-6, to=1-7]
        \arrow[Rightarrow, no head, from=1-1, to=3-1]
        \arrow["{(-1)^n\Omega^nf}", from=3-1, to=3-3]
        \arrow["{(-1)^n\Omega^ng}", from=3-3, to=3-5]
        \arrow[dashed, from=3-5, to=3-7]
        \arrow[Rightarrow, no head, from=1-3, to=3-3]
        \arrow[Rightarrow, no head, from=1-5, to=3-5]
        \arrow["{\eta_{\Omega^{n-1}X}}", from=1-7, to=3-7]
        \arrow["{(-1)^n\Omega^nh}"{description}, from=3-5, to=2-6]
    \end{tikzcd}\]
    (here the dashed arrow is the morphism which makes the diagram commute). The bottom row is distinguished by \autoref{big_shift_left_dt}. Then by axiom TR0, the top row is distinguished, and thus exact by \autoref{distinguished_tri_is_exact}. Thus we have shown exactness of (\ref{diag_asdjsaj}) at $\Omega^nY$ and $\Omega^nZ$. It remains to show exactness at $\Omega^{n-1}X$. In the case $n=1$, we want to show exactness at $X$ in the following diagram:
    % https://q.uiver.app/#q=WzAsNCxbMCwwLCJcXE9tZWdhIFoiXSxbMiwwLCJYIl0sWzQsMCwiWSJdLFsxLDEsIlxcT21lZ2FcXFNpZ21hIFgiXSxbMCwxLCItKFxcdmFyZV9YXFxjaXJjXFxPbWVnYSBoKSJdLFsxLDIsImYiXSxbMCwzLCItXFxPbWVnYSBoIiwyXSxbMywxLCJcXHZhcmVfWCIsMl1d
    \[\begin{tikzcd}
        {\Omega Z} && X && Y \\
        & {\Omega\Sigma X}
        \arrow["{-(\vare_X\circ\Omega h)}", from=1-1, to=1-3]
        \arrow["f", from=1-3, to=1-5]
        \arrow["{-\Omega h}"', from=1-1, to=2-2]
        \arrow["{\vare_X}"', from=2-2, to=1-3]
    \end{tikzcd}\]
    Unravelling definitions, $\vare_X\circ\Omega h$ is precisely the adjoint $\wt h:\Omega Z\to X$ of $h:Z\to\Sigma X$, in which case we have that the row in the above diagram fits into a distinguished triangle by \autoref{shift_dt_left}, and thus it is exact by \autoref{distinguished_tri_is_exact}. To see exactness at $\Omega^{n-1}X$ in diagram (\ref{diag_asdjsaj}), note that if we apply \autoref{shift_dt_left} to the sequence \autoref{big_shift_left_dt} for $n-1$, then we get that the following composition fits into a distinguished triangle, and is thus exact:
    \[\Omega^nZ\xrightarrow{-k}\Omega^{n-1}X\xrightarrow{(-1)^{n-1}\Omega^{n-1}f}\Omega^{n-1}Y,\]
    where $k:\Omega(\Omega^{n-1}Z)\to\Omega^{n-1}X$ is the adjoint of the composition
    \[\Omega^{n-1}Z\xrightarrow{(-1)^{n-1}\Omega^{n-1}h}\Omega^{n-1}\Sigma X\xrightarrow{\Omega^{n-2}\vare_X}\Omega^{n-2}X\xrightarrow{\eta_{\Omega^{n-2}X}}\Sigma\Omega^{n-1}X.\]
    Further expanding how adjoints are constructed, $k$ is the composition
    \[\Omega^nZ\xrightarrow{(-1)^{n-1}\Omega^nh}\Omega^{n}\Sigma X\xrightarrow{\Omega^{n-1}\vare_X}\Omega^{n-1}X\xrightarrow{\Omega\eta_{\Omega^{n-2}X}}\Omega\Sigma\Omega^{n-1}X\xrightarrow{\vare_{\Omega^{n-1}X}}\Omega^{n-1}X.\]
    Thus, in order to show exactness of (\ref{diag_asdjsaj}) at $\Sigma^{n-1}X$, it suffices to show that $k=(-1)^{n-1}\Omega^{n-1}(\vare_X\circ\Omega h)$. To that end, consider the following diagram:
    % https://q.uiver.app/#q=WzAsNixbMCwwLCJcXE9tZWdhXm5aIl0sWzEsMCwiXFxPbWVnYV5uXFxTaWdtYSBYIl0sWzIsMCwiXFxPbWVnYV57bi0xfVgiXSxbMywwLCJcXE9tZWdhXFxTaWdtYVxcT21lZ2Fee24tMX1YIl0sWzMsMiwiXFxPbWVnYV57bi0xfVgiXSxbMCwyLCJcXE9tZWdhXm5cXFNpZ21hIFgiXSxbMCwxLCIoLTEpXntuLTF9XFxPbWVnYV5uaCJdLFsxLDIsIlxcT21lZ2Fee24tMX1cXHZhcmVfWCJdLFsyLDMsIlxcT21lZ2FcXGV0YV97XFxPbWVnYV57bi0yfVh9Il0sWzMsNCwiXFx2YXJlX3tcXE9tZWdhXntuLTF9WH0iXSxbMCw1LCIoLTEpXntuLTF9XFxPbWVnYV5uaCIsMl0sWzUsNCwiXFxPbWVnYV57bi0xfVxcdmFyZV9YIiwyXSxbMiw0LCIiLDEseyJsZXZlbCI6Miwic3R5bGUiOnsiaGVhZCI6eyJuYW1lIjoibm9uZSJ9fX1dXQ==
    \[\begin{tikzcd}
        {\Omega^nZ} & {\Omega^n\Sigma X} & {\Omega^{n-1}X} & {\Omega\Sigma\Omega^{n-1}X} \\
        \\
        {\Omega^n\Sigma X} &&& {\Omega^{n-1}X}
        \arrow["{(-1)^{n-1}\Omega^nh}", from=1-1, to=1-2]
        \arrow["{\Omega^{n-1}\vare_X}", from=1-2, to=1-3]
        \arrow["{\Omega\eta_{\Omega^{n-2}X}}", from=1-3, to=1-4]
        \arrow["{\vare_{\Omega^{n-1}X}}", from=1-4, to=3-4]
        \arrow["{(-1)^{n-1}\Omega^nh}"', from=1-1, to=3-1]
        \arrow["{\Omega^{n-1}\vare_X}"', from=3-1, to=3-4]
        \arrow[Rightarrow, no head, from=1-3, to=3-4]
    \end{tikzcd}\]
    The top composition is $k$, while the bottom composition is $(-1)^{n-1}\Omega^{n-1}(\vare_X\circ\Omega h)$. The left region commutes by definition, while commutativity of the right region is precisely one of the zig-zag identities applied to $\Omega^{n-2}X$. Thus, we have shown that $-k=(-1)^n\Omega^{n-1}(\vare_X\circ\Omega h)$, so (\ref{diag_asdjsaj}) is exact at $\Omega^{n-1}X$, as desired.
\end{proof}

%\begin{lemma}\label{2-of-3_suffices_to_show_cofiber}
	%Let $\cE$ be a class of objects in a triangulated category $(\cC,\Sigma,\Omega)$ such that given any distinguished triangle
    %\[X\to Y\to Z\to\Sigma X,\]
    %if $X$ and $Y$ are in $\cE$ then so is $Z$. Then given any distinguished triangle like above, if \emph{any} two of three of $X$, $Y$, and $Z$ belong to $\cE$, than so does the third.
%\end{lemma}
%\begin{proof}
	%First, we claim that given some object $X$ in $\cC$, then $\Sigma X$ and $\Omega X$ belong to $\cE$. Recall by axiom TR1 that we have a distinguished triangle
    %\[X\xrightarrow{\id_X}X\to 0\to\Sigma X,\]
    %so that per our assumption $0$ belongs to $\cE$. Now, we can shift this triangle (TR4) to get a distinguished triangle of the form
    %\[X\to0\to\Sigma X\to\Sigma X.\]
    %Hence, $\Sigma X$ also belongs to $\cE$, as desired. Conversely, supppose $\Sigma X$ belongs to $\cE$. Then we may shift the above triangle once again to obtain a distinguished triangle of the form
    %\[0\to\Sigma X\to\Sigma X\to\Sigma 0.\]
    %Now, consider the distinguished triangle
    %\[\Omega X\xrightarrow{\id_{\Omega X}}\Omega X\to0\to\Sigma\Omega X.\]
    %We may shift it twice, and we have the following isomorphism of triangles:
    %% https://q.uiver.app/#q=WzAsOCxbMCwwLCIwIl0sWzEsMCwiXFxTaWdtYVxcT21lZ2EgWCJdLFsyLDAsIlxcU2lnbWFcXE9tZWdhIFgiXSxbMywwLCJcXFNpZ21hIDAiXSxbMCwxLCIwIl0sWzEsMSwiWCJdLFszLDEsIlxcU2lnbWEgMCJdLFsyLDEsIlxcU2lnbWFcXE9tZWdhIFgiXSxbMCwxXSxbMSwyXSxbMiwzXSxbMCw0LCIiLDIseyJsZXZlbCI6Miwic3R5bGUiOnsiaGVhZCI6eyJuYW1lIjoibm9uZSJ9fX1dLFsxLDUsIlxcY29uZyIsMl0sWzMsNiwiIiwwLHsibGV2ZWwiOjIsInN0eWxlIjp7ImhlYWQiOnsibmFtZSI6Im5vbmUifX19XSxbNCw1XSxbNSw3LCIiLDIseyJzdHlsZSI6eyJib2R5Ijp7Im5hbWUiOiJkYXNoZWQifX19XSxbNyw2XSxbMiw3LCIiLDEseyJsZXZlbCI6Miwic3R5bGUiOnsiaGVhZCI6eyJuYW1lIjoibm9uZSJ9fX1dXQ==
    %\[\begin{tikzcd}
        %0 & {\Sigma\Omega X} & {\Sigma\Omega X} & {\Sigma 0} \\
        %0 & X & {\Sigma\Omega X} & {\Sigma 0}
        %\arrow[from=1-1, to=1-2]
        %\arrow[from=1-2, to=1-3]
        %\arrow[from=1-3, to=1-4]
        %\arrow[Rightarrow, no head, from=1-1, to=2-1]
        %\arrow["\cong"', from=1-2, to=2-2]
        %\arrow[Rightarrow, no head, from=1-4, to=2-4]
        %\arrow[from=2-1, to=2-2]
        %\arrow[dashed, from=2-2, to=2-3]
        %\arrow[from=2-3, to=2-4]
        %\arrow[Rightarrow, no head, from=1-3, to=2-3]
    %\end{tikzcd}\]
    %where the dashed line is the unique arrow which makes the diagram commute. Thus, we get that
%    
    %Now, suppose we are given a distinguished triangle
    %\[X\to Y\to Z\to\Sigma X.\]
    %We would like to show that if any two of $X$, $Y$, and $Z$ belong to $\cE$, then so does the third. By definition this holds if $X$, $Y$ belong to $\cE$. First suppose $Y$, $Z$ belong to $\cE$. Then consider the triangle 
%\end{proof}

\subsection{Tensor triangulated categories}
Also important for our work is the concept of a \emph{tensor triangulated category}, that is, a triangulated symmetric monoidal category in which the triangulated structures are compatible, in the following sense:

\begin{definition}\label{tentri}
    A \textit{tensor triangulated category} is a triangulated symmetric monoidal category $(\cC,\otimes,S,\Sigma,\cD)$ such that:\begin{enumerate}[label=\textbf{TT\arabic*}]
        \item For all objects $X$ and $Y$ in $\cC$, there are natural isomorphisms
        \[e_{X,Y}:\Sigma X\otimes Y\xrightarrow\cong\Sigma(X\otimes Y).\]
        \item For each object $X$ in $\cC$, the functor $X\otimes(-)\cong(-)\otimes X$ is an additive functor.
        \item For each object $X$ in $\cC$, the functor $X\otimes(-)\cong(-)\otimes X$ preserves distinguished triangles, in that given a distinguished triangle/(co)fiber sequence
        \[A\xrightarrow fB\xrightarrow gC\xrightarrow h\Sigma A,\]
        then also
        \[X\otimes A\xrightarrow{X\otimes f}X\otimes B\xrightarrow{X\otimes g}X\otimes C\xrightarrow{X\otimes' h}\Sigma(X\otimes A)\]
        and
        \[A\otimes X\xrightarrow{f\otimes X}B\otimes X\xrightarrow{g\otimes X}C\otimes X\xrightarrow{h\otimes' X}\Sigma(A\otimes X)\]
        are distinguished triangles, where here we writing $X\otimes' h$ and $h\otimes' X$ to denote the compositions
        \[X\otimes C\xrightarrow{X\otimes h}X\otimes \Sigma A\xrightarrow\tau\Sigma A\otimes X\xrightarrow{e_{A,X}}\Sigma(A\otimes X)\xrightarrow{\Sigma\tau}\Sigma(X\otimes A)\]
        and
        \[C\otimes X\xrightarrow{h\otimes X}\Sigma A\otimes X\xrightarrow{e_{A,X}}\Sigma(A\otimes X),\]
        respectively.
        \item Given objects $X$, $Y$, and $Z$ in $\cC$, the following diagram must commute:
        % https://q.uiver.app/#q=WzAsNSxbMCwwLCIoXFxTaWdtYSBYXFxvdGltZXMgWSlcXG90aW1lcyBaIl0sWzEsMCwiXFxTaWdtYShYXFxvdGltZXMgWSlcXG90aW1lcyBaIl0sWzIsMCwiXFxTaWdtYSgoWFxcb3RpbWVzIFkpXFxvdGltZXMgWikiXSxbMiwxLCJcXFNpZ21hKFhcXG90aW1lcyhZXFxvdGltZXMgWikpIl0sWzAsMSwiXFxTaWdtYSBYXFxvdGltZXMoWVxcb3RpbWVzIFopIl0sWzAsMSwiZV97WCxZfVxcb3RpbWVzIFoiXSxbMSwyLCJlX3tYXFxvdGltZXMgWSxafSJdLFsyLDMsIlxcU2lnbWFcXGFscGhhIl0sWzAsNCwiXFxhbHBoYSIsMl0sWzQsMywiZV97WCxZXFxvdGltZXMgWn0iLDJdXQ==
        \[\begin{tikzcd}
            {(\Sigma X\otimes Y)\otimes Z} & {\Sigma(X\otimes Y)\otimes Z} & {\Sigma((X\otimes Y)\otimes Z)} \\
            {\Sigma X\otimes(Y\otimes Z)} && {\Sigma(X\otimes(Y\otimes Z))}
            \arrow["{e_{X,Y}\otimes Z}", from=1-1, to=1-2]
            \arrow["{e_{X\otimes Y,Z}}", from=1-2, to=1-3]
            \arrow["\Sigma\alpha", from=1-3, to=2-3]
            \arrow["\alpha"', from=1-1, to=2-1]
            \arrow["{e_{X,Y\otimes Z}}"', from=2-1, to=2-3]
        \end{tikzcd}\]
%        \item The following diagram must commute 
%        % https://q.uiver.app/#q=WzAsNixbMCwwLCJcXFNpZ21hIFNcXG90aW1lcyBcXFNpZ21hIFMiXSxbMCwxLCJcXFNpZ21hIFNcXG90aW1lc1xcU2lnbWEgUyJdLFsyLDEsIlxcU2lnbWFeMlMiXSxbMiwwLCJcXFNpZ21hXnsyfVMiXSxbMSwwLCJcXFNpZ21hKFNcXG90aW1lc1xcU2lnbWEgUykiXSxbMSwxLCJcXFNpZ21hKFNcXG90aW1lc1xcU2lnbWEgUykiXSxbMCwxLCJcXHRhdSIsMl0sWzMsMiwiLTEiXSxbMCw0LCJlX3tTLFN9Il0sWzQsMywiXFxTaWdtYVxcbGFtYmRhX3tcXFNpZ21hIFN9Il0sWzEsNSwiZV97UyxTfSJdLFs1LDIsIlxcU2lnbWFcXGxhbWJkYV97XFxTaWdtYSBTfSJdXQ==
%        \[\begin{tikzcd}
%            {\Sigma S\otimes \Sigma S} & {\Sigma(S\otimes\Sigma S)} & {\Sigma^{2}S} \\
%            {\Sigma S\otimes\Sigma S} & {\Sigma(S\otimes\Sigma S)} & {\Sigma^2S}
%            \arrow["\tau"', from=1-1, to=2-1]
%            \arrow["{-1}", from=1-3, to=2-3]
%            \arrow["{e_{S,S}}", from=1-1, to=1-2]
%            \arrow["{\Sigma\lambda_{\Sigma S}}", from=1-2, to=1-3]
%            \arrow["{e_{S,S}}", from=2-1, to=2-2]
%            \arrow["{\Sigma\lambda_{\Sigma S}}", from=2-2, to=2-3]
%        \end{tikzcd}\]
%        In other words, $\tau_{\Sigma S,\Sigma S}=-1$.
    \end{enumerate}
\end{definition}

Usually, most tensor triangulated categories that arise in nature will satisfy additional coherence axioms (see axioms TC1--TC5 in \cite{MayTri}), but the above definition will suffice for our purposes. In what follows, we fix a tensor triangulated category $(\cC,\otimes,S,\Sigma,e,\cD)$.
%Note that in the definition of the tensor triangulated category, we chose isomorphisms
%\[e_{X,Y}:\Sigma X\otimes Y\xrightarrow\cong\Sigma(X\otimes Y),\]
%but we just as well could have chosen isomorphisms
%\[e_{X,Y}':X\otimes\Sigma Y\xrightarrow\cong\Sigma(X\otimes Y).\]

\begin{definition}\label{e'_defn}
    There are natural isomorphisms
    \[e_{X,Y}':X\otimes\Sigma Y\xrightarrow\cong \Sigma(X\otimes Y)\]
    obtained via the composition
    \[X\otimes\Sigma Y\xrightarrow\tau\Sigma Y\otimes X\xrightarrow{e_{Y,X}}\Sigma(Y\otimes X)\xrightarrow{\Sigma\tau}\Sigma(X\otimes Y).\]
\end{definition}


%\begin{lemma}
%    For all $X$ and $Y$ in $\cC$, the following diagram commutes:
%    % https://q.uiver.app/#q=WzAsNCxbMCwwLCJcXFNpZ21hIFhcXG90aW1lc1xcU2lnbWEgWSJdLFsxLDAsIlxcU2lnbWEoWFxcb3RpbWVzIFxcU2lnbWEgWSkiXSxbMSwxLCJcXFNpZ21hXFxTaWdtYShYXFxvdGltZXMgWSkiXSxbMCwxLCJcXFNpZ21hKFxcU2lnbWEgWFxcb3RpbWVzIFkpIl0sWzAsMSwiLWVfe1gsXFxTaWdtYSBZfSJdLFsxLDIsIlxcU2lnbWEgZSdfe1gsWX0iXSxbMCwzLCJlJ197XFxTaWdtYSBYLFl9IiwyXSxbMywyLCJcXFNpZ21hIGVfe1gsWX0iLDJdXQ==
%    \[\begin{tikzcd}
%        {\Sigma X\otimes\Sigma Y} & {\Sigma(X\otimes \Sigma Y)} \\
%        {\Sigma(\Sigma X\otimes Y)} & {\Sigma\Sigma(X\otimes Y)}
%        \arrow["{-e_{X,\Sigma Y}}", from=1-1, to=1-2]
%        \arrow["{\Sigma e'_{X,Y}}", from=1-2, to=2-2]
%        \arrow["{e'_{\Sigma X,Y}}"', from=1-1, to=2-1]
%        \arrow["{\Sigma e_{X,Y}}"', from=2-1, to=2-2]
%    \end{tikzcd}\]
%    (note the sign on the top map).
%\end{lemma}
%\begin{proof}
%    Note there are natural isomorphisms
%    \[a_X:\Sigma X\xr\cong \Sigma S\otimes X\]
%    given by the composition
%    \[\Sigma X\xr{\Sigma\lambda_X}\Sigma(S\otimes X)\xr{e_{S,X}^{-1}}\Sigma S\otimes X.\]
%    Furthermore, under the isomorphism $a:\Sigma\cong \Sigma S\otimes-$, $e_{X,Y}:\Sigma X\otimes Y\cong\Sigma(X\otimes Y)$ corresponds to the associator $(\Sigma S\otimes X)\otimes Y\cong\Sigma S\otimes(X\otimes Y)$. Indeed, consider the following diagram:
%    \[\begin{tikzcd}
%        {\Sigma X\otimes Y} && {\Sigma(X\otimes Y)} \\
%        {\Sigma(S\otimes X)\otimes Y} & {\Sigma((S\otimes X)\otimes Y)} & {\Sigma(S\otimes(X\otimes Y)} \\
%        {(\Sigma S\otimes X)\otimes Y} && {\Sigma S\otimes(X\otimes Y)}
%        \arrow["{e_{X,Y}}", from=1-1, to=1-3]
%        \arrow["{\Sigma\lambda_X\otimes Y}"', from=1-1, to=2-1]
%        \arrow["{e_{S,X}^{-1}\otimes Y}"', from=2-1, to=3-1]
%        \arrow["{\Sigma(\lambda_X\otimes Y)}"{description}, from=1-3, to=2-2]
%        \arrow["\alpha"', from=3-1, to=3-3]
%        \arrow["{e_{S\otimes X,Y}}"', from=2-1, to=2-2]
%        \arrow["{\Sigma\lambda_{X\otimes Y}}", from=1-3, to=2-3]
%        \arrow["\Sigma\alpha"', from=2-2, to=2-3]
%        \arrow["{e_{S,X\otimes Y}^{-1}}", from=2-3, to=3-3]
%    \end{tikzcd}\]
%    The two vertical composites are $a_X\otimes Y$ and $a_{X\otimes Y}$, respectively. The top trapezoid commutes by naturality of $e$. The triangle commutes by coherence for a monoidal category. Finally, commutativity of the bottom rectangle is axiom TT4 for a tensor triangulated category.
%    
%    Similarly, under the isomorphism $a:\Sigma\cong\Sigma S\otimes-$, $e_{X,Y}':X\otimes\Sigma Y\cong\Sigma(X\otimes Y)$ corresponds to the map $X\otimes\Sigma S\otimes Y\xr{\tau_{X,\Sigma S}\otimes Y}X\otimes Y\otimes\Sigma S$. To see this, consider the following diagram:
%    % https://q.uiver.app/#q=WzAsMTEsWzAsMCwiWFxcb3RpbWVzXFxTaWdtYSBZIl0sWzEsMCwiXFxTaWdtYSBZXFxvdGltZXMgWCJdLFsyLDAsIlxcU2lnbWEoWVxcb3RpbWVzIFgpIl0sWzMsMCwiXFxTaWdtYShYXFxvdGltZXMgWSkiXSxbMCwzLCJYXFxvdGltZXNcXFNpZ21hIFNcXG90aW1lcyBZIl0sWzMsMywiXFxTaWdtYSBTXFxvdGltZXMgWFxcb3RpbWVzIFkiXSxbMCwxLCJYXFxvdGltZXNcXFNpZ21hKFNcXG90aW1lcyBZKSJdLFszLDEsIlxcU2lnbWEoU1xcb3RpbWVzIFhcXG90aW1lcyBZKSJdLFsxLDEsIlxcU2lnbWEoU1xcb3RpbWVzIFkpXFxvdGltZXMgWCJdLFsyLDEsIlxcU2lnbWEoU1xcb3RpbWVzIFlcXG90aW1lcyBYKSJdLFsxLDIsIlxcU2lnbWEgU1xcb3RpbWVzIFlcXG90aW1lcyBYIl0sWzAsMSwiXFx0YXVfe1gsXFxTaWdtYSBZfSJdLFsxLDIsImVfe1ksWH0iXSxbMiwzLCJcXFNpZ21hXFx0YXVfe1ksWH0iXSxbNCw1LCJcXHRhdV97WCxcXFNpZ21hIFN9XFxvdGltZXMgWSJdLFswLDYsIlhcXG90aW1lc1xcU2lnbWFcXGxhbWJkYV9ZIiwyXSxbNiw0LCJYXFxvdGltZXMgZV97UyxZfV57LTF9IiwyXSxbMyw3LCJcXFNpZ21hXFxsYW1iZGFfe1hcXG90aW1lcyBZfSJdLFs3LDUsImVfe1MsWFxcb3RpbWVzIFl9XnstMX0iXSxbNiw4LCJcXHRhdV97WCxcXFNpZ21hKFNcXG90aW1lcyBZKX0iXSxbOCw5LCJlX3tTXFxvdGltZXMgWSxYfSJdLFsyLDksIlxcU2lnbWEoXFxsYW1iZGFfWVxcb3RpbWVzIFgpIiwyXSxbMSw4LCJcXFNpZ21hXFxsYW1iZGFfWVxcb3RpbWVzIFgiLDJdLFsxMCw4LCJlX3tTLFl9XFxvdGltZXMgWCJdLFs5LDcsIlxcU2lnbWEoU1xcb3RpbWVzXFx0YXVfe1ksWH0pIl0sWzQsMTAsIlxcdGF1X3tYLFxcU2lnbWEgU1xcb3RpbWVzIFl9Il0sWzEwLDUsIlxcU2lnbWEgU1xcb3RpbWVzIFxcdGF1X3tZLFh9Il0sWzEwLDksImVfe1MsWVxcb3RpbWVzIFh9IiwyXV0=
%    \[\begin{tikzcd}
%        {X\otimes\Sigma Y} & {\Sigma Y\otimes X} & {\Sigma(Y\otimes X)} & {\Sigma(X\otimes Y)} \\
%        {X\otimes\Sigma(S\otimes Y)} & {\Sigma(S\otimes Y)\otimes X} & {\Sigma(S\otimes Y\otimes X)} & {\Sigma(S\otimes X\otimes Y)} \\
%        & {\Sigma S\otimes Y\otimes X} \\
%        {X\otimes\Sigma S\otimes Y} &&& {\Sigma S\otimes X\otimes Y}
%        \arrow["{\tau_{X,\Sigma Y}}", from=1-1, to=1-2]
%        \arrow["{e_{Y,X}}", from=1-2, to=1-3]
%        \arrow["{\Sigma\tau_{Y,X}}", from=1-3, to=1-4]
%        \arrow["{\tau_{X,\Sigma S}\otimes Y}", from=4-1, to=4-4]
%        \arrow["{X\otimes\Sigma\lambda_Y}"', from=1-1, to=2-1]
%        \arrow["{X\otimes e_{S,Y}^{-1}}"', from=2-1, to=4-1]
%        \arrow["{\Sigma\lambda_{X\otimes Y}}", from=1-4, to=2-4]
%        \arrow["{e_{S,X\otimes Y}^{-1}}", from=2-4, to=4-4]
%        \arrow["{\tau_{X,\Sigma(S\otimes Y)}}", from=2-1, to=2-2]
%        \arrow["{e_{S\otimes Y,X}}", from=2-2, to=2-3]
%        \arrow["{\Sigma(\lambda_Y\otimes X)}"', from=1-3, to=2-3]
%        \arrow["{\Sigma\lambda_Y\otimes X}"', from=1-2, to=2-2]
%        \arrow["{e_{S,Y}\otimes X}", from=3-2, to=2-2]
%        \arrow["{\Sigma(S\otimes\tau_{Y,X})}", from=2-3, to=2-4]
%        \arrow["{\tau_{X,\Sigma S\otimes Y}}", from=4-1, to=3-2]
%        \arrow["{\Sigma S\otimes \tau_{Y,X}}", from=3-2, to=4-4]
%        \arrow["{e_{S,Y\otimes X}}"', from=3-2, to=2-3]
%    \end{tikzcd}\]
%    Here we are taking the associators to be isomorphisms, by coherence for monoidal categories. The top horizontal composition is $e_{X,Y}'$, by definition. The vertical edge compositions are $X\otimes a_Y$ and $a_{X\otimes Y}$. The top left rectangle commutes by naturality of $\tau$. The top middle rectangle commutes by naturality of $e$. The top right triangle commutes by naturality of $\tau$. The bottom left trapezoid commutes by naturality of $\tau$. The small middle triangle commutes by axiom TT4 for a tensor triangulated category. The bottom triangle commutes by coherence for a symmetric monoidal category. Finally, the remaining region on the right commutes by naturality of $e$. Thus, in order to show the diagram in the statement of the lemma commutes, it suffices to show the following diagram commutes:
%    % https://q.uiver.app/#q=WzAsNCxbMCwwLCJcXFNpZ21hIFNcXG90aW1lcyBYXFxvdGltZXNcXFNpZ21hIFNcXG90aW1lcyBZIl0sWzEsMCwiXFxTaWdtYSBTXFxvdGltZXMgWFxcb3RpbWVzXFxTaWdtYSBTXFxvdGltZXMgWSJdLFsxLDEsIlxcU2lnbWEgU1xcb3RpbWVzIFxcU2lnbWEgU1xcb3RpbWVzIFhcXG90aW1lcyBZIl0sWzAsMSwiXFxTaWdtYSBTXFxvdGltZXMgXFxTaWdtYSBTXFxvdGltZXMgWFxcb3RpbWVzIFkiXSxbMCwxLCItXFxhbHBoYSJdLFsxLDIsIlxcU2lnbWEgU1xcb3RpbWVzIFxcdGF1X3tYLFxcU2lnbWEgU31cXG90aW1lcyBZIl0sWzAsMywiXFx0YXVfe1xcU2lnbWEgU1xcb3RpbWVzIFgsXFxTaWdtYSBTfVxcb3RpbWVzIFkiLDJdLFszLDIsIlxcYWxwaGEiLDJdXQ==
%    \[\begin{tikzcd}
%        {\Sigma S\otimes X\otimes\Sigma S\otimes Y} & {\Sigma S\otimes X\otimes\Sigma S\otimes Y} \\
%        {\Sigma S\otimes \Sigma S\otimes X\otimes Y} & {\Sigma S\otimes \Sigma S\otimes X\otimes Y}
%        \arrow["{-\alpha}", from=1-1, to=1-2]
%        \arrow["{\Sigma S\otimes \tau_{X,\Sigma S}\otimes Y}", from=1-2, to=2-2]
%        \arrow["{\tau_{\Sigma S\otimes X,\Sigma S}\otimes Y}"', from=1-1, to=2-1]
%        \arrow["\alpha"', from=2-1, to=2-2]
%    \end{tikzcd}\]
%    To see this diagram commutes, consider the following diagram:
%    % https://q.uiver.app/#q=WzAsNSxbMCwwLCJcXFNpZ21hIFNcXG90aW1lcyBYXFxvdGltZXNcXFNpZ21hIFNcXG90aW1lcyBZIl0sWzIsMCwiXFxTaWdtYSBTXFxvdGltZXMgWFxcb3RpbWVzXFxTaWdtYSBTXFxvdGltZXMgWSJdLFs0LDAsIlxcU2lnbWEgU1xcb3RpbWVzIFxcU2lnbWEgU1xcb3RpbWVzIFhcXG90aW1lcyBZIl0sWzAsMiwiXFxTaWdtYSBTXFxvdGltZXMgXFxTaWdtYSBTXFxvdGltZXMgWFxcb3RpbWVzIFkiXSxbNCwyLCJcXFNpZ21hIFNcXG90aW1lcyBcXFNpZ21hIFNcXG90aW1lcyBYXFxvdGltZXMgWSJdLFswLDEsIlxcYWxwaGEiXSxbMSwyLCJcXFNpZ21hIFNcXG90aW1lcyBcXHRhdV97WCxcXFNpZ21hIFN9XFxvdGltZXMgWSJdLFswLDMsIlxcdGF1X3tcXFNpZ21hIFNcXG90aW1lcyBYLFxcU2lnbWEgU31cXG90aW1lcyBZIiwyXSxbMyw0LCJcXGFscGhhIiwyXSxbMiw0LCJcXHRhdV97XFxTaWdtYSBTLFxcU2lnbWEgU31cXG90aW1lcyBYXFxvdGltZXMgWSJdXQ==
%    \[\begin{tikzcd}
%        {\Sigma S\otimes X\otimes\Sigma S\otimes Y} && {\Sigma S\otimes X\otimes\Sigma S\otimes Y} && {\Sigma S\otimes \Sigma S\otimes X\otimes Y} \\
%        \\
%        {\Sigma S\otimes \Sigma S\otimes X\otimes Y} &&&& {\Sigma S\otimes \Sigma S\otimes X\otimes Y}
%        \arrow["\alpha", from=1-1, to=1-3]
%        \arrow["{\Sigma S\otimes \tau_{X,\Sigma S}\otimes Y}", from=1-3, to=1-5]
%        \arrow["{\tau_{\Sigma S\otimes X,\Sigma S}\otimes Y}"', from=1-1, to=3-1]
%        \arrow["\alpha"', from=3-1, to=3-5]
%        \arrow["{\tau_{\Sigma S,\Sigma S}\otimes X\otimes Y}", from=1-5, to=3-5]
%    \end{tikzcd}\]
%    The diagram commutes by coherence for a symmetric monoidal category. The desired result follows by applying axiom TT5 for a tensor triangulated category, which tells us that $\tau_{\Sigma S,\Sigma S}=-1$, and additivity of $-\otimes-$ and composition.
%\end{proof}

\begin{lemma}\label{semi_tensor_exact_sequence_remains_exact}
    Let $A\xr aB\xr bC\xr cD$ be any sequence isomorphic to a distinguished triangle. Then given any $E$ in $\cC$, the sequences
    \[E\otimes A\xr{E\otimes a}E\otimes B\xr{E\otimes b}E\otimes C\xr{E\otimes c}E\otimes D\]
    and
    \[A\otimes E\xr{a\otimes E}B\otimes E\xr{b\otimes E}C\otimes E\xr{c\otimes E}D\otimes E\]
    are exact.
\end{lemma}
\begin{proof}
    Since $(a,b,c)$ is isomorphic to a distinguished triangle, there exists a commuting diagram in $\cSH$
    % https://q.uiver.app/#q=WzAsOCxbMCwwLCJYIl0sWzEsMCwiWSJdLFsyLDAsIloiXSxbMywwLCJcXFNpZ21hIFgiXSxbMCwxLCJBIl0sWzEsMSwiQiJdLFsyLDEsIkMiXSxbMywxLCJEIl0sWzAsMSwiZiJdLFsxLDIsImciXSxbMiwzLCJoIl0sWzAsNCwiXFxhbHBoYSIsMl0sWzQsNSwiYSIsMl0sWzUsNiwiYiIsMl0sWzYsNywiYyIsMl0sWzEsNSwiXFxiZXRhIiwyXSxbMiw2LCJcXGdhbW1hIiwyXSxbMyw3LCJcXGRlbHRhIiwyXV0=
    \[\begin{tikzcd}
        X & Y & Z & {\Sigma X} \\
        A & B & C & D
        \arrow["f", from=1-1, to=1-2]
        \arrow["g", from=1-2, to=1-3]
        \arrow["h", from=1-3, to=1-4]
        \arrow["\alpha"', from=1-1, to=2-1]
        \arrow["a"', from=2-1, to=2-2]
        \arrow["b"', from=2-2, to=2-3]
        \arrow["c"', from=2-3, to=2-4]
        \arrow["\beta"', from=1-2, to=2-2]
        \arrow["\gamma"', from=1-3, to=2-3]
        \arrow["\delta"', from=1-4, to=2-4]
    \end{tikzcd}\]
    where the top row is distinguished and the vertical arrows are isomorphisms. Then the following diagram commutes by functoriality of $-\otimes-$:
    % https://q.uiver.app/#q=WzAsOSxbMCwwLCJFXFxvdGltZXMgWCJdLFsyLDAsIkVcXG90aW1lcyBZIl0sWzQsMCwiRVxcb3RpbWVzIFoiXSxbNiwwLCJcXFNpZ21hKEVcXG90aW1lcyBYKSJdLFswLDIsIkVcXG90aW1lcyBBIl0sWzIsMiwiRVxcb3RpbWVzIEIiXSxbNCwyLCJFXFxvdGltZXMgQyJdLFs2LDIsIkVcXG90aW1lcyBEIl0sWzUsMSwiRVxcb3RpbWVzIFxcU2lnbWEgWCJdLFswLDEsIkVcXG90aW1lcyBmIl0sWzEsMiwiRVxcb3RpbWVzIGciXSxbMiwzLCJFXFxvdGltZXMnIGgiXSxbMCw0LCJFXFxvdGltZXMgXFxhbHBoYSIsMl0sWzQsNSwiRVxcb3RpbWVzIGEiLDJdLFs1LDYsIkVcXG90aW1lcyBiIiwyXSxbNiw3LCJFXFxvdGltZXMgYyIsMl0sWzEsNSwiRVxcb3RpbWVzIFxcYmV0YSIsMl0sWzIsNiwiRVxcb3RpbWVzIFxcZ2FtbWEiLDJdLFsyLDgsIkVcXG90aW1lcyBoIiwxXSxbOCwzLCJlJ197RSxYfSIsMV0sWzgsNywiRVxcb3RpbWVzIFxcZGVsdGEiLDFdLFszLDcsIihFXFxvdGltZXNcXGRlbHRhKVxcY2lyYyB7KGUnX3tFLFh9KX1eey0xfSIsMCx7InN0eWxlIjp7ImJvZHkiOnsibmFtZSI6ImRhc2hlZCJ9fX1dXQ==
    \[\begin{tikzcd}[sep=scriptsize]
        {E\otimes X} && {E\otimes Y} && {E\otimes Z} && {\Sigma(E\otimes X)} \\
        &&&&& {E\otimes \Sigma X} \\
        {E\otimes A} && {E\otimes B} && {E\otimes C} && {E\otimes D}
        \arrow["{E\otimes f}", from=1-1, to=1-3]
        \arrow["{E\otimes g}", from=1-3, to=1-5]
        \arrow["{E\otimes' h}", from=1-5, to=1-7]
        \arrow["{E\otimes \alpha}"', from=1-1, to=3-1]
        \arrow["{E\otimes a}"', from=3-1, to=3-3]
        \arrow["{E\otimes b}"', from=3-3, to=3-5]
        \arrow["{E\otimes c}"', from=3-5, to=3-7]
        \arrow["{E\otimes \beta}"', from=1-3, to=3-3]
        \arrow["{E\otimes \gamma}"', from=1-5, to=3-5]
        \arrow["{E\otimes h}"{description}, from=1-5, to=2-6]
        \arrow["{e'_{E,X}}"{description}, from=2-6, to=1-7]
        \arrow["{E\otimes \delta}"{description}, from=2-6, to=3-7]
        \arrow["{(E\otimes\delta)\circ {(e'_{E,X})}^{-1}}", dashed, from=1-7, to=3-7]
    \end{tikzcd}\]
    The top triangle is distinguished by axiom TT3 for a tensor triangulated category, thus exact by \autoref{distinguished_tri_is_exact}, so that the bottom triangle is also exact since the vertical arrows are isomorphisms and each square commutes. Similarly, the following diagram also commutes by functoriality of $-\otimes-$:
    % https://q.uiver.app/#q=WzAsOSxbMCwwLCJYXFxvdGltZXMgRSJdLFsyLDAsIllcXG90aW1lcyBFIl0sWzQsMCwiWlxcb3RpbWVzIEUiXSxbNiwwLCJcXFNpZ21hKFhcXG90aW1lcyBFKSJdLFswLDIsIkFcXG90aW1lcyBFIl0sWzIsMiwiQlxcb3RpbWVzIEUiXSxbNCwyLCJDXFxvdGltZXMgRSJdLFs2LDIsIkRcXG90aW1lcyBFIl0sWzUsMSwiXFxTaWdtYSBYXFxvdGltZXMgRSJdLFswLDEsImZcXG90aW1lcyBFIl0sWzEsMiwiZ1xcb3RpbWVzIEUiXSxbMiwzLCJoXFxvdGltZXMnRSJdLFswLDQsIlxcYWxwaGFcXG90aW1lcyBFIiwyXSxbNCw1LCJhXFxvdGltZXMgRSIsMl0sWzUsNiwiYlxcb3RpbWVzIEUiLDJdLFs2LDcsImNcXG90aW1lcyBFIiwyXSxbMSw1LCJcXGJldGFcXG90aW1lcyBFIiwyXSxbMiw2LCJcXGdhbW1hXFxvdGltZXMgRSIsMl0sWzMsNywiKFxcZGVsdGFcXG90aW1lcyBFKVxcY2lyYyBlX3tYLEV9XnstMX0iXSxbMiw4LCJoXFxvdGltZXMgRSIsMV0sWzgsMywiZV97WCxFfSIsMV0sWzgsNywiXFxkZWx0YVxcb3RpbWVzIEUiLDFdXQ==
    \[\begin{tikzcd}[sep=scriptsize]
        {X\otimes E} && {Y\otimes E} && {Z\otimes E} && {\Sigma(X\otimes E)} \\
        &&&&& {\Sigma X\otimes E} \\
        {A\otimes E} && {B\otimes E} && {C\otimes E} && {D\otimes E}
        \arrow["{f\otimes E}", from=1-1, to=1-3]
        \arrow["{g\otimes E}", from=1-3, to=1-5]
        \arrow["{h\otimes'E}", from=1-5, to=1-7]
        \arrow["{\alpha\otimes E}"', from=1-1, to=3-1]
        \arrow["{a\otimes E}"', from=3-1, to=3-3]
        \arrow["{b\otimes E}"', from=3-3, to=3-5]
        \arrow["{c\otimes E}"', from=3-5, to=3-7]
        \arrow["{\beta\otimes E}"', from=1-3, to=3-3]
        \arrow["{\gamma\otimes E}"', from=1-5, to=3-5]
        \arrow["{(\delta\otimes E)\circ e_{X,E}^{-1}}", from=1-7, to=3-7]
        \arrow["{h\otimes E}"{description}, from=1-5, to=2-6]
        \arrow["{e_{X,E}}"{description}, from=2-6, to=1-7]
        \arrow["{\delta\otimes E}"{description}, from=2-6, to=3-7]
    \end{tikzcd}\]
    The top row is distinguished by axiom TT3 for a tensor triangulated category, thus exact by \autoref{distinguished_tri_is_exact}, so that the bottom triangle is also exact since the vertical arrows are isomorphisms and each square commutes.
\end{proof}

\begin{proposition}\label{LES_remains_exact_after_tensor}
    Suppose we have a distinguished triangle
    \[X\to Y\to Z\to\Sigma X\]
    in $\cC$. Then given any object $E$ in $\cC$, the long exact sequence given in \autoref{dist_tri_LES} remains exact after applying $E\otimes-$ or $-\otimes E$.
\end{proposition}
\begin{proof}
    Recall that in the proof of \autoref{dist_tri_LES} we showed that the sequence was exact by showing that any two consecutive maps were isomorphic to a part of a distinguished triangle. Then the desired result follows from \autoref{semi_tensor_exact_sequence_remains_exact}.
\end{proof}

\begin{definition}
    An \emph{adjointly tensor triangulated} category is a tensor triangulated category $(\cC,\otimes,S,\Sigma,e,\cD)$ along with the structure of an adjointly triangulated category $(\cC,\Omega,\Sigma,\eta,\vare,\cD)$.
\end{definition}

From now on, we fix an adjointly tensor triangulated category $(\cC,\otimes,S,\Omega,\Sigma,\eta,\vare,e,\cD)$. 

\begin{definition}\label{o_X,Ys_in_AdjlyTenTri_cat}
    We may define natural isomorphisms $o_{X,Y}:\Omega X\otimes Y\xr\cong\Omega(X\otimes Y)$ and $o_{X,Y}':X\otimes\Omega Y\xr\cong\Omega(X\otimes Y)$ as the compositions
    \[o_{X,Y}:\Omega X\otimes Y\xr{\vare_{\Omega X\otimes Y}^{-1}}\Omega\Sigma(\Omega X\otimes Y)\xr{\Omega e_{\Omega X, Y}^{-1}}\Omega(\Sigma\Omega X\otimes Y)\xr{\Omega(\eta_X^{-1}\otimes Y)}\Omega(X\otimes Y)\]
    and
    \[o_{X,Y}':X\otimes\Omega Y\xr{\tau_{X,\Omega Y}}\Omega Y\otimes X\xr{o_{Y,X}}\Omega(Y\otimes X)\xr{\Omega\tau_{Y,X}}\Omega(X\otimes Y).\]
    These are both clearly natural by naturality of $\vare$, $e$, $\eta$, and $\tau$.
\end{definition}

\end{document}
