\documentclass[../main.tex]{subfiles}
\tikzcdset{scale cd/.style={every label/.append style={scale=#1},cells={nodes={scale=#1}}}}

\begin{document}

In the first section, we fix a symmetric monoidal category $(\cC,\otimes,S)$ withj left unitor, right unitor, associator, and symmetry isomorphism $\lambda$, $\rho$, $\alpha$, and $\tau$, respectively.

\begin{definition}\label{monoid_object}
    A \emph{monoid object} $(E,\mu,e)$ is an object $E$ in $\cC$ along with a multiplication map $\mu:E\otimes E\to E$ and a unit map $e:S\to E$ such that the following diagram commutes:
    % https://q.uiver.app/#q=WzAsOSxbMSwwLCJFXFxvdGltZXMgRSJdLFsxLDEsIkUiXSxbMiwwLCJTXFxvdGltZXMgRSJdLFswLDAsIkVcXG90aW1lcyBTIl0sWzMsMCwiKEVcXG90aW1lcyBFKVxcb3RpbWVzIEUiXSxbMywxLCJFXFxvdGltZXMoRVxcb3RpbWVzIEUpIl0sWzQsMSwiRVxcb3RpbWVzIEUiXSxbNSwxLCJFIl0sWzUsMCwiRVxcb3RpbWVzIEUiXSxbMCwxLCJcXG11Il0sWzIsMCwiZVxcb3RpbWVzIEUiLDJdLFszLDAsIkVcXG90aW1lcyBlIl0sWzIsMSwiXFxsYW1iZGEiXSxbMywxLCJcXHJobyIsMl0sWzQsNSwiXFxhbHBoYSIsMl0sWzUsNiwiRVxcb3RpbWVzXFxtdSJdLFs2LDcsIlxcbXUiXSxbNCw4LCJcXG11XFxvdGltZXMgRSJdLFs4LDcsIlxcbXUiXV0=
    \[\begin{tikzcd}
        {E\otimes S} & {E\otimes E} & {S\otimes E} & {(E\otimes E)\otimes E} && {E\otimes E} \\
        & E && {E\otimes(E\otimes E)} & {E\otimes E} & E
        \arrow["\mu", from=1-2, to=2-2]
        \arrow["{e\otimes E}"', from=1-3, to=1-2]
        \arrow["{E\otimes e}", from=1-1, to=1-2]
        \arrow["\lambda", from=1-3, to=2-2]
        \arrow["\rho"', from=1-1, to=2-2]
        \arrow["\alpha"', from=1-4, to=2-4]
        \arrow["E\otimes\mu", from=2-4, to=2-5]
        \arrow["\mu", from=2-5, to=2-6]
        \arrow["{\mu\otimes E}", from=1-4, to=1-6]
        \arrow["\mu", from=1-6, to=2-6]
    \end{tikzcd}\]
    The first diagram expresses unitality, while the second expressed associativity. If in addition the following diagram commutes, 
    % https://q.uiver.app/#q=WzAsMyxbMCwwLCJFXFxvdGltZXMgRSJdLFsyLDAsIkVcXG90aW1lcyBFIl0sWzEsMSwiRSJdLFswLDEsIlxcdGF1Il0sWzEsMiwiXFxtdSJdLFswLDIsIlxcbXUiLDJdXQ==
    \[\begin{tikzcd}
        {E\otimes E} && {E\otimes E} \\
        & E
        \arrow["\tau", from=1-1, to=1-3]
        \arrow["\mu", from=1-3, to=2-2]
        \arrow["\mu"', from=1-1, to=2-2]
    \end{tikzcd}\]
    then we say $(E,\mu,e)$ is a \emph{commutative} monoid object.
\end{definition}

%\begin{definition}
%	Let $(E,\mu,e)$ be a monoid object in $\cC$. Then a \emph{left module object} $(N,\kappa)$ over $(E,\mu,e)$ is the data of an object $N$ in $\cC$ and a morphism $\kappa:E\otimes N\to N$ such that the following two diagrams commute in $\cC$:
%	% https://q.uiver.app/#q=WzAsOCxbMCwwLCJTXFxvdGltZXMgTiJdLFsxLDAsIkVcXG90aW1lcyBOIl0sWzEsMSwiTiJdLFsyLDAsIihFXFxvdGltZXMgRSlcXG90aW1lcyBOIl0sWzQsMCwiRVxcb3RpbWVzIE4iXSxbNCwxLCJOIl0sWzIsMSwiRVxcb3RpbWVzKEVcXG90aW1lcyBOKSJdLFszLDEsIkVcXG90aW1lcyBOIl0sWzAsMSwiZVxcb3RpbWVzIE4iXSxbMSwyLCJcXGthcHBhIl0sWzAsMiwiXFxsYW1iZGFfTiIsMl0sWzMsNCwiXFxtdVxcb3RpbWVzIE4iXSxbNCw1LCJcXGthcHBhIl0sWzMsNiwiXFxhbHBoYSIsMl0sWzYsNywiRVxcb3RpbWVzIFxca2FwcGEiXSxbNyw1LCJcXGthcHBhIl1d
%	\[\begin{tikzcd}
%		{S\otimes N} & {E\otimes N} & {(E\otimes E)\otimes N} && {E\otimes N} \\
%		& N & {E\otimes(E\otimes N)} & {E\otimes N} & N
%		\arrow["{e\otimes N}", from=1-1, to=1-2]
%		\arrow["\kappa", from=1-2, to=2-2]
%		\arrow["{\lambda_N}"', from=1-1, to=2-2]
%		\arrow["{\mu\otimes N}", from=1-3, to=1-5]
%		\arrow["\kappa", from=1-5, to=2-5]
%		\arrow["\alpha"', from=1-3, to=2-3]
%		\arrow["{E\otimes \kappa}", from=2-3, to=2-4]
%		\arrow["\kappa", from=2-4, to=2-5]
%	\end{tikzcd}\]
%\end{definition}
%
%\begin{definition}
%	Let $(E,\mu,e)$ be a monoid object in $\cC$, and suppose we have two left module objects $(N,\kappa)$ and $(N',\kappa')$ over $(E,\mu,e)$. Then a morphism $f:N\to N'$ is a \emph{left $E$-module homomorphism} if the following diagram commutes in $\cC$:
%	% https://q.uiver.app/#q=WzAsNCxbMCwwLCJFXFxvdGltZXMgTiJdLFsxLDAsIkVcXG90aW1lcyBOJyJdLFswLDEsIk4iXSxbMSwxLCJOJyJdLFswLDEsIkVcXG90aW1lcyBmIl0sWzAsMiwiXFxrYXBwYSIsMl0sWzIsMywiZiJdLFsxLDMsIlxca2FwcGEnIl1d
%	\[\begin{tikzcd}
%		{E\otimes N} & {E\otimes N'} \\
%		N & {N'}
%		\arrow["{E\otimes f}", from=1-1, to=1-2]
%		\arrow["\kappa"', from=1-1, to=2-1]
%		\arrow["f", from=2-1, to=2-2]
%		\arrow["{\kappa'}", from=1-2, to=2-2]
%	\end{tikzcd}\]
%\end{definition}

%\begin{proposition}\label{product_of_monoids_is_a_monoid}
	%Let $(E_1,\mu_1,e_1)$ and $(E_2,\mu_2,e_2)$ be monoid objects in a symmetric monoidal category $(\cC,\otimes,S)$. Then $E_1\otimes E_2$ is canonically a ring spectrum via the maps
	%\[\mu:E_1\otimes E_2\otimes E_1\otimes E_2\xr{E_1\otimes\tau\otimes E_2}E_1\otimes E_1\otimes E_2\otimes E_2\xr{\mu_1\otimes\mu_2}E_1\otimes E_2\]
	%and
	%\[e:S\cong S\otimes S\xr{e_1\otimes e_2}E_1\otimes E_2.\]
%\end{proposition}
%\begin{proof}
	%\todo{todo}
%\end{proof}

From now on we fix a
%Karoubi complete
monoidal closed tensor triangulated category $(\cSH,\otimes,S,\Sigma,e,\cD)$ (\autoref{tentri}) with arbitrary (small) (co)products and sub-Picard grading $(A,\1,h,\{S^a\},\{\phi_{a,b}\})$ (\autoref{sub_Picard_grading_defn}), and we adopt the conventions outlined in \Cref{setup}. In all proofs that follow we will freely use the coherence theorem for symmetric monoidal categories. In particular, we will assume without loss of generality that the associators and unitors in $\cSH$ are identities.

\begin{proposition}\label{Sigma^a,Sigma^-a_adjoint_equiv}
	For each $a\in A$, the isomorphisms
	\[\eta^a_X:X\xr{\lambda_X^{-1}}S\otimes X\xr{\phi_{a,-a}\otimes X}(S^{a}\otimes S^{-a})\otimes X\xr\alpha S^a\otimes(S^{-a}\otimes X)=\Sigma^{a}\Omega^a X\]
	and 
	\[\vare^a_X:\Omega^a\Sigma^a X=S^{-a}\otimes(S^a\otimes X)\xrightarrow{\alpha^{-1}}(S^{-a}\otimes S^a)\otimes X\xrightarrow{\phi_{-a,a}^{-1}\otimes X}S\otimes X\xrightarrow{\lambda_X}X\]
	are natural in $X$, and furthermore, they are the unit and counit respectively of the adjoint autoequivalence $(\Omega^a,\Sigma^a,\eta^a,\vare^a)$ of $\cSH$. In particular, since $\Sigma\cong\Sigma^\1$, $\Omega:=\Omega^\1$ is a left adjoint for $\Sigma$, so that $(\cSH,\Omega,\Sigma,\eta,\vare,\cD)$ is an \emph{adjointly} triangulated category (\autoref{adjointly_triangulated_defn}), where $\eta$ and $\vare$ are the compositions 
	\[\eta:\Id_\cSH\xRightarrow{\eta^\1}\Sigma^\1\Omega\xRightarrow{\nu^{-1}\Omega}\Sigma\Omega\qquad\text{and}\qquad\vare:\Omega\Sigma\xRightarrow{\Omega\nu}\Omega\Sigma^\1\xRightarrow{\vare^\1}\Id_\cSH.\]
\end{proposition}
\begin{proof}
	In this proof, we will freely employ the coherence theorem for monoidal categories (see \cite{MLCoherence}), which essentially tells us that we may assume we are working in a strict monoidal category (i.e., that the associators and unitors and are identities). Then $\eta^a_X$ and $\vare^a_X$ become simply the maps
	\[\eta^a_X:X\xrightarrow{\phi_{a,-a}\otimes X}S^{a}\otimes S^{-a}\otimes X\qquad\text{and}\qquad\vare_X^a:S^{-a}\otimes S^{a}\otimes X\xrightarrow{\phi_{-a,a}^{-1}\otimes X}X.\]
	That these maps are natural in $X$ follows by functoriality of $-\otimes-$. Now, recall that in order to show that these natural isomorphisms form an \emph{adjoint} equivalence, it suffices to show that the natural isomorphisms $\eta^a:\Id_\cSH\Rightarrow\Omega^a\Sigma^a$ and $\vare^a:\Sigma^a\Omega^a\Rightarrow\Id_\cSH$ satisfy one of the two zig-zag identities:
	% https://q.uiver.app/#q=WzAsNixbMCwwLCJcXE9tZWdhXmEiXSxbMSwwLCJcXE9tZWdhXmFcXFNpZ21hXmFcXE9tZWdhXmEiXSxbMSwxLCJcXE9tZWdhXmEiXSxbMiwwLCJcXFNpZ21hXmFcXE9tZWdhXmFcXFNpZ21hXmEiXSxbMywwLCJcXFNpZ21hXmEiXSxbMiwxLCJcXFNpZ21hXmEiXSxbMCwxLCJcXE9tZWdhXmFcXGV0YV5hIl0sWzEsMiwiXFx2YXJlcHNpbG9uXmFcXE9tZWdhXmEiXSxbMyw1LCJcXFNpZ21hXmFcXHZhcmVwc2lsb25eYSIsMl0sWzQsNSwiIiwyLHsibGV2ZWwiOjIsInN0eWxlIjp7ImhlYWQiOnsibmFtZSI6Im5vbmUifX19XSxbMCwyLCIiLDIseyJsZXZlbCI6Miwic3R5bGUiOnsiaGVhZCI6eyJuYW1lIjoibm9uZSJ9fX1dLFs0LDMsIlxcZXRhXmFcXFNpZ21hXmEiLDJdXQ==
	\[\begin{tikzcd}
		{\Omega^a} & {\Omega^a\Sigma^a\Omega^a} & {\Sigma^a\Omega^a\Sigma^a} & {\Sigma^a} \\
		& {\Omega^a} & {\Sigma^a}
		\arrow["{\Omega^a\eta^a}", from=1-1, to=1-2]
		\arrow["{\varepsilon^a\Omega^a}", from=1-2, to=2-2]
		\arrow["{\Sigma^a\varepsilon^a}"', from=1-3, to=2-3]
		\arrow[Rightarrow, no head, from=1-4, to=2-3]
		\arrow[Rightarrow, no head, from=1-1, to=2-2]
		\arrow["{\eta^a\Sigma^a}"', from=1-4, to=1-3]
	\end{tikzcd}\]
	(that it suffices to show only one is~\cite[Lemma~3.2]{nlab:adjoint_equivalence}). We will show that the left is satisfied. Unravelling definitions, we simply wish to show that the following diagram commutes for all $X$ in $\cSH$:
	% https://q.uiver.app/#q=WzAsMyxbMCwwLCJTXnstYX1cXG90aW1lcyBYIl0sWzEsMCwiU157LWF9XFxvdGltZXMgU15hXFxvdGltZXMgU157LWF9XFxvdGltZXMgWCJdLFsxLDEsIlNeey1hfVxcb3RpbWVzIFgiXSxbMCwxLCJTXnstYX1cXG90aW1lcyBcXHBoaV97YSwtYX1cXG90aW1lcyBYIl0sWzEsMiwiXFxwaGlfey1hLGF9XnstMX1cXG90aW1lcyBTXnstYX1cXG90aW1lcyBYIl0sWzAsMiwiIiwyLHsibGV2ZWwiOjIsInN0eWxlIjp7ImhlYWQiOnsibmFtZSI6Im5vbmUifX19XV0=
	\[\begin{tikzcd}
		{S^{-a}\otimes X} & {S^{-a}\otimes S^a\otimes S^{-a}\otimes X} \\
		& {S^{-a}\otimes X}
		\arrow["{S^{-a}\otimes \phi_{a,-a}\otimes X}", from=1-1, to=1-2]
		\arrow["{\phi_{-a,a}^{-1}\otimes S^{-a}\otimes X}", from=1-2, to=2-2]
		\arrow[Rightarrow, no head, from=1-1, to=2-2]
	\end{tikzcd}\]
	Yet this is simply the diagram obtained by applying $-\otimes X$ to the associativity coherence diagram for the $\phi_{a,b}$'s (since $\phi_{a,0}$ and $\phi_{0,a}$ coincide with the unitors, and here we are taking the unitors and associators to be equalities), so it does commute, as desired.
\end{proof}

%\begin{lemma}\label{wedge_of_supsensions_of_monoid_object_is_module_object}
%	Let $(E,\mu,e)$ be a monoid object in $\cSH$, then given a family of integers $n_i\in\bZ$ indexed by some set $I$, the object $\bigoplus_i\Sigma^{n_i}E$ is canonically a left module object over $E$, where for $n>0$ we define $\Sigma^{-n}:=\Omega^n=(\Omega^\1)^n=(S^{-\1}\otimes-)^n$. Explicitly, the action map
%	\[E\otimes\bigoplus_i\Sigma^{n_i}E\to \bigoplus_i\Sigma^{n_i}E\]
%	is given by the composition
%	\[E\otimes\bigoplus_i\Sigma^{n_i}E\xr\cong\bigoplus_i(E\otimes\Sigma^{n_i}E)\xr\cong\bigoplus_i\Sigma^{n_i}(E\otimes E)\xr{\bigoplus_i\Sigma^{n_i}\mu}\bigoplus_i\Sigma^{n_i}E,\]
%	where the first arrow is the canonical isomorphism given by the fact that $-\otimes-$ commutes with arbitrary colimits in each argument (since $\cC$ is symmetric monoidal closed) and the second arrow is given componentwise by repeated applications of $e'$ if $n_i>0$ (\autoref{e'_defn}) or $o'$ if $n_i<0$ (see \autoref{o_X,Ys_in_AdjlyTenTri_cat}, which we get since $\cSH$ is adjointly triangulated by \autoref{Sigma^a,Sigma^-a_adjoint_equiv}).
%\end{lemma}
%\begin{proof}
%	\todo{TODO}
%\end{proof}

\begin{proposition}\label{pi_*E_is_ring_for_E_monoid_appendix}
	Let $(E,\mu,e)$ be a monoid object in $\cSH$, and consider the multiplication map $\pi_*(E)\times\pi_*(E)\to\pi_*(E)$ which sends classes $x:S^a\to E$ and $y:S^b\to E$ to the composition
	\[S^{a+b}\xr{\phi_{a,b}}S^a\otimes S^b\xr{x\otimes y}E\otimes E\xr\mu E.\]
	Then this endows $\pi_*(E)$ with the structure of an $A$-graded ring with unit $e\in\pi_0(E)=[S,E]$.
\end{proposition}
\begin{proof}
	Here we are using \autoref{A_graded_ring}, so it suffices to show the given assignment is associative and unital w.r.t.\ homogeneous elements. Suppose we have classes $x$, $y$, and $z$ in $\pi_a(E)$, $\pi_b(E)$, and $\pi_c(E)$, respectively. To see associativity, consider the following diagram:
	% https://q.uiver.app/#q=WzAsNixbMCwxLCJTXnthK2IrY30iXSxbMSwxLCJTXmFcXG90aW1lcyBTXmJcXG90aW1lcyBTXmMiXSxbMiwxLCIgIEVcXG90aW1lcyBFXFxvdGltZXMgRSJdLFszLDAsIkVcXG90aW1lcyBFIl0sWzMsMSwiRSJdLFszLDIsIkVcXG90aW1lcyBFIl0sWzAsMSwiXFxjb25nIl0sWzEsMiwieFxcb3RpbWVzIHlcXG90aW1lcyB6Il0sWzIsMywiXFxtdVxcb3RpbWVzIEUiXSxbMyw0LCJcXG11Il0sWzIsNSwiRVxcb3RpbWVzXFxtdSIsMl0sWzUsNCwiXFxtdSIsMl1d
	\[\begin{tikzcd}
		&&& {E\otimes E} \\
		{S^{a+b+c}} & {S^a\otimes S^b\otimes S^c} & {  E\otimes E\otimes E} & E \\
		&&& {E\otimes E}
		\arrow["\cong", from=2-1, to=2-2]
		\arrow["{x\otimes y\otimes z}", from=2-2, to=2-3]
		\arrow["{\mu\otimes E}", from=2-3, to=1-4]
		\arrow["\mu", from=1-4, to=2-4]
		\arrow["E\otimes\mu"', from=2-3, to=3-4]
		\arrow["\mu"', from=3-4, to=2-4]
	\end{tikzcd}\]
	(here the first arrow is the unique isomorphism obtained by composing products of $\phi_{a,b}$'s, see \autoref{unique_comp_Sas}). It commutes by associativity of $\mu$. It follows by functoriality of $-\otimes-$ that the top composition is $(x\cdot y)\cdot z$ while the bottom is $x\cdot(y\cdot z)$, so they are equal as desired. To see that $e\in\pi_0(E)$ is a left and right unit for this multiplication, consider the following diagram
	% https://q.uiver.app/#q=WzAsNSxbMiwwLCJTXmEiXSxbMiwxLCJFIl0sWzAsMSwiRVxcb3RpbWVzIEUiXSxbNCwxLCJFXFxvdGltZXMgRSJdLFsyLDIsIkUiXSxbMCwxLCJ4Il0sWzAsMiwiZVxcb3RpbWVzIHgiLDJdLFswLDMsInhcXG90aW1lcyBlIl0sWzEsNCwiIiwxLHsibGV2ZWwiOjIsInN0eWxlIjp7ImhlYWQiOnsibmFtZSI6Im5vbmUifX19XSxbMSwyLCJlXFxvdGltZXMgRSIsMl0sWzEsMywiRVxcb3RpbWVzIGUiXSxbMyw0LCJcXG11Il0sWzIsNCwiXFxtdSIsMl1d
	\[\begin{tikzcd}
		&& {S^a} \\
		{E\otimes E} && E && {E\otimes E} \\
		&& E
		\arrow["x", from=1-3, to=2-3]
		\arrow["{e\otimes x}"', from=1-3, to=2-1]
		\arrow["{x\otimes e}", from=1-3, to=2-5]
		\arrow[Rightarrow, no head, from=2-3, to=3-3]
		\arrow["{e\otimes E}"', from=2-3, to=2-1]
		\arrow["{E\otimes e}", from=2-3, to=2-5]
		\arrow["\mu", from=2-5, to=3-3]
		\arrow["\mu"', from=2-1, to=3-3]
	\end{tikzcd}\]
	Commutativity of the two top triangles is functoriality of $-\otimes-$. Commutativity of the bottom two triangles is unitality of $\mu$. Thus the diagram commutes, so $e\cdot x=x\cdot e$. Finally, to see this product is bilinear (distributive). Suppose we further have some $x'\in\pi_a(E)$ and $y'\in\pi_b(E)$, and consider the following diagrams:
	% https://q.uiver.app/#q=WzAsMTgsWzAsMCwiU157YStifSJdLFsxLDAsIlNeYVxcb3RpbWVzIFNeYiJdLFswLDEsIlNee2ErYn1cXG9wbHVzIFNee2ErYn0iXSxbMSwxLCIoU15hXFxvdGltZXMgU15iKVxcb3BsdXMoU15hXFxvdGltZXMgU15iKSJdLFsyLDAsIihTXmFcXG9wbHVzIFNeYSlcXG90aW1lcyBTXmIiXSxbMiwxLCIoRVxcb3RpbWVzIEUpXFxvcGx1cyhFXFxvdGltZXMgRSkiXSxbMywwLCIoRVxcb3BsdXMgRSlcXG90aW1lcyBFIl0sWzMsMSwiRVxcb3RpbWVzIEUiXSxbMCwyLCJTXnthK2J9Il0sWzEsMiwiU15hXFxvdGltZXMgU15iIl0sWzIsMiwiU15iXFxvdGltZXMoU15iXFxvcGx1cyBTXmIpIl0sWzMsMiwiRVxcb3RpbWVzKEVcXG9wbHVzIEUpIl0sWzMsMywiRVxcb3RpbWVzIEUiXSxbMiwzLCIoRVxcb3RpbWVzIEUpXFxvcGx1cyhFXFxvdGltZXMgRSkiXSxbMSwzLCIoU15hXFxvdGltZXMgU15iKVxcb3BsdXMoU15hXFxvdGltZXMgU15iKSJdLFswLDMsIlNee2ErYn1cXG9wbHVzIFNee2ErYn0iXSxbNCwxLCJFIl0sWzQsMywiRSJdLFswLDEsIlxccGhpX3thLGJ9Il0sWzAsMiwiXFxEZWx0YSIsMl0sWzIsMywiXFxwaGlfe2EsYn1cXG9wbHVzXFxwaGlfe2EsYn0iLDJdLFsxLDMsIlxcRGVsdGEiXSxbMSw0LCJcXERlbHRhXFxvdGltZXMgU15iIl0sWzQsMywiXFxjb25nIiwyXSxbMyw1LCIoeFxcb3RpbWVzIHkpXFxvcGx1cyh4J1xcb3RpbWVzIHkpIiwyXSxbNCw2LCIoeFxcb3BsdXMgeCcpXFxvdGltZXMgeSJdLFs2LDUsIlxcY29uZyIsMl0sWzUsNywiXFxuYWJsYSIsMl0sWzYsNywiXFxuYWJsYVxcb3RpbWVzIEUiXSxbOCw5LCJcXHBoaV97YSxifSJdLFs5LDEwLCJTXmFcXG90aW1lc1xcRGVsdGEiXSxbMTAsMTEsInhcXG90aW1lcyh5XFxvcGx1cyB5JykiXSxbMTEsMTIsIkVcXG90aW1lc1xcbmFibGEiXSxbMTEsMTMsIlxcY29uZyIsMl0sWzEzLDEyLCJcXG5hYmxhIiwyXSxbMTAsMTQsIlxcY29uZyIsMl0sWzE0LDEzLCIoeFxcb3RpbWVzIHkpXFxvcGx1cyh4XFxvdGltZXMgeScpIiwyXSxbOSwxNCwiXFxEZWx0YSJdLFs4LDE1LCJcXERlbHRhIiwyXSxbMTUsMTQsIlxccGhpX3thLGJ9XFxvcGx1c1xccGhpX3thLGJ9IiwyXSxbNywxNiwiXFxtdSJdLFsxMiwxNywiXFxtdSJdXQ==
	\[\begin{tikzcd}
		{S^{a+b}} & {S^a\otimes S^b} & {(S^a\oplus S^a)\otimes S^b} & {(E\oplus E)\otimes E} \\
		{S^{a+b}\oplus S^{a+b}} & {(S^a\otimes S^b)\oplus(S^a\otimes S^b)} & {(E\otimes E)\oplus(E\otimes E)} & {E\otimes E} & E \\
		{S^{a+b}} & {S^a\otimes S^b} & {S^b\otimes(S^b\oplus S^b)} & {E\otimes(E\oplus E)} \\
		{S^{a+b}\oplus S^{a+b}} & {(S^a\otimes S^b)\oplus(S^a\otimes S^b)} & {(E\otimes E)\oplus(E\otimes E)} & {E\otimes E} & E
		\arrow["{\phi_{a,b}}", from=1-1, to=1-2]
		\arrow["\Delta"', from=1-1, to=2-1]
		\arrow["{\phi_{a,b}\oplus\phi_{a,b}}"', from=2-1, to=2-2]
		\arrow["\Delta", from=1-2, to=2-2]
		\arrow["{\Delta\otimes S^b}", from=1-2, to=1-3]
		\arrow["\cong"', from=1-3, to=2-2]
		\arrow["{(x\otimes y)\oplus(x'\otimes y)}"', from=2-2, to=2-3]
		\arrow["{(x\oplus x')\otimes y}", from=1-3, to=1-4]
		\arrow["\cong"', from=1-4, to=2-3]
		\arrow["\nabla"', from=2-3, to=2-4]
		\arrow["{\nabla\otimes E}", from=1-4, to=2-4]
		\arrow["{\phi_{a,b}}", from=3-1, to=3-2]
		\arrow["{S^a\otimes\Delta}", from=3-2, to=3-3]
		\arrow["{x\otimes(y\oplus y')}", from=3-3, to=3-4]
		\arrow["E\otimes\nabla", from=3-4, to=4-4]
		\arrow["\cong"', from=3-4, to=4-3]
		\arrow["\nabla"', from=4-3, to=4-4]
		\arrow["\cong"', from=3-3, to=4-2]
		\arrow["{(x\otimes y)\oplus(x\otimes y')}"', from=4-2, to=4-3]
		\arrow["\Delta", from=3-2, to=4-2]
		\arrow["\Delta"', from=3-1, to=4-1]
		\arrow["{\phi_{a,b}\oplus\phi_{a,b}}"', from=4-1, to=4-2]
		\arrow["\mu", from=2-4, to=2-5]
		\arrow["\mu", from=4-4, to=4-5]
	\end{tikzcd}\]
	The unlabeled isomorphisms are those given by the fact that $-\otimes-$ is additive in each variable (since $\cSH$ is tensor triangulated). Commutativity of the left squares is naturality of $\Delta:X\to X\oplus X$ in an additive category. Commutativity of the rest of the diagram follows again from the fact that $-\otimes-$ is an additive functor in each variable. Hence, by functoriality of $-\otimes-$, these diagrams tell us that $(x+x')\cdot y=x\cdot y+x'\cdot y$ and $x\cdot(y+y')=x\cdot y+x\cdot y'$, respectively.
\end{proof}

%\begin{lemma}\label{E-module_N_implies_pi*N_is_pi*E_module}
%	Let $(E,\mu,e)$ be a monoid object and $(N,\kappa)$ a left $E$-module in $\cSH$. Then the map
%	\[\pi_*(E)\times\pi_*(N)\to\pi_*(N)\]
%	sending a class $x:S^a\to E$ and $y:S^b\to N$ to the composition
%	\[S^{a+b}\xr{\phi_{a,b}} S^a\otimes S^b\xr{x\otimes y}E\otimes N\xr\kappa N\]
%	endows $\pi_*(N)$ with the structure of an $A$-graded left $\pi_*(E)$-module.
%\end{lemma}
%\begin{proof}
%	\todo{TODO}
%\end{proof}
%
%\begin{lemma}
%	Let $(E,\mu,e)$ be a monoid object and $(N,\kappa)$ a left $E$-module in $\cSH$. Further suppose that $E$ and $N$ are both cellular (\autoref{cellular}) and that $\pi_*(N)$ is a projective left $\pi_*(E)$-module (\autoref{E-module_N_implies_pi*N_is_pi*E_module}). Then there exists a collection of objects $x_i\in\pi_*(N)$ indexed by some (small) set $I$ and a commuting diagram
%	% https://q.uiver.app/#q=WzAsMyxbMCwwLCJOIl0sWzEsMCwiXFxiaWdvcGx1c19pXFxTaWdtYV57fHhfaXx9RSJdLFsyLDAsIk4iXSxbMCwxLCJpIiwyXSxbMSwyLCJyIiwyXSxbMCwyLCIiLDAseyJjdXJ2ZSI6LTQsImxldmVsIjoyLCJzdHlsZSI6eyJoZWFkIjp7Im5hbWUiOiJub25lIn19fV1d
%	\[\begin{tikzcd}
%		N & {\bigoplus_i\Sigma^{|x_i|}E} & N
%		\arrow["i"', from=1-1, to=1-2]
%		\arrow["r"', from=1-2, to=1-3]
%		\arrow[curve={height=-24pt}, Rightarrow, no head, from=1-1, to=1-3]
%	\end{tikzcd},\]
%	where $i$ and $r$ are homomorphisms of left $E$-module objects (where $\bigoplus_i\Sigma^{|x_i|}E$ has the left $E$-module structure prescribed by \autoref{wedge_of_supsensions_of_monoid_object_is_module_object}).
%\end{lemma}
%\begin{proof}
%	\todo{TODO}
%\end{proof}

\begin{proposition}\label{pi_*(E)_is_A-graded_commutative_if_E_is_commutative}
	For all $a,b\in A$ there exists an element $\theta_{a,b}\in\pi_0(S)=[S,S]$ (determined by choice of coherent family $\{\phi_{a,b}\}$) such that given any commutative monoid object $(E,\mu,e)$ in $\cSH$, the $A$-graded ring structure on $\pi_\ast(E)$ (\autoref{pi_*E_is_ring_for_E_monoid}) has a commutativity formula given by
	\[x\cdot y=y\cdot x\cdot (e\circ\theta_{a,b})\]
	for all $x\in\pi_a(E)$ and $y\in\pi_b(E)$. In particular, $\theta_{a,b}\in\mathrm{Aut}(S)$ is the composition
	\[S\xr{\cong}S^{-a-b}\otimes S^a\otimes S^b\xr{S^{-a-b}\otimes\tau}S^{-a-b}\otimes S^b\otimes S^a\xr\cong S,\]
	where the outermost maps are the unique maps specified by \autoref{unique_comp_Sas}.
\end{proposition}
\begin{proof}
	Let $(E,\mu,e)$, $x$, and $y$ as in the statement of the proposition. Now consider the following diagram
	% https://q.uiver.app/#q=WzAsNyxbMCwwLCJTXnthK2J9Il0sWzAsMiwiU157YStifSJdLFsyLDIsIlNeYlxcb3RpbWVzIFNeYSJdLFsyLDAsIlNeYVxcb3RpbWVzIFNeYiJdLFs0LDAsIkVcXG90aW1lcyBFIl0sWzQsMiwiRVxcb3RpbWVzIEUiXSxbNiwxLCJFIl0sWzAsMSwiXFxwaGlfe2IsYX1eey0xfVxcY2lyY1xcdGF1XFxjaXJjXFxwaGlfe2EsYn0iLDIseyJzdHlsZSI6eyJib2R5Ijp7Im5hbWUiOiJkYXNoZWQifX19XSxbMSwyLCJcXHBoaV97YixhfSJdLFswLDMsIlxccGhpX3thLGJ9Il0sWzMsMiwiXFx0YXUiLDJdLFs0LDUsIlxcdGF1IiwyXSxbNCw2LCJcXG11Il0sWzIsNSwieVxcb3RpbWVzIHgiXSxbNSw2LCJcXG11IiwyXSxbMyw0LCJ4XFxvdGltZXMgeSJdXQ==
	\[\begin{tikzcd}[sep=small]
		{S^{a+b}} && {S^a\otimes S^b} && {E\otimes E} \\
		&&&&&& E \\
		{S^{a+b}} && {S^b\otimes S^a} && {E\otimes E}
		\arrow["{\phi_{b,a}^{-1}\circ\tau\circ\phi_{a,b}}"', dashed, from=1-1, to=3-1]
		\arrow["{\phi_{b,a}}", from=3-1, to=3-3]
		\arrow["{\phi_{a,b}}", from=1-1, to=1-3]
		\arrow["\tau"', from=1-3, to=3-3]
		\arrow["\tau"', from=1-5, to=3-5]
		\arrow["\mu", from=1-5, to=2-7]
		\arrow["{y\otimes x}", from=3-3, to=3-5]
		\arrow["\mu"', from=3-5, to=2-7]
		\arrow["{x\otimes y}", from=1-3, to=1-5]
	\end{tikzcd}\]
	The left square commutes by definition. The middle square commutes by naturality of the symmetry isomorphism. Finally, the right square commutes by commutativity of $E$. Unravelling definitions, we have shown that under the product on $\pi_\ast(E)$ induced by the $\phi_{a,b}$'s,
	\[x\cdot y=(y\cdot x)\circ(\phi_{b,a}^{-1}\circ\tau\circ\phi_{a,b}).\]
	Thus, in order to show the desired result it further suffices to show that
	\[(y\cdot x)\circ(\phi_{b,a}^{-1}\circ\tau\circ\phi_{a,b})=y\cdot x\cdot(e\circ\theta_{a,b}).\]
	Consider the following diagram:
	% https://q.uiver.app/#q=WzAsMTIsWzAsMCwiU157YStifSJdLFswLDEsIlNeYlxcb3RpbWVzIFNeYVxcb3RpbWVzIFNeey1hLWJ9XFxvdGltZXMgU15hXFxvdGltZXMgU15iIl0sWzAsMiwiU15iXFxvdGltZXMgU15hXFxvdGltZXMgU157LWEtYn1cXG90aW1lcyBTXmJcXG90aW1lcyBTXmEiXSxbMCw0LCJFXFxvdGltZXMgRVxcb3RpbWVzIEUiXSxbMiw0LCJFXFxvdGltZXMgRSJdLFsxLDIsIlNeYlxcb3RpbWVzIFNeYSJdLFsyLDAsIlNeYVxcb3RpbWVzIFNeYiJdLFsyLDEsIlNeYlxcb3RpbWVzIFNeYSJdLFsyLDIsIlNee2ErYn0iXSxbMiwzLCJFXFxvdGltZXMgRSJdLFswLDUsIkVcXG90aW1lcyBFIl0sWzIsNSwiRSJdLFswLDEsIlxcY29uZyIsMl0sWzAsNiwiXFxwaGlfe2EsYn0iXSxbNiw3LCJcXHRhdSJdLFs3LDgsIlxccGhpX3tiLGF9XnstMX0iXSxbOCw1LCJcXHBoaV97YixhfSIsMl0sWzIsNywiXFxjb25nIl0sWzEsNiwiXFxjb25nIiwyXSxbOSwzLCJFXFxvdGltZXMgRVxcb3RpbWVzIGUiLDJdLFsxLDIsIlNeYlxcb3RpbWVzIFNeYVxcb3RpbWVzIFNeey1hLWJ9XFxvdGltZXNcXHRhdSIsMl0sWzMsMTAsIlxcbXVcXG90aW1lcyBFIiwyXSxbMyw0LCJFXFxvdGltZXMgXFxtdSJdLFs0LDExLCJcXG11Il0sWzEwLDExLCJcXG11IiwyXSxbOSw0LCIiLDAseyJsZXZlbCI6Miwic3R5bGUiOnsiaGVhZCI6eyJuYW1lIjoibm9uZSJ9fX1dLFs1LDksInlcXG90aW1lcyB4Il0sWzUsMywieVxcb3RpbWVzIHhcXG90aW1lcyBlIiwyXSxbMiw1LCJcXGNvbmciXSxbNSw3LCIiLDIseyJsZXZlbCI6Miwic3R5bGUiOnsiaGVhZCI6eyJuYW1lIjoibm9uZSJ9fX1dXQ==
	\[\begin{tikzcd}
		{S^{a+b}} && {S^a\otimes S^b} \\
		{S^b\otimes S^a\otimes S^{-a-b}\otimes S^a\otimes S^b} && {S^b\otimes S^a} \\
		{S^b\otimes S^a\otimes S^{-a-b}\otimes S^b\otimes S^a} & {S^b\otimes S^a} & {S^{a+b}} \\
		&& {E\otimes E} \\
		{E\otimes E\otimes E} && {E\otimes E} \\
		{E\otimes E} && E
		\arrow["\cong"', from=1-1, to=2-1]
		\arrow["{\phi_{a,b}}", from=1-1, to=1-3]
		\arrow["\tau", from=1-3, to=2-3]
		\arrow["{\phi_{b,a}^{-1}}", from=2-3, to=3-3]
		\arrow["{\phi_{b,a}}"', from=3-3, to=3-2]
		\arrow["\cong", from=3-1, to=2-3]
		\arrow["\cong"', from=2-1, to=1-3]
		\arrow["{E\otimes E\otimes e}"', from=4-3, to=5-1]
		\arrow["{S^b\otimes S^a\otimes S^{-a-b}\otimes\tau}"', from=2-1, to=3-1]
		\arrow["{\mu\otimes E}"', from=5-1, to=6-1]
		\arrow["{E\otimes \mu}", from=5-1, to=5-3]
		\arrow["\mu", from=5-3, to=6-3]
		\arrow["\mu"', from=6-1, to=6-3]
		\arrow[Rightarrow, no head, from=4-3, to=5-3]
		\arrow["{y\otimes x}", from=3-2, to=4-3]
		\arrow["{y\otimes x\otimes e}"', from=3-2, to=5-1]
		\arrow["\cong", from=3-1, to=3-2]
		\arrow[Rightarrow, no head, from=3-2, to=2-3]
	\end{tikzcd}\]
	Here any map simply labelled $\cong$ is an appropriate composition of copies of $\phi_{a,b}$'s, associators, and their inverses, so that each of these maps are necessarily unique by \autoref{unique_comp_Sas}. The triangles in the top large rectangle commutes by coherence for the $\phi_{a,b}$'s. The parallelogram commutes by naturality of $\tau$ and coherence of the of $\phi_{a,b}$'s. The middle skewed triangle commutes by functoriality of $-\otimes-$. The triangle below that commutes by unitality of $\mu$. Finally, the bottom rectangle commmutes by associativity of $\mu$. Hence, by unravelling definitions and applying functoriality of $-\otimes-$, we get that the right composition is $(y\cdot x)\circ(\phi_{b,a}^{-1}\circ\tau\circ\phi_{a,b})$, while the left composition is $y\cdot x\cdot(e\circ\theta_{a,b})$, so they are equal as desired.
\end{proof}

\begin{proposition}\label{theta_a,0=theta_0,a=id_S}
	Given $a\in A$, we have $\theta_{0,a}=\theta_{a,0}=\id_S$.
\end{proposition}
\begin{proof}
	Recall $\theta_{a,0}$ is the composition
	\[S\xr{\phi_{-a,a}} S^{-a}\otimes S^a\xr{S^{-a}\otimes\phi_{a,0}} S^{-a}\otimes(S^a\otimes S)\xr{S^{-a}\otimes\tau}S^{-a}\otimes(S\otimes S^a)\xr{S^{-a}\otimes\phi_{0,a}^{-1}} S^{-a}\otimes S^a\xr{\phi_{-a,a}^{-1}}S\]
	By the coherence theorem for symmetric monoidal categories and the fact that $\phi_{a,0}$ and $\phi_{0,a}$ coincide with the unitors, we have that the composition
	\[S^a\xr{\phi_{a,0}=\rho_{S^a}^{-1}} S^a\otimes S\xr\tau S\otimes S^a\xr{\phi_{0,a}^{-1}=\lambda_{S^a}}S^a\]
	is precisely the identity map, so by functoriality of $-\otimes-$, we have that $\theta_{a,0}$ is the composition
	\[S\xr{\phi_{-a,a}}S^{-a}\otimes S^a\xr=S^{-a}\otimes S^{a}\xr{\phi_{-a,a}^{-1}}S,\]
	so $\theta_{a,0}=\id_S$, meaning
	\[x\cdot y=y\cdot x\cdot(e\circ\theta_{a,0})=y\cdot x\cdot e=y\cdot x,\]
	where the last equality follows by the fact that $e$ is the unit for the multiplication on $\pi_\ast(E)$. An entirely analagous argument yields that $\theta_{0,a}=\id_S$.
\end{proof}

\begin{proposition}\label{bilinear}
	Let $X$ and $Y$ be objects in $\cSH$. Then the $A$-graded pairing
	\[\pi_*(X)\times\pi_*(Y)\to\pi_*(X\otimes Y)\]
	sending $x:S^a\to X$ and $ y:S^b\to Y$ to the composition
	\[S^{a+b}\xr{\phi_{a,b}} S^a\otimes S^b\xr{x\otimes y}X\otimes Y\]
	is additive in each argument.
\end{proposition}
\begin{proof}
	Let $a,b\in A$, and let $x_1,x_2:S^a\to X$ and $ y:S^b\to Y$. Then consider the following diagram
	\[\begin{tikzcd}
		{S^{a+b}} & {S^a\otimes S^b} & {(S^a\oplus S^a)\otimes S^b} \\
		& {(S^a\otimes S^b)\oplus(S^a\otimes S^b)} & {(X\oplus X)\otimes Y} \\
		& {(X\otimes Y)\oplus(X\otimes Y)} & {X\otimes Y}
		\arrow["{\Delta\otimes S^b}", from=1-2, to=1-3]
		\arrow["\Delta"', from=1-2, to=2-2]
		\arrow["{( x_1\oplus x_2)\otimes y}", from=1-3, to=2-3]
		\arrow["{\nabla\otimes Y}", from=2-3, to=3-3]
		\arrow["{( x_1\otimes y)\oplus( x_2\otimes y)}"', from=2-2, to=3-2]
		\arrow["\nabla", from=3-2, to=3-3]
		\arrow["\cong"', from=1-3, to=2-2]
		\arrow["\cong"', from=2-3, to=3-2]
		\arrow["\cong", from=1-1, to=1-2]
	\end{tikzcd}\]
	The isomorphisms are given by the fact that $-\otimes-$ is additive in each variable. Both triangles and the parallelogram commute since $-\otimes-$ is additive. By functoriality of $-\otimes-$, the top composition is $( x_1+ x_2)\cdot y$ and the bottom composition is $ x_1\cdot y+ x_2\cdot y$, so they are equal, as desired. An entirely analagous argument yields that $ x\cdot( y_1+ y_2)= x\cdot y_1+ x\cdot y_2$ for $ x\in\pi_*(X)$ and $ y_1, y_2\in\pi_*(Y)$.
\end{proof}

\begin{proposition}[{\cite[Proposition 5.11]{nlab:introduction_to_stable_homotopy_theory_--_1-2}}]\label{module}
	Let $(E,\mu,e)$ be a monoid object in $\cSH$. Then $E_*(-)$ is a functor from $\cSH$ to left $A$-graded $\pi_*(E)$-modules, where given some $X$ in $\cSH$, $E_*(X)$ may be endowed with the structure of a left $A$-graded $\pi_*(E)$-module via the map 
	\[\pi_*(E)\times E_*(X)\to E_*(X)\]
	which given $a,b\in A$, sends $x:S^a\to E$ and $y:S^b\to E\otimes X$ to the composition
	\[x\cdot y:S^{a+b}\cong S^a\otimes S^b\xr{x\otimes y}E\otimes (E\otimes X)\cong (E\otimes E)\otimes X\xr{\mu\otimes X}E\otimes X.\]
	Similarly, the assignment $X\mapsto X_*(E)$ is a functor from $\cSH$ to right $A$-graded $\pi_*(E)$-modules, where the structure map
	\[X_*(E)\times\pi_*(E)\to X_*(E)\]
	sends $x:S^a\to X\otimes E$ and $y:S^b\to E$ to the composition
	\[x\cdot y:S^{a+b}\cong S^a\otimes S^b\xr{x\otimes y}(X\otimes E)\otimes E\cong X\otimes(E\otimes E)\xr{X\otimes\mu}X\otimes E.\]
	Finally, $E_*(E)$ is a $\pi_*(E)$-bimodule, in the sense that the left and right actions of $\pi_*(E)$ are compatible, so that given $y, z\in\pi_*(E)$ and $x\in E_*(E)$, $y\cdot(x\cdot z)=(y\cdot x)\cdot z$.
\end{proposition}
\begin{proof}
	First we show that the map $\pi_*(E)\times E_*(X)\to E_*(X)$ endows $E_*(X)$ with the structure of a left $\pi_*(E)$-module. Let $a,b,c\in A$ and $x,x':S^a\to E\otimes X$, $y:S^b\to E$, and $z, z'\in S^c\to E$. Then we wish to show that:
	\begin{enumerate}
		\item $ y\cdot( x+ x')= y\cdot x+ y\cdot x'$, 
		\item $( z+ z')\cdot x= z\cdot x+ z'\cdot x$,
		\item $(zy)\cdot  x= z\cdot( y\cdot x)$,
		\item $e\cdot x= x$.
	\end{enumerate}
	Axioms $(1)$ and $(2)$ follow by the fact that $E_*(X)=\pi_*(E\otimes X)$ and \autoref{bilinear}. To see $(3)$, consider the diagram:
	% https://q.uiver.app/#q=WzAsNixbMCwxLCJTXnthK2IrY30iXSxbMSwxLCJTXmNcXG90aW1lcyBTXmJcXG90aW1lcyBTXmEiXSxbMiwxLCJFXFxvdGltZXMgRVxcb3RpbWVzIEVcXG90aW1lcyBYIl0sWzMsMiwiRVxcb3RpbWVzIEVcXG90aW1lcyBYIl0sWzMsMCwiRVxcb3RpbWVzIEVcXG90aW1lcyBYIl0sWzMsMSwiRVxcb3RpbWVzIFgiXSxbMCwxLCJcXGNvbmciXSxbMSwyLCJ6XFxvdGltZXMgeVxcb3RpbWVzIHgiXSxbMiwzLCJcXG11XFxvdGltZXMgRVxcb3RpbWVzIFgiLDJdLFsyLDQsIkVcXG90aW1lc1xcbXVcXG90aW1lcyBYIl0sWzQsNSwiXFxtdVxcb3RpbWVzIFgiXSxbMyw1LCJcXG11XFxvdGltZXMgWCIsMV1d
	\[\begin{tikzcd}
		&&& {E\otimes E\otimes X} \\
		{S^{a+b+c}} & {S^c\otimes S^b\otimes S^a} & {E\otimes E\otimes E\otimes X} & {E\otimes X} \\
		&&& {E\otimes E\otimes X}
		\arrow["\cong", from=2-1, to=2-2]
		\arrow["{z\otimes y\otimes x}", from=2-2, to=2-3]
		\arrow["{\mu\otimes E\otimes X}"', from=2-3, to=3-4]
		\arrow["{E\otimes\mu\otimes X}", from=2-3, to=1-4]
		\arrow["{\mu\otimes X}", from=1-4, to=2-4]
		\arrow["{\mu\otimes X}"{description}, from=3-4, to=2-4]
	\end{tikzcd}\]
	It commutes by associativity of $\mu$. By functoriality of $-\otimes-$, the two outside compositions equal $z\cdot(y\cdot x)$ on the top and $(z\cdot y)\cdot x$ on the bottom. Hence, they are equal, as desired.

	Next, to see $(4)$, consider the following diagram:
	% https://q.uiver.app/#q=WzAsNCxbMCwwLCJTXmEiXSxbMSwxLCJFXFxvdGltZXMgWCJdLFsyLDAsIkVcXG90aW1lcyAgWCJdLFsxLDIsIkVcXG90aW1lcyBFXFxvdGltZXMgWCJdLFsxLDIsIiIsMSx7ImxldmVsIjoyLCJzdHlsZSI6eyJoZWFkIjp7Im5hbWUiOiJub25lIn19fV0sWzEsMywiZVxcb3RpbWVzIEVcXG90aW1lcyBYIiwxXSxbMCwyLCJ4Il0sWzMsMiwiXFxtdVxcb3RpbWVzIFgiLDIseyJjdXJ2ZSI6M31dLFswLDEsIngiLDJdLFswLDMsImVcXG90aW1lcyB4IiwyLHsiY3VydmUiOjN9XV0=
	\[\begin{tikzcd}
		{S^a} && {E\otimes  X} \\
		& {E\otimes X} \\
		& {E\otimes E\otimes X}
		\arrow[Rightarrow, no head, from=2-2, to=1-3]
		\arrow["{e\otimes E\otimes X}"{description}, from=2-2, to=3-2]
		\arrow["x", from=1-1, to=1-3]
		\arrow["{\mu\otimes X}"', curve={height=18pt}, from=3-2, to=1-3]
		\arrow["x"', from=1-1, to=2-2]
		\arrow["{e\otimes x}"', curve={height=18pt}, from=1-1, to=3-2]
	\end{tikzcd}\]
	The top triangle commutes by definition. The left triangle commutes by functoriality of $-\otimes-$. The right triangle commutes by unitality of $\mu$.
	The top composition is $ x$ while the bottom is $e\cdot x$, thus they are necessarily equal since the diagram commutes.

	Thus, we have shown that the indicated map does indeed endow $E_*(X)$ with the structure of a left $\pi_*(E)$-module. It remains to show that $E_*(-)$ sends maps in $\cSH$ to $A$-graded homomorphisms of left $A$-graded $\pi_*(E)$-modules. By definition, given $f:X\to Y$ in $\cSH$, $E_*(f)$ is the map which takes a class $x:S^a\to E\otimes X$ to the composition 
	\[S^a\xr xE\otimes X\xr{E\otimes f}E\otimes Y.\]
	To see this assignment is a homomorphism, suppose we are given some other $x':S^a\to E\otimes X$ and some scalar $y:S^b\to E$. Then we would like to show $E_*(f)(x+x')=E_*(f)(x)+E_*(f)(x')$ and $E_*(f)(y\cdot x)=y\cdot E_*(f)(x)$. To see the former, consider the following diagram:
	% https://q.uiver.app/#q=WzAsNixbMCwxLCJTXmEiXSxbMSwxLCJTXmFcXG9wbHVzIFNeYSJdLFsyLDEsIihFXFxvdGltZXMgWClcXG9wbHVzKEVcXG90aW1lcyBYKSJdLFszLDAsIihFXFxvdGltZXMgWSlcXG9wbHVzKEVcXG90aW1lcyBZKSJdLFszLDEsIkVcXG90aW1lcyBZIl0sWzMsMiwiRVxcb3RpbWVzIFgiXSxbMCwxLCJcXERlbHRhIl0sWzEsMiwieFxcb3BsdXMgeCciXSxbMiwzLCIoRVxcb3RpbWVzIGYpXFxvcGx1cyAoRVxcb3RpbWVzIGYpIl0sWzMsNCwiXFxuYWJsYSJdLFsyLDUsIlxcbmFibGEiLDJdLFs1LDQsIkVcXG90aW1lcyBmIiwyXV0=
	\[\begin{tikzcd}
		&&& {(E\otimes Y)\oplus(E\otimes Y)} \\
		{S^a} & {S^a\oplus S^a} & {(E\otimes X)\oplus(E\otimes X)} & {E\otimes Y} \\
		&&& {E\otimes X}
		\arrow["\Delta", from=2-1, to=2-2]
		\arrow["{x\oplus x'}", from=2-2, to=2-3]
		\arrow["{(E\otimes f)\oplus (E\otimes f)}", from=2-3, to=1-4]
		\arrow["\nabla", from=1-4, to=2-4]
		\arrow["\nabla"', from=2-3, to=3-4]
		\arrow["{E\otimes f}"', from=3-4, to=2-4]
	\end{tikzcd}\]
	It commutes by naturality of $\nabla$ in an additive category. The top composition is $E_*(f)(x)+E_*(f)(x')$, while the bottom is $E_*(f)(x+x')$, so they are equal as desired. To see that $E_*(f)(y\cdot x)=y\cdot E_*(f)(x)$, consider the following diagram:
	% https://q.uiver.app/#q=WzAsNixbMCwwLCJTXnthK2J9Il0sWzEsMCwiU15iXFxvdGltZXMgU15hIl0sWzIsMCwiRVxcb3RpbWVzIEVcXG90aW1lcyBYIl0sWzMsMCwiRVxcb3RpbWVzIEVcXG90aW1lcyBZIl0sWzMsMSwiRVxcb3RpbWVzIFkiXSxbMiwxLCJFXFxvdGltZXMgWCJdLFswLDEsIlxccGhpX3tiLGF9Il0sWzEsMiwieVxcb3RpbWVzIHgiXSxbMiwzLCJFXFxvdGltZXMgRVxcb3RpbWVzIGYiXSxbMyw0LCJcXG11XFxvdGltZXMgWSJdLFsyLDUsIlxcbXVcXG90aW1lcyBYIiwyXSxbNSw0LCJFXFxvdGltZXMgZiJdXQ==
	\[\begin{tikzcd}
		{S^{a+b}} & {S^b\otimes S^a} & {E\otimes E\otimes X} & {E\otimes E\otimes Y} \\
		&& {E\otimes X} & {E\otimes Y}
		\arrow["{\phi_{b,a}}", from=1-1, to=1-2]
		\arrow["{y\otimes x}", from=1-2, to=1-3]
		\arrow["{E\otimes E\otimes f}", from=1-3, to=1-4]
		\arrow["{\mu\otimes Y}", from=1-4, to=2-4]
		\arrow["{\mu\otimes X}"', from=1-3, to=2-3]
		\arrow["{E\otimes f}", from=2-3, to=2-4]
	\end{tikzcd}\]
	It commutes by functoriality of $-\otimes-$. The bottom composition is $E_*(f)(y\cdot x)$, while the top composition is $y\cdot E_*(f)(x)$, so they are equal, as desired.
	
	Showing that $X_*(E)$ has the structure of a right $\pi_*(E)$-module and that if $f:X\to Y$ is a morphism in $\cSH$ then the map
	\[X_*(E)=[S^*,X\otimes E]\xr{(f\otimes E)_*}[S^*,Y\otimes E]=Y_*(E)\]
	is an $A$-graded homomorphism of right $A$-graded $\pi_*(E)$-modules is entirely analagous.

    It remains to show that $E_*(E)$ is a $\pi_*(E)$-bimodule. Let $x:S^a\to E$, $y:S^b\to E\otimes E$, and $z:S^c\to E$, and consider the following diagram:
	% https://q.uiver.app/#q=WzAsNixbMCwxLCJTXnthK2IrY30iXSxbMSwxLCJTXmFcXG90aW1lcyBTXmJcXG90aW1lcyBTXmMiXSxbMiwxLCJFXFxvdGltZXMgRVxcb3RpbWVzIEVcXG90aW1lcyBFIl0sWzMsMCwiRVxcb3RpbWVzIEVcXG90aW1lcyBFIl0sWzMsMiwiRVxcb3RpbWVzIEVcXG90aW1lcyBFIl0sWzMsMSwiRVxcb3RpbWVzIEUiXSxbMCwxLCJcXGNvbmciXSxbMSwyLCJ4XFxvdGltZXMgeVxcb3RpbWVzIHoiXSxbMiwzLCJcXG11XFxvdGltZXMgRVxcb3RpbWVzIEUiXSxbMiw0LCJFXFxvdGltZXMgRVxcb3RpbWVzIFxcbXUiLDJdLFsyLDUsIlxcbXVcXG90aW1lc1xcbXUiXSxbMyw1LCJFXFxvdGltZXNcXG11Il0sWzQsNSwiXFxtdVxcb3RpbWVzIEUiLDJdXQ==
	\[\begin{tikzcd}
		&&& {E\otimes E\otimes E} \\
		{S^{a+b+c}} & {S^a\otimes S^b\otimes S^c} & {E\otimes E\otimes E\otimes E} & {E\otimes E} \\
		&&& {E\otimes E\otimes E}
		\arrow["\cong", from=2-1, to=2-2]
		\arrow["{x\otimes y\otimes z}", from=2-2, to=2-3]
		\arrow["{\mu\otimes E\otimes E}", from=2-3, to=1-4]
		\arrow["{E\otimes E\otimes \mu}"', from=2-3, to=3-4]
		\arrow["\mu\otimes\mu", from=2-3, to=2-4]
		\arrow["E\otimes\mu", from=1-4, to=2-4]
		\arrow["{\mu\otimes E}"', from=3-4, to=2-4]
	\end{tikzcd}\]
	Commutativity follows by functoriality of $-\otimes-$, which also tells us that the two outside compositions are $(x\cdot y)\cdot z$ (on top) and $x\cdot(y\cdot z)$ (on bottom). Hence they are equal, as desired.
\end{proof}

\begin{proposition}\label{X,Y*_LES_appendix}
	Suppose we are given a distinguished triangle
	\[X\xr fY\xr gZ\xr h\Sigma X\]
	and some object $W$ in $\cSH$. Then there is an infinite long exact sequence
	\begin{equation}
		\label{X,Y*_LES_diag}\cdots\to[W,Z]_{*+\n+\1}\xr\partial[W,X]_{*+\n}\xr{f_*}[W,Y]_{*+\n}\xr{g_*}[W,Z]_{*+\n}\xr{\partial}[W,Z]_{*+\n-\1}\to\cdots,
	\end{equation}
	where $\partial:[W,Z]_{*+\n+\1}\to[W,X]_{*+\n}$ sends a class $x:S^{a+\n+1}\otimes W\to Z$ to the composition
	\[S^{a+\n}\otimes W\cong S^{-\1}\otimes S^{a+\n+\1}\otimes W\xrightarrow{S^{-\1}\otimes x}S^{-\1}\otimes Z\xrightarrow{S^{-\1}\otimes h}S^{-\1}\otimes \Sigma X\xrightarrow{S^{-\1}\otimes \nu_X}S^{-\1}\otimes S^\1\otimes X\xrightarrow{\phi_{-\1,\1}^{-\1}\otimes X}X.\]
\end{proposition}
\begin{proof}
	Given $n>0$, we will write $\Sigma^{-n}$ to denote the functor $\Omega^n=(S^{-\1})^n\otimes-$ in this proof. 
	
	For all $n>0$, the $\phi_{a,b}$'s yield natural isomorphisms
	\[s^{-n}:\Omega^nX=(S^{-\1})^n\otimes X\xr{\phi\otimes X} S^{-\n}\otimes X=\Omega^\n X,\]
	where here we are writing $\phi$ as a stand-in for the unique isomorphism $(S^{-\1})^n\cong S^{-\n}$ that can be obtained by composing copies of tensor products of $\phi_{a,b}$'s, associators, unitors, and their inverses (\autoref{unique_comp_Sas}). Similarly, we have natural isomorphisms $s^n:\Sigma^nX\cong S^\n\otimes X$ given by the composition
	\[\Sigma^nX\xrightarrow{\nu^n_X}(S^\1)^n\otimes X\xrightarrow{\phi\otimes X}S^\n\otimes X=\Sigma^\n X,\]
	where again $\phi$ stands for the canonical isomorphism $(S^\1)^n\cong S^\n$, and we inductively define $\nu^1:=\nu$ and $\nu^{n+1}_X$ to be the composition
	\[\Sigma^{n+1}X=\Sigma^n\Sigma X\xrightarrow{\nu^n_{\Sigma X}}(S^\1)^n\otimes\Sigma X\xrightarrow{(S^\1)^n\otimes\nu_X}(S^\1)^{n}\otimes S^\1\otimes X=(S^\1)^{n+1}\otimes X.\]
	Finally, we define $s^0$ to be the identity natural transformation on $\cSH$. Then together with the natural isomorphisms $r^\n_{W,V}:[W,\Sigma^\n V]_*\cong[W,V]_{*-\n}$ given by \autoref{X,X,Sigma^aY*_eq_X,Y*-a}, we get the following natural isomorphisms of $A$-graded abelian groups for all $n\in\bZ$
	\[\ell^{n}_{V}:[W,\Sigma^n V]_*\xrightarrow{(s^n_V)_*}[W,\Sigma^\n V]_*\xr{r_{W,V}^\n}[W,V]_{*-\n}.\]
	Now, given $n\in\bZ$, consider the following diagram
	% https://q.uiver.app/#q=WzAsMTAsWzEsMCwiW1csXFxTaWdtYV5uWF1fKiJdLFsyLDAsIltXLFxcU2lnbWFeblldXyoiXSxbMywwLCJbVyxcXFNpZ21hXm5aXV8qIl0sWzEsMSwiW1csWF1feyotXFxufSJdLFsyLDEsIltXLFldX3sqLVxcbn0iXSxbMywxLCJbVyxaXV97Ki1cXG59Il0sWzQsMCwiW1csXFxTaWdtYV57bisxfVhdXyoiXSxbNCwxLCJbVyxYXV97Ki1cXG4tXFwxfSJdLFswLDAsIltXLFxcU2lnbWFee24tMX1aXV8qIl0sWzAsMSwiW1csWl1feyotXFxuK1xcMX0iXSxbMCwxLCJcXFNpZ21hXm5mXyoiXSxbMSwyLCJcXFNpZ21hXm5nXyoiXSxbMyw0LCJmXyoiLDJdLFs0LDUsImdfKiIsMl0sWzEsNCwiXFxlbGxebl9ZIiwyXSxbMiw1LCJcXGVsbF5uX1oiLDJdLFsyLDYsImhfbiJdLFs1LDcsIlxccGFydGlhbCIsMl0sWzYsNywiXFxlbGxee24rMX1fWCJdLFswLDMsIlxcZWxsXm5fWCIsMl0sWzgsOSwiXFxlbGxee24tMX1fWiIsMl0sWzksMywiXFxwYXJ0aWFsIiwyXSxbOCwwLCJoX3tuLTF9Il1d
	\[\begin{tikzcd}
		{[W,\Sigma^{n-1}Z]_*} & {[W,\Sigma^nX]_*} & {[W,\Sigma^nY]_*} & {[W,\Sigma^nZ]_*} & {[W,\Sigma^{n+1}X]_*} \\
		{[W,Z]_{*-\n+\1}} & {[W,X]_{*-\n}} & {[W,Y]_{*-\n}} & {[W,Z]_{*-\n}} & {[W,X]_{*-\n-\1}}
		\arrow["{\Sigma^nf_*}", from=1-2, to=1-3]
		\arrow["{\Sigma^ng_*}", from=1-3, to=1-4]
		\arrow["{f_*}"', from=2-2, to=2-3]
		\arrow["{g_*}"', from=2-3, to=2-4]
		\arrow["{\ell^n_Y}"', from=1-3, to=2-3]
		\arrow["{\ell^n_Z}"', from=1-4, to=2-4]
		\arrow["{h_n}", from=1-4, to=1-5]
		\arrow["\partial"', from=2-4, to=2-5]
		\arrow["{\ell^{n+1}_X}", from=1-5, to=2-5]
		\arrow["{\ell^n_X}"', from=1-2, to=2-2]
		\arrow["{\ell^{n-1}_Z}"', from=1-1, to=2-1]
		\arrow["\partial"', from=2-1, to=2-2]
		\arrow["{h_{n-1}}", from=1-1, to=1-2]
	\end{tikzcd}\]
	where $h_n=\Sigma^nh$ if $n\geq0$ and $h_n=\Sigma^{n+1}\wt h$ if $n<0$ (where $\wt h:\Omega Z\to X$ is the adjoint of $h:Z\to\Sigma X$). The top row is exact by \autoref{dist_tri_LES}, and we have constructed the vertical arrows to be isomorphisms. The inner two squares commute by naturality of $\ell^n$. Thus in order to show exactness of the bottom row, it suffices to show the outermost squares commute. Since our choice of $n\in\bZ$ is arbitrary, it further suffices to show the right square commutes. \todo{finish proof}
	% https://q.uiver.app/#q=WzAsMTMsWzMsMywiW1csWF1feyorXFxuLVxcMX0iXSxbMywwLCJbVyxcXE9tZWdhXntuLTF9WF1fKiJdLFsyLDAsIltXLFxcT21lZ2Fee259IFpdXyoiXSxbMiwzLCJbVyxcXE9tZWdhIFpdX3sqK1xcbi1cXDF9Il0sWzIsMSwiW1csXFxPbWVnYV57XFxuLVxcMX1cXE9tZWdhIFpdXyoiXSxbMywxLCJbVyxcXE9tZWdhXntcXG4tXFwxfVhdXyoiXSxbMiwyLCJbU157XFxuLVxcMX1cXG90aW1lcyBTXipcXG90aW1lcyBXLFxcT21lZ2EgWl0iXSxbMywyLCJbU157XFxuLVxcMX1cXG90aW1lcyBTXipcXG90aW1lcyBXLFhdIl0sWzIsNCwiW1NeXFwxXFxvdGltZXMgU157KitcXG4tXFwxfVxcb3RpbWVzIFcsWl0iXSxbMCwwLCJbVyxcXE9tZWdhXlxcbiBaXV8qIl0sWzAsNSwiW1NeXFxuXFxvdGltZXMgU14qXFxvdGltZXMgVyxaXSJdLFsyLDUsIltXLFpdX3sqK1xcbn0iXSxbMSwzLCJcXHN1YnN0YWNre1xcdGV4dHt0aGlzIHJlZ2lvbiBjb21tdXRlcyB9XFxcXFxcdGV4dHtieSBjb2hlcmVuY2V9fSJdLFsyLDEsIlxcT21lZ2Fee24tMX1cXHd0IGhfKiJdLFszLDAsIlxcd3QgaF8qIl0sWzIsNCwiKFxccGhpXFxvdGltZXNcXE9tZWdhIFopXyoiLDJdLFsxLDUsIihcXHBoaVxcb3RpbWVzIFgpXyoiXSxbNCw1LCJcXE9tZWdhXntcXG4tXFwxfVxcd3QgaF8qIl0sWzQsNiwiYWRqIiwyXSxbNSw3LCJhZGoiXSxbNiw3LCJcXHd0IGhfKiJdLFs2LDMsIihcXHBoaVxcb3RpbWVzIFcpXioiLDJdLFs3LDAsIihcXHBoaVxcb3RpbWVzIFcpXioiXSxbMyw4LCJhZGoiLDJdLFsyLDksIihcXHBoaVxcb3RpbWVzIFopXyoiLDJdLFs5LDEwLCJhZGoiLDJdLFs4LDExLCIoXFxwaGlcXG90aW1lcyBXKV4qIiwyXSxbMTEsMCwiXFxwYXJ0aWFsIiwyXSxbMTAsMTEsIihcXHBoaVxcb3RpbWVzIFcpXioiLDJdLFs5LDQsIihcXHBoaVxcb3RpbWVzIFopXyoiXSxbMTAsOCwiKFxccGhpXFxvdGltZXMgVyleKiJdXQ==
	\[\begin{tikzcd}
		{[W,\Omega^\n Z]_*} && {[W,\Omega^{n} Z]_*} & {[W,\Omega^{n-1}X]_*} \\
		&& {[W,\Omega^{\n-\1}\Omega Z]_*} & {[W,\Omega^{\n-\1}X]_*} \\
		&& {[S^{\n-\1}\otimes S^*\otimes W,\Omega Z]} & {[S^{\n-\1}\otimes S^*\otimes W,X]} \\
		& {\substack{\text{this region commutes }\\\text{by coherence}}} & {[W,\Omega Z]_{*+\n-\1}} & {[W,X]_{*+\n-\1}} \\
		&& {[S^\1\otimes S^{*+\n-\1}\otimes W,Z]} \\
		{[S^\n\otimes S^*\otimes W,Z]} && {[W,Z]_{*+\n}}
		\arrow["{\Omega^{n-1}\wt h_*}", from=1-3, to=1-4]
		\arrow["{\wt h_*}", from=4-3, to=4-4]
		\arrow["{(\phi\otimes\Omega Z)_*}"', from=1-3, to=2-3]
		\arrow["{(\phi\otimes X)_*}", from=1-4, to=2-4]
		\arrow["{\Omega^{\n-\1}\wt h_*}", from=2-3, to=2-4]
		\arrow["adj"', from=2-3, to=3-3]
		\arrow["adj", from=2-4, to=3-4]
		\arrow["{\wt h_*}", from=3-3, to=3-4]
		\arrow["{(\phi\otimes W)^*}"', from=3-3, to=4-3]
		\arrow["{(\phi\otimes W)^*}", from=3-4, to=4-4]
		\arrow["adj"', from=4-3, to=5-3]
		\arrow["{(\phi\otimes Z)_*}"', from=1-3, to=1-1]
		\arrow["adj"', from=1-1, to=6-1]
		\arrow["{(\phi\otimes W)^*}"', from=5-3, to=6-3]
		\arrow["\partial"', from=6-3, to=4-4]
		\arrow["{(\phi\otimes W)^*}"', from=6-1, to=6-3]
		\arrow["{(\phi\otimes Z)_*}", from=1-1, to=2-3]
		\arrow["{(\phi\otimes W)^*}", from=6-1, to=5-3]
	\end{tikzcd}\]
	% https://q.uiver.app/#q=WzAsMjIsWzAsMCwiW1csXFxTaWdtYV5uWl1fKiJdLFswLDEsIltXLFpdX3sqLVxcbn0iXSxbMSwwLCJbVyxcXFNpZ21hXntuKzF9WF1fKiJdLFsxLDEsIltXLFxcU2lnbWEgWF1feyotXFxufSJdLFsxLDIsIltXLFxcU2lnbWFeXFwxWF1feyotXFxufSJdLFsxLDMsIltXLFhdX3sqLVxcbi1cXDF9Il0sWzAsNSwiW1csXFxTaWdtYV57bisxfVhdXyoiXSxbMCw3LCJbVyxcXFNpZ21hXntcXG59XFxTaWdtYSBYXV8qIl0sWzAsOCwiW1Neey1cXG59XFxvdGltZXMgU14qXFxvdGltZXMgVyxcXFNpZ21hIFhdIl0sWzAsOSwiW1csXFxTaWdtYSBYXV97Ki1cXG59Il0sWzAsMTAsIltXLFxcU2lnbWFeXFwxWF1feyotXFxufSJdLFswLDExLCJbU157LVxcMX1cXG90aW1lcyBTXnsqLVxcbn1cXG90aW1lcyBXLFhdIl0sWzAsMTIsIltXLFhdX3sqLVxcbi1cXDF9Il0sWzMsNSwiW1csKFNeXFwxKV57bisxfVxcb3RpbWVzIFhdXyoiXSxbMCw2LCJbVywoU15cXDEpXm5cXG90aW1lc1xcU2lnbWEgWF0iXSxbMyw3LCJbVyxcXFNpZ21hXntcXG4rXFwxfVhdXyoiXSxbMywxMiwiW1Neey1cXG4tXFwxfVxcb3RpbWVzIFNeKlxcb3RpbWVzIFcsWF0iXSxbMSw3LCJbVyxcXFNpZ21hXlxcblxcU2lnbWFeXFwxIFhdXyoiXSxbMSw4LCJbU157LVxcbn1cXG90aW1lcyBTXipcXG90aW1lcyBXLFNeXFwxXFxvdGltZXMgWF0iXSxbMSw0LCJcXHN1YnN0YWNre3tcXHRleHR7ZGlhZ3JhbSBjaGFzZX19XFxcXFxcc3Vic3RhY2t7XFx0ZXh0e2Fyb3VuZCBhYm92ZX1cXFxcXFx0ZXh0e2NyZXNjZW50fX19OiJdLFsyLDEwLCJcXHN1YnN0YWNre1xcdGV4dHt0aGlzIHJlZ2lvbn1cXFxcXFxzdWJzdGFja3tcXHRleHR7Y29tbXV0ZXN9XFxcXFxcdGV4dHtieSBjb2hlcmVuY2V9fX0iXSxbMCwzLCJcXHN1YnN0YWNre1xcdGV4dHtzZWVpbmcgdGhhdH1cXFxcXFxzdWJzdGFja3tcXHRleHR7dGhpcyB0cmlhbmdsZX1cXFxcXFx0ZXh0e2NvbW11dGVzIGlzIGVhc3l9fX0iXSxbMCwxLCJcXGVsbF5uX1oiLDJdLFswLDIsIlxcU2lnbWFebmhfKiJdLFsyLDMsIlxcZWxsXntufV97XFxTaWdtYSBYfSJdLFsxLDMsImhfKiJdLFszLDQsIihcXG51X1gpXyoiXSxbNCw1LCJyXlxcMV97VyxYfSJdLFsxLDUsIlxccGFydGlhbCIsMl0sWzcsOCwiYWRqIiwyXSxbOCw5LCIoXFxwaGlcXG90aW1lcyBXKV4qIiwyXSxbMTAsMTEsImFkaiIsMl0sWzExLDEyLCIoXFxwaGlcXG90aW1lcyBXKV4qIiwyXSxbNiwxMywiKFxcbnVfWF57bisxfSlfKiJdLFs2LDE0LCIoXFxudV57bn1fe1xcU2lnbWEgWH0pXyoiLDJdLFsxNCw3LCIoXFxwaGlcXG90aW1lc1xcU2lnbWEgWClfKiIsMl0sWzE0LDEzLCIoKFNeXFwxKV5uXFxvdGltZXNcXG51X1gpXyoiLDFdLFs5LDEwLCIoXFxudV9YKV8qIiwyXSxbMTMsMTUsIihcXHBoaVxcb3RpbWVzIFgpXyoiXSxbMTUsMTYsImFkaiJdLFsxNiwxMiwiKFxccGhpXFxvdGltZXMgVyleKiJdLFs3LDE3LCIoXFxTaWdtYV5cXG5cXG51X1gpXyoiXSxbMTcsMTUsIihcXHBoaSBcXG90aW1lcyBYKV8qIl0sWzEzLDE3LCIoXFxwaGlcXG90aW1lcyBYKSIsMV0sWzgsMTgsIihcXG51X1gpXyoiXSxbMTcsMTgsImFkaiJdLFsxOCwxMCwiKFxccGhpXFxvdGltZXMgVyleKiJdLFsyLDUsIlxcZWxsXntuKzF9X1giLDAseyJjdXJ2ZSI6LTV9XSxbMTEsMTYsIihcXHBoaVxcb3RpbWVzIFcpXioiXV0=
	\[\begin{tikzcd}
		{[W,\Sigma^nZ]_*} & {[W,\Sigma^{n+1}X]_*} \\
		{[W,Z]_{*-\n}} & {[W,\Sigma X]_{*-\n}} \\
		& {[W,\Sigma^\1X]_{*-\n}} \\
		{\substack{\text{seeing that}\\\substack{\text{this triangle}\\\text{commutes is easy}}}} & {[W,X]_{*-\n-\1}} \\
		& {\substack{{\text{diagram chase}}\\\substack{\text{around above}\\\text{crescent}}}:} \\
		{[W,\Sigma^{n+1}X]_*} &&& {[W,(S^\1)^{n+1}\otimes X]_*} \\
		{[W,(S^\1)^n\otimes\Sigma X]} \\
		{[W,\Sigma^{\n}\Sigma X]_*} & {[W,\Sigma^\n\Sigma^\1 X]_*} && {[W,\Sigma^{\n+\1}X]_*} \\
		{[S^{-\n}\otimes S^*\otimes W,\Sigma X]} & {[S^{-\n}\otimes S^*\otimes W,S^\1\otimes X]} \\
		{[W,\Sigma X]_{*-\n}} \\
		{[W,\Sigma^\1X]_{*-\n}} && {\substack{\text{this region}\\\substack{\text{commutes}\\\text{by coherence}}}} \\
		{[S^{-\1}\otimes S^{*-\n}\otimes W,X]} \\
		{[W,X]_{*-\n-\1}} &&& {[S^{-\n-\1}\otimes S^*\otimes W,X]}
		\arrow["{\ell^n_Z}"', from=1-1, to=2-1]
		\arrow["{\Sigma^nh_*}", from=1-1, to=1-2]
		\arrow["{\ell^{n}_{\Sigma X}}", from=1-2, to=2-2]
		\arrow["{h_*}", from=2-1, to=2-2]
		\arrow["{(\nu_X)_*}", from=2-2, to=3-2]
		\arrow["{r^\1_{W,X}}", from=3-2, to=4-2]
		\arrow["\partial"', from=2-1, to=4-2]
		\arrow["adj"', from=8-1, to=9-1]
		\arrow["{(\phi\otimes W)^*}"', from=9-1, to=10-1]
		\arrow["adj"', from=11-1, to=12-1]
		\arrow["{(\phi\otimes W)^*}"', from=12-1, to=13-1]
		\arrow["{(\nu_X^{n+1})_*}", from=6-1, to=6-4]
		\arrow["{(\nu^{n}_{\Sigma X})_*}"', from=6-1, to=7-1]
		\arrow["{(\phi\otimes\Sigma X)_*}"', from=7-1, to=8-1]
		\arrow["{((S^\1)^n\otimes\nu_X)_*}"{description}, from=7-1, to=6-4]
		\arrow["{(\nu_X)_*}"', from=10-1, to=11-1]
		\arrow["{(\phi\otimes X)_*}", from=6-4, to=8-4]
		\arrow["adj", from=8-4, to=13-4]
		\arrow["{(\phi\otimes W)^*}", from=13-4, to=13-1]
		\arrow["{(\Sigma^\n\nu_X)_*}", from=8-1, to=8-2]
		\arrow["{(\phi \otimes X)_*}", from=8-2, to=8-4]
		\arrow["{(\phi\otimes X)}"{description}, from=6-4, to=8-2]
		\arrow["{(\nu_X)_*}", from=9-1, to=9-2]
		\arrow["adj", from=8-2, to=9-2]
		\arrow["{(\phi\otimes W)^*}", from=9-2, to=11-1]
		\arrow["{\ell^{n+1}_X}", curve={height=-30pt}, from=1-2, to=4-2]
		\arrow["{(\phi\otimes W)^*}", from=12-1, to=13-4]
	\end{tikzcd}\]
\end{proof}

\begin{proposition}[{\cite[Proposition 2.2]{nlab:introduction_to_the_adams_spectral_sequence}}]\label{Kunneth_map}
	Let $(E,\mu,e)$ be a monoid object in $\cSH$ and let $X$ be any object. Then the assignment
	\[E_*(E)\times E_*(X)\to E_*(E\otimes X)\]
	which sends $x:S^{a}\to E\otimes E$ and $ y:S^{b}\to E\otimes X$ to the composition
	\[x\cdot y:S^{a+b}\cong S^{a}\otimes S^{b}\xr{x\otimes y}E\otimes E\otimes E\otimes X\xr{E\otimes\mu\otimes X}E\otimes E\otimes X\]
	lifts to an $A$-graded homomorphism of left $A$-graded $\pi_*(E)$-modules
	\[\Phi_X:E_*(E)\otimes_{\pi_*(E)}E_*(X)\to E_*(E\otimes X)\]
	(where here $E_*(E)$ has a $\pi_*(E)$-bimodule structure and $E_*(X)$ has a left $\pi_*(E)$-module structure as specified by \autoref{module}, so $E_*(E)\otimes_{\pi_*(E)}E_*(X)$ is a left $A$-graded $\pi_*(E)$-module by \autoref{tensor_of_A_graded_is_A_graded}). Furthermore, this homomorphism is natural in $X$.
\end{proposition}
\begin{proof}
	First, recall by definition of the tensor product, in order to show the assignment $E_*(E)\times E_*(X)\to E_*(E\otimes X)$ induces a homomorphism $E_*(E)\otimes_{\pi_*(E)}E_*(X)\to E_*(E\otimes X)$ of $A$-graded abelian groups, it suffices to show that the assignment is $\pi_*(E)$-balanced, i.e., that it is linear in each argument and satisfies $xr\cdot y=x\cdot ry$ for $x\in E_*(E)$, $y\in E_*(X)$, and $r\in\pi_*(E)$.
	
	First, note that by the identifications $E_*(E)=\pi_*(E\otimes E)$, $E_*(X)=\pi_*(E\otimes X)$, and $E_*(E\otimes X)=\pi_*(E\otimes E\otimes X)$, and \autoref{bilinear}, it is straightforward to see that the assignment commutes with addition of maps in each argument. Now, let $a,b,c\in A$, $x:S^a\to E\otimes E$, $y:S^b\to E\otimes X$, and $z:S^c\to E$. Then we wish to show $x z\cdot y=x\cdot z y$. Consider the following diagram (where here we are passing to a permutative category):
	% https://q.uiver.app/#q=WzAsNixbMCwxLCJTXnthK2IrY30iXSxbMSwxLCJTXmFcXG90aW1lcyBTXmNcXG90aW1lcyBTXmIiXSxbMiwxLCJFXFxvdGltZXMgRVxcb3RpbWVzIEVcXG90aW1lcyBFXFxvdGltZXMgWCJdLFszLDAsIkVcXG90aW1lcyBFXFxvdGltZXMgRVxcb3RpbWVzIFgiXSxbMywxLCJFXFxvdGltZXMgRVxcb3RpbWVzIFgiXSxbMywyLCJFXFxvdGltZXMgRVxcb3RpbWVzIEVcXG90aW1lcyBYIl0sWzAsMSwiXFxjb25nIl0sWzEsMiwieFxcb3RpbWVzIHpcXG90aW1lcyB5Il0sWzIsMywiRVxcb3RpbWVzIFxcbXVcXG90aW1lcyBFXFxvdGltZXMgWCJdLFszLDQsIkVcXG90aW1lcyBcXG11XFxvdGltZXMgWCJdLFsyLDUsIkVcXG90aW1lcyBFXFxvdGltZXMgXFxtdVxcb3RpbWVzIFgiLDJdLFs1LDQsIkVcXG90aW1lcyBcXG11XFxvdGltZXMgWCIsMl1d
	\[\begin{tikzcd}
		&&& {E\otimes E\otimes E\otimes X} \\
		{S^{a+b+c}} & {S^a\otimes S^c\otimes S^b} & {E\otimes E\otimes E\otimes E\otimes X} & {E\otimes E\otimes X} \\
		&&& {E\otimes E\otimes E\otimes X}
		\arrow["\cong", from=2-1, to=2-2]
		\arrow["{x\otimes z\otimes y}", from=2-2, to=2-3]
		\arrow["{E\otimes \mu\otimes E\otimes X}", from=2-3, to=1-4]
		\arrow["{E\otimes \mu\otimes X}", from=1-4, to=2-4]
		\arrow["{E\otimes E\otimes \mu\otimes X}"', from=2-3, to=3-4]
		\arrow["{E\otimes \mu\otimes X}"', from=3-4, to=2-4]
	\end{tikzcd}\]
	It commutes by associativity of $\mu$. By functoriality of $-\otimes-$, the top composition is given by $(x z)\cdot y$ and the bottom composition is $x\cdot( z y)$, so we have they are equal, as desired. Thus, since the map $E_*(E)\times E_*(X)\to E_*(E\otimes X)$ is $\pi_*(E)$-balanced, we have that it induces a homomorphism of abelian groups. Furthermore, by \autoref{tensor_lift_of_A_graded_is_A_graded} it is $A$-graded.
	
	In order to see this map is a homomorphism of left $\pi_*(E)$-modules, we must show that $z(x\cdot y)=zx\cdot y$, where $x$, $y$, and $z$ are defined as above. Now consider the following diagram:
	% https://q.uiver.app/#q=WzAsNixbMCwxLCJTXnthK2IrY30iXSxbMSwxLCJTXmNcXG90aW1lcyBTXmFcXG90aW1lcyBTXmIiXSxbMiwxLCJFXFxvdGltZXMgRVxcb3RpbWVzIEVcXG90aW1lcyBFXFxvdGltZXMgWCJdLFszLDAsIkVcXG90aW1lcyBFXFxvdGltZXMgRVxcb3RpbWVzIFgiXSxbMywxLCJFXFxvdGltZXMgRVxcb3RpbWVzIFgiXSxbMywyLCJFXFxvdGltZXMgRVxcb3RpbWVzIEVcXG90aW1lcyBYIl0sWzAsMSwiXFxjb25nIl0sWzEsMiwielxcb3RpbWVzIHhcXG90aW1lcyB5Il0sWzIsMywiXFxtdVxcb3RpbWVzIEVcXG90aW1lcyBFXFxvdGltZXMgWCJdLFszLDQsIkVcXG90aW1lcyBcXG11XFxvdGltZXMgWCJdLFsyLDUsIkVcXG90aW1lcyBFXFxvdGltZXMgXFxtdVxcb3RpbWVzIFgiLDJdLFs1LDQsIlxcbXVcXG90aW1lcyBFXFxvdGltZXMgWCIsMl0sWzIsNCwiXFxtdVxcb3RpbWVzXFxtdVxcb3RpbWVzIFgiXV0=
	\[\begin{tikzcd}
		&&& {E\otimes E\otimes E\otimes X} \\
		{S^{a+b+c}} & {S^c\otimes S^a\otimes S^b} & {E\otimes E\otimes E\otimes E\otimes X} & {E\otimes E\otimes X} \\
		&&& {E\otimes E\otimes E\otimes X}
		\arrow["\cong", from=2-1, to=2-2]
		\arrow["{z\otimes x\otimes y}", from=2-2, to=2-3]
		\arrow["{\mu\otimes E\otimes E\otimes X}", from=2-3, to=1-4]
		\arrow["{E\otimes \mu\otimes X}", from=1-4, to=2-4]
		\arrow["{E\otimes E\otimes \mu\otimes X}"', from=2-3, to=3-4]
		\arrow["{\mu\otimes E\otimes X}"', from=3-4, to=2-4]
		\arrow["{\mu\otimes\mu\otimes X}", from=2-3, to=2-4]
	\end{tikzcd}\]
	Commutativity of the triangles is functoriality of $-\otimes-$. By functoriality of $-\otimes-$, the top composition is $zx\cdot y$, and the bottom composition is $z(x\cdot y)$. Hence they are equal, as desired, so that the map we have constructed
	\[E_*(E)\otimes_{\pi_*(E)}E_*(X)\to E_*(E\otimes X)\]
	is indeed an $A$-graded homomorphism of left $A$-graded $\pi_*(E)$-modules.

	Next, we would like to show that this homomorphism is natural in $X$. Let $f:X\to Y$ in $\cSH$. Then we would like to show the following diagram commutes:
	% https://q.uiver.app/#q=WzAsNCxbMCwwLCJFXyooRSlcXG90aW1lc197XFxwaV8qKEUpfUVfKihYKSJdLFswLDEsIkVfKihFKVxcb3RpbWVzX3tcXHBpXyooRSl9RV8qKFkpIl0sWzEsMSwiRV8qKEVcXG90aW1lcyBZKSJdLFsxLDAsIkVfKihFXFxvdGltZXMgWCkiXSxbMCwxLCJFXyooRSlcXG90aW1lc197XFxwaV8qKEUpfUVfKihmKSIsMl0sWzEsMiwiXFxQaGlfWSJdLFswLDMsIlxcUGhpX1giXSxbMywyLCJFXyooRVxcb3RpbWVzIGYpIl1d
	\begin{equation}\label{naturality_diagram_for_E*EoxE_*X-->E_*(EoxX)}\begin{tikzcd}
		{E_*(E)\otimes_{\pi_*(E)}E_*(X)} & {E_*(E\otimes X)} \\
		{E_*(E)\otimes_{\pi_*(E)}E_*(Y)} & {E_*(E\otimes Y)}
		\arrow["{E_*(E)\otimes_{\pi_*(E)}E_*(f)}"', from=1-1, to=2-1]
		\arrow["{\Phi_Y}", from=2-1, to=2-2]
		\arrow["{\Phi_X}", from=1-1, to=1-2]
		\arrow["{E_*(E\otimes f)}", from=1-2, to=2-2]
	\end{tikzcd}\end{equation}
	As all the maps here are homomorphisms, it suffices to chase generators around the diagram. In particular, suppose we are given $x:S^a\to E\otimes E$ and $y:S^b\to E\otimes X$, and consider the following diagram exhibiting the two possible ways to chase $x\otimes y$ around the diagram (as usual, we are passing to a symmetric strict monoidal category):
	% https://q.uiver.app/#q=WzAsNixbMCwwLCJTXnthK2J9Il0sWzEsMCwiU15hXFxvdGltZXMgU15iIl0sWzIsMCwiRVxcb3RpbWVzIEVcXG90aW1lcyBFXFxvdGltZXMgWCJdLFszLDAsIkVcXG90aW1lcyBFXFxvdGltZXMgWCJdLFszLDEsIkVcXG90aW1lcyBFXFxvdGltZXMgWSJdLFsyLDEsIkVcXG90aW1lcyBFXFxvdGltZXMgRVxcb3RpbWVzIFkiXSxbMCwxLCJcXHBoaV97YSxifSJdLFsxLDIsInhcXG90aW1lcyB5Il0sWzIsMywiRVxcb3RpbWVzIFxcbXVcXG90aW1lcyBYIl0sWzMsNCwiRVxcb3RpbWVzIEVcXG90aW1lcyBmIl0sWzIsNSwiRVxcb3RpbWVzIEVcXG90aW1lcyBFXFxvdGltZXMgZiIsMl0sWzUsNCwiRVxcb3RpbWVzIFxcbXVcXG90aW1lcyBZIl1d
	\[\begin{tikzcd}
		{S^{a+b}} & {S^a\otimes S^b} & {E\otimes E\otimes E\otimes X} & {E\otimes E\otimes X} \\
		&& {E\otimes E\otimes E\otimes Y} & {E\otimes E\otimes Y}
		\arrow["{\phi_{a,b}}", from=1-1, to=1-2]
		\arrow["{x\otimes y}", from=1-2, to=1-3]
		\arrow["{E\otimes \mu\otimes X}", from=1-3, to=1-4]
		\arrow["{E\otimes E\otimes f}", from=1-4, to=2-4]
		\arrow["{E\otimes E\otimes E\otimes f}"', from=1-3, to=2-3]
		\arrow["{E\otimes \mu\otimes Y}", from=2-3, to=2-4]
	\end{tikzcd}\]
	This diagram commutes by functoriality of $-\otimes-$. Thus we have that diagram (\ref{naturality_diagram_for_E*EoxE_*X-->E_*(EoxX)}) does indeed commute, as desired.
\end{proof}

\begin{lemma}\label{E_homology_suspension_iso_t^a's_appendix}
	Let $E$ and $X$ be objects in $\cSH$. Then for all $a\in A$, there is an $A$-graded isomorphism of $A$-graded abelian groups
	\[t^a_X:E_*(\Sigma^aX)\cong E_{*-a}(X)\]
	which sends a class $S^b\to E\otimes\Sigma^aX=E\otimes S^a\otimes X$ to the composition
	\[S^{b-a}\xr{\phi_{b,-a}} S^b\otimes S^{-a}\xr{x\otimes S^{-a}}E\otimes S^a\otimes X\otimes S^{-a}\xr{E\otimes S^a\otimes\tau_{X,S^{-a}}}E\otimes S^a\otimes S^{-a}\otimes X\xr{E\otimes\phi_{a,-a}^{-1}\otimes X}E\otimes X\]
	(where here we are ignoring associators and unitors). Furthermore this isomorphism is natural in $X$, and if $E$ is a monoid object in $\cSH$ then it is a natural isomorphism of $\pi_*(E)$-modules.
\end{lemma}
\begin{proof}
	Expressed in terms of hom-sets, $t^a_X$ is precisely the composition
	% https://q.uiver.app/#q=WzAsNyxbMSwwLCJbU14qLEVcXG90aW1lcyBTXmFcXG90aW1lcyBYXSJdLFsxLDEsIltTXipcXG90aW1lcyBTXnstYX0sRVxcb3RpbWVzIFNeYVxcb3RpbWVzIFhcXG90aW1lcyBTXnstYX1dIl0sWzEsMiwiIFtTXnsqLWF9LEVcXG90aW1lcyBTXmFcXG90aW1lcyBYXFxvdGltZXMgU157LWF9XSJdLFsxLDMsIiBbU157Ki1hfSxFXFxvdGltZXMgU15hXFxvdGltZXMgU157LWF9XFxvdGltZXMgWF0iXSxbMSw0LCIgW1NeeyotYX0sRVxcb3RpbWVzIFhdIl0sWzIsNCwiRV97Ki1hfShFXFxvdGltZXMgWCkiXSxbMCwwLCJFXyooXFxTaWdtYV5hWCkiXSxbMCwxLCItXFxvdGltZXMgU157LWF9Il0sWzEsMiwiKFxccGhpX3sqLC1hfSleKiJdLFsyLDMsIihFXFxvdGltZXMgU15hXFxvdGltZXNcXHRhdSlfKiJdLFszLDQsIihFXFxvdGltZXNcXHBoaV97YSwtYX1eey0xfVxcb3RpbWVzIFgpXyoiXSxbNCw1LCIiLDAseyJsZXZlbCI6Miwic3R5bGUiOnsiaGVhZCI6eyJuYW1lIjoibm9uZSJ9fX1dLFs2LDAsIiIsMCx7ImxldmVsIjoyLCJzdHlsZSI6eyJoZWFkIjp7Im5hbWUiOiJub25lIn19fV1d
	\[\begin{tikzcd}[column sep=tiny]
		{E_*(\Sigma^aX)} & {[S^*,E\otimes S^a\otimes X]} \\
		& {[S^*\otimes S^{-a},E\otimes S^a\otimes X\otimes S^{-a}]} \\
		& { [S^{*-a},E\otimes S^a\otimes X\otimes S^{-a}]} \\
		& { [S^{*-a},E\otimes S^a\otimes S^{-a}\otimes X]} \\
		& { [S^{*-a},E\otimes X]} & {E_{*-a}(E\otimes X)}
		\arrow["{-\otimes S^{-a}}", from=1-2, to=2-2]
		\arrow["{(\phi_{*,-a})^*}", from=2-2, to=3-2]
		\arrow["{(E\otimes S^a\otimes\tau)_*}", from=3-2, to=4-2]
		\arrow["{(E\otimes\phi_{a,-a}^{-1}\otimes X)_*}", from=4-2, to=5-2]
		\arrow[Rightarrow, no head, from=5-2, to=5-3]
		\arrow[Rightarrow, no head, from=1-1, to=1-2]
	\end{tikzcd}\]
	We know the first vertical arrow is an isomorphism of abelian groups as $-\otimes-$ is additive in each variable (since $\cSH$ is tensor triangulated) and $\Omega^a\cong -\otimes S^{-a}$ is an autoequivalence of $\cSH$ by \autoref{Sigma^a,Sigma^-a_adjoint_equiv}.  The three other vertical arrows are given by composing with an isomorphism in an additive category, so they are also isomorphisms.
	
	To see $t_X^a$ is a homomorphism of left $\pi_*(E)$-modules, suppose we are given classes $r:S^b\to E$ $\pi_b(E)$ and $x:S^c\to E\otimes S^a\otimes X$ in $E_c(\Sigma^aX)$. Then we wish to show that $t_X^a(r\cdot x)=r\cdot t_X^a(x)$. Consider the following diagram:
	% https://q.uiver.app/#q=WzAsOCxbMCwwLCJTXntiK2MtYX0iXSxbMCwyLCJTXmJcXG90aW1lcyBTXmNcXG90aW1lcyBTXnstYX0iXSxbMiwyLCJFXFxvdGltZXMgRVxcb3RpbWVzIFNeYVxcb3RpbWVzIFhcXG90aW1lcyBTXnstYX0iXSxbMiwwLCJFXFxvdGltZXMgU15hXFxvdGltZXMgWFxcb3RpbWVzIFNeey1hfSJdLFsyLDQsIkVcXG90aW1lcyBFXFxvdGltZXMgU15hXFxvdGltZXMgU157LWF9XFxvdGltZXMgWCJdLFs0LDQsIkVcXG90aW1lcyBFXFxvdGltZXMgWCJdLFs0LDIsIkVcXG90aW1lcyBYIl0sWzQsMCwiRVxcb3RpbWVzIFNeYVxcb3RpbWVzIFNeey1hfVxcb3RpbWVzIFgiXSxbMCwxLCJcXGNvbmciXSxbMSwyLCJyXFxvdGltZXMgeFxcb3RpbWVzIFNeey1hfSJdLFsyLDMsIlxcbXVcXG90aW1lcyBTXmFcXG90aW1lcyBYXFxvdGltZXMgU157LWF9Il0sWzIsNCwiRVxcb3RpbWVzIEVcXG90aW1lcyBTXmFcXG90aW1lcyBcXHRhdV97WCxTXnstYX19IiwyXSxbNCw1LCJFXFxvdGltZXMgRVxcb3RpbWVzIFxccGhpX3thLC1hfV57LTF9XFxvdGltZXMgWCIsMl0sWzUsNiwiXFxtdVxcb3RpbWVzIFgiLDJdLFszLDcsIkVcXG90aW1lcyBTXmFcXG90aW1lcyBcXHRhdV97WCxTXnstYX19Il0sWzcsNiwiRVxcb3RpbWVzIFxccGhpX3thLC1hfV57LTF9XFxvdGltZXMgWCJdLFs0LDcsIlxcbXVcXG90aW1lcyBTXmFcXG90aW1lcyBTXnstYX1cXG90aW1lcyBYIiwyXV0=
	\[\begin{tikzcd}
		{S^{b+c-a}} && {E\otimes S^a\otimes X\otimes S^{-a}} && {E\otimes S^a\otimes S^{-a}\otimes X} \\
		\\
		{S^b\otimes S^c\otimes S^{-a}} && {E\otimes E\otimes S^a\otimes X\otimes S^{-a}} && {E\otimes X} \\
		\\
		&& {E\otimes E\otimes S^a\otimes S^{-a}\otimes X} && {E\otimes E\otimes X}
		\arrow["\cong", from=1-1, to=3-1]
		\arrow["{r\otimes x\otimes S^{-a}}", from=3-1, to=3-3]
		\arrow["{\mu\otimes S^a\otimes X\otimes S^{-a}}", from=3-3, to=1-3]
		\arrow["{E\otimes E\otimes S^a\otimes \tau_{X,S^{-a}}}"', from=3-3, to=5-3]
		\arrow["{E\otimes E\otimes \phi_{a,-a}^{-1}\otimes X}"', from=5-3, to=5-5]
		\arrow["{\mu\otimes X}"', from=5-5, to=3-5]
		\arrow["{E\otimes S^a\otimes \tau_{X,S^{-a}}}", from=1-3, to=1-5]
		\arrow["{E\otimes \phi_{a,-a}^{-1}\otimes X}", from=1-5, to=3-5]
		\arrow["{\mu\otimes S^a\otimes S^{-a}\otimes X}"', from=5-3, to=1-5]
	\end{tikzcd}\]
	Both triangles commute by functoriality of $-\otimes-$. The top composition is $t_X^a(r\cdot x)$ while the bottom is $r\cdot t_X^a(x)$, so they are equal as desired.
	
	It remains to show $t^a_X$ is natural in $X$. let $f:X\to Y$ in $\cSH$, then we would like to show the following diagram commutes:
	% https://q.uiver.app/#q=WzAsNCxbMCwwLCJFXyooXFxTaWdtYV5hWCkiXSxbMSwwLCJFX3sqLWF9KFgpIl0sWzEsMSwiRV97Ki1hfShZKSJdLFswLDEsIkVfKihcXFNpZ21hXmFZKSJdLFswLDEsInReYV9YIl0sWzEsMiwiRV97Ki1hfShmKSJdLFswLDMsIkVfKihcXFNpZ21hXmFmKSIsMl0sWzMsMiwidF5hX1kiXV0=
	\begin{equation}\label{naturality_of_t^a_diagram}\begin{tikzcd}
		{E_*(\Sigma^aX)} & {E_{*-a}(X)} \\
		{E_*(\Sigma^aY)} & {E_{*-a}(Y)}
		\arrow["{t^a_X}", from=1-1, to=1-2]
		\arrow["{E_{*-a}(f)}", from=1-2, to=2-2]
		\arrow["{E_*(\Sigma^af)}"', from=1-1, to=2-1]
		\arrow["{t^a_Y}", from=2-1, to=2-2]
	\end{tikzcd}\end{equation}
	We may chase a generator around the diagram since all the arrows here are homomorphisms. Let $x:S^b\to E\otimes S^a\otimes X$ in $E_*(\Sigma^aX)$. Then consider the following diagram:
	% https://q.uiver.app/#q=WzAsOCxbMCwwLCJTXntiLWF9Il0sWzEsMCwiU15iXFxvdGltZXMgU157LWF9Il0sWzIsMCwiRVxcb3RpbWVzIFNeYVxcb3RpbWVzIFhcXG90aW1lcyBTXnstYX0iXSxbMywwLCJFXFxvdGltZXMgU15hXFxvdGltZXMgU157LWF9XFxvdGltZXMgWCJdLFs0LDAsIkVcXG90aW1lcyBYIl0sWzQsMSwiRVxcb3RpbWVzIFkiXSxbMiwxLCJFXFxvdGltZXMgU15hXFxvdGltZXMgWVxcb3RpbWVzIFNeey1hfSJdLFszLDEsIkVcXG90aW1lcyBTXmFcXG90aW1lcyBTXnstYX1cXG90aW1lcyBZIl0sWzAsMSwiXFxjb25nIl0sWzEsMiwieFxcb3RpbWVzIFNeey1hfSJdLFsyLDMsIkVcXG90aW1lcyBTXmFcXG90aW1lcyBcXHRhdSJdLFszLDQsIkVcXG90aW1lcyBcXHBoaV97YSwtYX1eey0xfVxcb3RpbWVzIFgiXSxbNCw1LCJFXFxvdGltZXMgZiJdLFsyLDYsIkVcXG90aW1lcyBTXnthfVxcb3RpbWVzIGZcXG90aW1lcyBTXnstYX0iLDJdLFs2LDcsIkVcXG90aW1lcyBTXmFcXG90aW1lcyBcXHRhdSIsMl0sWzcsNSwiRVxcb3RpbWVzIFxccGhpX3thLC1hfV57LTF9XFxvdGltZXMgWSIsMl0sWzMsNywiRVxcb3RpbWVzIFNeYVxcb3RpbWVzIFNeey1hfVxcb3RpbWVzIGYiLDJdXQ==
	\[\begin{tikzcd}
		{S^{b-a}} & {S^b\otimes S^{-a}} & {E\otimes S^a\otimes X\otimes S^{-a}} & {E\otimes S^a\otimes S^{-a}\otimes X} & {E\otimes X} \\
		&& {E\otimes S^a\otimes Y\otimes S^{-a}} & {E\otimes S^a\otimes S^{-a}\otimes Y} & {E\otimes Y}
		\arrow["\cong", from=1-1, to=1-2]
		\arrow["{x\otimes S^{-a}}", from=1-2, to=1-3]
		\arrow["{E\otimes S^a\otimes \tau}", from=1-3, to=1-4]
		\arrow["{E\otimes \phi_{a,-a}^{-1}\otimes X}", from=1-4, to=1-5]
		\arrow["{E\otimes f}", from=1-5, to=2-5]
		\arrow["{E\otimes S^{a}\otimes f\otimes S^{-a}}"', from=1-3, to=2-3]
		\arrow["{E\otimes S^a\otimes \tau}"', from=2-3, to=2-4]
		\arrow["{E\otimes \phi_{a,-a}^{-1}\otimes Y}"', from=2-4, to=2-5]
		\arrow["{E\otimes S^a\otimes S^{-a}\otimes f}"', from=1-4, to=2-4]
	\end{tikzcd}\]
	The left rectangle commutes by naturality of $\tau$, while the right rectangle commutes by functoriality of $-\otimes-$. The two outside compositions are the two ways to chase $x$ around diagram (\ref{naturality_of_t^a_diagram}), so the diagram commutes as desired.
\end{proof}

\begin{lemma}\label{t's_commute_with_Phi's}
	Given a monoid object $(E,\mu,e)$ in $\cSH$, the maps $\Phi_X$ constructed in \autoref{Kunneth_map} commute with the natural isomorphisms $t^a_X:E_*(\Sigma^aX)\xr\cong E_{*-a}(X)$ given in \autoref{E_homology_suspension_iso_t^a's_appendix}, in the sense that the following diagram commutes for all $a\in A$ and $X$ in $\cSH$:
	% https://q.uiver.app/#q=WzAsNCxbMCwwLCJFXyooRSlcXG90aW1lc197XFxwaV8qKEUpfUVfKihcXFNpZ21hXmFYKSJdLFsyLDAsIkVfKihFKVxcb3RpbWVzX3tcXHBpXyooRSl9RV97Ki1hfShYKSJdLFswLDEsIkVfKihFXFxvdGltZXNcXFNpZ21hXmFYKSJdLFsyLDEsIkVfeyotYX0oRVxcb3RpbWVzIFgpIl0sWzAsMSwiRV8qKEUpXFxvdGltZXMgdF5hX1giXSxbMCwyLCJcXFBoaV97XFxTaWdtYV5hWH0iLDJdLFsyLDMsInReYV9YIl0sWzEsMywiXFxQaGlfWCJdXQ==
	\[\begin{tikzcd}
		{E_*(E)\otimes_{\pi_*(E)}E_*(\Sigma^aX)} && {E_*(E)\otimes_{\pi_*(E)}E_{*-a}(X)} \\
		{E_*(E\otimes\Sigma^aX)} && {E_{*-a}(E\otimes X)}
		\arrow["{E_*(E)\otimes t^a_X}", from=1-1, to=1-3]
		\arrow["{\Phi_{\Sigma^aX}}"', from=1-1, to=2-1]
		\arrow["{t^a_X}", from=2-1, to=2-3]
		\arrow["{\Phi_X}", from=1-3, to=2-3]
	\end{tikzcd}\]
	where the top arrow is well-defined since $t_X^a$ is a left $\pi_*(E)$-modue homomorphism by the above lemma, and we are being abusive in that the bottom arrow is given by the composition
	\[E_*(E\otimes\Sigma^aX)\overset\alpha\cong(E\otimes E)_*(\Sigma^aX)\xr{t_X^a}(E\otimes E)_{*-a}(X)\overset\alpha\cong E_{*-a}(E\otimes X).\]
\end{lemma}
\begin{proof}
	Since all the maps in the above diagram are homomorphisms, we can chase generators around to show it commutes. Let $x:S^b\to E\otimes E$ and $y:S^c\to E\otimes\Sigma^aX=E\otimes S^a\otimes X$. Then consider the following diagram:
	% https://q.uiver.app/#q=WzAsOCxbMCwwLCJTXntiK2MtYX0iXSxbMCwyLCJTXmJcXG90aW1lcyBTXmNcXG90aW1lcyBTXnstYX0iXSxbMywyLCJFXFxvdGltZXMgRVxcb3RpbWVzIEVcXG90aW1lcyBTXmFcXG90aW1lcyBYXFxvdGltZXMgU157LWF9Il0sWzMsMCwiRVxcb3RpbWVzIEVcXG90aW1lcyBFXFxvdGltZXMgU15hXFxvdGltZXMgU157LWF9XFxvdGltZXMgWCJdLFs0LDAsIkVcXG90aW1lcyBFXFxvdGltZXMgRVxcb3RpbWVzIFgiXSxbNCwyLCJFXFxvdGltZXMgRVxcb3RpbWVzIFgiXSxbMyw0LCJFXFxvdGltZXMgRVxcb3RpbWVzIFNeYVxcb3RpbWVzIFhcXG90aW1lcyBTXnstYX0iXSxbNCw0LCJFXFxvdGltZXMgRVxcb3RpbWVzIFNeYVxcb3RpbWVzIFNeey1hfVxcb3RpbWVzIFgiXSxbMCwxLCJcXGNvbmciXSxbMiwzLCJFXFxvdGltZXMgRVxcb3RpbWVzIEVcXG90aW1lcyBTXmFcXG90aW1lcyBcXHRhdSJdLFszLDQsIkVcXG90aW1lcyBFXFxvdGltZXMgRVxcb3RpbWVzIFxccGhpX3thLC1hfV57LTF9XFxvdGltZXMgWCJdLFs0LDUsIkVcXG90aW1lcyBcXG11XFxvdGltZXMgWCJdLFsyLDYsIkVcXG90aW1lcyBcXG11XFxvdGltZXMgU15hXFxvdGltZXMgWFxcb3RpbWVzIFNeey1hfSIsMl0sWzYsNywiRVxcb3RpbWVzIEVcXG90aW1lcyBTXmFcXG90aW1lcyBcXHRhdSIsMl0sWzcsNSwiRVxcb3RpbWVzIEVcXG90aW1lcyBcXHBoaV97YSwtYX1eey0xfVxcb3RpbWVzIFgiLDJdLFszLDcsIkVcXG90aW1lcyBcXG11XFxvdGltZXMgU15hXFxvdGltZXMgU157LWF9XFxvdGltZXMgWCJdLFsxLDIsInhcXG90aW1lcyB5XFxvdGltZXMgU157LWF9Il1d
	\[\begin{tikzcd}
		{S^{b+c-a}} &&& {E\otimes E\otimes E\otimes S^a\otimes S^{-a}\otimes X} & {E\otimes E\otimes E\otimes X} \\
		\\
		{S^b\otimes S^c\otimes S^{-a}} &&& {E\otimes E\otimes E\otimes S^a\otimes X\otimes S^{-a}} & {E\otimes E\otimes X} \\
		\\
		&&& {E\otimes E\otimes S^a\otimes X\otimes S^{-a}} & {E\otimes E\otimes S^a\otimes S^{-a}\otimes X}
		\arrow["\cong", from=1-1, to=3-1]
		\arrow["{E\otimes E\otimes E\otimes S^a\otimes \tau}", from=3-4, to=1-4]
		\arrow["{E\otimes E\otimes E\otimes \phi_{a,-a}^{-1}\otimes X}", from=1-4, to=1-5]
		\arrow["{E\otimes \mu\otimes X}", from=1-5, to=3-5]
		\arrow["{E\otimes \mu\otimes S^a\otimes X\otimes S^{-a}}"', from=3-4, to=5-4]
		\arrow["{E\otimes E\otimes S^a\otimes \tau}"', from=5-4, to=5-5]
		\arrow["{E\otimes E\otimes \phi_{a,-a}^{-1}\otimes X}"', from=5-5, to=3-5]
		\arrow["{E\otimes \mu\otimes S^a\otimes S^{-a}\otimes X}", from=1-4, to=5-5]
		\arrow["{x\otimes y\otimes S^{-a}}", from=3-1, to=3-4]
	\end{tikzcd}\]
	Each triangle commutes by functoriality of $-\otimes-$. The two outside compositions are the two ways to chase $x\otimes y$ around the diagram in the statement of the lemma, so the diagram commutes as desired.
\end{proof}

\begin{corollary}\label{t_commutes_with_Phi's_corollary}
	For all $X$ in $\cSH$, we have natural isomorphisms $t_X:E_*E(\Sigma X)\xr\cong E_{*-\1}(X)$ given by the composition
	\[E_*(\Sigma X)\xr{E_*(\nu_X)}E_*(\Sigma^1 X)\xr{t^\1_X}E_{*-\1}(X).\]
	Furthermore, by naturality of $\Phi$ and the fact that $t_X^\1$ commutes with $\Phi$ (in the sense of the above lemma), this isomorphism also commutes with $\Phi$.
\end{corollary}

\begin{proposition}\label{Kunneth_iso_for_cellular_objects}
	Let $(E,\mu,e)$ be a flat monoid object in $\cSH$ (\autoref{flat}) and let $X$ be any cellular object in $\cSH$ (\autoref{cellular}). Then the natural homomorphism
	\[\Phi_X:E_*(E)\otimes_{\pi_*(E)} E_*(X)\to E_*(E\otimes X)\]
	given in \autoref{Kunneth_map} is an isomorphism of left $\pi_*(E)$-modules.
\end{proposition}
\begin{proof}
	In this proof, we will freely employ the coherence theorem for symmetric monoidal categories, and we will assume that associativity and unitality of $-\otimes-$ holds up to strict equality. To start, let $\cE$ be the collection of objects $X$ in $\cSH$ for which this map is an isomorphism. Then in order to show $\cE$ contains every cellular object, it suffices to show that $\cE$ satisfies the three conditions given for the class of cellular objects in \autoref{cellular}. First, we need to show that $\Phi$ is an isomorphism when $X=S^a$ for some $a\in A$.
%	Note that
%	\[E_*(S^a)=[S^*,E\otimes S^a]\cong[S^{-a}\otimes S^*,E]\cong[S^{*-a},E]=\pi_{*-a}(E),\]
%	where the first isomorphism follows by the adunction between $S^{-a}\otimes-$ and $-\otimes S^a\cong S^a\otimes-$ (\autoref{Sigma^a,Sigma^-a_adjoint_equiv}). Similarly, we have
%	\[E_*(E\otimes S^a)=[S^*,E\otimes E\otimes S^a]\cong[S^{*-a},E\otimes E]=E_{*-a}(E).\]
%	Hence by \autoref{tensor_shift_A_graded} we have isomorphisms
%	\[E_*(E)\otimes_{\pi_*(E)}E_*(S^a)\cong E_*(E)\otimes_{\pi_*(E)}\pi_{*-a}(E)\cong E_{*-a}(E)\cong E_*(E\otimes S^a).\]
	Indeed, consider the map
	\begin{align*}
		\Psi:E_*(E\otimes S^a)&\to E_*(E)\otimes_{\pi_*(E)}E_*(S^a)
	\end{align*}
	which sends a class $x:S^b\to E\otimes E\otimes S^a$ in $E_b(E\otimes S^a)$ to the pure tensor $\wt x\otimes\wt e$, where $\wt x\in E_{b-a}(E)$ is the composition
	\[S^{b-a}\cong S^b\otimes S^{-a}\xr{x\otimes S^{-a}}E\otimes E\otimes S^a\otimes S^{-a}\xr{E\otimes E\otimes\phi_{a,-a}^{-1}}E\otimes E\]
	and $\wt e\in E_a(S^a)$ is the composition
	\[S^a\cong S\otimes S^a\xr{e\otimes S^a}E\otimes S^a.\]
	First, note $\Psi$ is an ($A$-graded) homomorphism of abelian groups: Given $x,x'\in E_b(E\otimes S^a)$, we would like to show that $\wt x\otimes\wt e+\wt x'\otimes\wt e=\wt{x+x'}\otimes\wt e$. It suffices to show that $\wt x+\wt x'=\wt{x+x'}$. To see this, consider the following diagram (again, we are passing to a symmetric strict monoidal category):
	% https://q.uiver.app/#q=WzAsMTAsWzAsMCwiU157Yi1hfSJdLFsxLDAsIlNee2ItYX1cXG9wbHVzIFNee2ItYX0iXSxbMSwxLCIoU15iXFxvdGltZXMgU157LWF9KVxcb3BsdXMoU15iXFxvdGltZXMgU157LWF9KSJdLFsxLDIsIihFXFxvdGltZXMgRVxcb3RpbWVzIFNeYVxcb3RpbWVzIFNeey1hfSlcXG9wbHVzKEVcXG90aW1lcyBFXFxvdGltZXMgU15hXFxvdGltZXMgU157LWF9KSJdLFsxLDMsIihFXFxvdGltZXMgRSlcXG9wbHVzKEVcXG90aW1lcyBFKSJdLFsxLDQsIkVcXG90aW1lcyBFIl0sWzAsMSwiU15iXFxvdGltZXMgU157LWF9Il0sWzAsMiwiKFNeYlxcb3BsdXMgU15iKVxcb3RpbWVzIFNeey1hfSJdLFswLDMsIigoRVxcb3RpbWVzIEVcXG90aW1lcyBTXmEpXFxvcGx1cyAoRVxcb3RpbWVzIEVcXG90aW1lcyBTXmEpKVxcb3RpbWVzIFNeey1hfSJdLFswLDQsIkVcXG90aW1lcyBFXFxvdGltZXMgU15hXFxvdGltZXMgU157LWF9Il0sWzAsMSwiXFxEZWx0YSJdLFsxLDIsIlxccGhpX3tiLC1hfVxcb3BsdXNcXHBoaV97YiwtYX0iXSxbMiwzLCIoeFxcb3RpbWVzIFNeey1hfSlcXG9wbHVzKHgnXFxvdGltZXMgU157LWF9KSJdLFszLDQsIihFXFxvdGltZXMgRVxcb3RpbWVzXFxwaGlfe2EsLWF9XnstMX0pXFxvcGx1cyhFXFxvdGltZXMgRVxcb3RpbWVzXFxwaGlfe2EsLWF9XnstMX0pIl0sWzQsNSwiXFxuYWJsYSJdLFswLDYsIlxccGhpX3tiLWF9IiwyXSxbNiw3LCJcXERlbHRhXFxvdGltZXMgU157LWF9IiwyXSxbNyw4LCIoeFxcb3BsdXMgeCcpXFxvdGltZXMgU157LWF9IiwyXSxbOCw5LCJcXG5hYmxhXFxvdGltZXMgU157LWF9IiwyXSxbOSw1LCJFXFxvdGltZXMgRVxcb3RpbWVzXFxwaGlfe2EsLWF9XnstMX0iLDJdLFs3LDIsIlxcY29uZyJdLFs4LDMsIlxcY29uZyJdLFszLDksIlxcbmFibGEiXSxbNiwyLCJcXERlbHRhIl1d
	\[\begin{tikzcd}
		{S^{b-a}} & {S^{b-a}\oplus S^{b-a}} \\
		{S^b\otimes S^{-a}} & {(S^b\otimes S^{-a})\oplus(S^b\otimes S^{-a})} \\
		{(S^b\oplus S^b)\otimes S^{-a}} & {(E\otimes E\otimes S^a\otimes S^{-a})\oplus(E\otimes E\otimes S^a\otimes S^{-a})} \\
		{((E\otimes E\otimes S^a)\oplus (E\otimes E\otimes S^a))\otimes S^{-a}} & {(E\otimes E)\oplus(E\otimes E)} \\
		{E\otimes E\otimes S^a\otimes S^{-a}} & {E\otimes E}
		\arrow["\Delta", from=1-1, to=1-2]
		\arrow["{\phi_{b,-a}\oplus\phi_{b,-a}}", from=1-2, to=2-2]
		\arrow["{(x\otimes S^{-a})\oplus(x'\otimes S^{-a})}", from=2-2, to=3-2]
		\arrow["{(E\otimes E\otimes\phi_{a,-a}^{-1})\oplus(E\otimes E\otimes\phi_{a,-a}^{-1})}", from=3-2, to=4-2]
		\arrow["\nabla", from=4-2, to=5-2]
		\arrow["{\phi_{b-a}}"', from=1-1, to=2-1]
		\arrow["{\Delta\otimes S^{-a}}"', from=2-1, to=3-1]
		\arrow["{(x\oplus x')\otimes S^{-a}}"', from=3-1, to=4-1]
		\arrow["{\nabla\otimes S^{-a}}"', from=4-1, to=5-1]
		\arrow["{E\otimes E\otimes\phi_{a,-a}^{-1}}"', from=5-1, to=5-2]
		\arrow["\cong", from=3-1, to=2-2]
		\arrow["\cong", from=4-1, to=3-2]
		\arrow["\nabla", from=3-2, to=5-1]
		\arrow["\Delta", from=2-1, to=2-2]
	\end{tikzcd}\]
	The top rectangle commutes by naturality of $\Delta$ in an additive category. The bottom triangle commutes by naturality of $\nabla$ in an additive category. Finally, the remaining regions of the diagram commute by additivity of $-\otimes-$. By functoriality of $-\otimes-$, it follows that the left composition is $\wt{x+x'}$ and the right composition is $\wt x+\wt x'$, so they are equal as desired. Thus $\Psi$ is a homomorphism of abelian groups, as desired.

	Now, we claim that $\Psi$ is an inverse to $\Phi$, (which is enough to show $\Phi$ is an isomorphism of left $\pi_*(E)$-modules). Since $\Phi$ and $\Psi$ are homomorphisms it suffices to check that they are inverses on generators. First, let $x:S^b\to E\otimes E\otimes S^a$ in $E_b(E\otimes S^a)$. We would like to show that $\Phi(\Psi(x))=x$. Consider the following diagram, where here we are passing to a symmetric strict monoidal category:
	% https://q.uiver.app/#q=WzAsOCxbMCwwLCJTXmIiXSxbMiwwLCJTXmJcXG90aW1lcyBTXnstYX1cXG90aW1lcyBTXmEiXSxbNCwxLCJFXFxvdGltZXMgRVxcb3RpbWVzIFNeYVxcb3RpbWVzIFNeey1hfVxcb3RpbWVzIEVcXG90aW1lcyBTXmEiXSxbMCwzLCJFXFxvdGltZXMgRVxcb3RpbWVzIEVcXG90aW1lcyBTXmEiXSxbMCwyLCJFXFxvdGltZXMgRVxcb3RpbWVzIFNeYSJdLFsyLDMsIkVcXG90aW1lcyBFXFxvdGltZXMgRVxcb3RpbWVzIFNeYSJdLFsyLDEsIkVcXG90aW1lcyBFXFxvdGltZXMgU15hIFxcb3RpbWVzIFNeey1hfVxcb3RpbWVzIFNeYSJdLFsyLDIsIkVcXG90aW1lcyBFXFxvdGltZXMgU15hIl0sWzAsMSwiXFxjb25nIl0sWzEsMiwieFxcb3RpbWVzIFNeey1hfVxcb3RpbWVzIGVcXG90aW1lcyBTXmEiXSxbMyw0LCJFXFxvdGltZXMgXFxtdVxcb3RpbWVzIFNeYSJdLFswLDQsIngiLDJdLFs0LDUsIkVcXG90aW1lcyBFXFxvdGltZXMgZVxcb3RpbWVzIFNeYSIsMV0sWzUsMywiIiwxLHsibGV2ZWwiOjIsInN0eWxlIjp7ImhlYWQiOnsibmFtZSI6Im5vbmUifX19XSxbMSw2LCJ4XFxvdGltZXMgU157LWF9XFxvdGltZXMgU15hIiwxXSxbNCw2LCJFXFxvdGltZXMgRVxcb3RpbWVzIFNeYVxcb3RpbWVzIFxccGhpX3stYSxhfSJdLFs2LDIsIkVcXG90aW1lcyBFXFxvdGltZXMgU15hXFxvdGltZXMgU157LWF9XFxvdGltZXMgZVxcb3RpbWVzIFNeYSIsMl0sWzcsNiwiRVxcb3RpbWVzIEVcXG90aW1lcyBcXHBoaV97YSwtYX1cXG90aW1lcyBTXmEiLDFdLFs3LDUsIkVcXG90aW1lcyBFXFxvdGltZXMgZVxcb3RpbWVzIFNeYSIsMV0sWzcsNCwiIiwyLHsibGV2ZWwiOjIsInN0eWxlIjp7ImhlYWQiOnsibmFtZSI6Im5vbmUifX19XSxbMiw1LCJFXFxvdGltZXMgRVxcb3RpbWVzIFxccGhpX3thLC1hfV57LTF9XFxvdGltZXMgRVxcb3RpbWVzIFNeYSJdXQ==
	\[\begin{tikzcd}
		{S^b} && {S^b\otimes S^{-a}\otimes S^a} \\
		&& {E\otimes E\otimes S^a \otimes S^{-a}\otimes S^a} && {E\otimes E\otimes S^a\otimes S^{-a}\otimes E\otimes S^a} \\
		{E\otimes E\otimes S^a} && {E\otimes E\otimes S^a} \\
		{E\otimes E\otimes E\otimes S^a} && {E\otimes E\otimes E\otimes S^a}
		\arrow["\cong", from=1-1, to=1-3]
		\arrow["{x\otimes S^{-a}\otimes e\otimes S^a}", from=1-3, to=2-5]
		\arrow["{E\otimes \mu\otimes S^a}", from=4-1, to=3-1]
		\arrow["x"', from=1-1, to=3-1]
		\arrow["{E\otimes E\otimes e\otimes S^a}"{description}, from=3-1, to=4-3]
		\arrow[Rightarrow, no head, from=4-3, to=4-1]
		\arrow["{x\otimes S^{-a}\otimes S^a}"{description}, from=1-3, to=2-3]
		\arrow["{E\otimes E\otimes S^a\otimes \phi_{-a,a}}", from=3-1, to=2-3]
		\arrow["{E\otimes E\otimes S^a\otimes S^{-a}\otimes e\otimes S^a}"', from=2-3, to=2-5]
		\arrow["{E\otimes E\otimes \phi_{a,-a}\otimes S^a}"{description}, from=3-3, to=2-3]
		\arrow["{E\otimes E\otimes e\otimes S^a}"{description}, from=3-3, to=4-3]
		\arrow[Rightarrow, no head, from=3-3, to=3-1]
		\arrow["{E\otimes E\otimes \phi_{a,-a}^{-1}\otimes E\otimes S^a}", from=2-5, to=4-3]
	\end{tikzcd}\]
	The top left trapezoid commutes since the isomorphism $S^b\xr\cong S^b\otimes S^{-a}\otimes S^a$ may be given as $S^b\otimes\phi_{-a,a}$ (see \autoref{unique_comp_Sas}), in which case the trapezoid commmutes by functoriality of $-\otimes-$. The triangle below that commutes by coherence for the $\phi_{a,b}$'s. The triangle below that commutes by definition. The bottom left triangle commutes by unitality for $\mu$. The top right triangle commutes by functoriality of $-\otimes-$. Finally, the bottom right triangle commutes by functoriality of $-\otimes-$. It follows by unravelling definitions that the two outside compositions are $x$ and $\Phi(\Psi(x))$, so indeed we have $\Phi(\Psi(x))=x$ since the diagram commutes.

	On the other hand, suppose we are given a homogeneous pure tensor $x\otimes y$ in $E_*(E)\otimes_{\pi_*(E)}E_*(S^a)$, so $x:S^b\to E\otimes E$ and $y:S^c\to E\otimes S^a$ for some $b,c\in A$. Then we would like to show that $\Psi(\Phi(x\otimes y))=x\otimes y$. Unravelling definitions, $\Psi(\Phi(x\otimes y))$ is the homogeneous pure tensor $\wt{x y}\otimes\wt e$, where $\wt e:S^{a}\to E\otimes S^a$ is defined above, and by functoriality of $-\otimes-$, $\wt{xy}:S^{b+c-a}\to E\otimes E$ is the composition
	% https://q.uiver.app/#q=WzAsNyxbMCwwLCJTXntiK2MtYX0iXSxbMCwxLCJTXntiK2N9XFxvdGltZXMgU157LWF9Il0sWzAsMiwiU157Yn1cXG90aW1lcyBTXmNcXG90aW1lcyBTXnstYX0iXSxbMCwzLCJFXFxvdGltZXMgRVxcb3RpbWVzIEVcXG90aW1lcyBTXmFcXG90aW1lcyBTXnstYX0iXSxbMCw0LCJFXFxvdGltZXMgRVxcb3RpbWVzIFNeYVxcb3RpbWVzIFNeey1hfSJdLFswLDUsIkVcXG90aW1lcyBFXFxvdGltZXMgUyJdLFswLDYsIkVcXG90aW1lcyBFIl0sWzAsMSwiXFxwaGlfe2IrYywtYX0iXSxbMSwyLCJcXHBoaV97YixjfVxcb3RpbWVzICBTXnstYX0iXSxbMiwzLCJ4XFxvdGltZXMgeVxcb3RpbWVzIFNeey1hfSJdLFszLDQsIkVcXG90aW1lcyBcXG11XFxvdGltZXMgU15hXFxvdGltZXMgU157LWF9Il0sWzQsNSwiRVxcb3RpbWVzIEVcXG90aW1lcyBcXHBoaV97YSwtYX1eey0xfSJdLFs1LDYsIkVcXG90aW1lcyBcXHJob19FIl1d
	\[\begin{tikzcd}
		{S^{b+c-a}} \\
		{S^{b+c}\otimes S^{-a}} \\
		{S^{b}\otimes S^c\otimes S^{-a}} \\
		{E\otimes E\otimes E\otimes S^a\otimes S^{-a}} \\
		{E\otimes E\otimes S^a\otimes S^{-a}} \\
		{E\otimes E\otimes S} \\
		{E\otimes E.}
		\arrow["{\phi_{b+c,-a}}", from=1-1, to=2-1]
		\arrow["{\phi_{b,c}\otimes  S^{-a}}", from=2-1, to=3-1]
		\arrow["{x\otimes y\otimes S^{-a}}", from=3-1, to=4-1]
		\arrow["{E\otimes \mu\otimes S^a\otimes S^{-a}}", from=4-1, to=5-1]
		\arrow["{E\otimes E\otimes \phi_{a,-a}^{-1}}", from=5-1, to=6-1]
		\arrow["{E\otimes \rho_E}", from=6-1, to=7-1]
	\end{tikzcd}\]
	In order to see $x\otimes y=\wt{xy}\otimes\wt e$, it suffices to show there exists some scalar $r\in\pi_{c-a}(E)$ such that $x\cdot r=\wt{xy}$ and $r\cdot\wt e=y$, where here $\cdot$ denotes the right and left action of $\pi_*(E)$ on $E_*(E)$ and $E_*(S^a)$, respectively. Now, define $r$ to be the composition
	\[S^{c-a}\cong S^c\otimes S^{-a}\xr{y\otimes S^{-a}}E\otimes S^a\otimes S^{-a}\xr{E\otimes\phi_{a,-a}^{-1}}E\otimes S\xr{\rho_E}E.\]
	First, in order to see that $x\cdot r=\wt{xy}$, consider the following diagram, where here we are again passing to a symmetric strict monoidal category:
	% https://q.uiver.app/#q=WzAsNixbMCwwLCJTXntiK2MtYX0iXSxbMSwwLCJTXntifVxcb3RpbWVzIFNeY1xcb3RpbWVzIFNeey1hfSJdLFsyLDAsIkVcXG90aW1lcyBFXFxvdGltZXMgRVxcb3RpbWVzIFNeYVxcb3RpbWVzIFNeey1hfSJdLFszLDAsIkVcXG90aW1lcyBFXFxvdGltZXMgU15hXFxvdGltZXMgU157LWF9Il0sWzMsMSwiRVxcb3RpbWVzIEUiXSxbMiwxLCJFXFxvdGltZXMgRVxcb3RpbWVzIEUiXSxbMCwxLCJcXGNvbmciXSxbMSwyLCJ4XFxvdGltZXMgeVxcb3RpbWVzIFNeey1hfSJdLFsyLDMsIkVcXG90aW1lcyBcXG11XFxvdGltZXMgU15hXFxvdGltZXMgU157LWF9Il0sWzMsNCwiRVxcb3RpbWVzIEVcXG90aW1lcyBcXHBoaV97YSwtYX1eey0xfSJdLFsyLDUsIkVcXG90aW1lcyBFXFxvdGltZXMgRVxcb3RpbWVzIFxccGhpX3thLC1hfV57LTF9IiwyXSxbNSw0LCJFXFxvdGltZXMgXFxtdSIsMl0sWzIsNCwiRVxcb3RpbWVzXFxtdVxcb3RpbWVzXFxwaGlfe2EsLWF9XnstMX0iLDFdXQ==
	\[\begin{tikzcd}
		{S^{b+c-a}} & {S^{b}\otimes S^c\otimes S^{-a}} & {E\otimes E\otimes E\otimes S^a\otimes S^{-a}} & {E\otimes E\otimes S^a\otimes S^{-a}} \\
		&& {E\otimes E\otimes E} & {E\otimes E}
		\arrow["\cong", from=1-1, to=1-2]
		\arrow["{x\otimes y\otimes S^{-a}}", from=1-2, to=1-3]
		\arrow["{E\otimes \mu\otimes S^a\otimes S^{-a}}", from=1-3, to=1-4]
		\arrow["{E\otimes E\otimes \phi_{a,-a}^{-1}}", from=1-4, to=2-4]
		\arrow["{E\otimes E\otimes E\otimes \phi_{a,-a}^{-1}}"', from=1-3, to=2-3]
		\arrow["{E\otimes \mu}"', from=2-3, to=2-4]
		\arrow["{E\otimes\mu\otimes\phi_{a,-a}^{-1}}"{description}, from=1-3, to=2-4]
	\end{tikzcd}\]
	Commutativity is functoriality of $-\otimes-$, which also tells us that the two outside compositions are $\wt{xy}$ (on top) and $x\cdot r$ (on the bottom), so they are equal as desired. On the other hand, in order to see that $r\cdot\wt e=y$, consider the following diagram (where here we have passed to a symmetric strict monoidal category):
	% https://q.uiver.app/#q=WzAsNyxbMCwwLCJTXmMiXSxbMiwwLCJTXmNcXG90aW1lcyBTXnstYX1cXG90aW1lcyBTXmEiXSxbMiwyLCJFXFxvdGltZXMgU15hXFxvdGltZXMgU157LWF9XFxvdGltZXMgRVxcb3RpbWVzIFNeYSJdLFswLDMsIkVcXG90aW1lcyBFXFxvdGltZXMgU15hIl0sWzAsMiwiRVxcb3RpbWVzIFNeYSJdLFsxLDEsIkVcXG90aW1lcyBTXmFcXG90aW1lcyBTXnstYX1cXG90aW1lcyBTXmEiXSxbMiwzLCJFXFxvdGltZXMgRVxcb3RpbWVzIFNeYSJdLFswLDEsIlxcY29uZyJdLFsxLDIsInlcXG90aW1lcyBTXnstYX1cXG90aW1lcyBlXFxvdGltZXMgU15hIl0sWzMsNCwiXFxtdVxcb3RpbWVzIFNeYSJdLFswLDQsInkiLDJdLFs1LDIsIkVcXG90aW1lcyBTXnthfVxcb3RpbWVzIFNeey1hfVxcb3RpbWVzIGVcXG90aW1lcyBTXmEiLDFdLFs1LDQsIkVcXG90aW1lcyBTXmFcXG90aW1lcyBcXHBoaV97LWEsYX1eey0xfSIsMV0sWzQsNiwiRVxcb3RpbWVzIGVcXG90aW1lcyBTXmEiLDFdLFs2LDMsIiIsMCx7ImxldmVsIjoyLCJzdHlsZSI6eyJoZWFkIjp7Im5hbWUiOiJub25lIn19fV0sWzIsNiwiRVxcb3RpbWVzIFxccGhpX3thLC1hfV57LTF9XFxvdGltZXMgRVxcb3RpbWVzIFNee2F9Il0sWzEsNSwieVxcb3RpbWVzIFNeey1hfVxcb3RpbWVzIFNeYSIsMV1d
	\[\begin{tikzcd}
		{S^c} && {S^c\otimes S^{-a}\otimes S^a} \\
		& {E\otimes S^a\otimes S^{-a}\otimes S^a} \\
		{E\otimes S^a} && {E\otimes S^a\otimes S^{-a}\otimes E\otimes S^a} \\
		{E\otimes E\otimes S^a} && {E\otimes E\otimes S^a}
		\arrow["\cong", from=1-1, to=1-3]
		\arrow["{y\otimes S^{-a}\otimes e\otimes S^a}", from=1-3, to=3-3]
		\arrow["{\mu\otimes S^a}", from=4-1, to=3-1]
		\arrow["y"', from=1-1, to=3-1]
		\arrow["{E\otimes S^{a}\otimes S^{-a}\otimes e\otimes S^a}"{description}, from=2-2, to=3-3]
		\arrow["{E\otimes S^a\otimes \phi_{-a,a}^{-1}}"{description}, from=2-2, to=3-1]
		\arrow["{E\otimes e\otimes S^a}"{description}, from=3-1, to=4-3]
		\arrow[Rightarrow, no head, from=4-3, to=4-1]
		\arrow["{E\otimes \phi_{a,-a}^{-1}\otimes E\otimes S^{a}}", from=3-3, to=4-3]
		\arrow["{y\otimes S^{-a}\otimes S^a}"{description}, from=1-3, to=2-2]
	\end{tikzcd}\]
	The top left triangle commutes since we may take the isomorphism $S^c\xr{\cong}S^c\otimes S^{-a}\otimes S^a$ to be $S^c\otimes\phi_{-a,a}$, in which case commutativity of the triangle follows by functoriality of $-\otimes-$. Commutativity of the right triangle is also functoriality of $-\otimes-$. Commutativity of the bottom left triangle is unitality of $\mu$. Finally, commutativity of the remaining middle $4$-sided region is again functoriality of $-\otimes-$. It follows that $y$ is equal to the outer composition, which is $r\cdot\wt e$, as desired. Thus, we have shown that
	\[\Psi(\Phi(x\otimes y))=\wt{xy}\otimes\wt e=(x\cdot r)\otimes\wt e=x\otimes(r\cdot\wt e)=x\otimes y,\] 
	as desired, so that for each $a\in A$, the object $S^a$ belongs to the class $\cE$. 
	
	Now, we would like to show that given a distinguished triangle
	\[X\xr fY\xr gZ\xr h\Sigma X,\]
	if two of three of the objects $X$, $Y$, and $Z$ belong to $\cE$, then so does the third. From now on, write $L^E_*:\cSH\to\pi_*(E)\text-\Mod$ to denote the functor $X\mapsto E_*(E)\otimes_{\pi_*(E)}E_*(X)$, so $\Phi$ is a natural transformation $L_*^E\Rightarrow E_*(E\otimes-)$. First, note that by \autoref{LES_remains_exact_after_tensor}, we have the following exact sequence in $\cSH$: 
	\[E\otimes \Omega Y\xr{E\otimes \Omega g}E\otimes \Omega Z\xr{E\otimes \wt h}E\otimes X\xr{E\otimes f}E\otimes Y\xr{E\otimes g}E\otimes Z\xr{E\otimes h}E\otimes \Sigma X\xr{E\otimes \Sigma f}\Sigma Y.\]
	We can then apply $[S^*,-]$ to it, which yields the following exact sequence of $A$-graded $\pi_*(E)$-modules:
	\[E_*(\Omega Y)\xr{E_*(\Omega g)}E_*(\Omega Z)\xr{E_*(\wt h)}E_*(X)\xr{E_*(f)}E_*(Y)\xr{E_*(g)}E_*(Z)\xr{E_*(h)}E_*(\Sigma X)\xr{E_*(f)}E_*(\Sigma Y).\]
	Now, we can tensor this sequence with $E_*(E)$ over $\pi_*(E)$, and since $E_*(E)$ is a flat right $\pi_*(E)$ module, we get that the top row in the following sequence is exact:
	% https://q.uiver.app/#q=WzAsMTQsWzAsMCwiTF8qXkUoXFxPbWVnYSBZKSJdLFsxLDAsIkxfKl5FKFxcT21lZ2EgWikiXSxbMiwwLCJMXypeRShYKSJdLFszLDAsIkxfKl5FKFkpIl0sWzQsMCwiTF8qXkUoWikiXSxbNSwwLCJMXypeRShcXFNpZ21hIFgpIl0sWzYsMCwiTF8qXkUoXFxTaWdtYSBZKSJdLFswLDEsIkVfKihFXFxvdGltZXNcXE9tZWdhIFkpIl0sWzEsMSwiRV8qKEVcXG90aW1lc1xcT21lZ2EgWikiXSxbMiwxLCJFXyooRVxcb3RpbWVzIFgpIl0sWzMsMSwiRV8qKEVcXG90aW1lcyBZKSJdLFs0LDEsIkVfKihFXFxvdGltZXMgWikiXSxbNSwxLCJFXyooRVxcb3RpbWVzIFxcU2lnbWEgWCkiXSxbNiwxLCJFXyooRVxcb3RpbWVzIFxcU2lnbWEgWSkiXSxbMCwxLCJMXypeRShcXE9tZWdhIGcpIl0sWzEsMiwiTF8qXkUoXFx3dCBoKSJdLFsyLDMsIkxfKl5FKGYpIl0sWzMsNCwiTF8qXkUoZykiXSxbNCw1LCJMXypeRShoKSJdLFs1LDYsIkxfKl5FKFxcU2lnbWEgZikiXSxbMCw3LCJcXFBoaV97XFxPbWVnYSBZfSIsMl0sWzcsOCwiRV8qKEVcXG90aW1lcyBcXE9tZWdhIGcpIiwyXSxbMTEsMTIsIkVfKihFXFxvdGltZXMgaCkiLDJdLFsxMiwxMywiRV8qKEVcXG90aW1lc1xcU2lnbWEgZikiLDJdLFsxLDgsIlxcUGhpX3tcXE9tZWdhIFp9IiwyXSxbOSwxMCwiRV8qKEVcXG90aW1lcyBmKSIsMl0sWzIsOSwiXFxQaGlfe1h9IiwyXSxbOCw5LCJFXyooRVxcb3RpbWVzXFx3dCBoKSIsMl0sWzMsMTAsIlxcUGhpX1kiLDJdLFsxMCwxMSwiRV8qKEVcXG90aW1lcyBnKSIsMl0sWzQsMTEsIlxcUGhpX1oiLDJdLFs1LDEyLCJcXFBoaV97XFxTaWdtYSBYfSIsMl0sWzYsMTMsIlxcUGhpX3tcXFNpZ21hIFl9IiwyXV0=
	\[\begin{tikzcd}[column sep=tiny]
		{L_*^E(\Omega Y)} & {L_*^E(\Omega Z)} & {L_*^E(X)} & {L_*^E(Y)} & {L_*^E(Z)} & {L_*^E(\Sigma X)} & {L_*^E(\Sigma Y)} \\
		{E_*(E\otimes\Omega Y)} & {E_*(E\otimes\Omega Z)} & {E_*(E\otimes X)} & {E_*(E\otimes Y)} & {E_*(E\otimes Z)} & {E_*(E\otimes \Sigma X)} & {E_*(E\otimes \Sigma Y)}
		\arrow["{L_*^E(\Omega g)}", from=1-1, to=1-2]
		\arrow["{L_*^E(\wt h)}", from=1-2, to=1-3]
		\arrow["{L_*^E(f)}", from=1-3, to=1-4]
		\arrow["{L_*^E(g)}", from=1-4, to=1-5]
		\arrow["{L_*^E(h)}", from=1-5, to=1-6]
		\arrow["{L_*^E(\Sigma f)}", from=1-6, to=1-7]
		\arrow["{\Phi_{\Omega Y}}"', from=1-1, to=2-1]
		\arrow["{E_*(E\otimes \Omega g)}"', from=2-1, to=2-2]
		\arrow["{E_*(E\otimes h)}"', from=2-5, to=2-6]
		\arrow["{E_*(E\otimes\Sigma f)}"', from=2-6, to=2-7]
		\arrow["{\Phi_{\Omega Z}}"', from=1-2, to=2-2]
		\arrow["{E_*(E\otimes f)}"', from=2-3, to=2-4]
		\arrow["{\Phi_{X}}"', from=1-3, to=2-3]
		\arrow["{E_*(E\otimes\wt h)}"', from=2-2, to=2-3]
		\arrow["{\Phi_Y}"', from=1-4, to=2-4]
		\arrow["{E_*(E\otimes g)}"', from=2-4, to=2-5]
		\arrow["{\Phi_Z}"', from=1-5, to=2-5]
		\arrow["{\Phi_{\Sigma X}}"', from=1-6, to=2-6]
		\arrow["{\Phi_{\Sigma Y}}"', from=1-7, to=2-7]
	\end{tikzcd}\]
	The diagram commutes since $\Phi$ is natural. The following sequence is exact in $\cSH$ by \autoref{LES_remains_exact_after_tensor},
	\[E\otimes E\otimes \Omega Y\to E\otimes E\otimes \Omega Z\to E\otimes E\otimes X\to E\otimes E\otimes Y\to E\otimes E\otimes Z\to E\otimes E\otimes \Sigma X\to E\otimes E\otimes \Sigma Y,\]
	so that the bottom row in the above diagram is also exact. Now, suppose two of three of $X$, $Y$, and $Z$ belong to $\cE$. By \autoref{t's_commute_with_Phi's}, \autoref{t_commutes_with_Phi's_corollary}, if $\Phi_W$ is an isomorphism then $\Phi_{\Omega W}$ and $\Phi_{\Sigma W}$ are. Thus by the five lemma, it follows that the middle three vertical arrows in the above diagram are necessarily all isomorphisms, so we have shown that $\cE$ is closed under two-of-three for exact triangles, as desired.

	Finally, it remains to show that $\cE$ is closed under arbitrary coproducts. Let $\{X_i\}_{i\in I}$ be a collection of objects in $\cE$ indexed by some (small) set $I$. Then we'd like to show that $X:=\bigoplus_iX_i$ belongs to $\cE$. First of all, note that $E\otimes-$ preserves arbitrary coproducts, as it has a right adjoint $F(E,-)$. Thus without loss of generality we may take $\bigoplus_iE\otimes X_i=E\otimes\bigoplus_iX_i$ (as $E\otimes\bigoplus_iX_i$ \emph{is} a coproduct of all the $E\otimes X_i$'s). Now, recall that we have chosen each $S^a$ to be a compact object (\autoref{defn_compact}), so that the canonical map
	\[s:\bigoplus_i E_*(X_i)=\bigoplus_i[S^*,E\otimes X_i]\to[S^*,\bigoplus_iE\otimes X_i]=[S^*,E\otimes X]=E_*(X)\]
	is an isomorphism, natural in $X_i$ for each $i$. Note in particular that it is an isomorphism of left $\pi_*(E)$-modules. To see this, first note it suffices to check that $s(r\cdot x)=r\cdot s(x)$ for some homogeneous $x\in E_*(X_i)$ for some $i$, as such $x$ generate $\bigoplus_i E_*(X_i)$ by definition, and $s$ is a homomorphism of abelian groups. Then given $r:S^a\to E\otimes E$ and $x:S^b\to E\otimes X_i$, consider the following diagram	
	% https://q.uiver.app/#q=WzAsOCxbMCwwLCJTXnthK2J9Il0sWzEsMCwiU15hXFxvdGltZXMgU15iIl0sWzIsMCwiRVxcb3RpbWVzIEVcXG90aW1lcyBFXFxvdGltZXMgWF9pIl0sWzMsMCwiRVxcb3RpbWVzIEVcXG90aW1lcyBcXGJpZ29wbHVzX2koRVxcb3RpbWVzIFhfaSkiXSxbMywxLCJFXFxvdGltZXMgRVxcb3RpbWVzIEVcXG90aW1lcyBYIl0sWzMsMiwiRVxcb3RpbWVzIEVcXG90aW1lcyBYIl0sWzIsMywiRVxcb3RpbWVzIEVcXG90aW1lcyBYX2kiXSxbMywzLCJFXFxvdGltZXMgXFxiaWdvcGx1c19pKEVcXG90aW1lcyBYX2kpIl0sWzAsMSwiXFxwaGlfe2EsYn0iXSxbMSwyLCJ4XFxvdGltZXMgeSJdLFsyLDMsIkVcXG90aW1lcyBFXFxvdGltZXMgXFxpb3RhX3tFXFxvdGltZXMgWF9pfSJdLFszLDQsIiIsMCx7ImxldmVsIjoyLCJzdHlsZSI6eyJoZWFkIjp7Im5hbWUiOiJub25lIn19fV0sWzQsNSwiRVxcb3RpbWVzIFxcbXVcXG90aW1lcyBYIl0sWzIsNiwiRVxcb3RpbWVzIFxcbXVcXG90aW1lcyBYX2kiLDJdLFs2LDcsIkVcXG90aW1lcyBcXGlvdGFfe0VcXG90aW1lcyBYX2l9IiwyXSxbNyw1LCIiLDEseyJsZXZlbCI6Miwic3R5bGUiOnsiaGVhZCI6eyJuYW1lIjoibm9uZSJ9fX1dLFs2LDUsIkVcXG90aW1lcyBFXFxvdGltZXMgXFxpb3RhX3tYX2l9IiwxXSxbMiw0LCJFXFxvdGltZXMgRVxcb3RpbWVzIEVcXG90aW1lcyBcXGlvdGFfe1hfaX0iLDFdXQ==
	\[\begin{tikzcd}
		{S^{a+b}} & {S^a\otimes S^b} & {E\otimes E\otimes E\otimes X_i} & {E\otimes E\otimes \bigoplus_i(E\otimes X_i)} \\
		&&& {E\otimes E\otimes E\otimes X} \\
		&&& {E\otimes E\otimes X} \\
		&& {E\otimes E\otimes X_i} & {E\otimes \bigoplus_i(E\otimes X_i)}
		\arrow["{\phi_{a,b}}", from=1-1, to=1-2]
		\arrow["{x\otimes y}", from=1-2, to=1-3]
		\arrow["{E\otimes E\otimes \iota_{E\otimes X_i}}", from=1-3, to=1-4]
		\arrow[Rightarrow, no head, from=1-4, to=2-4]
		\arrow["{E\otimes \mu\otimes X}", from=2-4, to=3-4]
		\arrow["{E\otimes \mu\otimes X_i}"', from=1-3, to=4-3]
		\arrow["{E\otimes \iota_{E\otimes X_i}}"', from=4-3, to=4-4]
		\arrow[Rightarrow, no head, from=4-4, to=3-4]
		\arrow["{E\otimes E\otimes \iota_{X_i}}"{description}, from=4-3, to=3-4]
		\arrow["{E\otimes E\otimes E\otimes \iota_{X_i}}"{description}, from=1-3, to=2-4]
	\end{tikzcd}\]
	where $\iota_{E\otimes X_i}:E\otimes X_i\into\bigoplus_i(E\otimes X_i)$ and $\iota_{X_i}:X_i\into\bigoplus_iX_i$ are the maps determined by universal property of the coproduct. Commutativity of the two triangles is again by the fact that $E\otimes-$ is colimit preserving. Commutativity of the trapezoid is functoriality of $-\otimes-$. Thus, the top arrow in the following diagram is well-defined:
	% https://q.uiver.app/#q=WzAsNixbMSwwLCJFXyooRSlcXG90aW1lc197XFxwaV8qKEUpfVxcYmlnb3BsdXNfaUVfKihYX2kpIl0sWzAsMiwiXFxiaWdvcGx1c19pRV8qKEVcXG90aW1lcyBYX2kpIl0sWzIsMiwiRV8qKEVcXG90aW1lcyBYKSJdLFsyLDAsIkVfKihFKVxcb3RpbWVzX3tcXHBpXyooRSl9IEVfKihYKSJdLFswLDAsIlxcYmlnb3BsdXNfaUVfKihFKVxcb3RpbWVzX3tcXHBpXyooRSl9RV8qKFhfaSkiXSxbMSwyLCJFXyooXFxiaWdvcGx1c19pRVxcb3RpbWVzIFhfaSkiXSxbMCwzLCJFXyooRSlcXG90aW1lc197XFxwaV8qKEUpfSBzIl0sWzMsMiwiXFxQaGlfWCJdLFs0LDAsIiIsMCx7ImxldmVsIjoyLCJzdHlsZSI6eyJoZWFkIjp7Im5hbWUiOiJub25lIn19fV0sWzEsNSwicyJdLFs1LDIsIiIsMCx7ImxldmVsIjoyLCJzdHlsZSI6eyJoZWFkIjp7Im5hbWUiOiJub25lIn19fV0sWzQsMSwiXFxiaWdvcGx1c19pXFxQaGlfe1hfaX0iLDJdXQ==
	\begin{equation}\label{kunneth_iso_pt_3_diag}\begin{tikzcd}
		{\bigoplus_iE_*(E)\otimes_{\pi_*(E)}E_*(X_i)} & {E_*(E)\otimes_{\pi_*(E)}\bigoplus_iE_*(X_i)} & {E_*(E)\otimes_{\pi_*(E)} E_*(X)} \\
		\\
		{\bigoplus_iE_*(E\otimes X_i)} & {E_*(\bigoplus_iE\otimes X_i)} & {E_*(E\otimes X)}
		\arrow["{E_*(E)\otimes_{\pi_*(E)} s}", from=1-2, to=1-3]
		\arrow["{\Phi_X}", from=1-3, to=3-3]
		\arrow[Rightarrow, no head, from=1-1, to=1-2]
		\arrow["s", from=3-1, to=3-2]
		\arrow[Rightarrow, no head, from=3-2, to=3-3]
		\arrow["{\bigoplus_i\Phi_{X_i}}"', from=1-1, to=3-1]
	\end{tikzcd}\end{equation}
	We wish to show this diagram commutes. Again, since each map here is a homomorphism, it suffices to chase generators. By definition, a generator of the top left element is a homogeneous pure tensor in $E_*(E)\otimes_{\pi_{*}(E)}E_*(X_i)$ for some $i$ in $I$. Given classes $x:S^a\to E\otimes E$ and $y:S^b\to E\otimes X_i$, consider the following diagram:
	% https://q.uiver.app/#q=WzAsOCxbMCwwLCJTXnthK2J9Il0sWzEsMCwiU15hXFxvdGltZXMgU15iIl0sWzIsMCwiRVxcb3RpbWVzIEVcXG90aW1lcyBFXFxvdGltZXMgWF9pIl0sWzMsMCwiRVxcb3RpbWVzIEVcXG90aW1lcyBcXGJpZ29wbHVzX2lFXFxvdGltZXMgWF9pIl0sWzMsMiwiRVxcb3RpbWVzIEVcXG90aW1lcyBYIl0sWzIsMSwiRVxcb3RpbWVzIEVcXG90aW1lcyBYX2kiXSxbMywxLCJFXFxvdGltZXMgRVxcb3RpbWVzIEVcXG90aW1lcyBYIl0sWzIsMiwiXFxiaWdvcGx1c19pRVxcb3RpbWVzIEVcXG90aW1lcyBYX2kiXSxbMCwxLCJcXHBoaV97YSxifSJdLFsxLDIsInhcXG90aW1lcyB5Il0sWzIsMywiRVxcb3RpbWVzIEVcXG90aW1lcyBcXGlvdGFfe0VcXG90aW1lcyBYX2l9Il0sWzIsNSwiRVxcb3RpbWVzIFxcbXVcXG90aW1lcyBYX2kiLDJdLFszLDYsIiIsMCx7ImxldmVsIjoyLCJzdHlsZSI6eyJoZWFkIjp7Im5hbWUiOiJub25lIn19fV0sWzYsNCwiRVxcb3RpbWVzIFxcbXVcXG90aW1lcyBYIl0sWzUsNywiXFxpb3RhX3tFXFxvdGltZXMgRVxcb3RpbWVzIFhfaX0iLDJdLFs3LDQsIiIsMSx7ImxldmVsIjoyLCJzdHlsZSI6eyJoZWFkIjp7Im5hbWUiOiJub25lIn19fV0sWzUsNCwiRVxcb3RpbWVzIEVcXG90aW1lcyBcXGlvdGFfe1hfaX0iLDFdLFsyLDYsIkVcXG90aW1lcyBFXFxvdGltZXMgRVxcb3RpbWVzIFxcaW90YV97WF9pfSIsMV1d
	\[\begin{tikzcd}
		{S^{a+b}} & {S^a\otimes S^b} & {E\otimes E\otimes E\otimes X_i} & {E\otimes E\otimes \bigoplus_iE\otimes X_i} \\
		&& {E\otimes E\otimes X_i} & {E\otimes E\otimes E\otimes X} \\
		&& {\bigoplus_iE\otimes E\otimes X_i} & {E\otimes E\otimes X}
		\arrow["{\phi_{a,b}}", from=1-1, to=1-2]
		\arrow["{x\otimes y}", from=1-2, to=1-3]
		\arrow["{E\otimes E\otimes \iota_{E\otimes X_i}}", from=1-3, to=1-4]
		\arrow["{E\otimes \mu\otimes X_i}"', from=1-3, to=2-3]
		\arrow[Rightarrow, no head, from=1-4, to=2-4]
		\arrow["{E\otimes \mu\otimes X}", from=2-4, to=3-4]
		\arrow["{\iota_{E\otimes E\otimes X_i}}"', from=2-3, to=3-3]
		\arrow[Rightarrow, no head, from=3-3, to=3-4]
		\arrow["{E\otimes E\otimes \iota_{X_i}}"{description}, from=2-3, to=3-4]
		\arrow["{E\otimes E\otimes E\otimes \iota_{X_i}}"{description}, from=1-3, to=2-4]
	\end{tikzcd}\]
	Unravelling definitions, the two outside compositions are the two ways to chase $x\otimes y$ around diagram (\ref{kunneth_iso_pt_3_diag}). The two triangles commute again by the fact that $-\otimes-$ preserves colimits in each argument. Commutativity of the inner parallelogram is functoriality of $-\otimes-$. Thus diagram (\ref{kunneth_iso_pt_3_diag}) tells us $\Phi_X$ is an isomorphism, since $\Phi_{X_i}$ is an isomorphism for each $i$ in $I$, meaning $\bigoplus_i\Phi_{X_i}$ is as well.
\end{proof}

\begin{proposition}
	Let $(E,\mu,e)$ be a ring spectrum in $\cSH$, and let $X$ and $Y$ be two objects in $\cSH$. Consider the map
	\[[X,E\otimes Y]_*\to\Hom_{\pi_*(E)}^*(E_*(X),E_*(Y))\]
	which sends a class $f:S^a\otimes X\to E\otimes Y$ in $[X,E\otimes Y]_*$ to 

	Suppose there exists set $I$, a collection $\{a_i\}_{i\in I}\sseq A$, and maps $r$ and $i$ which fit into a retract diagram:
	% https://q.uiver.app/#q=WzAsMyxbMCwwLCJFXFxvdGltZXMgWCJdLFsxLDAsIlxcYmlnb3BsdXNfe2lcXGluIEl9XFxTaWdtYV57YV9pfUUiXSxbMSwxLCJFXFxvdGltZXMgWCJdLFswLDEsIlxcaW90YSJdLFsxLDIsInIiXSxbMCwyLCIiLDIseyJsZXZlbCI6Miwic3R5bGUiOnsiaGVhZCI6eyJuYW1lIjoibm9uZSJ9fX1dXQ==
	\[\begin{tikzcd}
		{E\otimes X} & {\bigoplus_{i\in I}\Sigma^{a_i}E} \\
		& {E\otimes X}
		\arrow["\iota", from=1-1, to=1-2]
		\arrow["r", from=1-2, to=2-2]
		\arrow[Rightarrow, no head, from=1-1, to=2-2]
	\end{tikzcd}\]
	Then 
\end{proposition}

%In the following definition, let $\vare:E_*(E)\to \pi_*(E)$ be the map which sends some $\alpha:S^a\to E\otimes E$ to the composition
%\[S^a\xr\alpha E\otimes E\xr\mu E.\]
%Also define $\Psi:E_*(E)\to E_*(E)\otimes_{\pi_*(E)}E_*(E)$ to be the map which factors as
%\[E_*(E)\to E_*(E\otimes E)\xr\cong E_*(E)\otimes_{\pi_*(E)}E_*(E)\]
%where the second arrow is the isomorphism prescribed by \autoref{Kunneth_map}, and the first arrow sends a class $\alpha:S^a\to E\otimes E$ to the composition
%\[S^a\xr\alpha E\otimes E\cong E\otimes S\otimes E\xr{E\otimes e\otimes E}E\otimes E\otimes E.\]
%
%\begin{lemma}[{\cite[Proposition 2.30, 2.33]{nlab:introduction_to_the_adams_spectral_sequence}}]\label{2.30_2.33}
	%Let $E$ be a flat commutative ring spectrum, and let $X$ and $Y$ be spectra such that $E_\aast(X)$ is a projective module over $\pi_\aast(E)$. Then for all $s\geq0$ and $t,w\in\bZ$, there is an isomorphism
	%\[\Phi:[X,E\wedge Y]_{t,w}\to\Hom_{E_\aast(E)}^{t,w}(E_\aast(X),E_\aast(E\wedge Y)),\]
	%obtained by sending a class $f:S^{t,w}\wedge X\to E\wedge Y$ in $[X,E\wedge Y]_{t,w}$ to the map
	%\[\Phi_f:E_\acast(X)\to E_{\ast+t,\ast+w}(X\wedge Y)\]
	%sending
	%\[[S^{a,b}\xr gE\wedge X]\mapsto[S^{a+t,b+w}\cong S^{a,b}\wedge S^{t,w}\xr{g\wedge S^{t,w}}E\wedge X\wedge S^{t,w}\cong E\wedge S^{t,w}\wedge X\xr{E\wedge f}E\wedge E\wedge Y].\]
%\end{lemma}
%\begin{proof}
	%Let $f:S^{t,w}\wedge X\to E\wedge Y$. First we want to show that $\Phi_f$ is actually an $E_\aast(E)$-comodule homomorphism.\todo{finish}
%\end{proof}

\end{document}
