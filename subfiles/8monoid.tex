\documentclass[../main.tex]{subfiles}
\tikzcdset{scale cd/.style={every label/.append style={scale=#1},cells={nodes={scale=#1}}}}

\begin{document}

In what follows, we fix a symmetric monoidal category $(\cC,\otimes,S)$ with left unitor, right unitor, associator, and symmetry isomorphisms $\lambda$, $\rho$, $\alpha$, and $\tau$, respectively.

\subsection{Monoid objects in a symmetric monoidal category}\label{monoid_objects_subsection_appendix}

\begin{definition}\label{monoid_object}
    Let $(\cC,\otimes,S)$ be a symmetric monoidal category with left unitor, right unitor, associator, and symmetry isomorphisms $\lambda$, $\rho$, $\alpha$, and $\tau$, respectively. A \emph{monoid object} $(E,\mu,e)$ is an object $E$ in $\cC$ along with a multiplication morphism $\mu:E\otimes E\to E$ and a unit map $e:S\to E$ such that the following diagrams commute:
	% https://q.uiver.app/#q=WzAsOSxbMSwwLCJFXFxvdGltZXMgRSJdLFsxLDEsIkUiXSxbMiwwLCJTXFxvdGltZXMgRSJdLFswLDAsIkVcXG90aW1lcyBTIl0sWzMsMCwiKEVcXG90aW1lcyBFKVxcb3RpbWVzIEUiXSxbMywxLCJFXFxvdGltZXMoRVxcb3RpbWVzIEUpIl0sWzQsMSwiRVxcb3RpbWVzIEUiXSxbNSwxLCJFIl0sWzUsMCwiRVxcb3RpbWVzIEUiXSxbMCwxLCJcXG11Il0sWzIsMCwiZVxcb3RpbWVzIEUiLDJdLFszLDAsIkVcXG90aW1lcyBlIl0sWzIsMSwiXFxsYW1iZGFfRSJdLFszLDEsIlxccmhvX0UiLDJdLFs0LDUsIlxcYWxwaGEiLDJdLFs1LDYsIkVcXG90aW1lc1xcbXUiXSxbNiw3LCJcXG11Il0sWzQsOCwiXFxtdVxcb3RpbWVzIEUiXSxbOCw3LCJcXG11Il1d
	\[\begin{tikzcd}
		{E\otimes S} & {E\otimes E} & {S\otimes E} & {(E\otimes E)\otimes E} && {E\otimes E} \\
		& E && {E\otimes(E\otimes E)} & {E\otimes E} & E
		\arrow["\mu", from=1-2, to=2-2]
		\arrow["{e\otimes E}"', from=1-3, to=1-2]
		\arrow["{E\otimes e}", from=1-1, to=1-2]
		\arrow["{\lambda_E}", from=1-3, to=2-2]
		\arrow["{\rho_E}"', from=1-1, to=2-2]
		\arrow["\alpha"', from=1-4, to=2-4]
		\arrow["E\otimes\mu", from=2-4, to=2-5]
		\arrow["\mu", from=2-5, to=2-6]
		\arrow["{\mu\otimes E}", from=1-4, to=1-6]
		\arrow["\mu", from=1-6, to=2-6]
	\end{tikzcd}\]
    The first diagram expresses unitality, while the second expressed associativity. If in addition the following diagram commutes, 
    % https://q.uiver.app/#q=WzAsMyxbMCwwLCJFXFxvdGltZXMgRSJdLFsyLDAsIkVcXG90aW1lcyBFIl0sWzEsMSwiRSJdLFswLDEsIlxcdGF1Il0sWzEsMiwiXFxtdSJdLFswLDIsIlxcbXUiLDJdXQ==
    \[\begin{tikzcd}
        {E\otimes E} && {E\otimes E} \\
        & E
        \arrow["\tau", from=1-1, to=1-3]
        \arrow["\mu", from=1-3, to=2-2]
        \arrow["\mu"', from=1-1, to=2-2]
    \end{tikzcd}\]
    then we say $(E,\mu,e)$ is a \emph{commutative} monoid object.
\end{definition}

\begin{definition}\label{Mon_C,CMon_C}
	Given two monoid objects $(E_1,\mu_1,e_1)$ and $(E_2,\mu_2,e_2)$ in a symmetric monoidal category $(\cC,\otimes,S)$, a \emph{monoid homomorphism} from $E_1$ to $E_2$ is a morphism $f:E_1\to E_2$ in $\cC$ such that the following diagrams commute:
	% https://q.uiver.app/#q=WzAsNyxbMCwwLCJFXzFcXG90aW1lcyBFXzEiXSxbMSwwLCJFXzJcXG90aW1lcyBFXzIiXSxbMSwxLCJFXzIiXSxbMCwxLCJFXzEiXSxbMywwLCJTIl0sWzIsMSwiRV8xIl0sWzQsMSwiRV8yIl0sWzAsMSwiZlxcb3RpbWVzIGYiXSxbMSwyLCJcXG11XzIiXSxbMCwzLCJcXG11XzEiLDJdLFszLDIsImYiXSxbNCw1LCJlXzEiLDJdLFs1LDYsImYiXSxbNCw2LCJlXzIiXV0=
	\[\begin{tikzcd}
		{E_1\otimes E_1} & {E_2\otimes E_2} && S \\
		{E_1} & {E_2} & {E_1} && {E_2}
		\arrow["{f\otimes f}", from=1-1, to=1-2]
		\arrow["{\mu_2}", from=1-2, to=2-2]
		\arrow["{\mu_1}"', from=1-1, to=2-1]
		\arrow["f", from=2-1, to=2-2]
		\arrow["{e_1}"', from=1-4, to=2-3]
		\arrow["f", from=2-3, to=2-5]
		\arrow["{e_2}", from=1-4, to=2-5]
	\end{tikzcd}\]
	It is straightforward to show that $\id_{E_1}$ is a homomorphism of monoid objects from $E_1$ to itself, and that the composition of monoid homomorphisms is still a monoid homomorphism. Thus, we have categories $\Mon_\cC$ and $\CMon_\cC$ of monoid objects and commutative monoid objects in $\cC$, respectively, with monoid homomorphisms between them.
\end{definition}

\begin{lemma}\label{product_of_monoids_is_monoid}
	Given two monoid objects $(E_1,\mu_1,e_1)$ and $(E_2,\mu_2,e_2)$ in a symmetric monoidal category $(\cC,\otimes,S)$, their tensor product $E_1\otimes E_2$ canonically becomes a monoid object in $\cC$ with unit map
	\[e:S\xr\cong S\otimes S\xr{e_1\otimes e_2}E_1\otimes E_2\]
	and multiplication map
	\[\mu:E_1\otimes E_2\otimes E_1\otimes E_2\xr{E_1\otimes\tau\otimes E_2}E_1\otimes E_1\otimes E_2\otimes E_2\xr{\mu_1\otimes\mu_2}E_1\otimes E_2\]
	(where here we are suppressing the associators from the notation). If in addition $(E_1,\mu_1,e_1)$ and $(E_2,\mu_2,e_2)$ are \emph{commutative} monoid objects, then $(E_1\otimes E_2,\mu,e)$ is as well.
\end{lemma}

\begin{lemma}\label{product_of_3+_monoids_no_ambiguity}
	Given monoid objects $(E_i,\mu_i,e_i)$ for $i=1,2,3$ in a symmetric monoidal category $\cC$, the associator $(E_1\otimes E_2)\otimes E_3\xr\cong E_1\otimes(E_2\otimes E_3)$ is an isomorphism of monoid objects. In other words, up to associativity, given a collection of monoid objects $E_1,\ldots,E_n$ in $\cC$, there is no ambiguity when talking about their tensor product $E_1\otimes\cdots\otimes E_n$ as a monoid object.
\end{lemma}
\begin{proof}
	Clearly, up to associativity, $(E_1\otimes E_2)\otimes E_3$ and $E_1\otimes (E_2\otimes E_3)$ have the same unit map $S\xr{e_1\otimes e_2\otimes e_3}E_1\otimes E_2\otimes E_3$. Thus, it remains to show that they have the same product map, up to associativity. To see this, consider the following diagram, where we've passed to a symmetric strict monoidal category:
	% https://q.uiver.app/#q=WzAsOSxbMCwwLCJFXzFcXG90aW1lcyhFXzJcXG90aW1lcyBFXzMpXFxvdGltZXMgRV8xXFxvdGltZXMoRV8yXFxvdGltZXMgRV8zKSJdLFswLDEsIkVfMVxcb3RpbWVzIEVfMVxcb3RpbWVzIEVfMlxcb3RpbWVzIEVfM1xcb3RpbWVzIEVfMlxcb3RpbWVzIEVfMyJdLFsyLDEsIkVfMVxcb3RpbWVzIEVfMlxcb3RpbWVzIEVfMVxcb3RpbWVzIEVfMlxcb3RpbWVzIEVfM1xcb3RpbWVzIEVfMyJdLFswLDIsIkVfezF9XFxvdGltZXMgRV8yXFxvdGltZXMgRV8yXFxvdGltZXMgRV8zXFxvdGltZXMgRV8zIl0sWzAsMywiRV97MX1cXG90aW1lcyBFX3syfVxcb3RpbWVzIEVfezN9Il0sWzIsMiwiRV8xXFxvdGltZXMgRV8xXFxvdGltZXMgRV8yXFxvdGltZXMgRV8yXFxvdGltZXMgRV97M30iXSxbMiwzLCJFX3sxfVxcb3RpbWVzIEVfezJ9XFxvdGltZXMgRV97M30iXSxbMSwyLCJFXzFcXG90aW1lcyBFXzFcXG90aW1lcyBFXzJcXG90aW1lcyBFXzJcXG90aW1lcyBFXzNcXG90aW1lcyBFXzMiXSxbMiwwLCIoRV8xXFxvdGltZXMgRV8yKVxcb3RpbWVzIEVfM1xcb3RpbWVzIChFXzFcXG90aW1lcyBFXzIpXFxvdGltZXMgRV8zIl0sWzAsMSwiRV8xXFxvdGltZXMgXFx0YXVfe0VfMlxcb3RpbWVzIEVfMyxFXzF9XFxvdGltZXMgRV8yXFxvdGltZXMgRV8zIiwyXSxbMSwzLCJcXG11XzFcXG90aW1lcyBFXzJcXG90aW1lcyBcXHRhdVxcb3RpbWVzIEVfMyIsMl0sWzMsNCwiRV8xXFxvdGltZXMgXFxtdV8yXFxvdGltZXMgXFxtdV8zIiwyXSxbMiw1LCJFXzFcXG90aW1lcyBcXHRhdVxcb3RpbWVzIEVfMlxcb3RpbWVzIFxcbXVfMyJdLFs1LDYsIlxcbXVfMVxcb3RpbWVzIFxcbXVfMlxcb3RpbWVzIEVfMyJdLFs0LDYsIlxcYWxwaGEiLDAseyJsZXZlbCI6Miwic3R5bGUiOnsiaGVhZCI6eyJuYW1lIjoibm9uZSJ9fX1dLFsxLDcsIkVfMVxcb3RpbWVzIEVfMVxcb3RpbWVzIEVfMlxcb3RpbWVzIFxcdGF1XFxvdGltZXMgRV8zIiwxXSxbMiw3LCJFXzFcXG90aW1lcyBcXHRhdVxcb3RpbWVzIEVfMlxcb3RpbWVzIEVfM1xcb3RpbWVzIEVfMyIsMV0sWzcsNSwiRV8xXFxvdGltZXMgRV8yXFxvdGltZXMgRV8yXFxvdGltZXMgRV8yXFxvdGltZXMgXFxtdV8zIiwyXSxbNywzLCJcXG11XzFcXG90aW1lcyBFXzJcXG90aW1lcyBFXzJcXG90aW1lcyBFXzNcXG90aW1lcyBFXzMiXSxbMCw4LCJcXGFscGhhIiwwLHsibGV2ZWwiOjIsInN0eWxlIjp7ImhlYWQiOnsibmFtZSI6Im5vbmUifX19XSxbOCwyLCJFXzFcXG90aW1lcyBFXzJcXG90aW1lcyBcXHRhdV97RV8zLEVfMVxcb3RpbWVzIEVfMn1cXG90aW1lcyBFXzMiXSxbNyw0LCJcXG11XzFcXG90aW1lcyBcXG11XzJcXG90aW1lcyBcXG11XzMiLDFdLFs3LDYsIlxcbXVfMVxcb3RpbWVzIFxcbXVfMlxcb3RpbWVzIFxcbXVfMyIsMV1d
	\[\begin{tikzcd}[column sep=tiny]
		{E_1\otimes(E_2\otimes E_3)\otimes E_1\otimes(E_2\otimes E_3)} && {(E_1\otimes E_2)\otimes E_3\otimes (E_1\otimes E_2)\otimes E_3} \\
		{E_1\otimes E_1\otimes E_2\otimes E_3\otimes E_2\otimes E_3} && {E_1\otimes E_2\otimes E_1\otimes E_2\otimes E_3\otimes E_3} \\
		{E_{1}\otimes E_2\otimes E_2\otimes E_3\otimes E_3} & {E_1\otimes E_1\otimes E_2\otimes E_2\otimes E_3\otimes E_3} & {E_1\otimes E_1\otimes E_2\otimes E_2\otimes E_{3}} \\
		{E_{1}\otimes E_{2}\otimes E_{3}} && {E_{1}\otimes E_{2}\otimes E_{3}}
		\arrow["{E_1\otimes \tau_{E_2\otimes E_3,E_1}\otimes E_2\otimes E_3}"', from=1-1, to=2-1]
		\arrow["{\mu_1\otimes E_2\otimes \tau\otimes E_3}"', from=2-1, to=3-1]
		\arrow["{E_1\otimes \mu_2\otimes \mu_3}"', from=3-1, to=4-1]
		\arrow["{E_1\otimes \tau\otimes E_2\otimes \mu_3}", from=2-3, to=3-3]
		\arrow["{\mu_1\otimes \mu_2\otimes E_3}", from=3-3, to=4-3]
		\arrow["\alpha", Rightarrow, no head, from=4-1, to=4-3]
		\arrow["{E_1\otimes E_1\otimes E_2\otimes \tau\otimes E_3}"{description}, from=2-1, to=3-2]
		\arrow["{E_1\otimes \tau\otimes E_2\otimes E_3\otimes E_3}"{description}, from=2-3, to=3-2]
		\arrow["{E_1\otimes E_2\otimes E_2\otimes E_2\otimes \mu_3}"', from=3-2, to=3-3]
		\arrow["{\mu_1\otimes E_2\otimes E_2\otimes E_3\otimes E_3}", from=3-2, to=3-1]
		\arrow["\alpha", Rightarrow, no head, from=1-1, to=1-3]
		\arrow["{E_1\otimes E_2\otimes \tau_{E_3,E_1\otimes E_2}\otimes E_3}", from=1-3, to=2-3]
		\arrow["{\mu_1\otimes \mu_2\otimes \mu_3}"{description}, from=3-2, to=4-1]
		\arrow["{\mu_1\otimes \mu_2\otimes \mu_3}"{description}, from=3-2, to=4-3]
	\end{tikzcd}\]
	The top pentagonal region commutes by coherence for the $\tau$'s in a symmetric monoidal category. The bottom triangle commutes by definition. The remaining four triangles commute by functoriality of $-\otimes-$. On the left is the product for $E_1\otimes(E_2\otimes E_3)$, while on the right is the product for $(E_1\otimes E_2)\otimes E_3$. Thus they are equal up to associativity, as desired.
\end{proof}

\begin{lemma}\label{E_ox_f,f_ox_E_are_monoid_homos}
	Suppose we have some monoid object $(E,\mu,e)$ in $\cC$ and some homomorphism of monoid objects $f:(E_1,\mu_1,e_1)\to (E_2,\mu_2,e_2)$ in $\Mon_\cC$. Then $E\otimes f:E\otimes E_1\to E\otimes E_2$ and $f\otimes E:E_1\otimes E\to E_2\otimes E$ are monoid homomorphisms, where here we are considering $E\otimes E_1$, $E\otimes E_2$, $E_1\otimes E$, and $E_2\otimes E$ to be monoid objects by \autoref{product_of_monoids_is_monoid}.
\end{lemma}
\begin{proof}
	We will show that $E\otimes f$ is a monoid object homomorphism, as showing $f\otimes E$ is a monoid homomorphism is entirely analagous. First consider the following diagram:
	% https://q.uiver.app/#q=WzAsOCxbMCwwLCJFXFxvdGltZXMgRV8xXFxvdGltZXMgRVxcb3RpbWVzIEVfMSJdLFszLDAsIkVcXG90aW1lcyBFXzJcXG90aW1lcyBFXFxvdGltZXMgRV8yIl0sWzAsMSwiRVxcb3RpbWVzIEVcXG90aW1lcyBFXzFcXG90aW1lcyBFXzEiXSxbMywxLCJFXFxvdGltZXMgRVxcb3RpbWVzIEVfMlxcb3RpbWVzIEVfMiJdLFswLDMsIkVcXG90aW1lcyBFXzEiXSxbMywzLCJFXFxvdGltZXMgRV8yIl0sWzEsMiwiRVxcb3RpbWVzIEVfMVxcb3RpbWVzIEVfMSJdLFsyLDIsIkVcXG90aW1lcyBFXzJcXG90aW1lcyBFXzIiXSxbMCwxLCJFXFxvdGltZXMgZlxcb3RpbWVzIEVcXG90aW1lcyBmIl0sWzAsMiwiRVxcb3RpbWVzIFxcdGF1XFxvdGltZXMgRV8xIiwyXSxbMSwzLCJFXFxvdGltZXMgXFx0YXVcXG90aW1lcyBFXzIiXSxbMiw0LCJcXG11XFxvdGltZXMgXFxtdV8xIiwyXSxbNCw1LCJFXFxvdGltZXMgZiJdLFszLDUsIlxcbXVcXG90aW1lcyBcXG11XzIiXSxbMiwzLCJFXFxvdGltZXMgRVxcb3RpbWVzIGZcXG90aW1lcyBmIl0sWzIsNiwiXFxtdVxcb3RpbWVzIEVfMVxcb3RpbWVzIEVfMiIsMV0sWzYsNCwiRVxcb3RpbWVzIFxcbXVfMSIsMV0sWzYsNywiRVxcb3RpbWVzIGZcXG90aW1lcyBmIl0sWzcsNSwiRVxcb3RpbWVzXFxtdV8yIiwxXSxbMyw3LCJcXG11XFxvdGltZXMgRV8yXFxvdGltZXMgRV8yIiwxXV0=
	\[\begin{tikzcd}
		{E\otimes E_1\otimes E\otimes E_1} &&& {E\otimes E_2\otimes E\otimes E_2} \\
		{E\otimes E\otimes E_1\otimes E_1} &&& {E\otimes E\otimes E_2\otimes E_2} \\
		& {E\otimes E_1\otimes E_1} & {E\otimes E_2\otimes E_2} \\
		{E\otimes E_1} &&& {E\otimes E_2}
		\arrow["{E\otimes f\otimes E\otimes f}", from=1-1, to=1-4]
		\arrow["{E\otimes \tau\otimes E_1}"', from=1-1, to=2-1]
		\arrow["{E\otimes \tau\otimes E_2}", from=1-4, to=2-4]
		\arrow["{\mu\otimes \mu_1}"', from=2-1, to=4-1]
		\arrow["{E\otimes f}", from=4-1, to=4-4]
		\arrow["{\mu\otimes \mu_2}", from=2-4, to=4-4]
		\arrow["{E\otimes E\otimes f\otimes f}", from=2-1, to=2-4]
		\arrow["{\mu\otimes E_1\otimes E_2}"{description}, from=2-1, to=3-2]
		\arrow["{E\otimes \mu_1}"{description}, from=3-2, to=4-1]
		\arrow["{E\otimes f\otimes f}", from=3-2, to=3-3]
		\arrow["{E\otimes\mu_2}"{description}, from=3-3, to=4-4]
		\arrow["{\mu\otimes E_2\otimes E_2}"{description}, from=2-4, to=3-3]
	\end{tikzcd}\]
	The top region commutes by naturality of $\tau$. The bottom trapezoid commutes since $f$ is a monoid homomorphism. The remaining three regions commute by functoriality of $-\otimes-$. Now, consider the following diagram:
	% https://q.uiver.app/#q=WzAsNCxbMiwwLCJTIl0sWzAsMywiRVxcb3RpbWVzIEVfMSJdLFs0LDMsIkVcXG90aW1lcyBFXzIiXSxbMiwyLCJFIl0sWzAsMSwiZVxcb3RpbWVzIGVfMSIsMl0sWzEsMiwiRVxcb3RpbWVzIGYiXSxbMCwyLCJlXFxvdGltZXMgZV8yIl0sWzAsMywiZSIsMV0sWzMsMSwiRVxcb3RpbWVzIGVfMSIsMV0sWzMsMiwiRVxcb3RpbWVzIGVfMiIsMV1d
	\[\begin{tikzcd}
		&& S \\
		\\
		&& E \\
		{E\otimes E_1} &&&& {E\otimes E_2}
		\arrow["{e\otimes e_1}"', from=1-3, to=4-1]
		\arrow["{E\otimes f}", from=4-1, to=4-5]
		\arrow["{e\otimes e_2}", from=1-3, to=4-5]
		\arrow["e"{description}, from=1-3, to=3-3]
		\arrow["{E\otimes e_1}"{description}, from=3-3, to=4-1]
		\arrow["{E\otimes e_2}"{description}, from=3-3, to=4-5]
	\end{tikzcd}\]
	The bottom region commutes since $f$ is a monoid homomorphism. The top two regions commute by functoriality of $-\otimes-$. Thus, we've shown $E\otimes f$ is a monoid object homomorphism, as desired.
\end{proof}

\subsection{Modules over monoid objects in a symmetric monoidal category}

\begin{definition}\label{left_module_object}
	Let $(E,\mu,e)$ be a monoid object in $\cC$. Then a \emph{(left) module object} $(N,\kappa)$ over $(E,\mu,e)$ is the data of an object $N$ in $\cC$ and a morphism $\kappa:E\otimes N\to N$ such that the following two diagrams commute in $\cC$:
	% https://q.uiver.app/#q=WzAsOCxbMCwwLCJTXFxvdGltZXMgTiJdLFsxLDAsIkVcXG90aW1lcyBOIl0sWzEsMSwiTiJdLFsyLDAsIihFXFxvdGltZXMgRSlcXG90aW1lcyBOIl0sWzQsMCwiRVxcb3RpbWVzIE4iXSxbNCwxLCJOIl0sWzIsMSwiRVxcb3RpbWVzKEVcXG90aW1lcyBOKSJdLFszLDEsIkVcXG90aW1lcyBOIl0sWzAsMSwiZVxcb3RpbWVzIE4iXSxbMSwyLCJcXGthcHBhIl0sWzAsMiwiXFxsYW1iZGFfTiIsMl0sWzMsNCwiXFxtdVxcb3RpbWVzIE4iXSxbNCw1LCJcXGthcHBhIl0sWzMsNiwiXFxhbHBoYSIsMl0sWzYsNywiRVxcb3RpbWVzIFxca2FwcGEiXSxbNyw1LCJcXGthcHBhIl1d
	\[\begin{tikzcd}
		{S\otimes N} & {E\otimes N} & {(E\otimes E)\otimes N} && {E\otimes N} \\
		& N & {E\otimes(E\otimes N)} & {E\otimes N} & N
		\arrow["{e\otimes N}", from=1-1, to=1-2]
		\arrow["\kappa", from=1-2, to=2-2]
		\arrow["{\lambda_N}"', from=1-1, to=2-2]
		\arrow["{\mu\otimes N}", from=1-3, to=1-5]
		\arrow["\kappa", from=1-5, to=2-5]
		\arrow["\alpha"', from=1-3, to=2-3]
		\arrow["{E\otimes \kappa}", from=2-3, to=2-4]
		\arrow["\kappa", from=2-4, to=2-5]
	\end{tikzcd}\]
\end{definition}

\begin{definition}\label{homomorphism_of_left_module_objects}
	Let $(E,\mu,e)$ be a monoid object in $\cC$, and suppose we have two (left) module objects $(N,\kappa)$ and $(N',\kappa')$ over $(E,\mu,e)$. Then a morphism $f:N\to N'$ is a \emph{(left) $E$-module homomorphism} if the following diagram commutes in $\cC$:
	% https://q.uiver.app/#q=WzAsNCxbMCwwLCJFXFxvdGltZXMgTiJdLFsxLDAsIkVcXG90aW1lcyBOJyJdLFswLDEsIk4iXSxbMSwxLCJOJyJdLFswLDEsIkVcXG90aW1lcyBmIl0sWzAsMiwiXFxrYXBwYSIsMl0sWzIsMywiZiJdLFsxLDMsIlxca2FwcGEnIl1d
	\[\begin{tikzcd}
		{E\otimes N} & {E\otimes N'} \\
		N & {N'}
		\arrow["{E\otimes f}", from=1-1, to=1-2]
		\arrow["\kappa"', from=1-1, to=2-1]
		\arrow["f", from=2-1, to=2-2]
		\arrow["{\kappa'}", from=1-2, to=2-2]
	\end{tikzcd}\]
\end{definition}

\begin{definition}
	Given a monoid object $(E,\mu,e)$ in $\cC$, we write $E\text-\Mod$ to denote the category of (left) module objects over $E$ and $E$-module homomorphisms between them. We denote the homset in $E\text-\Mod$ by
	\[\Hom_{E\text-\Mod}(M,N),\qquad\text{or simply}\qquad\Hom_E(M,N).\]
\end{definition}

For our purposes, we will only consider left module objects, so we will usually drop the quanitfier ``left'' and just refer to them as ``module objects''.

\begin{lemma}\label{module_if_iso_to_module}
	Let $(E,\mu,e)$ be a monoid object in $\cC$ and let $(N,\kappa)$ be an $E$ module object. Then given some object $X$ in $\cC$ and an isomorphism $\phi:N\xr\cong X$, $X$ inherits the structure of an $E$-module via the action map
	\[\kappa_\phi:E\otimes X\xr{E\otimes\phi^{-1}}E\otimes N\xr\kappa N\xr\phi X.\]
\end{lemma}
\begin{proof}
	We need to show the two coherence diagrams in \autoref{left_module_object} commute. To see the former commutes, consider the following diagram:
	% https://q.uiver.app/#q=WzAsNixbMCwwLCJYIl0sWzMsMywiWCJdLFszLDAsIkVcXG90aW1lcyBYIl0sWzMsMSwiRVxcb3RpbWVzIE4iXSxbMywyLCJOIl0sWzIsMSwiTiJdLFswLDEsIiIsMCx7ImxldmVsIjoyLCJzdHlsZSI6eyJoZWFkIjp7Im5hbWUiOiJub25lIn19fV0sWzAsMiwiZVxcb3RpbWVzIFgiXSxbMiwzLCJFXFxvdGltZXNcXHBoaV57LTF9Il0sWzMsNCwiXFxrYXBwYSJdLFs0LDEsIlxccGhpIl0sWzUsNCwiIiwwLHsibGV2ZWwiOjIsInN0eWxlIjp7ImhlYWQiOnsibmFtZSI6Im5vbmUifX19XSxbNSwzLCJlXFxvdGltZXMgTiJdLFswLDUsIlxccGhpXnstMX0iXV0=
	\[\begin{tikzcd}
		X &&& {E\otimes X} \\
		&& N & {E\otimes N} \\
		&&& N \\
		&&& X
		\arrow[Rightarrow, no head, from=1-1, to=4-4]
		\arrow["{e\otimes X}", from=1-1, to=1-4]
		\arrow["{E\otimes\phi^{-1}}", from=1-4, to=2-4]
		\arrow["\kappa", from=2-4, to=3-4]
		\arrow["\phi", from=3-4, to=4-4]
		\arrow[Rightarrow, no head, from=2-3, to=3-4]
		\arrow["{e\otimes N}", from=2-3, to=2-4]
		\arrow["{\phi^{-1}}", from=1-1, to=2-3]
	\end{tikzcd}\]
	The top trapezoid commutes by functoriality of $-\otimes-$. The middle small triangle commutes by unitality of $\kappa$. The remaining region commutes by definition. To see the second coherence diagram commutes, consider the following diagram:
	% https://q.uiver.app/#q=WzAsMTAsWzAsMCwiRVxcb3RpbWVzIEVcXG90aW1lcyBYIl0sWzMsMCwiRVxcb3RpbWVzIFgiXSxbMywxLCJFXFxvdGltZXMgTiJdLFszLDIsIk4iXSxbMywzLCJYIl0sWzAsMSwiRVxcb3RpbWVzIEVcXG90aW1lcyBOIl0sWzAsMiwiRVxcb3RpbWVzIE4iXSxbMCwzLCJFXFxvdGltZXMgWCJdLFsxLDMsIkVcXG90aW1lcyBOIl0sWzIsMywiTiJdLFswLDEsIlxcbXVcXG90aW1lcyBYIl0sWzEsMiwiRVxcb3RpbWVzXFxwaGleey0xfSJdLFsyLDMsIlxca2FwcGEiXSxbMyw0LCJcXHBoaSJdLFswLDUsIkVcXG90aW1lcyBFXFxvdGltZXNcXHBoaV57LTF9IiwyXSxbNSw2LCJFXFxvdGltZXNcXGthcHBhIiwyXSxbNiw3LCJFXFxvdGltZXNcXHBoaSIsMl0sWzcsOCwiRVxcb3RpbWVzXFxwaGleezEtfSIsMl0sWzgsOSwiXFxrYXBwYSIsMl0sWzksNCwiXFxwaGkiLDJdLFs1LDIsIlxcbXVcXG90aW1lcyBOIl0sWzYsMywiXFxrYXBwYSJdLFs2LDgsIiIsMCx7ImxldmVsIjoyLCJzdHlsZSI6eyJoZWFkIjp7Im5hbWUiOiJub25lIn19fV1d
	\[\begin{tikzcd}
		{E\otimes E\otimes X} &&& {E\otimes X} \\
		{E\otimes E\otimes N} &&& {E\otimes N} \\
		{E\otimes N} &&& N \\
		{E\otimes X} & {E\otimes N} & N & X
		\arrow["{\mu\otimes X}", from=1-1, to=1-4]
		\arrow["{E\otimes\phi^{-1}}", from=1-4, to=2-4]
		\arrow["\kappa", from=2-4, to=3-4]
		\arrow["\phi", from=3-4, to=4-4]
		\arrow["{E\otimes E\otimes\phi^{-1}}"', from=1-1, to=2-1]
		\arrow["E\otimes\kappa"', from=2-1, to=3-1]
		\arrow["E\otimes\phi"', from=3-1, to=4-1]
		\arrow["{E\otimes\phi^{1-}}"', from=4-1, to=4-2]
		\arrow["\kappa"', from=4-2, to=4-3]
		\arrow["\phi"', from=4-3, to=4-4]
		\arrow["{\mu\otimes N}", from=2-1, to=2-4]
		\arrow["\kappa", from=3-1, to=3-4]
		\arrow[Rightarrow, no head, from=3-1, to=4-2]
	\end{tikzcd}\]
	The top rectangle commutes by functoriality of $-\otimes-$. The middle rectangle commutes by coherence for $\kappa$. The bottom two regions commute by definition.
\end{proof}

\begin{proposition}\label{free_forgetful_E-Mod}
	Given a monoid object $(E,\mu,e)$ in $\cC$, the forgetful functor $E\text-\Mod\to\cC$ has a left adjoint $\cC\to E\text-\Mod$ sending an object $X$ in $\cC$ to $(E\otimes X,\kappa_X)$ where $\kappa_X$ is the composition
	\[E\otimes(E\otimes X)\xr{\alpha^{-1}}(E\otimes E)\otimes X\xr{\mu\otimes X}E\otimes X,\]
	and sending a morphism $f:X\to Y$ to $E\otimes f:E\otimes X\to E\otimes Y$.

	We call this functor $E\otimes-:\cC\to E\text-\Mod$ the \emph{free} functor, and we call $E$-modules in the image of the free functor \emph{free modules}.
\end{proposition}
\begin{proof}
	In this proof, we work in a symmetric strict monoidal category. First, we wish to show that $E\otimes-:\cC\to E\text-\Mod$ as constructed is well-defined. First, to see that $(X,\kappa_X)$ is actually a $E$-module, we need to show the two diagrams in \autoref{left_module_object} commute. Indeed, consider the following diagrams:
	% https://q.uiver.app/#q=WzAsNyxbMCwwLCJFXFxvdGltZXMgWCJdLFsxLDAsIkVcXG90aW1lcyBFXFxvdGltZXMgWCJdLFsxLDEsIkVcXG90aW1lcyBYIl0sWzIsMCwiRVxcb3RpbWVzIEVcXG90aW1lcyBFXFxvdGltZXMgWCJdLFszLDAsIkVcXG90aW1lcyBFXFxvdGltZXMgWCJdLFszLDEsIkVcXG90aW1lcyBYIl0sWzIsMSwiRVxcb3RpbWVzIEVcXG90aW1lcyBYIl0sWzAsMSwiZVxcb3RpbWVzIEVcXG90aW1lcyBYIl0sWzEsMiwiXFxtdVxcb3RpbWVzIFgiXSxbMCwyLCIiLDIseyJsZXZlbCI6Miwic3R5bGUiOnsiaGVhZCI6eyJuYW1lIjoibm9uZSJ9fX1dLFszLDQsIlxcbXVcXG90aW1lcyBFXFxvdGltZXMgWCJdLFs0LDUsIlxcbXVcXG90aW1lcyBYIl0sWzMsNiwiRVxcb3RpbWVzIFxcbXVcXG90aW1lcyBYIiwyXSxbNiw1LCJcXG11XFxvdGltZXMgWCIsMl1d
	\[\begin{tikzcd}
		{E\otimes X} & {E\otimes E\otimes X} & {E\otimes E\otimes E\otimes X} & {E\otimes E\otimes X} \\
		& {E\otimes X} & {E\otimes E\otimes X} & {E\otimes X}
		\arrow["{e\otimes E\otimes X}", from=1-1, to=1-2]
		\arrow["{\mu\otimes X}", from=1-2, to=2-2]
		\arrow[Rightarrow, no head, from=1-1, to=2-2]
		\arrow["{\mu\otimes E\otimes X}", from=1-3, to=1-4]
		\arrow["{\mu\otimes X}", from=1-4, to=2-4]
		\arrow["{E\otimes \mu\otimes X}"', from=1-3, to=2-3]
		\arrow["{\mu\otimes X}"', from=2-3, to=2-4]
	\end{tikzcd}\]
	These are precisely the diagrams obtained by applying $X\otimes-$ to the coherence diagrams for $\mu$, so that they commute as desired. Now, suppose $f:X\to Y$ is a morphism in $\cC$, then we would like to show that $E\otimes f:E\otimes X\to E\otimes Y$ is a morphism of $E$-module objects. Indeed, consider the following diagram:
	% https://q.uiver.app/#q=WzAsNCxbMCwwLCJFXFxvdGltZXMgRVxcb3RpbWVzIFgiXSxbMSwwLCJFXFxvdGltZXMgRVxcb3RpbWVzIFkiXSxbMSwxLCJFXFxvdGltZXMgWSJdLFswLDEsIkVcXG90aW1lcyBYIl0sWzAsMSwiRVxcb3RpbWVzIEVcXG90aW1lcyBmIl0sWzEsMiwiXFxtdVxcb3RpbWVzIFkiXSxbMCwzLCJcXG11XFxvdGltZXMgWCIsMl0sWzMsMiwiRVxcb3RpbWVzIGYiXV0=
	\[\begin{tikzcd}
		{E\otimes E\otimes X} & {E\otimes E\otimes Y} \\
		{E\otimes X} & {E\otimes Y}
		\arrow["{E\otimes E\otimes f}", from=1-1, to=1-2]
		\arrow["{\mu\otimes Y}", from=1-2, to=2-2]
		\arrow["{\mu\otimes X}"', from=1-1, to=2-1]
		\arrow["{E\otimes f}", from=2-1, to=2-2]
	\end{tikzcd}\]
	It commutes by functoriality of $-\otimes-$, so $E\otimes f$ is indeed an $E$-module homomorphism as desired.

	Now, in order to see that $E\otimes-$ is left adjoint to the forgetful functor, it suffices to construct a unit and counit for the adjunction and show they satisfy the zig-zag identities. Given $X$ in $\cC$ and $(N,\kappa)$ in $E\text-\Mod$, define $\eta_X:=e\otimes X:X\to E\otimes X$ and $\vare_{(N,\kappa)}:=\kappa:E\otimes N\to N$. $\eta_X$ is clearly natural in $X$ by functoriality of $-\otimes-$, and $\vare_{(N,\kappa)}$ is natural in $(N,\kappa)$ by how morphisms in $E\text-\Mod$ are defined. Now, to see these are actually the unit and counit of an adjunction, we need to show that the following diagrams commute for all $X$ in $\cC$ and $(N,\kappa)$ in $E\text-\Mod$:
	% https://q.uiver.app/#q=WzAsNixbMCwwLCJFXFxvdGltZXMgWCJdLFsyLDAsIkVcXG90aW1lcyBFXFxvdGltZXMgWCJdLFsyLDIsIkVcXG90aW1lcyBYIl0sWzUsMCwiRVxcb3RpbWVzIE4iXSxbNSwyLCJOIl0sWzcsMCwiTiJdLFswLDEsIkVcXG90aW1lc1xcZXRhX1g9RVxcb3RpbWVzIGVcXG90aW1lcyBYIl0sWzEsMiwiXFx2YXJlX3soRVxcb3RpbWVzIFgsXFxrYXBwYV9YKX09XFxtdVxcb3RpbWVzIFgiXSxbMCwyLCIiLDIseyJsZXZlbCI6Miwic3R5bGUiOnsiaGVhZCI6eyJuYW1lIjoibm9uZSJ9fX1dLFszLDQsIlxcdmFyZV97KE4sXFxrYXBwYSl9PVxca2FwcGEiLDJdLFs1LDMsIlxcZXRhX049ZVxcb3RpbWVzIE4iLDJdLFs1LDQsIiIsMCx7ImxldmVsIjoyLCJzdHlsZSI6eyJoZWFkIjp7Im5hbWUiOiJub25lIn19fV1d
	\[\begin{tikzcd}
		{E\otimes X} && {E\otimes E\otimes X} &&& {E\otimes N} && N \\
		\\
		&& {E\otimes X} &&& N
		\arrow["{E\otimes\eta_X=E\otimes e\otimes X}", from=1-1, to=1-3]
		\arrow["{\vare_{(E\otimes X,\kappa_X)}=\mu\otimes X}", from=1-3, to=3-3]
		\arrow[Rightarrow, no head, from=1-1, to=3-3]
		\arrow["{\vare_{(N,\kappa)}=\kappa}"', from=1-6, to=3-6]
		\arrow["{\eta_N=e\otimes N}"', from=1-8, to=1-6]
		\arrow[Rightarrow, no head, from=1-8, to=3-6]
	\end{tikzcd}\]
	Commutativity of the left diagram is unitality of $\mu$, while commutativity of the right diagram is unitality of $\kappa$. Thus indeed $E\otimes-:\cC\to E\text-\Mod$ is a left adjoint of the forgetful functor $E\text-\Mod\to\cC$, as desired.
\end{proof}

%\begin{definition}\label{free_module}
%	We call the functor $E\otimes-:\cC\to E\text-\Mod$ constructed above the \emph{free} functor, and we call $E$-modules in the image of the free functor \emph{free modules}.
%\end{definition}

\subsection{Monoid objects in \texorpdfstring{$\cSH$}{SH} and their associated rings}

For the remainder of this appendix, we fix a monoidal closed tensor triangulated category $(\cSH,\otimes,S,\Sigma,e,\cD)$ (\autoref{tentri}) with arbitrary (small) (co)products and sub-Picard grading $(A,\1,h,\{S^a\},\{\phi_{a,b}\})$ (\autoref{sub_Picard_grading_defn}), and we adopt the conventions outlined in \Cref{setup}. In all proofs that follow we will freely use the coherence theorem for symmetric monoidal categories. In particular, we will assume without loss of generality that the associators and unitors in $\cSH$ are identities.

\begin{proposition}\label{pi_*E_is_ring_for_E_monoid_appendix}
	The assignment $(E,\mu,e)\mapsto\pi_*(E)$ is a functor $\pi_*$ from the category $\Mon_\cSH$ of monoid objects in $\cSH$ (\autoref{Mon_C,CMon_C}) to the category of $A$-graded rings. In particular, given a monoid object $(E,\mu,e)$ in $\cSH$, $\pi_*(E)$ is canonically a ring with product $\pi_*(E)\times\pi_*(E)\to\pi_*(E)$ which sends classes $x:S^a\to E$ and $y:S^b\to E$ to the composition
	\[xy:S^{a+b}\xr{\phi_{a,b}}S^a\otimes S^b\xr{x\otimes y}E\otimes E\xr\mu E,\]
	and the unit of this ring is $e\in\pi_0(E)=[S,E]$.
\end{proposition}
\begin{proof}
	First, we show that $\pi_*(E)$ is actually a ring as indicated. By \autoref{A_graded_ring}, in order to make the $A$-graded abelian group $\pi_*(E)$ into an $A$-graded ring, it suffices to construct an associative and unital product only with respect to homogeneous elements. Suppose we have classes $x$, $y$, and $z$ in $\pi_a(E)$, $\pi_b(E)$, and $\pi_c(E)$, respectively. To see associativity, consider the following diagram:
	% https://q.uiver.app/#q=WzAsNixbMCwxLCJTXnthK2IrY30iXSxbMSwxLCJTXmFcXG90aW1lcyBTXmJcXG90aW1lcyBTXmMiXSxbMiwxLCIgIEVcXG90aW1lcyBFXFxvdGltZXMgRSJdLFszLDAsIkVcXG90aW1lcyBFIl0sWzMsMSwiRSJdLFszLDIsIkVcXG90aW1lcyBFIl0sWzAsMSwiXFxjb25nIl0sWzEsMiwieFxcb3RpbWVzIHlcXG90aW1lcyB6Il0sWzIsMywiXFxtdVxcb3RpbWVzIEUiXSxbMyw0LCJcXG11Il0sWzIsNSwiRVxcb3RpbWVzXFxtdSIsMl0sWzUsNCwiXFxtdSIsMl1d
	\[\begin{tikzcd}
		&&& {E\otimes E} \\
		{S^{a+b+c}} & {S^a\otimes S^b\otimes S^c} & {  E\otimes E\otimes E} & E \\
		&&& {E\otimes E}
		\arrow["\cong", from=2-1, to=2-2]
		\arrow["{x\otimes y\otimes z}", from=2-2, to=2-3]
		\arrow["{\mu\otimes E}", from=2-3, to=1-4]
		\arrow["\mu", from=1-4, to=2-4]
		\arrow["E\otimes\mu"', from=2-3, to=3-4]
		\arrow["\mu"', from=3-4, to=2-4]
	\end{tikzcd}\]
	(here the first arrow is the unique isomorphism obtained by composing products of $\phi_{a,b}$'s, see \autoref{unique_comp_Sas}). It commutes by associativity of $\mu$. It follows by functoriality of $-\otimes-$ that the top composition is $(x\cdot y)\cdot z$ while the bottom is $x\cdot(y\cdot z)$, so they are equal as desired. To see that $e\in\pi_0(E)$ is a left and right unit for this multiplication, consider the following diagram
	% https://q.uiver.app/#q=WzAsNSxbMiwwLCJTXmEiXSxbMiwxLCJFIl0sWzAsMSwiRVxcb3RpbWVzIEUiXSxbNCwxLCJFXFxvdGltZXMgRSJdLFsyLDIsIkUiXSxbMCwxLCJ4Il0sWzAsMiwiZVxcb3RpbWVzIHgiLDJdLFswLDMsInhcXG90aW1lcyBlIl0sWzEsNCwiIiwxLHsibGV2ZWwiOjIsInN0eWxlIjp7ImhlYWQiOnsibmFtZSI6Im5vbmUifX19XSxbMSwyLCJlXFxvdGltZXMgRSIsMl0sWzEsMywiRVxcb3RpbWVzIGUiXSxbMyw0LCJcXG11Il0sWzIsNCwiXFxtdSIsMl1d
	\[\begin{tikzcd}
		&& {S^a} \\
		{E\otimes E} && E && {E\otimes E} \\
		&& E
		\arrow["x", from=1-3, to=2-3]
		\arrow["{e\otimes x}"', from=1-3, to=2-1]
		\arrow["{x\otimes e}", from=1-3, to=2-5]
		\arrow[Rightarrow, no head, from=2-3, to=3-3]
		\arrow["{e\otimes E}"', from=2-3, to=2-1]
		\arrow["{E\otimes e}", from=2-3, to=2-5]
		\arrow["\mu", from=2-5, to=3-3]
		\arrow["\mu"', from=2-1, to=3-3]
	\end{tikzcd}\]
	Commutativity of the two top triangles is functoriality of $-\otimes-$. Commutativity of the bottom two triangles is unitality of $\mu$. Thus the diagram commutes, so $e\cdot x=x=x\cdot e$. Finally, we wish to show this product is bilinear (distributive). Suppose we further have some $x'\in\pi_a(E)$ and $y'\in\pi_b(E)$, and consider the following diagrams:
	% https://q.uiver.app/#q=WzAsMTgsWzAsMCwiU157YStifSJdLFsxLDAsIlNeYVxcb3RpbWVzIFNeYiJdLFswLDEsIlNee2ErYn1cXG9wbHVzIFNee2ErYn0iXSxbMSwxLCIoU15hXFxvdGltZXMgU15iKVxcb3BsdXMoU15hXFxvdGltZXMgU15iKSJdLFsyLDAsIihTXmFcXG9wbHVzIFNeYSlcXG90aW1lcyBTXmIiXSxbMiwxLCIoRVxcb3RpbWVzIEUpXFxvcGx1cyhFXFxvdGltZXMgRSkiXSxbMywwLCIoRVxcb3BsdXMgRSlcXG90aW1lcyBFIl0sWzMsMSwiRVxcb3RpbWVzIEUiXSxbMCwyLCJTXnthK2J9Il0sWzEsMiwiU15hXFxvdGltZXMgU15iIl0sWzIsMiwiU15iXFxvdGltZXMoU15iXFxvcGx1cyBTXmIpIl0sWzMsMiwiRVxcb3RpbWVzKEVcXG9wbHVzIEUpIl0sWzMsMywiRVxcb3RpbWVzIEUiXSxbMiwzLCIoRVxcb3RpbWVzIEUpXFxvcGx1cyhFXFxvdGltZXMgRSkiXSxbMSwzLCIoU15hXFxvdGltZXMgU15iKVxcb3BsdXMoU15hXFxvdGltZXMgU15iKSJdLFswLDMsIlNee2ErYn1cXG9wbHVzIFNee2ErYn0iXSxbNCwxLCJFIl0sWzQsMywiRSJdLFswLDEsIlxccGhpX3thLGJ9Il0sWzAsMiwiXFxEZWx0YSIsMl0sWzIsMywiXFxwaGlfe2EsYn1cXG9wbHVzXFxwaGlfe2EsYn0iLDJdLFsxLDMsIlxcRGVsdGEiXSxbMSw0LCJcXERlbHRhXFxvdGltZXMgU15iIl0sWzQsMywiXFxjb25nIiwyXSxbMyw1LCIoeFxcb3RpbWVzIHkpXFxvcGx1cyh4J1xcb3RpbWVzIHkpIiwyXSxbNCw2LCIoeFxcb3BsdXMgeCcpXFxvdGltZXMgeSJdLFs2LDUsIlxcY29uZyIsMl0sWzUsNywiXFxuYWJsYSIsMl0sWzYsNywiXFxuYWJsYVxcb3RpbWVzIEUiXSxbOCw5LCJcXHBoaV97YSxifSJdLFs5LDEwLCJTXmFcXG90aW1lc1xcRGVsdGEiXSxbMTAsMTEsInhcXG90aW1lcyh5XFxvcGx1cyB5JykiXSxbMTEsMTIsIkVcXG90aW1lc1xcbmFibGEiXSxbMTEsMTMsIlxcY29uZyIsMl0sWzEzLDEyLCJcXG5hYmxhIiwyXSxbMTAsMTQsIlxcY29uZyIsMl0sWzE0LDEzLCIoeFxcb3RpbWVzIHkpXFxvcGx1cyh4XFxvdGltZXMgeScpIiwyXSxbOSwxNCwiXFxEZWx0YSJdLFs4LDE1LCJcXERlbHRhIiwyXSxbMTUsMTQsIlxccGhpX3thLGJ9XFxvcGx1c1xccGhpX3thLGJ9IiwyXSxbNywxNiwiXFxtdSJdLFsxMiwxNywiXFxtdSJdXQ==
	\[\begin{tikzcd}
		{S^{a+b}} & {S^a\otimes S^b} & {(S^a\oplus S^a)\otimes S^b} & {(E\oplus E)\otimes E} \\
		{S^{a+b}\oplus S^{a+b}} & {(S^a\otimes S^b)\oplus(S^a\otimes S^b)} & {(E\otimes E)\oplus(E\otimes E)} & {E\otimes E} & E \\
		{S^{a+b}} & {S^a\otimes S^b} & {S^b\otimes(S^b\oplus S^b)} & {E\otimes(E\oplus E)} \\
		{S^{a+b}\oplus S^{a+b}} & {(S^a\otimes S^b)\oplus(S^a\otimes S^b)} & {(E\otimes E)\oplus(E\otimes E)} & {E\otimes E} & E
		\arrow["{\phi_{a,b}}", from=1-1, to=1-2]
		\arrow["\Delta"', from=1-1, to=2-1]
		\arrow["{\phi_{a,b}\oplus\phi_{a,b}}"', from=2-1, to=2-2]
		\arrow["\Delta", from=1-2, to=2-2]
		\arrow["{\Delta\otimes S^b}", from=1-2, to=1-3]
		\arrow["\cong"', from=1-3, to=2-2]
		\arrow["{(x\otimes y)\oplus(x'\otimes y)}"', from=2-2, to=2-3]
		\arrow["{(x\oplus x')\otimes y}", from=1-3, to=1-4]
		\arrow["\cong"', from=1-4, to=2-3]
		\arrow["\nabla"', from=2-3, to=2-4]
		\arrow["{\nabla\otimes E}", from=1-4, to=2-4]
		\arrow["{\phi_{a,b}}", from=3-1, to=3-2]
		\arrow["{S^a\otimes\Delta}", from=3-2, to=3-3]
		\arrow["{x\otimes(y\oplus y')}", from=3-3, to=3-4]
		\arrow["E\otimes\nabla", from=3-4, to=4-4]
		\arrow["\cong"', from=3-4, to=4-3]
		\arrow["\nabla"', from=4-3, to=4-4]
		\arrow["\cong"', from=3-3, to=4-2]
		\arrow["{(x\otimes y)\oplus(x\otimes y')}"', from=4-2, to=4-3]
		\arrow["\Delta", from=3-2, to=4-2]
		\arrow["\Delta"', from=3-1, to=4-1]
		\arrow["{\phi_{a,b}\oplus\phi_{a,b}}"', from=4-1, to=4-2]
		\arrow["\mu", from=2-4, to=2-5]
		\arrow["\mu", from=4-4, to=4-5]
	\end{tikzcd}\]
	The unlabeled isomorphisms are those given by the fact that $-\otimes-$ is additive in each variable (since $\cSH$ is tensor triangulated). Commutativity of the left squares is naturality of $\Delta:X\to X\oplus X$ in an additive category. Commutativity of the rest of the diagram follows again from the fact that $-\otimes-$ is an additive functor in each variable. Hence, by functoriality of $-\otimes-$, these diagrams tell us that $(x+x')\cdot y=x\cdot y+x'\cdot y$ and $x\cdot(y+y')=x\cdot y+x\cdot y'$, respectively. Thus, we have shown that if $(E,\mu,e)$ is a monoid object in $\cSH$ then $\pi_*(E)$ is a ring, as desired.

	It remains to show that given a homomorphism of monoid objects $f:(E_1,\mu_1,e_1)\to(E_2,\mu_2,e_2)$ in $\Mon_\cSH$ that $\pi_*(f):\pi_*(E_1)\to\pi_*(E_2)$ is an $A$-graded ring homomorphism. First of all, we know this is an $A$-graded abelian group homomorphism, since $\cSH$ is an additive category, meaning composition with $f$ is an abelian group homomorphism. Thus, in order to show it's a ring homomorphism, it remains to show that $\pi_*(f)(e_1)=e_2$ and that for all $x,y\in\pi_*(E)$ we have $\pi_*(f)(x\cdot y)=\pi_*(f)(x)\cdot\pi_*(f)(y)$. To see the former, note that $\pi_*(f)(e_1)=f\circ e_1$, and $f\circ e_1=e_2$ since $f$ is a monoid homomorphism in $\cSH$. To see the latter, first note by distributivity of multiplication in $\pi_*(E_1)$ and $\pi_*(E_2)$ and the fact that $\pi_*(f)$ is a group homomorphism, it suffices to consider the case that $x$ and $y$ are homogeneous of the form $x:S^a\to E_1$ and $y:S^b\to E_2$. In this case, consider the following diagram:
	% https://q.uiver.app/#q=WzAsNixbMCwwLCJTXnthK2J9Il0sWzEsMCwiU15hXFxvdGltZXMgU15iIl0sWzIsMCwiRV8xXFxvdGltZXMgRV8xIl0sWzMsMCwiRV8yXFxvdGltZXMgRV8yIl0sWzMsMSwiRV8yIl0sWzIsMSwiRV8xIl0sWzAsMSwiXFxwaGlfe2EsYn0iXSxbMSwyLCJ4XFxvdGltZXMgeSJdLFsyLDMsImZcXG90aW1lcyBmIl0sWzMsNCwiXFxtdV8yIl0sWzIsNSwiXFxtdV8xIiwyXSxbNSw0LCJmIl1d
	\[\begin{tikzcd}
		{S^{a+b}} & {S^a\otimes S^b} & {E_1\otimes E_1} & {E_2\otimes E_2} \\
		&& {E_1} & {E_2}
		\arrow["{\phi_{a,b}}", from=1-1, to=1-2]
		\arrow["{x\otimes y}", from=1-2, to=1-3]
		\arrow["{f\otimes f}", from=1-3, to=1-4]
		\arrow["{\mu_2}", from=1-4, to=2-4]
		\arrow["{\mu_1}"', from=1-3, to=2-3]
		\arrow["f", from=2-3, to=2-4]
	\end{tikzcd}\]
	The top composition is $\pi_*(f)(x)\cdot\pi_*(f)(y)$, while the bottom composition is $\pi_*(f)(x\cdot y)$. The diagram commutes since $f$ is a monoid object homomorphism. Thus $\pi_*(f)(x\cdot y)=\pi_*(f)(x)\cdot\pi_*(f)(y)$, as desired.
\end{proof}

\begin{proposition}\label{pi_*(E)_is_A-graded_commutative_if_E_is_commutative}
	For all $a,b\in A$ there exists an element $\theta_{a,b}\in\pi_0(S)=[S,S]$ such that given any commutative monoid object $(E,\mu,e)$ in $\cSH$, the $A$-graded ring structure on $\pi_\ast(E)$ (\autoref{pi_*E_is_ring_for_E_monoid}) has a commutativity formula given by
	\[x\cdot y=y\cdot x\cdot (e\circ\theta_{a,b})\]
	for all $x\in\pi_a(E)$ and $y\in\pi_b(E)$. In particular, $\theta_{a,b}\in\mathrm{Aut}(S)$ is the composition
	\[S\xr{\cong}S^{-a-b}\otimes S^a\otimes S^b\xr{S^{-a-b}\otimes\tau}S^{-a-b}\otimes S^b\otimes S^a\xr\cong S,\]
	where the outermost maps are the unique maps specified by \autoref{unique_comp_Sas}.
\end{proposition}
\begin{proof}
	Let $(E,\mu,e)$, $x$, and $y$ as in the statement of the proposition. Now consider the following diagram
	% https://q.uiver.app/#q=WzAsNyxbMCwwLCJTXnthK2J9Il0sWzAsMiwiU157YStifSJdLFsyLDIsIlNeYlxcb3RpbWVzIFNeYSJdLFsyLDAsIlNeYVxcb3RpbWVzIFNeYiJdLFs0LDAsIkVcXG90aW1lcyBFIl0sWzQsMiwiRVxcb3RpbWVzIEUiXSxbNiwxLCJFIl0sWzAsMSwiXFxwaGlfe2IsYX1eey0xfVxcY2lyY1xcdGF1XFxjaXJjXFxwaGlfe2EsYn0iLDIseyJzdHlsZSI6eyJib2R5Ijp7Im5hbWUiOiJkYXNoZWQifX19XSxbMSwyLCJcXHBoaV97YixhfSJdLFswLDMsIlxccGhpX3thLGJ9Il0sWzMsMiwiXFx0YXUiLDJdLFs0LDUsIlxcdGF1IiwyXSxbNCw2LCJcXG11Il0sWzIsNSwieVxcb3RpbWVzIHgiXSxbNSw2LCJcXG11IiwyXSxbMyw0LCJ4XFxvdGltZXMgeSJdXQ==
	\[\begin{tikzcd}[sep=small]
		{S^{a+b}} && {S^a\otimes S^b} && {E\otimes E} \\
		&&&&&& E \\
		{S^{a+b}} && {S^b\otimes S^a} && {E\otimes E}
		\arrow["{\phi_{b,a}^{-1}\circ\tau\circ\phi_{a,b}}"', dashed, from=1-1, to=3-1]
		\arrow["{\phi_{b,a}}", from=3-1, to=3-3]
		\arrow["{\phi_{a,b}}", from=1-1, to=1-3]
		\arrow["\tau"', from=1-3, to=3-3]
		\arrow["\tau"', from=1-5, to=3-5]
		\arrow["\mu", from=1-5, to=2-7]
		\arrow["{y\otimes x}", from=3-3, to=3-5]
		\arrow["\mu"', from=3-5, to=2-7]
		\arrow["{x\otimes y}", from=1-3, to=1-5]
	\end{tikzcd}\]
	The left square commutes by definition. The middle square commutes by naturality of the symmetry isomorphism. Finally, the right square commutes by commutativity of $E$. Unravelling definitions, we have shown that under the product on $\pi_\ast(E)$ induced by the $\phi_{a,b}$'s,
	\[x\cdot y=(y\cdot x)\circ(\phi_{b,a}^{-1}\circ\tau\circ\phi_{a,b}).\]
	Thus, in order to show the desired result it further suffices to show that
	\[(y\cdot x)\circ(\phi_{b,a}^{-1}\circ\tau\circ\phi_{a,b})=y\cdot x\cdot(e\circ\theta_{a,b}).\]
	Consider the following diagram:
	% https://q.uiver.app/#q=WzAsMTIsWzAsMCwiU157YStifSJdLFswLDEsIlNeYlxcb3RpbWVzIFNeYVxcb3RpbWVzIFNeey1hLWJ9XFxvdGltZXMgU15hXFxvdGltZXMgU15iIl0sWzAsMiwiU15iXFxvdGltZXMgU15hXFxvdGltZXMgU157LWEtYn1cXG90aW1lcyBTXmJcXG90aW1lcyBTXmEiXSxbMCw0LCJFXFxvdGltZXMgRVxcb3RpbWVzIEUiXSxbMiw0LCJFXFxvdGltZXMgRSJdLFsxLDIsIlNeYlxcb3RpbWVzIFNeYSJdLFsyLDAsIlNeYVxcb3RpbWVzIFNeYiJdLFsyLDEsIlNeYlxcb3RpbWVzIFNeYSJdLFsyLDIsIlNee2ErYn0iXSxbMiwzLCJFXFxvdGltZXMgRSJdLFswLDUsIkVcXG90aW1lcyBFIl0sWzIsNSwiRSJdLFswLDEsIlxcY29uZyIsMl0sWzAsNiwiXFxwaGlfe2EsYn0iXSxbNiw3LCJcXHRhdSJdLFs3LDgsIlxccGhpX3tiLGF9XnstMX0iXSxbOCw1LCJcXHBoaV97YixhfSIsMl0sWzIsNywiXFxjb25nIl0sWzEsNiwiXFxjb25nIiwyXSxbOSwzLCJFXFxvdGltZXMgRVxcb3RpbWVzIGUiLDJdLFsxLDIsIlNeYlxcb3RpbWVzIFNeYVxcb3RpbWVzIFNeey1hLWJ9XFxvdGltZXNcXHRhdSIsMl0sWzMsMTAsIlxcbXVcXG90aW1lcyBFIiwyXSxbMyw0LCJFXFxvdGltZXMgXFxtdSJdLFs0LDExLCJcXG11Il0sWzEwLDExLCJcXG11IiwyXSxbOSw0LCIiLDAseyJsZXZlbCI6Miwic3R5bGUiOnsiaGVhZCI6eyJuYW1lIjoibm9uZSJ9fX1dLFs1LDksInlcXG90aW1lcyB4Il0sWzUsMywieVxcb3RpbWVzIHhcXG90aW1lcyBlIiwyXSxbMiw1LCJcXGNvbmciXSxbNSw3LCIiLDIseyJsZXZlbCI6Miwic3R5bGUiOnsiaGVhZCI6eyJuYW1lIjoibm9uZSJ9fX1dXQ==
	\[\begin{tikzcd}
		{S^{a+b}} && {S^a\otimes S^b} \\
		{S^b\otimes S^a\otimes S^{-a-b}\otimes S^a\otimes S^b} && {S^b\otimes S^a} \\
		{S^b\otimes S^a\otimes S^{-a-b}\otimes S^b\otimes S^a} & {S^b\otimes S^a} & {S^{a+b}} \\
		&& {E\otimes E} \\
		{E\otimes E\otimes E} && {E\otimes E} \\
		{E\otimes E} && E
		\arrow["\cong"', from=1-1, to=2-1]
		\arrow["{\phi_{a,b}}", from=1-1, to=1-3]
		\arrow["\tau", from=1-3, to=2-3]
		\arrow["{\phi_{b,a}^{-1}}", from=2-3, to=3-3]
		\arrow["{\phi_{b,a}}"', from=3-3, to=3-2]
		\arrow["\cong", from=3-1, to=2-3]
		\arrow["\cong"', from=2-1, to=1-3]
		\arrow["{E\otimes E\otimes e}"', from=4-3, to=5-1]
		\arrow["{S^b\otimes S^a\otimes S^{-a-b}\otimes\tau}"', from=2-1, to=3-1]
		\arrow["{\mu\otimes E}"', from=5-1, to=6-1]
		\arrow["{E\otimes \mu}", from=5-1, to=5-3]
		\arrow["\mu", from=5-3, to=6-3]
		\arrow["\mu"', from=6-1, to=6-3]
		\arrow[Rightarrow, no head, from=4-3, to=5-3]
		\arrow["{y\otimes x}", from=3-2, to=4-3]
		\arrow["{y\otimes x\otimes e}"', from=3-2, to=5-1]
		\arrow["\cong", from=3-1, to=3-2]
		\arrow[Rightarrow, no head, from=3-2, to=2-3]
	\end{tikzcd}\]
	Here any map simply labelled $\cong$ is an appropriate composition of copies of $\phi_{a,b}$'s, associators, and their inverses, so that each of these maps are necessarily unique by \autoref{unique_comp_Sas}. The triangles in the top large rectangle commutes by coherence for the $\phi_{a,b}$'s. The parallelogram commutes by naturality of $\tau$ and coherence of the of $\phi_{a,b}$'s. The middle skewed triangle commutes by functoriality of $-\otimes-$. The triangle below that commutes by unitality of $\mu$. Finally, the bottom rectangle commmutes by associativity of $\mu$. Hence, by unravelling definitions and applying functoriality of $-\otimes-$, we get that the right composition is $(y\cdot x)\circ(\phi_{b,a}^{-1}\circ\tau\circ\phi_{a,b})$, while the left composition is $y\cdot x\cdot(e\circ\theta_{a,b})$, so they are equal as desired.
\end{proof}

\begin{lemma}\label{multipy_by_degree_0_is_same_as_compose}
	Suppose we have homogeneous elements $x,y\in\pi_*(S)$ with $x$ of degree $0$, then we have $x\cdot y=y\cdot x=x\circ y$ (where the $\cdot$ denotes the product given in \autoref{pi_*E_is_ring_for_E_monoid_appendix}).
\end{lemma}
\begin{proof}
	As morphisms, $y$ is an arrow $S^a\to S$ for some $a$ in $A$, and $x$ is a morphism $S\to S$. Then consider the following diagram:
	% https://q.uiver.app/#q=WzAsOSxbMiwwLCJTXmEiXSxbNCwwLCJTXmFcXG90aW1lcyBTIl0sWzQsMiwiU1xcb3RpbWVzIFMiXSxbMiwyLCJTIl0sWzIsMSwiUyJdLFswLDAsIlNcXG90aW1lcyBTXmEiXSxbMCwyLCJTXFxvdGltZXMgUyJdLFsxLDEsIlNcXG90aW1lcyBTIl0sWzMsMSwiU1xcb3RpbWVzIFMiXSxbMCwxLCJcXHBoaV97YSwwfT1cXHJob197U15hfV57LTF9Il0sWzIsMywiXFxwaGleey0xfV97MCwwfT1cXGxhbWJkYV9TIl0sWzAsNCwieSIsMl0sWzQsMywieCIsMl0sWzEsMiwieFxcb3RpbWVzIHkiXSxbMCw1LCJcXHBoaV97MCxhfT1cXGxhbWJkYV97U15hfV57LTF9IiwyXSxbNSw2LCJ5XFxvdGltZXMgeCIsMl0sWzYsMywiXFxwaGleey0xfV97MCwwfT1cXHJob19TIiwyXSxbNSw3LCJTXFxvdGltZXMgeSIsMV0sWzcsNiwieFxcb3RpbWVzIFMiLDFdLFsxLDgsInlcXG90aW1lcyBTIiwxXSxbOCwyLCJTXFxvdGltZXMgeCIsMV0sWzcsNCwiXFxsYW1iZGFfUz1cXHJob19TIl0sWzgsNCwiXFxyaG9fUz1cXGxhbWJkYV9TIiwyXV0=
	\[\begin{tikzcd}
		{S\otimes S^a} && {S^a} && {S^a\otimes S} \\
		& {S\otimes S} & S & {S\otimes S} \\
		{S\otimes S} && S && {S\otimes S}
		\arrow["{\phi_{a,0}=\rho_{S^a}^{-1}}", from=1-3, to=1-5]
		\arrow["{\phi^{-1}_{0,0}=\lambda_S}", from=3-5, to=3-3]
		\arrow["y"', from=1-3, to=2-3]
		\arrow["x"', from=2-3, to=3-3]
		\arrow["{x\otimes y}", from=1-5, to=3-5]
		\arrow["{\phi_{0,a}=\lambda_{S^a}^{-1}}"', from=1-3, to=1-1]
		\arrow["{y\otimes x}"', from=1-1, to=3-1]
		\arrow["{\phi^{-1}_{0,0}=\rho_S}"', from=3-1, to=3-3]
		\arrow["{S\otimes y}"{description}, from=1-1, to=2-2]
		\arrow["{x\otimes S}"{description}, from=2-2, to=3-1]
		\arrow["{y\otimes S}"{description}, from=1-5, to=2-4]
		\arrow["{S\otimes x}"{description}, from=2-4, to=3-5]
		\arrow["{\lambda_S=\rho_S}", from=2-2, to=2-3]
		\arrow["{\rho_S=\lambda_S}"', from=2-4, to=2-3]
	\end{tikzcd}\]
	The trapezoids commute by naturality of the unitors, and the triangles commute by functoriality of $-\otimes-$. The outside compositions are $y\cdot x$ on the left and $x\cdot y$ on the right, and the middle composition is $x\circ y$, so indeed we have $y\cdot x=x\cdot y=x\circ y$, as desired.
\end{proof}

\begin{lemma}\label{theta_a,0=theta_0,a=id_S}
	Given $a\in A$, we have $\theta_{0,a}=\theta_{a,0}=\id_S$.
\end{lemma}
\begin{proof}
	Recall $\theta_{a,0}$ is the composition
	\[S\xr{\phi_{-a,a}} S^{-a}\otimes S^a\xr{S^{-a}\otimes\phi_{a,0}} S^{-a}\otimes(S^a\otimes S)\xr{S^{-a}\otimes\tau}S^{-a}\otimes(S\otimes S^a)\xr{S^{-a}\otimes\phi_{0,a}^{-1}} S^{-a}\otimes S^a\xr{\phi_{-a,a}^{-1}}S\]
	By the coherence theorem for symmetric monoidal categories and the fact that $\phi_{a,0}$ and $\phi_{0,a}$ coincide with the unitors, we have that the composition
	\[S^a\xr{\phi_{a,0}=\rho_{S^a}^{-1}} S^a\otimes S\xr\tau S\otimes S^a\xr{\phi_{0,a}^{-1}=\lambda_{S^a}}S^a\]
	is precisely the identity map, so by functoriality of $-\otimes-$, we have that $\theta_{a,0}$ is the composition
	\[S\xr{\phi_{-a,a}}S^{-a}\otimes S^a\xr=S^{-a}\otimes S^{a}\xr{\phi_{-a,a}^{-1}}S,\]
	so $\theta_{a,0}=\id_S$, meaning
	\[x\cdot y=y\cdot x\cdot(e\circ\theta_{a,0})=y\cdot x\cdot e=y\cdot x,\]
	where the last equality follows by the fact that $e$ is the unit for the multiplication on $\pi_\ast(E)$. An entirely analagous argument yields that $\theta_{0,a}=\id_S$.
\end{proof}

\begin{lemma}\label{theta_ab.theta_ba=id}
	Let $a,b\in A$. Then $\theta_{a,b}\cdot\theta_{b,a}=\id_S$.
\end{lemma}
\begin{proof}
	By \autoref{multipy_by_degree_0_is_same_as_compose}, it suffices to show that $\theta_{a,b}\circ\theta_{b,a}=\id_S$. To see this, consider the following diagram:
	% https://q.uiver.app/#q=WzAsNyxbMCwwLCJTIl0sWzEsMCwiU157LWEtYn1cXG90aW1lcyBTXmJcXG90aW1lcyBTXmEiXSxbMiwwLCJTXnstYS1ifVxcb3RpbWVzIFNeYVxcb3RpbWVzIFNeYiJdLFszLDAsIlMiXSxbMywxLCJTXnstYS1ifVxcb3RpbWVzIFNeYVxcb3RpbWVzIFNeYiJdLFszLDIsIlNeey1hLWJ9XFxvdGltZXMgU15iXFxvdGltZXMgU15hIl0sWzMsMywiUyJdLFswLDEsIlxccGhpIl0sWzEsMiwiU157LWEtYn1cXG90aW1lcyBcXHRhdSJdLFsyLDMsIlxccGhpIl0sWzMsNCwiXFxwaGkiXSxbNCw1LCJTXnstYS1ifVxcb3RpbWVzIFxcdGF1Il0sWzUsNiwiXFxwaGkiXSxbMiw0LCIiLDEseyJsZXZlbCI6Miwic3R5bGUiOnsiaGVhZCI6eyJuYW1lIjoibm9uZSJ9fX1dLFswLDYsIiIsMix7ImxldmVsIjoyLCJzdHlsZSI6eyJoZWFkIjp7Im5hbWUiOiJub25lIn19fV0sWzEsNSwiIiwxLHsibGV2ZWwiOjIsInN0eWxlIjp7ImhlYWQiOnsibmFtZSI6Im5vbmUifX19XV0=
	\[\begin{tikzcd}
		S & {S^{-a-b}\otimes S^b\otimes S^a} & {S^{-a-b}\otimes S^a\otimes S^b} & S \\
		&&& {S^{-a-b}\otimes S^a\otimes S^b} \\
		&&& {S^{-a-b}\otimes S^b\otimes S^a} \\
		&&& S
		\arrow["\phi", from=1-1, to=1-2]
		\arrow["{S^{-a-b}\otimes \tau}", from=1-2, to=1-3]
		\arrow["\phi", from=1-3, to=1-4]
		\arrow["\phi", from=1-4, to=2-4]
		\arrow["{S^{-a-b}\otimes \tau}", from=2-4, to=3-4]
		\arrow["\phi", from=3-4, to=4-4]
		\arrow[Rightarrow, no head, from=1-3, to=2-4]
		\arrow[Rightarrow, no head, from=1-1, to=4-4]
		\arrow[Rightarrow, no head, from=1-2, to=3-4]
	\end{tikzcd}\]
	Here we are suppressing associators, and any map labelled $\phi$ is the appropriate composition of $\phi_{a,b}$'s, unitors, associators, identities, and their inverses (see \autoref{unique_comp_Sas}). Clearly each region commutes, the middle by the fact that $\tau^2=0$, and the other two regions by coherence for the $\phi$'s. Thus we have shwon $\theta_{a,b}\cdot\theta_{b,a}=\theta_{a,b}\cdot\theta_{b,a}=\id_S$, as desired.
\end{proof}

\begin{lemma}\label{theta_ab.theta_ac=theta_ab+c_and_theta_ba.theta_ca=theta_b+ca}
	Let $a,b,c\in A$. Then $\theta_{a,b}\cdot\theta_{a,c}=\theta_{a,b+c}$ and $\theta_{b,a}\cdot\theta_{c,a}=\theta_{b+c,a}$.
\end{lemma}
\begin{proof}
	By \autoref{multipy_by_degree_0_is_same_as_compose}, it suffices to show that $\theta_{a,b}\circ\theta_{a,c}=\theta_{a,b+c}$ and $\theta_{b,a}\circ\theta_{c,a}=\theta_{b+c,a}$. First we show $\theta_{a,b}\circ\theta_{a,c}=\theta_{a,b+c}$. To see this, consider the following diagram:
	% https://q.uiver.app/#q=WzAsMjQsWzAsMCwiUyJdLFsyLDAsIlNeey1hLWN9U15hU15jIl0sWzQsMCwiU157LWEtY31TXmNTXmEiXSxbNiwwLCJTIl0sWzYsMiwiU157LWEtYn1TXmFTXmIiXSxbNiw0LCJTXnstYS1ifVNeYlNeYSJdLFs2LDgsIlMiXSxbMiwyLCJTXnstYS1jfVNeey1ifVNeYVNeYlNeYyJdLFswLDIsIlNeey1hLWItY31TXmFTXntiK2N9Il0sWzAsNCwiU157LWEtYi1jfVNee2IrY31TXmEiXSxbNCwyLCJTXnstYS1jfVNeY1Neey1ifVNeYVNeYiJdLFs0LDQsIlNeey1hLWN9U15jU157LWJ9U15iU15hIl0sWzIsNCwiU157LWEtY31TXnstYn1TXmJTXmNTXmEiXSxbMCw4LCJTIl0sWzIsNiwiU157LWEtY31TXmNTXmEiXSxbNCw2LCJTXnstYS1jfVNeY1NeYSJdLFsxLDEsIihcXHRleHQgQSkiXSxbMywxLCIoXFx0ZXh0IEIpIl0sWzUsMSwiKFxcdGV4dCBDKSJdLFsxLDMsIihcXHRleHQgRCkiXSxbMywzLCIoXFx0ZXh0IEUpIl0sWzUsMywiKFxcdGV4dCBGKSJdLFszLDUsIihcXHRleHQgRykiXSxbMyw3LCIoXFx0ZXh0IEgpIl0sWzAsMSwiXFxwaGkiXSxbMSwyLCJTXnstYS1jfVxcdGF1Il0sWzIsMywiXFxwaGkiXSxbMyw0LCJcXHBoaSJdLFs0LDUsIlNeey1hLWJ9XFx0YXUiXSxbNSw2LCJcXHBoaSJdLFsxLDcsIlxccGhpIiwyXSxbMCw4LCJcXHBoaSIsMl0sWzgsOSwiU157LWEtYi1jfVxcdGF1IiwyXSxbNywxMCwiU157LWEtY31cXHRhdV97U157LWJ9U15hU15iLFNeY30iXSxbMiwxMCwiXFxwaGkiXSxbMTAsMTEsIlNeey1hLWN9U15jU157LWJ9XFx0YXUiLDFdLFs0LDEwLCJcXHBoaSIsMl0sWzUsMTEsIlxccGhpIiwyXSxbOCw3LCJcXHBoaSJdLFs5LDEyLCJcXHBoaSJdLFs3LDEyLCJTXnstYS1jfVNeey1ifVxcdGF1X3tTXmEsU15iU15jfSIsMV0sWzEyLDExLCJTXnstYS1jfVxcdGF1X3tTXnstYn1TXmIsU15jfVNeYSJdLFs5LDEzLCJcXHBoaSIsMl0sWzEzLDYsIiIsMix7ImxldmVsIjoyLCJzdHlsZSI6eyJoZWFkIjp7Im5hbWUiOiJub25lIn19fV0sWzE0LDEyLCJcXHBoaSJdLFsxNCwxNSwiIiwyLHsibGV2ZWwiOjIsInN0eWxlIjp7ImhlYWQiOnsibmFtZSI6Im5vbmUifX19XSxbMTUsMTEsIlxccGhpIiwyXV0=
	\begin{equation}\label{theta_ab_o_theta_ac}
		\begin{tikzcd}[sep=tiny]
			S && {S^{-a-c}S^aS^c} && {S^{-a-c}S^cS^a} && S \\
			& {(\text A)} && {(\text B)} && {(\text C)} \\
			{S^{-a-b-c}S^aS^{b+c}} && {S^{-a-c}S^{-b}S^aS^bS^c} && {S^{-a-c}S^cS^{-b}S^aS^b} && {S^{-a-b}S^aS^b} \\
			& {(\text D)} && {(\text E)} && {(\text F)} \\
			{S^{-a-b-c}S^{b+c}S^a} && {S^{-a-c}S^{-b}S^bS^cS^a} && {S^{-a-c}S^cS^{-b}S^bS^a} && {S^{-a-b}S^bS^a} \\
			&&& {(\text G)} \\
			&& {S^{-a-c}S^cS^a} && {S^{-a-c}S^cS^a} \\
			&&& {(\text H)} \\
			S &&&&&& S
			\arrow["\phi", from=1-1, to=1-3]
			\arrow["{S^{-a-c}\tau}", from=1-3, to=1-5]
			\arrow["\phi", from=1-5, to=1-7]
			\arrow["\phi", from=1-7, to=3-7]
			\arrow["{S^{-a-b}\tau}", from=3-7, to=5-7]
			\arrow["\phi", from=5-7, to=9-7]
			\arrow["\phi"', from=1-3, to=3-3]
			\arrow["\phi"', from=1-1, to=3-1]
			\arrow["{S^{-a-b-c}\tau}"', from=3-1, to=5-1]
			\arrow["{S^{-a-c}\tau_{S^{-b}S^aS^b,S^c}}", from=3-3, to=3-5]
			\arrow["\phi", from=1-5, to=3-5]
			\arrow["{S^{-a-c}S^cS^{-b}\tau}"{description}, from=3-5, to=5-5]
			\arrow["\phi"', from=3-7, to=3-5]
			\arrow["\phi"', from=5-7, to=5-5]
			\arrow["\phi", from=3-1, to=3-3]
			\arrow["\phi", from=5-1, to=5-3]
			\arrow["{S^{-a-c}S^{-b}\tau_{S^a,S^bS^c}}"{description}, from=3-3, to=5-3]
			\arrow["{S^{-a-c}\tau_{S^{-b}S^b,S^c}S^a}", from=5-3, to=5-5]
			\arrow["\phi"', from=5-1, to=9-1]
			\arrow[Rightarrow, no head, from=9-1, to=9-7]
			\arrow["\phi", from=7-3, to=5-3]
			\arrow[Rightarrow, no head, from=7-3, to=7-5]
			\arrow["\phi"', from=7-5, to=5-5]
		\end{tikzcd}
	\end{equation}
	Here we are omitting $\otimes$ from the notation, and each occurrence of an arrow labelled $\phi$ indicates it is the unique arrow that can be obtained as a formal composition of tensor products of copies of $\phi_{a,b}$'s, unitors, associators, and their inverses (\autoref{unique_comp_Sas}). Clearly the composition going around the top and then the right is $\theta_{a,b}\circ\theta_{a,c}$ while the composition going left around the bottom is $\theta_{a,b+c}$. Thus, we wish to show the above diagram commutes.
	
	Regions $(\text A)$, $(\text C)$, and $(\text H)$ commute by coherence for the $\phi$'s (see previous remark). Region $(\text E)$ commutes by coherence for the $\tau$'s. To see region $(\text B)$ commutes, consider the following diagram, which commutes by naturality of $\tau$:
	% https://q.uiver.app/#q=WzAsNixbMCwwLCJTXnstYS1jfVNeYSBTXmMiXSxbMiwwLCJTXnstYS1jfVNeY1NeYSJdLFswLDIsIlNeey1hLWN9U157YS1ifVNeYlNeYyJdLFsyLDIsIlNeey1hLWN9U15jU157YS1ifVNeYiJdLFsyLDQsIlNeey1hLWN9U15jU157LWJ9U157YX1TXmIiXSxbMCw0LCJTXnstYS1jfVNeey1ifVNeYVNeYlNeYyJdLFswLDEsIlNeey1hLWN9XFx0YXUiXSxbMCwyLCJTXnstYS1jfVxccGhpX3thLWIsYn1TXmMiLDJdLFsxLDMsIlNeey1hLWN9U15jXFxwaGlfe2EtYixifSJdLFsyLDMsIlNeey1hLWN9XFx0YXVfe1Nee2EtYn1TXmIsU15jfSJdLFszLDQsIlNeey1hLWN9U15jXFxwaGlfey1iLGF9U15iIl0sWzIsNSwiU157LWEtY31cXHBoaV97LWIsYX1TXmJTXmMiLDJdLFs1LDQsIlNeey1hLWN9XFx0YXVfe1Neey1ifVNeYVNeYixTXmN9Il1d
	\[\begin{tikzcd}
		{S^{-a-c}S^a S^c} && {S^{-a-c}S^cS^a} \\
		\\
		{S^{-a-c}S^{a-b}S^bS^c} && {S^{-a-c}S^cS^{a-b}S^b} \\
		\\
		{S^{-a-c}S^{-b}S^aS^bS^c} && {S^{-a-c}S^cS^{-b}S^{a}S^b}
		\arrow["{S^{-a-c}\tau}", from=1-1, to=1-3]
		\arrow["{S^{-a-c}\phi_{a-b,b}S^c}"', from=1-1, to=3-1]
		\arrow["{S^{-a-c}S^c\phi_{a-b,b}}", from=1-3, to=3-3]
		\arrow["{S^{-a-c}\tau_{S^{a-b}S^b,S^c}}", from=3-1, to=3-3]
		\arrow["{S^{-a-c}S^c\phi_{-b,a}S^b}", from=3-3, to=5-3]
		\arrow["{S^{-a-c}\phi_{-b,a}S^bS^c}"', from=3-1, to=5-1]
		\arrow["{S^{-a-c}\tau_{S^{-b}S^aS^b,S^c}}", from=5-1, to=5-3]
	\end{tikzcd}\]
	To see region $(\text D)$ commutes, note that it is simply the square
	% https://q.uiver.app/#q=WzAsNCxbMiwwLCJTXnstYS1jfVNeey1ifVNeYVNeYlNeYyJdLFswLDAsIlNeey1hLWItY31TXmFTXntiK2N9Il0sWzAsMiwiU157LWEtYi1jfVNee2IrY31TXmEiXSxbMiwyLCJTXnstYS1jfVNeey1ifVNeYlNeY1NeYSJdLFsxLDIsIlNeey1hLWItY31cXHRhdSIsMl0sWzEsMCwiXFxwaGlfey1hLWMsLWJ9U15hXFxwaGlfe2IsY30iXSxbMiwzLCJcXHBoaV97LWEtYywtYn1cXHBoaV97YixjfVNeYSJdLFswLDMsIlNeey1hLWN9U157LWJ9XFx0YXVfe1NeYSxTXmJTXmN9IiwxXV0=
	\[\begin{tikzcd}
		{S^{-a-b-c}S^aS^{b+c}} && {S^{-a-c}S^{-b}S^aS^bS^c} \\
		\\
		{S^{-a-b-c}S^{b+c}S^a} && {S^{-a-c}S^{-b}S^bS^cS^a}
		\arrow["{S^{-a-b-c}\tau}"', from=1-1, to=3-1]
		\arrow["{\phi_{-a-c,-b}S^a\phi_{b,c}}", from=1-1, to=1-3]
		\arrow["{\phi_{-a-c,-b}\phi_{b,c}S^a}", from=3-1, to=3-3]
		\arrow["{S^{-a-c}S^{-b}\tau_{S^a,S^bS^c}}"{description}, from=1-3, to=3-3]
	\end{tikzcd}\]
	This diagram commutes by naturality of $\tau$. To see region $(\text F)$ commutes, consider the following diagram, which commutes by functoriality of $-\otimes-$:
	% https://q.uiver.app/#q=WzAsNixbNCwwLCJTXnstYS1ifVNeYVNeYiJdLFs0LDIsIlNeey1hLWJ9U15iU15hIl0sWzAsMCwiU157LWEtY31TXmNTXnstYn1TXmFTXmIiXSxbMiwyLCJTXnstYS1jfVNee2MtYn1TXmJTXmEiXSxbMiwwLCJTXnstYS1jfVNee2MtYn1TXmFTXmIiXSxbMCwyLCJTXnstYS1jfVNeY1Neey1ifVNeYlNeYSJdLFswLDEsIlNeey1hLWJ9XFx0YXUiXSxbMSwzLCJcXHBoaV97LWEtYyxjLWJ9U15iU15hIiwyXSxbMCw0LCJcXHBoaV97LWEtYyxjLWJ9U15hU15iIiwyXSxbNCwzLCJTXnstYS1jfVNee2MtYn1cXHRhdSJdLFs0LDIsIlNeey1hLWN9XFxwaGlfe2MsLWJ9U15hU15iIiwyXSxbMiw1LCJTXnstYS1jfVNeY1Neey1ifVxcdGF1IiwyXSxbMyw1LCJTXnstYS1jfVxccGhpX3tjLC1ifVNeYlNeYSIsMl1d
	\[\begin{tikzcd}
		{S^{-a-c}S^cS^{-b}S^aS^b} && {S^{-a-c}S^{c-b}S^aS^b} && {S^{-a-b}S^aS^b} \\
		\\
		{S^{-a-c}S^cS^{-b}S^bS^a} && {S^{-a-c}S^{c-b}S^bS^a} && {S^{-a-b}S^bS^a}
		\arrow["{S^{-a-b}\tau}", from=1-5, to=3-5]
		\arrow["{\phi_{-a-c,c-b}S^bS^a}"', from=3-5, to=3-3]
		\arrow["{\phi_{-a-c,c-b}S^aS^b}"', from=1-5, to=1-3]
		\arrow["{S^{-a-c}S^{c-b}\tau}", from=1-3, to=3-3]
		\arrow["{S^{-a-c}\phi_{c,-b}S^aS^b}"', from=1-3, to=1-1]
		\arrow["{S^{-a-c}S^cS^{-b}\tau}"', from=1-1, to=3-1]
		\arrow["{S^{-a-c}\phi_{c,-b}S^bS^a}"', from=3-3, to=3-1]
	\end{tikzcd}\]
	Finally, to see region $(\text G)$ commutes, consider the following diagram:
	% https://q.uiver.app/#q=WzAsNixbMCwwLCJTXnstYS1jfVNeey1ifVNeYlNeY1NeYSJdLFswLDEsIlNeey1hLWN9U1NeY1NeYSJdLFsxLDEsIlNeey1hLWN9U15jU1NeYSJdLFsxLDAsIlNeey1hLWN9U15jU157LWJ9U15iU15hIl0sWzAsMiwiU157LWEtY31TXmNTXmEiXSxbMSwyLCJTXnstYS1jfVNeY1NeYSJdLFswLDMsIlNeey1hLWN9XFx0YXVfe1Neey1ifVNeYixTXmN9U15hIl0sWzEsMCwiU157LWEtY31cXHBoaV97LWIsYn1TXmNTXmEiXSxbMiwzLCJTXnstYS1jfVNeY1xccGhpX3stYixifVNeYSIsMl0sWzEsMiwiU157LWEtY31cXHRhdV97UyxTXmN9IFNeYSJdLFs0LDEsIlNeey1hLWN9XFxwaGlfezAsY31TXmE9U157LWEtY31cXGxhbWJkYV97U15jfV57LTF9U15hIl0sWzUsMiwiU157LWEtY31cXHBoaV97YywwfVNeYT1TXnstYS1jfVNcXHJob197U15jfV57LTF9U15hIiwyXSxbNCw1LCIiLDEseyJsZXZlbCI6Miwic3R5bGUiOnsiaGVhZCI6eyJuYW1lIjoibm9uZSJ9fX1dXQ==
	\[\begin{tikzcd}
		{S^{-a-c}S^{-b}S^bS^cS^a} & {S^{-a-c}S^cS^{-b}S^bS^a} \\
		{S^{-a-c}SS^cS^a} & {S^{-a-c}S^cSS^a} \\
		{S^{-a-c}S^cS^a} & {S^{-a-c}S^cS^a}
		\arrow["{S^{-a-c}\tau_{S^{-b}S^b,S^c}S^a}", from=1-1, to=1-2]
		\arrow["{S^{-a-c}\phi_{-b,b}S^cS^a}", from=2-1, to=1-1]
		\arrow["{S^{-a-c}S^c\phi_{-b,b}S^a}"', from=2-2, to=1-2]
		\arrow["{S^{-a-c}\tau_{S,S^c} S^a}", from=2-1, to=2-2]
		\arrow["{S^{-a-c}\phi_{0,c}S^a=S^{-a-c}\lambda_{S^c}^{-1}S^a}", from=3-1, to=2-1]
		\arrow["{S^{-a-c}\phi_{c,0}S^a=S^{-a-c}S\rho_{S^c}^{-1}S^a}"', from=3-2, to=2-2]
		\arrow[Rightarrow, no head, from=3-1, to=3-2]
	\end{tikzcd}\]
	The top region commutes by naturality of $\tau$, while the bottom region commutes by coherence for a symmetric monoidal category. Thus, we have shown that diagram (\ref{theta_ab_o_theta_ac}) commutes, so that $\theta_{a,b}\circ\theta_{a,c}=\theta_{a,b+c}$, as desired. Now, to see that $\theta_{b,a}\cdot\theta_{c,a}=\theta_{b+c,a}$, note that
	\[\theta_{b,a}\cdot\theta_{c,a}\overset{(\ast)}=\theta_{a,b}^{-1}\cdot\theta_{a,c}^{-1}=(\theta_{a,c}\cdot\theta_{a,b})^{-1}=\theta_{a,b+c}^{-1}\overset{(\ast)}=\theta_{b+c,a},\]
	where each occurrence of $(\ast)$ is \autoref{theta_ab.theta_ba=id}.
\end{proof}

%\begin{proposition}\label{theta_a,0=theta_0,a=id_S}
	%Given $a\in A$, we have $\theta_{0,a}=\theta_{a,0}=\id_S$.
%\end{proposition}
%\begin{proof}
	%Recall $\theta_{a,0}$ is the composition
	%\[S\xr{\phi_{-a,a}} S^{-a}\otimes S^a\xr{S^{-a}\otimes\phi_{a,0}} S^{-a}\otimes(S^a\otimes S)\xr{S^{-a}\otimes\tau}S^{-a}\otimes(S\otimes S^a)\xr{S^{-a}\otimes\phi_{0,a}^{-1}} S^{-a}\otimes S^a\xr{\phi_{-a,a}^{-1}}S\]
	%By the coherence theorem for symmetric monoidal categories and the fact that $\phi_{a,0}$ and $\phi_{0,a}$ coincide with the unitors, we have that the composition
	%\[S^a\xr{\phi_{a,0}=\rho_{S^a}^{-1}} S^a\otimes S\xr\tau S\otimes S^a\xr{\phi_{0,a}^{-1}=\lambda_{S^a}}S^a\]
	%is precisely the identity map, so by functoriality of $-\otimes-$, we have that $\theta_{a,0}$ is the composition
	%\[S\xr{\phi_{-a,a}}S^{-a}\otimes S^a\xr=S^{-a}\otimes S^{a}\xr{\phi_{-a,a}^{-1}}S,\]
	%so $\theta_{a,0}=\id_S$, meaning
	%\[x\cdot y=y\cdot x\cdot(e\circ\theta_{a,0})=y\cdot x\cdot e=y\cdot x,\]
	%where the last equality follows by the fact that $e$ is the unit for the multiplication on $\pi_\ast(E)$. An entirely analagous argument yields that $\theta_{0,a}=\id_S$.
%\end{proof}

\begin{lemma}\label{bilinear}
	Let $X$ and $Y$ be objects in $\cSH$. Then the $A$-graded pairing
	\[\pi_*(X)\times\pi_*(Y)\to\pi_*(X\otimes Y)\]
	sending $x:S^a\to X$ and $ y:S^b\to Y$ to the composition
	\[S^{a+b}\xr{\phi_{a,b}} S^a\otimes S^b\xr{x\otimes y}X\otimes Y\]
	is additive in each argument.
\end{lemma}
\begin{proof}
	Let $a,b\in A$, and let $x_1,x_2:S^a\to X$ and $ y:S^b\to Y$. Then consider the following diagram
	\[\begin{tikzcd}
		{S^{a+b}} & {S^a\otimes S^b} & {(S^a\oplus S^a)\otimes S^b} \\
		& {(S^a\otimes S^b)\oplus(S^a\otimes S^b)} & {(X\oplus X)\otimes Y} \\
		& {(X\otimes Y)\oplus(X\otimes Y)} & {X\otimes Y}
		\arrow["{\Delta\otimes S^b}", from=1-2, to=1-3]
		\arrow["\Delta"', from=1-2, to=2-2]
		\arrow["{( x_1\oplus x_2)\otimes y}", from=1-3, to=2-3]
		\arrow["{\nabla\otimes Y}", from=2-3, to=3-3]
		\arrow["{( x_1\otimes y)\oplus( x_2\otimes y)}"', from=2-2, to=3-2]
		\arrow["\nabla", from=3-2, to=3-3]
		\arrow["\cong"', from=1-3, to=2-2]
		\arrow["\cong"', from=2-3, to=3-2]
		\arrow["\cong", from=1-1, to=1-2]
	\end{tikzcd}\]
	The isomorphisms are given by the fact that $-\otimes-$ is additive in each variable. Both triangles and the parallelogram commute since $-\otimes-$ is additive. By functoriality of $-\otimes-$, the top composition is $( x_1+ x_2)\cdot y$ and the bottom composition is $ x_1\cdot y+ x_2\cdot y$, so they are equal, as desired. An entirely analagous argument yields that $ x\cdot( y_1+ y_2)= x\cdot y_1+ x\cdot y_2$ for $ x\in\pi_*(X)$ and $ y_1, y_2\in\pi_*(Y)$.
\end{proof}

\begin{proposition}\label{module}
	Let $(E,\mu,e)$ be a monoid object in $\cSH$. Then $E_*(-)$ is an additive functor from $\cSH$ to the category $\pi_*(E)\text-\Mod(A)$ of left $A$-graded modules over the ring $\pi_*(E)$ (\autoref{pi_*E_is_ring_for_E_monoid_appendix}) and degree-preserving homomorphisms between them, where given some $X$ in $\cSH$, $E_*(X)$ may be endowed with the structure of a left $A$-graded $\pi_*(E)$-module via the map 
	\[\pi_*(E)\times E_*(X)\to E_*(X)\]
	which given $a,b\in A$, sends $x:S^a\to E$ and $y:S^b\to E\otimes X$ to the composition
	\[x\cdot y:S^{a+b}\cong S^a\otimes S^b\xr{x\otimes y}E\otimes (E\otimes X)\cong (E\otimes E)\otimes X\xr{\mu\otimes X}E\otimes X.\]
	Similarly, the assignment $X\mapsto X_*(E)$ is a functor from $\cSH$ to right $A$-graded $\pi_*(E)$-modules, where the structure map
	\[X_*(E)\times\pi_*(E)\to X_*(E)\]
	sends $x:S^a\to X\otimes E$ and $y:S^b\to E$ to the composition
	\[x\cdot y:S^{a+b}\cong S^a\otimes S^b\xr{x\otimes y}(X\otimes E)\otimes E\cong X\otimes(E\otimes E)\xr{X\otimes\mu}X\otimes E.\]
	Finally, $E_*(E)$ is a $\pi_*(E)$-bimodule, in the sense that the left and right actions of $\pi_*(E)$ are compatible, so that given $y, z\in\pi_*(E)$ and $x\in E_*(E)$, $y\cdot(x\cdot z)=(y\cdot x)\cdot z$.
\end{proposition}
\begin{proof}
	By \autoref{A-graded_module}, in order to make the $A$-graded abelian group $E_*(X)$ into a left $A$-graded module over the $A$-graded ring $\pi_*(E)$, it suffices to define the action map $\pi_*(E)\times E_*(X)\to E_*(X)$ only for homogeneous elements, and to show that given homogeneous elements $x,x':S^a\to E\otimes X$ in $E_a(X)$, $y:S^b\to E$ in $\pi_b(E)$, and $z, z':S^c\to E$ in $\pi_c(E)$, that:
	\begin{enumerate}
		\item $y\cdot(x+x')=y\cdot x+y\cdot x'$, 
		\item $(z+ z')\cdot x=z\cdot x+z'\cdot x$,
		\item $(zy)\cdot x=z\cdot(y\cdot x)$,
		\item $e\cdot x=x$.
	\end{enumerate}
	Axioms $(1)$ and $(2)$ follow by the fact that $E_*(X)=\pi_*(E\otimes X)$ and \autoref{bilinear}. To see $(3)$, consider the diagram:
	% https://q.uiver.app/#q=WzAsNixbMCwxLCJTXnthK2IrY30iXSxbMSwxLCJTXmNcXG90aW1lcyBTXmJcXG90aW1lcyBTXmEiXSxbMiwxLCJFXFxvdGltZXMgRVxcb3RpbWVzIEVcXG90aW1lcyBYIl0sWzMsMiwiRVxcb3RpbWVzIEVcXG90aW1lcyBYIl0sWzMsMCwiRVxcb3RpbWVzIEVcXG90aW1lcyBYIl0sWzMsMSwiRVxcb3RpbWVzIFgiXSxbMCwxLCJcXGNvbmciXSxbMSwyLCJ6XFxvdGltZXMgeVxcb3RpbWVzIHgiXSxbMiwzLCJcXG11XFxvdGltZXMgRVxcb3RpbWVzIFgiLDJdLFsyLDQsIkVcXG90aW1lc1xcbXVcXG90aW1lcyBYIl0sWzQsNSwiXFxtdVxcb3RpbWVzIFgiXSxbMyw1LCJcXG11XFxvdGltZXMgWCIsMl1d
	\[\begin{tikzcd}
		&&& {E\otimes E\otimes X} \\
		{S^{a+b+c}} & {S^c\otimes S^b\otimes S^a} & {E\otimes E\otimes E\otimes X} & {E\otimes X} \\
		&&& {E\otimes E\otimes X}
		\arrow["\cong", from=2-1, to=2-2]
		\arrow["{z\otimes y\otimes x}", from=2-2, to=2-3]
		\arrow["{\mu\otimes E\otimes X}"', from=2-3, to=3-4]
		\arrow["{E\otimes\mu\otimes X}", from=2-3, to=1-4]
		\arrow["{\mu\otimes X}", from=1-4, to=2-4]
		\arrow["{\mu\otimes X}"', from=3-4, to=2-4]
	\end{tikzcd}\]
	It commutes by associativity of $\mu$. By functoriality of $-\otimes-$, the two outside compositions equal $z\cdot(y\cdot x)$ on the top and $(z\cdot y)\cdot x$ on the bottom. Hence, they are equal, as desired.

	Next, to see $(4)$, consider the following diagram:
	% https://q.uiver.app/#q=WzAsNCxbMCwwLCJTXmEiXSxbMSwxLCJFXFxvdGltZXMgWCJdLFsyLDAsIkVcXG90aW1lcyAgWCJdLFsxLDIsIkVcXG90aW1lcyBFXFxvdGltZXMgWCJdLFsxLDIsIiIsMSx7ImxldmVsIjoyLCJzdHlsZSI6eyJoZWFkIjp7Im5hbWUiOiJub25lIn19fV0sWzEsMywiZVxcb3RpbWVzIEVcXG90aW1lcyBYIiwxXSxbMCwyLCJ4Il0sWzMsMiwiXFxtdVxcb3RpbWVzIFgiLDIseyJjdXJ2ZSI6M31dLFswLDEsIngiLDJdLFswLDMsImVcXG90aW1lcyB4IiwyLHsiY3VydmUiOjN9XV0=
	\[\begin{tikzcd}
		{S^a} && {E\otimes  X} \\
		& {E\otimes X} \\
		& {E\otimes E\otimes X}
		\arrow[Rightarrow, no head, from=2-2, to=1-3]
		\arrow["{e\otimes E\otimes X}"{description}, from=2-2, to=3-2]
		\arrow["x", from=1-1, to=1-3]
		\arrow["{\mu\otimes X}"', curve={height=18pt}, from=3-2, to=1-3]
		\arrow["x"', from=1-1, to=2-2]
		\arrow["{e\otimes x}"', curve={height=18pt}, from=1-1, to=3-2]
	\end{tikzcd}\]
	The top triangle commutes by definition. The left triangle commutes by functoriality of $-\otimes-$. The right triangle commutes by unitality of $\mu$.
	The top composition is $ x$ while the bottom is $e\cdot x$, thus they are necessarily equal since the diagram commutes.

	Thus, we have shown that the indicated map does indeed endow $E_*(X)$ with the structure of a left $\pi_*(E)$-module. Next we would like to show that $E_*(-)$ sends maps in $\cSH$ to $A$-graded homomorphisms of left $A$-graded $\pi_*(E)$-modules. By definition, given $f:X\to Y$ in $\cSH$, $E_*(f)$ is the map which takes a class $x:S^a\to E\otimes X$ to the composition 
	\[S^a\xrightarrow xE\otimes X\xrightarrow{E\otimes f}E\otimes Y.\]
	To see this assignment is a homomorphism, suppose we are given some other $x':S^a\to E\otimes X$ and some scalar $y:S^b\to E$. Then we would like to show $E_*(f)(x+x')=E_*(f)(x)+E_*(f)(x')$ and $E_*(f)(y\cdot x)=y\cdot E_*(f)(x)$. To see the former, consider the following diagram:
	% https://q.uiver.app/#q=WzAsNixbMCwxLCJTXmEiXSxbMSwxLCJTXmFcXG9wbHVzIFNeYSJdLFsyLDEsIihFXFxvdGltZXMgWClcXG9wbHVzKEVcXG90aW1lcyBYKSJdLFszLDAsIihFXFxvdGltZXMgWSlcXG9wbHVzKEVcXG90aW1lcyBZKSJdLFszLDEsIkVcXG90aW1lcyBZIl0sWzMsMiwiRVxcb3RpbWVzIFgiXSxbMCwxLCJcXERlbHRhIl0sWzEsMiwieFxcb3BsdXMgeCciXSxbMiwzLCIoRVxcb3RpbWVzIGYpXFxvcGx1cyAoRVxcb3RpbWVzIGYpIl0sWzMsNCwiXFxuYWJsYSJdLFsyLDUsIlxcbmFibGEiLDJdLFs1LDQsIkVcXG90aW1lcyBmIiwyXV0=
	\[\begin{tikzcd}
		&&& {(E\otimes Y)\oplus(E\otimes Y)} \\
		{S^a} & {S^a\oplus S^a} & {(E\otimes X)\oplus(E\otimes X)} & {E\otimes Y} \\
		&&& {E\otimes X}
		\arrow["\Delta", from=2-1, to=2-2]
		\arrow["{x\oplus x'}", from=2-2, to=2-3]
		\arrow["{(E\otimes f)\oplus (E\otimes f)}", from=2-3, to=1-4]
		\arrow["\nabla", from=1-4, to=2-4]
		\arrow["\nabla"', from=2-3, to=3-4]
		\arrow["{E\otimes f}"', from=3-4, to=2-4]
	\end{tikzcd}\]
	It commutes by naturality of $\nabla$ in an additive category. The top composition is $E_*(f)(x)+E_*(f)(x')$, while the bottom is $E_*(f)(x+x')$, so they are equal as desired. To see that $E_*(f)(y\cdot x)=y\cdot E_*(f)(x)$, consider the following diagram:
	% https://q.uiver.app/#q=WzAsNixbMCwwLCJTXnthK2J9Il0sWzEsMCwiU15iXFxvdGltZXMgU15hIl0sWzIsMCwiRVxcb3RpbWVzIEVcXG90aW1lcyBYIl0sWzMsMCwiRVxcb3RpbWVzIEVcXG90aW1lcyBZIl0sWzMsMSwiRVxcb3RpbWVzIFkiXSxbMiwxLCJFXFxvdGltZXMgWCJdLFswLDEsIlxccGhpX3tiLGF9Il0sWzEsMiwieVxcb3RpbWVzIHgiXSxbMiwzLCJFXFxvdGltZXMgRVxcb3RpbWVzIGYiXSxbMyw0LCJcXG11XFxvdGltZXMgWSJdLFsyLDUsIlxcbXVcXG90aW1lcyBYIiwyXSxbNSw0LCJFXFxvdGltZXMgZiJdXQ==
	\[\begin{tikzcd}
		{S^{a+b}} & {S^b\otimes S^a} & {E\otimes E\otimes X} & {E\otimes E\otimes Y} \\
		&& {E\otimes X} & {E\otimes Y}
		\arrow["{\phi_{b,a}}", from=1-1, to=1-2]
		\arrow["{y\otimes x}", from=1-2, to=1-3]
		\arrow["{E\otimes E\otimes f}", from=1-3, to=1-4]
		\arrow["{\mu\otimes Y}", from=1-4, to=2-4]
		\arrow["{\mu\otimes X}"', from=1-3, to=2-3]
		\arrow["{E\otimes f}", from=2-3, to=2-4]
	\end{tikzcd}\]
	It commutes by functoriality of $-\otimes-$. The top composition is $E_*(f)(y\cdot x)$, while the bottom composition is $y\cdot E_*(f)(x)$, so they are equal, as desired.

	Thus, we've shown $E_*(-)$ yields a functor $\cSH\to\pi_*(E)\text-\Mod(A)$; it remains to show this functor is additive. This is clear, as given $f,g:X\to Y$ in $\cSH$, we have 
	\[E_*(f+g)=[S^*,E\otimes(f+g)]=[S^*,(E\otimes f)+(E\otimes g)]=E_*(f)+E_*(g),\]
	where the second equality follows since $-\otimes-$ is additive in each variable. 
	
	Showing that $X_*(E)$ has the structure of a right $\pi_*(E)$-module and that if $f:X\to Y$ is a morphism in $\cSH$ then the map
	\[X_*(E)=[S^*,X\otimes E]\xr{(f\otimes E)_*}[S^*,Y\otimes E]=Y_*(E)\]
	is an $A$-graded homomorphism of right $A$-graded $\pi_*(E)$-modules is entirely analagous.

    It remains to show that $E_*(E)$ is a $\pi_*(E)$-bimodule. Let $x:S^a\to E$, $y:S^b\to E\otimes E$, and $z:S^c\to E$, and consider the following diagram:
	% https://q.uiver.app/#q=WzAsNixbMCwxLCJTXnthK2IrY30iXSxbMSwxLCJTXmFcXG90aW1lcyBTXmJcXG90aW1lcyBTXmMiXSxbMiwxLCJFXFxvdGltZXMgRVxcb3RpbWVzIEVcXG90aW1lcyBFIl0sWzMsMCwiRVxcb3RpbWVzIEVcXG90aW1lcyBFIl0sWzMsMiwiRVxcb3RpbWVzIEVcXG90aW1lcyBFIl0sWzMsMSwiRVxcb3RpbWVzIEUiXSxbMCwxLCJcXGNvbmciXSxbMSwyLCJ4XFxvdGltZXMgeVxcb3RpbWVzIHoiXSxbMiwzLCJcXG11XFxvdGltZXMgRVxcb3RpbWVzIEUiXSxbMiw0LCJFXFxvdGltZXMgRVxcb3RpbWVzIFxcbXUiLDJdLFsyLDUsIlxcbXVcXG90aW1lc1xcbXUiXSxbMyw1LCJFXFxvdGltZXNcXG11Il0sWzQsNSwiXFxtdVxcb3RpbWVzIEUiLDJdXQ==
	\[\begin{tikzcd}
		&&& {E\otimes E\otimes E} \\
		{S^{a+b+c}} & {S^a\otimes S^b\otimes S^c} & {E\otimes E\otimes E\otimes E} & {E\otimes E} \\
		&&& {E\otimes E\otimes E}
		\arrow["\cong", from=2-1, to=2-2]
		\arrow["{x\otimes y\otimes z}", from=2-2, to=2-3]
		\arrow["{\mu\otimes E\otimes E}", from=2-3, to=1-4]
		\arrow["{E\otimes E\otimes \mu}"', from=2-3, to=3-4]
		\arrow["\mu\otimes\mu", from=2-3, to=2-4]
		\arrow["E\otimes\mu", from=1-4, to=2-4]
		\arrow["{\mu\otimes E}"', from=3-4, to=2-4]
	\end{tikzcd}\]
	Commutativity follows by functoriality of $-\otimes-$, which also tells us that the two outside compositions are $(x\cdot y)\cdot z$ (on top) and $x\cdot(y\cdot z)$ (on bottom). Hence they are equal, as desired.
\end{proof}

\begin{lemma}\label{E_homology_suspension_iso_t^a's_appendix}
	Let $E$ and $X$ be objects in $\cSH$. Then for all $a\in A$, there is an $A$-graded isomorphism of $A$-graded abelian groups
	\[t^a_X:E_*(\Sigma^aX)\cong E_{*-a}(X)\]
	which sends a class $x:S^b\to E\otimes\Sigma^aX=E\otimes S^a\otimes X$ to the composition
	\[S^{b-a}\xr{\phi_{b,-a}}S^b\otimes S^{-a}\xr{x\otimes S^{-a}}E\otimes S^a\otimes X\otimes S^{-a}\xr{E\otimes \tau\otimes S^{-a}}E\otimes X\otimes S^a\otimes S^{-a}\xr{E\otimes X\otimes \phi_{a,-a}^{-1}}E\otimes X\]
	with inverse ${(t^a_X)}^{-1}:E_{*-a}(X)\to E_*(\Sigma^aX)$ sending a class $x:S^{b-a}\to E\otimes X$ to the composition
	\[S^b\xr{\phi_{b-a,a}}S^{b-a}\otimes S^a\xr{x\otimes S^a}E\otimes X\otimes S^a\xr{E\otimes \tau}E\otimes S^a\otimes X\]
	(where here we are suppressing associators and unitors from the notation). Furthermore this isomorphism is natural in $X$, and if $E$ is a monoid object in $\cSH$ then it is a natural isomorphism of $\pi_*(E)$-modules.

	In particular, given a monoid object $(E,\mu,e)$ in $\cSH$, the functor $E_*(-):\cSH\to\pi_*(E)\text-\Mod(A)$ is a lax $A$-graded functor (\autoref{A-graded_functor_defn}).
\end{lemma}
\begin{proof}
	Expressed in terms of hom-sets, $t^a_X$ is precisely the composition
	% https://q.uiver.app/#q=WzAsNyxbMSwwLCJbU14qLEVcXG90aW1lcyBTXmFcXG90aW1lcyBYXSJdLFsxLDEsIltTXiosRVxcb3RpbWVzIFhcXG90aW1lcyBTXmFdIl0sWzIsNCwiRV97Ki1hfShFXFxvdGltZXMgWCkiXSxbMCwwLCJFXyooXFxTaWdtYV5hWCkiXSxbMSwyLCJbU14qXFxvdGltZXMgU157LWF9LEVcXG90aW1lcyBYXFxvdGltZXMgU15hXFxvdGltZXMgU157LWF9XSJdLFsxLDMsIltTXipcXG90aW1lcyBTXnstYX0sRVxcb3RpbWVzIFhdIl0sWzEsNCwiW1NeeyotYX0sRVxcb3RpbWVzIFhdIl0sWzAsMSwieyhFXFxvdGltZXMgXFx0YXUpfV8qIl0sWzMsMCwiIiwwLHsibGV2ZWwiOjIsInN0eWxlIjp7ImhlYWQiOnsibmFtZSI6Im5vbmUifX19XSxbMSw0LCItXFxvdGltZXMgU157LWF9Il0sWzQsNSwieyhFXFxvdGltZXMgWFxcb3RpbWVzIFxccGhpX3thLC1hfV57LTF9KX1fKiJdLFs2LDIsIiIsMCx7ImxldmVsIjoyLCJzdHlsZSI6eyJoZWFkIjp7Im5hbWUiOiJub25lIn19fV0sWzUsNiwieyhcXHBoaV97KiwtYX0pfV4qIl1d
	\[\begin{tikzcd}
		{E_*(\Sigma^aX)} & {[S^*,E\otimes S^a\otimes X]} \\
		& {[S^*,E\otimes X\otimes S^a]} \\
		& {[S^*\otimes S^{-a},E\otimes X\otimes S^a\otimes S^{-a}]} \\
		& {[S^*\otimes S^{-a},E\otimes X]} \\
		& {[S^{*-a},E\otimes X]} & {E_{*-a}(E\otimes X)}
		\arrow["{{(E\otimes \tau)}_*}", from=1-2, to=2-2]
		\arrow[Rightarrow, no head, from=1-1, to=1-2]
		\arrow["{-\otimes S^{-a}}", from=2-2, to=3-2]
		\arrow["{{(E\otimes X\otimes \phi_{a,-a}^{-1})}_*}", from=3-2, to=4-2]
		\arrow[Rightarrow, no head, from=5-2, to=5-3]
		\arrow["{{(\phi_{*,-a})}^*}", from=4-2, to=5-2]
	\end{tikzcd}\]
	We know the second vertical arrow is an isomorphism of abelian groups as $-\otimes-$ is additive in each variable (since $\cSH$ is tensor triangulated) and $\Omega^a\cong -\otimes S^{-a}$ is an autoequivalence of $\cSH$ by \autoref{Sigma^a,Sigma^-a_adjoint_equiv}.  The three other vertical arrows are given by composing with an isomorphism in an additive category, so they are also isomorphisms. Now, note the proposed inverse constructed above, can be factored into the following composition of hom-sets:
	% https://q.uiver.app/#q=WzAsNixbMSwwLCIgW1NeeyotYX0sRVxcb3RpbWVzIFhdIl0sWzAsMCwiRV97Ki1hfShFXFxvdGltZXMgWCkiXSxbMSwzLCIgW1NeKixFXFxvdGltZXMgU15hXFxvdGltZXMgWF0iXSxbMiwzLCJFXyooXFxTaWdtYV5hWCkiXSxbMSwxLCJbU157Ki1hfVxcb3RpbWVzIFNeYSxFXFxvdGltZXMgWFxcb3RpbWVzIFNeYV0iXSxbMSwyLCJbU14qLEVcXG90aW1lcyBYXFxvdGltZXMgU15hXSJdLFswLDEsIiIsMCx7ImxldmVsIjoyLCJzdHlsZSI6eyJoZWFkIjp7Im5hbWUiOiJub25lIn19fV0sWzIsMywiIiwwLHsibGV2ZWwiOjIsInN0eWxlIjp7ImhlYWQiOnsibmFtZSI6Im5vbmUifX19XSxbMCw0LCItXFxvdGltZXMgU15hIl0sWzQsNSwieyhcXHBoaV97Ki1hLGF9KX1eKiJdLFs1LDIsInsoRVxcb3RpbWVzXFx0YXUpfV8qIl1d
	\[\begin{tikzcd}
		{E_{*-a}(E\otimes X)} & { [S^{*-a},E\otimes X]} \\
		& {[S^{*-a}\otimes S^a,E\otimes X\otimes S^a]} \\
		& {[S^*,E\otimes X\otimes S^a]} \\
		& { [S^*,E\otimes S^a\otimes X]} & {E_*(\Sigma^aX)}
		\arrow[Rightarrow, no head, from=1-2, to=1-1]
		\arrow[Rightarrow, no head, from=4-2, to=4-3]
		\arrow["{-\otimes S^a}", from=1-2, to=2-2]
		\arrow["{{(\phi_{*-a,a})}^*}", from=2-2, to=3-2]
		\arrow["{{(E\otimes\tau)}_*}", from=3-2, to=4-2]
	\end{tikzcd}\]
	We leave it to the reader to check that this is an inverse for $t^a_X$, it is entirely straightforward (since we already know $t^a_X$ is an isomorphism, it suffices to show this composition is either a left or right inverse).
	
	
	Now, to see $t_X^a$ is a homomorphism of left $\pi_*(E)$-modules, suppose we are given classes $r:S^b\to E$ in $\pi_b(E)$ and $x:S^c\to E\otimes S^a\otimes X$ in $E_c(\Sigma^aX)$. Then we wish to show that $t_X^a(r\cdot x)=r\cdot t_X^a(x)$. To that end, consider the following diagram:
	% https://q.uiver.app/#q=WzAsOCxbMCwwLCJTXntiK2MtYX0iXSxbMCwyLCJTXmJcXG90aW1lcyBTXmNcXG90aW1lcyBTXnstYX0iXSxbMiwyLCJFXFxvdGltZXMgRVxcb3RpbWVzIFNeYVxcb3RpbWVzIFhcXG90aW1lcyBTXnstYX0iXSxbMiwwLCJFXFxvdGltZXMgU15hXFxvdGltZXMgWFxcb3RpbWVzIFNeey1hfSJdLFsyLDQsIkVcXG90aW1lcyBFXFxvdGltZXMgWFxcb3RpbWVzIFNeYVxcb3RpbWVzIFNeey1hfSJdLFs0LDQsIkVcXG90aW1lcyBFXFxvdGltZXMgWCJdLFs0LDIsIkVcXG90aW1lcyBYIl0sWzQsMCwiRVxcb3RpbWVzIFhcXG90aW1lcyBTXmFcXG90aW1lcyBTXnstYX0iXSxbMCwxLCJcXGNvbmciXSxbMSwyLCJyXFxvdGltZXMgeFxcb3RpbWVzIFNeey1hfSJdLFsyLDQsIkVcXG90aW1lcyBFXFxvdGltZXMgXFx0YXVcXG90aW1lcyBTXnstYX0iLDJdLFs0LDUsIkVcXG90aW1lcyBFXFxvdGltZXMgWFxcb3RpbWVzIFxccGhpX3thLC1hfV57LTF9IiwyXSxbNSw2LCJcXG11XFxvdGltZXMgWCIsMl0sWzMsNywiRVxcb3RpbWVzIFxcdGF1IFxcb3RpbWVzIFNeey1hfSJdLFs3LDYsIkVcXG90aW1lcyBYXFxvdGltZXMgXFxwaGlfe2EsLWF9XnstMX0iXSxbNCw3LCJcXG11XFxvdGltZXMgWFxcb3RpbWVzIFNeYVxcb3RpbWVzIFNeey1hfSIsMl0sWzIsMywiXFxtdVxcb3RpbWVzIFNeYVxcb3RpbWVzIFhcXG90aW1lcyBTXnstYX0iXV0=
	\[\begin{tikzcd}
		{S^{b+c-a}} && {E\otimes S^a\otimes X\otimes S^{-a}} && {E\otimes X\otimes S^a\otimes S^{-a}} \\
		\\
		{S^b\otimes S^c\otimes S^{-a}} && {E\otimes E\otimes S^a\otimes X\otimes S^{-a}} && {E\otimes X} \\
		\\
		&& {E\otimes E\otimes X\otimes S^a\otimes S^{-a}} && {E\otimes E\otimes X}
		\arrow["\cong", from=1-1, to=3-1]
		\arrow["{r\otimes x\otimes S^{-a}}", from=3-1, to=3-3]
		\arrow["{E\otimes E\otimes \tau\otimes S^{-a}}"', from=3-3, to=5-3]
		\arrow["{E\otimes E\otimes X\otimes \phi_{a,-a}^{-1}}"', from=5-3, to=5-5]
		\arrow["{\mu\otimes X}"', from=5-5, to=3-5]
		\arrow["{E\otimes \tau \otimes S^{-a}}", from=1-3, to=1-5]
		\arrow["{E\otimes X\otimes \phi_{a,-a}^{-1}}", from=1-5, to=3-5]
		\arrow["{\mu\otimes X\otimes S^a\otimes S^{-a}}"', from=5-3, to=1-5]
		\arrow["{\mu\otimes S^a\otimes X\otimes S^{-a}}", from=3-3, to=1-3]
	\end{tikzcd}\]
	Both triangles commute by functoriality of $-\otimes-$. The top composition is $t_X^a(r\cdot x)$ while the bottom is $r\cdot t_X^a(x)$, so they are equal as desired.
	
	It remains to show $t^a_X$ is natural in $X$. let $f:X\to Y$ in $\cSH$, then we would like to show the following diagram commutes:
	% https://q.uiver.app/#q=WzAsNCxbMCwwLCJFXyooXFxTaWdtYV5hWCkiXSxbMSwwLCJFX3sqLWF9KFgpIl0sWzEsMSwiRV97Ki1hfShZKSJdLFswLDEsIkVfKihcXFNpZ21hXmFZKSJdLFswLDEsInReYV9YIl0sWzEsMiwiRV97Ki1hfShmKSJdLFswLDMsIkVfKihcXFNpZ21hXmFmKSIsMl0sWzMsMiwidF5hX1kiXV0=
	\begin{equation}\label{naturality_of_t^a_diagram}\begin{tikzcd}
		{E_*(\Sigma^aX)} & {E_{*-a}(X)} \\
		{E_*(\Sigma^aY)} & {E_{*-a}(Y)}
		\arrow["{t^a_X}", from=1-1, to=1-2]
		\arrow["{E_{*-a}(f)}", from=1-2, to=2-2]
		\arrow["{E_*(\Sigma^af)}"', from=1-1, to=2-1]
		\arrow["{t^a_Y}", from=2-1, to=2-2]
	\end{tikzcd}\end{equation}
	We may chase a generator around the diagram since all the arrows here are homomorphisms. Let $x:S^b\to E\otimes S^a\otimes X$ in $E_*(\Sigma^aX)$. Then consider the following diagram:
	% https://q.uiver.app/#q=WzAsOCxbMCwwLCJTXntiLWF9Il0sWzEsMCwiU15iXFxvdGltZXMgU157LWF9Il0sWzIsMCwiRVxcb3RpbWVzIFNeYVxcb3RpbWVzIFhcXG90aW1lcyBTXnstYX0iXSxbMywwLCJFXFxvdGltZXMgWFxcb3RpbWVzIFNeYVxcb3RpbWVzIFNeey1hfSJdLFs0LDAsIkVcXG90aW1lcyBYIl0sWzQsMSwiRVxcb3RpbWVzIFkiXSxbMiwxLCJFXFxvdGltZXMgU15hXFxvdGltZXMgWVxcb3RpbWVzIFNeey1hfSJdLFszLDEsIkVcXG90aW1lcyBZXFxvdGltZXMgU15hXFxvdGltZXMgU157LWF9Il0sWzAsMSwiXFxjb25nIl0sWzEsMiwieFxcb3RpbWVzIFNeey1hfSJdLFsyLDMsIkVcXG90aW1lcyBcXHRhdVxcb3RpbWVzIFNeey1hfSJdLFszLDQsIkVcXG90aW1lcyBYXFxvdGltZXMgXFxwaGlfe2EsLWF9XnstMX0iXSxbNCw1LCJFXFxvdGltZXMgZiJdLFsyLDYsIkVcXG90aW1lcyBTXnthfVxcb3RpbWVzIGZcXG90aW1lcyBTXnstYX0iLDJdLFs2LDcsIkVcXG90aW1lcyBcXHRhdVxcb3RpbWVzIFNeey1hfSIsMl0sWzcsNSwiRVxcb3RpbWVzIFlcXG90aW1lcyBcXHBoaV97YSwtYX1eey0xfSIsMl0sWzMsNywiRVxcb3RpbWVzIGZcXG90aW1lcyBTXmFcXG90aW1lcyBTXnstYX0iLDFdXQ==
	\[\begin{tikzcd}
		{S^{b-a}} & {S^b\otimes S^{-a}} & {E\otimes S^a\otimes X\otimes S^{-a}} & {E\otimes X\otimes S^a\otimes S^{-a}} & {E\otimes X} \\
		&& {E\otimes S^a\otimes Y\otimes S^{-a}} & {E\otimes Y\otimes S^a\otimes S^{-a}} & {E\otimes Y}
		\arrow["\cong", from=1-1, to=1-2]
		\arrow["{x\otimes S^{-a}}", from=1-2, to=1-3]
		\arrow["{E\otimes \tau\otimes S^{-a}}", from=1-3, to=1-4]
		\arrow["{E\otimes X\otimes \phi_{a,-a}^{-1}}", from=1-4, to=1-5]
		\arrow["{E\otimes f}", from=1-5, to=2-5]
		\arrow["{E\otimes S^{a}\otimes f\otimes S^{-a}}"', from=1-3, to=2-3]
		\arrow["{E\otimes \tau\otimes S^{-a}}"', from=2-3, to=2-4]
		\arrow["{E\otimes Y\otimes \phi_{a,-a}^{-1}}"', from=2-4, to=2-5]
		\arrow["{E\otimes f\otimes S^a\otimes S^{-a}}"{description}, from=1-4, to=2-4]
	\end{tikzcd}\]
	The left rectangle commutes by naturality of $\tau$, while the right rectangle commutes by functoriality of $-\otimes-$. The two outside compositions are the two ways to chase $x$ around diagram (\ref{naturality_of_t^a_diagram}), so the diagram commutes as desired.
\end{proof}

%\begin{proposition}\label{product_of_monoids_is_a_monoid}
%	Let $(E_1,\mu_1,e_1)$ and $(E_2,\mu_2,e_2)$ be monoid objects in a symmetric monoidal category $(\cC,\otimes,S)$. Then $E_1\otimes E_2$ is canonically a ring spectrum via the maps
%	\[\mu:E_1\otimes E_2\otimes E_1\otimes E_2\xr{E_1\otimes\tau\otimes E_2}E_1\otimes E_1\otimes E_2\otimes E_2\xr{\mu_1\otimes\mu_2}E_1\otimes E_2\]
%	and
%	\[e:S\cong S\otimes S\xr{e_1\otimes e_2}E_1\otimes E_2.\]
%\end{proposition}
%\begin{proof}
%	\todo{todo}
%\end{proof}

\subsection{A K\"unneth isomorphism in \texorpdfstring{$\cSH$}{SH}}\label{subsection:Kunneth_appendix}

\begin{proposition}\label{Kunneth_map}
    Let $(E,\mu,e)$ be a monoid object and $Z$ and $W$ be objects in $\cSH$. Then there is a homomorphism of abelian groups
    \[\Phi_{Z,W}:Z_*(E)\otimes_{\pi_*(E)}E_*(W)\to\pi_*(Z\otimes E\otimes W)\]
    which given homogeneous elements $x:S^a\to Z\otimes E$ in $Z_*(E)=\pi_*(Z\otimes E)$  and $y:S^b\to E\otimes W$ in $E_*(W)=\pi_*(E\otimes W)$, sends the homogeneous pure tensor $x\otimes y$ in $Z_*(E)\otimes_{\pi_*(E)}E_*(W)$ to the composition
    \[\Phi_{Z,W}(x\otimes y):S^{a+b}\xr{\phi_{a,b}}S^a\otimes S^b\xr{x\otimes y}Z\otimes E\otimes E\otimes W\xr{Z\otimes\mu\otimes W}Z\otimes E\otimes W\]
    (where here we are considering the canonical $A$-graded right $\pi_*(E)$-module structure on $Z_*(E)$ and the canonical left $A$-graded $\pi_*(E)$-module structure on $E_*(W)$ given in \autoref{module_main}, so that $Z_*(E)\otimes_{\pi_*(E)}E_*(W)$ is a well-defined $A$-graded abelian group by \autoref{tensor_of_A_graded_is_A_graded}). Furthermore, this homomorphism is natural in both $Z$ and $W$.
\end{proposition}
\begin{proof}
	By \autoref{tensor_lift_of_A_graded_is_A_graded}, in order to get a homomorphism
	\[\Phi_{Z,W}:Z_*(E)\otimes_{\pi_*(E)}E_*(W)\to\pi_*(Z\otimes E\otimes W),\]
	it suffices to define an assignment $P:Z_*(E)\times E_*(W)\to\pi_*(Z\otimes E\otimes W)$ on homogeneous elements (which we have), and show that it is additive in each argument for homogeneous elements of the same degree, and that for all homogeneous $z\in Z_*(E)$, $r\in\pi_*(E)$, and $w\in E_*(W)$ that $P(zr,w)=P(z,rw)$, where concatenation denotes the module action.
	
	First, note that by \autoref{bilinear} it is straightforward to see that the assignment commutes with addition of maps of the same degree in each argument. Now, let $a,b,c\in A$, $z:S^a\to Z\otimes E$, $w:S^b\to E\otimes W$, and $r:S^c\to E$. Then we wish to show $P(zr,w)=P(z,rw)$. Consider the following diagram (where here we are passing to a symmetric strict monoidal category):
	% https://q.uiver.app/#q=WzAsNixbMCwxLCJTXnthK2IrY30iXSxbMSwxLCJTXmFcXG90aW1lcyBTXmNcXG90aW1lcyBTXmIiXSxbMiwxLCJaXFxvdGltZXMgRVxcb3RpbWVzIEVcXG90aW1lcyBFXFxvdGltZXMgVyJdLFszLDAsIlpcXG90aW1lcyBFXFxvdGltZXMgRVxcb3RpbWVzIFciXSxbMywxLCJaXFxvdGltZXMgRVxcb3RpbWVzIFciXSxbMywyLCJaXFxvdGltZXMgRVxcb3RpbWVzIEVcXG90aW1lcyBXIl0sWzAsMSwiXFxjb25nIl0sWzEsMiwielxcb3RpbWVzIHJcXG90aW1lcyB3Il0sWzIsMywiWlxcb3RpbWVzIFxcbXVcXG90aW1lcyBFXFxvdGltZXMgVyJdLFszLDQsIlpcXG90aW1lcyBcXG11XFxvdGltZXMgVyJdLFsyLDUsIlpcXG90aW1lcyBFXFxvdGltZXMgXFxtdVxcb3RpbWVzIFciLDJdLFs1LDQsIlpcXG90aW1lcyBcXG11XFxvdGltZXMgVyIsMl1d
	\[\begin{tikzcd}
		&&& {Z\otimes E\otimes E\otimes W} \\
		{S^{a+b+c}} & {S^a\otimes S^c\otimes S^b} & {Z\otimes E\otimes E\otimes E\otimes W} & {Z\otimes E\otimes W} \\
		&&& {Z\otimes E\otimes E\otimes W}
		\arrow["\cong", from=2-1, to=2-2]
		\arrow["{z\otimes r\otimes w}", from=2-2, to=2-3]
		\arrow["{Z\otimes \mu\otimes E\otimes W}", from=2-3, to=1-4]
		\arrow["{Z\otimes \mu\otimes W}", from=1-4, to=2-4]
		\arrow["{Z\otimes E\otimes \mu\otimes W}"', from=2-3, to=3-4]
		\arrow["{Z\otimes \mu\otimes W}"', from=3-4, to=2-4]
	\end{tikzcd}\]
	It commutes by associativity of $\mu$. By functoriality of $-\otimes-$, the top composition is given by $P(zr,w)$ and the bottom composition is $P(z,rw)$, so they are equal as desired. Thus, by \autoref{tensor_lift_of_A_graded_is_A_graded} we get the desired $A$-graded homomorphism $\pi_*(Z\otimes E)\otimes_{\pi_*(E)}\pi_*(E\otimes W)\to\pi_*(Z\otimes E\otimes W)$.
	
	%In order to see this map is a homomorphism of left $\pi_*(E)$-modules, we must show that $z(x\cdot y)=zx\cdot y$, where $x$, $y$, and $z$ are defined as above. Now consider the following diagram:
	%% https://q.uiver.app/#q=WzAsNixbMCwxLCJTXnthK2IrY30iXSxbMSwxLCJTXmNcXG90aW1lcyBTXmFcXG90aW1lcyBTXmIiXSxbMiwxLCJFXFxvdGltZXMgRVxcb3RpbWVzIEVcXG90aW1lcyBFXFxvdGltZXMgWCJdLFszLDAsIkVcXG90aW1lcyBFXFxvdGltZXMgRVxcb3RpbWVzIFgiXSxbMywxLCJFXFxvdGltZXMgRVxcb3RpbWVzIFgiXSxbMywyLCJFXFxvdGltZXMgRVxcb3RpbWVzIEVcXG90aW1lcyBYIl0sWzAsMSwiXFxjb25nIl0sWzEsMiwielxcb3RpbWVzIHhcXG90aW1lcyB5Il0sWzIsMywiXFxtdVxcb3RpbWVzIEVcXG90aW1lcyBFXFxvdGltZXMgWCJdLFszLDQsIkVcXG90aW1lcyBcXG11XFxvdGltZXMgWCJdLFsyLDUsIkVcXG90aW1lcyBFXFxvdGltZXMgXFxtdVxcb3RpbWVzIFgiLDJdLFs1LDQsIlxcbXVcXG90aW1lcyBFXFxvdGltZXMgWCIsMl0sWzIsNCwiXFxtdVxcb3RpbWVzXFxtdVxcb3RpbWVzIFgiXV0=
	%\[\begin{tikzcd}
		%&&& {E\otimes E\otimes E\otimes X} \\
		%{S^{a+b+c}} & {S^c\otimes S^a\otimes S^b} & {E\otimes E\otimes E\otimes E\otimes X} & {E\otimes E\otimes X} \\
		%&&& {E\otimes E\otimes E\otimes X}
		%\arrow["\cong", from=2-1, to=2-2]
		%\arrow["{z\otimes x\otimes y}", from=2-2, to=2-3]
		%\arrow["{\mu\otimes E\otimes E\otimes X}", from=2-3, to=1-4]
		%\arrow["{E\otimes \mu\otimes X}", from=1-4, to=2-4]
		%\arrow["{E\otimes E\otimes \mu\otimes X}"', from=2-3, to=3-4]
		%\arrow["{\mu\otimes E\otimes X}"', from=3-4, to=2-4]
		%\arrow["{\mu\otimes\mu\otimes X}", from=2-3, to=2-4]
	%\end{tikzcd}\]
	%Commutativity of the triangles is functoriality of $-\otimes-$. By functoriality of $-\otimes-$, the top composition is $zx\cdot y$, and the bottom composition is $z(x\cdot y)$. Hence they are equal, as desired, so that the map we have constructed
	%\[E_*(E)\otimes_{\pi_*(E)}E_*(X)\to E_*(E\otimes X)\]
	%is indeed an $A$-graded homomorphism of left $A$-graded $\pi_*(E)$-modules.

	Next, we would like to show that this homomorphism is natural in $Z$. Let $f:Z\to Z'$ in $\cSH$. Then we would like to show the following diagram commutes:
	% https://q.uiver.app/#q=WzAsNCxbMCwwLCJcXHBpXyooWlxcb3RpbWVzIEUpXFxvdGltZXNfe1xccGlfKihFKX1cXHBpXyooRVxcb3RpbWVzIFcpIl0sWzAsMSwiXFxwaV8qKFonXFxvdGltZXMgRSlcXG90aW1lc197XFxwaV8qKEUpfVxccGlfKihFXFxvdGltZXMgVykiXSxbMSwxLCJcXHBpXyooWidcXG90aW1lcyBFXFxvdGltZXMgVykiXSxbMSwwLCJcXHBpXyooWlxcb3RpbWVzIEVcXG90aW1lcyBXKSJdLFswLDEsIlxccGlfKihmXFxvdGltZXMgRSlcXG90aW1lc1xccGlfKihFXFxvdGltZXMgVykiLDJdLFsxLDIsIlxcUGhpX3taJyxXfSJdLFszLDIsIlxccGlfKihmXFxvdGltZXMgRVxcb3RpbWVzIFcpIl0sWzAsMywiXFxQaGlfe1osV30iXV0=
	\begin{equation}\label{naturality_diagram_for_E*EoxE_*X-->E_*(EoxX)}\begin{tikzcd}
		{\pi_*(Z\otimes E)\otimes_{\pi_*(E)}\pi_*(E\otimes W)} & {\pi_*(Z\otimes E\otimes W)} \\
		{\pi_*(Z'\otimes E)\otimes_{\pi_*(E)}\pi_*(E\otimes W)} & {\pi_*(Z'\otimes E\otimes W)}
		\arrow["{\pi_*(f\otimes E)\otimes\pi_*(E\otimes W)}"', from=1-1, to=2-1]
		\arrow["{\Phi_{Z',W}}", from=2-1, to=2-2]
		\arrow["{\pi_*(f\otimes E\otimes W)}", from=1-2, to=2-2]
		\arrow["{\Phi_{Z,W}}", from=1-1, to=1-2]
	\end{tikzcd}\end{equation}
	As all the maps here are homomorphisms, in order to show it commutes, it suffices to chase generators around the diagram. In particular, suppose we are given $z:S^a\to Z\otimes E$ and $w:S^b\to E\otimes W$, and consider the following diagram exhibiting the two possible ways to chase $z\otimes w$ around the diagram (as usual, we are passing to a symmetric strict monoidal category):
	% https://q.uiver.app/#q=WzAsNixbMCwwLCJTXnthK2J9Il0sWzEsMCwiU15hXFxvdGltZXMgU15iIl0sWzIsMCwiWlxcb3RpbWVzIEVcXG90aW1lcyBFXFxvdGltZXMgVyJdLFszLDAsIlpcXG90aW1lcyBFXFxvdGltZXMgVyJdLFszLDEsIlonXFxvdGltZXMgRVxcb3RpbWVzIFciXSxbMiwxLCJaXFxvdGltZXMgRVxcb3RpbWVzIEVcXG90aW1lcyBXIl0sWzAsMSwiXFxwaGlfe2EsYn0iXSxbMSwyLCJ6XFxvdGltZXMgdyJdLFsyLDMsIlpcXG90aW1lcyBcXG11XFxvdGltZXMgVyJdLFszLDQsImZcXG90aW1lcyBFXFxvdGltZXMgVyJdLFsyLDUsImZcXG90aW1lcyBFXFxvdGltZXMgRVxcb3RpbWVzIFciLDJdLFs1LDQsIlpcXG90aW1lcyBcXG11XFxvdGltZXMgVyJdXQ==
	\[\begin{tikzcd}
		{S^{a+b}} & {S^a\otimes S^b} & {Z\otimes E\otimes E\otimes W} & {Z\otimes E\otimes W} \\
		&& {Z\otimes E\otimes E\otimes W} & {Z'\otimes E\otimes W}
		\arrow["{\phi_{a,b}}", from=1-1, to=1-2]
		\arrow["{z\otimes w}", from=1-2, to=1-3]
		\arrow["{Z\otimes \mu\otimes W}", from=1-3, to=1-4]
		\arrow["{f\otimes E\otimes W}", from=1-4, to=2-4]
		\arrow["{f\otimes E\otimes E\otimes W}"', from=1-3, to=2-3]
		\arrow["{Z\otimes \mu\otimes W}", from=2-3, to=2-4]
	\end{tikzcd}\]
	This diagram commutes by functoriality of $-\otimes-$. Thus we have that diagram (\ref{naturality_diagram_for_E*EoxE_*X-->E_*(EoxX)}) does indeed commute, so that $\Phi_{Z,W}$ is natural in $Z$ as desired. Showing that $\Phi_{Z,W}$ is natural in $W$ is entirely analagous.
\end{proof}

\begin{lemma}\label{t_a's_commute_with_Kunneth_map}
	Let $(E,\mu,e)$ be a monoid object and $Z$ and $W$ be objects in $\cSH$. Then for all $a\in A$, the following diagram commutes
	% https://q.uiver.app/#q=WzAsNixbMCwwLCJaXyooRSlcXG90aW1lc197XFxwaV8qKEUpfUVfKihcXFNpZ21hXmFXKSJdLFsyLDAsIlpfKihFKVxcb3RpbWVzX3tcXHBpXyooRSl9RV97Ki1hfShXKSJdLFsyLDEsIlxccGlfeyotYX0oWlxcb3RpbWVzIEVcXG90aW1lcyBXKSJdLFswLDEsIlxccGlfKihaXFxvdGltZXMgRVxcb3RpbWVzXFxTaWdtYV5hVykiXSxbMCwyLCIoWlxcb3RpbWVzIEUpXyooXFxTaWdtYV5hVykiXSxbMiwyLCIoWlxcb3RpbWVzIEUpX3sqLWF9KFcpIl0sWzAsMSwiWl8qKEUpXFxvdGltZXNfe1xccGlfKihFKX10X2FeVyJdLFsxLDIsIlxcUGhpX3taLFd9Il0sWzAsMywiXFxQaGlfe1osXFxTaWdtYV5hV30iLDJdLFszLDQsIiIsMix7ImxldmVsIjoyLCJzdHlsZSI6eyJoZWFkIjp7Im5hbWUiOiJub25lIn19fV0sWzQsNSwidF9hXlciXSxbNSwyLCIiLDIseyJsZXZlbCI6Miwic3R5bGUiOnsiaGVhZCI6eyJuYW1lIjoibm9uZSJ9fX1dXQ==
	\[\begin{tikzcd}
		{Z_*(E)\otimes_{\pi_*(E)}E_*(\Sigma^aW)} && {Z_*(E)\otimes_{\pi_*(E)}E_{*-a}(W)} \\
		{\pi_*(Z\otimes E\otimes\Sigma^aW)} && {\pi_{*-a}(Z\otimes E\otimes W)} \\
		{(Z\otimes E)_*(\Sigma^aW)} && {(Z\otimes E)_{*-a}(W)}
		\arrow["{Z_*(E)\otimes_{\pi_*(E)}t_a^W}", from=1-1, to=1-3]
		\arrow["{\Phi_{Z,W}}", from=1-3, to=2-3]
		\arrow["{\Phi_{Z,\Sigma^aW}}"', from=1-1, to=2-1]
		\arrow[Rightarrow, no head, from=2-1, to=3-1]
		\arrow["{t_a^W}", from=3-1, to=3-3]
		\arrow[Rightarrow, no head, from=3-3, to=2-3]
	\end{tikzcd}\]
	where the maps $\Phi_{Z,W}$ are constructed in \autoref{Kunneth_map} and shown to be $A$-graded homomorphisms of abelian groups, and the maps $t_a$ are constructed and proven to be $A$-graded isomorphisms of abelian groups in \autoref{E_homology_suspension_iso_t^a's_appendix}.
\end{lemma}
\begin{proof}
	Note that in \autoref{E_homology_suspension_iso_t^a's_appendix}, it is shown that $t_a^W:E_*(\Sigma^aW)\to E_{*-a}(W)$ is not just an $A$-graded isomorphism of abelian groups, but it is furthermore a left $\pi_*(E)$-module isomorphism. Thus, the top arrow in the above diagram is well-defined. Since all the arrows involved are $A$-graded homomorphisms, in order to show the diagram commutes it suffices to chase a pure homogeneous tensor around, as they generate the top left object. To that end, let $x:S^b\to Z\otimes E$ in $Z_*(E)$ and $y:S^c\to E\otimes S^a\otimes W$ in $E_*(\Sigma^aW)$, and consider the following diagram exhibiting the two ways to chase $x\otimes y$ around:
	% https://q.uiver.app/#q=WzAsOCxbMCwwLCJTXntiK2MtYX0iXSxbMCwyLCJTXmJcXG90aW1lcyBTXmNcXG90aW1lcyBTXnstYX0iXSxbMiwyLCJaXFxvdGltZXMgRVxcb3RpbWVzIEVcXG90aW1lcyBTXmFcXG90aW1lcyBXXFxvdGltZXMgU157LWF9Il0sWzIsMCwiWlxcb3RpbWVzIEVcXG90aW1lcyBFXFxvdGltZXMgV1xcb3RpbWVzIFNeYVxcb3RpbWVzIFNeey1hfSJdLFs0LDAsIlpcXG90aW1lcyBFXFxvdGltZXMgRVxcb3RpbWVzIFciXSxbNCwyLCJaXFxvdGltZXMgRVxcb3RpbWVzIFciXSxbMiw0LCJaXFxvdGltZXMgRVxcb3RpbWVzIFNeYVxcb3RpbWVzIFdcXG90aW1lcyBTXnstYX0iXSxbNCw0LCJaXFxvdGltZXMgRVxcb3RpbWVzIFdcXG90aW1lcyBTXmFcXG90aW1lcyBTXnstYX0iXSxbMCwxLCJcXHBoaSJdLFsxLDIsInhcXG90aW1lcyB5XFxvdGltZXMgU157LWF9Il0sWzIsMywiWlxcb3RpbWVzIEVcXG90aW1lcyBFXFxvdGltZXMgXFx0YXVcXG90aW1lcyBTXnstYX0iXSxbMyw0LCJaXFxvdGltZXMgRVxcb3RpbWVzIEVcXG90aW1lcyBXXFxvdGltZXMgXFxwaGlfe2EsLWF9XnstMX0iXSxbNCw1LCJaXFxvdGltZXMgXFxtdVxcb3RpbWVzIFciXSxbMiw2LCJaXFxvdGltZXMgXFxtdVxcb3RpbWVzIFNeYVxcb3RpbWVzIFdcXG90aW1lcyBTXnstYX0iLDJdLFs2LDcsIlpcXG90aW1lcyBFXFxvdGltZXMgXFx0YXVcXG90aW1lcyBTXnstYX0iLDJdLFs3LDUsIlpcXG90aW1lcyBFXFxvdGltZXMgV1xcb3RpbWVzIFxccGhpX3thLC1hfV57LTF9IiwyXSxbMyw3LCJaXFxvdGltZXMgXFxtdVxcb3RpbWVzIFdcXG90aW1lcyBTXmFcXG90aW1lcyBTXnstYX0iXV0=
	\[\begin{tikzcd}
		{S^{b+c-a}} && {Z\otimes E\otimes E\otimes W\otimes S^a\otimes S^{-a}} && {Z\otimes E\otimes E\otimes W} \\
		\\
		{S^b\otimes S^c\otimes S^{-a}} && {Z\otimes E\otimes E\otimes S^a\otimes W\otimes S^{-a}} && {Z\otimes E\otimes W} \\
		\\
		&& {Z\otimes E\otimes S^a\otimes W\otimes S^{-a}} && {Z\otimes E\otimes W\otimes S^a\otimes S^{-a}}
		\arrow["\phi", from=1-1, to=3-1]
		\arrow["{x\otimes y\otimes S^{-a}}", from=3-1, to=3-3]
		\arrow["{Z\otimes E\otimes E\otimes \tau\otimes S^{-a}}", from=3-3, to=1-3]
		\arrow["{Z\otimes E\otimes E\otimes W\otimes \phi_{a,-a}^{-1}}", from=1-3, to=1-5]
		\arrow["{Z\otimes \mu\otimes W}", from=1-5, to=3-5]
		\arrow["{Z\otimes \mu\otimes S^a\otimes W\otimes S^{-a}}"', from=3-3, to=5-3]
		\arrow["{Z\otimes E\otimes \tau\otimes S^{-a}}"', from=5-3, to=5-5]
		\arrow["{Z\otimes E\otimes W\otimes \phi_{a,-a}^{-1}}"', from=5-5, to=3-5]
		\arrow["{Z\otimes \mu\otimes W\otimes S^a\otimes S^{-a}}", from=1-3, to=5-5]
	\end{tikzcd}\]
	Each triangle commutes by functoriality of $-\otimes-$, so the diagram commutes as desired.
\end{proof}

\begin{lemma}\label{Phi_Z,W_iso_implies_Phi_S^aZ,W_and_Phi_Z,S^aW_are_isos}
	Let $(E,\mu,e)$ be a monoid object and $Z$ and $W$ objects in $\cSH$, and suppose the map $\Phi_{Z,W}$ constructed in \autoref{Kunneth_map} is an isomorphism. Then $\Phi_{\Sigma^a Z,W}$ and $\Phi_{Z,\Sigma^aW}$ are isomorphisms for all $a\in A$. In particular, $\Phi_{\Sigma Z,W}$ an $\Phi_{Z,\Sigma W}$ are isomorphisms. 
\end{lemma}
\begin{proof}
	If $\Phi_{Z,W}$ is an isomorphism, it follows that $\Phi_{Z,\Sigma^aW}$ is an isomorphism by \autoref{t_a's_commute_with_Kunneth_map}, since every other arrow in that diagram is an isomorphism. On the other hand, in order to see $\Phi_{\Sigma^aZ,W}$ is an isomorphism, consider the following diagram:
	% https://q.uiver.app/#q=WzAsNCxbMCwwLCJcXHBpXyooXFxTaWdtYV5hWlxcb3RpbWVzIEUpXFxvdGltZXNfe1xccGlfKihFKX1cXHBpXyooRVxcb3RpbWVzIFcpIl0sWzEsMCwiXFxwaV8qKFxcU2lnbWFeYSBaXFxvdGltZXMgRVxcb3RpbWVzIFcpIl0sWzAsMSwiXFxwaV97Ki1hfShaXFxvdGltZXMgRSlcXG90aW1lc197XFxwaV8qKEUpfVxccGlfKihFXFxvdGltZXMgVykiXSxbMSwxLCJcXHBpX3sqLWF9KFpcXG90aW1lcyBFXFxvdGltZXMgVykiXSxbMCwxLCJcXFBoaV97XFxTaWdtYV5hWixXfSJdLFswLDIsIlxcbWF0aHJte2Fkan1cXG90aW1lc197XFxwaV8qKEUpfVxccGlfKihFXFxvdGltZXMgVykiLDJdLFsxLDMsIlxcbWF0aHJte2Fkan0iXSxbMiwzLCJcXFBoaV97WixXfSJdXQ==
	\begin{equation}\label{Phi_Sigma^aZ,W_diagram}\begin{tikzcd}
		{\pi_*(\Sigma^aZ\otimes E)\otimes_{\pi_*(E)}\pi_*(E\otimes W)} & {\pi_*(\Sigma^a Z\otimes E\otimes W)} \\
		{\pi_{*-a}(Z\otimes E)\otimes_{\pi_*(E)}\pi_*(E\otimes W)} & {\pi_{*-a}(Z\otimes E\otimes W)}
		\arrow["{\Phi_{\Sigma^aZ,W}}", from=1-1, to=1-2]
		\arrow["{\mathrm{adj}\otimes_{\pi_*(E)}\pi_*(E\otimes W)}"', from=1-1, to=2-1]
		\arrow["{\mathrm{adj}}", from=1-2, to=2-2]
		\arrow["{\Phi_{Z,W}}", from=2-1, to=2-2]
	\end{tikzcd}\end{equation}
	Here the left vertical arrow is constructed via the map $\mathrm{adj}:\pi_*(\Sigma^aZ\otimes E)\to\pi_{*-a}(Z\otimes E)$ which is given as as the composition
	\[\mathrm{adj}:[S^*,S^a\otimes Z\otimes E]\xr\cong[S^{-a}\otimes S^{*},Z\otimes E]\xr{{(\phi_{-a,*}^{-1})}^*}[S^{*-a},Z\otimes E]\]
	where the first arrow is the adjunction isomorphism from \autoref{Sigma^a,Sigma^-a_adjoint_equiv}. Explicitly, this map sends a class $x:S^b\to S^a\otimes Z\otimes E$ to the composition
	\[S^{b-a}\xr{\phi_{-a,b}}S^{-a}\otimes S^b\xr{S^{-a}\otimes x}S^{-a}\otimes S^{a}\otimes Z\otimes E\xr{\phi_{-a,a}^{-1}\otimes Z\otimes E}Z\otimes E.\]
	In order to show $\mathrm{adj}\otimes_{\pi_*(E)}\pi_*(E\otimes W)$ is well defined, it suffices to show $\mathrm{adj}$ is a degree $-a$ $A$-graded homomorphism of right $A$-graded $\pi_*(E)$-modules. It is clearly additive, as any adjunction between additive categories is automatically additive, as is composing with an morphism in an additive category. Thus, it remains to show $\mathrm{adj}$ commutes with scalar multiplication. By additivity, it suffices to consider only homogeneous elements. Let $x:S^a\to S^\1\otimes Z\otimes E$ in $\pi_*(S^\1\otimes Z\otimes E)$ and $r:S^b\to E$ in $\pi_*(E)$. Then we'd like to show that $\mathrm{adj}(x\cdot r)=\mathrm{adj}(x)\cdot r$. To see this, consider the following diagram:
	% https://q.uiver.app/#q=WzAsNixbMCwwLCJTXnthK2ItXFwxfSJdLFsxLDAsIlNeey1cXDF9XFxvdGltZXMgU15hXFxvdGltZXMgU15iIl0sWzIsMCwiU157LVxcMX1cXG90aW1lcyBTXlxcMVxcb3RpbWVzIFpcXG90aW1lcyBFXFxvdGltZXMgRSJdLFszLDAsIlpcXG90aW1lcyBFXFxvdGltZXMgRSJdLFszLDEsIlpcXG90aW1lcyBFIl0sWzIsMSwiU157LVxcMX1cXG90aW1lcyBTXlxcMVxcb3RpbWVzIFpcXG90aW1lcyBFIl0sWzAsMSwiXFxwaGkiXSxbMSwyLCJTXnstXFwxfVxcb3RpbWVzIHhcXG90aW1lcyB5Il0sWzIsMywiXFxwaGlfey1cXDEsXFwxfV57LTF9XFxvdGltZXMgWlxcb3RpbWVzIEVcXG90aW1lcyBFIl0sWzMsNCwiWlxcb3RpbWVzIFxcbXUiXSxbMiw1LCJTXnstXFwxfVxcb3RpbWVzIFNeXFwxXFxvdGltZXMgWlxcb3RpbWVzIFxcbXUiLDJdLFs1LDQsIlxccGhpX3stXFwxLFxcMX1eey0xfVxcb3RpbWVzIFpcXG90aW1lcyBFIiwyXV0=
	\[\begin{tikzcd}
		{S^{a+b-\1}} & {S^{-\1}\otimes S^a\otimes S^b} & {S^{-\1}\otimes S^\1\otimes Z\otimes E\otimes E} & {Z\otimes E\otimes E} \\
		&& {S^{-\1}\otimes S^\1\otimes Z\otimes E} & {Z\otimes E}
		\arrow["\phi", from=1-1, to=1-2]
		\arrow["{S^{-\1}\otimes x\otimes y}", from=1-2, to=1-3]
		\arrow["{\phi_{-\1,\1}^{-1}\otimes Z\otimes E\otimes E}", from=1-3, to=1-4]
		\arrow["{Z\otimes \mu}", from=1-4, to=2-4]
		\arrow["{S^{-\1}\otimes S^\1\otimes Z\otimes \mu}"', from=1-3, to=2-3]
		\arrow["{\phi_{-\1,\1}^{-1}\otimes Z\otimes E}"', from=2-3, to=2-4]
	\end{tikzcd}\]
	The top composition is $\mathrm{adj}(x)\cdot r$, while the bottom composition is $\mathrm{adj}(x\cdot r)$. The diagram commutes by functoriality of $-\otimes-$. Thus, it follows that $\mathrm{adj}(x)\cdot r=\mathrm{adj}(x\cdot r)$, so that $\mathrm{adj}$ is indeed an homomorphism of right $\pi_*(E)$-modules, in fact, an isomorphism as desired. Thus, since every arrow in diagram (\ref{Phi_Sigma^aZ,W_diagram}) is an isomorphism of abelian groups except the top arrow, in order to show $\Phi_{\Sigma^aZ,W}$ is an isomorphism, it suffices to show the diagram commutes. To that end, since all the arrows are homomorphisms, it suffices to chase a pure homogeneous tensor. So let $x:S^b\to\Sigma^aZ\otimes E$ and $y:S^c\to E\otimes W$, and consider the following diagram whose outside compositions exhibit the two ways to chase the pure tensor $x\otimes y$ around diagrama (\ref{Phi_Sigma^aZ,W_diagram}):
	% https://q.uiver.app/#q=WzAsNixbMCwwLCJTXntiK2MtYX0iXSxbMSwwLCJTXnstYX1cXG90aW1lcyBTXmJcXG90aW1lcyBTXmMiXSxbMiwwLCJTXnstYX1cXG90aW1lcyBTXmFcXG90aW1lcyBaXFxvdGltZXMgRVxcb3RpbWVzIEVcXG90aW1lcyBXIl0sWzMsMCwiU157LWF9XFxvdGltZXMgU15hXFxvdGltZXMgWlxcb3RpbWVzIEVcXG90aW1lcyBXIl0sWzMsMSwiWlxcb3RpbWVzIEVcXG90aW1lcyBXIl0sWzIsMSwiWlxcb3RpbWVzIEVcXG90aW1lcyBFXFxvdGltZXMgVyJdLFswLDEsIlxccGhpIl0sWzEsMiwiU157LWF9XFxvdGltZXMgeFxcb3RpbWVzIHkiXSxbMiwzLCJTXnstYX1cXG90aW1lcyBTXmFcXG90aW1lcyBaXFxvdGltZXMgXFxtdVxcb3RpbWVzIFciXSxbMyw0LCJcXHBoaV97LWEsYX1eey0xfVxcb3RpbWVzIFpcXG90aW1lcyBFXFxvdGltZXMgVyJdLFsyLDUsIlxccGhpX3stYSxhfV57LTF9XFxvdGltZXMgWlxcb3RpbWVzIEVcXG90aW1lcyBFXFxvdGltZXMgVyIsMl0sWzUsNCwiWlxcb3RpbWVzIFxcbXVcXG90aW1lcyBXIl1d
	\[\begin{tikzcd}
		{S^{b+c-a}} & {S^{-a}\otimes S^b\otimes S^c} & {S^{-a}\otimes S^a\otimes Z\otimes E\otimes E\otimes W} & {S^{-a}\otimes S^a\otimes Z\otimes E\otimes W} \\
		&& {Z\otimes E\otimes E\otimes W} & {Z\otimes E\otimes W}
		\arrow["\phi", from=1-1, to=1-2]
		\arrow["{S^{-a}\otimes x\otimes y}", from=1-2, to=1-3]
		\arrow["{S^{-a}\otimes S^a\otimes Z\otimes \mu\otimes W}", from=1-3, to=1-4]
		\arrow["{\phi_{-a,a}^{-1}\otimes Z\otimes E\otimes W}", from=1-4, to=2-4]
		\arrow["{\phi_{-a,a}^{-1}\otimes Z\otimes E\otimes E\otimes W}"', from=1-3, to=2-3]
		\arrow["{Z\otimes \mu\otimes W}", from=2-3, to=2-4]
	\end{tikzcd}\]
	The diagram clearly commutes by functoriality of $-\otimes-$, so that indeed diagram (\ref{Phi_Sigma^aZ,W_diagram}) commutes, so that $\Phi_{\Sigma^aZ,W}$ is indeed an isomorphism as desired. 

	Now, it remains to show that $\Phi_{Z,\Sigma W}$ and $\Phi_{\Sigma Z,W}$ are isomorphisms. To that end, consider the following diagram:
	% https://q.uiver.app/#q=WzAsNCxbMCwwLCJcXHBpXyooWlxcb3RpbWVzIEUpXFxvdGltZXNfe1xccGlfKihFKX1cXHBpXyooRVxcb3RpbWVzIFxcU2lnbWEgVykiXSxbMCwyLCJcXHBpXyooWlxcb3RpbWVzIEUpXFxvdGltZXNfe1xccGlfKihFKX1cXHBpXyooRVxcb3RpbWVzXFxTaWdtYV5cXDFXKSJdLFsxLDIsIlxccGlfKihaXFxvdGltZXMgRVxcb3RpbWVzXFxTaWdtYV5cXDFXKSJdLFsxLDAsIlxccGlfKihaXFxvdGltZXMgRVxcb3RpbWVzXFxTaWdtYSBXKSJdLFswLDEsIlxccGlfKihaXFxvdGltZXMgRSlcXG90aW1lc197XFxwaV8qKEUpfVxccGlfKihFXFxvdGltZXNcXG51X1cpIiwyXSxbMSwyLCJcXFBoaV97WixcXFNpZ21hXlxcMVd9IiwyXSxbMCwzLCJcXFBoaV97WixcXFNpZ21hIFd9Il0sWzMsMiwiXFxwaV8qKFpcXG90aW1lcyBFXFxvdGltZXMgXFxudV9XKSJdXQ==
	\[\begin{tikzcd}
		{\pi_*(Z\otimes E)\otimes_{\pi_*(E)}\pi_*(E\otimes \Sigma W)} & {\pi_*(Z\otimes E\otimes\Sigma W)} \\
		\\
		{\pi_*(Z\otimes E)\otimes_{\pi_*(E)}\pi_*(E\otimes\Sigma^\1W)} & {\pi_*(Z\otimes E\otimes\Sigma^\1W)}
		\arrow["{\pi_*(Z\otimes E)\otimes_{\pi_*(E)}\pi_*(E\otimes\nu_W)}"', from=1-1, to=3-1]
		\arrow["{\Phi_{Z,\Sigma^\1W}}"', from=3-1, to=3-2]
		\arrow["{\Phi_{Z,\Sigma W}}", from=1-1, to=1-2]
		\arrow["{\pi_*(Z\otimes E\otimes \nu_W)}", from=1-2, to=3-2]
	\end{tikzcd}\]
	It commutes by naturality of $\Phi$. Furthermore, assuming $\Phi_{Z,W}$ is an isomorphism, by what we have shown above we know that $\Phi_{Z,\Sigma^\1W}$ is an isomorphism, and since $\nu_W$ is an isomorphism, it follows that the above diagram commutes and all arrows except $\Phi_{Z,\Sigma W}$ are isomorphisms, so that $\Phi_{Z,\Sigma W}$ must be an isomorphism itself. Finally, an entirely analagous argument using naturality of $\Phi$ with respect to $\nu_Z$ yields that $\Phi_{\Sigma Z,W}$ is an isomorphism as well.
\end{proof}

\begin{proposition}\label{Kunneth_iso_for_cellular_objects}
	Let $(E,\mu,e)$ be a monoid object and $Z$ and $W$ objects in $\cSH$. Then if either:\begin{enumerate}
		\item $Z_*(E)$ is a flat right $\pi_*(E)$-module (via \autoref{module}) and $W$ is cellular (\autoref{cellular}), or
		\item $E_*(W)$ is a flat left $\pi_*(E)$-module (via \autoref{module}) and $Z$ is cellular (\autoref{cellular}),
	\end{enumerate} 
	then the natural homomorphism
	\[\Phi_{Z,W}:Z_*(E)\otimes_{\pi_*(E)}E_*(W)\to \pi_*(Z\otimes E\otimes W)\]
	given in \autoref{Kunneth_map} is an isomorphism of abelian groups.
\end{proposition}
\begin{proof}
	In this proof, we will freely employ the coherence theorem for symmetric monoidal categories, and we will assume that associativity and unitality of $-\otimes-$ holds up to strict equality. First we will consider the case that $\pi_*(Z\otimes E)=Z_*(E)$ is a flat right $\pi_*(E)$-module and $W$ is cellular. To start, let $\cE$ be the collection of objects $W$ in $\cSH$ for which this map is an isomorphism. Then in order to show $\cE$ contains every cellular object, it suffices to show that $\cE$ satisfies the three conditions given for the class of cellular objects in \autoref{cellular}. First, we need to show that $\Phi_{Z,W}$ is an isomorphism when $W=S^a$ for some $a\in A$.
%	Note that
%	\[E_*(S^a)=[S^*,E\otimes S^a]\cong[S^{-a}\otimes S^*,E]\cong[S^{*-a},E]=\pi_{*-a}(E),\]
%	where the first isomorphism follows by the adunction between $S^{-a}\otimes-$ and $-\otimes S^a\cong S^a\otimes-$ (\autoref{Sigma^a,Sigma^-a_adjoint_equiv}). Similarly, we have
%	\[E_*(E\otimes S^a)=[S^*,E\otimes E\otimes S^a]\cong[S^{*-a},E\otimes E]=E_{*-a}(E).\]
%	Hence by \autoref{tensor_shift_A_graded} we have isomorphisms
%	\[E_*(E)\otimes_{\pi_*(E)}E_*(S^a)\cong E_*(E)\otimes_{\pi_*(E)}\pi_{*-a}(E)\cong E_{*-a}(E)\cong E_*(E\otimes S^a).\]
	Indeed, consider the $A$-graded homomorphism
	\begin{align*}
		\Psi:\pi_*(Z\otimes E\otimes S^a)&\to \pi_*(Z\otimes E)\otimes_{\pi_*(E)}\pi_*(E\otimes S^a)
	\end{align*}
	which sends a class $x:S^b\to Z\otimes E\otimes S^a$ in $\pi_b(Z\otimes E\otimes S^a)$ to the pure tensor $\wt x\otimes\wt e$, where $\wt x\in \pi_{b-a}(Z\otimes E)$ is the composition
	\[S^{b-a}\xr{\phi_{b,-a}}S^b\otimes S^{-a}\xr{x\otimes S^{-a}}Z\otimes E\otimes S^a\otimes S^{-a}\xr{Z\otimes E\otimes\phi_{a,-a}^{-1}}Z\otimes E\]
	and $\wt e\in \pi_a(E\otimes S^a)$ is the composition
	\[S^a\xr{e\otimes S^a}E\otimes S^a.\]
	In order to see $\Psi$ is an ($A$-graded) homomorphism of abelian groups: Given $x,x'\in \pi_b(Z\otimes E\otimes S^a)$, we would like to show that $\wt x\otimes\wt e+\wt x'\otimes\wt e=\wt{x+x'}\otimes\wt e$. It suffices to show that $\wt x+\wt x'=\wt{x+x'}$. To see this, consider the following diagram (again, we are passing to a symmetric strict monoidal category):
	% https://q.uiver.app/#q=WzAsMTAsWzAsMCwiU157Yi1hfSJdLFsxLDAsIlNee2ItYX1cXG9wbHVzIFNee2ItYX0iXSxbMSwxLCIoU15iXFxvdGltZXMgU157LWF9KVxcb3BsdXMoU15iXFxvdGltZXMgU157LWF9KSJdLFsxLDIsIihaXFxvdGltZXMgRVxcb3RpbWVzIFNeYVxcb3RpbWVzIFNeey1hfSlcXG9wbHVzKFpcXG90aW1lcyBFXFxvdGltZXMgU15hXFxvdGltZXMgU157LWF9KSJdLFsxLDMsIihaXFxvdGltZXMgRSlcXG9wbHVzKFpcXG90aW1lcyBFKSJdLFsxLDQsIlpcXG90aW1lcyBFIl0sWzAsMSwiU15iXFxvdGltZXMgU157LWF9Il0sWzAsMiwiKFNeYlxcb3BsdXMgU15iKVxcb3RpbWVzIFNeey1hfSJdLFswLDMsIigoWlxcb3RpbWVzIEVcXG90aW1lcyBTXmEpXFxvcGx1cyAoWlxcb3RpbWVzIEVcXG90aW1lcyBTXmEpKVxcb3RpbWVzIFNeey1hfSJdLFswLDQsIlpcXG90aW1lcyBFXFxvdGltZXMgU15hXFxvdGltZXMgU157LWF9Il0sWzAsMSwiXFxEZWx0YSJdLFsxLDIsIlxccGhpX3tiLC1hfVxcb3BsdXNcXHBoaV97YiwtYX0iXSxbMiwzLCIoeFxcb3RpbWVzIFNeey1hfSlcXG9wbHVzKHgnXFxvdGltZXMgU157LWF9KSJdLFszLDQsIihaXFxvdGltZXMgRVxcb3RpbWVzXFxwaGlfe2EsLWF9XnstMX0pXFxvcGx1cyhaXFxvdGltZXMgRVxcb3RpbWVzXFxwaGlfe2EsLWF9XnstMX0pIl0sWzQsNSwiXFxuYWJsYSJdLFswLDYsIlxccGhpX3tiLWF9IiwyXSxbNiw3LCJcXERlbHRhXFxvdGltZXMgU157LWF9IiwyXSxbNyw4LCIoeFxcb3BsdXMgeCcpXFxvdGltZXMgU157LWF9IiwyXSxbOCw5LCJcXG5hYmxhXFxvdGltZXMgU157LWF9IiwyXSxbOSw1LCJaXFxvdGltZXMgRVxcb3RpbWVzXFxwaGlfe2EsLWF9XnstMX0iLDJdLFs3LDIsIlxcY29uZyJdLFs4LDMsIlxcY29uZyJdLFszLDksIlxcbmFibGEiXSxbNiwyLCJcXERlbHRhIl1d
	\[\begin{tikzcd}
		{S^{b-a}} & {S^{b-a}\oplus S^{b-a}} \\
		{S^b\otimes S^{-a}} & {(S^b\otimes S^{-a})\oplus(S^b\otimes S^{-a})} \\
		{(S^b\oplus S^b)\otimes S^{-a}} & {(Z\otimes E\otimes S^a\otimes S^{-a})\oplus(Z\otimes E\otimes S^a\otimes S^{-a})} \\
		{((Z\otimes E\otimes S^a)\oplus (Z\otimes E\otimes S^a))\otimes S^{-a}} & {(Z\otimes E)\oplus(Z\otimes E)} \\
		{Z\otimes E\otimes S^a\otimes S^{-a}} & {Z\otimes E}
		\arrow["\Delta", from=1-1, to=1-2]
		\arrow["{\phi_{b,-a}\oplus\phi_{b,-a}}", from=1-2, to=2-2]
		\arrow["{(x\otimes S^{-a})\oplus(x'\otimes S^{-a})}", from=2-2, to=3-2]
		\arrow["{(Z\otimes E\otimes\phi_{a,-a}^{-1})\oplus(Z\otimes E\otimes\phi_{a,-a}^{-1})}", from=3-2, to=4-2]
		\arrow["\nabla", from=4-2, to=5-2]
		\arrow["{\phi_{b-a}}"', from=1-1, to=2-1]
		\arrow["{\Delta\otimes S^{-a}}"', from=2-1, to=3-1]
		\arrow["{(x\oplus x')\otimes S^{-a}}"', from=3-1, to=4-1]
		\arrow["{\nabla\otimes S^{-a}}"', from=4-1, to=5-1]
		\arrow["{Z\otimes E\otimes\phi_{a,-a}^{-1}}"', from=5-1, to=5-2]
		\arrow["\cong", from=3-1, to=2-2]
		\arrow["\cong", from=4-1, to=3-2]
		\arrow["\nabla", from=3-2, to=5-1]
		\arrow["\Delta", from=2-1, to=2-2]
	\end{tikzcd}\]
	The top rectangle commutes by naturality of $\Delta$ in an additive category. The bottom triangle commutes by naturality of $\nabla$ in an additive category. Finally, the remaining regions of the diagram commute by additivity of $-\otimes-$. By functoriality of $-\otimes-$, it follows that the left composition is $\wt{x+x'}$ and the right composition is $\wt x+\wt x'$, so they are equal as desired. Thus $\Psi$ is a homomorphism of abelian groups, as desired.

	Now, we claim that $\Psi$ is an inverse to $\Phi_{Z,S^a}$. Since $\Phi_{Z,S^a}$ and $\Psi$ are homomorphisms it suffices to check that they are inverses on generators. First, let $x:S^b\to Z\otimes E\otimes S^a$ in $\pi_b(Z\otimes E\otimes S^a)$. We would like to show that $\Phi_{Z,S^a}(\Psi(x))=x$. Consider the following diagram, where here we are passing to a symmetric strict monoidal category:
	% https://q.uiver.app/#q=WzAsNyxbMCwwLCJTXmIiXSxbMiwwLCJTXmJcXG90aW1lcyBTXnstYX1cXG90aW1lcyBTXmEiXSxbNCwxLCJaXFxvdGltZXMgRVxcb3RpbWVzIFNeYVxcb3RpbWVzIFNeey1hfVxcb3RpbWVzIEVcXG90aW1lcyBTXmEiXSxbMCwyLCJaXFxvdGltZXMgRVxcb3RpbWVzIFNeYSJdLFsyLDMsIlpcXG90aW1lcyBFXFxvdGltZXMgRVxcb3RpbWVzIFNeYSJdLFsyLDEsIlpcXG90aW1lcyBFXFxvdGltZXMgU15hIFxcb3RpbWVzIFNeey1hfVxcb3RpbWVzIFNeYSJdLFsyLDIsIlpcXG90aW1lcyBFXFxvdGltZXMgU15hIl0sWzAsMSwiXFxjb25nIl0sWzEsMiwieFxcb3RpbWVzIFNeey1hfVxcb3RpbWVzIGVcXG90aW1lcyBTXmEiXSxbMCwzLCJ4IiwyXSxbMSw1LCJ4XFxvdGltZXMgU157LWF9XFxvdGltZXMgU15hIiwxXSxbMyw1LCJaXFxvdGltZXMgRVxcb3RpbWVzIFNeYVxcb3RpbWVzIFxccGhpX3stYSxhfSJdLFs1LDIsIlpcXG90aW1lcyBFXFxvdGltZXMgU15hXFxvdGltZXMgU157LWF9XFxvdGltZXMgZVxcb3RpbWVzIFNeYSIsMl0sWzYsNSwiWlxcb3RpbWVzIEVcXG90aW1lcyBcXHBoaV97YSwtYX1cXG90aW1lcyBTXmEiLDFdLFs2LDQsIlpcXG90aW1lcyBFXFxvdGltZXMgZVxcb3RpbWVzIFNeYSIsMV0sWzYsMywiIiwyLHsibGV2ZWwiOjIsInN0eWxlIjp7ImhlYWQiOnsibmFtZSI6Im5vbmUifX19XSxbMiw0LCJaXFxvdGltZXMgRVxcb3RpbWVzIFxccGhpX3thLC1hfV57LTF9XFxvdGltZXMgRVxcb3RpbWVzIFNeYSJdLFs0LDMsIlpcXG90aW1lcyBcXG11XFxvdGltZXMgU15hIl1d
	\[\begin{tikzcd}
		{S^b} && {S^b\otimes S^{-a}\otimes S^a} \\
		&& {Z\otimes E\otimes S^a \otimes S^{-a}\otimes S^a} && {Z\otimes E\otimes S^a\otimes S^{-a}\otimes E\otimes S^a} \\
		{Z\otimes E\otimes S^a} && {Z\otimes E\otimes S^a} \\
		&& {Z\otimes E\otimes E\otimes S^a}
		\arrow["\cong", from=1-1, to=1-3]
		\arrow["{x\otimes S^{-a}\otimes e\otimes S^a}", from=1-3, to=2-5]
		\arrow["x"', from=1-1, to=3-1]
		\arrow["{x\otimes S^{-a}\otimes S^a}"{description}, from=1-3, to=2-3]
		\arrow["{Z\otimes E\otimes S^a\otimes \phi_{-a,a}}", from=3-1, to=2-3]
		\arrow["{Z\otimes E\otimes S^a\otimes S^{-a}\otimes e\otimes S^a}"', from=2-3, to=2-5]
		\arrow["{Z\otimes E\otimes \phi_{a,-a}\otimes S^a}"{description}, from=3-3, to=2-3]
		\arrow["{Z\otimes E\otimes e\otimes S^a}"{description}, from=3-3, to=4-3]
		\arrow[Rightarrow, no head, from=3-3, to=3-1]
		\arrow["{Z\otimes E\otimes \phi_{a,-a}^{-1}\otimes E\otimes S^a}", from=2-5, to=4-3]
		\arrow["{Z\otimes \mu\otimes S^a}", from=4-3, to=3-1]
	\end{tikzcd}\]
	The top left trapezoid commutes since the isomorphism $S^b\xr\cong S^b\otimes S^{-a}\otimes S^a$ may be given as $S^b\otimes\phi_{-a,a}$ (see \autoref{unique_comp_Sas}), in which case the trapezoid commmutes by functoriality of $-\otimes-$. The triangle below that commutes by coherence for the $\phi_{a,b}$'s. The bottom left triangle commutes by unitality for $\mu$. The top right triangle commutes by functoriality of $-\otimes-$. Finally, the bottom right triangle commutes by functoriality of $-\otimes-$. It follows by unravelling definitions that the two outside compositions are $x$ and $\Phi_{Z,S^a}(\Psi(x))$, so indeed we have $\Phi_{Z,S^a}(\Psi(x))=x$ since the diagram commutes.

	On the other hand, suppose we are given a homogeneous pure tensor $x\otimes y$ in $\pi_*(Z\otimes E)\otimes_{\pi_*(E)}\pi_*(E\otimes S^a)$, so $x:S^b\to Z\otimes E$ and $y:S^c\to E\otimes S^a$ for some $b,c\in A$. Then we would like to show that $\Psi(\Phi_{Z,S^a}(x\otimes y))=x\otimes y$. Unravelling definitions, $\Psi(\Phi_{Z,S^a}(x\otimes y))$ is the homogeneous pure tensor $\wt{x y}\otimes\wt e$, where $\wt e$ is the map $e\otimes S^a:S^{a}\to E\otimes S^a$ is defined above, and by functoriality of $-\otimes-$, $\wt{xy}:S^{b+c-a}\to Z\otimes E$ is the composition
	% https://q.uiver.app/#q=WzAsNSxbMCwwLCJTXntiK2MtYX0iXSxbMCwxLCJTXmJcXG90aW1lcyBTXmNcXG90aW1lcyBTXnstYX0iXSxbMCwyLCJaXFxvdGltZXMgRVxcb3RpbWVzIEVcXG90aW1lcyBTXmFcXG90aW1lcyBTXnstYX0iXSxbMCwzLCJaXFxvdGltZXMgRVxcb3RpbWVzIFNeYVxcb3RpbWVzIFNeey1hfSJdLFswLDQsIlpcXG90aW1lcyBFIl0sWzAsMSwiXFxjb25nIl0sWzEsMiwieFxcb3RpbWVzIHlcXG90aW1lcyBTXnstYX0iXSxbMiwzLCJaXFxvdGltZXNcXG11XFxvdGltZXMgU15hXFxvdGltZXMgU157LWF9Il0sWzMsNCwiWlxcb3RpbWVzIEVcXG90aW1lc1xccGhpX3thLC1hfV57LTF9Il1d
	\[\begin{tikzcd}
		{S^{b+c-a}} \\
		{S^b\otimes S^c\otimes S^{-a}} \\
		{Z\otimes E\otimes E\otimes S^a\otimes S^{-a}} \\
		{Z\otimes E\otimes S^a\otimes S^{-a}} \\
		{Z\otimes E}
		\arrow["\cong", from=1-1, to=2-1]
		\arrow["{x\otimes y\otimes S^{-a}}", from=2-1, to=3-1]
		\arrow["{Z\otimes\mu\otimes S^a\otimes S^{-a}}", from=3-1, to=4-1]
		\arrow["{Z\otimes E\otimes\phi_{a,-a}^{-1}}", from=4-1, to=5-1]
	\end{tikzcd}\]
	Now, define $r\in\pi_{c-a}(E)$ to be the composition
	\[S^{c-a}\cong S^c\otimes S^{-a}\xr{y\otimes S^{-a}}E\otimes S^a\otimes S^{-a}\xr{E\otimes\phi_{a,-a}^{-1}}E.\]
	First, we claim that $x\cdot r=\wt{xy}$. To that end, consider the following diagram, where here we are again passing to a symmetric strict monoidal category:
	% https://q.uiver.app/#q=WzAsNixbMCwwLCJTXntiK2MtYX0iXSxbMSwwLCJTXntifVxcb3RpbWVzIFNeY1xcb3RpbWVzIFNeey1hfSJdLFsyLDAsIlpcXG90aW1lcyBFXFxvdGltZXMgRVxcb3RpbWVzIFNeYVxcb3RpbWVzIFNeey1hfSJdLFszLDAsIlpcXG90aW1lcyBFXFxvdGltZXMgU15hXFxvdGltZXMgU157LWF9Il0sWzMsMSwiWlxcb3RpbWVzIEUiXSxbMiwxLCJaXFxvdGltZXMgRVxcb3RpbWVzIEUiXSxbMCwxLCJcXGNvbmciXSxbMSwyLCJ4XFxvdGltZXMgeVxcb3RpbWVzIFNeey1hfSJdLFsyLDMsIlpcXG90aW1lcyBcXG11XFxvdGltZXMgU15hXFxvdGltZXMgU157LWF9Il0sWzIsNSwiWlxcb3RpbWVzIEVcXG90aW1lcyBFXFxvdGltZXMgXFxwaGlfe2EsLWF9XnstMX0iLDJdLFs1LDQsIlpcXG90aW1lcyBcXG11IiwyXSxbMyw0LCJaXFxvdGltZXMgRVxcb3RpbWVzIFxccGhpX3thLC1hfV57LTF9Il1d
	\[\begin{tikzcd}
		{S^{b+c-a}} & {S^{b}\otimes S^c\otimes S^{-a}} & {Z\otimes E\otimes E\otimes S^a\otimes S^{-a}} & {Z\otimes E\otimes S^a\otimes S^{-a}} \\
		&& {Z\otimes E\otimes E} & {Z\otimes E}
		\arrow["\cong", from=1-1, to=1-2]
		\arrow["{x\otimes y\otimes S^{-a}}", from=1-2, to=1-3]
		\arrow["{Z\otimes \mu\otimes S^a\otimes S^{-a}}", from=1-3, to=1-4]
		\arrow["{Z\otimes E\otimes E\otimes \phi_{a,-a}^{-1}}"', from=1-3, to=2-3]
		\arrow["{Z\otimes \mu}"', from=2-3, to=2-4]
		\arrow["{Z\otimes E\otimes \phi_{a,-a}^{-1}}", from=1-4, to=2-4]
	\end{tikzcd}\]
	Commutativity is functoriality of $-\otimes-$, which also tells us that the two outside compositions are $\wt{xy}$ (on top) and $x\cdot r$ (on the bottom), so they are equal as desired. On the other hand, we claim that $r\cdot\wt e=y$. To see this, consider the following diagram:
	% https://q.uiver.app/#q=WzAsOCxbMCwwLCJTXmMiXSxbMywwLCJTXmNcXG90aW1lcyBTXnstYX1cXG90aW1lcyBTXmEiXSxbMywxLCJFXFxvdGltZXMgU15hXFxvdGltZXMgU157LWF9XFxvdGltZXMgRVxcb3RpbWVzIFNeYSJdLFswLDMsIkVcXG90aW1lcyBFXFxvdGltZXMgU15hIl0sWzAsMiwiRVxcb3RpbWVzIFNeYSJdLFsxLDEsIkVcXG90aW1lcyBTXmFcXG90aW1lcyBTXnstYX1cXG90aW1lcyBTXmEiXSxbMywzLCJFXFxvdGltZXMgRVxcb3RpbWVzIFNeYSJdLFsxLDIsIkVcXG90aW1lcyBTXmEiXSxbMCwxLCJcXGNvbmciXSxbMSwyLCJ5XFxvdGltZXMgU157LWF9XFxvdGltZXMgZVxcb3RpbWVzIFNeYSJdLFszLDQsIlxcbXVcXG90aW1lcyBTXmEiXSxbMCw0LCJ5IiwyXSxbNSwyLCJFXFxvdGltZXMgU157YX1cXG90aW1lcyBTXnstYX1cXG90aW1lcyBlXFxvdGltZXMgU15hIiwyXSxbNSw0LCJFXFxvdGltZXMgU15hXFxvdGltZXMgXFxwaGlfey1hLGF9XnstMX0iLDFdLFs2LDMsIiIsMCx7ImxldmVsIjoyLCJzdHlsZSI6eyJoZWFkIjp7Im5hbWUiOiJub25lIn19fV0sWzIsNiwiRVxcb3RpbWVzIFxccGhpX3thLC1hfV57LTF9XFxvdGltZXMgRVxcb3RpbWVzIFNee2F9Il0sWzEsNSwieVxcb3RpbWVzIFNeey1hfVxcb3RpbWVzIFNeYSIsMV0sWzUsNywiRVxcb3RpbWVzIFxccGhpX3thLC1hfV57LTF9XFxvdGltZXMgU15hIl0sWzQsNywiIiwxLHsibGV2ZWwiOjIsInN0eWxlIjp7ImhlYWQiOnsibmFtZSI6Im5vbmUifX19XSxbNyw2LCJFXFxvdGltZXMgZVxcb3RpbWVzIFNeYSIsMV1d
	\[\begin{tikzcd}
		{S^c} &&& {S^c\otimes S^{-a}\otimes S^a} \\
		& {E\otimes S^a\otimes S^{-a}\otimes S^a} && {E\otimes S^a\otimes S^{-a}\otimes E\otimes S^a} \\
		{E\otimes S^a} & {E\otimes S^a} \\
		{E\otimes E\otimes S^a} &&& {E\otimes E\otimes S^a}
		\arrow["\cong", from=1-1, to=1-4]
		\arrow["{y\otimes S^{-a}\otimes e\otimes S^a}", from=1-4, to=2-4]
		\arrow["{\mu\otimes S^a}", from=4-1, to=3-1]
		\arrow["y"', from=1-1, to=3-1]
		\arrow["{E\otimes S^{a}\otimes S^{-a}\otimes e\otimes S^a}"', from=2-2, to=2-4]
		\arrow["{E\otimes S^a\otimes \phi_{-a,a}^{-1}}"{description}, from=2-2, to=3-1]
		\arrow[Rightarrow, no head, from=4-4, to=4-1]
		\arrow["{E\otimes \phi_{a,-a}^{-1}\otimes E\otimes S^{a}}", from=2-4, to=4-4]
		\arrow["{y\otimes S^{-a}\otimes S^a}"{description}, from=1-4, to=2-2]
		\arrow["{E\otimes \phi_{a,-a}^{-1}\otimes S^a}", from=2-2, to=3-2]
		\arrow[Rightarrow, no head, from=3-1, to=3-2]
		\arrow["{E\otimes e\otimes S^a}"{description}, from=3-2, to=4-4]
	\end{tikzcd}\]
	By \autoref{unique_comp_Sas}, we may take the top arrow to be $S^c\otimes \phi_{-a,a}$, in which case the top left triangle commutes by functoriality of $-\otimes-$. The bottom trapezoid commutes by unitality of $\mu$. Every other region commutes either by definition or by functoriality of $-\otimes-$. The top composition is $r\cdot\wt e$, so we have shown $r\cdot\wt e=y$ as desired. Thus, we have that
	\[\Psi(\Phi_{Z,S^a}(x\otimes y))=\wt{xy}\otimes\wt e=x\cdot r\otimes\wt e=x\otimes r\cdot\wt e=x\otimes y,\]
	as desired. Hence we have shown $\Psi$ is both a left and right inverse for $\Phi_{Z,S^a}$, so that indeed $S^a$ belongs to $\cE$ as desired.

	Now, we would like to show that given a distinguished triangle in $\cSH$
	\[X\xr fY\xr gW\xr h\Sigma X,\]
	if two of three of the objects $X$, $Y$, and $W$ belong to $\cE$, then so does the third. From now on, write $L^E_*$ to denote the functor from $\cSH$ to $A$-graded abelian groups sending $X\mapsto \pi_*(Z\otimes E)\otimes_{\pi_*(E)}\pi_*(E\otimes X)$. Then $\Phi_{Z,-}$ is a natural transformation $L_*^E\Rightarrow \pi_*(Z\otimes E\otimes-)=Z_*(E\otimes-)$. First, recall that it follows generally that in an adjointly tensor triangulated category (\autoref{adjointly_triangulated_defn}), which $\cSH$ is by \autoref{Sigma^a,Sigma^-a_adjoint_equiv}, given a distinguished triangle $(f,g,h)$ we have a long exact sequence (see \autoref{defn_exact} for the definition of an exact sequence in an additive category, and see \autoref{dist_tri_LES} for the explicit contruction of the LES associated to a distinguished triangle in an adjointly triangulated category):
	\[\Omega Y\xr{\Omega g}\Omega W\xr{\wt h}X\xr fY\xr gW\xr h\Sigma X\xr{\Sigma f}\Sigma Y,\]
	where $\wt h:\Omega W\to X$ is the adjoint of $h:W\to\Sigma X$. Then since $\cSH$ is further a tensor triangulated category (\autoref{tentri}), we have that the above sequence remains exact even after tensoring by $E$ on the left (see \autoref{LES_remains_exact_after_tensor} for details), so we have the following exact sequence in $\cSH$:
	\[E\otimes \Omega Y\xr{E\otimes \Omega g}E\otimes \Omega W\xr{E\otimes \wt h}E\otimes X\xr{E\otimes f}E\otimes Y\xr{E\otimes g}E\otimes W\xr{E\otimes h}E\otimes \Sigma X\xr{E\otimes \Sigma f}E\otimes \Sigma Y.\]
	We can then apply $[S^*,-]=\pi_*(-)$ to it, which yields the following exact sequence of $A$-graded abelian groups:
	\[E_*(\Omega Y)\xr{E_*(\Omega g)}E_*(\Omega W)\xr{E_*(\wt h)}E_*(X)\xr{E_*(f)}E_*(Y)\xr{E_*(g)}E_*(W)\xr{E_*(h)}E_*(\Sigma X)\xr{E_*(f)}E_*(\Sigma Y).\]
	Now, we can tensor this sequence with $\pi_*(Z\otimes E)$ on the left over $\pi_*(E)$, and since $\pi_*(Z\otimes E)$ is a flat right $\pi_*(E)$ module, we get that the top row in the following diagram is exact:
	% https://q.uiver.app/#q=WzAsMTQsWzAsMCwiTF8qXkUoXFxPbWVnYSBZKSJdLFsxLDAsIkxfKl5FKFxcT21lZ2EgVykiXSxbMiwwLCJMXypeRShYKSJdLFszLDAsIkxfKl5FKFkpIl0sWzQsMCwiTF8qXkUoVykiXSxbNSwwLCJMXypeRShcXFNpZ21hIFgpIl0sWzYsMCwiTF8qXkUoXFxTaWdtYSBZKSJdLFswLDEsIlpfKihFXFxvdGltZXNcXE9tZWdhIFkpIl0sWzEsMSwiWl8qKEVcXG90aW1lc1xcT21lZ2EgVykiXSxbMiwxLCJaXyooRVxcb3RpbWVzIFgpIl0sWzMsMSwiWl8qKEVcXG90aW1lcyBZKSJdLFs0LDEsIlpfKihFXFxvdGltZXMgVykiXSxbNSwxLCJaXyooRVxcb3RpbWVzIFxcU2lnbWEgWCkiXSxbNiwxLCJaXyooRVxcb3RpbWVzIFxcU2lnbWEgWSkiXSxbMCwxLCJMXypeRShcXE9tZWdhIGcpIl0sWzEsMiwiTF8qXkUoXFx3dCBoKSJdLFsyLDMsIkxfKl5FKGYpIl0sWzMsNCwiTF8qXkUoZykiXSxbNCw1LCJMXypeRShoKSJdLFs1LDYsIkxfKl5FKFxcU2lnbWEgZikiXSxbMCw3LCJcXFBoaV97WixcXE9tZWdhIFl9IiwyXSxbNyw4LCJaXyooRVxcb3RpbWVzIFxcT21lZ2EgZykiLDJdLFsxMSwxMiwiWl8qKEVcXG90aW1lcyBoKSIsMl0sWzEyLDEzLCJaXyooRVxcb3RpbWVzXFxTaWdtYSBmKSIsMl0sWzEsOCwiXFxQaGlfe1osXFxPbWVnYSBXfSIsMl0sWzksMTAsIlpfKihFXFxvdGltZXMgZikiLDJdLFsyLDksIlxcUGhpX3taLFh9IiwyXSxbMywxMCwiXFxQaGlfe1osWX0iLDJdLFsxMCwxMSwiWl8qKEVcXG90aW1lcyBnKSIsMl0sWzQsMTEsIlxcUGhpX3taLFd9IiwyXSxbNSwxMiwiXFxQaGlfe1osXFxTaWdtYSBYfSIsMl0sWzYsMTMsIlxcUGhpX3taLFxcU2lnbWEgWX0iLDJdLFs4LDksIlpfKihFXFxvdGltZXNcXHd0IGgpIiwyXV0=
	\[\begin{tikzcd}[column sep=tiny]
		{L_*^E(\Omega Y)} & {L_*^E(\Omega W)} & {L_*^E(X)} & {L_*^E(Y)} & {L_*^E(W)} & {L_*^E(\Sigma X)} & {L_*^E(\Sigma Y)} \\
		{Z_*(E\otimes\Omega Y)} & {Z_*(E\otimes\Omega W)} & {Z_*(E\otimes X)} & {Z_*(E\otimes Y)} & {Z_*(E\otimes W)} & {Z_*(E\otimes \Sigma X)} & {Z_*(E\otimes \Sigma Y)}
		\arrow["{L_*^E(\Omega g)}", from=1-1, to=1-2]
		\arrow["{L_*^E(\wt h)}", from=1-2, to=1-3]
		\arrow["{L_*^E(f)}", from=1-3, to=1-4]
		\arrow["{L_*^E(g)}", from=1-4, to=1-5]
		\arrow["{L_*^E(h)}", from=1-5, to=1-6]
		\arrow["{L_*^E(\Sigma f)}", from=1-6, to=1-7]
		\arrow["{\Phi_{Z,\Omega Y}}"', from=1-1, to=2-1]
		\arrow["{Z_*(E\otimes \Omega g)}"', from=2-1, to=2-2]
		\arrow["{Z_*(E\otimes h)}"', from=2-5, to=2-6]
		\arrow["{Z_*(E\otimes\Sigma f)}"', from=2-6, to=2-7]
		\arrow["{\Phi_{Z,\Omega W}}"', from=1-2, to=2-2]
		\arrow["{Z_*(E\otimes f)}"', from=2-3, to=2-4]
		\arrow["{\Phi_{Z,X}}"', from=1-3, to=2-3]
		\arrow["{\Phi_{Z,Y}}"', from=1-4, to=2-4]
		\arrow["{Z_*(E\otimes g)}"', from=2-4, to=2-5]
		\arrow["{\Phi_{Z,W}}"', from=1-5, to=2-5]
		\arrow["{\Phi_{Z,\Sigma X}}"', from=1-6, to=2-6]
		\arrow["{\Phi_{Z,\Sigma Y}}"', from=1-7, to=2-7]
		\arrow["{Z_*(E\otimes\wt h)}"', from=2-2, to=2-3]
	\end{tikzcd}\]
	This diagram further commutes by naturality of $\Phi_{Z,-}$. Now, supposing that two of three of $X$, $Y$, and $W$ belong to $\cE$, by  \autoref{Phi_Z,W_iso_implies_Phi_S^aZ,W_and_Phi_Z,S^aW_are_isos}, if $\Phi_{Z,V}$ is an isomorphism for some object $V$ in $\cSH$ then $\Phi_{Z,\Omega V}$ and $\Phi_{Z,\Sigma V}$ are. Thus by the five lemma, it follows that the middle three vertical arrows in the above diagram are necessarily all isomorphisms if any two of them are, so we have shown that $\cE$ is closed under two-of-three for exact triangles, as desired.

	Finally, it remains to show that $\cE$ is closed under arbitrary coproducts. Let $\{W_i\}_{i\in I}$ be a collection of objects in $\cE$ indexed by some (small) set $I$. Then we'd like to show that $W:=\bigoplus_iW_i$ belongs to $\cE$. First of all, note that $-\otimes-$ preserves arbitrary coproducts in each argument, as it has a right adjoint $F(-,-)$. Thus without loss of generality, given any object $X$ in $\cSH$, we may take $\bigoplus_iX\otimes W_i=X\otimes\bigoplus_iW_i$ (as $X\otimes\bigoplus_iW_i$ \emph{is} a coproduct of all the $X\otimes W_i$'s). Now, recall that we have chosen each $S^a$ to be a compact object (\autoref{defn_compact}), so that given any object $X$ and collection of objects $\{Y_i\}_{i\in I}$ in $\cSH$, if $Y:=\bigoplus_{i\in I}Y_i$, then the canonical map
	\[s_{X,Y_i}:\bigoplus_i X_*(Y_i)=\bigoplus_i[S^*,X\otimes Y_i]\to[S^*,\bigoplus_iX\otimes Y_i]=[S^*,X\otimes Y]=X_*(Y)\]
	is an isomorphism, natural in $Y_i$ for each $i$. Note in particular that $s_{E,W_i}$ is an isomorphism of left $\pi_*(E)$-modules. To see this, first note by additivity of $s_{E,W_i}$, it suffices to check that $s_{E,W_i}(r\cdot x)=r\cdot s_{E,W_i}(x)$ for each homogeneous $r\in\pi_*(E)$ and homogeneous $x\in E_*(W_i)$ for some $i$, as such $x$ generate $\bigoplus_i E_*(W_i)$ by definition. Then given $r:S^a\to E$ and $x:S^b\to E\otimes W_i$, consider the following diagram	
	% https://q.uiver.app/#q=WzAsOCxbMCwwLCJTXnthK2J9Il0sWzEsMCwiU15hXFxvdGltZXMgU15iIl0sWzIsMCwiRVxcb3RpbWVzIEVcXG90aW1lcyBXX2kiXSxbMywwLCJFXFxvdGltZXMgXFxiaWdvcGx1c19pKEVcXG90aW1lcyBXX2kpIl0sWzMsMSwiRVxcb3RpbWVzIEVcXG90aW1lcyBXIl0sWzMsMiwiRVxcb3RpbWVzIFciXSxbMiwzLCJFXFxvdGltZXMgV19pIl0sWzMsMywiXFxiaWdvcGx1c19pKEVcXG90aW1lcyBXX2kpIl0sWzAsMSwiXFxwaGlfe2EsYn0iXSxbMSwyLCJ4XFxvdGltZXMgeSJdLFsyLDMsIkVcXG90aW1lcyBcXGlvdGFfe0VcXG90aW1lcyBXX2l9Il0sWzMsNCwiIiwwLHsibGV2ZWwiOjIsInN0eWxlIjp7ImhlYWQiOnsibmFtZSI6Im5vbmUifX19XSxbMiw2LCJcXG11XFxvdGltZXMgV19pIiwyXSxbNiw3LCJcXGlvdGFfe0VcXG90aW1lcyBXX2l9IiwyXSxbNyw1LCIiLDEseyJsZXZlbCI6Miwic3R5bGUiOnsiaGVhZCI6eyJuYW1lIjoibm9uZSJ9fX1dLFs2LDUsIkVcXG90aW1lcyBcXGlvdGFfe1dfaX0iLDFdLFsyLDQsIkVcXG90aW1lcyBFXFxvdGltZXMgXFxpb3RhX3tXX2l9IiwxXSxbNCw1LCJcXG11XFxvdGltZXMgVyJdXQ==
	\[\begin{tikzcd}
		{S^{a+b}} & {S^a\otimes S^b} & {E\otimes E\otimes W_i} & {E\otimes \bigoplus_i(E\otimes W_i)} \\
		&&& {E\otimes E\otimes W} \\
		&&& {E\otimes W} \\
		&& {E\otimes W_i} & {\bigoplus_i(E\otimes W_i)}
		\arrow["{\phi_{a,b}}", from=1-1, to=1-2]
		\arrow["{x\otimes y}", from=1-2, to=1-3]
		\arrow["{E\otimes \iota_{E\otimes W_i}}", from=1-3, to=1-4]
		\arrow[Rightarrow, no head, from=1-4, to=2-4]
		\arrow["{\mu\otimes W_i}"', from=1-3, to=4-3]
		\arrow["{\iota_{E\otimes W_i}}"', from=4-3, to=4-4]
		\arrow[Rightarrow, no head, from=4-4, to=3-4]
		\arrow["{E\otimes \iota_{W_i}}"{description}, from=4-3, to=3-4]
		\arrow["{E\otimes E\otimes \iota_{W_i}}"{description}, from=1-3, to=2-4]
		\arrow["{\mu\otimes W}", from=2-4, to=3-4]
	\end{tikzcd}\]
	where $\iota_{E\otimes W_i}:E\otimes W_i\into\bigoplus_i(E\otimes W_i)$ and $\iota_{W_i}:W_i\into\bigoplus_iW_i$ are the maps determined by the definition of the coproduct. Commutativity of the two triangles is by the fact that $E\otimes-$ is colimit preserving. Commutativity of the trapezoid is functoriality of $-\otimes-$. Thus, since $s_{E,W_i}$ is a homomorphism of left $A$-graded $\pi_*(E)$-modules, the top right arrow in the following diagram is well-defined:
	% https://q.uiver.app/#q=WzAsNixbMSwwLCJaXyooRSlcXG90aW1lc197XFxwaV8qKEUpfVxcYmlnb3BsdXNfaUVfKihXX2kpIl0sWzAsMiwiXFxiaWdvcGx1c19pWl8qKEVcXG90aW1lcyBXX2kpIl0sWzIsMiwiWl8qKEVcXG90aW1lcyBXKSJdLFsyLDAsIlpfKihFKVxcb3RpbWVzX3tcXHBpXyooRSl9IEVfKihXKSJdLFswLDAsIlxcYmlnb3BsdXNfaVpfKihFKVxcb3RpbWVzX3tcXHBpXyooRSl9RV8qKFdfaSkiXSxbMSwyLCJaXyooXFxiaWdvcGx1c19pRVxcb3RpbWVzIFdfaSkiXSxbMCwzLCJaXyooRSlcXG90aW1lc197XFxwaV8qKEUpfSBzX3tFLFdfaX0iXSxbMywyLCJcXFBoaV97WixXfSJdLFs0LDAsIiIsMCx7ImxldmVsIjoyLCJzdHlsZSI6eyJoZWFkIjp7Im5hbWUiOiJub25lIn19fV0sWzEsNSwic197WixFXFxvdGltZXMgV19pfSJdLFs1LDIsIiIsMCx7ImxldmVsIjoyLCJzdHlsZSI6eyJoZWFkIjp7Im5hbWUiOiJub25lIn19fV0sWzQsMSwiXFxiaWdvcGx1c19pXFxQaGlfe1osV19pfSIsMl1d
	\begin{equation}\label{kunneth_iso_pt_3_diag}\begin{tikzcd}
		{\bigoplus_iZ_*(E)\otimes_{\pi_*(E)}E_*(W_i)} & {Z_*(E)\otimes_{\pi_*(E)}\bigoplus_iE_*(W_i)} & {Z_*(E)\otimes_{\pi_*(E)} E_*(W)} \\
		\\
		{\bigoplus_iZ_*(E\otimes W_i)} & {Z_*(\bigoplus_iE\otimes W_i)} & {Z_*(E\otimes W)}
		\arrow["{Z_*(E)\otimes_{\pi_*(E)} s_{E,W_i}}", from=1-2, to=1-3]
		\arrow["{\Phi_{Z,W}}", from=1-3, to=3-3]
		\arrow[Rightarrow, no head, from=1-1, to=1-2]
		\arrow["{s_{Z,E\otimes W_i}}", from=3-1, to=3-2]
		\arrow[Rightarrow, no head, from=3-2, to=3-3]
		\arrow["{\bigoplus_i\Phi_{Z,W_i}}"', from=1-1, to=3-1]
	\end{tikzcd}\end{equation}
	We wish to show this diagram commutes. Again, since each map here is a homomorphism, it suffices to chase generators. By definition, a generator of the top left element is a homogeneous pure tensor in $E_*(E)\otimes_{\pi_{*}(E)}E_*(W_i)$ for some $i$ in $I$. Given classes $x:S^a\to E\otimes E$ and $y:S^b\to E\otimes W_i$, consider the following diagram:
	% https://q.uiver.app/#q=WzAsOCxbMCwwLCJTXnthK2J9Il0sWzEsMCwiU15hXFxvdGltZXMgU15iIl0sWzIsMCwiWlxcb3RpbWVzIEVcXG90aW1lcyBFXFxvdGltZXMgV19pIl0sWzMsMCwiWlxcb3RpbWVzIEVcXG90aW1lcyBcXGJpZ29wbHVzX2lFXFxvdGltZXMgV19pIl0sWzMsMiwiWlxcb3RpbWVzIEVcXG90aW1lcyBXIl0sWzIsMSwiWlxcb3RpbWVzIEVcXG90aW1lcyBXX2kiXSxbMywxLCJaXFxvdGltZXMgRVxcb3RpbWVzIEVcXG90aW1lcyBXIl0sWzIsMiwiXFxiaWdvcGx1c19pWlxcb3RpbWVzIEVcXG90aW1lcyBXX2kiXSxbMCwxLCJcXHBoaV97YSxifSJdLFsxLDIsInhcXG90aW1lcyB5Il0sWzIsMywiWlxcb3RpbWVzIEVcXG90aW1lcyBcXGlvdGFfe0VcXG90aW1lcyBXX2l9Il0sWzIsNSwiWlxcb3RpbWVzIFxcbXVcXG90aW1lcyBXX2kiLDJdLFszLDYsIiIsMCx7ImxldmVsIjoyLCJzdHlsZSI6eyJoZWFkIjp7Im5hbWUiOiJub25lIn19fV0sWzYsNCwiWlxcb3RpbWVzIFxcbXVcXG90aW1lcyBXIl0sWzUsNywiXFxpb3RhX3taXFxvdGltZXMgRVxcb3RpbWVzIFdfaX0iLDJdLFs3LDQsIiIsMSx7ImxldmVsIjoyLCJzdHlsZSI6eyJoZWFkIjp7Im5hbWUiOiJub25lIn19fV0sWzUsNCwiWlxcb3RpbWVzIEVcXG90aW1lcyBcXGlvdGFfe1dfaX0iLDFdLFsyLDYsIlpcXG90aW1lcyBFXFxvdGltZXMgRVxcb3RpbWVzIFxcaW90YV97V19pfSIsMV1d
	\[\begin{tikzcd}
		{S^{a+b}} & {S^a\otimes S^b} & {Z\otimes E\otimes E\otimes W_i} & {Z\otimes E\otimes \bigoplus_iE\otimes W_i} \\
		&& {Z\otimes E\otimes W_i} & {Z\otimes E\otimes E\otimes W} \\
		&& {\bigoplus_iZ\otimes E\otimes W_i} & {Z\otimes E\otimes W}
		\arrow["{\phi_{a,b}}", from=1-1, to=1-2]
		\arrow["{x\otimes y}", from=1-2, to=1-3]
		\arrow["{Z\otimes E\otimes \iota_{E\otimes W_i}}", from=1-3, to=1-4]
		\arrow["{Z\otimes \mu\otimes W_i}"', from=1-3, to=2-3]
		\arrow[Rightarrow, no head, from=1-4, to=2-4]
		\arrow["{Z\otimes \mu\otimes W}", from=2-4, to=3-4]
		\arrow["{\iota_{Z\otimes E\otimes W_i}}"', from=2-3, to=3-3]
		\arrow[Rightarrow, no head, from=3-3, to=3-4]
		\arrow["{Z\otimes E\otimes \iota_{W_i}}"{description}, from=2-3, to=3-4]
		\arrow["{Z\otimes E\otimes E\otimes \iota_{W_i}}"{description}, from=1-3, to=2-4]
	\end{tikzcd}\]
	Unravelling definitions, the two outside compositions are the two ways to chase $x\otimes y$ around diagram (\ref{kunneth_iso_pt_3_diag}). The two triangles commute again by the fact that $-\otimes-$ preserves colimits in each argument. Commutativity of the inner parallelogram is functoriality of $-\otimes-$. Thus diagram (\ref{kunneth_iso_pt_3_diag}) tells us $\Phi_{Z,W}$ is an isomorphism, since $s_{E,W_i}$ and $s_{Z,E\otimes W_i}$ are isomorphisms, and $\Phi_{Z,W_i}$ is an isomorphism for each $i$ in $I$, meaning $\bigoplus_i\Phi_{W_i}$ is as well.

	Thus, we've shown the class $\cE$ of objects $W$ for which $\Phi_{Z,W}$ is an isomorphism contains the $S^a$'s, is closed under two-of-three for distinguished triangles, and is closed under arbitrary coproducts. Thus, it follows that $\cE$ contains the class of all cellular objects in $\cSH$, as desired.

	Now, suppose that $\pi_*(E\otimes W)$ is a flat left $\pi_*(E)$-module, then we'd like to show $\Phi_{Z,W}$ is an isomorphism for all cellular $Z$ in $\cSH$. Showing this is entirely analagous to above, so we only outline the argument. Let $\cE$ be the class of $Z$ in $\cSH$ such that $\Phi_{Z,W}$ is an isomorphism. Then in order to show $\cE$ contains every cellular object, it suffices to show it contains the $S^a$'s, is closed under two-of-three for distinguished triangles, and is closed under arbitrary coproducts. 
	
	To see $\cE$ contains the $S^a$'s, consider the map
	\[\Psi:\pi_*(S^a\otimes E\otimes W)\to \pi_*(S^a\otimes E)\otimes_{\pi_*(E)}\pi_*(E\otimes W)\]
	sending $x:S^b\to S^a\otimes E\otimes W$ to $\wt e\otimes\wt x$, where $\wt e\in\pi_a(S^a\otimes E)$ is the map $S^a\otimes e:S^a\to S^a\otimes E$, and $\wt x\in\pi_{b-a}(E\otimes W)$ is the map
	\[\wt x:S^{b-a}\xr{\phi_{-a,b}}S^{-a}\otimes S^b\xr{S^{-a}\otimes x}S^{-a}\otimes S^a\otimes E\otimes W\xr{\phi_{-a,a}^{-1}\otimes E\otimes W}E\otimes W.\]
	Then checking that $\Psi$ is a left and right inverse to $\Phi_{S^a,W}$ is entirely analagous, so that $S^a$ belongs to $\cE$ as desired.

	To see $\cE$ is closed under two-of-three for distinguished triangles, let
	\[X\xr fY\xr gZ\xr h\Sigma X\]
	be a distinguished triangle in $\cSH$. Then an analagous argument as above (using \autoref{dist_tri_LES} and \autoref{LES_remains_exact_after_tensor}) yields a long exact sequence of $A$-graded abelian groups
	% https://q.uiver.app/#q=WzAsNyxbMSwwLCJcXHBpXyooXFxPbWVnYSBZXFxvdGltZXMgRSkiXSxbMiwwLCJcXHBpXyooXFxPbWVnYSBaXFxvdGltZXMgRSkiXSxbMCwxLCJcXHBpXyooWFxcb3RpbWVzIEUpIl0sWzEsMSwiXFxwaV8qKFlcXG90aW1lcyBFKSJdLFsyLDEsIlxccGlfKihaXFxvdGltZXMgRSkiXSxbMCwyLCJcXHBpXyooXFxTaWdtYSBYXFxvdGltZXMgRSkiXSxbMSwyLCJcXHBpXyooXFxTaWdtYSBZXFxvdGltZXMgRSkiXSxbMCwxLCJcXHBpXyooXFxPbWVnYSBnXFxvdGltZXMgRSkiXSxbMSwyLCJcXHBpXyooXFx3dCBoXFxvdGltZXMgRSkiLDFdLFsyLDMsIlxccGlfKihmXFxvdGltZXMgRSkiLDJdLFszLDQsIlxccGlfKihnXFxvdGltZXMgRSkiXSxbNCw1LCJcXHBpXyooaFxcb3RpbWVzIEUpIiwxXSxbNSw2LCJcXHBpXyooXFxTaWdtYSBmXFxvdGltZXMgRSkiLDJdXQ==
	\[\begin{tikzcd}
		& {\pi_*(\Omega Y\otimes E)} & {\pi_*(\Omega Z\otimes E)} \\
		{\pi_*(X\otimes E)} & {\pi_*(Y\otimes E)} & {\pi_*(Z\otimes E)} \\
		{\pi_*(\Sigma X\otimes E)} & {\pi_*(\Sigma Y\otimes E)}
		\arrow["{\pi_*(\Omega g\otimes E)}", from=1-2, to=1-3]
		\arrow["{\pi_*(\wt h\otimes E)}"{description}, from=1-3, to=2-1]
		\arrow["{\pi_*(f\otimes E)}"', from=2-1, to=2-2]
		\arrow["{\pi_*(g\otimes E)}", from=2-2, to=2-3]
		\arrow["{\pi_*(h\otimes E)}"{description}, from=2-3, to=3-1]
		\arrow["{\pi_*(\Sigma f\otimes E)}"', from=3-1, to=3-2]
	\end{tikzcd}\]
	Then since $\pi_*(E\otimes W)$ is a flat left $\pi_*(E)$-module, we can tensor the above long exact sequence with $\pi_*(E\otimes W)$ on the right to obtain a long exact sequence which fits in the left column of the following commuting diagram:
	% https://q.uiver.app/#q=WzAsMTQsWzAsMCwiUl5FXyooXFxPbWVnYSBZKSJdLFswLDEsIlJeRV8qKFxcT21lZ2EgWikiXSxbMCwyLCJSXkVfKihYKSJdLFswLDMsIlJeRV8qKFkpIl0sWzAsNCwiUl5FXyooWikiXSxbMCw1LCJSXkVfKihcXFNpZ21hIFgpIl0sWzAsNiwiUl5FXyooXFxTaWdtYSBZKSJdLFsxLDAsIlxccGlfKihcXE9tZWdhIFlcXG90aW1lcyBFXFxvdGltZXMgVykiXSxbMSwxLCJcXHBpXyooXFxPbWVnYSBaXFxvdGltZXMgRVxcb3RpbWVzIFcpIl0sWzEsMiwiXFxwaV8qKFhcXG90aW1lcyBFXFxvdGltZXMgVykiXSxbMSwzLCJcXHBpXyooWVxcb3RpbWVzIEVcXG90aW1lcyBXKSJdLFsxLDQsIlxccGlfKihaXFxvdGltZXMgRVxcb3RpbWVzIFcpIl0sWzEsNSwiXFxwaV8qKFxcU2lnbWEgWFxcb3RpbWVzIEVcXG90aW1lcyBXKSJdLFsxLDYsIlxccGlfKihcXFNpZ21hIFlcXG90aW1lcyBFXFxvdGltZXMgVykiXSxbMCwxLCJSXypeRShcXE9tZWdhIGcpIiwyXSxbMSwyLCJSXypeRShcXHd0IGgpIiwyXSxbMiwzLCJSXypeRShmKSIsMl0sWzMsNCwiUl8qXkUoZykiLDJdLFs0LDUsIlJfKl5FKGgpIiwyXSxbNSw2LCJSXypeRShcXFNpZ21hIGYpIiwyXSxbMCw3LCJcXFBoaV97XFxPbWVnYSBZLFd9Il0sWzcsOCwiXFxwaV8qKFxcT21lZ2EgZ1xcb3RpbWVzIEVcXG90aW1lcyBXKSJdLFs4LDksIlxccGlfKihcXHd0IGhcXG90aW1lcyBFXFxvdGltZXMgVykiXSxbOSwxMCwiXFxwaV8qKGZcXG90aW1lcyBFXFxvdGltZXMgVykiXSxbMTAsMTEsIlxccGlfKihnXFxvdGltZXMgRVxcb3RpbWVzIFcpIl0sWzExLDEyLCJcXHBpXyooaFxcb3RpbWVzIEVcXG90aW1lcyBXKSJdLFsxMiwxMywiXFxwaV8qKFxcU2lnbWEgZlxcb3RpbWVzIEVcXG90aW1lcyBXKSJdLFsxLDgsIlxcUGhpX3tcXE9tZWdhIFosV30iXSxbMiw5LCJcXFBoaV97WCxXfSJdLFszLDEwLCJcXFBoaV97WSxXfSJdLFs0LDExLCJcXFBoaV97WixXfSJdLFs1LDEyLCJcXFBoaV97XFxTaWdtYSBYLFd9Il0sWzYsMTMsIlxcUGhpX3tcXFNpZ21hIFksV30iXV0=
	\[\begin{tikzcd}[row sep=small]
		{R^E_*(\Omega Y)} & {\pi_*(\Omega Y\otimes E\otimes W)} \\
		{R^E_*(\Omega Z)} & {\pi_*(\Omega Z\otimes E\otimes W)} \\
		{R^E_*(X)} & {\pi_*(X\otimes E\otimes W)} \\
		{R^E_*(Y)} & {\pi_*(Y\otimes E\otimes W)} \\
		{R^E_*(Z)} & {\pi_*(Z\otimes E\otimes W)} \\
		{R^E_*(\Sigma X)} & {\pi_*(\Sigma X\otimes E\otimes W)} \\
		{R^E_*(\Sigma Y)} & {\pi_*(\Sigma Y\otimes E\otimes W)}
		\arrow["{R_*^E(\Omega g)}"', from=1-1, to=2-1]
		\arrow["{R_*^E(\wt h)}"', from=2-1, to=3-1]
		\arrow["{R_*^E(f)}"', from=3-1, to=4-1]
		\arrow["{R_*^E(g)}"', from=4-1, to=5-1]
		\arrow["{R_*^E(h)}"', from=5-1, to=6-1]
		\arrow["{R_*^E(\Sigma f)}"', from=6-1, to=7-1]
		\arrow["{\Phi_{\Omega Y,W}}", from=1-1, to=1-2]
		\arrow["{\pi_*(\Omega g\otimes E\otimes W)}", from=1-2, to=2-2]
		\arrow["{\pi_*(\wt h\otimes E\otimes W)}", from=2-2, to=3-2]
		\arrow["{\pi_*(f\otimes E\otimes W)}", from=3-2, to=4-2]
		\arrow["{\pi_*(g\otimes E\otimes W)}", from=4-2, to=5-2]
		\arrow["{\pi_*(h\otimes E\otimes W)}", from=5-2, to=6-2]
		\arrow["{\pi_*(\Sigma f\otimes E\otimes W)}", from=6-2, to=7-2]
		\arrow["{\Phi_{\Omega Z,W}}", from=2-1, to=2-2]
		\arrow["{\Phi_{X,W}}", from=3-1, to=3-2]
		\arrow["{\Phi_{Y,W}}", from=4-1, to=4-2]
		\arrow["{\Phi_{Z,W}}", from=5-1, to=5-2]
		\arrow["{\Phi_{\Sigma X,W}}", from=6-1, to=6-2]
		\arrow["{\Phi_{\Sigma Y,W}}", from=7-1, to=7-2]
	\end{tikzcd}\]
	where $R_*^E$ denotes the functor from $\cSH$ to $A$-graded abelian groups sending $X\mapsto\pi_*(X\otimes E)\otimes_{\pi_*(E)}\pi_*(E\otimes W)$, so that $\Phi_{-,W}$ is a natural homomorphism $R_*^E(-)\Rightarrow \pi_*(-\otimes E\otimes W)$. Then finally by \autoref{Phi_Z,W_iso_implies_Phi_S^aZ,W_and_Phi_Z,S^aW_are_isos} and the five lemma, if any two of three of the middle three horizontal arrows are isomorphisms, then all three of the horizontal arrows are isomorphisms, as desired.

	Finally, in order to show $\cE$ is closed under arbitrary coproducts, suppose we have a collection of objects $\{Z_i\}_{i\in I}$ in $\cE$ indexed by some (small) set $\cE$. Then we'd like to show $Z:=\bigoplus_{i\in I}Z_i$ also belongs to $\cE$. First note that since the $S^a$'s are compact, for any object $Y$ we have isomorphisms
	\[s_{Z_i,Y}:\bigoplus_{i}{Z_i}_*(Y)=\bigoplus_i[S^*,Z_i\otimes Y]\to[S^*,\bigoplus_i(Z_i\otimes Y)]=[S^*,Z\otimes Y]=Z_*(Y).\]
	It is straightforward to verify that $s_{Z_i,E}:\bigoplus_i{Z_i}_*(E)\to Z_*(E)$ is not only an isomorphism of abelian groups, but an isomorphism of right $A$-graded $\pi_*(E)$-modules, so that the top arrow in the following diagram is well-defined:
	% https://q.uiver.app/#q=WzAsNSxbMSwwLCJcXGJpZ29wbHVzX2lcXGxlZnQoe1pfaX1fKihFKVxccmlnaHQpXFxvdGltZXNfe1xccGlfKihFKX1FXyooV19pKSJdLFswLDIsIlxcYmlnb3BsdXNfaXtaX2l9XyooRVxcb3RpbWVzIFcpIl0sWzIsMiwiWl8qKEVcXG90aW1lcyBXKSJdLFsyLDAsIlpfKihFKVxcb3RpbWVzX3tcXHBpXyooRSl9IEVfKihXKSJdLFswLDAsIlxcYmlnb3BsdXNfaVxcbGVmdCh7Wl9pfV8qKEUpXFxvdGltZXNfe1xccGlfKihFKX1FXyooVylcXHJpZ2h0KSJdLFswLDMsInNfe1pfaSxFfSJdLFszLDIsIlxcUGhpX3taLFd9Il0sWzQsMCwiIiwwLHsibGV2ZWwiOjIsInN0eWxlIjp7ImhlYWQiOnsibmFtZSI6Im5vbmUifX19XSxbNCwxLCJcXGJpZ29wbHVzX2lcXFBoaV97Wl9pLFd9IiwyXSxbMSwyLCJzX3taX2ksRVxcb3RpbWVzIFd9Il1d
	\[\begin{tikzcd}
		{\bigoplus_i\left({Z_i}_*(E)\otimes_{\pi_*(E)}E_*(W)\right)} & {\bigoplus_i\left({Z_i}_*(E)\right)\otimes_{\pi_*(E)}E_*(W_i)} & {Z_*(E)\otimes_{\pi_*(E)} E_*(W)} \\
		\\
		{\bigoplus_i{Z_i}_*(E\otimes W)} && {Z_*(E\otimes W)}
		\arrow["{s_{Z_i,E}}", from=1-2, to=1-3]
		\arrow["{\Phi_{Z,W}}", from=1-3, to=3-3]
		\arrow[Rightarrow, no head, from=1-1, to=1-2]
		\arrow["{\bigoplus_i\Phi_{Z_i,W}}"', from=1-1, to=3-1]
		\arrow["{s_{Z_i,E\otimes W}}", from=3-1, to=3-3]
	\end{tikzcd}\]
	Then a simple diagram chase yields the diagram commutes, so that $\Phi_{Z,W}$ is an isomorphism, assuming all the $\Phi_{Z_i,W}$'s are.
\end{proof}

\subsection{Modules over monoid objects in \texorpdfstring{$\cSH$}{SH}}

\begin{lemma}\label{suspension_of_module_object_is_module_object}
	Let $(E,\mu,e)$ be a monoid object in $\cSH$, and suppose $(N,\kappa)$ is a module object over $E$ (\autoref{left_module_object}). Then for all $a\in A$, the $a^\text{th}$ suspension $\Sigma^a N$ of $N$ is canonically an $E$-module object, with action map given by
	\[\kappa^a:E\otimes\Sigma^aN=E\otimes S^a\otimes N\xr{\tau\otimes N}S^a\otimes E\otimes N\xr{S^a\otimes\kappa}S^a\otimes N=\Sigma^aN.\]
	Furthermore, given an $E$-module homomorphism $f:(N,\kappa)\to (N',\kappa')$, $\Sigma^af:\Sigma^aN\to\Sigma^aN'$ is likewise an $E$-module homomorphism.
\end{lemma}
\begin{proof}
	In this proof, we are assuming that unitality and associativity hold up to strict equality, by the coherence theorem for monoidal categories. In order to show $(\Sigma^a N,\kappa^a)$ is a module object over $E$, we need to show $\kappa^a$ makes the two coherence diagrams in \autoref{left_module_object} commute. First, to see the first diagram commutes, consider the following diagram:
	% https://q.uiver.app/#q=WzAsNCxbMCwwLCJTXmFcXG90aW1lcyBOIl0sWzIsMCwiRVxcb3RpbWVzIFNeYVxcb3RpbWVzIE4iXSxbMiwyLCJTXmFcXG90aW1lcyBOIl0sWzIsMSwiU15hXFxvdGltZXMgRVxcb3RpbWVzIE4iXSxbMCwxLCJlXFxvdGltZXMgU15hXFxvdGltZXMgTiJdLFsxLDMsIlxcdGF1XFxvdGltZXMgTiJdLFszLDIsIlNeYVxcb3RpbWVzIFxca2FwcGEiXSxbMCwyLCIiLDIseyJsZXZlbCI6Miwic3R5bGUiOnsiaGVhZCI6eyJuYW1lIjoibm9uZSJ9fX1dLFswLDMsIlNeYVxcb3RpbWVzIGVcXG90aW1lcyBOIiwxXV0=
	\[\begin{tikzcd}
		{S^a\otimes N} && {E\otimes S^a\otimes N} \\
		&& {S^a\otimes E\otimes N} \\
		&& {S^a\otimes N}
		\arrow["{e\otimes S^a\otimes N}", from=1-1, to=1-3]
		\arrow["{\tau\otimes N}", from=1-3, to=2-3]
		\arrow["{S^a\otimes \kappa}", from=2-3, to=3-3]
		\arrow[Rightarrow, no head, from=1-1, to=3-3]
		\arrow["{S^a\otimes e\otimes N}"{description}, from=1-1, to=2-3]
	\end{tikzcd}\]
	The top inner triangle commutes by coherence for a symmetric monoidal category, and the bottom inner triangle commutes by the coherence condition for $\kappa$. To see the other module condition for $\wt\kappa$, consider the following diagram:
	% https://q.uiver.app/#q=WzAsOCxbMCwwLCJFXFxvdGltZXMgRVxcb3RpbWVzIFNeYVxcb3RpbWVzIE4iXSxbMiwwLCJFXFxvdGltZXMgU15hXFxvdGltZXMgTiJdLFsyLDEsIlNeYVxcb3RpbWVzIEVcXG90aW1lcyBOIl0sWzIsMiwiU15hXFxvdGltZXMgTiJdLFswLDEsIkVcXG90aW1lcyBTXmFcXG90aW1lcyBFXFxvdGltZXMgTiJdLFswLDIsIkVcXG90aW1lcyBTXmFcXG90aW1lcyBOIl0sWzEsMiwiU15hXFxvdGltZXMgRVxcb3RpbWVzIE4iXSxbMSwxLCJTXmFcXG90aW1lcyBFXFxvdGltZXMgRVxcb3RpbWVzIE4iXSxbMCwxLCJcXG11XFxvdGltZXMgU15hXFxvdGltZXMgTiJdLFsxLDIsIlxcdGF1XFxvdGltZXMgTiJdLFsyLDMsIlNeYVxcb3RpbWVzIFxca2FwcGEiXSxbMCw0LCJFXFxvdGltZXMgXFx0YXVcXG90aW1lcyBOIiwyXSxbNCw1LCJFXFxvdGltZXMgU15hXFxvdGltZXMgXFxrYXBwYSIsMl0sWzUsNiwiXFx0YXVcXG90aW1lcyBOIl0sWzYsMywiU15hXFxvdGltZXMgXFxrYXBwYSJdLFswLDcsIlxcdGF1X3tFXFxvdGltZXMgRSxTXmF9XFxvdGltZXMgTiIsMV0sWzcsMiwiU15hXFxvdGltZXMgXFxtdVxcb3RpbWVzIE4iXSxbNyw2LCJTXmFcXG90aW1lcyBFXFxvdGltZXMgXFxrYXBwYSIsMV0sWzQsNywiXFx0YXVcXG90aW1lcyBFXFxvdGltZXMgTiIsMl1d
	\[\begin{tikzcd}
		{E\otimes E\otimes S^a\otimes N} && {E\otimes S^a\otimes N} \\
		{E\otimes S^a\otimes E\otimes N} & {S^a\otimes E\otimes E\otimes N} & {S^a\otimes E\otimes N} \\
		{E\otimes S^a\otimes N} & {S^a\otimes E\otimes N} & {S^a\otimes N}
		\arrow["{\mu\otimes S^a\otimes N}", from=1-1, to=1-3]
		\arrow["{\tau\otimes N}", from=1-3, to=2-3]
		\arrow["{S^a\otimes \kappa}", from=2-3, to=3-3]
		\arrow["{E\otimes \tau\otimes N}"', from=1-1, to=2-1]
		\arrow["{E\otimes S^a\otimes \kappa}"', from=2-1, to=3-1]
		\arrow["{\tau\otimes N}", from=3-1, to=3-2]
		\arrow["{S^a\otimes \kappa}", from=3-2, to=3-3]
		\arrow["{\tau_{E\otimes E,S^a}\otimes N}"{description}, from=1-1, to=2-2]
		\arrow["{S^a\otimes \mu\otimes N}", from=2-2, to=2-3]
		\arrow["{S^a\otimes E\otimes \kappa}"{description}, from=2-2, to=3-2]
		\arrow["{\tau\otimes E\otimes N}"', from=2-1, to=2-2]
	\end{tikzcd}\]
	The top left triangle commutes by coherence for a symmetric monoidal category. The bottom left rectangle and top right trapezoid commute by naturality of $\tau$. Finally, the bottom right square commutes by the coherence condition for $\kappa$.

	Thus, we have shown that $\Sigma^aN$ is indeed an object in $E\text-\Mod$, as desired. Now let $f:(N,\kappa)\to(N',\kappa')$ be a morphism in $E\text-\Mod$, we would like to show $\Sigma^af:\Sigma^aN\to\Sigma^aN'$ is also a homomorphism of $E$-modules. To that end, consider the following diagram:
	% https://q.uiver.app/#q=WzAsNixbMCwwLCJFXFxvdGltZXMgU15hXFxvdGltZXMgTiJdLFsxLDAsIkVcXG90aW1lcyBTXmFcXG90aW1lcyBOJyJdLFswLDEsIlNeYVxcb3RpbWVzIEVcXG90aW1lcyBOIl0sWzEsMSwiU15hXFxvdGltZXMgRVxcb3RpbWVzIE4nIl0sWzAsMiwiU15hXFxvdGltZXMgTiJdLFsxLDIsIlNeYVxcb3RpbWVzIE4nIl0sWzAsMSwiRVxcb3RpbWVzIFNeYVxcb3RpbWVzIGYiXSxbMCwyLCJcXHRhdVxcb3RpbWVzIE4iLDJdLFsyLDMsIlNeYVxcb3RpbWVzIEVcXG90aW1lcyBmIl0sWzEsMywiXFx0YXVcXG90aW1lcyBOJyJdLFsyLDQsIlNeYVxcb3RpbWVzIFxca2FwcGEiLDJdLFszLDUsIlNeYVxcb3RpbWVzIFxca2FwcGEnIl0sWzQsNSwiU15hXFxvdGltZXMgZiIsMl1d
	\[\begin{tikzcd}
		{E\otimes S^a\otimes N} & {E\otimes S^a\otimes N'} \\
		{S^a\otimes E\otimes N} & {S^a\otimes E\otimes N'} \\
		{S^a\otimes N} & {S^a\otimes N'}
		\arrow["{E\otimes S^a\otimes f}", from=1-1, to=1-2]
		\arrow["{\tau\otimes N}"', from=1-1, to=2-1]
		\arrow["{S^a\otimes E\otimes f}", from=2-1, to=2-2]
		\arrow["{\tau\otimes N'}", from=1-2, to=2-2]
		\arrow["{S^a\otimes \kappa}"', from=2-1, to=3-1]
		\arrow["{S^a\otimes \kappa'}", from=2-2, to=3-2]
		\arrow["{S^a\otimes f}"', from=3-1, to=3-2]
	\end{tikzcd}\]
	The top rectangle commutes by functoriality of $-\otimes-$, while the bottom commutes since $f$ is an $E$-module homomorphism. Thus, $S^a\otimes f=\Sigma^af$ is an $E$-module homomorphism, as desired.
\end{proof}


\begin{proposition}
	Let $(E,\mu,e)$ be a monoid object in $\cSH$. Then the category $E\text-\Mod$ is canonically an $A$-graded category (\autoref{A-graded_cat_defn}), and the forgetful functor $E\text-\Mod\to\cSH$ is strictly $A$-graded (\autoref{A-graded_functor_defn}).
\end{proposition}
\begin{proof}
	In \autoref{suspension_of_module_object_is_module_object}, we showed that $E\text-\Mod$ is closed under the functor $\Sigma^a$ for all $a\in A$. Thus, in order to show $E\text-\Mod$ is an $A$-graded category, it suffices to show that for all objects $N$ in $E\text-\Mod$ that the maps
	\[\lambda_N:\Sigma^0N=S\otimes N\to N\qquad\text{and}\qquad\phi_{a,b}\otimes N:\Sigma^{a+b}N=S^{a+b}\otimes N\to S^a\otimes S^b\otimes N\]
	belong to $E\text-\Mod$. To see the former is, consider the following diagram:
	% https://q.uiver.app/#q=WzAsNSxbMCwwLCJFXFxvdGltZXMgU1xcb3RpbWVzIE4iXSxbMSwwLCJFXFxvdGltZXMgTiJdLFswLDEsIlNcXG90aW1lcyBFXFxvdGltZXMgTiJdLFswLDIsIlNcXG90aW1lcyBOIl0sWzEsMiwiTiJdLFswLDEsIkVcXG90aW1lc1xcbGFtYmRhX04iXSxbMCwyLCJcXHRhdVxcb3RpbWVzIE4iLDJdLFsyLDMsIlNcXG90aW1lc1xca2FwcGEiLDJdLFsxLDQsIlxca2FwcGEiXSxbMyw0LCJcXGxhbWJkYV9OIiwyXSxbMiwxLCJcXGxhbWJkYV97RVxcb3RpbWVzIE59IiwxXV0=
	\[\begin{tikzcd}
		{E\otimes S\otimes N} & {E\otimes N} \\
		{S\otimes E\otimes N} \\
		{S\otimes N} & N
		\arrow["{E\otimes\lambda_N}", from=1-1, to=1-2]
		\arrow["{\tau\otimes N}"', from=1-1, to=2-1]
		\arrow["S\otimes\kappa"', from=2-1, to=3-1]
		\arrow["\kappa", from=1-2, to=3-2]
		\arrow["{\lambda_N}"', from=3-1, to=3-2]
		\arrow["{\lambda_{E\otimes N}}"{description}, from=2-1, to=1-2]
	\end{tikzcd}\]
	The top left triangle commutes by coherence for a symmetric monoidal category, while the bottom triangle commutes by naturality of $\lambda$. To see $\phi_{a,b}\otimes N$ is an $E$-module homomorphism, consider the following diagram
	\[\begin{tikzcd}
		{E\otimes S^{a+b}\otimes N} && {E\otimes S^a\otimes S^b\otimes N} \\
		&& {S^a\otimes E\otimes S^b\otimes N} \\
		{S^{a+b}\otimes E\otimes N} && {S^a\otimes S^b\otimes E\otimes N} \\
		{S^{a+b}\otimes N} && {S^a\otimes S^b\otimes N}
		\arrow["{E\otimes \phi_{a,b}\otimes N}", from=1-1, to=1-3]
		\arrow["{\tau\otimes S^b\otimes N}", from=1-3, to=2-3]
		\arrow["{S^a\otimes\tau\otimes N}", from=2-3, to=3-3]
		\arrow["{S^a\otimes S^b\otimes\kappa}", from=3-3, to=4-3]
		\arrow["{\tau\otimes N}"', from=1-1, to=3-1]
		\arrow["{S^{a+b}\otimes \kappa}"', from=3-1, to=4-1]
		\arrow["{\phi_{a,b}\otimes N}", from=4-1, to=4-3]
		\arrow["{\phi_{a,b}\otimes E\otimes N}"', from=3-1, to=3-3]
		\arrow["{\tau_{E,S^a\otimes S^b}\otimes N}"'{pos=0.533}, shift right=10, curve={height=20pt}, from=1-3, to=3-3]
	\end{tikzcd}\]
	The top region commutes by coherence and naturality for the symmetries, while the bottom region commutes by functoriality of $-\otimes-$. Thus, $\phi_{a,b}\otimes N$ does indeed belong to $E\text-\Mod$, as desired.

	To recap, we have shown there are shift functors $\Sigma^a:E\text-\Mod\to E\text-\Mod$ and natural isomorphisms $\lambda:\Sigma^0\cong\mathrm{Id}_{E\text-\Mod}$ and $\phi_{a,b}:\Sigma^{a+b}\cong\Sigma^a\Sigma^b$ which satisfy the required coherence conditions. Furthermore, all these data are strictly preserved by the forgetful functor $E\text-\Mod\to\cSH$, so that it is strict $A$-graded, as desired.
\end{proof}


\begin{lemma}\label{free_susp_is_susp_of_free}
	Given a monoid object $(E,\mu,e)$ in $\cSH$, an object $X$ in $\cSH$, and some $a\in A$, the suspension of the free module $\Sigma^a(E\otimes X)$ is naturally isomorphic as an $E$-module object to the free $E$-module $E\otimes\Sigma^aX$.
\end{lemma}
\begin{proof}
	It suffices to show the map $S^a\otimes E\otimes X\xr{\tau\otimes X}E\otimes S^a\otimes X$ is a homomorphism of $E$-module objects, as we know it is an isomorphism and natural in $X$. To that end, consider the following diagram:
	% https://q.uiver.app/#q=WzAsNSxbMCwwLCJFXFxvdGltZXMgU15hXFxvdGltZXMgRVxcb3RpbWVzIFgiXSxbMCwxLCJTXmFcXG90aW1lcyBFXFxvdGltZXMgRVxcb3RpbWVzIFgiXSxbMCwyLCJTXmFcXG90aW1lcyBFXFxvdGltZXMgWCJdLFsyLDIsIkVcXG90aW1lcyBTXmFcXG90aW1lcyBYIl0sWzIsMCwiRVxcb3RpbWVzIEVcXG90aW1lcyBTXmFcXG90aW1lcyBYIl0sWzAsMSwiXFx0YXVcXG90aW1lcyBFXFxvdGltZXMgWCIsMl0sWzEsMiwiU15hXFxvdGltZXMgXFxtdVxcb3RpbWVzIFgiLDJdLFsyLDMsIlxcdGF1XFxvdGltZXMgWCJdLFswLDQsIkVcXG90aW1lcyBcXHRhdVxcb3RpbWVzIFgiXSxbNCwzLCJcXG11XFxvdGltZXMgU15hXFxvdGltZXMgWCJdLFsxLDQsIlxcdGF1X3tTXmEsRVxcb3RpbWVzIEV9XFxvdGltZXMgWCIsMl1d
	\[\begin{tikzcd}
		{E\otimes S^a\otimes E\otimes X} && {E\otimes E\otimes S^a\otimes X} \\
		{S^a\otimes E\otimes E\otimes X} \\
		{S^a\otimes E\otimes X} && {E\otimes S^a\otimes X}
		\arrow["{\tau\otimes E\otimes X}"', from=1-1, to=2-1]
		\arrow["{S^a\otimes \mu\otimes X}"', from=2-1, to=3-1]
		\arrow["{\tau\otimes X}", from=3-1, to=3-3]
		\arrow["{E\otimes \tau\otimes X}", from=1-1, to=1-3]
		\arrow["{\mu\otimes S^a\otimes X}", from=1-3, to=3-3]
		\arrow["{\tau_{S^a,E\otimes E}\otimes X}"', from=2-1, to=1-3]
	\end{tikzcd}\]
	The top triangle commutes by coherence for a symmetric monoidal category. The bottom trapezoid commutes by naturality of $\tau$. 
\end{proof}

\begin{proposition}\label{coproduct_of_E_modules_is_coproduct_in_E_mod}
	Let $(E,\mu,e)$ be a monoid object in $\cSH$, and suppose we have a family of $E$-module objects $(N_i,\kappa_i)$ indexed by some small set $I$. Then $N:=\bigoplus_{i\in I}N_i$ is canonically an $E$-module, with action map given by the composition
	\[\kappa:E\otimes\bigoplus_iN_i\xr\cong\bigoplus_i(E\otimes N_i)\xrightarrow{\bigoplus_i\kappa_i}\bigoplus_iN_i,\]
	where the first isomorphism is given by the fact that $E\otimes-$ preserves coproducts, since it is a left adjoint as $\cSH$ is monoidal closed. Furthermore, $N$ is the coproduct of all the $N_i$'s in $E\text-\Mod$, so that $E\text-\Mod$ has arbitrary coproducts.
\end{proposition}
\begin{proof}
	We need to show the action map $\kappa$ makes the diagrams in \autoref{left_module_object} commute. To see the first (unitality) diagram commutes, consider the following diagram:
	% https://q.uiver.app/#q=WzAsNCxbMCwwLCJcXGJpZ29wbHVzX2lOX2kiXSxbNCwwLCJFXFxvdGltZXNcXGJpZ29wbHVzX2lOX2kiXSxbNCwyLCJcXGJpZ29wbHVzX2koRVxcb3RpbWVzIE5faSkiXSxbNCw0LCJcXGJpZ29wbHVzX2lOX2kiXSxbMCwxLCJlXFxvdGltZXNcXGJpZ29wbHVzX2lOX2kiXSxbMSwyLCJcXGNvbmciXSxbMiwzLCJcXGJpZ29wbHVzX2lcXGthcHBhX2kiXSxbMCwzLCIiLDIseyJsZXZlbCI6Miwic3R5bGUiOnsiaGVhZCI6eyJuYW1lIjoibm9uZSJ9fX1dLFswLDIsIlxcYmlnb3BsdXNfaShlXFxvdGltZXMgTl9pKSIsMV1d
	\[\begin{tikzcd}
		{\bigoplus_iN_i} &&&& {E\otimes\bigoplus_iN_i} \\
		\\
		&&&& {\bigoplus_i(E\otimes N_i)} \\
		\\
		&&&& {\bigoplus_iN_i}
		\arrow["{e\otimes\bigoplus_iN_i}", from=1-1, to=1-5]
		\arrow["\cong", from=1-5, to=3-5]
		\arrow["{\bigoplus_i\kappa_i}", from=3-5, to=5-5]
		\arrow[Rightarrow, no head, from=1-1, to=5-5]
		\arrow["{\bigoplus_i(e\otimes N_i)}"{description}, from=1-1, to=3-5]
	\end{tikzcd}\]
	The top triangle commutes by additivity of $E\otimes-$ The bottom triangle commutes by unitality of each of the $\kappa_i$'s. To see the second coherence diagram commutes, consider the following diagram:
	% https://q.uiver.app/#q=WzAsOCxbMCwwLCJFXFxvdGltZXMgRVxcb3RpbWVzXFxiaWdvcGx1c19pTl9pIl0sWzIsMCwiRVxcb3RpbWVzIFxcYmlnb3BsdXNfaU5faSJdLFswLDEsIkVcXG90aW1lc1xcYmlnb3BsdXNfaShFXFxvdGltZXMgTl9pKSJdLFswLDIsIkVcXG90aW1lc1xcYmlnb3BsdXNfaU5faSJdLFsxLDIsIlxcYmlnb3BsdXNfaShFXFxvdGltZXMgTl9pKSJdLFsyLDIsIlxcYmlnb3BsdXNfaU5faSJdLFsyLDEsIlxcYmlnb3BsdXNfaShFXFxvdGltZXMgTl9pKSJdLFsxLDEsIlxcYmlnb3BsdXNfaShFXFxvdGltZXMgRVxcb3RpbWVzIE5faSkiXSxbMCwxLCJcXG11XFxvcGx1c1xcYmlnb3BsdXNfaU5faSJdLFswLDIsIkVcXG90aW1lc1xcY29uZyIsMl0sWzIsMywiRVxcb3RpbWVzXFxiaWdvcGx1c19pXFxrYXBwYV9pIiwyXSxbMyw0LCJcXGNvbmciLDJdLFs0LDUsIlxcYmlnb3BsdXNfaVxca2FwcGFfaSIsMl0sWzEsNiwiXFxjb25nIl0sWzYsNSwiXFxiaWdvcGx1c19pXFxrYXBwYV9pIl0sWzAsNywiXFxjb25nIl0sWzIsNywiXFxjb25nIl0sWzcsNiwiXFxiaWdvcGx1c19pKFxcbXVcXG90aW1lcyBOX2kpIl0sWzcsNCwiXFxiaWdvcGx1c19pKEVcXG90aW1lc1xca2FwcGFfaSkiLDJdXQ==
	\[\begin{tikzcd}
		{E\otimes E\otimes\bigoplus_iN_i} && {E\otimes \bigoplus_iN_i} \\
		{E\otimes\bigoplus_i(E\otimes N_i)} & {\bigoplus_i(E\otimes E\otimes N_i)} & {\bigoplus_i(E\otimes N_i)} \\
		{E\otimes\bigoplus_iN_i} & {\bigoplus_i(E\otimes N_i)} & {\bigoplus_iN_i}
		\arrow["{\mu\oplus\bigoplus_iN_i}", from=1-1, to=1-3]
		\arrow["E\otimes\cong"', from=1-1, to=2-1]
		\arrow["{E\otimes\bigoplus_i\kappa_i}"', from=2-1, to=3-1]
		\arrow["\cong"', from=3-1, to=3-2]
		\arrow["{\bigoplus_i\kappa_i}"', from=3-2, to=3-3]
		\arrow["\cong", from=1-3, to=2-3]
		\arrow["{\bigoplus_i\kappa_i}", from=2-3, to=3-3]
		\arrow["\cong", from=1-1, to=2-2]
		\arrow["\cong", from=2-1, to=2-2]
		\arrow["{\bigoplus_i(\mu\otimes N_i)}", from=2-2, to=2-3]
		\arrow["{\bigoplus_i(E\otimes\kappa_i)}"', from=2-2, to=3-2]
	\end{tikzcd}\]
	The bottom right square commutes by coherence for the $\kappa_i$'s. Every other region commutes by additivity of $-\otimes-$ in each variable. Thus $N=\bigoplus_iN_i$ is indeed an $E$-module object, as desired.

	Now, we claim that $(N,\kappa)$ is the coproduct of the $(N_i,\kappa_i)$'s in $E\text-\Mod$. First, we need to show that the canonical maps $\iota_i:N_i\into N$ are morphisms in $E\text-\Mod$ for all $i\in I$. To see $\iota_i$ is a homomorphism of $E$-module objects, consider the following diagram:
	% https://q.uiver.app/#q=WzAsNSxbMCwwLCJFXFxvdGltZXMgTl9pIl0sWzIsMCwiRVxcb3RpbWVzXFxiaWdvcGx1c19pTl9pIl0sWzAsMiwiTl9pIl0sWzIsMiwiXFxiaWdvcGx1c19pTl9pIl0sWzIsMSwiXFxiaWdvcGx1c19pKEVcXG90aW1lcyBOX2kpIl0sWzAsMSwiRVxcb3RpbWVzXFxpb3RhX2kiXSxbMCwyLCJcXGthcHBhX2kiLDJdLFsyLDMsIlxcaW90YV9pIiwyLHsic3R5bGUiOnsidGFpbCI6eyJuYW1lIjoiaG9vayIsInNpZGUiOiJ0b3AifX19XSxbMSw0LCJcXGNvbmciXSxbNCwzLCJcXGJpZ29wbHVzX2lcXGthcHBhX2kiXSxbMCw0LCJcXGlvdGFfe0VcXG90aW1lcyBOX2l9IiwxLHsic3R5bGUiOnsidGFpbCI6eyJuYW1lIjoiaG9vayIsInNpZGUiOiJ0b3AifX19XV0=
	\[\begin{tikzcd}
		{E\otimes N_i} && {E\otimes\bigoplus_iN_i} \\
		&& {\bigoplus_i(E\otimes N_i)} \\
		{N_i} && {\bigoplus_iN_i}
		\arrow["{E\otimes\iota_i}", from=1-1, to=1-3]
		\arrow["{\kappa_i}"', from=1-1, to=3-1]
		\arrow["{\iota_i}"', hook, from=3-1, to=3-3]
		\arrow["\cong", from=1-3, to=2-3]
		\arrow["{\bigoplus_i\kappa_i}", from=2-3, to=3-3]
		\arrow["{\iota_{E\otimes N_i}}"{description}, hook, from=1-1, to=2-3]
	\end{tikzcd}\]
	The top triangle commutes by additivity of $E\otimes-$. The bottom trapezoid commutes since, by univeral property of the coproduct, $\bigoplus_i\kappa_i$ is the unique arrow which makes the trapezoid commute for all $i\in I$. Now, it remains to show that given an $E$-module object $(N',\kappa')$ and homomorphisms $f_i:N_i\to N'$ of $E$-module objects for all $i\in I$, that the unique arrow $f:N\to N'$ in $\cSH$ satisfying $f\circ\iota_i=f_i$ for all $i\in I$ is a homomorphism of $E$-module objects, so that $N$ is actually the coproduct of the $N_i$'s. To see this, first let $h:\bigoplus_i(E\otimes N_i)\to E\otimes N'$ be the arrow determined by the maps $E\otimes N_i\xrightarrow{E\otimes f_i}E\otimes N'$. Then consider the following diagram:
	% https://q.uiver.app/#q=WzAsNyxbMCwwLCJFXFxvdGltZXNcXGJpZ29wbHVzX2lOX2kiXSxbMiwwLCJFXFxvdGltZXMgTiciXSxbMCwxLCJcXGJpZ29wbHVzX2koRVxcb3RpbWVzIE5faSkiXSxbMCwzLCJcXGJpZ29wbHVzX2lOX2kiXSxbMiwzLCJOJyJdLFsxLDEsIlxcYmlnb3BsdXNfaShFXFxvdGltZXMgTicpIl0sWzEsMiwiXFxiaWdvcGx1c19pTiciXSxbMCwxLCJFXFxvdGltZXMgZiJdLFswLDIsIlxcY29uZyIsMl0sWzIsMywiXFxiaWdvcGx1c19pXFxrYXBwYV9pIiwyXSxbMyw0LCJmIiwyXSxbMSw0LCJcXGthcHBhJyJdLFsyLDUsIlxcYmlnb3BsdXNfaShFXFxvdGltZXMgZl9pKSIsMl0sWzIsMSwiaCJdLFs1LDEsIlxcbmFibGEiLDJdLFs1LDYsIlxcYmlnb3BsdXNfaVxca2FwcGEnIl0sWzYsNCwiXFxuYWJsYSJdLFszLDYsIlxcYmlnb3BsdXNfaWZfaSIsMV1d
	\[\begin{tikzcd}
		{E\otimes\bigoplus_iN_i} && {E\otimes N'} \\
		{\bigoplus_i(E\otimes N_i)} & {\bigoplus_i(E\otimes N')} \\
		& {\bigoplus_iN'} \\
		{\bigoplus_iN_i} && {N'}
		\arrow["{E\otimes f}", from=1-1, to=1-3]
		\arrow["\cong"', from=1-1, to=2-1]
		\arrow["{\bigoplus_i\kappa_i}"', from=2-1, to=4-1]
		\arrow["f"', from=4-1, to=4-3]
		\arrow["{\kappa'}", from=1-3, to=4-3]
		\arrow["{\bigoplus_i(E\otimes f_i)}"', from=2-1, to=2-2]
		\arrow["h", from=2-1, to=1-3]
		\arrow["\nabla"', from=2-2, to=1-3]
		\arrow["{\bigoplus_i\kappa'}", from=2-2, to=3-2]
		\arrow["\nabla", from=3-2, to=4-3]
		\arrow["{\bigoplus_if_i}"{description}, from=4-1, to=3-2]
	\end{tikzcd}\]
	The top triangle commutes by additivity of $E\otimes-$. The triangle below that commutes by the universal property of the coproduct, since it is straightforward to check that $\nabla\circ\bigoplus_i(E\otimes f_i)$ and $h$ both satisfy the universal property of the colimit. The left trapezoid commutes by functoriality of $-\oplus-$ and the fact that $f_i$ is a homomorphism of $E$-module objects for all $i$ in $I$. The right trapezoid commutes by naturality of $\nabla$. Finally, the bottom triangle commutes by the universal product of the coproduct, by showing that $\nabla\circ\bigoplus_if_i$ in place of $f$ also satisfies the universal property of the colimit. Hence $f$ is inded a homomorphism of $E$-module objects, as desired.

	To recap, we have shown that given a set of $E$-module objects $\{(N_i,\kappa_i)\}_{i\in I}$, that the inclusion maps $\iota_i:N_i\into \bigoplus_iN_i$ are morphisms in $E\text-\Mod$, and that given morphism $f_i:(N_i,\kappa_i)\to(N',\kappa')$ for all $i\in I$, the unique induced map $\bigoplus_iN_i\to N'$ is a morphism in $E\text-\Mod$. Thus, $E\text-\Mod$ does indeed have arbitrary coproducts, and the forgetful functor $E\text-\Mod\to\cSH$ preserves them.
\end{proof}

\begin{proposition}\label{E-Mod,free,forgetful_are_additive}
	Let $(E,\mu,e)$ be a monoid object in $\cSH$. Then $E\text-\Mod$ is an additive category, so that in particular the forgetful functor $E\text-\Mod\to\cSH$ and the free functor $\cSH\to E\text-\Mod$ are additive.
\end{proposition}
\begin{proof}
	It is a general fact that adjoint functors between additive categories are necessarily additive. In order to show $E\text-\Mod$ is an additive category, it suffices to show it has finite coproducts, that $\Hom_{E\text-\Mod}(N,N')$ is an abelian group for all $E$-modules $N$ and $N'$, and that composition is bilinear. We know that $E\text-\Mod$ has coproducts which are preserved by the forgetful functor $E\text-\Mod\to\cSH$ by \autoref{coproduct_of_E_modules_is_coproduct_in_E_mod} (which is clearly faithful). Thus, because $\cSH$ is $\Ab$-enriched and $\Hom_{E\text-\Mod}(N,N')\sseq\cSH(N,N')$, it suffices to show that $\Hom_{E\text-\Mod}(N,N')$ is closed under addition and taking inverses. To see the former, let $f,g:N\to N'$ be $E$-module homomorphisms, and consider the following diagram:
	% https://q.uiver.app/#q=WzAsMTAsWzAsMCwiRVxcb3RpbWVzIE4iXSxbMSwwLCJFXFxvdGltZXMgKE5cXG9wbHVzIE4pIl0sWzIsMCwiRVxcb3RpbWVzIChOJ1xcb3BsdXMgTicpIl0sWzMsMCwiRVxcb3RpbWVzIE4nIl0sWzMsMiwiTiciXSxbMCwyLCJOIl0sWzEsMSwiKEVcXG90aW1lcyBOKVxcb3BsdXMoRVxcb3RpbWVzIE4pIl0sWzIsMSwiKEVcXG90aW1lcyBOJylcXG90aW1lcyAoRVxcb3RpbWVzIE4nKSJdLFsxLDIsIk5cXG9wbHVzIE4iXSxbMiwyLCJOJ1xcb3BsdXMgTiciXSxbMCwxLCJFXFxvdGltZXNcXERlbHRhX04iXSxbMSwyLCJFXFxvdGltZXMgKGZcXG9wbHVzIGcpIl0sWzIsMywiRVxcb3RpbWVzIFxcbmFibGFfe04nfSJdLFszLDQsIlxca2FwcGEnIl0sWzAsNSwiXFxrYXBwYSIsMl0sWzAsNiwiXFxEZWx0YV97RVxcb3RpbWVzIE59IiwyXSxbNiw3LCIoRVxcb3RpbWVzIGYpXFxvcGx1cyhFXFxvdGltZXMgZykiXSxbMiw3LCJcXGNvbmciXSxbMSw2LCJcXGNvbmciLDJdLFs3LDMsIlxcbmFibGFfe0VcXG90aW1lcyBOJ30iLDJdLFs1LDgsIlxcRGVsdGFfTiJdLFs4LDksImZcXG9wbHVzIGciXSxbOSw0LCJcXG5hYmxhX3tOJ30iXSxbNiw4LCJcXGthcHBhXFxvcGx1c1xca2FwcGEiLDJdLFs3LDksIlxca2FwcGEnXFxvcGx1c1xca2FwcGEnIl1d
	\[\begin{tikzcd}
		{E\otimes N} & {E\otimes (N\oplus N)} & {E\otimes (N'\oplus N')} & {E\otimes N'} \\
		& {(E\otimes N)\oplus(E\otimes N)} & {(E\otimes N')\otimes (E\otimes N')} \\
		N & {N\oplus N} & {N'\oplus N'} & {N'}
		\arrow["{E\otimes\Delta_N}", from=1-1, to=1-2]
		\arrow["{E\otimes (f\oplus g)}", from=1-2, to=1-3]
		\arrow["{E\otimes \nabla_{N'}}", from=1-3, to=1-4]
		\arrow["{\kappa'}", from=1-4, to=3-4]
		\arrow["\kappa"', from=1-1, to=3-1]
		\arrow["{\Delta_{E\otimes N}}"', from=1-1, to=2-2]
		\arrow["{(E\otimes f)\oplus(E\otimes g)}", from=2-2, to=2-3]
		\arrow["\cong", from=1-3, to=2-3]
		\arrow["\cong"', from=1-2, to=2-2]
		\arrow["{\nabla_{E\otimes N'}}"', from=2-3, to=1-4]
		\arrow["{\Delta_N}", from=3-1, to=3-2]
		\arrow["{f\oplus g}", from=3-2, to=3-3]
		\arrow["{\nabla_{N'}}", from=3-3, to=3-4]
		\arrow["\kappa\oplus\kappa"', from=2-2, to=3-2]
		\arrow["{\kappa'\oplus\kappa'}", from=2-3, to=3-3]
	\end{tikzcd}\]
	The outermost trapezoids commute by naturality of $\Delta$ and $\nabla$. The triangles in the top corners and the top middle rectangle commute by additivity of $E\otimes-$. Finally, the middle bottom rectangle commutes by functoriality of $-\oplus-$ and $-\otimes-$, and the fact that $f$ and $g$ are $E$-module homomorphisms. Commutativity of the above diagram shows that $f+g$ is a homomorphism of $E$-modules as desired. Finally, to see $-f$ is a $E$-module homomorphism if $f$ is, we would like to show that $\kappa'\circ(E\otimes(-f))=(-f)\circ\kappa$. This follows by the fact that $\kappa'\circ(E\otimes f)=f\circ\kappa$ and additivity of $-\otimes-$ and composition.
\end{proof}

\begin{lemma}\label{E-module_N_implies_pi*N_is_pi*E_module}
	Let $(E,\mu,e)$ be a monoid object in $\cSH$. Then the assignment $\pi_*:(N,\kappa)\mapsto\pi_*(N)$ yields an additive from $E\text-\Mod$ to the category $\pi_*(E)\text-\Mod(A)$ of $A$-graded left $\pi_*(E)$-modules and degree-preserving homomorphisms between them. In particular, if $(N,\kappa)$ is an $E$-module in $\cSH$, then we view it with its \emph{canonical} $A$-graded left $\pi_*(E)$-module structure given by the graded map
	\[\pi_*(E)\times\pi_*(N)\to\pi_*(N)\]
	sending a class $r:S^a\to E$ and $x:S^b\to N$ to the composition
	\[r\cdot x:S^{a+b}\xr{\phi_{a,b}} S^a\otimes S^b\xr{r\otimes x}E\otimes N\xr\kappa N.\]
	Furthermore, this functor is lax $A$-graded (\autoref{A-graded_functor_defn}), with structure map $\pi_*(\Sigma^aN)\cong\pi_{*-a}(N)$ which sends $x:S^b\to S^a\otimes N$ to the composition
	\[S^{b-a}\xr{\phi_{b,-a}}S^b\otimes S^{-a}\xr{x\otimes S^{-a}}S^a\otimes N\otimes S^{-a}\xr{\tau\otimes S^{-a}}N\otimes S^a\otimes S^{-a}\xr{N\otimes\phi_{a,-a}^{-1}}N.\]
\end{lemma}
\begin{proof}
	First let $(N,\kappa)$ be an $E$-module object. Let $a,b,c\in A$ and $x,x':S^a\to N$, $y:S^b\to E$, and $z, z'\in S^c\to E$. Then by \autoref{A-graded_module}, it suffices to show that
	\begin{enumerate}
		\item $y\cdot( x+ x')= y\cdot x+ y\cdot x'$, 
		\item $( z+ z')\cdot x= z\cdot x+ z'\cdot x$,
		\item $(zy)\cdot  x= z\cdot( y\cdot x)$,
		\item $e\cdot x= x$.
	\end{enumerate}
	The first two axioms follow by \autoref{bilinear}. To see $(3)$, consider the diagram:
	% https://q.uiver.app/#q=WzAsNixbMCwxLCJTXnthK2IrY30iXSxbMSwxLCJTXmNcXG90aW1lcyBTXmJcXG90aW1lcyBTXmEiXSxbMiwxLCJFXFxvdGltZXMgRVxcb3RpbWVzIE4iXSxbMywwLCJFXFxvdGltZXMgTiJdLFszLDEsIk4iXSxbMywyLCJFXFxvdGltZXMgTiJdLFswLDEsIlxcY29uZyJdLFsxLDIsInpcXG90aW1lcyB5XFxvdGltZXMgeCJdLFsyLDMsIkVcXG90aW1lc1xca2FwcGEiXSxbMyw0LCJcXGthcHBhIl0sWzIsNSwiXFxtdVxcb3RpbWVzIE4iLDJdLFs1LDQsIlxca2FwcGEiLDJdXQ==
	\[\begin{tikzcd}
		&&& {E\otimes N} \\
		{S^{a+b+c}} & {S^c\otimes S^b\otimes S^a} & {E\otimes E\otimes N} & N \\
		&&& {E\otimes N}
		\arrow["\cong", from=2-1, to=2-2]
		\arrow["{z\otimes y\otimes x}", from=2-2, to=2-3]
		\arrow["E\otimes\kappa", from=2-3, to=1-4]
		\arrow["\kappa", from=1-4, to=2-4]
		\arrow["{\mu\otimes N}"', from=2-3, to=3-4]
		\arrow["\kappa"', from=3-4, to=2-4]
	\end{tikzcd}\]
	It commutes by coherence for $\kappa$. By functoriality of $-\otimes-$, the two outside compositions equal $z\cdot(y\cdot x)$ on the top and $(z\cdot y)\cdot x$ on the bottom. Hence, they are equal, as desired.

	Next, to see $(4)$, consider the following diagram:
	% https://q.uiver.app/#q=WzAsNCxbMCwwLCJTXmEiXSxbMSwxLCJOIl0sWzIsMCwiTiJdLFsxLDIsIkVcXG90aW1lcyBOIl0sWzEsMiwiIiwxLHsibGV2ZWwiOjIsInN0eWxlIjp7ImhlYWQiOnsibmFtZSI6Im5vbmUifX19XSxbMSwzLCJlXFxvdGltZXMgTiIsMV0sWzAsMiwieCJdLFszLDIsIlxca2FwcGEiLDIseyJjdXJ2ZSI6M31dLFswLDEsIngiLDJdLFswLDMsImVcXG90aW1lcyB4IiwyLHsiY3VydmUiOjN9XV0=
	\[\begin{tikzcd}
		{S^a} && N \\
		& N \\
		& {E\otimes N}
		\arrow[Rightarrow, no head, from=2-2, to=1-3]
		\arrow["{e\otimes N}"{description}, from=2-2, to=3-2]
		\arrow["x", from=1-1, to=1-3]
		\arrow["\kappa"', curve={height=18pt}, from=3-2, to=1-3]
		\arrow["x"', from=1-1, to=2-2]
		\arrow["{e\otimes x}"', curve={height=18pt}, from=1-1, to=3-2]
	\end{tikzcd}\]
	The top triangle commutes by definition. The left triangle commutes by functoriality of $-\otimes-$. The right triangle commutes by unitality of $\kappa$. The top composition is $ x$ while the bottom is $e\cdot x$, thus they are necessarily equal since the diagram commutes.

	Now, we'd like to show that if $f:(N,\kappa)\to(N',\kappa)$ is a homomorphism of $E$-module objects, then $\pi_*(f):\pi_*(N)\to\pi_*(N')$ is a homomorphism of left $\pi_*(E)$-modules. To see this, let $r:S^a\to E$ in $\pi_a(E)$ and $x,x:S^b\to N$ in $\pi_b(N)$. We'd like to show that $\pi_*(f)(x+x')=\pi_*(f)(x)+\pi_*(f)(x')$ and $\pi_*(f)(r\cdot x)=r\cdot\pi_*(f)(x)$. To see the former, consider the following diagram:
	% https://q.uiver.app/#q=WzAsNixbMCwxLCJTXmEiXSxbMSwxLCJTXmFcXG9wbHVzIFNeYSJdLFsyLDEsIk5cXG9wbHVzIE4iXSxbMywwLCJOJ1xcb3BsdXMgTiciXSxbMywxLCJOJyJdLFszLDIsIk4iXSxbMCwxLCJcXERlbHRhIl0sWzEsMiwieFxcb3BsdXMgeCciXSxbMiwzLCJmXFxvcGx1cyBmIl0sWzMsNCwiXFxuYWJsYSJdLFsyLDUsIlxcbmFibGEiLDJdLFs1LDQsImYiLDJdXQ==
	\[\begin{tikzcd}
		&&& {N'\oplus N'} \\
		{S^a} & {S^a\oplus S^a} & {N\oplus N} & {N'} \\
		&&& N
		\arrow["\Delta", from=2-1, to=2-2]
		\arrow["{x\oplus x'}", from=2-2, to=2-3]
		\arrow["{f\oplus f}", from=2-3, to=1-4]
		\arrow["\nabla", from=1-4, to=2-4]
		\arrow["\nabla"', from=2-3, to=3-4]
		\arrow["f"', from=3-4, to=2-4]
	\end{tikzcd}\]
	It commutes by naturality of $\nabla$ in an additive category. The top composition is $\pi_*(f)(x)+\pi_*(f)(x')$, while the bottom is $\pi_*(f)(x+x')$, so they are equal as desired. To see that $\pi_*(f)(r\cdot x)=r\cdot \pi_*(f)(x)$, consider the following diagram:
	% https://q.uiver.app/#q=WzAsNixbMCwxLCJTXnthK2J9Il0sWzEsMSwiU15iXFxvdGltZXMgU15hIl0sWzIsMSwiRVxcb3RpbWVzIE4iXSxbMywwLCJFXFxvdGltZXMgTiciXSxbMywxLCJOJyJdLFszLDIsIk4iXSxbMCwxLCJcXHBoaV97YixhfSJdLFsxLDIsInJcXG90aW1lcyB4Il0sWzIsMywiRVxcb3RpbWVzIGYiXSxbMyw0LCJcXGthcHBhJyJdLFsyLDUsIlxca2FwcGEiLDJdLFs1LDQsImYiLDJdXQ==
	\[\begin{tikzcd}
		&&& {E\otimes N'} \\
		{S^{a+b}} & {S^b\otimes S^a} & {E\otimes N} & {N'} \\
		&&& N
		\arrow["{\phi_{b,a}}", from=2-1, to=2-2]
		\arrow["{r\otimes x}", from=2-2, to=2-3]
		\arrow["{E\otimes f}", from=2-3, to=1-4]
		\arrow["{\kappa'}", from=1-4, to=2-4]
		\arrow["\kappa"', from=2-3, to=3-4]
		\arrow["f"', from=3-4, to=2-4]
	\end{tikzcd}\]
	It commutes by the fact that $f$ is a homomorphism of $E$-module objects. The bottom composition is $\pi_*(f)(r\cdot x)$, while the top composition is $r\cdot \pi_*(f)(x)$, so they are equal, as desired.
 
	Next we claim this functor is additive. It suffices to show it preserves the zero object and preserves coproducts. To see the former, note that $\pi_*(0)=[S^*,0]=0$ by definition, since $0$ is terminal. To see the latter, we need to show that given $(N,\kappa),(N',\kappa')\in E\text-\Mod$ that $\pi_*(N)\oplus\pi_*(N')\cong\pi_*(N\oplus N')$, and that the following diagram commutes:
	% https://q.uiver.app/#q=WzAsMyxbMSwxLCJcXHBpXyooTlxcb3BsdXMgTicpIl0sWzAsMCwiXFxwaV8qKE4pIl0sWzAsMSwiXFxwaV8qKE4pXFxvcGx1c1xccGlfKihOJykiXSxbMSwwLCJcXHBpXyooXFxpb3RhX04pIl0sWzIsMCwiXFxjb25nIiwyXSxbMSwyLCJcXGlvdGFfe1xccGlfKihOKX0iLDJdXQ==
	\[\begin{tikzcd}
		{\pi_*(N)} \\
		{\pi_*(N)\oplus\pi_*(N')} & {\pi_*(N\oplus N')}
		\arrow["{\pi_*(\iota_N)}", from=1-1, to=2-2]
		\arrow["\cong"', from=2-1, to=2-2]
		\arrow["{\iota_{\pi_*(N)}}"', from=1-1, to=2-1]
	\end{tikzcd}\]
	Since each $S^a$ is compact, for all $a,b\in A$ we have isomorphisms
	\[\pi_a(N)\oplus\pi_a(N')=[S^a,N]\oplus[S^a,N']\cong[S^a,N\oplus N']=\pi_a(N,N'),\]
	and these combine together to yield $A$-graded isomorphisms $\pi_*(N)\oplus\pi_*(N')\xr\cong\pi_*(N\oplus N')$. To see the above diagram commutes, note that since everything is an $A$-graded homomorphism of $A$-graded abelian groups, it suffices to chase homogeneous elements around to show it commutes. Indeed, it is entirely straightforward, by unravelling definitions, that both compositions around the diagram take a generator $x:S^a\to N$ in $\pi_a(N)$ to the composition
	\[S^a\xr xN\xr{\iota_N}N\oplus N'.\]
	Thus, we have shown that $\pi_*$ preserves all finite coproducts, so it is additive.

	Finally, we claim this functor is lax $A$-graded. To that end, we need to show the given map
	\[t^a_{(N,\kappa)}\pi_*(\Sigma^aN)\to\pi_{*-a}(N),\]
	is an isomorphism of left $\pi_*(E)$-modules for all $(N,\kappa)$ in $E\text-\Mod$. First, note that by unravelling definitions it factors as
	% https://q.uiver.app/#q=WzAsNyxbMCwwLCJcXHBpXyooXFxTaWdtYV5hTikiXSxbMSwwLCJbU14qLFNeYVxcb3RpbWVzIE5dIl0sWzEsMSwiW1NeKlxcb3RpbWVzIFNeey1hfSxTXmFcXG90aW1lcyBOXFxvdGltZXMgU157LWF9XSJdLFsxLDIsIltTXipcXG90aW1lcyBTXnstYX0sTlxcb3RpbWVzIFNeYVxcb3RpbWVzIFNeey1hfV0iXSxbMSwzLCJbU14qXFxvdGltZXMgU157LWF9LE5dIl0sWzEsNCwiW1NeeyotYX0sTl0iXSxbMiw0LCJcXHBpX3sqLWF9KE4pIl0sWzAsMSwiIiwwLHsibGV2ZWwiOjIsInN0eWxlIjp7ImhlYWQiOnsibmFtZSI6Im5vbmUifX19XSxbMSwyLCItXFxvdGltZXMgU157LWF9Il0sWzIsMywieyhcXHRhdVxcb3RpbWVzIFNeey1hfSl9XyoiXSxbMyw0LCJ7KE5cXG90aW1lcyBcXHBoaV97YSwtYX1eey0xfSl9XyoiXSxbNCw1LCJ7KFxccGhpX3sqLC1hfSl9XioiXSxbNSw2LCIiLDAseyJsZXZlbCI6Miwic3R5bGUiOnsiaGVhZCI6eyJuYW1lIjoibm9uZSJ9fX1dXQ==
	\[\begin{tikzcd}
		{\pi_*(\Sigma^aN)} & {[S^*,S^a\otimes N]} \\
		& {[S^*\otimes S^{-a},S^a\otimes N\otimes S^{-a}]} \\
		& {[S^*\otimes S^{-a},N\otimes S^a\otimes S^{-a}]} \\
		& {[S^*\otimes S^{-a},N]} \\
		& {[S^{*-a},N]} & {\pi_{*-a}(N)}
		\arrow[Rightarrow, no head, from=1-1, to=1-2]
		\arrow["{-\otimes S^{-a}}", from=1-2, to=2-2]
		\arrow["{{(\tau\otimes S^{-a})}_*}", from=2-2, to=3-2]
		\arrow["{{(N\otimes \phi_{a,-a}^{-1})}_*}", from=3-2, to=4-2]
		\arrow["{{(\phi_{*,-a})}^*}", from=4-2, to=5-2]
		\arrow[Rightarrow, no head, from=5-2, to=5-3]
	\end{tikzcd}\]
	The first vertical arrow is an isomorphism since $-\otimes S^a$ is an additive equivalence of $\cSH$ (\autoref{Sigma^a,Sigma^-a_adjoint_equiv}), and the other three arrows are given by composing with isomorphisms in an additive category. Thus, we have constructed $A$-graded isomorphisms of abelian groups $\pi_*(\Sigma^aX)\to\pi_{*-a}(X)$. It remains to show that this map is a homomorphism of left $\pi_*(E)$-modules. To that end, let $r:S^b\to E$ in $\pi_*(E)$ and $x:S^c\to S^a\otimes N$ in $\pi_*(\Sigma^aN)$, and consider the following diagram:
	% https://q.uiver.app/#q=WzAsOSxbMCwwLCJTXntiK2MtYX0iXSxbMCwxLCJTXmJcXG90aW1lcyBTXmNcXG90aW1lcyBTXnstYX0iXSxbMCwyLCJFXFxvdGltZXMgU15hXFxvdGltZXMgTlxcb3RpbWVzIFNeey1hfSJdLFsxLDIsIkVcXG90aW1lcyBOXFxvdGltZXMgU15hXFxvdGltZXMgU157LWF9Il0sWzIsMiwiRU4iXSxbMCw0LCJTXmFcXG90aW1lcyBFXFxvdGltZXMgTlxcb3RpbWVzIFNeey1hfSJdLFsxLDQsIlNeYVxcb3RpbWVzIE5cXG90aW1lcyBTXnstYX0iXSxbMiw0LCJOXFxvdGltZXMgU15hXFxvdGltZXMgU157LWF9Il0sWzIsMywiTiJdLFswLDEsIlxccGhpIiwyXSxbMSwyLCJyXFxvdGltZXMgeFxcb3RpbWVzIFNeey1hfSIsMl0sWzIsMywiRVxcb3RpbWVzIFxcdGF1XFxvdGltZXMgIFNeey1hfSJdLFszLDQsIkVcXG90aW1lcyBOXFxvdGltZXMgXFxwaGlfe2EsLWF9XnstMX0iXSxbMiw1LCJcXHRhdVxcb3RpbWVzICBOXFxvdGltZXMgU157LWF9IiwyXSxbNSw2LCJTXmFcXG90aW1lcyBcXGthcHBhXFxvdGltZXMgIFNeey1hfSJdLFs2LDcsIlxcdGF1XFxvdGltZXMgU157LWF9Il0sWzQsOCwiXFxrYXBwYSJdLFs3LDgsIk5cXG90aW1lcyBcXHBoaV97YSwtYX1ee18xfSIsMV0sWzMsNywiXFxrYXBwYVxcb3RpbWVzIFNeYVxcb3RpbWVzIFNeey1hfSIsMV0sWzUsMywiXFx0YXVfe1NeYSxFXFxvdGltZXMgTn1cXG90aW1lcyBTXnstYX0iLDFdXQ==
	\[\begin{tikzcd}
		{S^{b+c-a}} \\
		{S^b\otimes S^c\otimes S^{-a}} \\
		{E\otimes S^a\otimes N\otimes S^{-a}} & {E\otimes N\otimes S^a\otimes S^{-a}} & EN \\
		&& N \\
		{S^a\otimes E\otimes N\otimes S^{-a}} & {S^a\otimes N\otimes S^{-a}} & {N\otimes S^a\otimes S^{-a}}
		\arrow["\phi"', from=1-1, to=2-1]
		\arrow["{r\otimes x\otimes S^{-a}}"', from=2-1, to=3-1]
		\arrow["{E\otimes \tau\otimes  S^{-a}}", from=3-1, to=3-2]
		\arrow["{E\otimes N\otimes \phi_{a,-a}^{-1}}", from=3-2, to=3-3]
		\arrow["{\tau\otimes  N\otimes S^{-a}}"', from=3-1, to=5-1]
		\arrow["{S^a\otimes \kappa\otimes  S^{-a}}", from=5-1, to=5-2]
		\arrow["{\tau\otimes S^{-a}}", from=5-2, to=5-3]
		\arrow["\kappa", from=3-3, to=4-3]
		\arrow["{N\otimes \phi_{a,-a}^{_1}}"{description}, from=5-3, to=4-3]
		\arrow["{\kappa\otimes S^a\otimes S^{-a}}"{description}, from=3-2, to=5-3]
		\arrow["{\tau_{S^a,E\otimes N}\otimes S^{-a}}"{description}, from=5-1, to=3-2]
	\end{tikzcd}\]
	The top composition is $r\cdot t^a_{(N,\kappa)}(x)$, while the bottom composition is $t^a_{(N,\kappa)}(r\cdot x)$. The left triangle commutes by coherence for the symmetries. The bottom triangle commutes by naturality of $\tau$. The right triangle commutes by functoriality of $-\otimes-$. Thus, indeed $\pi_*:E\text-\Mod\to\pi_*(E)\text-\Mod(A)$ is lax $A$-graded, as desired.
\end{proof}

%\begin{definition}
%	Let $(E,\mu,e)$ be a monoid object in $\cSH$, and suppose $(N,\kappa)$ and $(N',\kappa')$ are $E$-module objects in $E\text-\Mod$. Then the hom-sets in $E\text-\Mod$ can be extended to $A$-graded abelian groups $\Hom^*_{E\text-\Mod}(N,N')$, by defining 
%	\[\Hom^a_{E\text-\Mod}(N,N'):=\Hom_{E\text-\Mod}(\Sigma^a N,N')\]
%	for each $a\in A$ (where $\Sigma^{*}N$ has the $E$-module structure given by \autoref{suspension_of_module_object_is_module_object}).
%\end{definition}

\subsection{A universal coefficient theorem in \texorpdfstring{$\cSH$}{SH}}

%\begin{lemma}\label{tw^a_isos}
	%Let $(E,\mu,e)$ be a monoid object in $\cSH$, $(N,\kappa)$ an $E$-module, and $a\in A$. Then the assignment
	%\[\mathrm{tw}^a:\pi_{*-a}(N)\to\pi_*(\Sigma^aN)\]
	%sending $x:S^{b-a}\to N$ to the composition
	%\[S^b\xr{\phi_{a,b-a}}S^a\otimes S^{b-a}\xr{S^a\otimes x}S^a\otimes N=\Sigma^aN\]
	%is an $A$-graded isomorphism of left $A$-graded $\pi_*(E)$-modules (where here $\pi_*(N)$ is a left $\pi_*(E)$-module by \autoref{E-module_N_implies_pi*N_is_pi*E_module}, and $\pi_*(\Sigma^aN)$ has the left $\pi_*(E)$-module structure given by \autoref{suspension_of_module_object_is_module_object} and \autoref{E-module_N_implies_pi*N_is_pi*E_module}).
%\end{lemma}
%\begin{proof}
	%Unravelling definitions, the map $\mathrm{tw}^a:\pi_{*-a}(N)\to\pi_*(\Sigma^aN)$ factors as
	%\[\pi_{*-a}(N)=[S^{*-a},N]\xr{-\otimes S^a}[S^a\otimes S^{*-a},S^a\otimes N]\xr{{(\phi_{a,b-a})}^*}[S^*,S^a\otimes N]=\pi_*(\Sigma^aN).\]
	%The arrow labeled $-\otimes S^a$ is an isomorphism of abelian groups because $-\otimes S^a\cong \Sigma^a$ is an autoequivalence of $\cSH$ (\autoref{Sigma^a,Sigma^-a_adjoint_equiv}). Hence, we have shown the map is an isomorphism of abelian groups. Clearly the map preserves degree, so it is an $A$-graded homomorphism as desired. Finally, it remains to show that this map is a homomorphism of left $\pi_*(E)$-modules, i.e., that given $r:S^b\to E$ in $\pi_*(E)$ and $x:S^{c-a}\to N$ in $\pi_{*-a}(N)$ that $\mathrm{tw}^a(r\cdot x)=r\cdot\mathrm{tw}^a(x)$. To that end, consider the following diagram:
	%% https://q.uiver.app/#q=WzAsNyxbMCwxLCJTXntiK2N9Il0sWzEsMSwiU15iXFxvdGltZXMgU157Yy1hfVxcb3RpbWVzIFNeYSJdLFsyLDEsIkVcXG90aW1lcyBOXFxvdGltZXMgU15hIl0sWzIsMCwiRVxcb3RpbWVzIFNeYVxcb3RpbWVzIE4iXSxbMywwLCJTXmFcXG90aW1lcyBFXFxvdGltZXMgTiJdLFszLDEsIlNeYVxcb3RpbWVzIE4iXSxbMywyLCJOXFxvdGltZXMgU15hIl0sWzAsMSwiXFxwaGkiXSxbMSwyLCJyXFxvdGltZXMgeFxcb3RpbWVzIFNeYSJdLFszLDQsIlxcdGF1XFxvdGltZXMgTiJdLFs0LDUsIlNeYVxcb3RpbWVzXFxrYXBwYSJdLFsyLDYsIlxca2FwcGFcXG90aW1lcyBTXmEiLDJdLFs2LDUsIlxcdGF1IiwyXSxbMiw0LCJcXHRhdV97RVxcb3RpbWVzIE4sU15hfSIsMV0sWzIsMywiRVxcb3RpbWVzIFxcdGF1Il1d
	%\[\begin{tikzcd}
		%&& {E\otimes S^a\otimes N} & {S^a\otimes E\otimes N} \\
		%{S^{b+c}} & {S^b\otimes S^{c-a}\otimes S^a} & {E\otimes N\otimes S^a} & {S^a\otimes N} \\
		%&&& {N\otimes S^a}
		%\arrow["\phi", from=2-1, to=2-2]
		%\arrow["{r\otimes x\otimes S^a}", from=2-2, to=2-3]
		%\arrow["{\tau\otimes N}", from=1-3, to=1-4]
		%\arrow["{S^a\otimes\kappa}", from=1-4, to=2-4]
		%\arrow["{\kappa\otimes S^a}"', from=2-3, to=3-4]
		%\arrow["\tau"', from=3-4, to=2-4]
		%\arrow["{\tau_{E\otimes N,S^a}}"{description}, from=2-3, to=1-4]
		%\arrow["{E\otimes \tau}", from=2-3, to=1-3]
	%\end{tikzcd}\]
	%The top triangle commutes by coherence for a symmetric monoidal category, while the right triangle commutes by naturality of $\tau$.
%\end{proof}
%
%\begin{corollary}\label{pi_*:Hom^*(N,N')toHom^*(pi*N,pi*N')}
%	The homomorphisms
%	\[\pi_*:\Hom_{E\text-\Mod}(N,N')\to\Hom_{\pi_*(E)}(\pi_*(N),\pi_*(N'))\]
%	constructed in \autoref{E-module_N_implies_pi*N_is_pi*E_module} extend to $A$-graded homomorphisms of abelian groups
%	\[\pi_*:\Hom_{E\text-\Mod}^*(N,N')\to\Hom_{\pi_*(E)}^*(\pi_*(N),\pi_*(N'))\]
%	as
%	\[\Hom_{E\text-\Mod}(\Sigma^aN,N')\xr{\pi_*}\Hom_{\pi_*(E)}(\pi_*(\Sigma^aN),\pi_*(N'))\xr{(\mathrm{tw}^a)^*}\Hom_{\pi_*(E)}(\pi_{*-a}(N),\pi_*(N')).\]
%	Explicitly, this map sends a class $f:S^a\otimes N\to N'$ to the map $\pi_{*-a}(N)\to\pi_*(N')$ which sends a class $x:S^{b-a}\to N$ to the composition
%	\[S^b\xr{\phi}S^{b-a}\otimes S^a\xr{x\otimes S^a}N\otimes S^a\xr\tau S^a\otimes N\xr fN'.\]
%\end{corollary}

\begin{lemma}\label{susp_of_coprod_is_iso_in_E-Mod_to_coprod_of_sus}
	Let $(E,\mu,e)$ be a monoid object in $\cSH$, and suppose we have a collection of objects $(N_i,\kappa_i)$ in $E\text-\Mod$. Then for all $a\in A$, since $\Sigma^a$ has a right adjoint $\Sigma^{-a}$ (\autoref{Sigma^a,Sigma^-a_adjoint_equiv}), it preserves coproducts in $\cSH$, (which are coproducts in $E\text-\Mod$ by \autoref{coproduct_of_E_modules_is_coproduct_in_E_mod}), so we have an isomorphism
	\[\Sigma^a\bigoplus_iN_i\cong\bigoplus_i\Sigma^aN_i.\]
	Then this isomorphism is an $E$-module homomorphism.
\end{lemma}
\begin{proof}
	\todo{TODO}
\end{proof}

\begin{lemma}\label{pi_*_iso_when_N_retract_of_wedge_of_susps}
	Let $(E,\mu,e)$ be a monoid object and $(N,\kappa)$ an $E$-module object in $\cSH$. Then given a collection of $a_i\in A$ indexed by some set $I$, if $(N,\kappa)$ is a retract of $\bigoplus_i(E\otimes S^{a_i})$ in $E\text-\Mod$,\footnote{Here $\bigoplus_i(E\otimes S^{a_i})$ is a coproduct (\autoref{coproduct_of_E_modules_is_coproduct_in_E_mod}) of a bunch of free $E$-module objects (\autoref{free_forgetful_E-Mod}), so it is itself an $E$-module object.} then for all $E$-module objects $(N',\kappa')$ in $\cSH$, the functor $\pi_*$ (\autoref{E-module_N_implies_pi*N_is_pi*E_module}) induces an $A$-graded isomorphism of abelian groups
	\[\pi_*:\Hom_{E\text-\Mod}^*(N,N')\to\Hom_{\pi_*(E)}^*(\pi_*(N),\pi_*(N')).\]
\end{lemma}
\begin{proof}
	First, since the functor $\pi_*:E\text-\Mod\to\pi_*(E)\text-\Mod(A)$ is lax $A$-graded (by \autoref{E-module_N_implies_pi*N_is_pi*E_module}), by \autoref{lift_A-graded_functor} we have induced $A$-graded homomorphisms
	\[\Hom_{E\text-\Mod}^*(N,N')=\Hom_{E\text-\Mod}(\Sigma^*N,N')\to\Hom_{\pi_*(E)}^*(\pi_*(N),\pi_*(N'))\]
	which send $f:\Sigma^aN\to N'$ to the composition
	\[\pi_{*-a}(N)\xr\cong\pi_*(\Sigma^aN)\xr{\pi_*(f)}\pi_*(N').\]
	Furthermore, the aforementioned lemma says that the $A$-graded homomorphism
	\[\pi_*:\Hom^*_{E\text-\Mod}(N,N')\to\Hom_{\pi_*(E)}^*(\pi_*(N),\pi_*(N'))\]
	is an isomorphism iff
	\[\pi_*:\Hom_{E\text-\Mod}(\Sigma^aN,N')\to\Hom_{\pi_*(E)}(\pi_*(\Sigma^aN),\pi_*(N'))\]
	is an isomorphism for all $a\in A$. First, we show that if $N$ is a retract of $\bigoplus_i(E\otimes S^{a_i})$ in $E\text-\Mod$, then
	\[\pi_*:\Hom_{E\text-\Mod}(N,N')\to\Hom_{\pi_*(E)}(\pi_*(N),\pi_*(N'))\]
	is an isomorphism. To start, we consider the case $N=\bigoplus_i(E\otimes S^{a_i})$. Consider the following diagram:
	% https://q.uiver.app/#q=WzAsOSxbMCwwLCJcXEhvbV97RVxcdGV4dC1cXE1vZH0oXFxiaWdvcGx1c19pIChFXFxvdGltZXMgU157YV9pfSksTicpIl0sWzEsMCwiXFxIb21fe1xccGlfKihFKX0oXFxwaV8qKFxcYmlnb3BsdXNfaSAoRVxcb3RpbWVzIFNee2FfaX0pKSxcXHBpXyooTicpKSJdLFswLDEsIlxccHJvZF9pXFxIb21fe0VcXHRleHQtXFxNb2R9KEVcXG90aW1lcyBTXnthX2l9LE4nKSJdLFswLDIsIlxccHJvZF9pW1Nee2FfaX0sTiddIl0sWzEsMSwiXFxIb21fe1xccGlfKihFKX0oXFxiaWdvcGx1c19pXFxwaV8qKEVcXG90aW1lcyBTXnthX2l9KSxcXHBpXyooTicpKSJdLFsxLDIsIlxccHJvZF9pXFxIb21fe1xccGlfKihFKX0oXFxwaV8qKEVcXG90aW1lcyBTXnthX2l9KSxcXHBpXyooTicpKSJdLFsxLDMsIlxccHJvZF9pXFxIb21fe1xccGlfKihFKX0oXFxwaV97Ki1hX2l9KEUpLFxccGlfKihOJykpIl0sWzEsNCwiXFxwcm9kX2lcXEhvbV57YV9pfV97XFxwaV8qKEUpfShcXHBpX3sqfShFKSxcXHBpXyooTicpKSJdLFswLDQsIlxccHJvZF9pXFxwaV97YV9pfShOJykiXSxbMCwxLCJcXHBpXyoiXSxbMCwyLCJcXGNvbmciLDJdLFsyLDMsIlxcY29uZyIsMl0sWzEsNCwiXFxjb25nIl0sWzQsNSwiXFxjb25nIl0sWzUsNiwiXFxjb25nIl0sWzYsNywiIiwwLHsibGV2ZWwiOjIsInN0eWxlIjp7ImhlYWQiOnsibmFtZSI6Im5vbmUifX19XSxbNyw4LCJcXHByb2RfaVxcbWF0aHJte2V2fV8xIiwyXSxbMyw4LCIiLDIseyJsZXZlbCI6Miwic3R5bGUiOnsiaGVhZCI6eyJuYW1lIjoibm9uZSJ9fX1dXQ==
	\[\begin{tikzcd}
		{\Hom_{E\text-\Mod}(\bigoplus_i (E\otimes S^{a_i}),N')} & {\Hom_{\pi_*(E)}(\pi_*(\bigoplus_i (E\otimes S^{a_i})),\pi_*(N'))} \\
		{\prod_i\Hom_{E\text-\Mod}(E\otimes S^{a_i},N')} & {\Hom_{\pi_*(E)}(\bigoplus_i\pi_*(E\otimes S^{a_i}),\pi_*(N'))} \\
		{\prod_i[S^{a_i},N']} & {\prod_i\Hom_{\pi_*(E)}(\pi_*(E\otimes S^{a_i}),\pi_*(N'))} \\
		& {\prod_i\Hom_{\pi_*(E)}(\pi_{*-a_i}(E),\pi_*(N'))} \\
		{\prod_i\pi_{a_i}(N')} & {\prod_i\Hom^{a_i}_{\pi_*(E)}(\pi_{*}(E),\pi_*(N'))}
		\arrow["{\pi_*}", from=1-1, to=1-2]
		\arrow["\cong"', from=1-1, to=2-1]
		\arrow["\cong"', from=2-1, to=3-1]
		\arrow["\cong", from=1-2, to=2-2]
		\arrow["\cong", from=2-2, to=3-2]
		\arrow["\cong", from=3-2, to=4-2]
		\arrow[Rightarrow, no head, from=4-2, to=5-2]
		\arrow["{\prod_i\mathrm{ev}_1}"', from=5-2, to=5-1]
		\arrow[Rightarrow, no head, from=3-1, to=5-1]
	\end{tikzcd}\]
	Here the top left vertical isomorphism exibits the universal property of the coproduct in $E\text-\Mod$, and middle left vertical isomorphism below that is the free-forgetful adjunction for $E$-modules (\autoref{free_forgetful_E-Mod}). The bottom horizontal isomorphism is the product of the evaluation-at-$1$ isomorphisms (\autoref{ev_at_1_is_iso}). On the other side, the top right vertical isomorphism is given by the fact that $S^a$ is compact for each $a\in A$, so we have isomorphisms
	\[\bigoplus_i\pi_*(E\otimes S^{a_i})=\bigoplus_{a\in A}\bigoplus_{i}[S^a,E\otimes S^{a_i}]\cong\bigoplus_{a\in A}[S^a,\bigoplus_{i}(E\otimes S^{a_i})]=\pi_*(\bigoplus_i(E\otimes S^{a_i})),\]
	where the middle isomorphism takes a generator $x:S^a\xr E\otimes S^{a_i}$ to the composition $S^a\xr xE\otimes S^{a_i}\into\bigoplus_i(E\otimes S^{a_i})$. The middle right vertical isomorphism exhibits the universal property of the coproduct of modules. Finally the bottom right vertical isomorphism is given by the isomorphisms
	\[\pi_{*-a_i}(E\otimes S^{a_i})=[S^{*-a_i},E\otimes S^{a_i}]\xr{-\otimes S^{a_i}}[S^{*-a_i}\otimes S^{a_i},E\otimes S^{a_i}]\xr{\phi^*}[S^*,E\otimes S^{a_i}]=\pi_*(E\otimes S^{a_i}),\]
	where $-\otimes S^{a_i}\cong\Sigma^{a_i}$ is an isomorphism by \autoref{Sigma^a,Sigma^-a_adjoint_equiv}. Now, we claim this diagram commutes. This really simply amounts to unravelling definitions, and chasing a homomorphism $f:\bigoplus_i(E\otimes S^{a_i})\to N'$ of $E$-module objects both ways around the diagram yields the composition
	\[\prod_i(S^{a_i}\xr{e\otimes S^{a_i}}E\otimes S^{a_i}\into\bigoplus_i(E\otimes S^{a_i})\xr fN').\]
	Thus, since the diagram commutes, we have that 
	\[\pi_*:\Hom_{E\text-\Mod}(\bigoplus_i(E\otimes S^{a_i}),N')\to\Hom_{\pi_*(E)}(\pi_*(\bigoplus_i(E\otimes S^{a_i})),\pi_*(N'))\] is an isomorphism, as desired.

	Now, consider the case that $N$ is a retract of $\bigoplus_i(E\otimes S^{a_i})$ in $E\text-\Mod$, so there exists a commuting diagram of $E$-module object homomorphisms:
	% https://q.uiver.app/#q=WzAsMyxbMCwwLCJOIl0sWzEsMCwiXFxiaWdvcGx1c19pKEVcXG90aW1lcyBTXnthX2l9KSJdLFsyLDAsIk4iXSxbMCwxLCJcXGlvdGEiLDJdLFsxLDIsInIiLDJdLFswLDIsIiIsMix7ImN1cnZlIjotNCwibGV2ZWwiOjIsInN0eWxlIjp7ImhlYWQiOnsibmFtZSI6Im5vbmUifX19XV0=
	\[\begin{tikzcd}
		N & {\bigoplus_i(E\otimes S^{a_i})} & N
		\arrow["\iota"', from=1-1, to=1-2]
		\arrow["r"', from=1-2, to=1-3]
		\arrow[curve={height=-24pt}, Rightarrow, no head, from=1-1, to=1-3]
	\end{tikzcd}\]
	Now consider the following diagram:
	% https://q.uiver.app/#q=WzAsNixbMiwwLCJcXEhvbV97RVxcdGV4dC1cXE1vZH0oTixOJykiXSxbMSwwLCJcXEhvbV97RVxcdGV4dC1cXE1vZH0oXFxiaWdvcGx1c19pKEVcXG90aW1lcyBTXnthX2l9KSxOJykiXSxbMCwwLCJcXEhvbV97RVxcdGV4dC1cXE1vZH0oTixOJykiXSxbMSwxLCJcXEhvbV97XFxwaV8qKEUpfShcXHBpXyooXFxiaWdvcGx1c19pKEVcXG90aW1lcyBTXnthX2l9KSksXFxwaV8qKE4nKSkiXSxbMCwxLCJcXEhvbV97XFxwaV8qKEUpfShcXHBpXyooTiksXFxwaV8qKE4nKSkiXSxbMiwxLCJcXEhvbV97XFxwaV8qKEUpfShcXHBpXyooTiksXFxwaV8qKE4nKSkiXSxbMiwxLCJyXioiXSxbMSwwLCJcXGlvdGFeKiJdLFsyLDAsIiIsMix7ImN1cnZlIjotNCwibGV2ZWwiOjIsInN0eWxlIjp7ImhlYWQiOnsibmFtZSI6Im5vbmUifX19XSxbMSwzLCJcXHBpXyoiXSxbMiw0LCJcXHBpXyoiLDJdLFs0LDMsIihcXHBpXyoocikpXioiXSxbMyw1LCIoXFxwaV8qKFxcaW90YSkpXioiXSxbMCw1LCJcXHBpXyoiLDJdLFs0LDUsIiIsMix7ImN1cnZlIjo0LCJsZXZlbCI6Miwic3R5bGUiOnsiaGVhZCI6eyJuYW1lIjoibm9uZSJ9fX1dXQ==
	\[\begin{tikzcd}
		{\Hom_{E\text-\Mod}(N,N')} & {\Hom_{E\text-\Mod}(\bigoplus_i(E\otimes S^{a_i}),N')} & {\Hom_{E\text-\Mod}(N,N')} \\
		{\Hom_{\pi_*(E)}(\pi_*(N),\pi_*(N'))} & {\Hom_{\pi_*(E)}(\pi_*(\bigoplus_i(E\otimes S^{a_i})),\pi_*(N'))} & {\Hom_{\pi_*(E)}(\pi_*(N),\pi_*(N'))}
		\arrow["{r^*}", from=1-1, to=1-2]
		\arrow["{\iota^*}", from=1-2, to=1-3]
		\arrow[curve={height=-24pt}, Rightarrow, no head, from=1-1, to=1-3]
		\arrow["{\pi_*}", from=1-2, to=2-2]
		\arrow["{\pi_*}"', from=1-1, to=2-1]
		\arrow["{(\pi_*(r))^*}", from=2-1, to=2-2]
		\arrow["{(\pi_*(\iota))^*}", from=2-2, to=2-3]
		\arrow["{\pi_*}"', from=1-3, to=2-3]
		\arrow[curve={height=24pt}, Rightarrow, no head, from=2-1, to=2-3]
	\end{tikzcd}\]
	Each square commutes by functoriality of $\pi_*$. We have shown the middle vertical arrow is an isomorphism. Thus the outside arrows are isomorphisms as well, as a retract of an isomorphism is an isomorphism.

	Now, it remains to show that 
	\[\pi_*:\Hom_{E\text-\Mod}(\Sigma^aN,N')\to\Hom_{\pi_*(E)}:(\pi_*(\Sigma^aN),\pi_*(N'))\]
	is an isomorphism when $N$ is a retract of $\bigoplus_i(E\otimes S^{a_i})$. By the above results, it suffices to show that $\Sigma^aN$ is a retract in $E\text-\Mod$ of $\bigoplus_i(E\otimes S^{a+a_i})$. It further suffices to show that $\bigoplus_i(E\otimes S^{a+a_i})$ is isomorphic in $E\text-\Mod$ to $\Sigma^a\bigoplus_i(E\otimes S^{a_i})$. To that end, note we have isomorphisms in $\cSH$:
	\[\Sigma^a\bigoplus_i(E\otimes S^{a_i})\cong\bigoplus_i(\Sigma^a\otimes E\otimes S^{a_i})\xr{\cong}\bigoplus_i(E\otimes \Sigma^aS^{a_i})\xr{\bigoplus_i(E\otimes\phi_{a,a_i}^{-1})}\bigoplus_i(E\otimes S^{a+a_i}),\]
	where the first isomorphism is that given in \autoref{susp_of_coprod_is_iso_in_E-Mod_to_coprod_of_sus}, the second isomorphism is \autoref{free_susp_is_susp_of_free}, and the last arrow is an $E$-module homomorphism, as it is a coproduct of homomorphisms of free $E$-modules (\autoref{free_forgetful_E-Mod}). Thus, we've shown that $\Sigma^aN$ is a retract of $\bigoplus_i(E\otimes S^{a+a_i})\cong\Sigma^a\bigoplus_i(E\otimes S^{a_i})$ in $E\text-\Mod$, so that by the above argument for all $a\in A$ we have isomorphisms
	\[\pi_*:\Hom_{E\text-\Mod}(\Sigma^aN,N')\to\Hom_{\pi_*(E)}:(\pi_*(\Sigma^aN),\pi_*(N')).\]
	Then by \autoref{lift_A-graded_functor}, the induced map
	\[\pi_*:\Hom_{E\text-\Mod}(N,N')\to\Hom^*_{\pi_*(E)}:(\pi_*(N),\pi_*(N'))\]
	is an isomorphism of $A$-graded abelian groups.
\end{proof}

\todo{EVERYTHING BELOW NEEDS TO BE FIXED}

\begin{corollary}\label{UCT_for_retract}
	Let $(E,\mu,e)$ be a monoid object and $X$ an object in $\cSH$. If there is a collection of $a_i\in A$ indexed by some set $I$ such that $E\otimes X$ is a retract of $\bigoplus_i(E\otimes S^{a_i})$ in $E\text-\Mod$,\footnote{Here $\bigoplus_i(E\otimes S^{a_i})$ is a coproduct (\autoref{coproduct_of_E_modules_is_coproduct_in_E_mod}) of a bunch of free $E$-module objects (\autoref{free_forgetful_E-Mod}), so it is itself a $E$-module object.} then for all $E$-module objects $(N,\kappa)$, the assignment
	\[[X,N]_*\to\Hom_{\pi_*(E)}^*(E_*(X),\pi_*(N))\]
	sending $f:S^a\otimes X\to N$ to the map $E_{*-a}(X)\to\pi_*(N)$ which sends a class $x:S^{b-a}\to E\otimes X$ to the composition
	\[S^b\cong S^{b-a}S^a\xr{xS^a}EXS^a\xr{E\phi XS^a}ES^aS^{-a}XS^a\xr{ES^a\tau S^a}ES^aXS^{-a}S^a\xr{ES^aX\phi^{-1}}ES^aX\xr{Ef}EN\xr\kappa N\]
	is an $A$-graded isomorphism of $A$-graded abelian groups (where here we are omitting $\otimes$, associators, and unitors so the composition fits on the page).
\end{corollary}
\begin{proof}
	By the universal property of the coproduct, it suffices to show that for all $a\in A$, the restriction
	\[T_a:[X,N]_a\to\Hom_{\pi_*(E)}^a(E_*(X),\pi_*(N))\]
	is a well-defined isomorphism of abelian groups. First of all, by unravelling definitions it is straightforward to see that $T_a$ factors as
	% https://q.uiver.app/#q=WzAsNixbMSwwLCJbXFxTaWdtYV5hWCxOXSJdLFsxLDEsIlxcSG9tX3tFXFx0ZXh0LVxcTW9kfShFXFxvdGltZXMgXFxTaWdtYV5hWCxOKSJdLFsxLDMsIlxcSG9tX3tcXHBpXyooRSl9KEVfeyotYX0oWCksXFxwaV8qKE4pKSJdLFswLDAsIltYLE5dX2EiXSxbMiwzLCJcXEhvbV97XFxwaV8qKEUpfV5hKEVfKihYKSxcXHBpXyooTikpIl0sWzEsMiwiXFxIb21fe1xccGlfKihFKX0oRV8qKFxcU2lnbWFeYVgpLFxccGlfKihOKSJdLFswLDEsIlxcdGV4dHthZGp9Il0sWzMsMCwiIiwwLHsibGV2ZWwiOjIsInN0eWxlIjp7ImhlYWQiOnsibmFtZSI6Im5vbmUifX19XSxbMiw0LCIiLDAseyJsZXZlbCI6Miwic3R5bGUiOnsiaGVhZCI6eyJuYW1lIjoibm9uZSJ9fX1dLFsxLDUsIlxccGlfKigtKSJdLFs1LDIsInsoeyh0X1heYSl9XnstMX0pfV4qIl1d
	\[\begin{tikzcd}
		{[X,N]_a} & {[\Sigma^aX,N]} \\
		& {\Hom_{E\text-\Mod}(E\otimes \Sigma^aX,N)} \\
		& {\Hom_{\pi_*(E)}(E_*(\Sigma^aX),\pi_*(N)} \\
		& {\Hom_{\pi_*(E)}(E_{*-a}(X),\pi_*(N))} & {\Hom_{\pi_*(E)}^a(E_*(X),\pi_*(N))}
		\arrow["{\text{adj}}", from=1-2, to=2-2]
		\arrow[Rightarrow, no head, from=1-1, to=1-2]
		\arrow[Rightarrow, no head, from=4-2, to=4-3]
		\arrow["{\pi_*(-)}", from=2-2, to=3-2]
		\arrow["{{({(t_X^a)}^{-1})}^*}", from=3-2, to=4-2]
	\end{tikzcd}\]
	where the first isomorphism is the free-forgetful adjunction for $E$-modules (\autoref{free_forgetful_E-Mod}), the second map is that induced by the functor $\pi_*$ constructed in \autoref{E-module_N_implies_pi*N_is_pi*E_module}, and the third map is induced by the $A$-graded isomorphism of left $\pi_*(E)$-modules $(t_X^a)^{-1}:E_{*-a}(X)\to E_*(\Sigma^aX)$ from (\autoref{E_homology_suspension_iso_t^a's_appendix}). Furthermore, since we have isomorphisms
	\[E\otimes\Sigma^aX=E\otimes S^a\otimes X\cong S^a\otimes E\otimes X\]
	and
	\[S^a\otimes\bigoplus_i(E\otimes S^{a_i})\cong\bigoplus_i(S^a\otimes E\otimes S^{a_i})\cong \bigoplus_i (E\otimes S^{a}\otimes S^{a_i})\cong \bigoplus_i (E\otimes S^{a+a_i}),\]
	we have that $E\otimes\Sigma^aX$ is a retract of $\bigoplus_i E\otimes S^{a+a_i}$, as $E\otimes X$ is a retract of $\bigoplus_i(E\otimes S^{a_i})$, so that by \autoref{pi_*_iso_when_N_retract_of_wedge_of_susps}, the map
	\[\pi_*:\Hom_{E\text-\Mod}(E\otimes\Sigma^aX,N)\to\Hom_{\pi_*(E)}(E_*(\Sigma^aX),\pi_*(N))\]
	is an isomorphism. Thus, $T_a:[X,N]_a\to\Hom_{\pi_*(E)}^a(E_*(X),\pi_*(N))$ is an isomorphism, as desired.
\end{proof}

\begin{lemma}\label{retract_of_module_whose_idempotent_is_module_homomorphism_is_module}
	Let $(E,\mu,e)$ be a monoid object in $\cSH$. Further suppose we have some object $X$ in $\cSH$ and an $E$-module object $(N,\kappa)$, along with a commuting diagram in $\cSH$
	% https://q.uiver.app/#q=WzAsMyxbMCwwLCJYIl0sWzEsMCwiTiJdLFsyLDAsIlgiXSxbMCwxLCJcXGlvdGEiLDJdLFsxLDIsInIiLDJdLFswLDIsIiIsMix7ImN1cnZlIjotMywibGV2ZWwiOjIsInN0eWxlIjp7ImhlYWQiOnsibmFtZSI6Im5vbmUifX19XV0=
	\[\begin{tikzcd}
		X & N & X
		\arrow["\iota"', from=1-1, to=1-2]
		\arrow["r"', from=1-2, to=1-3]
		\arrow[curve={height=-18pt}, Rightarrow, no head, from=1-1, to=1-3]
	\end{tikzcd}\]
	Then if $\ell:=\iota\circ r:N\to N$ is an $E$-module homomorphism, then $X$ is canonically an $E$-module object with structure map
	\[\kappa_X:E\otimes X\xr{E\otimes\iota}E\otimes N\xr{\kappa}N\xr{r}X,\]
	and furthermore, the maps $\iota:X\to N$ and $r:N\to X$ are $E$-module homomorphisms.
\end{lemma}
\begin{proof}
	First, in order to show $(X,\kappa_X)$ is an $E$-module, we need to show the two diagrams in \autoref{left_module_object} commute. To see the unitality diagram holds, consider the following diagram:
	% https://q.uiver.app/#q=WzAsNyxbMCwwLCJTXFxvdGltZXMgWCJdLFsyLDAsIkVcXG90aW1lcyBYIl0sWzIsMSwiRVxcb3RpbWVzIE4iXSxbMiwyLCJOIl0sWzIsMywiWCJdLFswLDMsIlgiXSxbMSwxLCJTXFxvdGltZXMgTiJdLFswLDEsImVcXG90aW1lcyBYIl0sWzEsMiwiRVxcb3RpbWVzXFxpb3RhIl0sWzIsMywiXFxrYXBwYSJdLFszLDQsInIiXSxbMCw1LCJcXGxhbWJkYV9YIiwyXSxbNSw0LCIiLDIseyJsZXZlbCI6Miwic3R5bGUiOnsiaGVhZCI6eyJuYW1lIjoibm9uZSJ9fX1dLFswLDYsIlNcXG90aW1lc1xcaW90YSIsMV0sWzYsMiwiZVxcb3RpbWVzIE4iXSxbNiwzLCJcXGxhbWJkYV9OIiwxXSxbNSwzLCJcXGlvdGEiXV0=
	\[\begin{tikzcd}
		{S\otimes X} && {E\otimes X} \\
		& {S\otimes N} & {E\otimes N} \\
		&& N \\
		X && X
		\arrow["{e\otimes X}", from=1-1, to=1-3]
		\arrow["E\otimes\iota", from=1-3, to=2-3]
		\arrow["\kappa", from=2-3, to=3-3]
		\arrow["r", from=3-3, to=4-3]
		\arrow["{\lambda_X}"', from=1-1, to=4-1]
		\arrow[Rightarrow, no head, from=4-1, to=4-3]
		\arrow["S\otimes\iota"{description}, from=1-1, to=2-2]
		\arrow["{e\otimes N}", from=2-2, to=2-3]
		\arrow["{\lambda_N}"{description}, from=2-2, to=3-3]
		\arrow["\iota", from=4-1, to=3-3]
	\end{tikzcd}\]
	The large left triangle commutes by naturality of $\lambda$. The top trapezoid commutes by functoriality of $-\otimes-$. The small middle right triangle commutes by unitality of $\kappa$. Finally, the bottom triangle commutes by definition, since we are assuming $r\circ\iota=\id_X$. Now the right composition is $\kappa_X$, so we have shown $\kappa_X\circ(e\otimes X)=\lambda_X$, as desired. Now, consider the following diagram:
	% https://q.uiver.app/#q=WzAsMTEsWzAsMCwiRVxcb3RpbWVzIEVcXG90aW1lcyBYIl0sWzMsMCwiRVxcb3RpbWVzIFgiXSxbMywxLCJFXFxvdGltZXMgTiJdLFszLDIsIk4iXSxbMywzLCJYIl0sWzAsMSwiRVxcb3RpbWVzIEVcXG90aW1lcyBOIl0sWzAsMiwiRVxcb3RpbWVzIE4iXSxbMCwzLCJFXFxvdGltZXMgWCJdLFsxLDMsIkVcXG90aW1lcyBOIl0sWzIsMywiTiJdLFsxLDEsIkVcXG90aW1lcyBFXFxvdGltZXMgTiJdLFswLDEsIlxcbXVcXG90aW1lcyBYIl0sWzEsMiwiRVxcb3RpbWVzXFxpb3RhIl0sWzIsMywiXFxrYXBwYSJdLFszLDQsInIiXSxbMCw1LCJFXFxvdGltZXMgRVxcb3RpbWVzXFxpb3RhIiwyXSxbNSw2LCJFXFxvdGltZXNcXGthcHBhIiwyXSxbNiw3LCJFXFxvdGltZXMgciIsMl0sWzcsOCwiRVxcb3RpbWVzXFxpb3RhIiwyXSxbOCw5LCJcXGthcHBhIiwyXSxbOSw0LCJyIiwyXSxbNSwxMCwiRVxcb3RpbWVzIEVcXG90aW1lc1xcZWxsIiwyXSxbMCwxMCwiRVxcb3RpbWVzIEVcXG90aW1lc1xcaW90YSIsMV0sWzEwLDgsIkVcXG90aW1lc1xca2FwcGEiXSxbNiw4LCJFXFxvdGltZXNcXGVsbCJdLFsxMCwyLCJcXG11XFxvdGltZXMgTiJdLFszLDksIiIsMSx7ImxldmVsIjoyLCJzdHlsZSI6eyJoZWFkIjp7Im5hbWUiOiJub25lIn19fV1d
	\[\begin{tikzcd}
		{E\otimes E\otimes X} &&& {E\otimes X} \\
		{E\otimes E\otimes N} & {E\otimes E\otimes N} && {E\otimes N} \\
		{E\otimes N} &&& N \\
		{E\otimes X} & {E\otimes N} & N & X
		\arrow["{\mu\otimes X}", from=1-1, to=1-4]
		\arrow["E\otimes\iota", from=1-4, to=2-4]
		\arrow["\kappa", from=2-4, to=3-4]
		\arrow["r", from=3-4, to=4-4]
		\arrow["{E\otimes E\otimes\iota}"', from=1-1, to=2-1]
		\arrow["E\otimes\kappa"', from=2-1, to=3-1]
		\arrow["{E\otimes r}"', from=3-1, to=4-1]
		\arrow["E\otimes\iota"', from=4-1, to=4-2]
		\arrow["\kappa"', from=4-2, to=4-3]
		\arrow["r"', from=4-3, to=4-4]
		\arrow["{E\otimes E\otimes\ell}"', from=2-1, to=2-2]
		\arrow["{E\otimes E\otimes\iota}"{description}, from=1-1, to=2-2]
		\arrow["E\otimes\kappa", from=2-2, to=4-2]
		\arrow["E\otimes\ell", from=3-1, to=4-2]
		\arrow["{\mu\otimes N}", from=2-2, to=2-4]
		\arrow[Rightarrow, no head, from=3-4, to=4-3]
	\end{tikzcd}\]
	The top trapezoid commutes by funtoriality of $-\otimes-$. The top left triangle commutes by functoriality of $-\otimes-$ and the fact that $\ell\circ\iota=\iota\circ r\circ\iota=\iota\circ\id_X=\iota$.  The middle left trapezoid commutes by since $\ell$ is an $E$-module homomorphism, by assumption. The bottom left triangle commutes by functoriality of $-\otimes-$ and the fact that $\iota\circ r=\ell$. Thus, we have shown that $(X,\kappa_X)$ is an $E$-module object, as desired.

	Now, it remains to show that $\iota:X\to N$ and $r:N\to X$ are $E$-module homomorphisms. To that end, consider the following two diagrams:
	% https://q.uiver.app/#q=WzAsMTIsWzAsMCwiRVxcb3RpbWVzIFgiXSxbMiwwLCJFXFxvdGltZXMgTiJdLFsyLDMsIk4iXSxbMCwxLCJFXFxvdGltZXMgTiJdLFswLDIsIk4iXSxbMCwzLCJYIl0sWzMsMCwiRVxcb3RpbWVzIFgiXSxbMywzLCJYIl0sWzMsMSwiRVxcb3RpbWVzIE4iXSxbMywyLCJOIl0sWzEsMCwiRVxcb3RpbWVzIE4iXSxbMSwzLCJOIl0sWzEsMiwiXFxrYXBwYSIsMl0sWzAsMywiRVxcb3RpbWVzXFxpb3RhIiwyXSxbMyw0LCJcXGthcHBhIiwyXSxbNCw1LCJyIiwyXSxbMSw2LCJFXFxvdGltZXMgciJdLFsyLDcsInIiXSxbNiw4LCJFXFxvdGltZXNcXGlvdGEiXSxbOCw5LCJcXGthcHBhIl0sWzksNywiciJdLFsxLDgsIkVcXG90aW1lc1xcZWxsIiwxXSxbMiw5LCJcXGVsbCIsMV0sWzEwLDExLCJcXGthcHBhIl0sWzAsMTAsIkVcXG90aW1lc1xcaW90YSJdLFs1LDExLCJcXGlvdGEiLDJdLFs0LDExLCJcXGVsbCIsMV0sWzMsMTAsIkVcXG90aW1lcyBcXGVsbCIsMV1d
	\[\begin{tikzcd}
		{E\otimes X} & {E\otimes N} & {E\otimes N} & {E\otimes X} \\
		{E\otimes N} &&& {E\otimes N} \\
		N &&& N \\
		X & N & N & X
		\arrow["\kappa"', from=1-3, to=4-3]
		\arrow["E\otimes\iota"', from=1-1, to=2-1]
		\arrow["\kappa"', from=2-1, to=3-1]
		\arrow["r"', from=3-1, to=4-1]
		\arrow["{E\otimes r}", from=1-3, to=1-4]
		\arrow["r", from=4-3, to=4-4]
		\arrow["E\otimes\iota", from=1-4, to=2-4]
		\arrow["\kappa", from=2-4, to=3-4]
		\arrow["r", from=3-4, to=4-4]
		\arrow["E\otimes\ell"{description}, from=1-3, to=2-4]
		\arrow["\ell"{description}, from=4-3, to=3-4]
		\arrow["\kappa", from=1-2, to=4-2]
		\arrow["E\otimes\iota", from=1-1, to=1-2]
		\arrow["\iota"', from=4-1, to=4-2]
		\arrow["\ell"{description}, from=3-1, to=4-2]
		\arrow["{E\otimes \ell}"{description}, from=2-1, to=1-2]
	\end{tikzcd}\]
	The trapezoids in each diagram commute since we are assuming $\ell$ is a $E$-module homomorphism. The four triangles commute since $\ell\circ\iota=\iota$ and $r\circ\ell=r$. Thus, we have shown that $\kappa_X\circ(E\otimes r)=r\circ\kappa$ and $\kappa\circ(E\otimes\iota)=\iota\circ\kappa_X$, so we indeed have that $\iota$ and $r$ are $E$-module homomorphisms, as desired.
\end{proof}

\begin{proposition}\label{if_pi_*N_graded_proj_then_retract_of_wedge_of_susps}
	Let $(E,\mu,e)$ be a monoid object and $(N,\kappa)$ an $E$-module object in $\cSH$. Further suppose that $E$ and $N$ are cellular and that $\pi_*(N)$ is a \emph{graded projective} (\autoref{graded_projective_module}) left $\pi_*(E)$-module (via \autoref{E-module_N_implies_pi*N_is_pi*E_module}). Then given some homogeneous generating set $\{x_i\}_{i\in I}\sseq \pi_*(N)$, $N$ is a retract of $\bigoplus_i(E\otimes S^{|x_i|})$ in $E\text-\Mod$.\footnote{Here $\bigoplus_i(E\otimes S^{a_i})$ is a coproduct (\autoref{coproduct_of_E_modules_is_coproduct_in_E_mod}) of a bunch of free $E$-module objects (\autoref{free_forgetful_E-Mod}), so it is itself a $E$-module object.}
\end{proposition}
\begin{proof}
	Let $M:=\bigoplus_i(E\otimes S^{|x_i|})$. We have a map
	\[r:M\to N\]
	induced by the maps
	\[r_i:E\otimes S^{|x_i|}\xrightarrow{E\otimes x_i}E\otimes N\xr{\kappa} N.\]
	This is a homomorphism of $E$-module objects:
	% https://q.uiver.app/#q=WzAsOCxbMCwwLCJFXFxvdGltZXNcXGJpZ29wbHVzX2koRVxcb3RpbWVzIFNee3x4X2l8fSkiXSxbMiwwLCJFXFxvdGltZXMgTiJdLFsyLDQsIk4iXSxbMCwyLCJcXGJpZ29wbHVzX2koRVxcb3RpbWVzIEVcXG90aW1lcyBTXnt8eF9pfH0pIl0sWzAsNCwiXFxiaWdvcGx1c19pKEVcXG90aW1lcyBTXnt8eF9pfH0pIl0sWzEsMiwiXFxiaWdvcGx1c19pKEVcXG90aW1lcyBOKSJdLFsxLDMsIlxcYmlnb3BsdXNfaSBOIl0sWzEsMSwiRVxcb3RpbWVzXFxiaWdvcGx1c19pTiJdLFswLDEsIkVcXG90aW1lcyByIl0sWzEsMiwiXFxrYXBwYSJdLFswLDMsIlxcY29uZyIsMl0sWzMsNCwiXFxiaWdvcGx1c19pKFxcbXVcXG90aW1lcyBTXnt8eF9pfH0pIiwyXSxbNCwyLCJyIl0sWzMsNSwiXFxiaWdvcGx1c19pKEVcXG90aW1lcyByX2kpIl0sWzUsNiwiXFxiaWdvcGx1c19pXFxrYXBwYSJdLFs2LDIsIlxcbmFibGEiXSxbNCw2LCJcXGJpZ29wbHVzX2lyX2kiXSxbMCw3LCJFXFxvdGltZXNcXGJpZ29wbHVzX2lyX2kiLDFdLFs3LDUsIlxcY29uZyIsMl0sWzcsMSwiRVxcb3RpbWVzXFxuYWJsYSIsMV0sWzUsMSwiXFxuYWJsYSIsMl1d
	\[\begin{tikzcd}
		{E\otimes\bigoplus_i(E\otimes S^{|x_i|})} && {E\otimes N} \\
		& {E\otimes\bigoplus_iN} \\
		{\bigoplus_i(E\otimes E\otimes S^{|x_i|})} & {\bigoplus_i(E\otimes N)} \\
		& {\bigoplus_i N} \\
		{\bigoplus_i(E\otimes S^{|x_i|})} && N
		\arrow["{E\otimes r}", from=1-1, to=1-3]
		\arrow["\kappa", from=1-3, to=5-3]
		\arrow["\cong"', from=1-1, to=3-1]
		\arrow["{\bigoplus_i(\mu\otimes S^{|x_i|})}"', from=3-1, to=5-1]
		\arrow["r", from=5-1, to=5-3]
		\arrow["{\bigoplus_i(E\otimes r_i)}", from=3-1, to=3-2]
		\arrow["{\bigoplus_i\kappa}", from=3-2, to=4-2]
		\arrow["\nabla", from=4-2, to=5-3]
		\arrow["{\bigoplus_ir_i}", from=5-1, to=4-2]
		\arrow["{E\otimes\bigoplus_ir_i}"{description}, from=1-1, to=2-2]
		\arrow["\cong"', from=2-2, to=3-2]
		\arrow["E\otimes\nabla"{description}, from=2-2, to=1-3]
		\arrow["\nabla"', from=3-2, to=1-3]
	\end{tikzcd}\]
	The right trapezoid commutes by naturality of $\nabla$. The bottom triangle commutes by the fact that $\nabla\circ\bigoplus_ir_i$ and $r$ satisfy the same universal property for the coproduct. Every other region commutes by additivity of $E\otimes-$, except the left trapezoid: Note that by expanding out how $r_i$ is defined, it becomes
	% https://q.uiver.app/#q=WzAsNixbMCwwLCJcXGJpZ29wbHVzX2koRVxcb3RpbWVzIEVcXG90aW1lcyBTXnt8eF9pfH0pIl0sWzQsMCwiXFxiaWdvcGx1c19pKEVcXG90aW1lcyBFXFxvdGltZXMgWCkiXSxbMiwwLCJcXGJpZ29wbHVzX2koRVxcb3RpbWVzIEVcXG90aW1lcyBOKSJdLFs0LDEsIlxcYmlnb3BsdXNfaShFXFxvdGltZXMgWCkiXSxbMCwxLCJcXGJpZ29wbHVzX2koRVxcb3RpbWVzIFNee3x4X2l8fSkiXSxbMiwxLCJcXGJpZ29wbHVzX2koRVxcb3RpbWVzIE4pIl0sWzAsMiwiXFxiaWdvcGx1c19pKEVcXG90aW1lcyBFXFxvdGltZXMgeF9pKSJdLFsyLDEsIlxcYmlnb3BsdXNfaShFXFxvdGltZXNcXGthcHBhKSJdLFsxLDMsIlxcYmlnb3BsdXNfaVxca2FwcGEiXSxbMCw0LCJcXGJpZ29wbHVzX2koXFxtdVxcb3RpbWVzIFNee3x4X2l8fSkiLDJdLFs0LDUsIlxcYmlnb3BsdXNfaShFXFxvdGltZXMgeF9pKSIsMl0sWzUsMywiXFxiaWdvcGx1c19pXFxrYXBwYSIsMl0sWzIsNSwiXFxiaWdvcGx1c19pKFxcbXVcXG90aW1lcyBYKSJdXQ==
	\[\begin{tikzcd}
		{\bigoplus_i(E\otimes E\otimes S^{|x_i|})} && {\bigoplus_i(E\otimes E\otimes N)} && {\bigoplus_i(E\otimes E\otimes X)} \\
		{\bigoplus_i(E\otimes S^{|x_i|})} && {\bigoplus_i(E\otimes N)} && {\bigoplus_i(E\otimes X)}
		\arrow["{\bigoplus_i(E\otimes E\otimes x_i)}", from=1-1, to=1-3]
		\arrow["{\bigoplus_i(E\otimes\kappa)}", from=1-3, to=1-5]
		\arrow["{\bigoplus_i\kappa}", from=1-5, to=2-5]
		\arrow["{\bigoplus_i(\mu\otimes S^{|x_i|})}"', from=1-1, to=2-1]
		\arrow["{\bigoplus_i(E\otimes x_i)}"', from=2-1, to=2-3]
		\arrow["{\bigoplus_i\kappa}"', from=2-3, to=2-5]
		\arrow["{\bigoplus_i(\mu\otimes X)}", from=1-3, to=2-3]
	\end{tikzcd}\]
	The left square commutes by functoriality of $-\otimes-$, and the right square commutes by coherence for $\kappa$. Hence, we've shown that $r$ is a homomorphism of $E$-modules, as desired. Thus, $r$ induces a homomorphism of left $\pi_*(E)$-modules $\pi_*(r)\in\Hom_{\pi_*(E)}(\pi_*(M),\pi_*(N))$. Further note that for all $i\in I$, $x_i$ is in the image of $\pi_*(r)$, as by definition $\pi_*(r)$ sends the class 
	\[S^{|x_i|}\xr{e\otimes S^{|x_i|}}E\otimes S^{|x_i|}\into M\]
	in $\pi_{|x_i|}(M)$ to the composition
	\[S^{|x_i|}\xr{e\otimes S^{|x_i|}}E\otimes S^{|x_i|}\xr{E\otimes x_i}E\otimes N\xr\kappa N,\]
	and by unitality of $\kappa$ this composition is simply $x_i:S^{|x_i|}\to N$. Thus, we have constructed a surjective $A$-graded homomorphism $\pi_*(r):\pi_*(M)\to \pi_*(N)$ of left $\pi_*(E)$-modules, so that since $\pi_*(N)$ is projective graded module there exists an $A$-graded left $\pi_*(E)$-module homomorphism $\iota:\pi_*(N)\to\pi_*(M)$ which makes the following diagram commute:
	% https://q.uiver.app/#q=WzAsMyxbMCwxLCJcXHBpXyooTikiXSxbMSwxLCJcXHBpXyooTikiXSxbMSwwLCJcXHBpXyooTSkiXSxbMCwxLCIiLDAseyJsZXZlbCI6Miwic3R5bGUiOnsiaGVhZCI6eyJuYW1lIjoibm9uZSJ9fX1dLFsyLDEsIlxccGlfKihyKSJdLFswLDIsIlxcaW90YSJdXQ==
	\[\begin{tikzcd}
		& {\pi_*(M)} \\
		{\pi_*(N)} & {\pi_*(N)}
		\arrow[Rightarrow, no head, from=2-1, to=2-2]
		\arrow["{\pi_*(r)}", from=1-2, to=2-2]
		\arrow["\iota", from=2-1, to=1-2]
	\end{tikzcd}\]
	Thus we have an idempotent of left $A$-graded $\pi_*(E)$-modules:
	% https://q.uiver.app/#q=WzAsMyxbMCwwLCJcXHBpXyooTSkiXSxbMSwwLCJcXHBpXyooTikiXSxbMiwwLCJcXHBpXyooTSkiXSxbMCwxLCJcXHBpXyoocikiXSxbMSwyLCJcXGlvdGEiXV0=
	\[\begin{tikzcd}
		{\pi_*(M)} & {\pi_*(N)} & {\pi_*(M)}
		\arrow["{\pi_*(r)}", from=1-1, to=1-2]
		\arrow["\iota", from=1-2, to=1-3]
	\end{tikzcd}\]
	Now, by \autoref{pi_*_iso_when_N_retract_of_wedge_of_susps}, since $M=\bigoplus_i(E\otimes S^{|x_i|})$, we have that the map
	\[\pi_*:\Hom_{E\text-\Mod}(M,M)\to\Hom_{\pi_*(E)\text-\Mod}(\pi_*(M),\pi_*(M))\]
	is an isomorphism of abelian groups, so that the above idempotent is induced by some endomorphism $\ell:M\to M$ of $E$-module objects. Further note that by functoriality of $\pi_*$,
	\[\pi_*(\ell\circ\ell)=\pi_*(\ell)\circ\pi_*(\ell)=\pi_*(\ell),\]
	and again since $\pi_*$ is an isomorphism here, we have that $\ell\circ\ell=\ell$, so that $\ell$ is an idempotent in $\cSH$. By \autoref{idempotent_splits_in_tri_cat_with_countable_coproducts}, every idempotent in $\cSH$ splits, meaning $\ell$ factors in $\cSH$ as
	\[\ell:M\xr{r'}X\xr{\iota'}M\]
	with $r'\circ \iota'=\id_X$. Since $X$ is a retract of an $E$-module object, and the corresponding idempotent is an $E$-module homomorphism, by \autoref{retract_of_module_whose_idempotent_is_module_homomorphism_is_module} we have that $X$ may be canonically viewed as an $E$-module object, and that $r':M\to X$ and $\iota':X\to M$ are homomorphisms of $E$-module objects. Note that since $E$ and each $S^{|x_i|}$ are cellular, $E\otimes S^{|x_i|}$ is cellular for all $i\in I$ (by \autoref{cellular_closed_under_tensor}), so that $M=\bigoplus_i(E\otimes S^{|x_i|})$ is cellular, as by definition an arbitrary coproduct of cellular objects is cellular. Thus by \autoref{cellular_idempotent_splits_cellularly}, $X$ is cellular as well. Now consider the following commutative diagram
	% https://q.uiver.app/#q=WzAsOSxbMywxLCJcXHBpXyooTSkiXSxbMiwyLCJcXHBpXyooWCkiXSxbNCwwLCJcXHBpXyooTikiXSxbNCwyLCJcXHBpXyooWCkiXSxbNSwxLCJcXHBpXyooTSkiXSxbNiwxLCJcXHBpXyooWCkiXSxbMCwxLCJcXHBpXyooTikiXSxbMSwxLCJcXHBpXyooTSkiXSxbMiwwLCJcXHBpXyooTikiXSxbMSwwLCJcXHBpXyooXFxpb3RhJykiLDJdLFswLDIsIlxccGlfKihyKSJdLFswLDMsIlxccGlfKihyJykiLDJdLFszLDQsIlxccGlfKihcXGlvdGEnKSIsMl0sWzIsNCwiXFxpb3RhIl0sWzQsNSwiXFxwaV8qKHInKSJdLFszLDUsIiIsMSx7ImN1cnZlIjozLCJsZXZlbCI6Miwic3R5bGUiOnsiaGVhZCI6eyJuYW1lIjoibm9uZSJ9fX1dLFs2LDcsIlxcaW90YSIsMl0sWzcsMSwiXFxwaV8qKHInKSIsMl0sWzcsOCwiXFxwaV8qKHIpIl0sWzgsMCwiXFxpb3RhIl0sWzYsOCwiIiwwLHsiY3VydmUiOi0zLCJsZXZlbCI6Miwic3R5bGUiOnsiaGVhZCI6eyJuYW1lIjoibm9uZSJ9fX1dLFsxLDMsIiIsMCx7ImxldmVsIjoyLCJzdHlsZSI6eyJoZWFkIjp7Im5hbWUiOiJub25lIn19fV0sWzgsMiwiIiwwLHsibGV2ZWwiOjIsInN0eWxlIjp7ImhlYWQiOnsibmFtZSI6Im5vbmUifX19XSxbNywwLCJcXHBpXyooXFxlbGwpIl0sWzAsNCwiXFxwaV8qKFxcZWxsKSJdXQ==
	\[\begin{tikzcd}
		&& {\pi_*(N)} && {\pi_*(N)} \\
		{\pi_*(N)} & {\pi_*(M)} && {\pi_*(M)} && {\pi_*(M)} & {\pi_*(X)} \\
		&& {\pi_*(X)} && {\pi_*(X)}
		\arrow["{\pi_*(\iota')}"', from=3-3, to=2-4]
		\arrow["{\pi_*(r)}", from=2-4, to=1-5]
		\arrow["{\pi_*(r')}"', from=2-4, to=3-5]
		\arrow["{\pi_*(\iota')}"', from=3-5, to=2-6]
		\arrow["\iota", from=1-5, to=2-6]
		\arrow["{\pi_*(r')}", from=2-6, to=2-7]
		\arrow[curve={height=18pt}, Rightarrow, no head, from=3-5, to=2-7]
		\arrow["\iota"', from=2-1, to=2-2]
		\arrow["{\pi_*(r')}"', from=2-2, to=3-3]
		\arrow["{\pi_*(r)}", from=2-2, to=1-3]
		\arrow["\iota", from=1-3, to=2-4]
		\arrow[curve={height=-18pt}, Rightarrow, no head, from=2-1, to=1-3]
		\arrow[Rightarrow, no head, from=3-3, to=3-5]
		\arrow[Rightarrow, no head, from=1-3, to=1-5]
		\arrow["{\pi_*(\ell)}", from=2-2, to=2-4]
		\arrow["{\pi_*(\ell)}", from=2-4, to=2-6]
	\end{tikzcd}\]
	From this diagram we read off that the middle diagonal composition
	\[\pi_*(X)\xr{\pi_*(\iota')}\pi_*(M)\xr{\pi_*(r)}\pi_*(N)\]
	is an isomorphism with inverse $\pi_*(r')\circ\iota$. Now, since $X$ and $N$ are cellular, and $\pi_*(r\circ\iota')$ is an isomorphism, by \autoref{cellular_closed_under_iso} we have that $r\circ\iota'$ is an isomorphism, say with inverse $p$. Thus we have a commuting diagram
	% https://q.uiver.app/#q=WzAsNCxbMiwwLCJNIl0sWzQsMCwiTiJdLFswLDAsIk4iXSxbMSwxLCJYIl0sWzAsMSwiciIsMl0sWzIsMCwiXFxpb3RhJ1xcY2lyYyBwIiwyXSxbMiwzLCJwIiwyXSxbMywwLCJcXGlvdGEnIiwyXSxbMiwxLCIiLDAseyJjdXJ2ZSI6LTQsImxldmVsIjoyLCJzdHlsZSI6eyJoZWFkIjp7Im5hbWUiOiJub25lIn19fV1d
	\[\begin{tikzcd}
		N && M && N \\
		& X
		\arrow["r"', from=1-3, to=1-5]
		\arrow["{\iota'\circ p}"', from=1-1, to=1-3]
		\arrow["p"', from=1-1, to=2-2]
		\arrow["{\iota'}"', from=2-2, to=1-3]
		\arrow[curve={height=-24pt}, Rightarrow, no head, from=1-1, to=1-5]
	\end{tikzcd}\]
	and the middle row exhibits $N$ as a retract of $M=\bigoplus_i(E\otimes S^{|x_i|})$, as desired. It remains to show this is a retract in $E\text-\Mod$, i.e., that $r$ and $\iota'\circ p$ are homomorphisms of $E$-module objects. Above we constructed $r$ to be a homomorphism of $E$-modules. We also showed that $X$ is an $E$-module and that $\iota'$ is an $E$-module homomorphism. Thus, it remains to show that $p:N\to X$ is an $E$-module homomorphism. But we know that $p$ is the inverse of $r\circ\iota'$ in $\cSH$, and we know $r$ and $\iota'$ are morphisms in $E\text-\Mod$, so that $p$ is the inverse of $r\circ\iota'$ in $E\text-\Mod$, meaning $p$ is indeed an $E$-module homomorphism as desired.
\end{proof}

\begin{corollary}\label{UCT_for_graded_projective}
	Let $(E,\mu,e)$ be a monoid object and let $X$ and $Y$ be objects in $\cSH$. Then if $E$ and $X$ are cellular and $E_*(X)$ is a graded projective (\autoref{graded_projective_module}) left $\pi_*(E)$-module (\autoref{module}), then the map
	\[[X,E\otimes Y]_*\to\Hom_{\pi_*(E)}^*(E_*(X),E_*(Y))\]
	sending $f:S^a\otimes X\to E\otimes Y$ to the map $E_{*-a}(X)\to E_*(Y)$ which sends a class $x:S^{b-a}\to E\otimes X$ to the composition
	\[S^b\cong S^{b-a}S^a\xr{xS^a}EXS^a\xr{E\phi XS^a}ES^aS^{-a}XS^a\xr{ES^a\tau S^a}ES^aXS^{-a}S^a\xr{ES^aX\phi^{-1}}ES^aX\xr{Ef}EEY\xr{\mu Y}EY\]
	is an $A$-graded isomorphism of $A$-graded abelian groups (where here we are omitting $\otimes$, associators, and unitors so the composition fits on the page).
\end{corollary}
\begin{proof}
	Since: (1) $E\otimes X$ is a free $E$-module object (\autoref{free_forgetful_E-Mod}), (2) $E_*(X)=\pi_*(E\otimes X)$ is a graded projective left $\pi_*(E)$-module, and (3) $E\otimes X$ is cellular (\autoref{cellular_closed_under_tensor}), by \autoref{if_pi_*N_graded_proj_then_retract_of_wedge_of_susps} it follows that $E\otimes X$ is a retract of $\bigoplus_i(E\otimes S^{a_i})$ in $E\text-\Mod$ for some collection of $a_i\in A$ indexed by some set $I$. Thus the desired result follows by \autoref{UCT_for_retract} with $N=E\otimes Y$ (which is an $E$-module by \autoref{free_forgetful_E-Mod}).
\end{proof}

\begin{proposition}\label{[X,EY]-->Hom_E*E(E_*X,E_*EY)_is_iso_for_nice_E,X,Y_appendix}
    Let $(E,\mu,e)$ commutative monoid object, and $X$ and $Y$ objects in $\cSH$. Suppose that\begin{enumerate}
        \item $E$ is flat (\autoref{flat}) and cellular (\autoref{cellular}),
        \item $X$ is cellular and $E_*(X)$ is a graded projective (\autoref{graded_projective_module}) left $\pi_*(E)$-module (via \autoref{module_main}),
        \item $Y$ is cellular \emph{or} $E_*(Y)$ is a graded projective left $\pi_*(E)$ module (via \autoref{module_main}).
    \end{enumerate}
    Then the map 
    \[\Psi_{X,Y}:[X,E\otimes Y]_*\to\Hom_{E_*(E)\text-\CoMod}^*(E_*(X),E_*(E\otimes Y))\]
    sending $f:S^a\otimes X\to E\otimes Y$ to the map $E_{*-a}(X)\to E_*(E\otimes Y)$ which sends $x:S^{b-a}\to E\otimes X$ to the composition
	\[S^b\cong S^{b-a}S^a\xr{xS^a}EXS^a\xr{E\phi XS^a}ES^aS^{-a}XS^a\xr{ES^a\tau S^a}ES^aXS^{-a}S^a\xr{ES^aX\phi^{-1}}ES^aX\xr{Ef}EEY\]
    is a well-defined $A$-graded isomorphism of $A$-graded abelian groups (where here we are omitting $\otimes$, associators, and unitors so the composition fits on the page).
\end{proposition}
\begin{proof}
	Consider the following diagram
	% https://q.uiver.app/#q=WzAsNCxbMCwwLCJbWCxFXFxvdGltZXMgWV1fKiJdLFsyLDAsIlxcbWF0aHJte0hvbX1fe0VfKihFKX1eKihFXyooWCksRV8qKEVcXG90aW1lcyBZKSkiXSxbMiwyLCJcXG1hdGhybXtIb219X3tFXyooRSl9XiooRV8qKFgpLEVfKkVcXG90aW1lc197XFxwaV8qKEUpfSBFXyooWSkpIl0sWzAsMiwiXFxtYXRocm17SG9tfV97XFxwaV8qKEUpfV4qKEVfKihYKSxFXyooWSkpIl0sWzAsMSwiXFxQc2lfe1gsWX0iXSxbMiwxLCJ7KFxcUGhpX3tFLFl9KX1fKiIsMl0sWzAsMywiXFxjb25nIiwyXSxbMSwzLCJcXHBpXyooXFxtdVxcb3RpbWVzIFkpXFxjaXJjKC0pIiwxXSxbMiwzLCJcXGNvbmciLDJdXQ==
	\begin{equation}\label{diagram_PsiXY}\begin{tikzcd}
		{[X,E\otimes Y]_*} && {\mathrm{Hom}_{E_*(E)}^*(E_*(X),E_*(E\otimes Y))} \\
		\\
		{\mathrm{Hom}_{\pi_*(E)}^*(E_*(X),E_*(Y))} && {\mathrm{Hom}_{E_*(E)}^*(E_*(X),E_*E\otimes_{\pi_*(E)} E_*(Y))}
		\arrow["{\Psi_{X,Y}}", from=1-1, to=1-3]
		\arrow["{{(\Phi_{E,Y})}_*}"', from=3-3, to=1-3]
		\arrow["\cong"', from=1-1, to=3-1]
		\arrow["{\pi_*(\mu\otimes Y)\circ(-)}"{description}, from=1-3, to=3-1]
		\arrow["\cong"', from=3-3, to=3-1]
	\end{tikzcd}\end{equation}
	Here the left vertical isomorphism is that from \autoref{UCT_for_graded_projective}, and the bottom horizontal isomorphism is the forgetful-cofree adjunction (\autoref{comodule_co-free_adjunction}) for $A$-graded left comodules over the dual $E$-Steenrod algebra. The right vertical arrow is a well-defined isomorphism, as $\Phi_{E,Y}$ is a homomorphism of $A$-graded left $E_*(E)$-comodules (\autoref{Phi_E,X_is_comodule_homo_main}), and in fact it is an isomorphism by \autoref{Kunneth_iso_for_cellular_objects}, since $E$ is flat and cellular, and $Y$ is cellular or $E_*(Y)$ is flat. Thus in order to see $\Psi_{X,Y}$ is an isomorphism, it suffices to show that the diagram commutes. Showing the top left triangle commutes is entirely straightforward, and simply amounts to unravelling definitions. Now we claim the right triangle commutes. First, recall that by how the how forgetful-cofree adjunction for left comodules over a Hopf algebroid is defined, that the bottom vertical arrow sends an $A$-graded homomorphism of left $E_*(E)$-comodules $\psi:E_{*-a}(X)\to E_*(E)\otimes_{\pi_*(E)}E_*(Y)$ to the composition
	\[E_{*-a}(X)\xr\psi E_*(E)\otimes_{\pi_*(E)}E_*(Y)\xr{\pi_*(\mu)\otimes E_*(Y)}\pi_*(E)\otimes_{\pi_*(E)}E_*(Y)\xr\cong E_*(Y).\]
	Thus, in order to show the bottom right triangle in diagram (\ref{diagram_PsiXY}) commutes, it suffices to show the following diagram commutes:
	% https://q.uiver.app/#q=WzAsNCxbMCwwLCJFXyooRSlcXG90aW1lc197XFxwaV8qKEUpfUVfKihZKSJdLFsyLDAsIlxccGlfKihFKVxcb3RpbWVzX3tcXHBpXyooRSl9RV8qKFkpIl0sWzIsMiwiRV8qKFkpIl0sWzAsMiwiRV8qKEVcXG90aW1lcyBZKSJdLFswLDEsIlxccGlfKihcXG11KVxcb3RpbWVzIEVfKihZKSJdLFsxLDIsIlxcY29uZyJdLFswLDMsIlxcUGhpX3tFLFl9IiwyXSxbMywyLCJcXHBpXyooXFxtdVxcb3RpbWVzIFkpIl1d
	\[\begin{tikzcd}
		{E_*(E)\otimes_{\pi_*(E)}E_*(Y)} && {\pi_*(E)\otimes_{\pi_*(E)}E_*(Y)} \\
		\\
		{E_*(E\otimes Y)} && {E_*(Y)}
		\arrow["{\pi_*(\mu)\otimes E_*(Y)}", from=1-1, to=1-3]
		\arrow["\cong", from=1-3, to=3-3]
		\arrow["{\Phi_{E,Y}}"', from=1-1, to=3-1]
		\arrow["{\pi_*(\mu\otimes Y)}", from=3-1, to=3-3]
	\end{tikzcd}\]
	Since all the arrows here are homomorphisms of abelian groups, in order to show the diagram commutes, it suffices to chase pure homogeneous tensors around. To that end, let $x:S^a\to E\otimes E$ and $y:S^b\to E\otimes Y$, and consider the following diagram exhibiting the two ways to chase $x\otimes y$ around:
	% https://q.uiver.app/#q=WzAsNixbMCwwLCJTXnthK2J9Il0sWzEsMCwiU15hXFxvdGltZXMgU15iIl0sWzIsMCwiRVxcb3RpbWVzIEVcXG90aW1lcyBFXFxvdGltZXMgWSJdLFszLDAsIkVcXG90aW1lcyBFXFxvdGltZXMgWSJdLFszLDEsIkVcXG90aW1lcyBZIl0sWzIsMSwiRVxcb3RpbWVzIEVcXG90aW1lcyBZIl0sWzAsMSwiXFxwaGlfe2EsYn0iXSxbMSwyLCJ4XFxvdGltZXMgeSJdLFsyLDMsIlxcbXVcXG90aW1lcyBFXFxvdGltZXMgWSJdLFszLDQsIlxcbXVcXG90aW1lcyBZIl0sWzIsNSwiRVxcb3RpbWVzIFxcbXVcXG90aW1lcyBZIiwyXSxbNSw0LCJcXG11XFxvdGltZXMgWSJdXQ==
	\[\begin{tikzcd}
		{S^{a+b}} & {S^a\otimes S^b} & {E\otimes E\otimes E\otimes Y} & {E\otimes E\otimes Y} \\
		&& {E\otimes E\otimes Y} & {E\otimes Y}
		\arrow["{\phi_{a,b}}", from=1-1, to=1-2]
		\arrow["{x\otimes y}", from=1-2, to=1-3]
		\arrow["{\mu\otimes E\otimes Y}", from=1-3, to=1-4]
		\arrow["{\mu\otimes Y}", from=1-4, to=2-4]
		\arrow["{E\otimes \mu\otimes Y}"', from=1-3, to=2-3]
		\arrow["{\mu\otimes Y}", from=2-3, to=2-4]
	\end{tikzcd}\]
	The diagram commutes by associtiavity of $\mu$. Thus, diagram (\ref{diagram_PsiXY}) commutes, so that $\Psi_{X,Y}$ is indeed an isomorphism, as desired.
\end{proof}

%In the following definition, let $\vare:E_*(E)\to \pi_*(E)$ be the map which sends some $\alpha:S^a\to E\otimes E$ to the composition
%\[S^a\xr\alpha E\otimes E\xr\mu E.\]
%Also define $\Psi:E_*(E)\to E_*(E)\otimes_{\pi_*(E)}E_*(E)$ to be the map which factors as
%\[E_*(E)\to E_*(E\otimes E)\xr\cong E_*(E)\otimes_{\pi_*(E)}E_*(E)\]
%where the second arrow is the isomorphism prescribed by \autoref{Kunneth_map}, and the first arrow sends a class $\alpha:S^a\to E\otimes E$ to the composition
%\[S^a\xr\alpha E\otimes E\cong E\otimes S\otimes E\xr{E\otimes e\otimes E}E\otimes E\otimes E.\]
%
%\begin{lemma}[{\cite[Proposition 2.30, 2.33]{nlab:introduction_to_the_adams_spectral_sequence}}]\label{2.30_2.33}
	%Let $E$ be a flat commutative ring spectrum, and let $X$ and $Y$ be spectra such that $E_\aast(X)$ is a projective module over $\pi_\aast(E)$. Then for all $s\geq0$ and $t,w\in\bZ$, there is an isomorphism
	%\[\Phi:[X,E\wedge Y]_{t,w}\to\Hom_{E_\aast(E)}^{t,w}(E_\aast(X),E_\aast(E\wedge Y)),\]
	%obtained by sending a class $f:S^{t,w}\wedge X\to E\wedge Y$ in $[X,E\wedge Y]_{t,w}$ to the map
	%\[\Phi_f:E_\acast(X)\to E_{\ast+t,\ast+w}(X\wedge Y)\]
	%sending
	%\[[S^{a,b}\xr gE\wedge X]\mapsto[S^{a+t,b+w}\cong S^{a,b}\wedge S^{t,w}\xr{g\wedge S^{t,w}}E\wedge X\wedge S^{t,w}\cong E\wedge S^{t,w}\wedge X\xr{E\wedge f}E\wedge E\wedge Y].\]
%\end{lemma}
%\begin{proof}
	%Let $f:S^{t,w}\wedge X\to E\wedge Y$. First we want to show that $\Phi_f$ is actually an $E_\aast(E)$-comodule homomorphism.\todo{finish}
%\end{proof}

\end{document}
