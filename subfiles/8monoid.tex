\documentclass[../main.tex]{subfiles}
\tikzcdset{scale cd/.style={every label/.append style={scale=#1},cells={nodes={scale=#1}}}}

\begin{document}

In what follows, we fix a symmetric monoidal category $(\cC,\otimes,S)$ with left unitor, right unitor, associator, and symmetry isomorphisms $\lambda$, $\rho$, $\alpha$, and $\tau$, respectively.

\begin{definition}\label{left_module_object}
	Let $(E,\mu,e)$ be a monoid object in $\cC$. Then a \emph{left module object} $(N,\kappa)$ over $(E,\mu,e)$ is the data of an object $N$ in $\cC$ and a morphism $\kappa:E\otimes N\to N$ such that the following two diagrams commute in $\cC$:
	% https://q.uiver.app/#q=WzAsOCxbMCwwLCJTXFxvdGltZXMgTiJdLFsxLDAsIkVcXG90aW1lcyBOIl0sWzEsMSwiTiJdLFsyLDAsIihFXFxvdGltZXMgRSlcXG90aW1lcyBOIl0sWzQsMCwiRVxcb3RpbWVzIE4iXSxbNCwxLCJOIl0sWzIsMSwiRVxcb3RpbWVzKEVcXG90aW1lcyBOKSJdLFszLDEsIkVcXG90aW1lcyBOIl0sWzAsMSwiZVxcb3RpbWVzIE4iXSxbMSwyLCJcXGthcHBhIl0sWzAsMiwiXFxsYW1iZGFfTiIsMl0sWzMsNCwiXFxtdVxcb3RpbWVzIE4iXSxbNCw1LCJcXGthcHBhIl0sWzMsNiwiXFxhbHBoYSIsMl0sWzYsNywiRVxcb3RpbWVzIFxca2FwcGEiXSxbNyw1LCJcXGthcHBhIl1d
	\[\begin{tikzcd}
		{S\otimes N} & {E\otimes N} & {(E\otimes E)\otimes N} && {E\otimes N} \\
		& N & {E\otimes(E\otimes N)} & {E\otimes N} & N
		\arrow["{e\otimes N}", from=1-1, to=1-2]
		\arrow["\kappa", from=1-2, to=2-2]
		\arrow["{\lambda_N}"', from=1-1, to=2-2]
		\arrow["{\mu\otimes N}", from=1-3, to=1-5]
		\arrow["\kappa", from=1-5, to=2-5]
		\arrow["\alpha"', from=1-3, to=2-3]
		\arrow["{E\otimes \kappa}", from=2-3, to=2-4]
		\arrow["\kappa", from=2-4, to=2-5]
	\end{tikzcd}\]
\end{definition}

\begin{definition}\label{homomorphism_of_left_module_objects}
	Let $(E,\mu,e)$ be a monoid object in $\cC$, and suppose we have two left module objects $(N,\kappa)$ and $(N',\kappa')$ over $(E,\mu,e)$. Then a morphism $f:N\to N'$ is a \emph{left $E$-module homomorphism} if the following diagram commutes in $\cC$:
	% https://q.uiver.app/#q=WzAsNCxbMCwwLCJFXFxvdGltZXMgTiJdLFsxLDAsIkVcXG90aW1lcyBOJyJdLFswLDEsIk4iXSxbMSwxLCJOJyJdLFswLDEsIkVcXG90aW1lcyBmIl0sWzAsMiwiXFxrYXBwYSIsMl0sWzIsMywiZiJdLFsxLDMsIlxca2FwcGEnIl1d
	\[\begin{tikzcd}
		{E\otimes N} & {E\otimes N'} \\
		N & {N'}
		\arrow["{E\otimes f}", from=1-1, to=1-2]
		\arrow["\kappa"', from=1-1, to=2-1]
		\arrow["f", from=2-1, to=2-2]
		\arrow["{\kappa'}", from=1-2, to=2-2]
	\end{tikzcd}\]
\end{definition}

\begin{definition}
	Given a monoid object $(E,\mu,e)$ in $\cSH$, we write $E\text-\Mod$ to denote the category of left module objects over $E$ and left $E$-module homomorphisms between them. We denote the homset in $E\text-\Mod$ by
	\[\Hom_{E\text-\Mod}(M,N),\qquad\text{or simply}\qquad\Hom_E(M,N).\]
\end{definition}

\begin{lemma}\label{module_if_iso_to_module}
	Let $(E,\mu,e)$ be a monoid object in $\cC$ and let $(N,\kappa)$ be a left $E$ module object. Then given some object $X$ in $\cC$ and an isomorphism $\phi:N\xr\cong X$, $X$ inherits the structure of a left $E$-module via the action map
	\[\kappa_\phi:E\otimes X\xr{E\otimes\phi^{-1}}E\otimes N\xr\kappa N\xr\phi X.\]
\end{lemma}
\begin{proof}
	We need to show the two coherence diagrams in \autoref{left_module_object} commute. To see the former commutes, consider the following diagram:
	% https://q.uiver.app/#q=WzAsNixbMCwwLCJYIl0sWzMsMywiWCJdLFszLDAsIkVcXG90aW1lcyBYIl0sWzMsMSwiRVxcb3RpbWVzIE4iXSxbMywyLCJOIl0sWzIsMSwiTiJdLFswLDEsIiIsMCx7ImxldmVsIjoyLCJzdHlsZSI6eyJoZWFkIjp7Im5hbWUiOiJub25lIn19fV0sWzAsMiwiZVxcb3RpbWVzIFgiXSxbMiwzLCJFXFxvdGltZXNcXHBoaV57LTF9Il0sWzMsNCwiXFxrYXBwYSJdLFs0LDEsIlxccGhpIl0sWzUsNCwiIiwwLHsibGV2ZWwiOjIsInN0eWxlIjp7ImhlYWQiOnsibmFtZSI6Im5vbmUifX19XSxbNSwzLCJlXFxvdGltZXMgTiJdLFswLDUsIlxccGhpXnstMX0iXV0=
	\[\begin{tikzcd}
		X &&& {E\otimes X} \\
		&& N & {E\otimes N} \\
		&&& N \\
		&&& X
		\arrow[Rightarrow, no head, from=1-1, to=4-4]
		\arrow["{e\otimes X}", from=1-1, to=1-4]
		\arrow["{E\otimes\phi^{-1}}", from=1-4, to=2-4]
		\arrow["\kappa", from=2-4, to=3-4]
		\arrow["\phi", from=3-4, to=4-4]
		\arrow[Rightarrow, no head, from=2-3, to=3-4]
		\arrow["{e\otimes N}", from=2-3, to=2-4]
		\arrow["{\phi^{-1}}", from=1-1, to=2-3]
	\end{tikzcd}\]
	The top trapezoid commutes by functoriality of $-\otimes-$. The middle small triangle commutes by unitality of $\kappa$. The remaining region commutes by definition. To see the second coherence diagram commutes, consider the following diagram:
	% https://q.uiver.app/#q=WzAsMTAsWzAsMCwiRVxcb3RpbWVzIEVcXG90aW1lcyBYIl0sWzMsMCwiRVxcb3RpbWVzIFgiXSxbMywxLCJFXFxvdGltZXMgTiJdLFszLDIsIk4iXSxbMywzLCJYIl0sWzAsMSwiRVxcb3RpbWVzIEVcXG90aW1lcyBOIl0sWzAsMiwiRVxcb3RpbWVzIE4iXSxbMCwzLCJFXFxvdGltZXMgWCJdLFsxLDMsIkVcXG90aW1lcyBOIl0sWzIsMywiTiJdLFswLDEsIlxcbXVcXG90aW1lcyBYIl0sWzEsMiwiRVxcb3RpbWVzXFxwaGleey0xfSJdLFsyLDMsIlxca2FwcGEiXSxbMyw0LCJcXHBoaSJdLFswLDUsIkVcXG90aW1lcyBFXFxvdGltZXNcXHBoaV57LTF9IiwyXSxbNSw2LCJFXFxvdGltZXNcXGthcHBhIiwyXSxbNiw3LCJFXFxvdGltZXNcXHBoaSIsMl0sWzcsOCwiRVxcb3RpbWVzXFxwaGleezEtfSIsMl0sWzgsOSwiXFxrYXBwYSIsMl0sWzksNCwiXFxwaGkiLDJdLFs1LDIsIlxcbXVcXG90aW1lcyBOIl0sWzYsMywiXFxrYXBwYSJdLFs2LDgsIiIsMCx7ImxldmVsIjoyLCJzdHlsZSI6eyJoZWFkIjp7Im5hbWUiOiJub25lIn19fV1d
	\[\begin{tikzcd}
		{E\otimes E\otimes X} &&& {E\otimes X} \\
		{E\otimes E\otimes N} &&& {E\otimes N} \\
		{E\otimes N} &&& N \\
		{E\otimes X} & {E\otimes N} & N & X
		\arrow["{\mu\otimes X}", from=1-1, to=1-4]
		\arrow["{E\otimes\phi^{-1}}", from=1-4, to=2-4]
		\arrow["\kappa", from=2-4, to=3-4]
		\arrow["\phi", from=3-4, to=4-4]
		\arrow["{E\otimes E\otimes\phi^{-1}}"', from=1-1, to=2-1]
		\arrow["E\otimes\kappa"', from=2-1, to=3-1]
		\arrow["E\otimes\phi"', from=3-1, to=4-1]
		\arrow["{E\otimes\phi^{1-}}"', from=4-1, to=4-2]
		\arrow["\kappa"', from=4-2, to=4-3]
		\arrow["\phi"', from=4-3, to=4-4]
		\arrow["{\mu\otimes N}", from=2-1, to=2-4]
		\arrow["\kappa", from=3-1, to=3-4]
		\arrow[Rightarrow, no head, from=3-1, to=4-2]
	\end{tikzcd}\]
	The top rectangle commutes by functoriality of $-\otimes-$. The middle rectangle commutes by coherence for $\kappa$. The bottom two regions commute by definition.
\end{proof}

\begin{proposition}\label{free_forgetful_E-Mod}
	Given a monoid object $(E,\mu,e)$ in $\cC$, the forgetful functor $E\text-\Mod\to\cC$ has a left adjoint $\cC\to E\text-\Mod$ sending an object $X\mapsto (E\otimes X,\kappa_X)$ where $\kappa_X$ is the composition
	\[E\otimes(E\otimes X)\xr{\alpha^{-1}}(E\otimes E)\otimes X\xr{\mu\otimes X}E\otimes X,\]
	and sending a morphism $f:X\to Y$ to $E\otimes f:E\otimes X\to E\otimes Y$.
\end{proposition}
\begin{proof}
	In this proof, we work in a symmetric strict monoidal category. First, we wish to show that $E\otimes-:\cC\to E\text-\Mod$ as constructed is well-defined. First, to see that $(X,\kappa_X)$ is actually a left $E$-module, we need to show the two diagrams in \autoref{left_module_object} commute. Indeed, consider the following diagrams:
	% https://q.uiver.app/#q=WzAsNyxbMCwwLCJFXFxvdGltZXMgWCJdLFsxLDAsIkVcXG90aW1lcyBFXFxvdGltZXMgWCJdLFsxLDEsIkVcXG90aW1lcyBYIl0sWzIsMCwiRVxcb3RpbWVzIEVcXG90aW1lcyBFXFxvdGltZXMgWCJdLFszLDAsIkVcXG90aW1lcyBFXFxvdGltZXMgWCJdLFszLDEsIkVcXG90aW1lcyBYIl0sWzIsMSwiRVxcb3RpbWVzIEVcXG90aW1lcyBYIl0sWzAsMSwiZVxcb3RpbWVzIEVcXG90aW1lcyBYIl0sWzEsMiwiXFxtdVxcb3RpbWVzIFgiXSxbMCwyLCIiLDIseyJsZXZlbCI6Miwic3R5bGUiOnsiaGVhZCI6eyJuYW1lIjoibm9uZSJ9fX1dLFszLDQsIlxcbXVcXG90aW1lcyBFXFxvdGltZXMgWCJdLFs0LDUsIlxcbXVcXG90aW1lcyBYIl0sWzMsNiwiRVxcb3RpbWVzIFxcbXVcXG90aW1lcyBYIiwyXSxbNiw1LCJcXG11XFxvdGltZXMgWCIsMl1d
	\[\begin{tikzcd}
		{E\otimes X} & {E\otimes E\otimes X} & {E\otimes E\otimes E\otimes X} & {E\otimes E\otimes X} \\
		& {E\otimes X} & {E\otimes E\otimes X} & {E\otimes X}
		\arrow["{e\otimes E\otimes X}", from=1-1, to=1-2]
		\arrow["{\mu\otimes X}", from=1-2, to=2-2]
		\arrow[Rightarrow, no head, from=1-1, to=2-2]
		\arrow["{\mu\otimes E\otimes X}", from=1-3, to=1-4]
		\arrow["{\mu\otimes X}", from=1-4, to=2-4]
		\arrow["{E\otimes \mu\otimes X}"', from=1-3, to=2-3]
		\arrow["{\mu\otimes X}"', from=2-3, to=2-4]
	\end{tikzcd}\]
	These are precisely the diagrams obtained by applying $X\otimes-$ to the coherence diagrams for $\mu$, so that they commute as desired. Now, suppose $f:X\to Y$ is a morphism in $\cC$, then we would like to show that $E\otimes f:E\otimes X\to E\otimes Y$ is a morphism of left $E$-module objects. Indeed, consider the following diagram:
	% https://q.uiver.app/#q=WzAsNCxbMCwwLCJFXFxvdGltZXMgRVxcb3RpbWVzIFgiXSxbMSwwLCJFXFxvdGltZXMgRVxcb3RpbWVzIFkiXSxbMSwxLCJFXFxvdGltZXMgWSJdLFswLDEsIkVcXG90aW1lcyBYIl0sWzAsMSwiRVxcb3RpbWVzIEVcXG90aW1lcyBmIl0sWzEsMiwiXFxtdVxcb3RpbWVzIFkiXSxbMCwzLCJcXG11XFxvdGltZXMgWCIsMl0sWzMsMiwiRVxcb3RpbWVzIGYiXV0=
	\[\begin{tikzcd}
		{E\otimes E\otimes X} & {E\otimes E\otimes Y} \\
		{E\otimes X} & {E\otimes Y}
		\arrow["{E\otimes E\otimes f}", from=1-1, to=1-2]
		\arrow["{\mu\otimes Y}", from=1-2, to=2-2]
		\arrow["{\mu\otimes X}"', from=1-1, to=2-1]
		\arrow["{E\otimes f}", from=2-1, to=2-2]
	\end{tikzcd}\]
	It commutes by functoriality of $-\otimes-$, so $E\otimes f$ is indeed a left $E$-module homomorphism as desired.

	Now, in order to see that $E\otimes-$ is left adjoint to the forgetful functor, it suffices to construct a unit and counit for the adjunction and show they satisfy the zig-zag identities. Given $X$ in $\cC$ and $(N,\kappa)$ in $E\text-\Mod$, define $\eta_X:=e\otimes X:X\to E\otimes X$ and $\vare_{(N,\kappa)}:=\kappa:E\otimes N\to N$. $\eta_X$ is clearly natural in $X$ by functoriality of $-\otimes-$, and $\vare_{(N,\kappa)}$ is natural in $(N,\kappa)$ by how morphisms in $E\text-\Mod$ are defined. Now, to see these are actually the unit and counit of an adjunction, we need to show that the following diagrams commute for all $X$ in $\cC$ and $(N,\kappa)$ in $E\text-\Mod$:
	% https://q.uiver.app/#q=WzAsNixbMCwwLCJFXFxvdGltZXMgWCJdLFsyLDAsIkVcXG90aW1lcyBFXFxvdGltZXMgWCJdLFsyLDIsIkVcXG90aW1lcyBYIl0sWzUsMCwiRVxcb3RpbWVzIE4iXSxbNSwyLCJOIl0sWzcsMCwiTiJdLFswLDEsIkVcXG90aW1lc1xcZXRhX1g9RVxcb3RpbWVzIGVcXG90aW1lcyBYIl0sWzEsMiwiXFx2YXJlX3soRVxcb3RpbWVzIFgsXFxrYXBwYV9YKX09XFxtdVxcb3RpbWVzIFgiXSxbMCwyLCIiLDIseyJsZXZlbCI6Miwic3R5bGUiOnsiaGVhZCI6eyJuYW1lIjoibm9uZSJ9fX1dLFszLDQsIlxcdmFyZV97KE4sXFxrYXBwYSl9PVxca2FwcGEiLDJdLFs1LDMsIlxcZXRhX049ZVxcb3RpbWVzIE4iLDJdLFs1LDQsIiIsMCx7ImxldmVsIjoyLCJzdHlsZSI6eyJoZWFkIjp7Im5hbWUiOiJub25lIn19fV1d
	\[\begin{tikzcd}
		{E\otimes X} && {E\otimes E\otimes X} &&& {E\otimes N} && N \\
		\\
		&& {E\otimes X} &&& N
		\arrow["{E\otimes\eta_X=E\otimes e\otimes X}", from=1-1, to=1-3]
		\arrow["{\vare_{(E\otimes X,\kappa_X)}=\mu\otimes X}", from=1-3, to=3-3]
		\arrow[Rightarrow, no head, from=1-1, to=3-3]
		\arrow["{\vare_{(N,\kappa)}=\kappa}"', from=1-6, to=3-6]
		\arrow["{\eta_N=e\otimes N}"', from=1-8, to=1-6]
		\arrow[Rightarrow, no head, from=1-8, to=3-6]
	\end{tikzcd}\]
	Commutativity of the left diagram is unitality of $\mu$, while commutativity of the right diagram is unitality of $\kappa$. Thus indeed $E\otimes-:\cC\to E\text-\Mod$ is a left adjoint of the forgetful functor $E\text-\Mod\to\cC$, as desired.
\end{proof}

\begin{definition}\label{free_module}
	We call the functor $E\otimes-:\cC\to E\text-\Mod$ constructed above the \emph{free} functor, and we call left $E$-modules in the image of the free functor \emph{free modules}.
\end{definition}

%\begin{proposition}\label{product_of_monoids_is_a_monoid}
%	Let $(E_1,\mu_1,e_1)$ and $(E_2,\mu_2,e_2)$ be monoid objects in a symmetric monoidal category $(\cC,\otimes,S)$. Then $E_1\otimes E_2$ is canonically a ring spectrum via the maps
%	\[\mu:E_1\otimes E_2\otimes E_1\otimes E_2\xr{E_1\otimes\tau\otimes E_2}E_1\otimes E_1\otimes E_2\otimes E_2\xr{\mu_1\otimes\mu_2}E_1\otimes E_2\]
%	and
%	\[e:S\cong S\otimes S\xr{e_1\otimes e_2}E_1\otimes E_2.\]
%\end{proposition}
%\begin{proof}
%	\todo{todo}
%\end{proof}

From now on we fix a
monoidal closed tensor triangulated category $(\cSH,\otimes,S,\Sigma,e,\cD)$ (\autoref{tentri}) with arbitrary (small) (co)products and sub-Picard grading $(A,\1,h,\{S^a\},\{\phi_{a,b}\})$ (\autoref{sub_Picard_grading_defn}), and we adopt the conventions outlined in \Cref{setup}. In all proofs that follow we will freely use the coherence theorem for symmetric monoidal categories. In particular, we will assume without loss of generality that the associators and unitors in $\cSH$ are identities.

\begin{lemma}\label{suspension_of_module_object_is_module_object}
	Let $(E,\mu,e)$ be a monoid object in $\cSH$, and suppose $(N,\kappa)$ is a left module object over $E$. Then for all $a\in A$, $\Sigma^a N$ is canonically a left $E$-module object, with action map given by
	\[\kappa^a:E\otimes S^a\otimes N\]
	\[\kappa^a:E\otimes\Sigma^aN=E\otimes S^a\otimes N\xr{\tau\otimes N}S^a\otimes E\otimes N\xr{S^a\otimes\kappa}S^a\otimes N=\Sigma^aN.\]
\end{lemma}
\begin{proof}
	In this proof, we are assuming that unitality and associativity hold up to strict equality, by the coherence theorem for monoidal categories. In order to show $(\Sigma^a N,\kappa^a)$ is a left module object over $E$, we need to show $\kappa^a$ makes the two coherence diagrams in \autoref{left_module_object} commute. First, to see the first diagram commutes, consider the following diagram:
	% https://q.uiver.app/#q=WzAsNCxbMCwwLCJTXmFcXG90aW1lcyBOIl0sWzIsMCwiRVxcb3RpbWVzIFNeYVxcb3RpbWVzIE4iXSxbMiwyLCJTXmFcXG90aW1lcyBOIl0sWzIsMSwiU15hXFxvdGltZXMgRVxcb3RpbWVzIE4iXSxbMCwxLCJlXFxvdGltZXMgU15hXFxvdGltZXMgTiJdLFsxLDMsIlxcdGF1XFxvdGltZXMgTiJdLFszLDIsIlNeYVxcb3RpbWVzIFxca2FwcGEiXSxbMCwyLCIiLDIseyJsZXZlbCI6Miwic3R5bGUiOnsiaGVhZCI6eyJuYW1lIjoibm9uZSJ9fX1dLFswLDMsIlNeYVxcb3RpbWVzIGVcXG90aW1lcyBOIiwxXV0=
	\[\begin{tikzcd}
		{S^a\otimes N} && {E\otimes S^a\otimes N} \\
		&& {S^a\otimes E\otimes N} \\
		&& {S^a\otimes N}
		\arrow["{e\otimes S^a\otimes N}", from=1-1, to=1-3]
		\arrow["{\tau\otimes N}", from=1-3, to=2-3]
		\arrow["{S^a\otimes \kappa}", from=2-3, to=3-3]
		\arrow[Rightarrow, no head, from=1-1, to=3-3]
		\arrow["{S^a\otimes e\otimes N}"{description}, from=1-1, to=2-3]
	\end{tikzcd}\]
	The top inner triangle commutes by coherence for a symmetric monoidal category, and the bottom inner triangle commutes by the coherence condition for $\kappa$. To see the other module condition for $\wt\kappa$, consider the following diagram:
	% https://q.uiver.app/#q=WzAsOCxbMCwwLCJFXFxvdGltZXMgRVxcb3RpbWVzIFNeYVxcb3RpbWVzIE4iXSxbMiwwLCJFXFxvdGltZXMgU15hXFxvdGltZXMgTiJdLFsyLDEsIlNeYVxcb3RpbWVzIEVcXG90aW1lcyBOIl0sWzIsMiwiU15hXFxvdGltZXMgTiJdLFswLDEsIkVcXG90aW1lcyBTXmFcXG90aW1lcyBFXFxvdGltZXMgTiJdLFswLDIsIkVcXG90aW1lcyBTXmFcXG90aW1lcyBOIl0sWzEsMiwiU15hXFxvdGltZXMgRVxcb3RpbWVzIE4iXSxbMSwxLCJTXmFcXG90aW1lcyBFXFxvdGltZXMgRVxcb3RpbWVzIE4iXSxbMCwxLCJcXG11XFxvdGltZXMgU15hXFxvdGltZXMgTiJdLFsxLDIsIlxcdGF1XFxvdGltZXMgTiJdLFsyLDMsIlNeYVxcb3RpbWVzIFxca2FwcGEiXSxbMCw0LCJFXFxvdGltZXMgXFx0YXVcXG90aW1lcyBOIiwyXSxbNCw1LCJFXFxvdGltZXMgU15hXFxvdGltZXMgXFxrYXBwYSIsMl0sWzUsNiwiXFx0YXVcXG90aW1lcyBOIl0sWzYsMywiU15hXFxvdGltZXMgXFxrYXBwYSJdLFswLDcsIlxcdGF1X3tFXFxvdGltZXMgRSxTXmF9XFxvdGltZXMgTiIsMV0sWzcsMiwiU15hXFxvdGltZXMgXFxtdVxcb3RpbWVzIE4iXSxbNyw2LCJTXmFcXG90aW1lcyBFXFxvdGltZXMgXFxrYXBwYSIsMV0sWzQsNywiXFx0YXVcXG90aW1lcyBFXFxvdGltZXMgTiIsMl1d
	\[\begin{tikzcd}
		{E\otimes E\otimes S^a\otimes N} && {E\otimes S^a\otimes N} \\
		{E\otimes S^a\otimes E\otimes N} & {S^a\otimes E\otimes E\otimes N} & {S^a\otimes E\otimes N} \\
		{E\otimes S^a\otimes N} & {S^a\otimes E\otimes N} & {S^a\otimes N}
		\arrow["{\mu\otimes S^a\otimes N}", from=1-1, to=1-3]
		\arrow["{\tau\otimes N}", from=1-3, to=2-3]
		\arrow["{S^a\otimes \kappa}", from=2-3, to=3-3]
		\arrow["{E\otimes \tau\otimes N}"', from=1-1, to=2-1]
		\arrow["{E\otimes S^a\otimes \kappa}"', from=2-1, to=3-1]
		\arrow["{\tau\otimes N}", from=3-1, to=3-2]
		\arrow["{S^a\otimes \kappa}", from=3-2, to=3-3]
		\arrow["{\tau_{E\otimes E,S^a}\otimes N}"{description}, from=1-1, to=2-2]
		\arrow["{S^a\otimes \mu\otimes N}", from=2-2, to=2-3]
		\arrow["{S^a\otimes E\otimes \kappa}"{description}, from=2-2, to=3-2]
		\arrow["{\tau\otimes E\otimes N}"', from=2-1, to=2-2]
	\end{tikzcd}\]
	The top left triangle commutes by coherence for a symmetric monoidal category. The bottom left rectangle and top right trapezoid commute by naturality of $\tau$. Finally, the bottom right square commutes by the coherence condition for $\kappa$.
\end{proof}

\begin{lemma}\label{free_susp_is_susp_of_free}
	Given a monoid object $(E,\mu,e)$ in $\cSH$, an object $X$ in $\cSH$, and some $a\in A$, the suspension of the free module $\Sigma^a(E\otimes X)$ is naturally isomorphic as a left $E$-module object to the free $E$-module $E\otimes\Sigma^aX$.
\end{lemma}
\begin{proof}
	It suffices to show the isomorphism $S^a\otimes E\otimes X\xr{\tau\otimes X}E\otimes S^a\otimes X$ is a homomorphism of left $E$-module objects. To see this, consider the following diagram:
	% https://q.uiver.app/#q=WzAsNSxbMCwwLCJFXFxvdGltZXMgU15hXFxvdGltZXMgRVxcb3RpbWVzIFgiXSxbMCwxLCJTXmFcXG90aW1lcyBFXFxvdGltZXMgRVxcb3RpbWVzIFgiXSxbMCwyLCJTXmFcXG90aW1lcyBFXFxvdGltZXMgWCJdLFsyLDIsIkVcXG90aW1lcyBTXmFcXG90aW1lcyBYIl0sWzIsMCwiRVxcb3RpbWVzIEVcXG90aW1lcyBTXmFcXG90aW1lcyBYIl0sWzAsMSwiXFx0YXVcXG90aW1lcyBFXFxvdGltZXMgWCIsMl0sWzEsMiwiU15hXFxvdGltZXMgXFxtdVxcb3RpbWVzIFgiLDJdLFsyLDMsIlxcdGF1XFxvdGltZXMgWCJdLFswLDQsIkVcXG90aW1lcyBcXHRhdVxcb3RpbWVzIFgiXSxbNCwzLCJcXG11XFxvdGltZXMgU15hXFxvdGltZXMgWCJdLFsxLDQsIlxcdGF1X3tTXmEsRVxcb3RpbWVzIEV9XFxvdGltZXMgWCIsMl1d
	\[\begin{tikzcd}
		{E\otimes S^a\otimes E\otimes X} && {E\otimes E\otimes S^a\otimes X} \\
		{S^a\otimes E\otimes E\otimes X} \\
		{S^a\otimes E\otimes X} && {E\otimes S^a\otimes X}
		\arrow["{\tau\otimes E\otimes X}"', from=1-1, to=2-1]
		\arrow["{S^a\otimes \mu\otimes X}"', from=2-1, to=3-1]
		\arrow["{\tau\otimes X}", from=3-1, to=3-3]
		\arrow["{E\otimes \tau\otimes X}", from=1-1, to=1-3]
		\arrow["{\mu\otimes S^a\otimes X}", from=1-3, to=3-3]
		\arrow["{\tau_{S^a,E\otimes E}\otimes X}"', from=2-1, to=1-3]
	\end{tikzcd}\]
	The top triangle commutes by coherence for a symmetric monoidal category. The bottom trapezoid commutes by naturality of $\tau$. 
\end{proof}

\begin{definition}
	Let $(E,\mu,e)$ be a monoid object in $\cSH$, and suppose $(N,\kappa)$ and $(N',\kappa')$ are left $E$-module objects in $E\text-\Mod$. Then the hom-sets in $E\text-\Mod$ can be extended to $A$-graded abelian groups $\Hom^*_{E\text-\Mod}(N,N')$, by defining 
	\[\Hom^a_{E\text-\Mod}(N,N'):=\Hom_{E\text-\Mod}(\Sigma^a N,N')\]
	for each $a\in A$ (where $\Sigma^{*}N$ has the left $E$-module structure given by \autoref{suspension_of_module_object_is_module_object}).
\end{definition}

\begin{proposition}\label{coproduct_of_E_modules_is_coproduct_in_E_mod}
	Let $(E,\mu,e)$ be a monoid object in $\cSH$, and suppose we have a family of left $E$-module objects $(N_i,\kappa_i)$ indexed by some small set $I$. Then $N:=\bigoplus_{i\in I}N_i$ is canonically a left $E$-module, with action map given by the composition
	\[\kappa:E\otimes\bigoplus_iN_i\xr\cong\bigoplus_i(E\otimes N_i)\xrightarrow{\bigoplus_i\kappa_i}\bigoplus_iN_i,\]
	where the first isomorphism is given by the fact that $E\otimes-$ preserves coproducts, since it is a left adjoint as $\cSH$ is monoidal closed. Furthermore, $N$ is the coproduct of all the $N_i$'s in $E\text-\Mod$, so that $E\text-\Mod$ has arbitrary coproducts.
\end{proposition}
\begin{proof}
	We need to show the action map $\kappa$ makes the diagrams in \autoref{left_module_object} commute. To see the first (unitality) diagram commutes, consider the following diagram:
	% https://q.uiver.app/#q=WzAsNCxbMCwwLCJcXGJpZ29wbHVzX2lOX2kiXSxbNCwwLCJFXFxvdGltZXNcXGJpZ29wbHVzX2lOX2kiXSxbNCwyLCJcXGJpZ29wbHVzX2koRVxcb3RpbWVzIE5faSkiXSxbNCw0LCJcXGJpZ29wbHVzX2lOX2kiXSxbMCwxLCJlXFxvdGltZXNcXGJpZ29wbHVzX2lOX2kiXSxbMSwyLCJcXGNvbmciXSxbMiwzLCJcXGJpZ29wbHVzX2lcXGthcHBhX2kiXSxbMCwzLCIiLDIseyJsZXZlbCI6Miwic3R5bGUiOnsiaGVhZCI6eyJuYW1lIjoibm9uZSJ9fX1dLFswLDIsIlxcYmlnb3BsdXNfaShlXFxvdGltZXMgTl9pKSIsMV1d
	\[\begin{tikzcd}
		{\bigoplus_iN_i} &&&& {E\otimes\bigoplus_iN_i} \\
		\\
		&&&& {\bigoplus_i(E\otimes N_i)} \\
		\\
		&&&& {\bigoplus_iN_i}
		\arrow["{e\otimes\bigoplus_iN_i}", from=1-1, to=1-5]
		\arrow["\cong", from=1-5, to=3-5]
		\arrow["{\bigoplus_i\kappa_i}", from=3-5, to=5-5]
		\arrow[Rightarrow, no head, from=1-1, to=5-5]
		\arrow["{\bigoplus_i(e\otimes N_i)}"{description}, from=1-1, to=3-5]
	\end{tikzcd}\]
	The top triangle commutes by additivity of $E\otimes-$ The bottom triangle commutes by unitality of each of the $\kappa_i$'s. To see the second coherence diagram commutes, consider the following diagram:
	% https://q.uiver.app/#q=WzAsOCxbMCwwLCJFXFxvdGltZXMgRVxcb3RpbWVzXFxiaWdvcGx1c19pTl9pIl0sWzIsMCwiRVxcb3RpbWVzIFxcYmlnb3BsdXNfaU5faSJdLFswLDEsIkVcXG90aW1lc1xcYmlnb3BsdXNfaShFXFxvdGltZXMgTl9pKSJdLFswLDIsIkVcXG90aW1lc1xcYmlnb3BsdXNfaU5faSJdLFsxLDIsIlxcYmlnb3BsdXNfaShFXFxvdGltZXMgTl9pKSJdLFsyLDIsIlxcYmlnb3BsdXNfaU5faSJdLFsyLDEsIlxcYmlnb3BsdXNfaShFXFxvdGltZXMgTl9pKSJdLFsxLDEsIlxcYmlnb3BsdXNfaShFXFxvdGltZXMgRVxcb3RpbWVzIE5faSkiXSxbMCwxLCJcXG11XFxvcGx1c1xcYmlnb3BsdXNfaU5faSJdLFswLDIsIkVcXG90aW1lc1xcY29uZyIsMl0sWzIsMywiRVxcb3RpbWVzXFxiaWdvcGx1c19pXFxrYXBwYV9pIiwyXSxbMyw0LCJcXGNvbmciLDJdLFs0LDUsIlxcYmlnb3BsdXNfaVxca2FwcGFfaSIsMl0sWzEsNiwiXFxjb25nIl0sWzYsNSwiXFxiaWdvcGx1c19pXFxrYXBwYV9pIl0sWzAsNywiXFxjb25nIl0sWzIsNywiXFxjb25nIl0sWzcsNiwiXFxiaWdvcGx1c19pKFxcbXVcXG90aW1lcyBOX2kpIl0sWzcsNCwiXFxiaWdvcGx1c19pKEVcXG90aW1lc1xca2FwcGFfaSkiLDJdXQ==
	\[\begin{tikzcd}
		{E\otimes E\otimes\bigoplus_iN_i} && {E\otimes \bigoplus_iN_i} \\
		{E\otimes\bigoplus_i(E\otimes N_i)} & {\bigoplus_i(E\otimes E\otimes N_i)} & {\bigoplus_i(E\otimes N_i)} \\
		{E\otimes\bigoplus_iN_i} & {\bigoplus_i(E\otimes N_i)} & {\bigoplus_iN_i}
		\arrow["{\mu\oplus\bigoplus_iN_i}", from=1-1, to=1-3]
		\arrow["E\otimes\cong"', from=1-1, to=2-1]
		\arrow["{E\otimes\bigoplus_i\kappa_i}"', from=2-1, to=3-1]
		\arrow["\cong"', from=3-1, to=3-2]
		\arrow["{\bigoplus_i\kappa_i}"', from=3-2, to=3-3]
		\arrow["\cong", from=1-3, to=2-3]
		\arrow["{\bigoplus_i\kappa_i}", from=2-3, to=3-3]
		\arrow["\cong", from=1-1, to=2-2]
		\arrow["\cong", from=2-1, to=2-2]
		\arrow["{\bigoplus_i(\mu\otimes N_i)}", from=2-2, to=2-3]
		\arrow["{\bigoplus_i(E\otimes\kappa_i)}"', from=2-2, to=3-2]
	\end{tikzcd}\]
	The bottom right square commutes by coherence for the $\kappa_i$'s. Every other region commutes by additivity of $-\otimes-$ in each variable. Thus $N=\bigoplus_iN_i$ is indeed a left $E$-module object, as desired.

	Now, we claim that $(N,\kappa)$ is the coproduct of the $(N_i,\kappa_i)$'s in $E\text-\Mod$. First, we need to show that the canonical maps $\iota_i:N_i\into N$ are morphisms in $E\text-\Mod$ for all $i\in I$. To see $\iota_i$ is a homomorphism of left $E$-module objects, consider the following diagram:
	% https://q.uiver.app/#q=WzAsNSxbMCwwLCJFXFxvdGltZXMgTl9pIl0sWzIsMCwiRVxcb3RpbWVzXFxiaWdvcGx1c19pTl9pIl0sWzAsMiwiTl9pIl0sWzIsMiwiXFxiaWdvcGx1c19pTl9pIl0sWzIsMSwiXFxiaWdvcGx1c19pKEVcXG90aW1lcyBOX2kpIl0sWzAsMSwiRVxcb3RpbWVzXFxpb3RhX2kiXSxbMCwyLCJcXGthcHBhX2kiLDJdLFsyLDMsIlxcaW90YV9pIiwyLHsic3R5bGUiOnsidGFpbCI6eyJuYW1lIjoiaG9vayIsInNpZGUiOiJ0b3AifX19XSxbMSw0LCJcXGNvbmciXSxbNCwzLCJcXGJpZ29wbHVzX2lcXGthcHBhX2kiXSxbMCw0LCJcXGlvdGFfe0VcXG90aW1lcyBOX2l9IiwxLHsic3R5bGUiOnsidGFpbCI6eyJuYW1lIjoiaG9vayIsInNpZGUiOiJ0b3AifX19XV0=
	\[\begin{tikzcd}
		{E\otimes N_i} && {E\otimes\bigoplus_iN_i} \\
		&& {\bigoplus_i(E\otimes N_i)} \\
		{N_i} && {\bigoplus_iN_i}
		\arrow["{E\otimes\iota_i}", from=1-1, to=1-3]
		\arrow["{\kappa_i}"', from=1-1, to=3-1]
		\arrow["{\iota_i}"', hook, from=3-1, to=3-3]
		\arrow["\cong", from=1-3, to=2-3]
		\arrow["{\bigoplus_i\kappa_i}", from=2-3, to=3-3]
		\arrow["{\iota_{E\otimes N_i}}"{description}, hook, from=1-1, to=2-3]
	\end{tikzcd}\]
	The top triangle commutes by additivity of $E\otimes-$. The bottom trapezoid commutes since, by univeral property of the coproduct, $\bigoplus_i\kappa_i$ is the unique arrow which makes the trapezoid commute for all $i\in I$. Now, it remains to show that given a left $E$-module object $(N',\kappa')$ and homomorphisms $f_i:N_i\to N'$ of left $E$-module objects for all $i\in I$, that the unique arrow $f:N\to N'$ in $\cSH$ satisfying $f\circ\iota_i=f_i$ for all $i\in I$ is a homomorphism of left $E$-module objects, so that $N$ is actually the coproduct of the $N_i$'s. To see this, first let $h:\bigoplus_i(E\otimes N_i)\to E\otimes N'$ be the arrow determined by the maps $E\otimes N_i\xrightarrow{E\otimes f_i}E\otimes N'$. Then consider the following diagram:
	% https://q.uiver.app/#q=WzAsNyxbMCwwLCJFXFxvdGltZXNcXGJpZ29wbHVzX2lOX2kiXSxbMiwwLCJFXFxvdGltZXMgTiciXSxbMCwxLCJcXGJpZ29wbHVzX2koRVxcb3RpbWVzIE5faSkiXSxbMCwzLCJcXGJpZ29wbHVzX2lOX2kiXSxbMiwzLCJOJyJdLFsxLDEsIlxcYmlnb3BsdXNfaShFXFxvdGltZXMgTicpIl0sWzEsMiwiXFxiaWdvcGx1c19pTiciXSxbMCwxLCJFXFxvdGltZXMgZiJdLFswLDIsIlxcY29uZyIsMl0sWzIsMywiXFxiaWdvcGx1c19pXFxrYXBwYV9pIiwyXSxbMyw0LCJmIiwyXSxbMSw0LCJcXGthcHBhJyJdLFsyLDUsIlxcYmlnb3BsdXNfaShFXFxvdGltZXMgZl9pKSIsMl0sWzIsMSwiaCJdLFs1LDEsIlxcbmFibGEiLDJdLFs1LDYsIlxcYmlnb3BsdXNfaVxca2FwcGEnIl0sWzYsNCwiXFxuYWJsYSJdLFszLDYsIlxcYmlnb3BsdXNfaWZfaSIsMV1d
	\[\begin{tikzcd}
		{E\otimes\bigoplus_iN_i} && {E\otimes N'} \\
		{\bigoplus_i(E\otimes N_i)} & {\bigoplus_i(E\otimes N')} \\
		& {\bigoplus_iN'} \\
		{\bigoplus_iN_i} && {N'}
		\arrow["{E\otimes f}", from=1-1, to=1-3]
		\arrow["\cong"', from=1-1, to=2-1]
		\arrow["{\bigoplus_i\kappa_i}"', from=2-1, to=4-1]
		\arrow["f"', from=4-1, to=4-3]
		\arrow["{\kappa'}", from=1-3, to=4-3]
		\arrow["{\bigoplus_i(E\otimes f_i)}"', from=2-1, to=2-2]
		\arrow["h", from=2-1, to=1-3]
		\arrow["\nabla"', from=2-2, to=1-3]
		\arrow["{\bigoplus_i\kappa'}", from=2-2, to=3-2]
		\arrow["\nabla", from=3-2, to=4-3]
		\arrow["{\bigoplus_if_i}"{description}, from=4-1, to=3-2]
	\end{tikzcd}\]
	The top triangle commutes by additivity of $E\otimes-$. The triangle below that commutes by the universal property of the coproduct, since it is straightforward to check that $\nabla\circ\bigoplus_i(E\otimes f_i)$ and $h$ both satisfy the universal property of the colimit. The left trapezoid commutes by functoriality of $-\oplus-$ and the fact that $f_i$ is a homomorphism of left $E$-module objects for all $i$ in $I$. The right trapezoid commutes by naturality of $\nabla$. Finally, the bottom triangle commutes by the universal product of the coproduct, by showing that $\nabla\circ\bigoplus_if_i$ in place of $f$ also satisfies the universal property of the colimit. Hence $f$ is inded a homomorphism of left $E$-module objects, as desired.

	To recap, we have shown that given a set of left $E$-module objects $\{(N_i,\kappa_i)\}_{i\in I}$, that the inclusion maps $\iota_i:N_i\into \bigoplus_iN_i$ are morphisms in $E\text-\Mod$, and that given morphism $f_i:(N_i,\kappa_i)\to(N',\kappa')$ for all $i\in I$, the unique induced map $\bigoplus_iN_i\to N'$ is a morphism in $E\text-\Mod$. Thus, $E\text-\Mod$ does indeed have arbitrary coproducts, and the forgetful functor $E\text-\Mod\to\cSH$ preserves them.
\end{proof}

\begin{proposition}\label{E-Mod,free,forgetful_are_additive}
	Let $(E,\mu,e)$ be a monoid object in $\cSH$. Then $E\text-\Mod$ is an additive category, so that in particular the forgetful functor $E\text-\Mod\to\cSH$ and the free functor $\cSH\to E\text-\Mod$ are additive.
\end{proposition}
\begin{proof}
	It is a general fact that adjoint functors between additive categories are necessarily additive. In order to show $E\text-\Mod$ is an additive category, it suffices to show it has finite coproducts, that $\Hom_{E\text-\Mod}(N,N')$ is an abelian group for all left $E$-modules $N$ and $N'$, and that composition is bilinear. We know that $E\text-\Mod$ has coproducts which are preserved by the forgetful functor $E\text-\Mod\to\cSH$ by \autoref{coproduct_of_E_modules_is_coproduct_in_E_mod} (which is clearly faithful). Thus, because $\cSH$ is $\Ab$-enriched and $\Hom_{E\text-\Mod}(N,N')\sseq\cSH(N,N')$, it suffices to show that $\Hom_{E\text-\Mod}(N,N')$ is closed under addition and taking inverses. To see the former, let $f,g:N\to N'$ be left $E$-module homomorphisms, and consider the following diagram:
	% https://q.uiver.app/#q=WzAsMTIsWzAsMCwiRVxcb3RpbWVzIE4iXSxbMSwwLCJFXFxvdGltZXMgKE5cXG9wbHVzIE4pIl0sWzMsMCwiRVxcb3RpbWVzIChOJ1xcb3BsdXMgTicpIl0sWzQsMCwiRVxcb3RpbWVzIE4nIl0sWzQsNCwiTiciXSxbMCw0LCJOIl0sWzEsMSwiKEVcXG90aW1lcyBOKVxcb3BsdXMoRVxcb3RpbWVzIE4pIl0sWzMsMSwiKEVcXG90aW1lcyBOJylcXG90aW1lcyAoRVxcb3RpbWVzIE4nKSJdLFsyLDIsIihFXFxvdGltZXMgTicpXFxvdGltZXMgKEVcXG90aW1lcyBOKSJdLFsxLDQsIk5cXG9wbHVzIE4iXSxbMyw0LCJOJ1xcb3BsdXMgTiciXSxbMiwzLCJOJ1xcb3BsdXMgTiJdLFswLDEsIkVcXG90aW1lc1xcRGVsdGFfTiJdLFsxLDIsIkVcXG90aW1lcyAoZlxcb3BsdXMgZykiXSxbMiwzLCJFXFxvdGltZXMgXFxuYWJsYV97Tid9Il0sWzMsNCwiXFxrYXBwYSciXSxbMCw1LCJcXGthcHBhIiwyXSxbMCw2LCJcXERlbHRhX3tFXFxvdGltZXMgTn0iLDJdLFs2LDcsIihFXFxvdGltZXMgZilcXG9wbHVzKEVcXG90aW1lcyBnKSJdLFsyLDcsIlxcY29uZyJdLFsxLDYsIlxcY29uZyIsMl0sWzcsMywiXFxuYWJsYV97RVxcb3RpbWVzIE4nfSIsMl0sWzYsOCwiKEVcXG90aW1lcyBmKVxcb3BsdXMoRVxcb3RpbWVzIE4pIiwxXSxbOCw3LCIoRVxcb3RpbWVzIE4nKVxcb3BsdXMoRVxcb3RpbWVzIGcpIiwxXSxbNSw5LCJcXERlbHRhX04iXSxbOSwxMCwiZlxcb3BsdXMgZyJdLFsxMCw0LCJcXG5hYmxhX3tOJ30iXSxbOSwxMSwiZlxcb3BsdXMgTiJdLFsxMSwxMCwiTidcXG9wbHVzIGciXSxbNiw5LCJcXGthcHBhXFxvcGx1c1xca2FwcGEiLDJdLFs3LDEwLCJcXGthcHBhJ1xcb3BsdXNcXGthcHBhJyJdLFs4LDExLCJcXGthcHBhJ1xcb3BsdXNcXGthcHBhIl1d
	\[\begin{tikzcd}[column sep=tiny]
		{E\otimes N} & {E\otimes (N\oplus N)} && {E\otimes (N'\oplus N')} & {E\otimes N'} \\
		& {(E\otimes N)\oplus(E\otimes N)} && {(E\otimes N')\otimes (E\otimes N')} \\
		&& {(E\otimes N')\otimes (E\otimes N)} \\
		&& {N'\oplus N} \\
		N & {N\oplus N} && {N'\oplus N'} & {N'}
		\arrow["{E\otimes\Delta_N}", from=1-1, to=1-2]
		\arrow["{E\otimes (f\oplus g)}", from=1-2, to=1-4]
		\arrow["{E\otimes \nabla_{N'}}", from=1-4, to=1-5]
		\arrow["{\kappa'}", from=1-5, to=5-5]
		\arrow["\kappa"', from=1-1, to=5-1]
		\arrow["{\Delta_{E\otimes N}}"', from=1-1, to=2-2]
		\arrow["{(E\otimes f)\oplus(E\otimes g)}", from=2-2, to=2-4]
		\arrow["\cong", from=1-4, to=2-4]
		\arrow["\cong"', from=1-2, to=2-2]
		\arrow["{\nabla_{E\otimes N'}}"', from=2-4, to=1-5]
		\arrow["{(E\otimes f)\oplus(E\otimes N)}"{description}, from=2-2, to=3-3]
		\arrow["{(E\otimes N')\oplus(E\otimes g)}"{description}, from=3-3, to=2-4]
		\arrow["{\Delta_N}", from=5-1, to=5-2]
		\arrow["{f\oplus g}", from=5-2, to=5-4]
		\arrow["{\nabla_{N'}}", from=5-4, to=5-5]
		\arrow["{f\oplus N}", from=5-2, to=4-3]
		\arrow["{N'\oplus g}", from=4-3, to=5-4]
		\arrow["\kappa\oplus\kappa"', from=2-2, to=5-2]
		\arrow["{\kappa'\oplus\kappa'}", from=2-4, to=5-4]
		\arrow["{\kappa'\oplus\kappa}", from=3-3, to=4-3]
	\end{tikzcd}\]
	The outermost trapezoids commute by naturality of $\Delta$ and $\nabla$. The triangles in the top corners and the top middle rectangle commute by additivity of $E\otimes-$. The middle triangle commutes by functoriality of $\oplus$ and $\otimes$. The middle trapezoids commute by the fact that $f$ and $g$ are homomorphisms of left $E$-modules. Finally, the middle bottom triangle commutes by functoriality of $-\oplus-$. Commutativity of the above diagram shows that $f+g$ is a homomorphism of left $E$-modules as desired. Finally, to see $-f$ is a left $E$-module homomorphism if $f$ is, we would like to show that $\kappa'\circ(E\otimes(-f))=(-f)\circ\kappa$. This follows by the fact that $\kappa'\circ(E\otimes f)=f\circ\kappa$ and additivity of $-\otimes-$ and composition.
\end{proof}

\begin{proposition}\label{pi_*E_is_ring_for_E_monoid_appendix}
	Let $(E,\mu,e)$ be a monoid object in $\cSH$, and consider the multiplication map $\pi_*(E)\times\pi_*(E)\to\pi_*(E)$ which sends classes $x:S^a\to E$ and $y:S^b\to E$ to the composition
	\[S^{a+b}\xr{\phi_{a,b}}S^a\otimes S^b\xr{x\otimes y}E\otimes E\xr\mu E.\]
	Then this endows $\pi_*(E)$ with the structure of an $A$-graded ring with unit $e\in\pi_0(E)=[S,E]$.
\end{proposition}
\begin{proof}
	Here we are using \autoref{A_graded_ring}, so it suffices to show the given assignment is associative and unital w.r.t.\ homogeneous elements. Suppose we have classes $x$, $y$, and $z$ in $\pi_a(E)$, $\pi_b(E)$, and $\pi_c(E)$, respectively. To see associativity, consider the following diagram:
	% https://q.uiver.app/#q=WzAsNixbMCwxLCJTXnthK2IrY30iXSxbMSwxLCJTXmFcXG90aW1lcyBTXmJcXG90aW1lcyBTXmMiXSxbMiwxLCIgIEVcXG90aW1lcyBFXFxvdGltZXMgRSJdLFszLDAsIkVcXG90aW1lcyBFIl0sWzMsMSwiRSJdLFszLDIsIkVcXG90aW1lcyBFIl0sWzAsMSwiXFxjb25nIl0sWzEsMiwieFxcb3RpbWVzIHlcXG90aW1lcyB6Il0sWzIsMywiXFxtdVxcb3RpbWVzIEUiXSxbMyw0LCJcXG11Il0sWzIsNSwiRVxcb3RpbWVzXFxtdSIsMl0sWzUsNCwiXFxtdSIsMl1d
	\[\begin{tikzcd}
		&&& {E\otimes E} \\
		{S^{a+b+c}} & {S^a\otimes S^b\otimes S^c} & {  E\otimes E\otimes E} & E \\
		&&& {E\otimes E}
		\arrow["\cong", from=2-1, to=2-2]
		\arrow["{x\otimes y\otimes z}", from=2-2, to=2-3]
		\arrow["{\mu\otimes E}", from=2-3, to=1-4]
		\arrow["\mu", from=1-4, to=2-4]
		\arrow["E\otimes\mu"', from=2-3, to=3-4]
		\arrow["\mu"', from=3-4, to=2-4]
	\end{tikzcd}\]
	(here the first arrow is the unique isomorphism obtained by composing products of $\phi_{a,b}$'s, see \autoref{unique_comp_Sas}). It commutes by associativity of $\mu$. It follows by functoriality of $-\otimes-$ that the top composition is $(x\cdot y)\cdot z$ while the bottom is $x\cdot(y\cdot z)$, so they are equal as desired. To see that $e\in\pi_0(E)$ is a left and right unit for this multiplication, consider the following diagram
	% https://q.uiver.app/#q=WzAsNSxbMiwwLCJTXmEiXSxbMiwxLCJFIl0sWzAsMSwiRVxcb3RpbWVzIEUiXSxbNCwxLCJFXFxvdGltZXMgRSJdLFsyLDIsIkUiXSxbMCwxLCJ4Il0sWzAsMiwiZVxcb3RpbWVzIHgiLDJdLFswLDMsInhcXG90aW1lcyBlIl0sWzEsNCwiIiwxLHsibGV2ZWwiOjIsInN0eWxlIjp7ImhlYWQiOnsibmFtZSI6Im5vbmUifX19XSxbMSwyLCJlXFxvdGltZXMgRSIsMl0sWzEsMywiRVxcb3RpbWVzIGUiXSxbMyw0LCJcXG11Il0sWzIsNCwiXFxtdSIsMl1d
	\[\begin{tikzcd}
		&& {S^a} \\
		{E\otimes E} && E && {E\otimes E} \\
		&& E
		\arrow["x", from=1-3, to=2-3]
		\arrow["{e\otimes x}"', from=1-3, to=2-1]
		\arrow["{x\otimes e}", from=1-3, to=2-5]
		\arrow[Rightarrow, no head, from=2-3, to=3-3]
		\arrow["{e\otimes E}"', from=2-3, to=2-1]
		\arrow["{E\otimes e}", from=2-3, to=2-5]
		\arrow["\mu", from=2-5, to=3-3]
		\arrow["\mu"', from=2-1, to=3-3]
	\end{tikzcd}\]
	Commutativity of the two top triangles is functoriality of $-\otimes-$. Commutativity of the bottom two triangles is unitality of $\mu$. Thus the diagram commutes, so $e\cdot x=x=x\cdot e$. Finally, to see this product is bilinear (distributive). Suppose we further have some $x'\in\pi_a(E)$ and $y'\in\pi_b(E)$, and consider the following diagrams:
	% https://q.uiver.app/#q=WzAsMTgsWzAsMCwiU157YStifSJdLFsxLDAsIlNeYVxcb3RpbWVzIFNeYiJdLFswLDEsIlNee2ErYn1cXG9wbHVzIFNee2ErYn0iXSxbMSwxLCIoU15hXFxvdGltZXMgU15iKVxcb3BsdXMoU15hXFxvdGltZXMgU15iKSJdLFsyLDAsIihTXmFcXG9wbHVzIFNeYSlcXG90aW1lcyBTXmIiXSxbMiwxLCIoRVxcb3RpbWVzIEUpXFxvcGx1cyhFXFxvdGltZXMgRSkiXSxbMywwLCIoRVxcb3BsdXMgRSlcXG90aW1lcyBFIl0sWzMsMSwiRVxcb3RpbWVzIEUiXSxbMCwyLCJTXnthK2J9Il0sWzEsMiwiU15hXFxvdGltZXMgU15iIl0sWzIsMiwiU15iXFxvdGltZXMoU15iXFxvcGx1cyBTXmIpIl0sWzMsMiwiRVxcb3RpbWVzKEVcXG9wbHVzIEUpIl0sWzMsMywiRVxcb3RpbWVzIEUiXSxbMiwzLCIoRVxcb3RpbWVzIEUpXFxvcGx1cyhFXFxvdGltZXMgRSkiXSxbMSwzLCIoU15hXFxvdGltZXMgU15iKVxcb3BsdXMoU15hXFxvdGltZXMgU15iKSJdLFswLDMsIlNee2ErYn1cXG9wbHVzIFNee2ErYn0iXSxbNCwxLCJFIl0sWzQsMywiRSJdLFswLDEsIlxccGhpX3thLGJ9Il0sWzAsMiwiXFxEZWx0YSIsMl0sWzIsMywiXFxwaGlfe2EsYn1cXG9wbHVzXFxwaGlfe2EsYn0iLDJdLFsxLDMsIlxcRGVsdGEiXSxbMSw0LCJcXERlbHRhXFxvdGltZXMgU15iIl0sWzQsMywiXFxjb25nIiwyXSxbMyw1LCIoeFxcb3RpbWVzIHkpXFxvcGx1cyh4J1xcb3RpbWVzIHkpIiwyXSxbNCw2LCIoeFxcb3BsdXMgeCcpXFxvdGltZXMgeSJdLFs2LDUsIlxcY29uZyIsMl0sWzUsNywiXFxuYWJsYSIsMl0sWzYsNywiXFxuYWJsYVxcb3RpbWVzIEUiXSxbOCw5LCJcXHBoaV97YSxifSJdLFs5LDEwLCJTXmFcXG90aW1lc1xcRGVsdGEiXSxbMTAsMTEsInhcXG90aW1lcyh5XFxvcGx1cyB5JykiXSxbMTEsMTIsIkVcXG90aW1lc1xcbmFibGEiXSxbMTEsMTMsIlxcY29uZyIsMl0sWzEzLDEyLCJcXG5hYmxhIiwyXSxbMTAsMTQsIlxcY29uZyIsMl0sWzE0LDEzLCIoeFxcb3RpbWVzIHkpXFxvcGx1cyh4XFxvdGltZXMgeScpIiwyXSxbOSwxNCwiXFxEZWx0YSJdLFs4LDE1LCJcXERlbHRhIiwyXSxbMTUsMTQsIlxccGhpX3thLGJ9XFxvcGx1c1xccGhpX3thLGJ9IiwyXSxbNywxNiwiXFxtdSJdLFsxMiwxNywiXFxtdSJdXQ==
	\[\begin{tikzcd}
		{S^{a+b}} & {S^a\otimes S^b} & {(S^a\oplus S^a)\otimes S^b} & {(E\oplus E)\otimes E} \\
		{S^{a+b}\oplus S^{a+b}} & {(S^a\otimes S^b)\oplus(S^a\otimes S^b)} & {(E\otimes E)\oplus(E\otimes E)} & {E\otimes E} & E \\
		{S^{a+b}} & {S^a\otimes S^b} & {S^b\otimes(S^b\oplus S^b)} & {E\otimes(E\oplus E)} \\
		{S^{a+b}\oplus S^{a+b}} & {(S^a\otimes S^b)\oplus(S^a\otimes S^b)} & {(E\otimes E)\oplus(E\otimes E)} & {E\otimes E} & E
		\arrow["{\phi_{a,b}}", from=1-1, to=1-2]
		\arrow["\Delta"', from=1-1, to=2-1]
		\arrow["{\phi_{a,b}\oplus\phi_{a,b}}"', from=2-1, to=2-2]
		\arrow["\Delta", from=1-2, to=2-2]
		\arrow["{\Delta\otimes S^b}", from=1-2, to=1-3]
		\arrow["\cong"', from=1-3, to=2-2]
		\arrow["{(x\otimes y)\oplus(x'\otimes y)}"', from=2-2, to=2-3]
		\arrow["{(x\oplus x')\otimes y}", from=1-3, to=1-4]
		\arrow["\cong"', from=1-4, to=2-3]
		\arrow["\nabla"', from=2-3, to=2-4]
		\arrow["{\nabla\otimes E}", from=1-4, to=2-4]
		\arrow["{\phi_{a,b}}", from=3-1, to=3-2]
		\arrow["{S^a\otimes\Delta}", from=3-2, to=3-3]
		\arrow["{x\otimes(y\oplus y')}", from=3-3, to=3-4]
		\arrow["E\otimes\nabla", from=3-4, to=4-4]
		\arrow["\cong"', from=3-4, to=4-3]
		\arrow["\nabla"', from=4-3, to=4-4]
		\arrow["\cong"', from=3-3, to=4-2]
		\arrow["{(x\otimes y)\oplus(x\otimes y')}"', from=4-2, to=4-3]
		\arrow["\Delta", from=3-2, to=4-2]
		\arrow["\Delta"', from=3-1, to=4-1]
		\arrow["{\phi_{a,b}\oplus\phi_{a,b}}"', from=4-1, to=4-2]
		\arrow["\mu", from=2-4, to=2-5]
		\arrow["\mu", from=4-4, to=4-5]
	\end{tikzcd}\]
	The unlabeled isomorphisms are those given by the fact that $-\otimes-$ is additive in each variable (since $\cSH$ is tensor triangulated). Commutativity of the left squares is naturality of $\Delta:X\to X\oplus X$ in an additive category. Commutativity of the rest of the diagram follows again from the fact that $-\otimes-$ is an additive functor in each variable. Hence, by functoriality of $-\otimes-$, these diagrams tell us that $(x+x')\cdot y=x\cdot y+x'\cdot y$ and $x\cdot(y+y')=x\cdot y+x\cdot y'$, respectively.
\end{proof}

\begin{proposition}\label{pi_*(E)_is_A-graded_commutative_if_E_is_commutative}
	For all $a,b\in A$ there exists an element $\theta_{a,b}\in\pi_0(S)=[S,S]$ such that given any commutative monoid object $(E,\mu,e)$ in $\cSH$, the $A$-graded ring structure on $\pi_\ast(E)$ (\autoref{pi_*E_is_ring_for_E_monoid}) has a commutativity formula given by
	\[x\cdot y=y\cdot x\cdot (e\circ\theta_{a,b})\]
	for all $x\in\pi_a(E)$ and $y\in\pi_b(E)$. In particular, $\theta_{a,b}\in\mathrm{Aut}(S)$ is the composition
	\[S\xr{\cong}S^{-a-b}\otimes S^a\otimes S^b\xr{S^{-a-b}\otimes\tau}S^{-a-b}\otimes S^b\otimes S^a\xr\cong S,\]
	where the outermost maps are the unique maps specified by \autoref{unique_comp_Sas}.
\end{proposition}
\begin{proof}
	Let $(E,\mu,e)$, $x$, and $y$ as in the statement of the proposition. Now consider the following diagram
	% https://q.uiver.app/#q=WzAsNyxbMCwwLCJTXnthK2J9Il0sWzAsMiwiU157YStifSJdLFsyLDIsIlNeYlxcb3RpbWVzIFNeYSJdLFsyLDAsIlNeYVxcb3RpbWVzIFNeYiJdLFs0LDAsIkVcXG90aW1lcyBFIl0sWzQsMiwiRVxcb3RpbWVzIEUiXSxbNiwxLCJFIl0sWzAsMSwiXFxwaGlfe2IsYX1eey0xfVxcY2lyY1xcdGF1XFxjaXJjXFxwaGlfe2EsYn0iLDIseyJzdHlsZSI6eyJib2R5Ijp7Im5hbWUiOiJkYXNoZWQifX19XSxbMSwyLCJcXHBoaV97YixhfSJdLFswLDMsIlxccGhpX3thLGJ9Il0sWzMsMiwiXFx0YXUiLDJdLFs0LDUsIlxcdGF1IiwyXSxbNCw2LCJcXG11Il0sWzIsNSwieVxcb3RpbWVzIHgiXSxbNSw2LCJcXG11IiwyXSxbMyw0LCJ4XFxvdGltZXMgeSJdXQ==
	\[\begin{tikzcd}[sep=small]
		{S^{a+b}} && {S^a\otimes S^b} && {E\otimes E} \\
		&&&&&& E \\
		{S^{a+b}} && {S^b\otimes S^a} && {E\otimes E}
		\arrow["{\phi_{b,a}^{-1}\circ\tau\circ\phi_{a,b}}"', dashed, from=1-1, to=3-1]
		\arrow["{\phi_{b,a}}", from=3-1, to=3-3]
		\arrow["{\phi_{a,b}}", from=1-1, to=1-3]
		\arrow["\tau"', from=1-3, to=3-3]
		\arrow["\tau"', from=1-5, to=3-5]
		\arrow["\mu", from=1-5, to=2-7]
		\arrow["{y\otimes x}", from=3-3, to=3-5]
		\arrow["\mu"', from=3-5, to=2-7]
		\arrow["{x\otimes y}", from=1-3, to=1-5]
	\end{tikzcd}\]
	The left square commutes by definition. The middle square commutes by naturality of the symmetry isomorphism. Finally, the right square commutes by commutativity of $E$. Unravelling definitions, we have shown that under the product on $\pi_\ast(E)$ induced by the $\phi_{a,b}$'s,
	\[x\cdot y=(y\cdot x)\circ(\phi_{b,a}^{-1}\circ\tau\circ\phi_{a,b}).\]
	Thus, in order to show the desired result it further suffices to show that
	\[(y\cdot x)\circ(\phi_{b,a}^{-1}\circ\tau\circ\phi_{a,b})=y\cdot x\cdot(e\circ\theta_{a,b}).\]
	Consider the following diagram:
	% https://q.uiver.app/#q=WzAsMTIsWzAsMCwiU157YStifSJdLFswLDEsIlNeYlxcb3RpbWVzIFNeYVxcb3RpbWVzIFNeey1hLWJ9XFxvdGltZXMgU15hXFxvdGltZXMgU15iIl0sWzAsMiwiU15iXFxvdGltZXMgU15hXFxvdGltZXMgU157LWEtYn1cXG90aW1lcyBTXmJcXG90aW1lcyBTXmEiXSxbMCw0LCJFXFxvdGltZXMgRVxcb3RpbWVzIEUiXSxbMiw0LCJFXFxvdGltZXMgRSJdLFsxLDIsIlNeYlxcb3RpbWVzIFNeYSJdLFsyLDAsIlNeYVxcb3RpbWVzIFNeYiJdLFsyLDEsIlNeYlxcb3RpbWVzIFNeYSJdLFsyLDIsIlNee2ErYn0iXSxbMiwzLCJFXFxvdGltZXMgRSJdLFswLDUsIkVcXG90aW1lcyBFIl0sWzIsNSwiRSJdLFswLDEsIlxcY29uZyIsMl0sWzAsNiwiXFxwaGlfe2EsYn0iXSxbNiw3LCJcXHRhdSJdLFs3LDgsIlxccGhpX3tiLGF9XnstMX0iXSxbOCw1LCJcXHBoaV97YixhfSIsMl0sWzIsNywiXFxjb25nIl0sWzEsNiwiXFxjb25nIiwyXSxbOSwzLCJFXFxvdGltZXMgRVxcb3RpbWVzIGUiLDJdLFsxLDIsIlNeYlxcb3RpbWVzIFNeYVxcb3RpbWVzIFNeey1hLWJ9XFxvdGltZXNcXHRhdSIsMl0sWzMsMTAsIlxcbXVcXG90aW1lcyBFIiwyXSxbMyw0LCJFXFxvdGltZXMgXFxtdSJdLFs0LDExLCJcXG11Il0sWzEwLDExLCJcXG11IiwyXSxbOSw0LCIiLDAseyJsZXZlbCI6Miwic3R5bGUiOnsiaGVhZCI6eyJuYW1lIjoibm9uZSJ9fX1dLFs1LDksInlcXG90aW1lcyB4Il0sWzUsMywieVxcb3RpbWVzIHhcXG90aW1lcyBlIiwyXSxbMiw1LCJcXGNvbmciXSxbNSw3LCIiLDIseyJsZXZlbCI6Miwic3R5bGUiOnsiaGVhZCI6eyJuYW1lIjoibm9uZSJ9fX1dXQ==
	\[\begin{tikzcd}
		{S^{a+b}} && {S^a\otimes S^b} \\
		{S^b\otimes S^a\otimes S^{-a-b}\otimes S^a\otimes S^b} && {S^b\otimes S^a} \\
		{S^b\otimes S^a\otimes S^{-a-b}\otimes S^b\otimes S^a} & {S^b\otimes S^a} & {S^{a+b}} \\
		&& {E\otimes E} \\
		{E\otimes E\otimes E} && {E\otimes E} \\
		{E\otimes E} && E
		\arrow["\cong"', from=1-1, to=2-1]
		\arrow["{\phi_{a,b}}", from=1-1, to=1-3]
		\arrow["\tau", from=1-3, to=2-3]
		\arrow["{\phi_{b,a}^{-1}}", from=2-3, to=3-3]
		\arrow["{\phi_{b,a}}"', from=3-3, to=3-2]
		\arrow["\cong", from=3-1, to=2-3]
		\arrow["\cong"', from=2-1, to=1-3]
		\arrow["{E\otimes E\otimes e}"', from=4-3, to=5-1]
		\arrow["{S^b\otimes S^a\otimes S^{-a-b}\otimes\tau}"', from=2-1, to=3-1]
		\arrow["{\mu\otimes E}"', from=5-1, to=6-1]
		\arrow["{E\otimes \mu}", from=5-1, to=5-3]
		\arrow["\mu", from=5-3, to=6-3]
		\arrow["\mu"', from=6-1, to=6-3]
		\arrow[Rightarrow, no head, from=4-3, to=5-3]
		\arrow["{y\otimes x}", from=3-2, to=4-3]
		\arrow["{y\otimes x\otimes e}"', from=3-2, to=5-1]
		\arrow["\cong", from=3-1, to=3-2]
		\arrow[Rightarrow, no head, from=3-2, to=2-3]
	\end{tikzcd}\]
	Here any map simply labelled $\cong$ is an appropriate composition of copies of $\phi_{a,b}$'s, associators, and their inverses, so that each of these maps are necessarily unique by \autoref{unique_comp_Sas}. The triangles in the top large rectangle commutes by coherence for the $\phi_{a,b}$'s. The parallelogram commutes by naturality of $\tau$ and coherence of the of $\phi_{a,b}$'s. The middle skewed triangle commutes by functoriality of $-\otimes-$. The triangle below that commutes by unitality of $\mu$. Finally, the bottom rectangle commmutes by associativity of $\mu$. Hence, by unravelling definitions and applying functoriality of $-\otimes-$, we get that the right composition is $(y\cdot x)\circ(\phi_{b,a}^{-1}\circ\tau\circ\phi_{a,b})$, while the left composition is $y\cdot x\cdot(e\circ\theta_{a,b})$, so they are equal as desired.
\end{proof}

\begin{lemma}\label{multipy_by_degree_0_is_same_as_compose}
	Suppose we have homogeneous elements $x,y\in\pi_*(S)$ with $x$ of degree $0$, then we have $x\cdot y=y\cdot x=x\circ y$ (where the $\cdot$ denotes the product given in \autoref{pi_*E_is_ring_for_E_monoid_appendix}).
\end{lemma}
\begin{proof}
	As morphisms, $y$ is an arrow $S^a\to S$ for some $a$ in $A$, and $x$ is a morphism $S\to S$. Then consider the following diagram:
	% https://q.uiver.app/#q=WzAsOSxbMiwwLCJTXmEiXSxbNCwwLCJTXmFcXG90aW1lcyBTIl0sWzQsMiwiU1xcb3RpbWVzIFMiXSxbMiwyLCJTIl0sWzIsMSwiUyJdLFswLDAsIlNcXG90aW1lcyBTXmEiXSxbMCwyLCJTXFxvdGltZXMgUyJdLFsxLDEsIlNcXG90aW1lcyBTIl0sWzMsMSwiU1xcb3RpbWVzIFMiXSxbMCwxLCJcXHBoaV97YSwwfT1cXHJob197U15hfV57LTF9Il0sWzIsMywiXFxwaGleey0xfV97MCwwfT1cXGxhbWJkYV9TIl0sWzAsNCwieSIsMl0sWzQsMywieCIsMl0sWzEsMiwieFxcb3RpbWVzIHkiXSxbMCw1LCJcXHBoaV97MCxhfT1cXGxhbWJkYV97U15hfV57LTF9IiwyXSxbNSw2LCJ5XFxvdGltZXMgeCIsMl0sWzYsMywiXFxwaGleey0xfV97MCwwfT1cXHJob19TIiwyXSxbNSw3LCJTXFxvdGltZXMgeSIsMV0sWzcsNiwieFxcb3RpbWVzIFMiLDFdLFsxLDgsInlcXG90aW1lcyBTIiwxXSxbOCwyLCJTXFxvdGltZXMgeCIsMV0sWzcsNCwiXFxsYW1iZGFfUz1cXHJob19TIl0sWzgsNCwiXFxyaG9fUz1cXGxhbWJkYV9TIiwyXV0=
	\[\begin{tikzcd}
		{S\otimes S^a} && {S^a} && {S^a\otimes S} \\
		& {S\otimes S} & S & {S\otimes S} \\
		{S\otimes S} && S && {S\otimes S}
		\arrow["{\phi_{a,0}=\rho_{S^a}^{-1}}", from=1-3, to=1-5]
		\arrow["{\phi^{-1}_{0,0}=\lambda_S}", from=3-5, to=3-3]
		\arrow["y"', from=1-3, to=2-3]
		\arrow["x"', from=2-3, to=3-3]
		\arrow["{x\otimes y}", from=1-5, to=3-5]
		\arrow["{\phi_{0,a}=\lambda_{S^a}^{-1}}"', from=1-3, to=1-1]
		\arrow["{y\otimes x}"', from=1-1, to=3-1]
		\arrow["{\phi^{-1}_{0,0}=\rho_S}"', from=3-1, to=3-3]
		\arrow["{S\otimes y}"{description}, from=1-1, to=2-2]
		\arrow["{x\otimes S}"{description}, from=2-2, to=3-1]
		\arrow["{y\otimes S}"{description}, from=1-5, to=2-4]
		\arrow["{S\otimes x}"{description}, from=2-4, to=3-5]
		\arrow["{\lambda_S=\rho_S}", from=2-2, to=2-3]
		\arrow["{\rho_S=\lambda_S}"', from=2-4, to=2-3]
	\end{tikzcd}\]
	The trapezoids commute by naturality of the unitors, and the triangles commute by functoriality of $-\otimes-$. The outside compositions are $y\cdot x$ on the left and $x\cdot y$ on the right, and the middle composition is $x\circ y$, so indeed we have $y\cdot x=x\cdot y=x\circ y$, as desired.
\end{proof}

\begin{lemma}\label{theta_a,0=theta_0,a=id_S}
	Given $a\in A$, we have $\theta_{0,a}=\theta_{a,0}=\id_S$.
\end{lemma}
\begin{proof}
	Recall $\theta_{a,0}$ is the composition
	\[S\xr{\phi_{-a,a}} S^{-a}\otimes S^a\xr{S^{-a}\otimes\phi_{a,0}} S^{-a}\otimes(S^a\otimes S)\xr{S^{-a}\otimes\tau}S^{-a}\otimes(S\otimes S^a)\xr{S^{-a}\otimes\phi_{0,a}^{-1}} S^{-a}\otimes S^a\xr{\phi_{-a,a}^{-1}}S\]
	By the coherence theorem for symmetric monoidal categories and the fact that $\phi_{a,0}$ and $\phi_{0,a}$ coincide with the unitors, we have that the composition
	\[S^a\xr{\phi_{a,0}=\rho_{S^a}^{-1}} S^a\otimes S\xr\tau S\otimes S^a\xr{\phi_{0,a}^{-1}=\lambda_{S^a}}S^a\]
	is precisely the identity map, so by functoriality of $-\otimes-$, we have that $\theta_{a,0}$ is the composition
	\[S\xr{\phi_{-a,a}}S^{-a}\otimes S^a\xr=S^{-a}\otimes S^{a}\xr{\phi_{-a,a}^{-1}}S,\]
	so $\theta_{a,0}=\id_S$, meaning
	\[x\cdot y=y\cdot x\cdot(e\circ\theta_{a,0})=y\cdot x\cdot e=y\cdot x,\]
	where the last equality follows by the fact that $e$ is the unit for the multiplication on $\pi_\ast(E)$. An entirely analagous argument yields that $\theta_{0,a}=\id_S$.
\end{proof}

\begin{lemma}\label{theta_ab.theta_ba=id}
	Let $a,b\in A$. Then $\theta_{a,b}\cdot\theta_{b,a}=\id_S$.
\end{lemma}
\begin{proof}
	By \autoref{multipy_by_degree_0_is_same_as_compose}, it suffices to show that $\theta_{a,b}\circ\theta_{b,a}=\id_S$. To see this, consider the following diagram:
	% https://q.uiver.app/#q=WzAsNyxbMCwwLCJTIl0sWzEsMCwiU157LWEtYn1cXG90aW1lcyBTXmJcXG90aW1lcyBTXmEiXSxbMiwwLCJTXnstYS1ifVxcb3RpbWVzIFNeYVxcb3RpbWVzIFNeYiJdLFszLDAsIlMiXSxbMywxLCJTXnstYS1ifVxcb3RpbWVzIFNeYVxcb3RpbWVzIFNeYiJdLFszLDIsIlNeey1hLWJ9XFxvdGltZXMgU15iXFxvdGltZXMgU15hIl0sWzMsMywiUyJdLFswLDEsIlxccGhpIl0sWzEsMiwiU157LWEtYn1cXG90aW1lcyBcXHRhdSJdLFsyLDMsIlxccGhpIl0sWzMsNCwiXFxwaGkiXSxbNCw1LCJTXnstYS1ifVxcb3RpbWVzIFxcdGF1Il0sWzUsNiwiXFxwaGkiXSxbMiw0LCIiLDEseyJsZXZlbCI6Miwic3R5bGUiOnsiaGVhZCI6eyJuYW1lIjoibm9uZSJ9fX1dLFswLDYsIiIsMix7ImxldmVsIjoyLCJzdHlsZSI6eyJoZWFkIjp7Im5hbWUiOiJub25lIn19fV0sWzEsNSwiIiwxLHsibGV2ZWwiOjIsInN0eWxlIjp7ImhlYWQiOnsibmFtZSI6Im5vbmUifX19XV0=
	\[\begin{tikzcd}
		S & {S^{-a-b}\otimes S^b\otimes S^a} & {S^{-a-b}\otimes S^a\otimes S^b} & S \\
		&&& {S^{-a-b}\otimes S^a\otimes S^b} \\
		&&& {S^{-a-b}\otimes S^b\otimes S^a} \\
		&&& S
		\arrow["\phi", from=1-1, to=1-2]
		\arrow["{S^{-a-b}\otimes \tau}", from=1-2, to=1-3]
		\arrow["\phi", from=1-3, to=1-4]
		\arrow["\phi", from=1-4, to=2-4]
		\arrow["{S^{-a-b}\otimes \tau}", from=2-4, to=3-4]
		\arrow["\phi", from=3-4, to=4-4]
		\arrow[Rightarrow, no head, from=1-3, to=2-4]
		\arrow[Rightarrow, no head, from=1-1, to=4-4]
		\arrow[Rightarrow, no head, from=1-2, to=3-4]
	\end{tikzcd}\]
	Here we are suppressing associators, and any map labelled $\phi$ is the appropriate composition of $\phi_{a,b}$'s, unitors, associators, identities, and their inverses (see \autoref{unique_comp_Sas}). Clearly each region commutes, the middle by the fact that $\tau^2=0$, and the other two regions by coherence for the $\phi$'s. Thus we have shwon $\theta_{a,b}\cdot\theta_{b,a}=\theta_{a,b}\cdot\theta_{b,a}=\id_S$, as desired.
\end{proof}

\begin{lemma}\label{theta_ab.theta_ac=theta_ab+c_and_theta_ba.theta_ca=theta_b+ca}
	Let $a,b,c\in A$. Then $\theta_{a,b}\cdot\theta_{a,c}=\theta_{a,b+c}$ and $\theta_{b,a}\cdot\theta_{c,a}=\theta_{b+c,a}$.
\end{lemma}
\begin{proof}
	By \autoref{multipy_by_degree_0_is_same_as_compose}, it suffices to show that $\theta_{a,b}\circ\theta_{a,c}=\theta_{a,b+c}$ and $\theta_{b,a}\circ\theta_{c,a}=\theta_{b+c,a}$. First we show $\theta_{a,b}\circ\theta_{a,c}=\theta_{a,b+c}$. To see this, consider the following diagram:
	% https://q.uiver.app/#q=WzAsMjQsWzAsMCwiUyJdLFsyLDAsIlNeey1hLWN9U15hU15jIl0sWzQsMCwiU157LWEtY31TXmNTXmEiXSxbNiwwLCJTIl0sWzYsMiwiU157LWEtYn1TXmFTXmIiXSxbNiw0LCJTXnstYS1ifVNeYlNeYSJdLFs2LDgsIlMiXSxbMiwyLCJTXnstYS1jfVNeey1ifVNeYVNeYlNeYyJdLFswLDIsIlNeey1hLWItY31TXmFTXntiK2N9Il0sWzAsNCwiU157LWEtYi1jfVNee2IrY31TXmEiXSxbNCwyLCJTXnstYS1jfVNeY1Neey1ifVNeYVNeYiJdLFs0LDQsIlNeey1hLWN9U15jU157LWJ9U15iU15hIl0sWzIsNCwiU157LWEtY31TXnstYn1TXmJTXmNTXmEiXSxbMCw4LCJTIl0sWzIsNiwiU157LWEtY31TXmNTXmEiXSxbNCw2LCJTXnstYS1jfVNeY1NeYSJdLFsxLDEsIihcXHRleHQgQSkiXSxbMywxLCIoXFx0ZXh0IEIpIl0sWzUsMSwiKFxcdGV4dCBDKSJdLFsxLDMsIihcXHRleHQgRCkiXSxbMywzLCIoXFx0ZXh0IEUpIl0sWzUsMywiKFxcdGV4dCBGKSJdLFszLDUsIihcXHRleHQgRykiXSxbMyw3LCIoXFx0ZXh0IEgpIl0sWzAsMSwiXFxwaGkiXSxbMSwyLCJTXnstYS1jfVxcdGF1Il0sWzIsMywiXFxwaGkiXSxbMyw0LCJcXHBoaSJdLFs0LDUsIlNeey1hLWJ9XFx0YXUiXSxbNSw2LCJcXHBoaSJdLFsxLDcsIlxccGhpIiwyXSxbMCw4LCJcXHBoaSIsMl0sWzgsOSwiU157LWEtYi1jfVxcdGF1IiwyXSxbNywxMCwiU157LWEtY31cXHRhdV97U157LWJ9U15hU15iLFNeY30iXSxbMiwxMCwiXFxwaGkiXSxbMTAsMTEsIlNeey1hLWN9U15jU157LWJ9XFx0YXVfe2IsYX0iLDFdLFs0LDEwLCJcXHBoaSIsMl0sWzUsMTEsIlxccGhpIiwyXSxbOCw3LCJcXHBoaSJdLFs5LDEyLCJcXHBoaSJdLFs3LDEyLCJTXnstYS1jfVNeey1ifVxcdGF1X3tTXmEsU15iU15jfSIsMV0sWzEyLDExLCJTXnstYS1jfVxcdGF1X3tTXnstYn1TXmIsU15jfVNeYSJdLFs5LDEzLCJcXHBoaSIsMl0sWzEzLDYsIiIsMix7ImxldmVsIjoyLCJzdHlsZSI6eyJoZWFkIjp7Im5hbWUiOiJub25lIn19fV0sWzE0LDEyLCJcXHBoaSJdLFsxNCwxNSwiIiwyLHsibGV2ZWwiOjIsInN0eWxlIjp7ImhlYWQiOnsibmFtZSI6Im5vbmUifX19XSxbMTUsMTEsIlxccGhpIiwyXV0=
	\begin{equation}\label{theta_ab_o_theta_ac}\begin{tikzcd}[sep=tiny]
		S && {S^{-a-c}S^aS^c} && {S^{-a-c}S^cS^a} && S \\
		& {(\text A)} && {(\text B)} && {(\text C)} \\
		{S^{-a-b-c}S^aS^{b+c}} && {S^{-a-c}S^{-b}S^aS^bS^c} && {S^{-a-c}S^cS^{-b}S^aS^b} && {S^{-a-b}S^aS^b} \\
		& {(\text D)} && {(\text E)} && {(\text F)} \\
		{S^{-a-b-c}S^{b+c}S^a} && {S^{-a-c}S^{-b}S^bS^cS^a} && {S^{-a-c}S^cS^{-b}S^bS^a} && {S^{-a-b}S^bS^a} \\
		&&& {(\text G)} \\
		&& {S^{-a-c}S^cS^a} && {S^{-a-c}S^cS^a} \\
		&&& {(\text H)} \\
		S &&&&&& S
		\arrow["\phi", from=1-1, to=1-3]
		\arrow["{S^{-a-c}\tau}", from=1-3, to=1-5]
		\arrow["\phi", from=1-5, to=1-7]
		\arrow["\phi", from=1-7, to=3-7]
		\arrow["{S^{-a-b}\tau}", from=3-7, to=5-7]
		\arrow["\phi", from=5-7, to=9-7]
		\arrow["\phi"', from=1-3, to=3-3]
		\arrow["\phi"', from=1-1, to=3-1]
		\arrow["{S^{-a-b-c}\tau}"', from=3-1, to=5-1]
		\arrow["{S^{-a-c}\tau_{S^{-b}S^aS^b,S^c}}", from=3-3, to=3-5]
		\arrow["\phi", from=1-5, to=3-5]
		\arrow["{S^{-a-c}S^cS^{-b}\tau_{b,a}}"{description}, from=3-5, to=5-5]
		\arrow["\phi"', from=3-7, to=3-5]
		\arrow["\phi"', from=5-7, to=5-5]
		\arrow["\phi", from=3-1, to=3-3]
		\arrow["\phi", from=5-1, to=5-3]
		\arrow["{S^{-a-c}S^{-b}\tau_{S^a,S^bS^c}}"{description}, from=3-3, to=5-3]
		\arrow["{S^{-a-c}\tau_{S^{-b}S^b,S^c}S^a}", from=5-3, to=5-5]
		\arrow["\phi"', from=5-1, to=9-1]
		\arrow[Rightarrow, no head, from=9-1, to=9-7]
		\arrow["\phi", from=7-3, to=5-3]
		\arrow[Rightarrow, no head, from=7-3, to=7-5]
		\arrow["\phi"', from=7-5, to=5-5]
	\end{tikzcd}\end{equation}
	Here we are omitting $\otimes$ from the notation, and each occurrence of an arrow labelled $\phi$ indicates it is the unique arrow that can be obtained as a formal composition of tensor products of copies of $\phi_{a,b}$'s, unitors, associators, and their inverses (\autoref{unique_comp_Sas}). Clearly the composition going around the top and then the right is $\theta_{a,b}\circ\theta_{a,c}$ while the composition going left around the bottom is $\theta_{a,b+c}$. Thus, we wish to show the above diagram commutes.
	
	Regions $(\text A)$, $(\text C)$, and $(\text H)$ commute by coherence for the $\phi$'s (see previous remark). Region $(\text E)$ commutes by coherence for the $\tau$'s. To see region $(\text B)$ commutes, consider the following diagram, which commutes by naturality of $\tau$:
	% https://q.uiver.app/#q=WzAsNixbMCwwLCJTXnstYS1jfVNeYSBTXmMiXSxbMiwwLCJTXnstYS1jfVNeY1NeYSJdLFswLDIsIlNeey1hLWN9U157YS1ifVNeYlNeYyJdLFsyLDIsIlNeey1hLWN9U15jU157YS1ifVNeYiJdLFsyLDQsIlNeey1hLWN9U15jU157LWJ9U157YX1TXmIiXSxbMCw0LCJTXnstYS1jfVNeey1ifVNeYVNeYlNeYyJdLFswLDEsIlNeey1hLWN9XFx0YXUiXSxbMCwyLCJTXnstYS1jfVxccGhpX3thLWIsYn1TXmMiLDJdLFsxLDMsIlNeey1hLWN9U15jXFxwaGlfe2EtYixifSJdLFsyLDMsIlNeey1hLWN9XFx0YXVfe1Nee2EtYn1TXmIsU15jfSJdLFszLDQsIlNeey1hLWN9U15jXFxwaGlfey1iLGF9U15iIl0sWzIsNSwiU157LWEtY31cXHBoaV97LWIsYX1TXmJTXmMiLDJdLFs1LDQsIlNeey1hLWN9XFx0YXVfe1Neey1ifVNeYVNeYixTXmN9Il1d
	\[\begin{tikzcd}
		{S^{-a-c}S^a S^c} && {S^{-a-c}S^cS^a} \\
		\\
		{S^{-a-c}S^{a-b}S^bS^c} && {S^{-a-c}S^cS^{a-b}S^b} \\
		\\
		{S^{-a-c}S^{-b}S^aS^bS^c} && {S^{-a-c}S^cS^{-b}S^{a}S^b}
		\arrow["{S^{-a-c}\tau}", from=1-1, to=1-3]
		\arrow["{S^{-a-c}\phi_{a-b,b}S^c}"', from=1-1, to=3-1]
		\arrow["{S^{-a-c}S^c\phi_{a-b,b}}", from=1-3, to=3-3]
		\arrow["{S^{-a-c}\tau_{S^{a-b}S^b,S^c}}", from=3-1, to=3-3]
		\arrow["{S^{-a-c}S^c\phi_{-b,a}S^b}", from=3-3, to=5-3]
		\arrow["{S^{-a-c}\phi_{-b,a}S^bS^c}"', from=3-1, to=5-1]
		\arrow["{S^{-a-c}\tau_{S^{-b}S^aS^b,S^c}}", from=5-1, to=5-3]
	\end{tikzcd}\]
	To see region $(\text D)$ commutes, note that it is simply the square
	% https://q.uiver.app/#q=WzAsNCxbMiwwLCJTXnstYS1jfVNeey1ifVNeYVNeYlNeYyJdLFswLDAsIlNeey1hLWItY31TXmFTXntiK2N9Il0sWzAsMiwiU157LWEtYi1jfVNee2IrY31TXmEiXSxbMiwyLCJTXnstYS1jfVNeey1ifVNeYlNeY1NeYSJdLFsxLDIsIlNeey1hLWItY31cXHRhdSIsMl0sWzEsMCwiXFxwaGlfey1hLWMsLWJ9U15hXFxwaGlfe2IsY30iXSxbMiwzLCJcXHBoaV97LWEtYywtYn1cXHBoaV97YixjfVNeYSJdLFswLDMsIlNeey1hLWN9U157LWJ9XFx0YXVfe1NeYSxTXmJTXmN9IiwxXV0=
	\[\begin{tikzcd}
		{S^{-a-b-c}S^aS^{b+c}} && {S^{-a-c}S^{-b}S^aS^bS^c} \\
		\\
		{S^{-a-b-c}S^{b+c}S^a} && {S^{-a-c}S^{-b}S^bS^cS^a}
		\arrow["{S^{-a-b-c}\tau}"', from=1-1, to=3-1]
		\arrow["{\phi_{-a-c,-b}S^a\phi_{b,c}}", from=1-1, to=1-3]
		\arrow["{\phi_{-a-c,-b}\phi_{b,c}S^a}", from=3-1, to=3-3]
		\arrow["{S^{-a-c}S^{-b}\tau_{S^a,S^bS^c}}"{description}, from=1-3, to=3-3]
	\end{tikzcd}\]
	This diagram commutes by naturality of $\tau$. To see region $(\text F)$ commutes, consider the following diagram, which commutes by functoriality of $-\otimes-$:
	% https://q.uiver.app/#q=WzAsNixbNCwwLCJTXnstYS1ifVNeYVNeYiJdLFs0LDIsIlNeey1hLWJ9U15iU15hIl0sWzAsMCwiU157LWEtY31TXmNTXnstYn1TXmFTXmIiXSxbMiwyLCJTXnstYS1jfVNee2MtYn1TXmJTXmEiXSxbMiwwLCJTXnstYS1jfVNee2MtYn1TXmFTXmIiXSxbMCwyLCJTXnstYS1jfVNeY1Neey1ifVNeYlNeYSJdLFswLDEsIlNeey1hLWJ9XFx0YXUiXSxbMSwzLCJcXHBoaV97LWEtYyxjLWJ9U15iU15hIiwyXSxbMCw0LCJcXHBoaV97LWEtYyxjLWJ9U15hU15iIiwyXSxbNCwzLCJTXnstYS1jfVNee2MtYn1cXHRhdSJdLFs0LDIsIlNeey1hLWN9XFxwaGlfe2MsLWJ9U15hU15iIiwyXSxbMiw1LCJTXnstYS1jfVNeY1Neey1ifVxcdGF1IiwyXSxbMyw1LCJTXnstYS1jfVxccGhpX3tjLC1ifVNeYlNeYSIsMl1d
	\[\begin{tikzcd}
		{S^{-a-c}S^cS^{-b}S^aS^b} && {S^{-a-c}S^{c-b}S^aS^b} && {S^{-a-b}S^aS^b} \\
		\\
		{S^{-a-c}S^cS^{-b}S^bS^a} && {S^{-a-c}S^{c-b}S^bS^a} && {S^{-a-b}S^bS^a}
		\arrow["{S^{-a-b}\tau}", from=1-5, to=3-5]
		\arrow["{\phi_{-a-c,c-b}S^bS^a}"', from=3-5, to=3-3]
		\arrow["{\phi_{-a-c,c-b}S^aS^b}"', from=1-5, to=1-3]
		\arrow["{S^{-a-c}S^{c-b}\tau}", from=1-3, to=3-3]
		\arrow["{S^{-a-c}\phi_{c,-b}S^aS^b}"', from=1-3, to=1-1]
		\arrow["{S^{-a-c}S^cS^{-b}\tau}"', from=1-1, to=3-1]
		\arrow["{S^{-a-c}\phi_{c,-b}S^bS^a}"', from=3-3, to=3-1]
	\end{tikzcd}\]
	Finally, to see region $(\text G)$ commutes, consider the following diagram:
	% https://q.uiver.app/#q=WzAsNixbMCwwLCJTXnstYS1jfVNeey1ifVNeYlNeY1NeYSJdLFswLDEsIlNeey1hLWN9U1NeY1NeYSJdLFsxLDEsIlNeey1hLWN9U15jU1NeYSJdLFsxLDAsIlNeey1hLWN9U15jU157LWJ9U15iU15hIl0sWzAsMiwiU157LWEtY31TXmNTXmEiXSxbMSwyLCJTXnstYS1jfVNeY1NeYSJdLFswLDMsIlNeey1hLWN9XFx0YXVfe1Neey1ifVNeYixTXmN9U15hIl0sWzEsMCwiU157LWEtY31cXHBoaV97LWIsYn1TXmNTXmEiXSxbMiwzLCJTXnstYS1jfVNeY1xccGhpX3stYixifVNeYSIsMl0sWzEsMiwiU157LWEtY31cXHRhdV97UyxTXmN9IFNeYSJdLFs0LDEsIlNeey1hLWN9XFxwaGlfezAsY31TXmE9U157LWEtY31cXGxhbWJkYV97U15jfV57LTF9U15hIl0sWzUsMiwiU157LWEtY31cXHBoaV97YywwfVNeYT1TXnstYS1jfVNcXHJob197U15jfV57LTF9U15hIiwyXSxbNCw1LCIiLDEseyJsZXZlbCI6Miwic3R5bGUiOnsiaGVhZCI6eyJuYW1lIjoibm9uZSJ9fX1dXQ==
	\[\begin{tikzcd}
		{S^{-a-c}S^{-b}S^bS^cS^a} & {S^{-a-c}S^cS^{-b}S^bS^a} \\
		{S^{-a-c}SS^cS^a} & {S^{-a-c}S^cSS^a} \\
		{S^{-a-c}S^cS^a} & {S^{-a-c}S^cS^a}
		\arrow["{S^{-a-c}\tau_{S^{-b}S^b,S^c}S^a}", from=1-1, to=1-2]
		\arrow["{S^{-a-c}\phi_{-b,b}S^cS^a}", from=2-1, to=1-1]
		\arrow["{S^{-a-c}S^c\phi_{-b,b}S^a}"', from=2-2, to=1-2]
		\arrow["{S^{-a-c}\tau_{S,S^c} S^a}", from=2-1, to=2-2]
		\arrow["{S^{-a-c}\phi_{0,c}S^a=S^{-a-c}\lambda_{S^c}^{-1}S^a}", from=3-1, to=2-1]
		\arrow["{S^{-a-c}\phi_{c,0}S^a=S^{-a-c}S\rho_{S^c}^{-1}S^a}"', from=3-2, to=2-2]
		\arrow[Rightarrow, no head, from=3-1, to=3-2]
	\end{tikzcd}\]
	The top region commutes by naturality of $\tau$, while the bottom region commutes by coherence for a symmetric monoidal category. Thus, we have shown that diagram (\ref{theta_ab_o_theta_ac=theta_ab+c}) commutes, so that $\theta_{a,b}\circ\theta_{a,c}=\theta_{a,b+c}$, as desired. Now, to see that $\theta_{b,a}\cdot\theta_{c,a}=\theta_{b+c,a}$, note that
	\[\theta_{b,a}\cdot\theta_{c,a}\overset{(\ast)}=\theta_{a,b}^{-1}\cdot\theta_{a,c}^{-1}=(\theta_{a,c}\cdot\theta_{a,b})^{-1}=\theta_{a,b+c}^{-1}\overset{(\ast)}=\theta_{b+c,a},\]
	where each occurrence of $(\ast)$ is \autoref{theta_ab.theta_ba=id},
\end{proof}

%\begin{proposition}\label{theta_a,0=theta_0,a=id_S}
	%Given $a\in A$, we have $\theta_{0,a}=\theta_{a,0}=\id_S$.
%\end{proposition}
%\begin{proof}
	%Recall $\theta_{a,0}$ is the composition
	%\[S\xr{\phi_{-a,a}} S^{-a}\otimes S^a\xr{S^{-a}\otimes\phi_{a,0}} S^{-a}\otimes(S^a\otimes S)\xr{S^{-a}\otimes\tau}S^{-a}\otimes(S\otimes S^a)\xr{S^{-a}\otimes\phi_{0,a}^{-1}} S^{-a}\otimes S^a\xr{\phi_{-a,a}^{-1}}S\]
	%By the coherence theorem for symmetric monoidal categories and the fact that $\phi_{a,0}$ and $\phi_{0,a}$ coincide with the unitors, we have that the composition
	%\[S^a\xr{\phi_{a,0}=\rho_{S^a}^{-1}} S^a\otimes S\xr\tau S\otimes S^a\xr{\phi_{0,a}^{-1}=\lambda_{S^a}}S^a\]
	%is precisely the identity map, so by functoriality of $-\otimes-$, we have that $\theta_{a,0}$ is the composition
	%\[S\xr{\phi_{-a,a}}S^{-a}\otimes S^a\xr=S^{-a}\otimes S^{a}\xr{\phi_{-a,a}^{-1}}S,\]
	%so $\theta_{a,0}=\id_S$, meaning
	%\[x\cdot y=y\cdot x\cdot(e\circ\theta_{a,0})=y\cdot x\cdot e=y\cdot x,\]
	%where the last equality follows by the fact that $e$ is the unit for the multiplication on $\pi_\ast(E)$. An entirely analagous argument yields that $\theta_{0,a}=\id_S$.
%\end{proof}

\begin{lemma}\label{bilinear}
	Let $X$ and $Y$ be objects in $\cSH$. Then the $A$-graded pairing
	\[\pi_*(X)\times\pi_*(Y)\to\pi_*(X\otimes Y)\]
	sending $x:S^a\to X$ and $ y:S^b\to Y$ to the composition
	\[S^{a+b}\xr{\phi_{a,b}} S^a\otimes S^b\xr{x\otimes y}X\otimes Y\]
	is additive in each argument.
\end{lemma}
\begin{proof}
	Let $a,b\in A$, and let $x_1,x_2:S^a\to X$ and $ y:S^b\to Y$. Then consider the following diagram
	\[\begin{tikzcd}
		{S^{a+b}} & {S^a\otimes S^b} & {(S^a\oplus S^a)\otimes S^b} \\
		& {(S^a\otimes S^b)\oplus(S^a\otimes S^b)} & {(X\oplus X)\otimes Y} \\
		& {(X\otimes Y)\oplus(X\otimes Y)} & {X\otimes Y}
		\arrow["{\Delta\otimes S^b}", from=1-2, to=1-3]
		\arrow["\Delta"', from=1-2, to=2-2]
		\arrow["{( x_1\oplus x_2)\otimes y}", from=1-3, to=2-3]
		\arrow["{\nabla\otimes Y}", from=2-3, to=3-3]
		\arrow["{( x_1\otimes y)\oplus( x_2\otimes y)}"', from=2-2, to=3-2]
		\arrow["\nabla", from=3-2, to=3-3]
		\arrow["\cong"', from=1-3, to=2-2]
		\arrow["\cong"', from=2-3, to=3-2]
		\arrow["\cong", from=1-1, to=1-2]
	\end{tikzcd}\]
	The isomorphisms are given by the fact that $-\otimes-$ is additive in each variable. Both triangles and the parallelogram commute since $-\otimes-$ is additive. By functoriality of $-\otimes-$, the top composition is $( x_1+ x_2)\cdot y$ and the bottom composition is $ x_1\cdot y+ x_2\cdot y$, so they are equal, as desired. An entirely analagous argument yields that $ x\cdot( y_1+ y_2)= x\cdot y_1+ x\cdot y_2$ for $ x\in\pi_*(X)$ and $ y_1, y_2\in\pi_*(Y)$.
\end{proof}

\begin{lemma}\label{E-module_N_implies_pi*N_is_pi*E_module}
	Let $(E,\mu,e)$ be a monoid object. Then the assignment $\pi_*:(N,\kappa)\mapsto\pi_*(N)$ yields an additive functor from $E\text-\Mod$ to $A$-graded left $\pi_*(E)$-modules. In particular, if $(N,\kappa)$ is a left $E$-module in $\cSH$ then the map
	\[\pi_*(E)\times\pi_*(N)\to\pi_*(N)\]
	sending a class $r:S^a\to E$ and $x:S^b\to N$ to the composition
	\[r\cdot x:S^{a+b}\xr{\phi_{a,b}} S^a\otimes S^b\xr{r\otimes x}E\otimes N\xr\kappa N\]
	endows $\pi_*(N)$ with the structure of an $A$-graded left $\pi_*(E)$-module.
\end{lemma}
\begin{proof}
	First let $(N,\kappa)$ be an $E$-module object. Let $a,b,c\in A$ and $x,x':S^a\to N$, $y:S^b\to E$, and $z, z'\in S^c\to E$. Then by \autoref{A-graded_module}, it suffices to show that
	\begin{enumerate}
		\item $y\cdot( x+ x')= y\cdot x+ y\cdot x'$, 
		\item $( z+ z')\cdot x= z\cdot x+ z'\cdot x$,
		\item $(zy)\cdot  x= z\cdot( y\cdot x)$,
		\item $e\cdot x= x$.
	\end{enumerate}
	The first two axioms follow by \autoref{bilinear}. To see $(3)$, consider the diagram:
	% https://q.uiver.app/#q=WzAsNixbMCwxLCJTXnthK2IrY30iXSxbMSwxLCJTXmNcXG90aW1lcyBTXmJcXG90aW1lcyBTXmEiXSxbMiwxLCJFXFxvdGltZXMgRVxcb3RpbWVzIE4iXSxbMywwLCJFXFxvdGltZXMgTiJdLFszLDEsIk4iXSxbMywyLCJFXFxvdGltZXMgTiJdLFswLDEsIlxcY29uZyJdLFsxLDIsInpcXG90aW1lcyB5XFxvdGltZXMgeCJdLFsyLDMsIkVcXG90aW1lc1xca2FwcGEiXSxbMyw0LCJcXGthcHBhIl0sWzIsNSwiXFxtdVxcb3RpbWVzIE4iLDJdLFs1LDQsIlxca2FwcGEiLDJdXQ==
	\[\begin{tikzcd}
		&&& {E\otimes N} \\
		{S^{a+b+c}} & {S^c\otimes S^b\otimes S^a} & {E\otimes E\otimes N} & N \\
		&&& {E\otimes N}
		\arrow["\cong", from=2-1, to=2-2]
		\arrow["{z\otimes y\otimes x}", from=2-2, to=2-3]
		\arrow["E\otimes\kappa", from=2-3, to=1-4]
		\arrow["\kappa", from=1-4, to=2-4]
		\arrow["{\mu\otimes N}"', from=2-3, to=3-4]
		\arrow["\kappa"', from=3-4, to=2-4]
	\end{tikzcd}\]
	It commutes by coherence for $\kappa$. By functoriality of $-\otimes-$, the two outside compositions equal $z\cdot(y\cdot x)$ on the top and $(z\cdot y)\cdot x$ on the bottom. Hence, they are equal, as desired.

	Next, to see $(4)$, consider the following diagram:
	% https://q.uiver.app/#q=WzAsNCxbMCwwLCJTXmEiXSxbMSwxLCJOIl0sWzIsMCwiTiJdLFsxLDIsIkVcXG90aW1lcyBOIl0sWzEsMiwiIiwxLHsibGV2ZWwiOjIsInN0eWxlIjp7ImhlYWQiOnsibmFtZSI6Im5vbmUifX19XSxbMSwzLCJlXFxvdGltZXMgTiIsMV0sWzAsMiwieCJdLFszLDIsIlxca2FwcGEiLDIseyJjdXJ2ZSI6M31dLFswLDEsIngiLDJdLFswLDMsImVcXG90aW1lcyB4IiwyLHsiY3VydmUiOjN9XV0=
	\[\begin{tikzcd}
		{S^a} && N \\
		& N \\
		& {E\otimes N}
		\arrow[Rightarrow, no head, from=2-2, to=1-3]
		\arrow["{e\otimes N}"{description}, from=2-2, to=3-2]
		\arrow["x", from=1-1, to=1-3]
		\arrow["\kappa"', curve={height=18pt}, from=3-2, to=1-3]
		\arrow["x"', from=1-1, to=2-2]
		\arrow["{e\otimes x}"', curve={height=18pt}, from=1-1, to=3-2]
	\end{tikzcd}\]
	The top triangle commutes by definition. The left triangle commutes by functoriality of $-\otimes-$. The right triangle commutes by unitality of $\kappa$. The top composition is $ x$ while the bottom is $e\cdot x$, thus they are necessarily equal since the diagram commutes.

	Now, we'd like to show that if $f:(N,\kappa)\to(N',\kappa)$ is a homomorphism of left $E$-module objects, then $\pi_*(f):\pi_*(N)\to\pi_*(N')$ is a homomorphism of left $\pi_*(E)$-modules. To see this, let $r:S^a\to E$ in $\pi_a(E)$ and $x,x:S^b\to N$ in $\pi_b(N)$. We'd like to show that $\pi_*(f)(x+x')=\pi_*(f)(x)+\pi_*(f)(x')$ and $\pi_*(f)(r\cdot x)=r\cdot\pi_*(f)(x)$. To see the former, consider the following diagram:
	% https://q.uiver.app/#q=WzAsNixbMCwxLCJTXmEiXSxbMSwxLCJTXmFcXG9wbHVzIFNeYSJdLFsyLDEsIk5cXG9wbHVzIE4iXSxbMywwLCJOJ1xcb3BsdXMgTiciXSxbMywxLCJOJyJdLFszLDIsIk4iXSxbMCwxLCJcXERlbHRhIl0sWzEsMiwieFxcb3BsdXMgeCciXSxbMiwzLCJmXFxvcGx1cyBmIl0sWzMsNCwiXFxuYWJsYSJdLFsyLDUsIlxcbmFibGEiLDJdLFs1LDQsImYiLDJdXQ==
	\[\begin{tikzcd}
		&&& {N'\oplus N'} \\
		{S^a} & {S^a\oplus S^a} & {N\oplus N} & {N'} \\
		&&& N
		\arrow["\Delta", from=2-1, to=2-2]
		\arrow["{x\oplus x'}", from=2-2, to=2-3]
		\arrow["{f\oplus f}", from=2-3, to=1-4]
		\arrow["\nabla", from=1-4, to=2-4]
		\arrow["\nabla"', from=2-3, to=3-4]
		\arrow["f"', from=3-4, to=2-4]
	\end{tikzcd}\]
	It commutes by naturality of $\nabla$ in an additive category. The top composition is $\pi_*(f)(x)+\pi_*(f)(x')$, while the bottom is $\pi_*(f)(x+x')$, so they are equal as desired. To see that $\pi_*(f)(r\cdot x)=r\cdot \pi_*(f)(x)$, consider the following diagram:
	% https://q.uiver.app/#q=WzAsNixbMCwxLCJTXnthK2J9Il0sWzEsMSwiU15iXFxvdGltZXMgU15hIl0sWzIsMSwiRVxcb3RpbWVzIE4iXSxbMywwLCJFXFxvdGltZXMgTiciXSxbMywxLCJOJyJdLFszLDIsIk4iXSxbMCwxLCJcXHBoaV97YixhfSJdLFsxLDIsInJcXG90aW1lcyB4Il0sWzIsMywiRVxcb3RpbWVzIGYiXSxbMyw0LCJcXGthcHBhJyJdLFsyLDUsIlxca2FwcGEiLDJdLFs1LDQsImYiLDJdXQ==
	\[\begin{tikzcd}
		&&& {E\otimes N'} \\
		{S^{a+b}} & {S^b\otimes S^a} & {E\otimes N} & {N'} \\
		&&& N
		\arrow["{\phi_{b,a}}", from=2-1, to=2-2]
		\arrow["{r\otimes x}", from=2-2, to=2-3]
		\arrow["{E\otimes f}", from=2-3, to=1-4]
		\arrow["{\kappa'}", from=1-4, to=2-4]
		\arrow["\kappa"', from=2-3, to=3-4]
		\arrow["f"', from=3-4, to=2-4]
	\end{tikzcd}\]
	It commutes by the fact that $f$ is a homomorphism of left $E$-module objects. The bottom composition is $\pi_*(f)(r\cdot x)$, while the top composition is $r\cdot \pi_*(f)(x)$, so they are equal, as desired.
 
	It remains to show this functor is additive. It suffices to show it preserves the zero object and preserves coproducts. To see the former, note that $\pi_*(0)=[S^*,0]=0$ by definition, since $0$ is terminal. To see the latter, we need to show that given $(N,\kappa),(N',\kappa')\in E\text-\Mod$ that $\pi_*(N)\oplus\pi_*(N')\cong\pi_*(N\oplus N')$, and that the following diagram commutes:
	% https://q.uiver.app/#q=WzAsMyxbMSwxLCJcXHBpXyooTlxcb3BsdXMgTicpIl0sWzAsMCwiXFxwaV8qKE4pIl0sWzAsMSwiXFxwaV8qKE4pXFxvcGx1c1xccGlfKihOJykiXSxbMSwwLCJcXHBpXyooXFxpb3RhX04pIl0sWzIsMCwiXFxjb25nIiwyXSxbMSwyLCJcXGlvdGFfe1xccGlfKihOKX0iLDJdXQ==
	\[\begin{tikzcd}
		{\pi_*(N)} \\
		{\pi_*(N)\oplus\pi_*(N')} & {\pi_*(N\oplus N')}
		\arrow["{\pi_*(\iota_N)}", from=1-1, to=2-2]
		\arrow["\cong"', from=2-1, to=2-2]
		\arrow["{\iota_{\pi_*(N)}}"', from=1-1, to=2-1]
	\end{tikzcd}\]
	Since each $S^a$ is compact, for all $a,b\in A$ we have isomorphisms
	\[\pi_a(N)\oplus\pi_a(N')=[S^a,N]\oplus[S^a,N']\cong[S^a,N\oplus N']=\pi_a(N,N'),\]
	and these combine together to yield $A$-graded isomorphisms $\pi_*(N)\oplus\pi_*(N')\xr\cong\pi_*(N\oplus N')$. To see the above diagram commutes, note that since everything is an $A$-graded homomorphism of $A$-graded abelian groups, it suffices to chase homogeneous elements around to show it commutes. Indeed, it is entirely straightforward, by unravelling definitions, that both compositions around the diagram take a generator $x:S^a\to N$ in $\pi_a(N)$ to the composition
	\[S^a\xr xN\xr{\iota_N}N\oplus N'.\]
	Thus, we have shown that $\pi_*$ preserves all finite coproducts, so it is additive.
\end{proof}

\begin{proposition}[{\cite[Proposition 5.11]{nlab:introduction_to_stable_homotopy_theory_--_1-2}}]\label{module}
	Let $(E,\mu,e)$ be a monoid object in $\cSH$. Then $E_*(-)$ is a functor from $\cSH$ to left $A$-graded $\pi_*(E)$-modules, where given some $X$ in $\cSH$, $E_*(X)$ may be endowed with the structure of a left $A$-graded $\pi_*(E)$-module via the map 
	\[\pi_*(E)\times E_*(X)\to E_*(X)\]
	which given $a,b\in A$, sends $x:S^a\to E$ and $y:S^b\to E\otimes X$ to the composition
	\[x\cdot y:S^{a+b}\cong S^a\otimes S^b\xr{x\otimes y}E\otimes (E\otimes X)\cong (E\otimes E)\otimes X\xr{\mu\otimes X}E\otimes X.\]
	Similarly, the assignment $X\mapsto X_*(E)$ is a functor from $\cSH$ to right $A$-graded $\pi_*(E)$-modules, where the structure map
	\[X_*(E)\times\pi_*(E)\to X_*(E)\]
	sends $x:S^a\to X\otimes E$ and $y:S^b\to E$ to the composition
	\[x\cdot y:S^{a+b}\cong S^a\otimes S^b\xr{x\otimes y}(X\otimes E)\otimes E\cong X\otimes(E\otimes E)\xr{X\otimes\mu}X\otimes E.\]
	Finally, $E_*(E)$ is a $\pi_*(E)$-bimodule, in the sense that the left and right actions of $\pi_*(E)$ are compatible, so that given $y, z\in\pi_*(E)$ and $x\in E_*(E)$, $y\cdot(x\cdot z)=(y\cdot x)\cdot z$.
\end{proposition}
\begin{proof}
	Note that $E_*(X)=\pi_*(E\otimes X)$, so that $E_*(-)$ is the composition of the free $E$-module functor (\autoref{free_module}, which is additive by \autoref{E-Mod,free,forgetful_are_additive}) with the additive functor $\pi_*$ from $E\text-\Mod$ to left $A$-graded $\pi_*(E)$-modules (\autoref{E-module_N_implies_pi*N_is_pi*E_module}). Thus indeed $E_*(X)$ is a left $\pi_*(E)$-module for all $X$, and given $f:X\to Y$ in $\cSH$, $E_*(f)$ is a homomorphism of left $\pi_*(E)$-modules.
	
	Showing that $X_*(E)$ has the structure of a right $\pi_*(E)$-module and that if $f:X\to Y$ is a morphism in $\cSH$ then the map
	\[X_*(E)=[S^*,X\otimes E]\xr{(f\otimes E)_*}[S^*,Y\otimes E]=Y_*(E)\]
	is an $A$-graded homomorphism of right $A$-graded $\pi_*(E)$-modules is entirely analagous.

    It remains to show that $E_*(E)$ is a $\pi_*(E)$-bimodule. Let $x:S^a\to E$, $y:S^b\to E\otimes E$, and $z:S^c\to E$, and consider the following diagram:
	% https://q.uiver.app/#q=WzAsNixbMCwxLCJTXnthK2IrY30iXSxbMSwxLCJTXmFcXG90aW1lcyBTXmJcXG90aW1lcyBTXmMiXSxbMiwxLCJFXFxvdGltZXMgRVxcb3RpbWVzIEVcXG90aW1lcyBFIl0sWzMsMCwiRVxcb3RpbWVzIEVcXG90aW1lcyBFIl0sWzMsMiwiRVxcb3RpbWVzIEVcXG90aW1lcyBFIl0sWzMsMSwiRVxcb3RpbWVzIEUiXSxbMCwxLCJcXGNvbmciXSxbMSwyLCJ4XFxvdGltZXMgeVxcb3RpbWVzIHoiXSxbMiwzLCJcXG11XFxvdGltZXMgRVxcb3RpbWVzIEUiXSxbMiw0LCJFXFxvdGltZXMgRVxcb3RpbWVzIFxcbXUiLDJdLFsyLDUsIlxcbXVcXG90aW1lc1xcbXUiXSxbMyw1LCJFXFxvdGltZXNcXG11Il0sWzQsNSwiXFxtdVxcb3RpbWVzIEUiLDJdXQ==
	\[\begin{tikzcd}
		&&& {E\otimes E\otimes E} \\
		{S^{a+b+c}} & {S^a\otimes S^b\otimes S^c} & {E\otimes E\otimes E\otimes E} & {E\otimes E} \\
		&&& {E\otimes E\otimes E}
		\arrow["\cong", from=2-1, to=2-2]
		\arrow["{x\otimes y\otimes z}", from=2-2, to=2-3]
		\arrow["{\mu\otimes E\otimes E}", from=2-3, to=1-4]
		\arrow["{E\otimes E\otimes \mu}"', from=2-3, to=3-4]
		\arrow["\mu\otimes\mu", from=2-3, to=2-4]
		\arrow["E\otimes\mu", from=1-4, to=2-4]
		\arrow["{\mu\otimes E}"', from=3-4, to=2-4]
	\end{tikzcd}\]
	Commutativity follows by functoriality of $-\otimes-$, which also tells us that the two outside compositions are $(x\cdot y)\cdot z$ (on top) and $x\cdot(y\cdot z)$ (on bottom). Hence they are equal, as desired.
\end{proof}

\begin{lemma}\label{tw^a_isos}
	Let $(E,\mu,e)$ be a monoid object in $\cSH$, $(N,\kappa)$ a left $E$-module, and $a\in A$. Then the assignment
	\[\mathrm{tw}^a:\pi_{*-a}(N)\to\pi_*(\Sigma^aN)\]
	sending $x:S^{b-a}\to N$ to the composition
	\[S^b\xr{\phi_{b-a,a}}S^{b-a}\otimes S^a\xr{x\otimes S^a}N\otimes S^a\xr\tau S^a\otimes N=\Sigma^aN\]
	is an $A$-graded isomorphism of left $A$-graded $\pi_*(E)$-modules (where here $\pi_*(N)$ is a left $\pi_*(E)$-module by \autoref{E-module_N_implies_pi*N_is_pi*E_module}, and $\pi_*(\Sigma^aN)$ has the left $\pi_*(E)$ module given by \autoref{suspension_of_module_object_is_module_object} and \autoref{E-module_N_implies_pi*N_is_pi*E_module}).
\end{lemma}
\begin{proof}
	Unravelling definitions, the map $\mathrm{tw}^a:\pi_{*-a}(N)\to\pi_*(\Sigma^aN)$ factors as
	\[\pi_{*-a}(N)=[S^{*-a},N]\xr{-\otimes S^a}[S^{*-a}\otimes S^a,N\otimes S^a]\xrightarrow{(\phi_{*-a,a})^*}[S^*,N\otimes S^a]\xrightarrow{\tau_*}[S^*,S^a\otimes N]=\pi_*(\Sigma^aN).\]
	The arrow labeled $-\otimes S^a$ is an isomorphism of abelian groups because $-\otimes S^a\cong \Sigma^a$ is an autoequivalence of $\cSH$ (\autoref{Sigma^a,Sigma^-a_adjoint_equiv}). Hence, we have shown the map is an isomorphism of abelian groups. Clearly the map preserves degree, so it is an $A$-graded homomorphism as desired. Finally, it remains to show that this map is a homomorphism of left $\pi_*(E)$-modules, i.e., that given $r:S^b\to E$ in $\pi_*(E)$ and $x:S^{c-a}\to N$ in $\pi_{*-a}(N)$ that $\mathrm{tw}^a(r\cdot x)=r\cdot\mathrm{tw}^a(x)$. Unravelling definitions, $\mathrm{tw}^a(r\cdot x)$ is the composition
	\[S^{b+c}\xr\cong S^b\otimes S^{c-a}\otimes S^a\xrightarrow{r\otimes x\otimes S^a}E\otimes N\otimes S^a\xrightarrow{\kappa\otimes  S^a}N\otimes S^a\xrightarrow\tau S^a\otimes N,\]
	while on the other hand $r\cdot\mathrm{tw}^a(x)$ is the composition
	\[S^{b+c}\xr\cong S^b\otimes S^{c-a}\otimes S^a\xrightarrow{r\otimes x\otimes S^a}E\otimes N\otimes S^a\xrightarrow{E\otimes \tau}E\otimes S^a\otimes N\xrightarrow{\tau\otimes  N}S^a\otimes E\otimes N\xrightarrow{S^a\otimes \kappa}S^a\otimes N.\]
	To see these are equal, consider the following diagram:
	% https://q.uiver.app/#q=WzAsNyxbMCwxLCJTXntiK2N9Il0sWzEsMSwiU15iXFxvdGltZXMgU157Yy1hfVxcb3RpbWVzIFNeYSJdLFsyLDEsIkVcXG90aW1lcyBOXFxvdGltZXMgU15hIl0sWzIsMCwiRVxcb3RpbWVzIFNeYVxcb3RpbWVzIE4iXSxbMywwLCJTXmFcXG90aW1lcyBFXFxvdGltZXMgTiJdLFszLDEsIlNeYVxcb3RpbWVzIE4iXSxbMywyLCJOXFxvdGltZXMgU15hIl0sWzAsMSwiXFxwaGkiXSxbMSwyLCJyXFxvdGltZXMgeFxcb3RpbWVzIFNeYSJdLFszLDQsIlxcdGF1XFxvdGltZXMgTiJdLFs0LDUsIlNeYVxcb3RpbWVzXFxrYXBwYSJdLFsyLDYsIlxca2FwcGFcXG90aW1lcyBTXmEiLDJdLFs2LDUsIlxcdGF1IiwyXSxbMiw0LCJcXHRhdV97RVxcb3RpbWVzIE4sU15hfSIsMV0sWzIsMywiRVxcb3RpbWVzIFxcdGF1Il1d
	\[\begin{tikzcd}
		&& {E\otimes S^a\otimes N} & {S^a\otimes E\otimes N} \\
		{S^{b+c}} & {S^b\otimes S^{c-a}\otimes S^a} & {E\otimes N\otimes S^a} & {S^a\otimes N} \\
		&&& {N\otimes S^a}
		\arrow["\phi", from=2-1, to=2-2]
		\arrow["{r\otimes x\otimes S^a}", from=2-2, to=2-3]
		\arrow["{\tau\otimes N}", from=1-3, to=1-4]
		\arrow["{S^a\otimes\kappa}", from=1-4, to=2-4]
		\arrow["{\kappa\otimes S^a}"', from=2-3, to=3-4]
		\arrow["\tau"', from=3-4, to=2-4]
		\arrow["{\tau_{E\otimes N,S^a}}"{description}, from=2-3, to=1-4]
		\arrow["{E\otimes \tau}", from=2-3, to=1-3]
	\end{tikzcd}\]
	The top triangle commutes by coherence for a symmetric monoidal category, while the right triangle commutes by naturality of $\tau$.
\end{proof}

%\begin{corollary}\label{pi_*:Hom^*(N,N')toHom^*(pi*N,pi*N')}
%	The homomorphisms
%	\[\pi_*:\Hom_{E\text-\Mod}(N,N')\to\Hom_{\pi_*(E)}(\pi_*(N),\pi_*(N'))\]
%	constructed in \autoref{E-module_N_implies_pi*N_is_pi*E_module} extend to $A$-graded homomorphisms of abelian groups
%	\[\pi_*:\Hom_{E\text-\Mod}^*(N,N')\to\Hom_{\pi_*(E)}^*(\pi_*(N),\pi_*(N'))\]
%	as
%	\[\Hom_{E\text-\Mod}(\Sigma^aN,N')\xr{\pi_*}\Hom_{\pi_*(E)}(\pi_*(\Sigma^aN),\pi_*(N'))\xr{(\mathrm{tw}^a)^*}\Hom_{\pi_*(E)}(\pi_{*-a}(N),\pi_*(N')).\]
%	Explicitly, this map sends a class $f:S^a\otimes N\to N'$ to the map $\pi_{*-a}(N)\to\pi_*(N')$ which sends a class $x:S^{b-a}\to N$ to the composition
%	\[S^b\xr{\phi}S^{b-a}\otimes S^a\xr{x\otimes S^a}N\otimes S^a\xr\tau S^a\otimes N\xr fN'.\]
%\end{corollary}

\begin{lemma}\label{pi_*_iso_when_N_retract_of_wedge_of_susps}
	Let $(E,\mu,e)$ be a monoid object and $(N,\kappa)$ a left $E$-module object in $\cSH$. Then given a collection of $a_i\in A$ indexed by some set $I$, if $(N,\kappa)$ is a retract of $\bigoplus_i(E\otimes S^{a_i})$ in $E\text-\Mod$,\footnote{Here $\bigoplus_i(E\otimes S^{a_i})$ is a coproduct (\autoref{coproduct_of_E_modules_is_coproduct_in_E_mod}) of a bunch of left free $E$-module objects (\autoref{free_forgetful_E-Mod}), so it is itself a left $E$-module object.} then for all left $E$-module objects $(N',\kappa')$ in $\cSH$, the functor $\pi_*$ (\autoref{E-module_N_implies_pi*N_is_pi*E_module}) induces an isomorphism
	\[\pi_*:\Hom_{E\text-\Mod}(N,N')\to\Hom_{\pi_*(E)}(\pi_*(N),\pi_*(N')).\]
\end{lemma}
\begin{proof}
	First, we consider the case $N=\bigoplus_i(E\otimes S^{a_i})$. Consider the following diagram:
	% https://q.uiver.app/#q=WzAsOSxbMCwwLCJcXEhvbV97RVxcdGV4dC1cXE1vZH0oXFxiaWdvcGx1c19pIChFXFxvdGltZXMgU157YV9pfSksTicpIl0sWzEsMCwiXFxIb21fe1xccGlfKihFKX0oXFxwaV8qKFxcYmlnb3BsdXNfaSAoRVxcb3RpbWVzIFNee2FfaX0pKSxcXHBpXyooTicpKSJdLFswLDEsIlxccHJvZF9pXFxIb21fe0VcXHRleHQtXFxNb2R9KEVcXG90aW1lcyBTXnthX2l9LE4nKSJdLFswLDIsIlxccHJvZF9pW1Nee2FfaX0sTiddIl0sWzEsMSwiXFxIb21fe1xccGlfKihFKX0oXFxiaWdvcGx1c19pXFxwaV8qKEVcXG90aW1lcyBTXnthX2l9KSxcXHBpXyooTicpKSJdLFsxLDIsIlxccHJvZF9pXFxIb21fe1xccGlfKihFKX0oXFxwaV8qKEVcXG90aW1lcyBTXnthX2l9KSxcXHBpXyooTicpKSJdLFsxLDMsIlxccHJvZF9pXFxIb21fe1xccGlfKihFKX0oXFxwaV97Ki1hX2l9KEUpLFxccGlfKihOJykpIl0sWzEsNCwiXFxwcm9kX2lcXEhvbV57YV9pfV97XFxwaV8qKEUpfShcXHBpX3sqfShFKSxcXHBpXyooTicpKSJdLFswLDQsIlxccHJvZF9pXFxwaV97YV9pfShOJykiXSxbMCwxLCJcXHBpXyoiXSxbMCwyLCJcXGNvbmciLDJdLFsyLDMsIlxcY29uZyIsMl0sWzEsNCwiXFxjb25nIl0sWzQsNSwiXFxjb25nIl0sWzUsNiwiXFxjb25nIl0sWzYsNywiIiwwLHsibGV2ZWwiOjIsInN0eWxlIjp7ImhlYWQiOnsibmFtZSI6Im5vbmUifX19XSxbNyw4LCJcXHByb2RfaVxcbWF0aHJte2V2fV8xIiwyXSxbMyw4LCIiLDIseyJsZXZlbCI6Miwic3R5bGUiOnsiaGVhZCI6eyJuYW1lIjoibm9uZSJ9fX1dXQ==
	\[\begin{tikzcd}
		{\Hom_{E\text-\Mod}(\bigoplus_i (E\otimes S^{a_i}),N')} & {\Hom_{\pi_*(E)}(\pi_*(\bigoplus_i (E\otimes S^{a_i})),\pi_*(N'))} \\
		{\prod_i\Hom_{E\text-\Mod}(E\otimes S^{a_i},N')} & {\Hom_{\pi_*(E)}(\bigoplus_i\pi_*(E\otimes S^{a_i}),\pi_*(N'))} \\
		{\prod_i[S^{a_i},N']} & {\prod_i\Hom_{\pi_*(E)}(\pi_*(E\otimes S^{a_i}),\pi_*(N'))} \\
		& {\prod_i\Hom_{\pi_*(E)}(\pi_{*-a_i}(E),\pi_*(N'))} \\
		{\prod_i\pi_{a_i}(N')} & {\prod_i\Hom^{a_i}_{\pi_*(E)}(\pi_{*}(E),\pi_*(N'))}
		\arrow["{\pi_*}", from=1-1, to=1-2]
		\arrow["\cong"', from=1-1, to=2-1]
		\arrow["\cong"', from=2-1, to=3-1]
		\arrow["\cong", from=1-2, to=2-2]
		\arrow["\cong", from=2-2, to=3-2]
		\arrow["\cong", from=3-2, to=4-2]
		\arrow[Rightarrow, no head, from=4-2, to=5-2]
		\arrow["{\prod_i\mathrm{ev}_1}"', from=5-2, to=5-1]
		\arrow[Rightarrow, no head, from=3-1, to=5-1]
	\end{tikzcd}\]
	Here the top left vertical isomorphism exibits the universal property of the coproduct in $E\text-\Mod$, and middle left vertical isomorphism below that is the free-forgetful adjunction for $E$-modules (\autoref{free_forgetful_E-Mod}). The bottom horizontal isomorphism is the product of the evaluation-at-$1$ isomorphisms (\autoref{ev_at_1_is_iso}). On the other side, the top right vertical isomorphism is given by the fact that $S^a$ is compact for each $a\in A$, so we have isomorphisms
	\[\bigoplus_i\pi_*(E\otimes S^{a_i})=\bigoplus_{a\in A}\bigoplus_{i}[S^a,E\otimes S^{a_i}]\cong\bigoplus_{a\in A}[S^a,\bigoplus_{i}(E\otimes S^{a_i})]=\pi_*(\bigoplus_i(E\otimes S^{a_i})),\]
	where the middle isomorphism takes a generator $x:S^a\xr E\otimes S^{a_i}$ to the composition $S^a\xr xE\otimes S^{a_i}\into\bigoplus_i(E\otimes S^{a_i})$. The middle right vertical isomorphism exhibits the universal property of the coproduct of modules. Finally the bottom right vertical isomorphism is given by the isomorphisms
	\[\pi_{*-a_i}(E\otimes S^{a_i})=[S^{*-a_i},E\otimes S^{a_i}]\xr{-\otimes S^{a_i}}[S^{*-a_i}\otimes S^{a_i},E\otimes S^{a_i}]\xr{\phi^*}[S^*,E\otimes S^{a_i}]=\pi_*(E\otimes S^{a_i}),\]
	where $-\otimes S^{a_i}\cong\Sigma^{a_i}$ is an isomorphism by \autoref{Sigma^a,Sigma^-a_adjoint_equiv}. Now, we claim this diagram commutes. This really simply amounts to unravelling definitions, and chasing a homomorphism $f:\bigoplus_i(E\otimes S^{a_i})\to N'$ of left $E$-module objects both ways around the diagram yields the composition
	\[\prod_i(S^{a_i}\xr{e\otimes S^{a_i}}E\otimes S^{a_i}\into\bigoplus_i(E\otimes S^{a_i})\xr fN').\]
	Thus, since the diagram commutes, we have that 
	\[\pi_*:\Hom_{E\text-\Mod}(\bigoplus_i(E\otimes S^{a_i}),N')\to\Hom_{\pi_*(E)}(\pi_*(\bigoplus_i(E\otimes S^{a_i})),\pi_*(N'))\] is an isomorphism, as desired.

	Now, consider the case that $N$ is a retract of $\bigoplus_i(E\otimes S^{a_i})$ in $E\text-\Mod$, so there exists a commuting diagram of left $E$-module object homomorphisms:
	% https://q.uiver.app/#q=WzAsMyxbMCwwLCJOIl0sWzEsMCwiXFxiaWdvcGx1c19pKEVcXG90aW1lcyBTXnthX2l9KSJdLFsyLDAsIk4iXSxbMCwxLCJcXGlvdGEiLDJdLFsxLDIsInIiLDJdLFswLDIsIiIsMix7ImN1cnZlIjotNCwibGV2ZWwiOjIsInN0eWxlIjp7ImhlYWQiOnsibmFtZSI6Im5vbmUifX19XV0=
	\[\begin{tikzcd}
		N & {\bigoplus_i(E\otimes S^{a_i})} & N
		\arrow["\iota"', from=1-1, to=1-2]
		\arrow["r"', from=1-2, to=1-3]
		\arrow[curve={height=-24pt}, Rightarrow, no head, from=1-1, to=1-3]
	\end{tikzcd}\]
	Now consider the following diagram:
	% https://q.uiver.app/#q=WzAsNixbMiwwLCJcXEhvbV4qX3tFXFx0ZXh0LVxcTW9kfShOLE4nKSJdLFsxLDAsIlxcSG9tX3tFXFx0ZXh0LVxcTW9kfV4qKFxcYmlnb3BsdXNfaShFXFxvdGltZXMgU157YV9pfSksTicpIl0sWzAsMCwiXFxIb21eKl97RVxcdGV4dC1cXE1vZH0oTixOJykiXSxbMSwxLCJcXEhvbV97XFxwaV8qKEUpfV4qKFxccGlfKihcXGJpZ29wbHVzX2koRVxcb3RpbWVzIFNee2FfaX0pKSxcXHBpXyooTicpKSJdLFswLDEsIlxcSG9tXipfe1xccGlfKihFKX0oXFxwaV8qKE4pLFxccGlfKihOJykpIl0sWzIsMSwiXFxIb21eKl97XFxwaV8qKEUpfShcXHBpXyooTiksXFxwaV8qKE4nKSkiXSxbMiwxLCJyXioiXSxbMSwwLCJcXGlvdGFeKiJdLFsyLDAsIiIsMix7ImN1cnZlIjotNCwibGV2ZWwiOjIsInN0eWxlIjp7ImhlYWQiOnsibmFtZSI6Im5vbmUifX19XSxbMSwzLCJcXHBpXyoiXSxbMiw0LCJcXHBpXyoiLDJdLFs0LDMsIihcXHBpXyoocikpXioiXSxbMyw1LCIoXFxwaV8qKFxcaW90YSkpXioiXSxbMCw1LCJcXHBpXyoiLDJdLFs0LDUsIiIsMix7ImN1cnZlIjo0LCJsZXZlbCI6Miwic3R5bGUiOnsiaGVhZCI6eyJuYW1lIjoibm9uZSJ9fX1dXQ==
	\[\begin{tikzcd}[column sep=small]
		{\Hom^*_{E\text-\Mod}(N,N')} & {\Hom_{E\text-\Mod}^*(\bigoplus_i(E\otimes S^{a_i}),N')} & {\Hom^*_{E\text-\Mod}(N,N')} \\
		{\Hom^*_{\pi_*(E)}(\pi_*(N),\pi_*(N'))} & {\Hom_{\pi_*(E)}^*(\pi_*(\bigoplus_i(E\otimes S^{a_i})),\pi_*(N'))} & {\Hom^*_{\pi_*(E)}(\pi_*(N),\pi_*(N'))}
		\arrow["{r^*}", from=1-1, to=1-2]
		\arrow["{\iota^*}", from=1-2, to=1-3]
		\arrow[curve={height=-24pt}, Rightarrow, no head, from=1-1, to=1-3]
		\arrow["{\pi_*}", from=1-2, to=2-2]
		\arrow["{\pi_*}"', from=1-1, to=2-1]
		\arrow["{(\pi_*(r))^*}", from=2-1, to=2-2]
		\arrow["{(\pi_*(\iota))^*}", from=2-2, to=2-3]
		\arrow["{\pi_*}"', from=1-3, to=2-3]
		\arrow[curve={height=24pt}, Rightarrow, no head, from=2-1, to=2-3]
	\end{tikzcd}\]
	Each square commutes by functoriality of $\pi_*$. We have shown the middle vertical arrow is an isomorphism. Thus the outside arrows are isomorphisms as well, as a retract of an isomorphism is an isomorphism.
\end{proof}

\begin{proposition}\label{UCT_for_retract}
	Let $(E,\mu,e)$ be a monoid object and $X$ an object in $\cSH$. If there is a collection of $a_i\in A$ indexed by some set $I$ such that $E\otimes X$ is a retract of $\bigoplus_i(E\otimes S^{a_i})$ in $E\text-\Mod$,\footnote{Here $\bigoplus_i(E\otimes S^{a_i})$ is a coproduct (\autoref{coproduct_of_E_modules_is_coproduct_in_E_mod}) of a bunch of left free $E$-module objects (\autoref{free_forgetful_E-Mod}), so it is itself a left $E$-module object.} then for all left $E$-module objects $(N,\kappa)$ the assignment
	\[\Psi:[X,N]_*\to\Hom_{\pi_*(E)}^*(E_*(X),\pi_*(N))\]
	sending $f:S^a\otimes X\to N$ to the map $E_{*-a}(X)\to\pi_*(N)$ which sends a class $x:S^{b-a}\to E\otimes X$ to the composition
	\[\Psi(f)(x):S^b\xr\phi S^{b-a}\otimes S^a\xrightarrow{x\otimes S^a}E\otimes X\otimes S^a\xr{E\otimes\tau}E\otimes S^a\otimes X\xr{E\otimes f}E\otimes N\xr\kappa N\]
	is an $A$-graded isomorphism of $A$-graded abelian groups.
\end{proposition}
\begin{proof}
	Clearly as constructed, assuming $\Psi(f)$ as defined is actually a homomorphism of left $\pi_*(E)$-modules, this map is $A$-graded. Thus, it suffices to show that for all $a\in A$, the restriction
	\[\Psi_a:[X,N]_a\to\Hom_{\pi_*(E)}^a(E_*(X),\pi_*(N))\]
	is an isomorphism. First of all, note that $\Psi_a$ factors as
	% https://q.uiver.app/#q=WzAsNyxbMSwwLCJbXFxTaWdtYV5hWCxOXSJdLFsxLDEsIlxcSG9tX3tFXFx0ZXh0LVxcTW9kfShFXFxvdGltZXNcXFNpZ21hXmFYLE4pIl0sWzEsMiwiXFxIb21fe0VcXHRleHQtXFxNb2R9KFxcU2lnbWFeYShFXFxvdGltZXMgWCksTikiXSxbMSwzLCJcXEhvbV97XFxwaV8qKEUpfShcXHBpXyooXFxTaWdtYV5hKEVcXG90aW1lcyBYKSksXFxwaV8qKE4pKSJdLFsxLDQsIlxcSG9tX3tcXHBpXyooRSl9KEVfeyotYX0oWCksXFxwaV8qKE4pKSJdLFswLDAsIltYLE5dX2EiXSxbMiw0LCJcXEhvbV97XFxwaV8qKEUpfV5hKEVfKihYKSxcXHBpXyooTikpIl0sWzAsMSwiXFxjb25nIl0sWzEsMiwiXFxjb25nIl0sWzIsMywiXFxwaV8qIl0sWzMsNCwiKFxcbWF0aHJte3R3fV5hKV4qIl0sWzUsMCwiIiwwLHsibGV2ZWwiOjIsInN0eWxlIjp7ImhlYWQiOnsibmFtZSI6Im5vbmUifX19XSxbNCw2LCIiLDAseyJsZXZlbCI6Miwic3R5bGUiOnsiaGVhZCI6eyJuYW1lIjoibm9uZSJ9fX1dXQ==
	\[\begin{tikzcd}
		{[X,N]_a} & {[\Sigma^aX,N]} \\
		& {\Hom_{E\text-\Mod}(E\otimes\Sigma^aX,N)} \\
		& {\Hom_{E\text-\Mod}(\Sigma^a(E\otimes X),N)} \\
		& {\Hom_{\pi_*(E)}(\pi_*(\Sigma^a(E\otimes X)),\pi_*(N))} \\
		& {\Hom_{\pi_*(E)}(E_{*-a}(X),\pi_*(N))} & {\Hom_{\pi_*(E)}^a(E_*(X),\pi_*(N))}
		\arrow["\cong", from=1-2, to=2-2]
		\arrow["\cong", from=2-2, to=3-2]
		\arrow["{\pi_*}", from=3-2, to=4-2]
		\arrow["{(\mathrm{tw}^a)^*}", from=4-2, to=5-2]
		\arrow[Rightarrow, no head, from=1-1, to=1-2]
		\arrow[Rightarrow, no head, from=5-2, to=5-3]
	\end{tikzcd}\]
	where the first isomorphism is the free-forgetful adjunction for $E$-modules (\autoref{free_forgetful_E-Mod}), the second isomorphism is given by \autoref{free_susp_is_susp_of_free}, the third map is that induced by the functor $\pi_*$ constructed in \autoref{E-module_N_implies_pi*N_is_pi*E_module}, and the final map is induced by the isomorphism $\mathrm{tw}^a:\pi_{*-a}(E\otimes X)\xr\cong\pi_*(\Sigma^a(E\otimes X))$ (\autoref{tw^a_isos}). Unravelling definitions, this composition sends a class $f:S^a\otimes X\to N$ to the map $E_{*-a}(X)\to\pi_*(N)$ which sends a class $x:S^{b-a}\to E\otimes X$ to the composition
	\[S^b\xrightarrow\cong S^{b-a}\otimes S^a\xrightarrow{x\otimes S^a}E\otimes X\otimes S^a\xrightarrow{\tau_{E\otimes X,S^a}}S^a\otimes E\otimes X\xrightarrow{\tau\otimes X}E\otimes S^a\otimes X\xrightarrow{E\otimes f}E\otimes N\xrightarrow{\kappa}N,\]
	and clearly this equals $\Psi(f)(x)$, by coherence for the symmetries. Thus, it suffices to show that
	\[\pi_*:\Hom_{E\text-\Mod}(\Sigma^a(E\otimes X),N)\to\Hom_{\pi_*(E)}(\pi_{*}(\Sigma^a(E\otimes X)),\pi_*(N))\]
	is an isomorphism when $E\otimes X$ is a retract of $\bigoplus_i(E\otimes S^{a_i})$ in $E\text-\Mod$. This is precisely \autoref{pi_*_iso_when_N_retract_of_wedge_of_susps}.
\end{proof}

\begin{proposition}\label{if_pi_*N_graded_proj_then_retract_of_wedge_of_susps}
	Let $(E,\mu,e)$ be a monoid object and $(N,\kappa)$ a left $E$-module object in $\cSH$. Further suppose that $E$ and $N$ are cellular and that $\pi_*(N)$ is a \emph{graded projective} (\autoref{graded_projective_module}) left $\pi_*(E)$-module (\autoref{E-module_N_implies_pi*N_is_pi*E_module}). Then given some homogeneous generating set $\{x_i\}_{i\in I}\sseq \pi_*(N)$, $N$ is a retract of $\bigoplus_i(E\otimes S^{|x_i|})$ in $E\text-\Mod$.\footnote{Here $\bigoplus_i(E\otimes S^{a_i})$ is a coproduct (\autoref{coproduct_of_E_modules_is_coproduct_in_E_mod}) of a bunch of left free $E$-module objects (\autoref{free_forgetful_E-Mod}), so it is itself a left $E$-module object.}
\end{proposition}
\begin{proof}
	Let $M:=\bigoplus_i(E\otimes S^{|x_i|})$. We have a map
	\[r:M\to N\]
	induced by the maps
	\[r_i:E\otimes S^{|x_i|}\xrightarrow{E\otimes x_i}E\otimes N\xr{\kappa} N.\]
	This is a homomorphism of left $E$-module objects:
	% https://q.uiver.app/#q=WzAsOCxbMCwwLCJFXFxvdGltZXNcXGJpZ29wbHVzX2koRVxcb3RpbWVzIFNee3x4X2l8fSkiXSxbMiwwLCJFXFxvdGltZXMgTiJdLFsyLDQsIk4iXSxbMCwyLCJcXGJpZ29wbHVzX2koRVxcb3RpbWVzIEVcXG90aW1lcyBTXnt8eF9pfH0pIl0sWzAsNCwiXFxiaWdvcGx1c19pKEVcXG90aW1lcyBTXnt8eF9pfH0pIl0sWzEsMiwiXFxiaWdvcGx1c19pKEVcXG90aW1lcyBOKSJdLFsxLDMsIlxcYmlnb3BsdXNfaSBOIl0sWzEsMSwiRVxcb3RpbWVzXFxiaWdvcGx1c19pTiJdLFswLDEsIkVcXG90aW1lcyByIl0sWzEsMiwiXFxrYXBwYSJdLFswLDMsIlxcY29uZyIsMl0sWzMsNCwiXFxiaWdvcGx1c19pKFxcbXVcXG90aW1lcyBTXnt8eF9pfH0pIiwyXSxbNCwyLCJyIl0sWzMsNSwiXFxiaWdvcGx1c19pKEVcXG90aW1lcyByX2kpIl0sWzUsNiwiXFxiaWdvcGx1c19pXFxrYXBwYSJdLFs2LDIsIlxcbmFibGEiXSxbNCw2LCJcXGJpZ29wbHVzX2lyX2kiXSxbMCw3LCJFXFxvdGltZXNcXGJpZ29wbHVzX2lyX2kiLDFdLFs3LDUsIlxcY29uZyIsMl0sWzcsMSwiRVxcb3RpbWVzXFxuYWJsYSIsMV0sWzUsMSwiXFxuYWJsYSIsMl1d
	\[\begin{tikzcd}
		{E\otimes\bigoplus_i(E\otimes S^{|x_i|})} && {E\otimes N} \\
		& {E\otimes\bigoplus_iN} \\
		{\bigoplus_i(E\otimes E\otimes S^{|x_i|})} & {\bigoplus_i(E\otimes N)} \\
		& {\bigoplus_i N} \\
		{\bigoplus_i(E\otimes S^{|x_i|})} && N
		\arrow["{E\otimes r}", from=1-1, to=1-3]
		\arrow["\kappa", from=1-3, to=5-3]
		\arrow["\cong"', from=1-1, to=3-1]
		\arrow["{\bigoplus_i(\mu\otimes S^{|x_i|})}"', from=3-1, to=5-1]
		\arrow["r", from=5-1, to=5-3]
		\arrow["{\bigoplus_i(E\otimes r_i)}", from=3-1, to=3-2]
		\arrow["{\bigoplus_i\kappa}", from=3-2, to=4-2]
		\arrow["\nabla", from=4-2, to=5-3]
		\arrow["{\bigoplus_ir_i}", from=5-1, to=4-2]
		\arrow["{E\otimes\bigoplus_ir_i}"{description}, from=1-1, to=2-2]
		\arrow["\cong"', from=2-2, to=3-2]
		\arrow["E\otimes\nabla"{description}, from=2-2, to=1-3]
		\arrow["\nabla"', from=3-2, to=1-3]
	\end{tikzcd}\]
	The right trapezoid commutes by naturality of $\nabla$. The bottom triangle commutes by the fact that $\nabla\circ\bigoplus_ir_i$ and $r$ satisfy the same unviersal property for the coproduct. Every other region commutes by additivity of $E\otimes-$, except the left trapezoid: Note that by expanding out how $r_i$ is defined, it becomes
	% https://q.uiver.app/#q=WzAsNixbMCwwLCJcXGJpZ29wbHVzX2koRVxcb3RpbWVzIEVcXG90aW1lcyBTXnt8eF9pfH0pIl0sWzQsMCwiXFxiaWdvcGx1c19pKEVcXG90aW1lcyBFXFxvdGltZXMgWCkiXSxbMiwwLCJcXGJpZ29wbHVzX2koRVxcb3RpbWVzIEVcXG90aW1lcyBOKSJdLFs0LDEsIlxcYmlnb3BsdXNfaShFXFxvdGltZXMgWCkiXSxbMCwxLCJcXGJpZ29wbHVzX2koRVxcb3RpbWVzIFNee3x4X2l8fSkiXSxbMiwxLCJcXGJpZ29wbHVzX2koRVxcb3RpbWVzIE4pIl0sWzAsMiwiXFxiaWdvcGx1c19pKEVcXG90aW1lcyBFXFxvdGltZXMgeF9pKSJdLFsyLDEsIlxcYmlnb3BsdXNfaShFXFxvdGltZXNcXGthcHBhKSJdLFsxLDMsIlxcYmlnb3BsdXNfaVxca2FwcGEiXSxbMCw0LCJcXGJpZ29wbHVzX2koXFxtdVxcb3RpbWVzIFNee3x4X2l8fSkiLDJdLFs0LDUsIlxcYmlnb3BsdXNfaShFXFxvdGltZXMgeF9pKSIsMl0sWzUsMywiXFxiaWdvcGx1c19pXFxrYXBwYSIsMl0sWzIsNSwiXFxiaWdvcGx1c19pKFxcbXVcXG90aW1lcyBYKSJdXQ==
	\[\begin{tikzcd}
		{\bigoplus_i(E\otimes E\otimes S^{|x_i|})} && {\bigoplus_i(E\otimes E\otimes N)} && {\bigoplus_i(E\otimes E\otimes X)} \\
		{\bigoplus_i(E\otimes S^{|x_i|})} && {\bigoplus_i(E\otimes N)} && {\bigoplus_i(E\otimes X)}
		\arrow["{\bigoplus_i(E\otimes E\otimes x_i)}", from=1-1, to=1-3]
		\arrow["{\bigoplus_i(E\otimes\kappa)}", from=1-3, to=1-5]
		\arrow["{\bigoplus_i\kappa}", from=1-5, to=2-5]
		\arrow["{\bigoplus_i(\mu\otimes S^{|x_i|})}"', from=1-1, to=2-1]
		\arrow["{\bigoplus_i(E\otimes x_i)}"', from=2-1, to=2-3]
		\arrow["{\bigoplus_i\kappa}"', from=2-3, to=2-5]
		\arrow["{\bigoplus_i(\mu\otimes X)}", from=1-3, to=2-3]
	\end{tikzcd}\]
	The left square commutes by functoriality of $-\otimes-$, and the right square commutes by coherence for $\kappa$. Hence, we've shown that $r$ is a homomorphism of left $E$-modules, as desired. Thus, $r$ induces a homomorphism of left $\pi_*(E)$-modules $\pi_*(r)\in\Hom_{\pi_*(E)}(\pi_*(M),\pi_*(N)))$ Further note that for all $i\in I$, $x_i$ is in the image of $\pi_*(r)$, as by definition $\pi_*(r)$ sends the class 
	\[S^{|x_i|}\xr{e\otimes S^{|x_i|}}E\otimes S^{|x_i|}\into M\]
	in $\pi_{|x_i|}(M)$ to the composition
	\[S^{|x_i|}\xr{e\otimes S^{|x_i|}}E\otimes S^{|x_i|}\xr{E\otimes x_i}E\otimes N\xr\kappa N,\]
	and by unitality of $\kappa$ this composition is simply $x_i:S^{|x_i|}\to N$. Thus, we have constructed a surjective $A$-graded homomorphism $\pi_*(r):\pi_*(M)\to \pi_*(N)$ of left $\pi_*(E)$-modules, so that since $\pi_*(N)$ is projective graded module there exists an $A$-graded left $\pi_*(E)$-module homomorphism $\iota:\pi_*(N)\to\pi_*(M)$ which makes the following diagram commute:
	% https://q.uiver.app/#q=WzAsMyxbMCwxLCJcXHBpXyooTikiXSxbMSwxLCJcXHBpXyooTikiXSxbMSwwLCJcXHBpXyooTSkiXSxbMCwxLCIiLDAseyJsZXZlbCI6Miwic3R5bGUiOnsiaGVhZCI6eyJuYW1lIjoibm9uZSJ9fX1dLFsyLDEsIlxccGlfKihyKSJdLFswLDIsIlxcaW90YSJdXQ==
	\[\begin{tikzcd}
		& {\pi_*(M)} \\
		{\pi_*(N)} & {\pi_*(N)}
		\arrow[Rightarrow, no head, from=2-1, to=2-2]
		\arrow["{\pi_*(r)}", from=1-2, to=2-2]
		\arrow["\iota", from=2-1, to=1-2]
	\end{tikzcd}\]
	which further induces the corresponding idempotent of left $\pi_*(E)$-modules:
	% https://q.uiver.app/#q=WzAsMyxbMCwwLCJcXHBpXyooTSkiXSxbMSwwLCJcXHBpXyooTikiXSxbMiwwLCJcXHBpXyooTSkiXSxbMCwxLCJcXHBpXyoocikiXSxbMSwyLCJcXGlvdGEiXV0=
	\[\begin{tikzcd}
		{\pi_*(M)} & {\pi_*(N)} & {\pi_*(M)}
		\arrow["{\pi_*(r)}", from=1-1, to=1-2]
		\arrow["\iota", from=1-2, to=1-3]
	\end{tikzcd}\]
	Now, by \autoref{pi_*_iso_when_N_retract_of_wedge_of_susps}, since $M=\bigoplus_i(E\otimes S^{|x_i|})$, we have that this map is actually induced by some endomorphism $\ell:M\to M$ of left $E$-module objects. Now $\ell$ splits by \autoref{idempotent_splits_in_tri_cat_with_countable_coproducts}, meaning there exists a diagram in $\cSH$ of the form
	\[\ell:M\xr{r'}X\xr{\iota'}M\]
	with $r'\circ \iota'=\id_X$. Note that since $E$ and each $S^{|x_i|}$ are cellular, $E\otimes S^{|x_i|}$ is cellular for all $i\in I$ (\autoref{cellular_closed_under_tensor}), so that $M=\bigoplus_i(E\otimes S^{|x_i|})$ is cellular, as by definition an arbitrary coproduct of cellular objects is cellular. Thus by \autoref{cellular_idempotent_splits_cellularly} $X$ is cellular as well. Now consider the following commutative diagram
	% https://q.uiver.app/#q=WzAsOSxbMywxLCJcXHBpXyooTSkiXSxbMiwyLCJcXHBpXyooWCkiXSxbNCwwLCJcXHBpXyooTikiXSxbNCwyLCJcXHBpXyooWCkiXSxbNSwxLCJcXHBpXyooTSkiXSxbNiwxLCJcXHBpXyooWCkiXSxbMCwxLCJcXHBpXyooTikiXSxbMSwxLCJcXHBpXyooTSkiXSxbMiwwLCJcXHBpXyooTikiXSxbMSwwLCJcXHBpXyooXFxpb3RhJykiLDJdLFswLDIsIlxccGlfKihyKSJdLFswLDMsIlxccGlfKihyJykiLDJdLFszLDQsIlxccGlfKihcXGlvdGEnKSIsMl0sWzIsNCwiXFxpb3RhIl0sWzQsNSwiXFxwaV8qKHInKSJdLFszLDUsIiIsMSx7ImN1cnZlIjozLCJsZXZlbCI6Miwic3R5bGUiOnsiaGVhZCI6eyJuYW1lIjoibm9uZSJ9fX1dLFs2LDcsIlxcaW90YSIsMl0sWzcsMSwiXFxwaV8qKHInKSIsMl0sWzcsOCwiXFxwaV8qKHIpIl0sWzgsMCwiXFxpb3RhIl0sWzYsOCwiIiwwLHsiY3VydmUiOi0zLCJsZXZlbCI6Miwic3R5bGUiOnsiaGVhZCI6eyJuYW1lIjoibm9uZSJ9fX1dLFsxLDMsIiIsMCx7ImxldmVsIjoyLCJzdHlsZSI6eyJoZWFkIjp7Im5hbWUiOiJub25lIn19fV0sWzgsMiwiIiwwLHsibGV2ZWwiOjIsInN0eWxlIjp7ImhlYWQiOnsibmFtZSI6Im5vbmUifX19XSxbNywwLCJcXHBpXyooXFxlbGwpIl0sWzAsNCwiXFxwaV8qKFxcZWxsKSJdXQ==
	\[\begin{tikzcd}
		&& {\pi_*(N)} && {\pi_*(N)} \\
		{\pi_*(N)} & {\pi_*(M)} && {\pi_*(M)} && {\pi_*(M)} & {\pi_*(X)} \\
		&& {\pi_*(X)} && {\pi_*(X)}
		\arrow["{\pi_*(\iota')}"', from=3-3, to=2-4]
		\arrow["{\pi_*(r)}", from=2-4, to=1-5]
		\arrow["{\pi_*(r')}"', from=2-4, to=3-5]
		\arrow["{\pi_*(\iota')}"', from=3-5, to=2-6]
		\arrow["\iota", from=1-5, to=2-6]
		\arrow["{\pi_*(r')}", from=2-6, to=2-7]
		\arrow[curve={height=18pt}, Rightarrow, no head, from=3-5, to=2-7]
		\arrow["\iota"', from=2-1, to=2-2]
		\arrow["{\pi_*(r')}"', from=2-2, to=3-3]
		\arrow["{\pi_*(r)}", from=2-2, to=1-3]
		\arrow["\iota", from=1-3, to=2-4]
		\arrow[curve={height=-18pt}, Rightarrow, no head, from=2-1, to=1-3]
		\arrow[Rightarrow, no head, from=3-3, to=3-5]
		\arrow[Rightarrow, no head, from=1-3, to=1-5]
		\arrow["{\pi_*(\ell)}", from=2-2, to=2-4]
		\arrow["{\pi_*(\ell)}", from=2-4, to=2-6]
	\end{tikzcd}\]
	From this diagram we read off that the middle diagonal composition
	\[\pi_*(X)\xr{\pi_*(\iota')}\pi_*(M)\xr{\pi_*(r)}\pi_*(N)\]
	is an isomorphism with inverse $\pi_*(r')\circ\iota$. Now, since $X$ and $N$ are cellular, and $\pi_*(r\circ\iota')$ is an isomorphism, by \autoref{cellular_closed_under_iso} we have that $r\circ\iota'$ is an isomorphism, say with inverse $p$. Thus we have a commuting diagram
	% https://q.uiver.app/#q=WzAsNCxbMiwwLCJNIl0sWzQsMCwiTiJdLFswLDAsIk4iXSxbMSwxLCJYIl0sWzAsMSwiciIsMl0sWzIsMCwiXFxpb3RhJ1xcY2lyYyBwIiwyXSxbMiwzLCJwIiwyXSxbMywwLCJcXGlvdGEnIiwyXSxbMiwxLCIiLDAseyJjdXJ2ZSI6LTQsImxldmVsIjoyLCJzdHlsZSI6eyJoZWFkIjp7Im5hbWUiOiJub25lIn19fV1d
	\[\begin{tikzcd}
		N && M && N \\
		& X
		\arrow["r"', from=1-3, to=1-5]
		\arrow["{\iota'\circ p}"', from=1-1, to=1-3]
		\arrow["p"', from=1-1, to=2-2]
		\arrow["{\iota'}"', from=2-2, to=1-3]
		\arrow[curve={height=-24pt}, Rightarrow, no head, from=1-1, to=1-5]
	\end{tikzcd}\]
	and the middle row exhibits $N$ as a retract of $M=\bigoplus_i(E\otimes S^{|x_i|})$, as desired.
\end{proof}

\begin{corollary}\label{UCT_for_graded_projective}
	Let $(E,\mu,e)$ be a monoid object and let $X$ and $Y$ be objects in $\cSH$. Then if $E$ and $X$ are cellular and $E_*(X)$ is a graded projective (\autoref{graded_projective_module}) left $\pi_*(E)$-module (\autoref{module}), then the map
	\[\Psi_{X,Y}:[X,E\otimes Y]_*\to\Hom_{\pi_*(E)}^*(E_*(X),E_*(Y))\]
	sending $f:S^a\otimes X\to E\otimes Y$ to the map $E_{*-a}(X)\to E_*(Y)$ which sends a class $x:S^{b-a}\to E\otimes X$ to the composition
	\[\Psi_{X,Y}(f)(x):S^b\xr\phi S^{b-a}\otimes S^a\xr{x\otimes S^a}E\otimes X\otimes S^a\xr{E\otimes \tau}E\otimes S^a\otimes X\xr{E\otimes f}E\otimes E\otimes Y\xr{\mu\otimes Y}E\otimes Y\]
	is an $A$-graded isomorphism of $A$-graded abelian groups.
\end{corollary}
\begin{proof}
	By \autoref{if_pi_*N_graded_proj_then_retract_of_wedge_of_susps}, since $E\otimes X$ is a left $E$-module object (\autoref{free_module}), $E_*(X)=\pi_*(E\otimes X)$ is a graded projective left $\pi_*(E)$-module, and $E\otimes X$ is cellular (\autoref{cellular_closed_under_tensor}), it follows that $E\otimes X$ is a retract of $\bigoplus_i(E\otimes S^{a_i})$ in $E\text-\Mod$ for some collection of $a_i\in A$ indexed by some set $I$. Thus the desired result follows by \autoref{UCT_for_retract} with $N=E\otimes Y$ (which is an $E$-module by \autoref{free_module}).
\end{proof}


\begin{proposition}[{\cite[Proposition 2.2]{nlab:introduction_to_the_adams_spectral_sequence}}]\label{Kunneth_map}
	Let $(E,\mu,e)$ be a monoid object in $\cSH$ and let $X$ be any object. Then the assignment
	\[E_*(E)\times E_*(X)\to E_*(E\otimes X)\]
	which sends $x:S^{a}\to E\otimes E$ and $ y:S^{b}\to E\otimes X$ to the composition
	\[x\cdot y:S^{a+b}\cong S^{a}\otimes S^{b}\xr{x\otimes y}E\otimes E\otimes E\otimes X\xr{E\otimes\mu\otimes X}E\otimes E\otimes X\]
	lifts to an $A$-graded homomorphism of left $A$-graded $\pi_*(E)$-modules
	\[\Phi_X:E_*(E)\otimes_{\pi_*(E)}E_*(X)\to E_*(E\otimes X)\]
	(where here $E_*(E)$ has a $\pi_*(E)$-bimodule structure and $E_*(X)$ has a left $\pi_*(E)$-module structure as specified by \autoref{module}, so $E_*(E)\otimes_{\pi_*(E)}E_*(X)$ is a left $A$-graded $\pi_*(E)$-module by \autoref{tensor_of_A_graded_is_A_graded}). Furthermore, this homomorphism is natural in $X$.
\end{proposition}
\begin{proof}
	First, recall by definition of the tensor product, in order to show the assignment $E_*(E)\times E_*(X)\to E_*(E\otimes X)$ induces a homomorphism $E_*(E)\otimes_{\pi_*(E)}E_*(X)\to E_*(E\otimes X)$ of $A$-graded abelian groups, it suffices to show that the assignment is $\pi_*(E)$-balanced, i.e., that it is linear in each argument and satisfies $xr\cdot y=x\cdot ry$ for $x\in E_*(E)$, $y\in E_*(X)$, and $r\in\pi_*(E)$.
	
	First, note that by the identifications $E_*(E)=\pi_*(E\otimes E)$, $E_*(X)=\pi_*(E\otimes X)$, and $E_*(E\otimes X)=\pi_*(E\otimes E\otimes X)$, and \autoref{bilinear}, it is straightforward to see that the assignment commutes with addition of maps in each argument. Now, let $a,b,c\in A$, $x:S^a\to E\otimes E$, $y:S^b\to E\otimes X$, and $z:S^c\to E$. Then we wish to show $x z\cdot y=x\cdot z y$. Consider the following diagram (where here we are passing to a symmetric strict monoidal category):
	% https://q.uiver.app/#q=WzAsNixbMCwxLCJTXnthK2IrY30iXSxbMSwxLCJTXmFcXG90aW1lcyBTXmNcXG90aW1lcyBTXmIiXSxbMiwxLCJFXFxvdGltZXMgRVxcb3RpbWVzIEVcXG90aW1lcyBFXFxvdGltZXMgWCJdLFszLDAsIkVcXG90aW1lcyBFXFxvdGltZXMgRVxcb3RpbWVzIFgiXSxbMywxLCJFXFxvdGltZXMgRVxcb3RpbWVzIFgiXSxbMywyLCJFXFxvdGltZXMgRVxcb3RpbWVzIEVcXG90aW1lcyBYIl0sWzAsMSwiXFxjb25nIl0sWzEsMiwieFxcb3RpbWVzIHpcXG90aW1lcyB5Il0sWzIsMywiRVxcb3RpbWVzIFxcbXVcXG90aW1lcyBFXFxvdGltZXMgWCJdLFszLDQsIkVcXG90aW1lcyBcXG11XFxvdGltZXMgWCJdLFsyLDUsIkVcXG90aW1lcyBFXFxvdGltZXMgXFxtdVxcb3RpbWVzIFgiLDJdLFs1LDQsIkVcXG90aW1lcyBcXG11XFxvdGltZXMgWCIsMl1d
	\[\begin{tikzcd}
		&&& {E\otimes E\otimes E\otimes X} \\
		{S^{a+b+c}} & {S^a\otimes S^c\otimes S^b} & {E\otimes E\otimes E\otimes E\otimes X} & {E\otimes E\otimes X} \\
		&&& {E\otimes E\otimes E\otimes X}
		\arrow["\cong", from=2-1, to=2-2]
		\arrow["{x\otimes z\otimes y}", from=2-2, to=2-3]
		\arrow["{E\otimes \mu\otimes E\otimes X}", from=2-3, to=1-4]
		\arrow["{E\otimes \mu\otimes X}", from=1-4, to=2-4]
		\arrow["{E\otimes E\otimes \mu\otimes X}"', from=2-3, to=3-4]
		\arrow["{E\otimes \mu\otimes X}"', from=3-4, to=2-4]
	\end{tikzcd}\]
	It commutes by associativity of $\mu$. By functoriality of $-\otimes-$, the top composition is given by $(x z)\cdot y$ and the bottom composition is $x\cdot( z y)$, so we have they are equal, as desired. Thus, by \autoref{tensor_lift_of_A_graded_is_A_graded} we get the desired $A$-graded homomorphism $E_*(E)\otimes_{\pi_*(E)} E_*(X)\to E_*(E\otimes X)$.
	
	In order to see this map is a homomorphism of left $\pi_*(E)$-modules, we must show that $z(x\cdot y)=zx\cdot y$, where $x$, $y$, and $z$ are defined as above. Now consider the following diagram:
	% https://q.uiver.app/#q=WzAsNixbMCwxLCJTXnthK2IrY30iXSxbMSwxLCJTXmNcXG90aW1lcyBTXmFcXG90aW1lcyBTXmIiXSxbMiwxLCJFXFxvdGltZXMgRVxcb3RpbWVzIEVcXG90aW1lcyBFXFxvdGltZXMgWCJdLFszLDAsIkVcXG90aW1lcyBFXFxvdGltZXMgRVxcb3RpbWVzIFgiXSxbMywxLCJFXFxvdGltZXMgRVxcb3RpbWVzIFgiXSxbMywyLCJFXFxvdGltZXMgRVxcb3RpbWVzIEVcXG90aW1lcyBYIl0sWzAsMSwiXFxjb25nIl0sWzEsMiwielxcb3RpbWVzIHhcXG90aW1lcyB5Il0sWzIsMywiXFxtdVxcb3RpbWVzIEVcXG90aW1lcyBFXFxvdGltZXMgWCJdLFszLDQsIkVcXG90aW1lcyBcXG11XFxvdGltZXMgWCJdLFsyLDUsIkVcXG90aW1lcyBFXFxvdGltZXMgXFxtdVxcb3RpbWVzIFgiLDJdLFs1LDQsIlxcbXVcXG90aW1lcyBFXFxvdGltZXMgWCIsMl0sWzIsNCwiXFxtdVxcb3RpbWVzXFxtdVxcb3RpbWVzIFgiXV0=
	\[\begin{tikzcd}
		&&& {E\otimes E\otimes E\otimes X} \\
		{S^{a+b+c}} & {S^c\otimes S^a\otimes S^b} & {E\otimes E\otimes E\otimes E\otimes X} & {E\otimes E\otimes X} \\
		&&& {E\otimes E\otimes E\otimes X}
		\arrow["\cong", from=2-1, to=2-2]
		\arrow["{z\otimes x\otimes y}", from=2-2, to=2-3]
		\arrow["{\mu\otimes E\otimes E\otimes X}", from=2-3, to=1-4]
		\arrow["{E\otimes \mu\otimes X}", from=1-4, to=2-4]
		\arrow["{E\otimes E\otimes \mu\otimes X}"', from=2-3, to=3-4]
		\arrow["{\mu\otimes E\otimes X}"', from=3-4, to=2-4]
		\arrow["{\mu\otimes\mu\otimes X}", from=2-3, to=2-4]
	\end{tikzcd}\]
	Commutativity of the triangles is functoriality of $-\otimes-$. By functoriality of $-\otimes-$, the top composition is $zx\cdot y$, and the bottom composition is $z(x\cdot y)$. Hence they are equal, as desired, so that the map we have constructed
	\[E_*(E)\otimes_{\pi_*(E)}E_*(X)\to E_*(E\otimes X)\]
	is indeed an $A$-graded homomorphism of left $A$-graded $\pi_*(E)$-modules.

	Next, we would like to show that this homomorphism is natural in $X$. Let $f:X\to Y$ in $\cSH$. Then we would like to show the following diagram commutes:
	% https://q.uiver.app/#q=WzAsNCxbMCwwLCJFXyooRSlcXG90aW1lc197XFxwaV8qKEUpfUVfKihYKSJdLFswLDEsIkVfKihFKVxcb3RpbWVzX3tcXHBpXyooRSl9RV8qKFkpIl0sWzEsMSwiRV8qKEVcXG90aW1lcyBZKSJdLFsxLDAsIkVfKihFXFxvdGltZXMgWCkiXSxbMCwxLCJFXyooRSlcXG90aW1lc197XFxwaV8qKEUpfUVfKihmKSIsMl0sWzEsMiwiXFxQaGlfWSJdLFswLDMsIlxcUGhpX1giXSxbMywyLCJFXyooRVxcb3RpbWVzIGYpIl1d
	\begin{equation}\label{naturality_diagram_for_E*EoxE_*X-->E_*(EoxX)}\begin{tikzcd}
		{E_*(E)\otimes_{\pi_*(E)}E_*(X)} & {E_*(E\otimes X)} \\
		{E_*(E)\otimes_{\pi_*(E)}E_*(Y)} & {E_*(E\otimes Y)}
		\arrow["{E_*(E)\otimes_{\pi_*(E)}E_*(f)}"', from=1-1, to=2-1]
		\arrow["{\Phi_Y}", from=2-1, to=2-2]
		\arrow["{\Phi_X}", from=1-1, to=1-2]
		\arrow["{E_*(E\otimes f)}", from=1-2, to=2-2]
	\end{tikzcd}\end{equation}
	As all the maps here are homomorphisms, it suffices to chase generators around the diagram. In particular, suppose we are given $x:S^a\to E\otimes E$ and $y:S^b\to E\otimes X$, and consider the following diagram exhibiting the two possible ways to chase $x\otimes y$ around the diagram (as usual, we are passing to a symmetric strict monoidal category):
	% https://q.uiver.app/#q=WzAsNixbMCwwLCJTXnthK2J9Il0sWzEsMCwiU15hXFxvdGltZXMgU15iIl0sWzIsMCwiRVxcb3RpbWVzIEVcXG90aW1lcyBFXFxvdGltZXMgWCJdLFszLDAsIkVcXG90aW1lcyBFXFxvdGltZXMgWCJdLFszLDEsIkVcXG90aW1lcyBFXFxvdGltZXMgWSJdLFsyLDEsIkVcXG90aW1lcyBFXFxvdGltZXMgRVxcb3RpbWVzIFkiXSxbMCwxLCJcXHBoaV97YSxifSJdLFsxLDIsInhcXG90aW1lcyB5Il0sWzIsMywiRVxcb3RpbWVzIFxcbXVcXG90aW1lcyBYIl0sWzMsNCwiRVxcb3RpbWVzIEVcXG90aW1lcyBmIl0sWzIsNSwiRVxcb3RpbWVzIEVcXG90aW1lcyBFXFxvdGltZXMgZiIsMl0sWzUsNCwiRVxcb3RpbWVzIFxcbXVcXG90aW1lcyBZIl1d
	\[\begin{tikzcd}
		{S^{a+b}} & {S^a\otimes S^b} & {E\otimes E\otimes E\otimes X} & {E\otimes E\otimes X} \\
		&& {E\otimes E\otimes E\otimes Y} & {E\otimes E\otimes Y}
		\arrow["{\phi_{a,b}}", from=1-1, to=1-2]
		\arrow["{x\otimes y}", from=1-2, to=1-3]
		\arrow["{E\otimes \mu\otimes X}", from=1-3, to=1-4]
		\arrow["{E\otimes E\otimes f}", from=1-4, to=2-4]
		\arrow["{E\otimes E\otimes E\otimes f}"', from=1-3, to=2-3]
		\arrow["{E\otimes \mu\otimes Y}", from=2-3, to=2-4]
	\end{tikzcd}\]
	This diagram commutes by functoriality of $-\otimes-$. Thus we have that diagram (\ref{naturality_diagram_for_E*EoxE_*X-->E_*(EoxX)}) does indeed commute, as desired.
\end{proof}

\begin{lemma}\label{E_homology_suspension_iso_t^a's_appendix}
	Let $E$ and $X$ be objects in $\cSH$. Then for all $a\in A$, there is an $A$-graded isomorphism of $A$-graded abelian groups
	\[t^a_X:E_*(\Sigma^aX)\cong E_{*-a}(X)\]
	which sends a class $S^b\to E\otimes\Sigma^aX=E\otimes S^a\otimes X$ to the composition
	\[S^{b-a}\xr{\phi_{b,-a}} S^b\otimes S^{-a}\xr{x\otimes S^{-a}}E\otimes S^a\otimes X\otimes S^{-a}\xr{E\otimes S^a\otimes\tau_{X,S^{-a}}}E\otimes S^a\otimes S^{-a}\otimes X\xr{E\otimes\phi_{a,-a}^{-1}\otimes X}E\otimes X\]
	(where here we are ignoring associators and unitors). Furthermore this isomorphism is natural in $X$, and if $E$ is a monoid object in $\cSH$ then it is a natural isomorphism of $\pi_*(E)$-modules.
\end{lemma}
\begin{proof}
	Expressed in terms of hom-sets, $t^a_X$ is precisely the composition
	% https://q.uiver.app/#q=WzAsNyxbMSwwLCJbU14qLEVcXG90aW1lcyBTXmFcXG90aW1lcyBYXSJdLFsxLDEsIltTXipcXG90aW1lcyBTXnstYX0sRVxcb3RpbWVzIFNeYVxcb3RpbWVzIFhcXG90aW1lcyBTXnstYX1dIl0sWzEsMiwiIFtTXnsqLWF9LEVcXG90aW1lcyBTXmFcXG90aW1lcyBYXFxvdGltZXMgU157LWF9XSJdLFsxLDMsIiBbU157Ki1hfSxFXFxvdGltZXMgU15hXFxvdGltZXMgU157LWF9XFxvdGltZXMgWF0iXSxbMSw0LCIgW1NeeyotYX0sRVxcb3RpbWVzIFhdIl0sWzIsNCwiRV97Ki1hfShFXFxvdGltZXMgWCkiXSxbMCwwLCJFXyooXFxTaWdtYV5hWCkiXSxbMCwxLCItXFxvdGltZXMgU157LWF9Il0sWzEsMiwiKFxccGhpX3sqLC1hfSleKiJdLFsyLDMsIihFXFxvdGltZXMgU15hXFxvdGltZXNcXHRhdSlfKiJdLFszLDQsIihFXFxvdGltZXNcXHBoaV97YSwtYX1eey0xfVxcb3RpbWVzIFgpXyoiXSxbNCw1LCIiLDAseyJsZXZlbCI6Miwic3R5bGUiOnsiaGVhZCI6eyJuYW1lIjoibm9uZSJ9fX1dLFs2LDAsIiIsMCx7ImxldmVsIjoyLCJzdHlsZSI6eyJoZWFkIjp7Im5hbWUiOiJub25lIn19fV1d
	\[\begin{tikzcd}[column sep=tiny]
		{E_*(\Sigma^aX)} & {[S^*,E\otimes S^a\otimes X]} \\
		& {[S^*\otimes S^{-a},E\otimes S^a\otimes X\otimes S^{-a}]} \\
		& { [S^{*-a},E\otimes S^a\otimes X\otimes S^{-a}]} \\
		& { [S^{*-a},E\otimes S^a\otimes S^{-a}\otimes X]} \\
		& { [S^{*-a},E\otimes X]} & {E_{*-a}(E\otimes X)}
		\arrow["{-\otimes S^{-a}}", from=1-2, to=2-2]
		\arrow["{(\phi_{*,-a})^*}", from=2-2, to=3-2]
		\arrow["{(E\otimes S^a\otimes\tau)_*}", from=3-2, to=4-2]
		\arrow["{(E\otimes\phi_{a,-a}^{-1}\otimes X)_*}", from=4-2, to=5-2]
		\arrow[Rightarrow, no head, from=5-2, to=5-3]
		\arrow[Rightarrow, no head, from=1-1, to=1-2]
	\end{tikzcd}\]
	We know the first vertical arrow is an isomorphism of abelian groups as $-\otimes-$ is additive in each variable (since $\cSH$ is tensor triangulated) and $\Omega^a\cong -\otimes S^{-a}$ is an autoequivalence of $\cSH$ by \autoref{Sigma^a,Sigma^-a_adjoint_equiv}.  The three other vertical arrows are given by composing with an isomorphism in an additive category, so they are also isomorphisms.
	
	To see $t_X^a$ is a homomorphism of left $\pi_*(E)$-modules, suppose we are given classes $r:S^b\to E$ $\pi_b(E)$ and $x:S^c\to E\otimes S^a\otimes X$ in $E_c(\Sigma^aX)$. Then we wish to show that $t_X^a(r\cdot x)=r\cdot t_X^a(x)$. Consider the following diagram:
	% https://q.uiver.app/#q=WzAsOCxbMCwwLCJTXntiK2MtYX0iXSxbMCwyLCJTXmJcXG90aW1lcyBTXmNcXG90aW1lcyBTXnstYX0iXSxbMiwyLCJFXFxvdGltZXMgRVxcb3RpbWVzIFNeYVxcb3RpbWVzIFhcXG90aW1lcyBTXnstYX0iXSxbMiwwLCJFXFxvdGltZXMgU15hXFxvdGltZXMgWFxcb3RpbWVzIFNeey1hfSJdLFsyLDQsIkVcXG90aW1lcyBFXFxvdGltZXMgU15hXFxvdGltZXMgU157LWF9XFxvdGltZXMgWCJdLFs0LDQsIkVcXG90aW1lcyBFXFxvdGltZXMgWCJdLFs0LDIsIkVcXG90aW1lcyBYIl0sWzQsMCwiRVxcb3RpbWVzIFNeYVxcb3RpbWVzIFNeey1hfVxcb3RpbWVzIFgiXSxbMCwxLCJcXGNvbmciXSxbMSwyLCJyXFxvdGltZXMgeFxcb3RpbWVzIFNeey1hfSJdLFsyLDMsIlxcbXVcXG90aW1lcyBTXmFcXG90aW1lcyBYXFxvdGltZXMgU157LWF9Il0sWzIsNCwiRVxcb3RpbWVzIEVcXG90aW1lcyBTXmFcXG90aW1lcyBcXHRhdV97WCxTXnstYX19IiwyXSxbNCw1LCJFXFxvdGltZXMgRVxcb3RpbWVzIFxccGhpX3thLC1hfV57LTF9XFxvdGltZXMgWCIsMl0sWzUsNiwiXFxtdVxcb3RpbWVzIFgiLDJdLFszLDcsIkVcXG90aW1lcyBTXmFcXG90aW1lcyBcXHRhdV97WCxTXnstYX19Il0sWzcsNiwiRVxcb3RpbWVzIFxccGhpX3thLC1hfV57LTF9XFxvdGltZXMgWCJdLFs0LDcsIlxcbXVcXG90aW1lcyBTXmFcXG90aW1lcyBTXnstYX1cXG90aW1lcyBYIiwyXV0=
	\[\begin{tikzcd}
		{S^{b+c-a}} && {E\otimes S^a\otimes X\otimes S^{-a}} && {E\otimes S^a\otimes S^{-a}\otimes X} \\
		\\
		{S^b\otimes S^c\otimes S^{-a}} && {E\otimes E\otimes S^a\otimes X\otimes S^{-a}} && {E\otimes X} \\
		\\
		&& {E\otimes E\otimes S^a\otimes S^{-a}\otimes X} && {E\otimes E\otimes X}
		\arrow["\cong", from=1-1, to=3-1]
		\arrow["{r\otimes x\otimes S^{-a}}", from=3-1, to=3-3]
		\arrow["{\mu\otimes S^a\otimes X\otimes S^{-a}}", from=3-3, to=1-3]
		\arrow["{E\otimes E\otimes S^a\otimes \tau_{X,S^{-a}}}"', from=3-3, to=5-3]
		\arrow["{E\otimes E\otimes \phi_{a,-a}^{-1}\otimes X}"', from=5-3, to=5-5]
		\arrow["{\mu\otimes X}"', from=5-5, to=3-5]
		\arrow["{E\otimes S^a\otimes \tau_{X,S^{-a}}}", from=1-3, to=1-5]
		\arrow["{E\otimes \phi_{a,-a}^{-1}\otimes X}", from=1-5, to=3-5]
		\arrow["{\mu\otimes S^a\otimes S^{-a}\otimes X}"', from=5-3, to=1-5]
	\end{tikzcd}\]
	Both triangles commute by functoriality of $-\otimes-$. The top composition is $t_X^a(r\cdot x)$ while the bottom is $r\cdot t_X^a(x)$, so they are equal as desired.
	
	It remains to show $t^a_X$ is natural in $X$. let $f:X\to Y$ in $\cSH$, then we would like to show the following diagram commutes:
	% https://q.uiver.app/#q=WzAsNCxbMCwwLCJFXyooXFxTaWdtYV5hWCkiXSxbMSwwLCJFX3sqLWF9KFgpIl0sWzEsMSwiRV97Ki1hfShZKSJdLFswLDEsIkVfKihcXFNpZ21hXmFZKSJdLFswLDEsInReYV9YIl0sWzEsMiwiRV97Ki1hfShmKSJdLFswLDMsIkVfKihcXFNpZ21hXmFmKSIsMl0sWzMsMiwidF5hX1kiXV0=
	\begin{equation}\label{naturality_of_t^a_diagram}\begin{tikzcd}
		{E_*(\Sigma^aX)} & {E_{*-a}(X)} \\
		{E_*(\Sigma^aY)} & {E_{*-a}(Y)}
		\arrow["{t^a_X}", from=1-1, to=1-2]
		\arrow["{E_{*-a}(f)}", from=1-2, to=2-2]
		\arrow["{E_*(\Sigma^af)}"', from=1-1, to=2-1]
		\arrow["{t^a_Y}", from=2-1, to=2-2]
	\end{tikzcd}\end{equation}
	We may chase a generator around the diagram since all the arrows here are homomorphisms. Let $x:S^b\to E\otimes S^a\otimes X$ in $E_*(\Sigma^aX)$. Then consider the following diagram:
	% https://q.uiver.app/#q=WzAsOCxbMCwwLCJTXntiLWF9Il0sWzEsMCwiU15iXFxvdGltZXMgU157LWF9Il0sWzIsMCwiRVxcb3RpbWVzIFNeYVxcb3RpbWVzIFhcXG90aW1lcyBTXnstYX0iXSxbMywwLCJFXFxvdGltZXMgU15hXFxvdGltZXMgU157LWF9XFxvdGltZXMgWCJdLFs0LDAsIkVcXG90aW1lcyBYIl0sWzQsMSwiRVxcb3RpbWVzIFkiXSxbMiwxLCJFXFxvdGltZXMgU15hXFxvdGltZXMgWVxcb3RpbWVzIFNeey1hfSJdLFszLDEsIkVcXG90aW1lcyBTXmFcXG90aW1lcyBTXnstYX1cXG90aW1lcyBZIl0sWzAsMSwiXFxjb25nIl0sWzEsMiwieFxcb3RpbWVzIFNeey1hfSJdLFsyLDMsIkVcXG90aW1lcyBTXmFcXG90aW1lcyBcXHRhdSJdLFszLDQsIkVcXG90aW1lcyBcXHBoaV97YSwtYX1eey0xfVxcb3RpbWVzIFgiXSxbNCw1LCJFXFxvdGltZXMgZiJdLFsyLDYsIkVcXG90aW1lcyBTXnthfVxcb3RpbWVzIGZcXG90aW1lcyBTXnstYX0iLDJdLFs2LDcsIkVcXG90aW1lcyBTXmFcXG90aW1lcyBcXHRhdSIsMl0sWzcsNSwiRVxcb3RpbWVzIFxccGhpX3thLC1hfV57LTF9XFxvdGltZXMgWSIsMl0sWzMsNywiRVxcb3RpbWVzIFNeYVxcb3RpbWVzIFNeey1hfVxcb3RpbWVzIGYiLDJdXQ==
	\[\begin{tikzcd}
		{S^{b-a}} & {S^b\otimes S^{-a}} & {E\otimes S^a\otimes X\otimes S^{-a}} & {E\otimes S^a\otimes S^{-a}\otimes X} & {E\otimes X} \\
		&& {E\otimes S^a\otimes Y\otimes S^{-a}} & {E\otimes S^a\otimes S^{-a}\otimes Y} & {E\otimes Y}
		\arrow["\cong", from=1-1, to=1-2]
		\arrow["{x\otimes S^{-a}}", from=1-2, to=1-3]
		\arrow["{E\otimes S^a\otimes \tau}", from=1-3, to=1-4]
		\arrow["{E\otimes \phi_{a,-a}^{-1}\otimes X}", from=1-4, to=1-5]
		\arrow["{E\otimes f}", from=1-5, to=2-5]
		\arrow["{E\otimes S^{a}\otimes f\otimes S^{-a}}"', from=1-3, to=2-3]
		\arrow["{E\otimes S^a\otimes \tau}"', from=2-3, to=2-4]
		\arrow["{E\otimes \phi_{a,-a}^{-1}\otimes Y}"', from=2-4, to=2-5]
		\arrow["{E\otimes S^a\otimes S^{-a}\otimes f}"', from=1-4, to=2-4]
	\end{tikzcd}\]
	The left rectangle commutes by naturality of $\tau$, while the right rectangle commutes by functoriality of $-\otimes-$. The two outside compositions are the two ways to chase $x$ around diagram (\ref{naturality_of_t^a_diagram}), so the diagram commutes as desired.
\end{proof}

\begin{lemma}\label{t's_commute_with_Phi's}
	Given a monoid object $(E,\mu,e)$ in $\cSH$, the maps $\Phi_X$ constructed in \autoref{Kunneth_map} commute with the natural isomorphisms $t^a_X:E_*(\Sigma^aX)\xr\cong E_{*-a}(X)$ given in \autoref{E_homology_suspension_iso_t^a's_appendix}, in the sense that the following diagram commutes for all $a\in A$ and $X$ in $\cSH$:
	% https://q.uiver.app/#q=WzAsNCxbMCwwLCJFXyooRSlcXG90aW1lc197XFxwaV8qKEUpfUVfKihcXFNpZ21hXmFYKSJdLFsyLDAsIkVfKihFKVxcb3RpbWVzX3tcXHBpXyooRSl9RV97Ki1hfShYKSJdLFswLDEsIkVfKihFXFxvdGltZXNcXFNpZ21hXmFYKSJdLFsyLDEsIkVfeyotYX0oRVxcb3RpbWVzIFgpIl0sWzAsMSwiRV8qKEUpXFxvdGltZXMgdF5hX1giXSxbMCwyLCJcXFBoaV97XFxTaWdtYV5hWH0iLDJdLFsyLDMsInReYV9YIl0sWzEsMywiXFxQaGlfWCJdXQ==
	\[\begin{tikzcd}
		{E_*(E)\otimes_{\pi_*(E)}E_*(\Sigma^aX)} && {E_*(E)\otimes_{\pi_*(E)}E_{*-a}(X)} \\
		{E_*(E\otimes\Sigma^aX)} && {E_{*-a}(E\otimes X)}
		\arrow["{E_*(E)\otimes t^a_X}", from=1-1, to=1-3]
		\arrow["{\Phi_{\Sigma^aX}}"', from=1-1, to=2-1]
		\arrow["{t^a_X}", from=2-1, to=2-3]
		\arrow["{\Phi_X}", from=1-3, to=2-3]
	\end{tikzcd}\]
	where the top arrow is well-defined since $t_X^a$ is a left $\pi_*(E)$-modue homomorphism by the above lemma, and we are being abusive in that the bottom arrow is given by the composition
	\[E_*(E\otimes\Sigma^aX)\overset\alpha\cong(E\otimes E)_*(\Sigma^aX)\xr{t_X^a}(E\otimes E)_{*-a}(X)\overset\alpha\cong E_{*-a}(E\otimes X).\]
\end{lemma}
\begin{proof}
	Since all the maps in the above diagram are homomorphisms, we can chase generators around to show it commutes. Let $x:S^b\to E\otimes E$ and $y:S^c\to E\otimes\Sigma^aX=E\otimes S^a\otimes X$. Then consider the following diagram:
	% https://q.uiver.app/#q=WzAsOCxbMCwwLCJTXntiK2MtYX0iXSxbMCwyLCJTXmJcXG90aW1lcyBTXmNcXG90aW1lcyBTXnstYX0iXSxbMywyLCJFXFxvdGltZXMgRVxcb3RpbWVzIEVcXG90aW1lcyBTXmFcXG90aW1lcyBYXFxvdGltZXMgU157LWF9Il0sWzMsMCwiRVxcb3RpbWVzIEVcXG90aW1lcyBFXFxvdGltZXMgU15hXFxvdGltZXMgU157LWF9XFxvdGltZXMgWCJdLFs0LDAsIkVcXG90aW1lcyBFXFxvdGltZXMgRVxcb3RpbWVzIFgiXSxbNCwyLCJFXFxvdGltZXMgRVxcb3RpbWVzIFgiXSxbMyw0LCJFXFxvdGltZXMgRVxcb3RpbWVzIFNeYVxcb3RpbWVzIFhcXG90aW1lcyBTXnstYX0iXSxbNCw0LCJFXFxvdGltZXMgRVxcb3RpbWVzIFNeYVxcb3RpbWVzIFNeey1hfVxcb3RpbWVzIFgiXSxbMCwxLCJcXGNvbmciXSxbMiwzLCJFXFxvdGltZXMgRVxcb3RpbWVzIEVcXG90aW1lcyBTXmFcXG90aW1lcyBcXHRhdSJdLFszLDQsIkVcXG90aW1lcyBFXFxvdGltZXMgRVxcb3RpbWVzIFxccGhpX3thLC1hfV57LTF9XFxvdGltZXMgWCJdLFs0LDUsIkVcXG90aW1lcyBcXG11XFxvdGltZXMgWCJdLFsyLDYsIkVcXG90aW1lcyBcXG11XFxvdGltZXMgU15hXFxvdGltZXMgWFxcb3RpbWVzIFNeey1hfSIsMl0sWzYsNywiRVxcb3RpbWVzIEVcXG90aW1lcyBTXmFcXG90aW1lcyBcXHRhdSIsMl0sWzcsNSwiRVxcb3RpbWVzIEVcXG90aW1lcyBcXHBoaV97YSwtYX1eey0xfVxcb3RpbWVzIFgiLDJdLFszLDcsIkVcXG90aW1lcyBcXG11XFxvdGltZXMgU15hXFxvdGltZXMgU157LWF9XFxvdGltZXMgWCJdLFsxLDIsInhcXG90aW1lcyB5XFxvdGltZXMgU157LWF9Il1d
	\[\begin{tikzcd}
		{S^{b+c-a}} &&& {E\otimes E\otimes E\otimes S^a\otimes S^{-a}\otimes X} & {E\otimes E\otimes E\otimes X} \\
		\\
		{S^b\otimes S^c\otimes S^{-a}} &&& {E\otimes E\otimes E\otimes S^a\otimes X\otimes S^{-a}} & {E\otimes E\otimes X} \\
		\\
		&&& {E\otimes E\otimes S^a\otimes X\otimes S^{-a}} & {E\otimes E\otimes S^a\otimes S^{-a}\otimes X}
		\arrow["\cong", from=1-1, to=3-1]
		\arrow["{E\otimes E\otimes E\otimes S^a\otimes \tau}", from=3-4, to=1-4]
		\arrow["{E\otimes E\otimes E\otimes \phi_{a,-a}^{-1}\otimes X}", from=1-4, to=1-5]
		\arrow["{E\otimes \mu\otimes X}", from=1-5, to=3-5]
		\arrow["{E\otimes \mu\otimes S^a\otimes X\otimes S^{-a}}"', from=3-4, to=5-4]
		\arrow["{E\otimes E\otimes S^a\otimes \tau}"', from=5-4, to=5-5]
		\arrow["{E\otimes E\otimes \phi_{a,-a}^{-1}\otimes X}"', from=5-5, to=3-5]
		\arrow["{E\otimes \mu\otimes S^a\otimes S^{-a}\otimes X}", from=1-4, to=5-5]
		\arrow["{x\otimes y\otimes S^{-a}}", from=3-1, to=3-4]
	\end{tikzcd}\]
	Each triangle commutes by functoriality of $-\otimes-$. The two outside compositions are the two ways to chase $x\otimes y$ around the diagram in the statement of the lemma, so the diagram commutes as desired.
\end{proof}

\begin{corollary}\label{t_commutes_with_Phi's_corollary}
	For all $X$ in $\cSH$, we have natural isomorphisms $t_X:E_*E(\Sigma X)\xr\cong E_{*-\1}(X)$ given by the composition
	\[E_*(\Sigma X)\xr{E_*(\nu_X)}E_*(\Sigma^1 X)\xr{t^\1_X}E_{*-\1}(X).\]
	Furthermore, by naturality of $\Phi$ and the fact that $t_X^\1$ commutes with $\Phi$ (in the sense of the above lemma), this isomorphism also commutes with $\Phi$.
\end{corollary}

\begin{proposition}\label{Kunneth_iso_for_cellular_objects}
	Let $(E,\mu,e)$ be a flat monoid object in $\cSH$ (\autoref{flat}) and let $X$ be any cellular object in $\cSH$ (\autoref{cellular}). Then the natural homomorphism
	\[\Phi_X:E_*(E)\otimes_{\pi_*(E)} E_*(X)\to E_*(E\otimes X)\]
	given in \autoref{Kunneth_map} is an isomorphism of left $\pi_*(E)$-modules.
\end{proposition}
\begin{proof}
	In this proof, we will freely employ the coherence theorem for symmetric monoidal categories, and we will assume that associativity and unitality of $-\otimes-$ holds up to strict equality. To start, let $\cE$ be the collection of objects $X$ in $\cSH$ for which this map is an isomorphism. Then in order to show $\cE$ contains every cellular object, it suffices to show that $\cE$ satisfies the three conditions given for the class of cellular objects in \autoref{cellular}. First, we need to show that $\Phi$ is an isomorphism when $X=S^a$ for some $a\in A$.
%	Note that
%	\[E_*(S^a)=[S^*,E\otimes S^a]\cong[S^{-a}\otimes S^*,E]\cong[S^{*-a},E]=\pi_{*-a}(E),\]
%	where the first isomorphism follows by the adunction between $S^{-a}\otimes-$ and $-\otimes S^a\cong S^a\otimes-$ (\autoref{Sigma^a,Sigma^-a_adjoint_equiv}). Similarly, we have
%	\[E_*(E\otimes S^a)=[S^*,E\otimes E\otimes S^a]\cong[S^{*-a},E\otimes E]=E_{*-a}(E).\]
%	Hence by \autoref{tensor_shift_A_graded} we have isomorphisms
%	\[E_*(E)\otimes_{\pi_*(E)}E_*(S^a)\cong E_*(E)\otimes_{\pi_*(E)}\pi_{*-a}(E)\cong E_{*-a}(E)\cong E_*(E\otimes S^a).\]
	Indeed, consider the map
	\begin{align*}
		\Psi:E_*(E\otimes S^a)&\to E_*(E)\otimes_{\pi_*(E)}E_*(S^a)
	\end{align*}
	which sends a class $x:S^b\to E\otimes E\otimes S^a$ in $E_b(E\otimes S^a)$ to the pure tensor $\wt x\otimes\wt e$, where $\wt x\in E_{b-a}(E)$ is the composition
	\[S^{b-a}\cong S^b\otimes S^{-a}\xr{x\otimes S^{-a}}E\otimes E\otimes S^a\otimes S^{-a}\xr{E\otimes E\otimes\phi_{a,-a}^{-1}}E\otimes E\]
	and $\wt e\in E_a(S^a)$ is the composition
	\[S^a\cong S\otimes S^a\xr{e\otimes S^a}E\otimes S^a.\]
	First, note $\Psi$ is an ($A$-graded) homomorphism of abelian groups: Given $x,x'\in E_b(E\otimes S^a)$, we would like to show that $\wt x\otimes\wt e+\wt x'\otimes\wt e=\wt{x+x'}\otimes\wt e$. It suffices to show that $\wt x+\wt x'=\wt{x+x'}$. To see this, consider the following diagram (again, we are passing to a symmetric strict monoidal category):
	% https://q.uiver.app/#q=WzAsMTAsWzAsMCwiU157Yi1hfSJdLFsxLDAsIlNee2ItYX1cXG9wbHVzIFNee2ItYX0iXSxbMSwxLCIoU15iXFxvdGltZXMgU157LWF9KVxcb3BsdXMoU15iXFxvdGltZXMgU157LWF9KSJdLFsxLDIsIihFXFxvdGltZXMgRVxcb3RpbWVzIFNeYVxcb3RpbWVzIFNeey1hfSlcXG9wbHVzKEVcXG90aW1lcyBFXFxvdGltZXMgU15hXFxvdGltZXMgU157LWF9KSJdLFsxLDMsIihFXFxvdGltZXMgRSlcXG9wbHVzKEVcXG90aW1lcyBFKSJdLFsxLDQsIkVcXG90aW1lcyBFIl0sWzAsMSwiU15iXFxvdGltZXMgU157LWF9Il0sWzAsMiwiKFNeYlxcb3BsdXMgU15iKVxcb3RpbWVzIFNeey1hfSJdLFswLDMsIigoRVxcb3RpbWVzIEVcXG90aW1lcyBTXmEpXFxvcGx1cyAoRVxcb3RpbWVzIEVcXG90aW1lcyBTXmEpKVxcb3RpbWVzIFNeey1hfSJdLFswLDQsIkVcXG90aW1lcyBFXFxvdGltZXMgU15hXFxvdGltZXMgU157LWF9Il0sWzAsMSwiXFxEZWx0YSJdLFsxLDIsIlxccGhpX3tiLC1hfVxcb3BsdXNcXHBoaV97YiwtYX0iXSxbMiwzLCIoeFxcb3RpbWVzIFNeey1hfSlcXG9wbHVzKHgnXFxvdGltZXMgU157LWF9KSJdLFszLDQsIihFXFxvdGltZXMgRVxcb3RpbWVzXFxwaGlfe2EsLWF9XnstMX0pXFxvcGx1cyhFXFxvdGltZXMgRVxcb3RpbWVzXFxwaGlfe2EsLWF9XnstMX0pIl0sWzQsNSwiXFxuYWJsYSJdLFswLDYsIlxccGhpX3tiLWF9IiwyXSxbNiw3LCJcXERlbHRhXFxvdGltZXMgU157LWF9IiwyXSxbNyw4LCIoeFxcb3BsdXMgeCcpXFxvdGltZXMgU157LWF9IiwyXSxbOCw5LCJcXG5hYmxhXFxvdGltZXMgU157LWF9IiwyXSxbOSw1LCJFXFxvdGltZXMgRVxcb3RpbWVzXFxwaGlfe2EsLWF9XnstMX0iLDJdLFs3LDIsIlxcY29uZyJdLFs4LDMsIlxcY29uZyJdLFszLDksIlxcbmFibGEiXSxbNiwyLCJcXERlbHRhIl1d
	\[\begin{tikzcd}
		{S^{b-a}} & {S^{b-a}\oplus S^{b-a}} \\
		{S^b\otimes S^{-a}} & {(S^b\otimes S^{-a})\oplus(S^b\otimes S^{-a})} \\
		{(S^b\oplus S^b)\otimes S^{-a}} & {(E\otimes E\otimes S^a\otimes S^{-a})\oplus(E\otimes E\otimes S^a\otimes S^{-a})} \\
		{((E\otimes E\otimes S^a)\oplus (E\otimes E\otimes S^a))\otimes S^{-a}} & {(E\otimes E)\oplus(E\otimes E)} \\
		{E\otimes E\otimes S^a\otimes S^{-a}} & {E\otimes E}
		\arrow["\Delta", from=1-1, to=1-2]
		\arrow["{\phi_{b,-a}\oplus\phi_{b,-a}}", from=1-2, to=2-2]
		\arrow["{(x\otimes S^{-a})\oplus(x'\otimes S^{-a})}", from=2-2, to=3-2]
		\arrow["{(E\otimes E\otimes\phi_{a,-a}^{-1})\oplus(E\otimes E\otimes\phi_{a,-a}^{-1})}", from=3-2, to=4-2]
		\arrow["\nabla", from=4-2, to=5-2]
		\arrow["{\phi_{b-a}}"', from=1-1, to=2-1]
		\arrow["{\Delta\otimes S^{-a}}"', from=2-1, to=3-1]
		\arrow["{(x\oplus x')\otimes S^{-a}}"', from=3-1, to=4-1]
		\arrow["{\nabla\otimes S^{-a}}"', from=4-1, to=5-1]
		\arrow["{E\otimes E\otimes\phi_{a,-a}^{-1}}"', from=5-1, to=5-2]
		\arrow["\cong", from=3-1, to=2-2]
		\arrow["\cong", from=4-1, to=3-2]
		\arrow["\nabla", from=3-2, to=5-1]
		\arrow["\Delta", from=2-1, to=2-2]
	\end{tikzcd}\]
	The top rectangle commutes by naturality of $\Delta$ in an additive category. The bottom triangle commutes by naturality of $\nabla$ in an additive category. Finally, the remaining regions of the diagram commute by additivity of $-\otimes-$. By functoriality of $-\otimes-$, it follows that the left composition is $\wt{x+x'}$ and the right composition is $\wt x+\wt x'$, so they are equal as desired. Thus $\Psi$ is a homomorphism of abelian groups, as desired.

	Now, we claim that $\Psi$ is an inverse to $\Phi$, (which is enough to show $\Phi$ is an isomorphism of left $\pi_*(E)$-modules). Since $\Phi$ and $\Psi$ are homomorphisms it suffices to check that they are inverses on generators. First, let $x:S^b\to E\otimes E\otimes S^a$ in $E_b(E\otimes S^a)$. We would like to show that $\Phi(\Psi(x))=x$. Consider the following diagram, where here we are passing to a symmetric strict monoidal category:
	% https://q.uiver.app/#q=WzAsOCxbMCwwLCJTXmIiXSxbMiwwLCJTXmJcXG90aW1lcyBTXnstYX1cXG90aW1lcyBTXmEiXSxbNCwxLCJFXFxvdGltZXMgRVxcb3RpbWVzIFNeYVxcb3RpbWVzIFNeey1hfVxcb3RpbWVzIEVcXG90aW1lcyBTXmEiXSxbMCwzLCJFXFxvdGltZXMgRVxcb3RpbWVzIEVcXG90aW1lcyBTXmEiXSxbMCwyLCJFXFxvdGltZXMgRVxcb3RpbWVzIFNeYSJdLFsyLDMsIkVcXG90aW1lcyBFXFxvdGltZXMgRVxcb3RpbWVzIFNeYSJdLFsyLDEsIkVcXG90aW1lcyBFXFxvdGltZXMgU15hIFxcb3RpbWVzIFNeey1hfVxcb3RpbWVzIFNeYSJdLFsyLDIsIkVcXG90aW1lcyBFXFxvdGltZXMgU15hIl0sWzAsMSwiXFxjb25nIl0sWzEsMiwieFxcb3RpbWVzIFNeey1hfVxcb3RpbWVzIGVcXG90aW1lcyBTXmEiXSxbMyw0LCJFXFxvdGltZXMgXFxtdVxcb3RpbWVzIFNeYSJdLFswLDQsIngiLDJdLFs0LDUsIkVcXG90aW1lcyBFXFxvdGltZXMgZVxcb3RpbWVzIFNeYSIsMV0sWzUsMywiIiwxLHsibGV2ZWwiOjIsInN0eWxlIjp7ImhlYWQiOnsibmFtZSI6Im5vbmUifX19XSxbMSw2LCJ4XFxvdGltZXMgU157LWF9XFxvdGltZXMgU15hIiwxXSxbNCw2LCJFXFxvdGltZXMgRVxcb3RpbWVzIFNeYVxcb3RpbWVzIFxccGhpX3stYSxhfSJdLFs2LDIsIkVcXG90aW1lcyBFXFxvdGltZXMgU15hXFxvdGltZXMgU157LWF9XFxvdGltZXMgZVxcb3RpbWVzIFNeYSIsMl0sWzcsNiwiRVxcb3RpbWVzIEVcXG90aW1lcyBcXHBoaV97YSwtYX1cXG90aW1lcyBTXmEiLDFdLFs3LDUsIkVcXG90aW1lcyBFXFxvdGltZXMgZVxcb3RpbWVzIFNeYSIsMV0sWzcsNCwiIiwyLHsibGV2ZWwiOjIsInN0eWxlIjp7ImhlYWQiOnsibmFtZSI6Im5vbmUifX19XSxbMiw1LCJFXFxvdGltZXMgRVxcb3RpbWVzIFxccGhpX3thLC1hfV57LTF9XFxvdGltZXMgRVxcb3RpbWVzIFNeYSJdXQ==
	\[\begin{tikzcd}
		{S^b} && {S^b\otimes S^{-a}\otimes S^a} \\
		&& {E\otimes E\otimes S^a \otimes S^{-a}\otimes S^a} && {E\otimes E\otimes S^a\otimes S^{-a}\otimes E\otimes S^a} \\
		{E\otimes E\otimes S^a} && {E\otimes E\otimes S^a} \\
		{E\otimes E\otimes E\otimes S^a} && {E\otimes E\otimes E\otimes S^a}
		\arrow["\cong", from=1-1, to=1-3]
		\arrow["{x\otimes S^{-a}\otimes e\otimes S^a}", from=1-3, to=2-5]
		\arrow["{E\otimes \mu\otimes S^a}", from=4-1, to=3-1]
		\arrow["x"', from=1-1, to=3-1]
		\arrow["{E\otimes E\otimes e\otimes S^a}"{description}, from=3-1, to=4-3]
		\arrow[Rightarrow, no head, from=4-3, to=4-1]
		\arrow["{x\otimes S^{-a}\otimes S^a}"{description}, from=1-3, to=2-3]
		\arrow["{E\otimes E\otimes S^a\otimes \phi_{-a,a}}", from=3-1, to=2-3]
		\arrow["{E\otimes E\otimes S^a\otimes S^{-a}\otimes e\otimes S^a}"', from=2-3, to=2-5]
		\arrow["{E\otimes E\otimes \phi_{a,-a}\otimes S^a}"{description}, from=3-3, to=2-3]
		\arrow["{E\otimes E\otimes e\otimes S^a}"{description}, from=3-3, to=4-3]
		\arrow[Rightarrow, no head, from=3-3, to=3-1]
		\arrow["{E\otimes E\otimes \phi_{a,-a}^{-1}\otimes E\otimes S^a}", from=2-5, to=4-3]
	\end{tikzcd}\]
	The top left trapezoid commutes since the isomorphism $S^b\xr\cong S^b\otimes S^{-a}\otimes S^a$ may be given as $S^b\otimes\phi_{-a,a}$ (see \autoref{unique_comp_Sas}), in which case the trapezoid commmutes by functoriality of $-\otimes-$. The triangle below that commutes by coherence for the $\phi_{a,b}$'s. The triangle below that commutes by definition. The bottom left triangle commutes by unitality for $\mu$. The top right triangle commutes by functoriality of $-\otimes-$. Finally, the bottom right triangle commutes by functoriality of $-\otimes-$. It follows by unravelling definitions that the two outside compositions are $x$ and $\Phi(\Psi(x))$, so indeed we have $\Phi(\Psi(x))=x$ since the diagram commutes.

	On the other hand, suppose we are given a homogeneous pure tensor $x\otimes y$ in $E_*(E)\otimes_{\pi_*(E)}E_*(S^a)$, so $x:S^b\to E\otimes E$ and $y:S^c\to E\otimes S^a$ for some $b,c\in A$. Then we would like to show that $\Psi(\Phi(x\otimes y))=x\otimes y$. Unravelling definitions, $\Psi(\Phi(x\otimes y))$ is the homogeneous pure tensor $\wt{x y}\otimes\wt e$, where $\wt e:S^{a}\to E\otimes S^a$ is defined above, and by functoriality of $-\otimes-$, $\wt{xy}:S^{b+c-a}\to E\otimes E$ is the composition
	% https://q.uiver.app/#q=WzAsNyxbMCwwLCJTXntiK2MtYX0iXSxbMCwxLCJTXntiK2N9XFxvdGltZXMgU157LWF9Il0sWzAsMiwiU157Yn1cXG90aW1lcyBTXmNcXG90aW1lcyBTXnstYX0iXSxbMCwzLCJFXFxvdGltZXMgRVxcb3RpbWVzIEVcXG90aW1lcyBTXmFcXG90aW1lcyBTXnstYX0iXSxbMCw0LCJFXFxvdGltZXMgRVxcb3RpbWVzIFNeYVxcb3RpbWVzIFNeey1hfSJdLFswLDUsIkVcXG90aW1lcyBFXFxvdGltZXMgUyJdLFswLDYsIkVcXG90aW1lcyBFIl0sWzAsMSwiXFxwaGlfe2IrYywtYX0iXSxbMSwyLCJcXHBoaV97YixjfVxcb3RpbWVzICBTXnstYX0iXSxbMiwzLCJ4XFxvdGltZXMgeVxcb3RpbWVzIFNeey1hfSJdLFszLDQsIkVcXG90aW1lcyBcXG11XFxvdGltZXMgU15hXFxvdGltZXMgU157LWF9Il0sWzQsNSwiRVxcb3RpbWVzIEVcXG90aW1lcyBcXHBoaV97YSwtYX1eey0xfSJdLFs1LDYsIkVcXG90aW1lcyBcXHJob19FIl1d
	\[\begin{tikzcd}
		{S^{b+c-a}} \\
		{S^{b+c}\otimes S^{-a}} \\
		{S^{b}\otimes S^c\otimes S^{-a}} \\
		{E\otimes E\otimes E\otimes S^a\otimes S^{-a}} \\
		{E\otimes E\otimes S^a\otimes S^{-a}} \\
		{E\otimes E\otimes S} \\
		{E\otimes E.}
		\arrow["{\phi_{b+c,-a}}", from=1-1, to=2-1]
		\arrow["{\phi_{b,c}\otimes  S^{-a}}", from=2-1, to=3-1]
		\arrow["{x\otimes y\otimes S^{-a}}", from=3-1, to=4-1]
		\arrow["{E\otimes \mu\otimes S^a\otimes S^{-a}}", from=4-1, to=5-1]
		\arrow["{E\otimes E\otimes \phi_{a,-a}^{-1}}", from=5-1, to=6-1]
		\arrow["{E\otimes \rho_E}", from=6-1, to=7-1]
	\end{tikzcd}\]
	In order to see $x\otimes y=\wt{xy}\otimes\wt e$, it suffices to show there exists some scalar $r\in\pi_{c-a}(E)$ such that $x\cdot r=\wt{xy}$ and $r\cdot\wt e=y$, where here $\cdot$ denotes the right and left action of $\pi_*(E)$ on $E_*(E)$ and $E_*(S^a)$, respectively. Now, define $r$ to be the composition
	\[S^{c-a}\cong S^c\otimes S^{-a}\xr{y\otimes S^{-a}}E\otimes S^a\otimes S^{-a}\xr{E\otimes\phi_{a,-a}^{-1}}E\otimes S\xr{\rho_E}E.\]
	First, in order to see that $x\cdot r=\wt{xy}$, consider the following diagram, where here we are again passing to a symmetric strict monoidal category:
	% https://q.uiver.app/#q=WzAsNixbMCwwLCJTXntiK2MtYX0iXSxbMSwwLCJTXntifVxcb3RpbWVzIFNeY1xcb3RpbWVzIFNeey1hfSJdLFsyLDAsIkVcXG90aW1lcyBFXFxvdGltZXMgRVxcb3RpbWVzIFNeYVxcb3RpbWVzIFNeey1hfSJdLFszLDAsIkVcXG90aW1lcyBFXFxvdGltZXMgU15hXFxvdGltZXMgU157LWF9Il0sWzMsMSwiRVxcb3RpbWVzIEUiXSxbMiwxLCJFXFxvdGltZXMgRVxcb3RpbWVzIEUiXSxbMCwxLCJcXGNvbmciXSxbMSwyLCJ4XFxvdGltZXMgeVxcb3RpbWVzIFNeey1hfSJdLFsyLDMsIkVcXG90aW1lcyBcXG11XFxvdGltZXMgU15hXFxvdGltZXMgU157LWF9Il0sWzMsNCwiRVxcb3RpbWVzIEVcXG90aW1lcyBcXHBoaV97YSwtYX1eey0xfSJdLFsyLDUsIkVcXG90aW1lcyBFXFxvdGltZXMgRVxcb3RpbWVzIFxccGhpX3thLC1hfV57LTF9IiwyXSxbNSw0LCJFXFxvdGltZXMgXFxtdSIsMl0sWzIsNCwiRVxcb3RpbWVzXFxtdVxcb3RpbWVzXFxwaGlfe2EsLWF9XnstMX0iLDFdXQ==
	\[\begin{tikzcd}
		{S^{b+c-a}} & {S^{b}\otimes S^c\otimes S^{-a}} & {E\otimes E\otimes E\otimes S^a\otimes S^{-a}} & {E\otimes E\otimes S^a\otimes S^{-a}} \\
		&& {E\otimes E\otimes E} & {E\otimes E}
		\arrow["\cong", from=1-1, to=1-2]
		\arrow["{x\otimes y\otimes S^{-a}}", from=1-2, to=1-3]
		\arrow["{E\otimes \mu\otimes S^a\otimes S^{-a}}", from=1-3, to=1-4]
		\arrow["{E\otimes E\otimes \phi_{a,-a}^{-1}}", from=1-4, to=2-4]
		\arrow["{E\otimes E\otimes E\otimes \phi_{a,-a}^{-1}}"', from=1-3, to=2-3]
		\arrow["{E\otimes \mu}"', from=2-3, to=2-4]
		\arrow["{E\otimes\mu\otimes\phi_{a,-a}^{-1}}"{description}, from=1-3, to=2-4]
	\end{tikzcd}\]
	Commutativity is functoriality of $-\otimes-$, which also tells us that the two outside compositions are $\wt{xy}$ (on top) and $x\cdot r$ (on the bottom), so they are equal as desired. On the other hand, in order to see that $r\cdot\wt e=y$, consider the following diagram (where here we have passed to a symmetric strict monoidal category):
	% https://q.uiver.app/#q=WzAsNyxbMCwwLCJTXmMiXSxbMiwwLCJTXmNcXG90aW1lcyBTXnstYX1cXG90aW1lcyBTXmEiXSxbMiwyLCJFXFxvdGltZXMgU15hXFxvdGltZXMgU157LWF9XFxvdGltZXMgRVxcb3RpbWVzIFNeYSJdLFswLDMsIkVcXG90aW1lcyBFXFxvdGltZXMgU15hIl0sWzAsMiwiRVxcb3RpbWVzIFNeYSJdLFsxLDEsIkVcXG90aW1lcyBTXmFcXG90aW1lcyBTXnstYX1cXG90aW1lcyBTXmEiXSxbMiwzLCJFXFxvdGltZXMgRVxcb3RpbWVzIFNeYSJdLFswLDEsIlxcY29uZyJdLFsxLDIsInlcXG90aW1lcyBTXnstYX1cXG90aW1lcyBlXFxvdGltZXMgU15hIl0sWzMsNCwiXFxtdVxcb3RpbWVzIFNeYSJdLFswLDQsInkiLDJdLFs1LDIsIkVcXG90aW1lcyBTXnthfVxcb3RpbWVzIFNeey1hfVxcb3RpbWVzIGVcXG90aW1lcyBTXmEiLDFdLFs1LDQsIkVcXG90aW1lcyBTXmFcXG90aW1lcyBcXHBoaV97LWEsYX1eey0xfSIsMV0sWzQsNiwiRVxcb3RpbWVzIGVcXG90aW1lcyBTXmEiLDFdLFs2LDMsIiIsMCx7ImxldmVsIjoyLCJzdHlsZSI6eyJoZWFkIjp7Im5hbWUiOiJub25lIn19fV0sWzIsNiwiRVxcb3RpbWVzIFxccGhpX3thLC1hfV57LTF9XFxvdGltZXMgRVxcb3RpbWVzIFNee2F9Il0sWzEsNSwieVxcb3RpbWVzIFNeey1hfVxcb3RpbWVzIFNeYSIsMV1d
	\[\begin{tikzcd}
		{S^c} && {S^c\otimes S^{-a}\otimes S^a} \\
		& {E\otimes S^a\otimes S^{-a}\otimes S^a} \\
		{E\otimes S^a} && {E\otimes S^a\otimes S^{-a}\otimes E\otimes S^a} \\
		{E\otimes E\otimes S^a} && {E\otimes E\otimes S^a}
		\arrow["\cong", from=1-1, to=1-3]
		\arrow["{y\otimes S^{-a}\otimes e\otimes S^a}", from=1-3, to=3-3]
		\arrow["{\mu\otimes S^a}", from=4-1, to=3-1]
		\arrow["y"', from=1-1, to=3-1]
		\arrow["{E\otimes S^{a}\otimes S^{-a}\otimes e\otimes S^a}"{description}, from=2-2, to=3-3]
		\arrow["{E\otimes S^a\otimes \phi_{-a,a}^{-1}}"{description}, from=2-2, to=3-1]
		\arrow["{E\otimes e\otimes S^a}"{description}, from=3-1, to=4-3]
		\arrow[Rightarrow, no head, from=4-3, to=4-1]
		\arrow["{E\otimes \phi_{a,-a}^{-1}\otimes E\otimes S^{a}}", from=3-3, to=4-3]
		\arrow["{y\otimes S^{-a}\otimes S^a}"{description}, from=1-3, to=2-2]
	\end{tikzcd}\]
	The top left triangle commutes since we may take the isomorphism $S^c\xr{\cong}S^c\otimes S^{-a}\otimes S^a$ to be $S^c\otimes\phi_{-a,a}$, in which case commutativity of the triangle follows by functoriality of $-\otimes-$. Commutativity of the right triangle is also functoriality of $-\otimes-$. Commutativity of the bottom left triangle is unitality of $\mu$. Finally, commutativity of the remaining middle $4$-sided region is again functoriality of $-\otimes-$. It follows that $y$ is equal to the outer composition, which is $r\cdot\wt e$, as desired. Thus, we have shown that
	\[\Psi(\Phi(x\otimes y))=\wt{xy}\otimes\wt e=(x\cdot r)\otimes\wt e=x\otimes(r\cdot\wt e)=x\otimes y,\] 
	as desired, so that for each $a\in A$, the object $S^a$ belongs to the class $\cE$. 
	
	Now, we would like to show that given a distinguished triangle
	\[X\xr fY\xr gZ\xr h\Sigma X,\]
	if two of three of the objects $X$, $Y$, and $Z$ belong to $\cE$, then so does the third. From now on, write $L^E_*:\cSH\to\pi_*(E)\text-\Mod$ to denote the functor $X\mapsto E_*(E)\otimes_{\pi_*(E)}E_*(X)$, so $\Phi$ is a natural transformation $L_*^E\Rightarrow E_*(E\otimes-)$. First, note that by \autoref{LES_remains_exact_after_tensor}, we have the following exact sequence in $\cSH$: 
	\[E\otimes \Omega Y\xr{E\otimes \Omega g}E\otimes \Omega Z\xr{E\otimes \wt h}E\otimes X\xr{E\otimes f}E\otimes Y\xr{E\otimes g}E\otimes Z\xr{E\otimes h}E\otimes \Sigma X\xr{E\otimes \Sigma f}\Sigma Y.\]
	We can then apply $[S^*,-]$ to it, which yields the following exact sequence of $A$-graded $\pi_*(E)$-modules:
	\[E_*(\Omega Y)\xr{E_*(\Omega g)}E_*(\Omega Z)\xr{E_*(\wt h)}E_*(X)\xr{E_*(f)}E_*(Y)\xr{E_*(g)}E_*(Z)\xr{E_*(h)}E_*(\Sigma X)\xr{E_*(f)}E_*(\Sigma Y).\]
	Now, we can tensor this sequence with $E_*(E)$ over $\pi_*(E)$, and since $E_*(E)$ is a flat right $\pi_*(E)$ module, we get that the top row in the following sequence is exact:
	% https://q.uiver.app/#q=WzAsMTQsWzAsMCwiTF8qXkUoXFxPbWVnYSBZKSJdLFsxLDAsIkxfKl5FKFxcT21lZ2EgWikiXSxbMiwwLCJMXypeRShYKSJdLFszLDAsIkxfKl5FKFkpIl0sWzQsMCwiTF8qXkUoWikiXSxbNSwwLCJMXypeRShcXFNpZ21hIFgpIl0sWzYsMCwiTF8qXkUoXFxTaWdtYSBZKSJdLFswLDEsIkVfKihFXFxvdGltZXNcXE9tZWdhIFkpIl0sWzEsMSwiRV8qKEVcXG90aW1lc1xcT21lZ2EgWikiXSxbMiwxLCJFXyooRVxcb3RpbWVzIFgpIl0sWzMsMSwiRV8qKEVcXG90aW1lcyBZKSJdLFs0LDEsIkVfKihFXFxvdGltZXMgWikiXSxbNSwxLCJFXyooRVxcb3RpbWVzIFxcU2lnbWEgWCkiXSxbNiwxLCJFXyooRVxcb3RpbWVzIFxcU2lnbWEgWSkiXSxbMCwxLCJMXypeRShcXE9tZWdhIGcpIl0sWzEsMiwiTF8qXkUoXFx3dCBoKSJdLFsyLDMsIkxfKl5FKGYpIl0sWzMsNCwiTF8qXkUoZykiXSxbNCw1LCJMXypeRShoKSJdLFs1LDYsIkxfKl5FKFxcU2lnbWEgZikiXSxbMCw3LCJcXFBoaV97XFxPbWVnYSBZfSIsMl0sWzcsOCwiRV8qKEVcXG90aW1lcyBcXE9tZWdhIGcpIiwyXSxbMTEsMTIsIkVfKihFXFxvdGltZXMgaCkiLDJdLFsxMiwxMywiRV8qKEVcXG90aW1lc1xcU2lnbWEgZikiLDJdLFsxLDgsIlxcUGhpX3tcXE9tZWdhIFp9IiwyXSxbOSwxMCwiRV8qKEVcXG90aW1lcyBmKSIsMl0sWzIsOSwiXFxQaGlfe1h9IiwyXSxbOCw5LCJFXyooRVxcb3RpbWVzXFx3dCBoKSIsMl0sWzMsMTAsIlxcUGhpX1kiLDJdLFsxMCwxMSwiRV8qKEVcXG90aW1lcyBnKSIsMl0sWzQsMTEsIlxcUGhpX1oiLDJdLFs1LDEyLCJcXFBoaV97XFxTaWdtYSBYfSIsMl0sWzYsMTMsIlxcUGhpX3tcXFNpZ21hIFl9IiwyXV0=
	\[\begin{tikzcd}[column sep=tiny]
		{L_*^E(\Omega Y)} & {L_*^E(\Omega Z)} & {L_*^E(X)} & {L_*^E(Y)} & {L_*^E(Z)} & {L_*^E(\Sigma X)} & {L_*^E(\Sigma Y)} \\
		{E_*(E\otimes\Omega Y)} & {E_*(E\otimes\Omega Z)} & {E_*(E\otimes X)} & {E_*(E\otimes Y)} & {E_*(E\otimes Z)} & {E_*(E\otimes \Sigma X)} & {E_*(E\otimes \Sigma Y)}
		\arrow["{L_*^E(\Omega g)}", from=1-1, to=1-2]
		\arrow["{L_*^E(\wt h)}", from=1-2, to=1-3]
		\arrow["{L_*^E(f)}", from=1-3, to=1-4]
		\arrow["{L_*^E(g)}", from=1-4, to=1-5]
		\arrow["{L_*^E(h)}", from=1-5, to=1-6]
		\arrow["{L_*^E(\Sigma f)}", from=1-6, to=1-7]
		\arrow["{\Phi_{\Omega Y}}"', from=1-1, to=2-1]
		\arrow["{E_*(E\otimes \Omega g)}"', from=2-1, to=2-2]
		\arrow["{E_*(E\otimes h)}"', from=2-5, to=2-6]
		\arrow["{E_*(E\otimes\Sigma f)}"', from=2-6, to=2-7]
		\arrow["{\Phi_{\Omega Z}}"', from=1-2, to=2-2]
		\arrow["{E_*(E\otimes f)}"', from=2-3, to=2-4]
		\arrow["{\Phi_{X}}"', from=1-3, to=2-3]
		\arrow["{E_*(E\otimes\wt h)}"', from=2-2, to=2-3]
		\arrow["{\Phi_Y}"', from=1-4, to=2-4]
		\arrow["{E_*(E\otimes g)}"', from=2-4, to=2-5]
		\arrow["{\Phi_Z}"', from=1-5, to=2-5]
		\arrow["{\Phi_{\Sigma X}}"', from=1-6, to=2-6]
		\arrow["{\Phi_{\Sigma Y}}"', from=1-7, to=2-7]
	\end{tikzcd}\]
	The diagram commutes since $\Phi$ is natural. The following sequence is exact in $\cSH$ by \autoref{LES_remains_exact_after_tensor},
	\[E\otimes E\otimes \Omega Y\to E\otimes E\otimes \Omega Z\to E\otimes E\otimes X\to E\otimes E\otimes Y\to E\otimes E\otimes Z\to E\otimes E\otimes \Sigma X\to E\otimes E\otimes \Sigma Y,\]
	so that the bottom row in the above diagram is also exact. Now, suppose two of three of $X$, $Y$, and $Z$ belong to $\cE$. By \autoref{t's_commute_with_Phi's}, \autoref{t_commutes_with_Phi's_corollary}, if $\Phi_W$ is an isomorphism then $\Phi_{\Omega W}$ and $\Phi_{\Sigma W}$ are. Thus by the five lemma, it follows that the middle three vertical arrows in the above diagram are necessarily all isomorphisms, so we have shown that $\cE$ is closed under two-of-three for exact triangles, as desired.

	Finally, it remains to show that $\cE$ is closed under arbitrary coproducts. Let $\{X_i\}_{i\in I}$ be a collection of objects in $\cE$ indexed by some (small) set $I$. Then we'd like to show that $X:=\bigoplus_iX_i$ belongs to $\cE$. First of all, note that $E\otimes-$ preserves arbitrary coproducts, as it has a right adjoint $F(E,-)$. Thus without loss of generality we may take $\bigoplus_iE\otimes X_i=E\otimes\bigoplus_iX_i$ (as $E\otimes\bigoplus_iX_i$ \emph{is} a coproduct of all the $E\otimes X_i$'s). Now, recall that we have chosen each $S^a$ to be a compact object (\autoref{defn_compact}), so that the canonical map
	\[s:\bigoplus_i E_*(X_i)=\bigoplus_i[S^*,E\otimes X_i]\to[S^*,\bigoplus_iE\otimes X_i]=[S^*,E\otimes X]=E_*(X)\]
	is an isomorphism, natural in $X_i$ for each $i$. Note in particular that it is an isomorphism of left $\pi_*(E)$-modules. To see this, first note it suffices to check that $s(r\cdot x)=r\cdot s(x)$ for some homogeneous $x\in E_*(X_i)$ for some $i$, as such $x$ generate $\bigoplus_i E_*(X_i)$ by definition, and $s$ is a homomorphism of abelian groups. Then given $r:S^a\to E\otimes E$ and $x:S^b\to E\otimes X_i$, consider the following diagram	
	% https://q.uiver.app/#q=WzAsOCxbMCwwLCJTXnthK2J9Il0sWzEsMCwiU15hXFxvdGltZXMgU15iIl0sWzIsMCwiRVxcb3RpbWVzIEVcXG90aW1lcyBFXFxvdGltZXMgWF9pIl0sWzMsMCwiRVxcb3RpbWVzIEVcXG90aW1lcyBcXGJpZ29wbHVzX2koRVxcb3RpbWVzIFhfaSkiXSxbMywxLCJFXFxvdGltZXMgRVxcb3RpbWVzIEVcXG90aW1lcyBYIl0sWzMsMiwiRVxcb3RpbWVzIEVcXG90aW1lcyBYIl0sWzIsMywiRVxcb3RpbWVzIEVcXG90aW1lcyBYX2kiXSxbMywzLCJFXFxvdGltZXMgXFxiaWdvcGx1c19pKEVcXG90aW1lcyBYX2kpIl0sWzAsMSwiXFxwaGlfe2EsYn0iXSxbMSwyLCJ4XFxvdGltZXMgeSJdLFsyLDMsIkVcXG90aW1lcyBFXFxvdGltZXMgXFxpb3RhX3tFXFxvdGltZXMgWF9pfSJdLFszLDQsIiIsMCx7ImxldmVsIjoyLCJzdHlsZSI6eyJoZWFkIjp7Im5hbWUiOiJub25lIn19fV0sWzQsNSwiRVxcb3RpbWVzIFxcbXVcXG90aW1lcyBYIl0sWzIsNiwiRVxcb3RpbWVzIFxcbXVcXG90aW1lcyBYX2kiLDJdLFs2LDcsIkVcXG90aW1lcyBcXGlvdGFfe0VcXG90aW1lcyBYX2l9IiwyXSxbNyw1LCIiLDEseyJsZXZlbCI6Miwic3R5bGUiOnsiaGVhZCI6eyJuYW1lIjoibm9uZSJ9fX1dLFs2LDUsIkVcXG90aW1lcyBFXFxvdGltZXMgXFxpb3RhX3tYX2l9IiwxXSxbMiw0LCJFXFxvdGltZXMgRVxcb3RpbWVzIEVcXG90aW1lcyBcXGlvdGFfe1hfaX0iLDFdXQ==
	\[\begin{tikzcd}
		{S^{a+b}} & {S^a\otimes S^b} & {E\otimes E\otimes E\otimes X_i} & {E\otimes E\otimes \bigoplus_i(E\otimes X_i)} \\
		&&& {E\otimes E\otimes E\otimes X} \\
		&&& {E\otimes E\otimes X} \\
		&& {E\otimes E\otimes X_i} & {E\otimes \bigoplus_i(E\otimes X_i)}
		\arrow["{\phi_{a,b}}", from=1-1, to=1-2]
		\arrow["{x\otimes y}", from=1-2, to=1-3]
		\arrow["{E\otimes E\otimes \iota_{E\otimes X_i}}", from=1-3, to=1-4]
		\arrow[Rightarrow, no head, from=1-4, to=2-4]
		\arrow["{E\otimes \mu\otimes X}", from=2-4, to=3-4]
		\arrow["{E\otimes \mu\otimes X_i}"', from=1-3, to=4-3]
		\arrow["{E\otimes \iota_{E\otimes X_i}}"', from=4-3, to=4-4]
		\arrow[Rightarrow, no head, from=4-4, to=3-4]
		\arrow["{E\otimes E\otimes \iota_{X_i}}"{description}, from=4-3, to=3-4]
		\arrow["{E\otimes E\otimes E\otimes \iota_{X_i}}"{description}, from=1-3, to=2-4]
	\end{tikzcd}\]
	where $\iota_{E\otimes X_i}:E\otimes X_i\into\bigoplus_i(E\otimes X_i)$ and $\iota_{X_i}:X_i\into\bigoplus_iX_i$ are the maps determined by universal property of the coproduct. Commutativity of the two triangles is again by the fact that $E\otimes-$ is colimit preserving. Commutativity of the trapezoid is functoriality of $-\otimes-$. Thus, the top arrow in the following diagram is well-defined:
	% https://q.uiver.app/#q=WzAsNixbMSwwLCJFXyooRSlcXG90aW1lc197XFxwaV8qKEUpfVxcYmlnb3BsdXNfaUVfKihYX2kpIl0sWzAsMiwiXFxiaWdvcGx1c19pRV8qKEVcXG90aW1lcyBYX2kpIl0sWzIsMiwiRV8qKEVcXG90aW1lcyBYKSJdLFsyLDAsIkVfKihFKVxcb3RpbWVzX3tcXHBpXyooRSl9IEVfKihYKSJdLFswLDAsIlxcYmlnb3BsdXNfaUVfKihFKVxcb3RpbWVzX3tcXHBpXyooRSl9RV8qKFhfaSkiXSxbMSwyLCJFXyooXFxiaWdvcGx1c19pRVxcb3RpbWVzIFhfaSkiXSxbMCwzLCJFXyooRSlcXG90aW1lc197XFxwaV8qKEUpfSBzIl0sWzMsMiwiXFxQaGlfWCJdLFs0LDAsIiIsMCx7ImxldmVsIjoyLCJzdHlsZSI6eyJoZWFkIjp7Im5hbWUiOiJub25lIn19fV0sWzEsNSwicyJdLFs1LDIsIiIsMCx7ImxldmVsIjoyLCJzdHlsZSI6eyJoZWFkIjp7Im5hbWUiOiJub25lIn19fV0sWzQsMSwiXFxiaWdvcGx1c19pXFxQaGlfe1hfaX0iLDJdXQ==
	\begin{equation}\label{kunneth_iso_pt_3_diag}\begin{tikzcd}
		{\bigoplus_iE_*(E)\otimes_{\pi_*(E)}E_*(X_i)} & {E_*(E)\otimes_{\pi_*(E)}\bigoplus_iE_*(X_i)} & {E_*(E)\otimes_{\pi_*(E)} E_*(X)} \\
		\\
		{\bigoplus_iE_*(E\otimes X_i)} & {E_*(\bigoplus_iE\otimes X_i)} & {E_*(E\otimes X)}
		\arrow["{E_*(E)\otimes_{\pi_*(E)} s}", from=1-2, to=1-3]
		\arrow["{\Phi_X}", from=1-3, to=3-3]
		\arrow[Rightarrow, no head, from=1-1, to=1-2]
		\arrow["s", from=3-1, to=3-2]
		\arrow[Rightarrow, no head, from=3-2, to=3-3]
		\arrow["{\bigoplus_i\Phi_{X_i}}"', from=1-1, to=3-1]
	\end{tikzcd}\end{equation}
	We wish to show this diagram commutes. Again, since each map here is a homomorphism, it suffices to chase generators. By definition, a generator of the top left element is a homogeneous pure tensor in $E_*(E)\otimes_{\pi_{*}(E)}E_*(X_i)$ for some $i$ in $I$. Given classes $x:S^a\to E\otimes E$ and $y:S^b\to E\otimes X_i$, consider the following diagram:
	% https://q.uiver.app/#q=WzAsOCxbMCwwLCJTXnthK2J9Il0sWzEsMCwiU15hXFxvdGltZXMgU15iIl0sWzIsMCwiRVxcb3RpbWVzIEVcXG90aW1lcyBFXFxvdGltZXMgWF9pIl0sWzMsMCwiRVxcb3RpbWVzIEVcXG90aW1lcyBcXGJpZ29wbHVzX2lFXFxvdGltZXMgWF9pIl0sWzMsMiwiRVxcb3RpbWVzIEVcXG90aW1lcyBYIl0sWzIsMSwiRVxcb3RpbWVzIEVcXG90aW1lcyBYX2kiXSxbMywxLCJFXFxvdGltZXMgRVxcb3RpbWVzIEVcXG90aW1lcyBYIl0sWzIsMiwiXFxiaWdvcGx1c19pRVxcb3RpbWVzIEVcXG90aW1lcyBYX2kiXSxbMCwxLCJcXHBoaV97YSxifSJdLFsxLDIsInhcXG90aW1lcyB5Il0sWzIsMywiRVxcb3RpbWVzIEVcXG90aW1lcyBcXGlvdGFfe0VcXG90aW1lcyBYX2l9Il0sWzIsNSwiRVxcb3RpbWVzIFxcbXVcXG90aW1lcyBYX2kiLDJdLFszLDYsIiIsMCx7ImxldmVsIjoyLCJzdHlsZSI6eyJoZWFkIjp7Im5hbWUiOiJub25lIn19fV0sWzYsNCwiRVxcb3RpbWVzIFxcbXVcXG90aW1lcyBYIl0sWzUsNywiXFxpb3RhX3tFXFxvdGltZXMgRVxcb3RpbWVzIFhfaX0iLDJdLFs3LDQsIiIsMSx7ImxldmVsIjoyLCJzdHlsZSI6eyJoZWFkIjp7Im5hbWUiOiJub25lIn19fV0sWzUsNCwiRVxcb3RpbWVzIEVcXG90aW1lcyBcXGlvdGFfe1hfaX0iLDFdLFsyLDYsIkVcXG90aW1lcyBFXFxvdGltZXMgRVxcb3RpbWVzIFxcaW90YV97WF9pfSIsMV1d
	\[\begin{tikzcd}
		{S^{a+b}} & {S^a\otimes S^b} & {E\otimes E\otimes E\otimes X_i} & {E\otimes E\otimes \bigoplus_iE\otimes X_i} \\
		&& {E\otimes E\otimes X_i} & {E\otimes E\otimes E\otimes X} \\
		&& {\bigoplus_iE\otimes E\otimes X_i} & {E\otimes E\otimes X}
		\arrow["{\phi_{a,b}}", from=1-1, to=1-2]
		\arrow["{x\otimes y}", from=1-2, to=1-3]
		\arrow["{E\otimes E\otimes \iota_{E\otimes X_i}}", from=1-3, to=1-4]
		\arrow["{E\otimes \mu\otimes X_i}"', from=1-3, to=2-3]
		\arrow[Rightarrow, no head, from=1-4, to=2-4]
		\arrow["{E\otimes \mu\otimes X}", from=2-4, to=3-4]
		\arrow["{\iota_{E\otimes E\otimes X_i}}"', from=2-3, to=3-3]
		\arrow[Rightarrow, no head, from=3-3, to=3-4]
		\arrow["{E\otimes E\otimes \iota_{X_i}}"{description}, from=2-3, to=3-4]
		\arrow["{E\otimes E\otimes E\otimes \iota_{X_i}}"{description}, from=1-3, to=2-4]
	\end{tikzcd}\]
	Unravelling definitions, the two outside compositions are the two ways to chase $x\otimes y$ around diagram (\ref{kunneth_iso_pt_3_diag}). The two triangles commute again by the fact that $-\otimes-$ preserves colimits in each argument. Commutativity of the inner parallelogram is functoriality of $-\otimes-$. Thus diagram (\ref{kunneth_iso_pt_3_diag}) tells us $\Phi_X$ is an isomorphism, since $\Phi_{X_i}$ is an isomorphism for each $i$ in $I$, meaning $\bigoplus_i\Phi_{X_i}$ is as well.
\end{proof}

\begin{proposition}
	Let $(E,\mu,e)$ be a ring spectrum in $\cSH$, and let $X$ and $Y$ be two objects in $\cSH$ such that $E$ and $X$ are both cellular (\autoref{cellular}) and $E_*(X)$ is a projective left $\pi_*(E)$-module (\autoref{module}). Then the map
	\[{[X,E\otimes Y]}_*\to\Hom_{\pi_*(E)}^*(E_*(X),E_*(Y))\]
	which sends a generator $f:S^a\otimes X\to E\otimes Y$ in ${[X,E\otimes Y]}_*$ to the assignment which sends a generator $x:S^b\to E\otimes X$ in $E_*(X)$ to the composition
	\[S^{a+b}\to S^a\otimes S^b\xr{S^a\otimes x}S^a\otimes E\otimes X\xr{\tau\otimes X}E\otimes S^a\otimes X\xr{E\otimes f}E\otimes E\otimes Y\xr{\mu}E\otimes Y\]
	is an $A$-graded isomorphism of $A$-graded abelian groups.
\end{proposition}

%In the following definition, let $\vare:E_*(E)\to \pi_*(E)$ be the map which sends some $\alpha:S^a\to E\otimes E$ to the composition
%\[S^a\xr\alpha E\otimes E\xr\mu E.\]
%Also define $\Psi:E_*(E)\to E_*(E)\otimes_{\pi_*(E)}E_*(E)$ to be the map which factors as
%\[E_*(E)\to E_*(E\otimes E)\xr\cong E_*(E)\otimes_{\pi_*(E)}E_*(E)\]
%where the second arrow is the isomorphism prescribed by \autoref{Kunneth_map}, and the first arrow sends a class $\alpha:S^a\to E\otimes E$ to the composition
%\[S^a\xr\alpha E\otimes E\cong E\otimes S\otimes E\xr{E\otimes e\otimes E}E\otimes E\otimes E.\]
%
%\begin{lemma}[{\cite[Proposition 2.30, 2.33]{nlab:introduction_to_the_adams_spectral_sequence}}]\label{2.30_2.33}
	%Let $E$ be a flat commutative ring spectrum, and let $X$ and $Y$ be spectra such that $E_\aast(X)$ is a projective module over $\pi_\aast(E)$. Then for all $s\geq0$ and $t,w\in\bZ$, there is an isomorphism
	%\[\Phi:[X,E\wedge Y]_{t,w}\to\Hom_{E_\aast(E)}^{t,w}(E_\aast(X),E_\aast(E\wedge Y)),\]
	%obtained by sending a class $f:S^{t,w}\wedge X\to E\wedge Y$ in $[X,E\wedge Y]_{t,w}$ to the map
	%\[\Phi_f:E_\acast(X)\to E_{\ast+t,\ast+w}(X\wedge Y)\]
	%sending
	%\[[S^{a,b}\xr gE\wedge X]\mapsto[S^{a+t,b+w}\cong S^{a,b}\wedge S^{t,w}\xr{g\wedge S^{t,w}}E\wedge X\wedge S^{t,w}\cong E\wedge S^{t,w}\wedge X\xr{E\wedge f}E\wedge E\wedge Y].\]
%\end{lemma}
%\begin{proof}
	%Let $f:S^{t,w}\wedge X\to E\wedge Y$. First we want to show that $\Phi_f$ is actually an $E_\aast(E)$-comodule homomorphism.\todo{finish}
%\end{proof}

\end{document}
