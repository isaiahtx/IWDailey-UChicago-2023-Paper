\documentclass[../main.tex]{subfiles}
\begin{document}

In this appendix, we fix an abelian group $A$ once and for all. We assume the reader is familiar with the basic theory of (non-commutative, unital) rings and modules over them.

\subsection{\texorpdfstring{$A$}{A}-graded abelian groups, rings, and modules}

\begin{definition}\label{graded_abgrp}
	An \emph{$A$-graded abelian group} is an abelian group $B$ along with a subgroup $B_a\leq B$ for each $a\in A$ such that the canonical map
	\[\bigoplus_{a\in A}B_a\to B\]
	sending $(x_a)_{a\in A}$ to $\sum_{a\in A}x_a$ is an isomorphism. Given two $A$-graded abelian groups $B$ and $C$, a homomorphism $f:B\to C$ is a \textit{homomorphism of $A$-graded abelian groups}, or just an \emph{$A$-graded homomorphism}, if it preserves the grading, i.e., if it restricts to a map $B_a\to C_a$ for all $a\in A$. 

	We denote the category of $A$-graded abelian groups and $A$-graded homomorphisms between them by $\Ab^A$
\end{definition}

It is easy to see that an $A$-graded abelian group $B$ is generated by its \emph{homogeneous} elements, that is, nonzero elements $x\in B$ such that there exists some $a\in A$ with $x\in B_a$. Furthermore, by the universal property of the coproduct, given two $A$-graded abelian groups $B$ and $C$, the data of an $A$-graded homomorphism $\varphi_a:B\to C$ is precisely the data of homomorphisms $\varphi_a:B_a\to C_a$. 

\begin{remark}
	Clearly the condition that the canonical map $\bigoplus_{a\in A}B_a\to B$ is an isomorphism requires that $B_a\cap B_b=0$ if $a\neq b$. In particular, given a homogeneous element $x\in B$, there exists precisely one $a\in A$ such that $x\in B_a$. We call this $a$ the \emph{degree} of $x$, and we write $|x|=a$.
\end{remark}

\begin{definition}
	An \emph{$A$-graded ring} is a ring $R$ such that its underlying abelian group $R$ is $A$-graded and the multiplication map $R\times R\to R$ restricts to $R_a\times R_b\to R_{a+b}$ for all $a,b\in A$. A morphism of $A$-graded rings is a ring homomorphism whose underlying homomorphism of abelian groups is $A$-graded.
\end{definition}

Explicitly, given an $A$-graded ring $R$ and homogeneous elements $x,y\in R$, we must have $|xy|=|x|+|y|$. For example, given some field $k$, the ring $R=k[x,y]$ is $\bZ^2$-graded, where given $(n,m)\in\bZ^2$, $R_{n,m}$ is the subgroup of those monomials of the form $ax^ny^m$ for some $a\in k$. 

\begin{definition}
	Let $R$ be an $A$-graded ring. A \emph{left $A$-graded $R$-module} $M$ is a left $R$-module $M$ such that $M$ is an $A$-graded abelian group and the action map $R\times M\to M$ restricts to a map $R_a\times M_b\to M_{a+b}$ for all $a,b\in A$. Right $A$-graded $R$-modules are defined similarly. Finally, an $A$-graded $R$-bimodule is an $A$-graded abelian group $M$ which has the structure of both an $A$-graded left and right $R$-module such that given $r,s\in R$ and $m\in M$, $r\cdot(m\cdot s)=(r\cdot m)\cdot s$. 
	
	Morphisms between $A$-graded $R$-modules are precisely $R$-module homomorphisms whose underlying group homomorphisms are $A$-graded. We write $R\text-\Mod^A$ for the category of left $A$-graded $R$-modules and $\Mod^A\text-R$ for the category of right $A$-graded $R$-modules.
\end{definition}

\begin{remark}
	It is straightforward to see that an $A$-graded abelian group is equivalently an $A$-graded $\bZ$-module, where here we are considering $\bZ$ as an $A$-graded ring concentrated in degree $0$. Thus any result below about $A$-graded modules applies equally to $A$-graded abelian groups.
\end{remark}

\begin{remark}
	We often will denote an $A$-graded $R$-module $M$ by $M_*$. Given some $a\in A$, we can define the shifted $A$-graded abelian group $M_{*+a}$ whose $b^\text{th}$ component is $M_{b+a}$. We will also sometimes write $\Sigma^aM$ to denote the shifted module $M_{*-a}$.
\end{remark}

\begin{definition}
    More generally, given two $A$-graded $R$-modules $M$ and $N$ and some $d\in A$, an $R$-module homomorphism $f:M\to N$ is \emph{an $A$-graded homomorphism of degree $d$} if it restricts to a map $M_{a}\to N_{a+d}$ for all $a\in A$. Thus, an $A$-graded homomorphism of degree $d$ from $M$ to $N$ is equivalently an $A$-graded homomorphism $M_{*}\to N_{*+d}$ or an $A$-graded homomorphism $M_{*-d}\to N$. Given some $a\in A$ and left (resp.\ right) $R$-modules $M$ and $N$, we will write 
	\[\Hom_R^d(M,N)=\Hom_R(M_*,N_{*+d})=\Hom_R(M_{*-d},N_*)\]
	to denote the set of $A$-graded homomorphisms of degree $d$ from $M$ to $N$, and simply
	\[\Hom_R(M,N)\] 
	to denote the set of degree-$0$ $A$-graded homomorphisms from $M$ to $N$. Clearly $A$-graded homomorphisms may be added and subtracted, so these are further abelian groups. Thus we have an $A$-graded abelian group
	\[\Hom_R^*(M,N).\]
\end{definition}

Unless stated otherwise, an ``$A$-graded homomorphism'' will always refer to an $A$-graded homomorphism of degree $0$. 


Oftentimes when constructing $A$-graded rings, we do so only by defining the product of homogeneous elements, like so:

\begin{lemma}\label{A_graded_ring}
	Suppose we have an $A$-graded abelian group $R$, a distinguished element $1\in R_0$, and $\bZ$-bilinear maps $m_{a,b}:R_a\times R_b\to R_{a+b}$ for all $a,b\in A$. Further suppose that for all $x\in R_a$, $y\in R_b$, and $z\in R_c$, we have
	\[m_{a+b,c}(m_{a,b}(x,y),z)=m_{a,b+c}(x,m_{b,c}(y,z))\qquad\text{and}\qquad m_{a,0}(x,1)=m_{0,a}(1,x)=x.\]
	Then there exists a unique multiplication map $m:R\times R\to R$ which endows $R$ with the structure of an $A$-graded ring and restricts to $m_{a,b}$ for all $a,b\in A$.
\end{lemma}
\begin{proof}
	Given $r,s\in R$, since $R\cong\bigoplus_{a\in A}R_a$, we may uniquely decompose $r$ and $s$ into homogeneous elements as $r=\sum_{a\in A}r_a$ and $s=\sum_{a\in A}s_a$ with each $r_a,s_a\in R_a$ such that only finitely many of the $r_a$'s and $s_a$'s are nonzero. Then in order to define a distributive product $R\times R\to R$ which restricts to $m_{a,b}:R_a\times R_b\to R_{a+b}$, note we \emph{must} define
	\[r\cdot s=\(\sum_{a\in A}r_a\)\cdot\(\sum_{b\in A}s_b\)=\sum_{a,b\in A}r_a\cdot s_b=\sum_{a,b\in A}m_{a,b}(r_a,s_b).\]
	Thus, we have shown uniqueness. It remains to show this product actually gives $R$ the structure of a ring.  First we claim that the sum on the right is actually finite. Note there exists only finitely many nonzero $r_a$'s and $s_b$'s, and if $s_b=0$ then 
	\[m_{a,b}(r_a,0)=m_{a,b}(r_a,0+0)\overset{(*)}=m_{a,b}(r_a,0)+m_{a,b}(r_a,0)\implies m_{a,b}(r_a,0)=0,\]
	where $(*)$ follows from bilinearity of $m_{a,b}$. A similar argument yields that $m_{a,b}(0,s_b)=0$ for all $a,b\in A$. Hence indeed $m_{a,b}(r_a,s_b)$ is zero for all but finitely many pairs $(a,b)\in A^2$, as desired. Observe that in particular
	\[(r\cdot s)_a=\sum_{b+c=a}m_{b,c}(r_b,s_c)=\sum_{b\in A}m_{b,a-b}(r_b,s_{a-b})=\sum_{c\in A}m_{a-c,c}(r_{a-c},s_{c}).\]
	Now we claim this multiplication is associative. Given $t=\sum_{a\in A}t_a\in R$, we have
	\begin{align*}
		(r\cdot s)\cdot t&=\sum_{a,b\in A}m_{a,b}((r\cdot s)_a,t_b) \\
		&=\sum_{a,b\in A}m_{a,b}\(\sum_{c\in A}m_{a-c,c}(r_{a-c},s_{c}),t_b\) \\
		&\overset{(1)}=\sum_{a,b,c\in A}m_{a,b}(m_{a-c,c}(r_{a-c},s_{c}),t_b) \\
		&\overset{(2)}=\sum_{a,b,c\in A}m_{c,a+b-c}(r_c,m_{a-c,b}(s_{a-c},t_b)) \\
		&\overset{(3)}=\sum_{a,b,c\in A}m_{a,c}(r_a,m_{b,c-b}(s_b,t_{c-b})) \\
		&\overset{(1)}=\sum_{a,c\in A}m_{a,c}\left(r_a,\sum_{b\in A}m_{b,c-b}(s_b,t_{c-b})\right) \\
		&=\sum_{a,c\in A}m_{a,c}(r_a,(s\cdot t)_c)=r\cdot(s\cdot t),
	\end{align*}
	where each occurrence of $(1)$ follows by bilinearity of the $m_{a,b}$'s, each occurrence of $(2)$ is associativity of the $m_{a,b}$'s, and $(3)$ is obtained by re-indexing by re-defining $a:=c$, $b:=a-c$, and $c:=a+b-c$. Next, we wish to show that the distinguished element $1\in R_0$ is a unit with respect to this multiplication. Indeed, we have
	\[1\cdot r\overset{(1)}=\sum_{a\in A}m_{0,a}(1,r_a)\overset{(2)}=\sum_{a\in A}r_a=r\qquad\text{and}\qquad r\cdot 1\overset{(1)}=\sum_{a\in A}m_{a,0}(r_a,1)\overset{(2)}=\sum_{a\in A}r_a=r,\]
	where $(1)$ follows by the fact that $m_{a,b}(0,-)=m_{a,b}(-,0)=0$, which we have shown above, and $(2)$ follows by unitality of the $m_{0,a}$'s and $m_{0,a}$'s, respectively. Finally, we wish to show that this product is distributive. Indeed, we have
	\begin{align*}
		r\cdot(s+t)&=\sum_{a,b\in A}m_{a,b}(r_a,(s+t)_b) \\
		&=\sum_{a,b\in A}m_{a,b}(r_a,s_b+t_b) \\
		&\overset{(*)}=\sum_{a,b\in A}m_{a,b}(r_a,s_b)+\sum_{a,b\in A}m_{a,b}(r_a,t_b)=(r\cdot s)+(r\cdot t),
	\end{align*}
	where $(*)$ follows by bilinearity of $m_{a,b}$. An entirely analagous argument yields that $(r+s)\cdot t=(r\cdot t)+(s\cdot t)$.
\end{proof}

Similarly, when defining $A$-graded modules, we will only define the action maps for homogeneous elements:

\begin{lemma}\label{A-graded_module}
	Let $R$ be an $A$-graded ring, $M$ an $A$-graded abelian group, and suppose there exists $\bZ$-bilinear maps $\kappa_{a,b}:R_a\times M_b\to M_{a+b}$ for all $a,b\in A$. Further suppose that for all $r\in R_a$, $r'\in R_b$, and $m\in M_c$ that
	\[\kappa_{a+b,c}(r\cdot r',m)=\kappa_{a,b+c}(r,\kappa_{b,c}(r',m))\qquad\text{and}\qquad\kappa_{0,c}(1,m)=m.\]
	Then there is a unique map $\kappa:R\times M\to M$ which endows $M$ with the structure of a left $A$-graded $R$-module and restricts to $\kappa_{a,b}$ for all $a,b\in A$.

	On the other hand, suppose there exists $\bZ$-bilinear maps $\kappa_{a,b}:M_a\times R_b\to M_{a+b}$ for all $a,b\in A$. Further suppose that for all $r\in R_a$, $r'\in R_b$, and $m\in M_c$ that
	\[\kappa_{c,a+b}(m,r\cdot r')=\kappa_{c+a,b}(\kappa_{c,a}(m,r),r')\qquad\text{and}\qquad\kappa_{c,0}(m,1)=m.\]
	Then there is a unique map $\kappa:M\times R\to M$ which endows $M$ with the structure of a right $A$-graded $R$-module and restricts to $\kappa_{a,b}$ for all $a,b\in A$.

	Finally, if we have maps $\lambda_{a,b}:R_a\times M_b\to M_{a+b}$ and $\rho_{a,b}:M_a\times R_b\to M_{a+b}$ satisfying all of the above conditions, and if we further have that
	\[\lambda_{a,b+c}(r,\rho_{b,c}(x,s))=\rho_{a+b,c}(\lambda_{a,b}(r,x),s)\]
	for all $r\in R_a$, $x\in M_b$, and $s\in R_c$, then the left and right $A$-graded $R$-module structures induced on $M$ by the $\lambda$'s and $\rho$'s give $M$ the structure of an $A$-graded $R$-bimodule.
\end{lemma}
\begin{proof}
	Checking this all is straightforward albeit tedious; we leave the proof as an exercise for the reader.
\end{proof}

When working with $A$-graded rings and modules, we will often freely use the above propositions without comment. 

\begin{lemma}\label{ev_at_1_is_iso}
	Let $R$ be an $A$-graded ring, and let $M$ be an $A$-graded left (resp.\ right) $R$-module. Then for all $d\in A$, the evaluation map
	\begin{align*}
		\mathrm{ev_1}:\Hom_{R}^d(R,M)&\to M_d \\
		\varphi&\mapsto\varphi(1)
	\end{align*}
	is an isomorphism of abelian groups.
\end{lemma}
\begin{proof}
	We consider the case that $M$ is a left $A$-graded $R$-module, as showing it when $M$ is a right module is entirely analagous. First of all, this map is clearly a homomorphism, as given degree $d$ $A$-graded homomorphisms $\varphi,\psi:R\to M$, we have 
	\[\mathrm{ev}_1(\varphi+\psi)=(\varphi+\psi)(1)=\varphi(1)+\psi(1)=\mathrm{ev}_1(\varphi)+\mathrm{ev}_1(\psi).\]
	Now, to see it is surjective, let $m\in M_d$, and define $\varphi_m:R\to M$ to send $r\mapsto r\cdot m$. First of all, $\varphi_m$ is a module homomorphism, as given $r,s\in R$, 
	\[\varphi_m(r+s)=(r+s)\cdot m=r\cdot m+s\cdot m=\varphi_m(r)+\varphi_m(s)\quad\text{and}\quad\varphi_m(r\cdot s)=r\cdot s\cdot m=r\cdot\varphi_m(s).\]
	Furthermore, it is clearly $A$-graded of degree $d$, as given a homogeneous element $r\in R_a$ for some $a\in A$, we have $\varphi_m(r)=r\cdot m\in R_{a+d}$, since $m$ is homogeneous of degree $d$. Finally, clearly 
	\[\mathrm{ev}_1(\varphi_m)=\varphi_m(1)=1\cdot m=m,\]
	so indeed $\mathrm{ev}_1$ is surjective. On the other hand, to see it is injective, suppose we are given $\varphi,\psi\in\Hom_R^d(R,M)$ such that $\varphi(1)=\psi(1)$. Then given $r\in R$, we must have
	\[\varphi(r)=\varphi(r\cdot 1)=r\cdot\varphi(1)=r\cdot\psi(1)=\psi(r\cdot 1)=\psi(r),\]
	so $\varphi$ and $\psi$ are exactly the same map. Thus, $\mathrm{ev}_1$ is injective, as desired.
\end{proof}

\subsection{Tensor products of \texorpdfstring{$A$}{A}-graded modules}

\begin{lemma}\label{product_of_A_graded}
	Given an $A$-graded ring $R$ and two left (resp.\ right) $A$-graded $R$-modules $M$ and $N$, their direct sum $M\oplus N$ is naturally a left (resp.\ right) $A$-graded $R$-module by defining
	\[(M\oplus N)_a:=M_a\oplus N_a.\]
\end{lemma}
\begin{proof}
	The canonical map $\bigoplus_{a\in A}(M_a\oplus N_a)\to M\oplus N$ factors as
	\[\bigoplus_{a\in A}(M_a\oplus N_a)\xr\cong\bigoplus_{a\in A}M_a\oplus\bigoplus_{a\in A}N_a\xr\cong M\oplus N.\qedhere\]
\end{proof}

Recall that given a ring $R$, a left $R$-module $M$, a right $R$-module $N$, and an abelian group $A$, an \emph{$R$-balanced map} $\varphi:M\times N\to B$ is one which satisifies
\begin{align*}
	\varphi(m,n+n')&=\varphi(m,n)+\varphi(m,n') \\
	\varphi(m+m',n)&=\varphi(m,n)+\varphi(m',n) \\
	\varphi(m\cdot r,n)&=\varphi(m,r\cdot n)
\end{align*}
for all $m,m'\in M$, $n,n'\in N$, and $r\in R$. Then the tensor product $M\otimes_RN$ is the universal abelian group equipped with an $R$-balanced map $\otimes:M\times N\to M\otimes_RN$ such that for every abelian group $B$ and every $R$-balanced map $\varphi:M\times N\to B$, there is a \emph{unique} group homomorphism $\wt\varphi:M\otimes_RN\to B$ such that $\wt f\circ\otimes=f$. We call elements in the image of $\otimes:M\times N\to M\otimes_RN$ \emph{pure tensors}. It is a standard fact that $M\otimes_RN$ is generated as an abelian group by its pure tensors.

\begin{definition}\label{A-graded_R-balanced_map_defn}
	Suppose we have a right $A$-graded $R$-module $M$, a left $A$-graded $R$-module $N$, and an $A$-graded abelian group $B$. Then an \emph{$A$-graded $R$-balanced map} $\varphi:M\times N\to B$ is an $R$-balanced map which restricts to $M_a\times N_b\to B_{a+b}$ for all $a,b\in A$.
\end{definition}

\begin{proposition}\label{tensor_of_A_graded_is_A_graded}
	Suppose we have a right $A$-graded $R$-module $M$ and a left $A$-graded $R$-module $N$. Then the tensor product
	\[M\otimes_RN\]
	naturally inherits the structure of an $A$-graded abelian group by defining $(M\otimes_RN)_a$ to be the subgroup generated by \emph{homogeneous} pure tensors, i.e., those elements $m\otimes n$ with $m\in M_b$ and $n\in N_c$ such that $b+c=a$. Furthermore, if either $M$ (resp.\ $N$) is an $A$-graded bimodule, then this decomposition makes $M\otimes_RN$ into a left (resp.\ right) $A$-graded $R$-module. In particular, if both $M$ and $N$ are $R$-bimodules, then $M\otimes_RN$ is an $A$-graded $R$-bimodule.
\end{proposition}
\begin{proof}
	By definition, since $M$ and $N$ are $A$-graded abelian groups, they are generated (as abelian groups) by their homogeneous elements. Thus it follows that $M\otimes_RN$ is generated by its homogeneous pure tensors, as defined above. Now, given a homogeneous pure tensor $m\otimes n$ in $M\otimes_RN$, it is clear that defining its degree by the formula $|m\otimes n|:=|m|+|n|$ is perfectly well-defined, as given homogeneous elements $m\in M$, $n\in N$, and $r\in R$ we have that
	\[|(m\cdot r)\otimes n|=|m\cdot r|+|n|=|m|+|r|+|n|=|m|+|r\cdot n|=|m\otimes(r\cdot n)|.\]
	Thus, we may define $(M\otimes_RN)_a$ to be the subgroup of $M\otimes_RN$ generated by those pure homogeneous tensors of degree $a$. Now, consider the map
	\[\Psi:M\times N\to\bigoplus_{a\in A}(M\otimes_RN)_a\]
	which takes a pair $(m,n)=\sum_{a\in A}(m_a,n_a)$ to the element $\Psi(m,n)$ whose $a^\text{th}$ component is
	\[(\Psi(m,n))_a:=\sum_{b+c=a}m_b\otimes n_c.\]
	It is straightforward to see that this map is $R$-balanced, in the sense that it is additive in each argument and $\Psi(m\cdot r,n)=\Psi(m,r\cdot n)$ for all $m\in M$, $n\in N$, and $r\in R$. Thus by the universal property of $M\otimes_RN$, we get a homomorphism of abelian groups $\wt\Psi:M\otimes_RN\to\bigoplus_{a\in A}(M\otimes_RN)_a$ lifting $\Psi$ along the canonical map $M\times N\to M\otimes_RN$. Now, also consider the canonical map
	\[\Phi:\bigoplus_{a\in A}(M\otimes_RN)_a\to M\otimes_RN.\]
	We would like to show $\wt\Psi$ and $\Phi$ are inverses of eah other. Since $\wt\Psi$ and $\Phi$ are both homomorphisms, it suffices to show this on generators. Let $m\otimes n$ be a homogeneous pure tensor with $m=m_a\in M_a$ and $n=n_b\in N_b$. Then we have
	\[\Phi(\wt\Psi(m\otimes n))=\Phi\(\bigoplus_{a\in A}\sum_{b+c=a}m_b\otimes n_c\)\overset{(*)}=\Phi(m\otimes n)=m\otimes n,\]
	and
	\[\wt\Psi(\Phi(m\otimes n))=\wt\Psi(m\otimes n)=\bigoplus_{a\in A}\sum_{b+c=a}m_b\otimes n_c\overset{(*)}=m\otimes n,\]
	where both occurrences of $(\ast)$ follow by the fact that $m_b\otimes n_c=0$ unless $b=c=a$, in which case $m_a\otimes n_a=m\otimes n$. Thus since $\Phi$ is an isomorphism, $M\otimes_RN$ is indeed an $A$-graded abelian group, as desired.

	Now, suppose that $M$ is an $A$-graded $R$-bimodule, so there exists left and right $A$-graded actions of $R$ on $M$ such that given $r,s\in R$ and $m\in M$ we have $r\cdot(m\cdot s)=(r\cdot m)\cdot s$. Then we would like to show that given a left $A$-graded $R$-module $N$ that $M\otimes_RN$ is canonically a left $A$-graded $R$-module. Indeed, define the action of $R$ on $M\otimes_RN$ on pure tensors by the formula
	\[r\cdot(m\otimes n)=(r\cdot m)\otimes n.\]
	First of all, clearly this map is $A$-graded, as if $r\in R_a$, $m\in M_b$, and $n\in N_c$ then $(r\cdot m)\otimes n$, by definition, has degree $|r\cdot m|+|n|=|r|+|m|+|n|$ (the last equality follows since the left action of $R$ on $M$ is $A$-graded). In order to show the above map defines a left module structure, it suffices to show that given pure tensors $m\otimes n,m'\otimes n'\in M\otimes_RN$ and elements $r,r'\in R$ that
	\begin{enumerate}
		\item $r\cdot(m\otimes n+m'\otimes n')=r\cdot(m\otimes n)+r\cdot( m'\otimes n')$,
		\item $(r+r')\cdot(m\otimes n)=r\cdot(m\otimes n)+r'\cdot(m'\otimes n')$,
		\item $(rr')\cdot(m\otimes n)=r\cdot(r'\cdot(m\otimes n))$, and
		\item $1\cdot (m\otimes n)=m\otimes n$.
	\end{enumerate}
	Axiom $(1)$ holds by definition. To see $(2)$, note that by the fact that $R$ acts on $M$ on the left that
	\[(r+r')\cdot(m\otimes n)=((r+r')\cdot m)\otimes n=(r\cdot m+r'\cdot m)\otimes n=r\cdot m\otimes n+r'\cdot m\otimes n.\]
	That $(3)$ and $(4)$ hold follows similarly by the fact that $(rr')\cdot m=r\cdot(r'\cdot m)$ and $1\cdot m=m$.

	Conversely, if $N$ is an $A$-graded $R$-bimodule, then showing $M\otimes_RN$ is canonically a right $A$-graded $R$-module via the rule
	\[(m\otimes n)\cdot r=m\otimes(n\cdot r)\]
	is entirely analagous.

	Finally, if both $M$ and $N$ are $R$-bimodules, then by what we have shown, $M\otimes_RN$ is both a left and right $R$-module. To see these coincide to give $M\otimes_RN$ an $R$-bimodule structure, note that given $m\in M$, $n\in N$, and $r,r'\in R$ that
	\[(r\cdot(m\otimes n))\cdot r'=((r\cdot m)\otimes n)\cdot r'=(r\cdot m)\otimes(n\cdot r')=r\cdot(m\otimes(n\cdot r'))=r\cdot((m\otimes n)\cdot r').\qedhere\]
\end{proof}

\begin{lemma}\label{tensor_lift_of_A_graded_is_A_graded}
	Let $R$ be an $A$-graded ring, $B$ an $A$-graded abelian group, $M$ a right $A$-graded $R$-module, and $N$ a left $A$-graded $R$-module. Further suppose we are given a map $\varphi_{a,b}:M_a\times N_b\to B_{a+b}$ for all $a,b\in A$ which commutes with addition in each argument, and such that for all $m\in M_a$, $n\in N_b$, and $r\in R_c$ that
	\[\varphi_{a+b,c}(m\cdot r,n)=\varphi_{a,b+c}(m,r\cdot n).\]
	Then there is a unique $A$-graded $R$-balanced map $\varphi:M\times N\to B$ which restricts to $\varphi_{a,b}$ for all $a,b\in A$, and furthermore, the induced homorphism $\wt\varphi:M\otimes_RN\to B$ is an $A$-graded homomorphism of abelian groups.
\end{lemma}
\begin{proof}
	Checking this is straightforward, we leave it as an exercise for the reader.
\end{proof}

\subsection{\texorpdfstring{$A$}{A}-graded submodules and quotient modules}

In what follows, fix an $A$-graded ring $R$. We will simply say ``$A$-graded $R$-module'' when we are freely considering either left or right $A$-graded $R$-modules. 

Recall that given a ring $R$, an $R$-module $P$ is \emph{projective} if, for all diagrams of $R$-module homomorphisms of the form
% https://q.uiver.app/#q=WzAsMyxbMCwxLCJQIl0sWzEsMSwiTiJdLFsxLDAsIk0iXSxbMCwxLCJmIiwyXSxbMiwxLCJnIiwwLHsic3R5bGUiOnsiaGVhZCI6eyJuYW1lIjoiZXBpIn19fV1d
\[\begin{tikzcd}
	& M \\
	P & N
	\arrow["f"', from=2-1, to=2-2]
	\arrow["g", two heads, from=1-2, to=2-2]
\end{tikzcd}\]
with $g$ an epimorphism, there exists a lift $h:P\to M$ satisfying $g\circ h=f$
% https://q.uiver.app/#q=WzAsMyxbMCwxLCJQIl0sWzEsMSwiTiJdLFsxLDAsIk0iXSxbMCwxLCJmIiwyXSxbMiwxLCJnIiwwLHsic3R5bGUiOnsiaGVhZCI6eyJuYW1lIjoiZXBpIn19fV0sWzAsMiwiaCIsMCx7InN0eWxlIjp7ImJvZHkiOnsibmFtZSI6ImRhc2hlZCJ9fX1dXQ==
\[\begin{tikzcd}
	& M \\
	P & N
	\arrow["f"', from=2-1, to=2-2]
	\arrow["g", two heads, from=1-2, to=2-2]
	\arrow["h", dashed, from=2-1, to=1-2]
\end{tikzcd}\]
(Note $h$ is not required to be unique.)

\begin{definition}\label{graded_projective_module}
	Let $R$ be an $A$-graded ring, and let $P$ be an $A$-graded $R$-module. Then $P$ is a \emph{graded projective} module if, for all diagrams of $A$-graded $R$-module homomorphisms of the form 
	% https://q.uiver.app/#q=WzAsMyxbMCwxLCJQIl0sWzEsMSwiTiJdLFsxLDAsIk0iXSxbMCwxLCJmIiwyXSxbMiwxLCJnIiwwLHsic3R5bGUiOnsiaGVhZCI6eyJuYW1lIjoiZXBpIn19fV1d
	\[\begin{tikzcd}
		& M \\
		P & N
		\arrow["f"', from=2-1, to=2-2]
		\arrow["g", two heads, from=1-2, to=2-2]
	\end{tikzcd}\]
	with $g$ an epimorphism, there exists an $A$-graded homomorphism $h:P\to M$ satisfying $g\circ h=f$.
	% https://q.uiver.app/#q=WzAsMyxbMCwxLCJQIl0sWzEsMSwiTiJdLFsxLDAsIk0iXSxbMCwxLCJmIiwyXSxbMiwxLCJnIiwwLHsic3R5bGUiOnsiaGVhZCI6eyJuYW1lIjoiZXBpIn19fV0sWzAsMiwiaCIsMCx7InN0eWxlIjp7ImJvZHkiOnsibmFtZSI6ImRhc2hlZCJ9fX1dXQ==
	\[\begin{tikzcd}
		& M \\
		P & N
		\arrow["f"', from=2-1, to=2-2]
		\arrow["g", two heads, from=1-2, to=2-2]
		\arrow["h", dashed, from=2-1, to=1-2]
	\end{tikzcd}\]
	(Note $h$ is not required to be unique.)
\end{definition}

\begin{definition}
	Let $M$ be an $A$-graded $R$-module. Then an \emph{$A$-graded $R$-submodule} is an $A$-graded $R$-module $N$ which is a subset of $M$ and for which the inclusion $N\into M$ is an $A$-graded homomorphism of $R$-modules. Equivalently, it is a submodule $N$ for which the canonical map
	\[\bigoplus_{a\in A} N\cap M_a\to N\]
	is an isomorphism.
\end{definition}

\begin{lemma}\label{submodule_lemma}
	Let $M$ be an $A$-graded $R$-module. Then an $R$-submodule $N\leq M$ is an $A$-graded submodule if and only if it is generated as an $R$-module by homogeneous elements of $M$.
\end{lemma}
\begin{proof}
	If $N\leq M$ is a $A$-graded submodule, it is generated by the set of all its homogeneous elements, which are also homogeneous elements in $M$, by definition.

	Conversely, suppose $N\leq M$ is a submodule which is generated by homogeneous elements of $M$. Then define $N_a:=N\cap M_a$, and consider the canonical map
	\[\Phi:\bigoplus_{a\in A}N_a\to N.\]
	First of all, it is surjective, as each generator of $N$ belongs to some $N_a$, by definition. To see it is injective, consider the following commutative diagram:
	% https://q.uiver.app/#q=WzAsNCxbMCwwLCJcXGJpZ29wbHVzX3thXFxpbiBBfU5fYSJdLFsxLDAsIlxcYmlnb3BsdXNfe2FcXGluIEF9TV9hIl0sWzEsMSwiTSJdLFswLDEsIk4iXSxbMCwxLCIiLDAseyJzdHlsZSI6eyJ0YWlsIjp7Im5hbWUiOiJob29rIiwic2lkZSI6InRvcCJ9fX1dLFsxLDIsIlxcY29uZyJdLFswLDMsIlxcUGhpIiwyXSxbMywyLCIiLDIseyJzdHlsZSI6eyJ0YWlsIjp7Im5hbWUiOiJob29rIiwic2lkZSI6InRvcCJ9fX1dXQ==
	\[\begin{tikzcd}
		{\bigoplus_{a\in A}N_a} & {\bigoplus_{a\in A}M_a} \\
		N & M
		\arrow[hook, from=1-1, to=1-2]
		\arrow["\cong", from=1-2, to=2-2]
		\arrow["\Phi"', from=1-1, to=2-1]
		\arrow[hook, from=2-1, to=2-2]
	\end{tikzcd}\]
	Since $\Phi$ composes with an injection to get an injection, clearly $\Phi$ must be injective itself. We have the desired result.
\end{proof}

\begin{proposition}\label{image_and_kernel_of_A_graded_map_is_A_graded}
	Given two left (resp.\ right) $A$-graded $R$-modules $M$ and $N$ and an $A$-graded $R$-module homomorphism $\varphi:M\to N$ (of possibly nonzero degree), the kernel and images of $\varphi$ are $A$-graded submodules of $M$ and $N$, respectively.
\end{proposition}
\begin{proof}
	First recall that a degree $d$ $A$-graded homomorphism $M\to N$ is simply an $A$-graded homomorphism $M_*\to N_{*+d}$, so it suffices to consider the case $\varphi$ is of degree $0$. Next, note that since the forgetful functor from $R$-modules to abelian groups preserves kernels and images, it suffices to consider the case that $\varphi$ is a homomorphism of $A$-graded abelian groups. Finally, by \autoref{submodule_lemma}, it suffices to show that $\ker\varphi$ and $\imm\varphi$ are generated by homogeneous elements of $M$ and $N$, respectively.

	Note that by the universal property of the coproduct in $\Ab$, the data of an $A$-graded homomorphism of abelian groups $\varphi:M\to N$ is precisely the data of an $A$-indexed collection of abelian group homomorphisms $\varphi_a:M_a\to N_a$, in which case the following diagram commutes:
	% https://q.uiver.app/#q=WzAsNCxbMCwwLCJcXGJpZ29wbHVzX2FNX2EiXSxbMSwwLCJcXGJpZ29wbHVzX2FOX2EiXSxbMCwxLCJNIl0sWzEsMSwiTiJdLFswLDEsIlxcYmlnb3BsdXNfYVxcdmFycGhpX2EiXSxbMCwyLCJcXGNvbmciLDJdLFsyLDMsIlxcdmFycGhpIl0sWzEsMywiXFxjb25nIl1d
	\[\begin{tikzcd}
		{\bigoplus_aM_a} & {\bigoplus_aN_a} \\
		M & N
		\arrow["{\bigoplus_a\varphi_a}", from=1-1, to=1-2]
		\arrow["\cong"', from=1-1, to=2-1]
		\arrow["\varphi", from=2-1, to=2-2]
		\arrow["\cong", from=1-2, to=2-2]
	\end{tikzcd}\]
	Finally, the desired result follows by the purely formal fact that taking images and kernels commutes with arbitrary direct sums.
\end{proof}

\begin{proposition}\label{preimage_of_A_graded_is_A_graded}
	Given two left (resp.\ right) $A$-graded $R$-modules $M$ and $N$, an $A$-graded submodule $K\leq N$, and an $A$-graded $R$-module homomorphism $\varphi:M\to N$ (of possibly nonzero degree), the submodule $\varphi^{-1}(K)$ of $M$ is $A$-graded.
\end{proposition}
\begin{proof}
	Recall that a degree $d$ $A$-graded homomorphism $M\to N$ is simply an $A$-graded homomorphism $M_*\to N_{*+d}$, so it suffices to consider the case $\varphi$ is of degree $0$. Now, let $x\in L:=\varphi^{-1}(K)$. As an element of $M$, we may uniquely write $x=\sum_{a\in A}x_a$ where each $x_a\in M_a$. Similarly, if we set $y:=\varphi(x)$, then we may uniquely write $y=\sum_{a\in A}y_a$ where each $y_a\in N_a$. Then since $K$ is an $A$-graded submodule of $N$ and $y\in K$, by definition, we have that $y_a\in K$ for each $a$. Finally, note that
	\[\sum_{a\in A}y_a=y=\varphi(x)=\sum_{a\in A}\varphi(x_a),\]
	so that $\varphi(x_a)=y_a\in K$ for all $a\in A$, so that $x_a\in L$ for all $a\in A$. Thus we have shown that each element in $L$ can be written as a sum of homogeneous elements in $M$, as desired.
\end{proof}

\begin{proposition}\label{quotient_of_A_graded_is_A_graded}
	Given an $A$-graded $R$-module $M$ and an $A$-graded subgroup $N\leq M$, the quotient $M/N$ is canonically $A$-graded by defining $(M/N)_a$ to be the subgroup generated by cosets represented by homogeneous elements of degree $a$ in $M$. Furthermore, the canonical maps $M_a/N_a\to (M/N)_a$ taking a coset $m+N_a$ to $m+N$ are isomorphisms.
\end{proposition}
\begin{proof}
	Consider the canonical map
	\[\Phi:\bigoplus_a (M/N)_a\to M/N.\]
	First of all, surjectivity of $\Phi$ follows by commutativity of the following diagram:
	% https://q.uiver.app/#q=WzAsNCxbMCwwLCJcXGJpZ29wbHVzX2EgTV9hIl0sWzEsMCwiTSJdLFsxLDEsIk0vTiJdLFswLDEsIlxcYmlnb3BsdXNfYShNL04pX2EiXSxbMCwxLCJcXGNvbmciXSxbMSwyLCIiLDAseyJzdHlsZSI6eyJoZWFkIjp7Im5hbWUiOiJlcGkifX19XSxbMCwzLCIiLDIseyJzdHlsZSI6eyJoZWFkIjp7Im5hbWUiOiJlcGkifX19XSxbMywyLCJcXFBoaSJdXQ==
	\[\begin{tikzcd}
		{\bigoplus_a M_a} & M \\
		{\bigoplus_a(M/N)_a} & {M/N}
		\arrow["\cong", from=1-1, to=1-2]
		\arrow[two heads, from=1-2, to=2-2]
		\arrow[two heads, from=1-1, to=2-1]
		\arrow["\Phi", from=2-1, to=2-2]
	\end{tikzcd}\]
	where the vertical left map sends a generator $m\in M_a$ to the coset $m+N$ in $(M/N)_a\sseq M/N$. To see $\Phi$ is injective, suppose we are given some element $(m_a+N)_{a\in A}$ in $\bigoplus_a(M/C)_a$ such that $\sum_{a\in A}(m_a+N)=0$ in $M/N$. Thus $\sum_{a\in A}m_a\in N$, and since $N$ is $A$-graded this implies that each $m_a$ belongs to $N\cap M_a=N_a$, so that in particular $m_a+N$ is zero in $(M/N)_a\sseq M/N$, so that $(m_a+N)_{a\in A}=0$ in $\bigoplus_{a}(M/N)_a$, as desired.

	It remains to show that the canonical map
	\[\varphi_a:M_a/N_a\to (M/N)_a\]
	is an isomorphism. It is clearly surjective, as $(M/N)_a$ is generated by elements $m+N$ for $m\in M_a$, and these elements make up precisely the image of $\varphi_a$. Thus $\varphi_a$ hits every generator of $(M/N)_a$, so $\varphi_a$ is surjective. On the other hand, suppose we are given some $m\in M_a$ such that $\varphi(m+N_a)=m+N=0$. Thus $m\in N$, and $m\in M_a$, so that $m\in M_a\cap N=N_a$, meaning $m+N_a=0$ in $M_a/N_a$, as desired.
\end{proof}

\subsection{Pushouts of \texorpdfstring{$A$}{A}-graded anticommutative rings}\label{subsection:A-graded_anticommutative_rings_properties}

The goal of this section is to show that given an $A$-graded anticommutative ring $R$ (\autoref{A-graded_anticommutative_ring_defn}) that the category $R\text-\GrCAlg(A)$ of $A$-graded anticommutative $R$-algebras (\autoref{R-GrCAlg_defn}) has pushouts and binary coproducts, which are formed by taking the tensor product of the underlying $A$-graded modules and endowing it with an anticommutative product. The proofs here are entirely analagous to showing that the standard category of anticommutative $\bZ$-graded rings has pushouts, so rather than giving complete proofs in this section we simply outline what needs to be shown, and leave it to the reader to fill in the details.

\begin{proposition}\label{B-tensor_product_in_R-GrCAlg}
	Suppose we have an $A$-graded anticommutative ring $R$ (\autoref{A-graded_anticommutative_ring_defn}) and two morphisms $f:(B,\varphi_B)\to(C,\varphi_C)$ and $g:(B,\varphi_B)\to(D,\varphi_D)$ in $R\text-\GrCAlg(A)$ (\autoref{R-GrCAlg_defn}). Then $f$ and $g$ make $C$ and $D$ both $B$-bimodules, respectively,\footnote{Explicitly, it is a standard fact that given a ring homomorphism $\varphi:R\to S$ that $S$ canonically becomes an $R$-bimodule with left action $r\cdot s:=\varphi(r)s$ and right action $s\cdot r:=s\varphi(r)$, so that in particular if $\varphi$ is an $A$-graded homomorphism of $A$-graded rings, then $\varphi$ makes $S$ an $A$-graded $R$-bimodule.} so we may form their tensor product $C\otimes_BD$, which is itself an $A$-graded $B$-bimodule (\autoref{tensor_of_A_graded_is_A_graded}). Then $C\otimes_BD$ canonically inherits the structure of an $A$-graded $R$-commutative ring with unit $1_C\otimes 1_D$ via a product
	\[(C\otimes_BD)\times(C\otimes_BD)\to C\otimes_BD\]
	which sends a pair $(x\otimes y,x'\otimes y')$ of homogeneous pure tensors to the element
	\[\varphi_B(\theta_{|x|,|y'|})\cdot(xx'\otimes yy')=\varphi_C(\theta_{|x|,|y'|})xx'\otimes yy',\]
	(where here $\cdot$ denotes the left module action of $B$ on $C\otimes_BD$), and with structure map
	\begin{align*}
		\varphi:R&\to C\otimes_BD \\ 
		r&\mapsto \varphi_B(r)\cdot(1_C\otimes 1_D)=(\varphi_C(r)\otimes 1_D)=(1_C\otimes\varphi_D(r)).
	\end{align*}
\end{proposition}
\begin{proof}[Proof sketch]
	We simply lay out everything that needs to be shown, and we leave it to the reader to fill in the details. First to show that the indicated product is actually well-defined and distributive, by \autoref{tensor_lift_of_A_graded_is_A_graded} it suffices to show that for all homogeneous $c,c',c''\in C$, $d,d',d''\in D$, and $b\in B$ with $|c'|=|c''|$ and $|d'|=|d''|$, that
	\begin{align*}
		\varphi_B(\theta_{|d|,|c'+c''|})\cdot(c(c'+c'')\otimes dd')&=\varphi_B(\theta_{|d|,|c'|})\cdot(cc'\otimes dd')+\varphi_B(\theta_{|d|,|c''|})\cdot (cc''\otimes dd') \\
		\varphi_B(\theta_{|d|,|c'|})\cdot (cc'\otimes d(d'+d''))&=\varphi_B(\theta_{|d|,|c'|})\cdot(cc'\otimes dd')+\varphi_B(\theta_{|d|,|c'|})\cdot(cc'\otimes dd'') \\
		\varphi_B(\theta_{|d|,|c'\cdot b|})\cdot (c(c'\cdot b)\otimes dd')&=\varphi_B(\theta_{|d|,|c'|})\cdot(cc'\otimes d(b\cdot d')) \\
		\varphi_B(\theta_{|d'|,|c|})\cdot((c'+c'')c\otimes d'd)&=\varphi_B(\theta_{|d'|,|c|})\cdot(c'c\otimes d'd)+\varphi_B(\theta_{|d'|,|c|})\cdot(c''c\otimes d'd) \\
		\varphi_B(\theta_{|d'+d''|,|c|})\cdot(c'c\otimes (d'+d'')d)&=\varphi_B(\theta_{|d'|,|c|})\cdot(c'c\otimes d'd)+\varphi_B(\theta_{|d''|,|c|})\cdot(c'c\otimes d''d) \\
		\varphi_B(\theta_{|d'|,|c|})((c'\cdot b)c\otimes d'd)&=\varphi_B(\theta_{|c|,|b\cdot d'|})\cdot (c'c\otimes (b\cdot d')d),
	\end{align*}
	where each occurrence of $\cdot$ denotes the left or right module action of $B$. These tell us that for all $x\in C\otimes_BD$ that the maps $C\otimes_BD\to C\otimes_BD$ sending $y\mapsto xy$ and $y\mapsto yx$ are well-defined $A$-graded homomorphisms of abelian groups, so we have a distributive product $(x,y)\mapsto xy$. Then to show that this product makes $C\otimes_BD$ an $A$-graded ring, we need to show it is associative and unital. By \autoref{A_graded_ring}, it suffices to show that for all \emph{homogeneous} $x,y,z\in C\otimes_BD$ that $(xy)z=x(yz)$ and $x(1_C\otimes 1_D)=x=(1_C\otimes 1_D)x$. By distributivity, it further suffices to consider the case that $x$, $y$, and $z$ are homogeneous \emph{pure tensors} in $C\otimes_BD$, i.e., it suffices to show that for all homogeneous $c,c',c''\in C$ and $d,d',d''\in D$ that
	\[((c\otimes d)(c'\otimes d'))(c''\otimes d'')=(c\otimes d)((c'\otimes d')(c''\otimes d''))\]
	and
	\[(c\otimes d)(1_C\otimes 1_D)=(c\otimes d)=(1_C\otimes 1_D)(c\otimes d).\]
	Thus, proving these hold will show $C\otimes_BD$ has the structure of an $A$-graded ring, as desired. Now, we wish to show that the given map $\varphi:R\to C\otimes_BD$ is a ring homomorphism. Clearly it sends $1$ to $1_C\otimes 1_D$, and again by linearity, it suffices to show that given \emph{homogeneous} $r,s\in R$ that
	\[\varphi(r+s)=\varphi_B(r+s)(1_C\otimes 1_D)=\varphi_B(r)(1_C\otimes 1_D)+\varphi_B(s)(1_C\otimes 1_D)=\varphi(r)+\varphi(s)\]
	and
	\[\varphi(rs)=\varphi_B(rs)(1_C\otimes 1_D)=(\varphi_B(r)(1_C\otimes 1_D))(\varphi_B(s)(1_C\otimes 1_D))=\varphi(r)\varphi(s).\]
	Finally, we need to show that $C\otimes_BD$ satisfies the graded commutativity condition, for which again by linearity it suffices to show that given homogeneous $c,c'\in C$ and $d,d'\in D$ that
	\[(c\otimes d)(c'\otimes d')=\varphi(\theta_{|c\otimes d|,|c'\otimes d'|})(c'\otimes d')(c\otimes d)=\varphi(\theta_{|c|+|d|,|c'|+|d'|})(c'\otimes d')(c\otimes d).\]
	Showing all of these is relatively straightforward.
\end{proof}

\begin{proposition}\label{R-GrCAlg_has_pushouts_and_binary_coproducts}
	Given an $A$-graded anticommutative ring $(R,\theta)$, the category $R\text-\GrCAlg(A)$ has pushouts, where given $f:(B,\varphi_B)\to (C,\varphi_C)$ and $g:(B,\varphi_B)\to(D,\varphi_D)$, their pushout is the object $(C\otimes_BD,\varphi)$ constructed in \autoref{B-tensor_product_in_R-GrCAlg}, along with the canonical maps $(C,\varphi_C)\to(C\otimes_BD,\varphi)$ sending $c\mapsto c\otimes1_D$ and $(D,\varphi_D)\to(C\otimes_BD,\varphi)$ sending $d\mapsto  1_C\otimes d$. In particular, since $(R,\id_R)$ is initial, $R\text-\GrCAlg(A)$ has binary coproducts.
\end{proposition}
\begin{proof}[Proof sketch]
	First, we need to show that the given maps $i_C:(C,\varphi_C)\to(C\otimes_BD,\varphi)$ and $i_D:(D,\varphi_D)\to(C\otimes_BD,\varphi)$ are actually morphisms in $R\text-\GrCAlg(A)$, i.e., that they are ring homomorphisms and that the following diagram commutes:
	% https://q.uiver.app/#q=WzAsNCxbMSwxLCJDXFxvdGltZXNfQkQiXSxbMCwxLCJDIl0sWzIsMSwiRCJdLFsxLDAsIlIiXSxbMSwwLCJpX0MiLDJdLFsyLDAsImlfRCJdLFszLDEsIlxcdmFycGhpX0MiLDJdLFszLDIsIlxcdmFycGhpX0QiXSxbMywwLCJcXHZhcnBoaSIsMV1d
	\[\begin{tikzcd}
		& R \\
		C & {C\otimes_BD} & D
		\arrow["{i_C}"', from=2-1, to=2-2]
		\arrow["{i_D}", from=2-3, to=2-2]
		\arrow["{\varphi_C}"', from=1-2, to=2-1]
		\arrow["{\varphi_D}", from=1-2, to=2-3]
		\arrow["\varphi"{description}, from=1-2, to=2-2]
	\end{tikzcd}\]
	Showing this is entirely straightforward. Furthermore, $i_C$ and $i_D$ clearly make the following diagram commute:
	% https://q.uiver.app/#q=WzAsNCxbMCwwLCJCIl0sWzEsMCwiRCJdLFsxLDEsIkNcXG90aW1lc19CRCJdLFswLDEsIkMiXSxbMCwxLCJnIl0sWzEsMiwiaV9EIl0sWzAsMywiZiIsMl0sWzMsMiwiaV9DIiwyXV0=
	\[\begin{tikzcd}
		B & D \\
		C & {C\otimes_BD}
		\arrow["g", from=1-1, to=1-2]
		\arrow["{i_D}", from=1-2, to=2-2]
		\arrow["f"', from=1-1, to=2-1]
		\arrow["{i_C}"', from=2-1, to=2-2]
	\end{tikzcd}\]
	It remains to show that $i_C$ and $i_D$ are the universal such arrows. Suppose we have some object $(E,\varphi_E)$ in $R\text-\GrCAlg(A)$ and a commuting diagram
	% https://q.uiver.app/#q=WzAsNCxbMCwwLCJCIl0sWzAsMSwiQyJdLFsxLDAsIkQiXSxbMSwxLCJFIl0sWzAsMSwiZiIsMl0sWzAsMiwiZyJdLFsxLDMsImgiXSxbMiwzLCJrIl1d
	\[\begin{tikzcd}
		B & D \\
		C & E
		\arrow["f"', from=1-1, to=2-1]
		\arrow["g", from=1-1, to=1-2]
		\arrow["h", from=2-1, to=2-2]
		\arrow["k", from=1-2, to=2-2]
	\end{tikzcd}\]
	of morphisms in $R\text-\GrCAlg(A)$. Then we'd like to show there exists a unique morphism $\ell:C\otimes_BD\to E$ in $R\text-\GrCAlg(A)$ which makes the following diagram commute:
	% https://q.uiver.app/#q=WzAsNSxbMCwwLCJCIl0sWzAsMSwiQyJdLFsxLDAsIkQiXSxbMiwyLCJFIl0sWzEsMSwiQ1xcb3RpbWVzX0JEIl0sWzAsMSwiZiIsMl0sWzAsMiwiZyJdLFsxLDMsImgiLDIseyJjdXJ2ZSI6M31dLFsyLDMsImsiLDAseyJjdXJ2ZSI6LTN9XSxbMSw0LCJpX0MiXSxbMiw0LCJpX0QiLDJdLFs0LDMsIlxcZWxsIiwxLHsic3R5bGUiOnsiYm9keSI6eyJuYW1lIjoiZGFzaGVkIn19fV1d
	\[\begin{tikzcd}
		B & D \\
		C & {C\otimes_BD} \\
		&& E
		\arrow["f"', from=1-1, to=2-1]
		\arrow["g", from=1-1, to=1-2]
		\arrow["h"', curve={height=18pt}, from=2-1, to=3-3]
		\arrow["k", curve={height=-18pt}, from=1-2, to=3-3]
		\arrow["{i_C}", from=2-1, to=2-2]
		\arrow["{i_D}"', from=1-2, to=2-2]
		\arrow["\ell"{description}, dashed, from=2-2, to=3-3]
	\end{tikzcd}\]
	First we show uniqueness. Supposing such an arrow $\ell$ existed, given elements $c\in C$ and $d\in D$, we must have
	\[\ell(c\otimes d)=\ell((c\otimes 1_D)(1_C\otimes d))=\ell(c\otimes 1_D)\ell(1_C\otimes d)=\ell(i_C(c))\ell(i_D(d))=h(c)k(d).\]
	Since pure tensors generate $C\otimes_BD$, we have uniquely determined $\ell$, and clearly it makes the above diagram commute. Now, it remains to show that as defined $\ell$ is a morphism in $R\text-\GrCAlg(A)$, i.e., that it is an $A$-graded ring homomorphism and that the following diagram commutes:
	% https://q.uiver.app/#q=WzAsMyxbMSwwLCJSIl0sWzAsMSwiQ1xcb3RpbWVzX0JEIl0sWzIsMSwiRSJdLFswLDEsIlxcdmFycGhpIiwyXSxbMSwyLCJcXGVsbCJdLFswLDIsIlxcdmFycGhpX0UiXV0=
	\[\begin{tikzcd}
		& R \\
		{C\otimes_BD} && E
		\arrow["\varphi"', from=1-2, to=2-1]
		\arrow["\ell", from=2-1, to=2-3]
		\arrow["{\varphi_E}", from=1-2, to=2-3]
	\end{tikzcd}\]
	This is all entirely straightforward to show.
\end{proof}

\end{document}
