\documentclass[../main.tex]{subfiles}
\begin{document}

In this section, we aim to provide a nicer characterization of the $E_1$ page. Here we will often work in a symmetric strict monoidal category by the coherence theorem for symmetric monoidal categories, and we will do so without comment. Recall that by how the Adams spectral sequence for the computation of $[X,Y]_*$ is constructed, that the $E_1$ page is the $\bZ\times A$-graded abelian group given by
\[E_1^{s,a}(X,Y)=[X,W_s]_a=[S^a\otimes X,{\ol E}^s\otimes Y],\]
where $\ol E$ is the fiber (\autoref{fiber}) of the unit map $e:S\to E$.  In this section, we will show that under suitable conditions, these groups may alternatively be computed as hom-groups of morphisms of comodules over the dual $E$-Steenrod algebra (to be defined).

\subsection{Flat monoid objects in \texorpdfstring{$\cSH$}{SH}}

\begin{definition}\label{flat}
    Call a monoid object $(E,\mu,e)$ in $\cSH$ (\autoref{monoid_object}) \emph{flat} if the canonical right $\pi_*(E)$-module structure on $E_*(E)$ (\autoref{module_main}) is that of a flat module.
\end{definition}

\begin{proposition}[\autoref{Kunneth_map}]\label{Kunneth_map_construction_main}
    Let $(E,\mu,e)$ be a monoid object (\autoref{monoid_object}) and $Z$ and $W$ be objects in $\cSH$. Then there is a homomorphism of abelian groups
    \[\Phi_{Z,W}:\pi_*(Z\otimes E)\otimes_{\pi_*(E)}\pi_*(E\otimes W)\to\pi_*(Z\otimes E\otimes W)\]
    which given homogeneous elements $x:S^a\to Z\otimes E$ in $\pi_*(Z\otimes E)$  and $y:S^b\to E\otimes W$ in $\pi_*(E\otimes W)$, sends the homogeneous pure tensor $x\otimes y$ in $\pi_*(Z\otimes E)\otimes_{\pi_*(E)}\pi_*(E\otimes W)$ to the composition
    \[S^{a+b}\xr{\phi_{a,b}}S^a\otimes S^b\xr{x\otimes y}Z\otimes E\otimes E\otimes W\xr{Z\otimes\mu\otimes W}Z\otimes E\otimes W\]
    (where here we are considering the canonical $A$-graded right $\pi_*(E)$-module structure on $\pi_*(Z\otimes E)=Z_*(E)$ and the canonical left $A$-graded $\pi_*(E)$-module structure on $\pi_*(E\otimes W)=E_*(W)$ given in \autoref{module_main}, so that $\pi_*(Z\otimes E)\otimes_{\pi_*(E)}\pi_*(E\otimes W)$ is a well-defined $A$-graded abelian group by \autoref{tensor_of_A_graded_is_A_graded}). Furthermore, this homomorphism is natural in both $Z$ and $W$.
\end{proposition}

The key consequence of the assumption that $E$ is flat in the sense of \autoref{flat} is that $\Phi_{E,W}$ is an isomorphism for cellular objects $W$ in $\cSH$:

\begin{proposition}[\autoref{Kunneth_iso_for_cellular_objects}]\label{Kunneth_map_iso_main}
	Let $(E,\mu,e)$ be a monoid object in $\cSH$. Then if either:\begin{enumerate}
		\item $\pi_*(Z\otimes E)=Z_*(E)$ is a flat right $\pi_*(E)$-module (via \autoref{module}) and $W$ is cellular (\autoref{cellular}), or
		\item $\pi_*(E\otimes W)=E_*(W)$ is a flat left $\pi_*(E)$-module (via \autoref{module}) and $Z$ is cellular (\autoref{cellular}),
	\end{enumerate} 
	then the natural homomorphism
	\[\Phi_{Z,W}:\pi_*(Z\otimes E)\otimes_{\pi_*(E)}\pi_*(E\otimes W)\to \pi_*(Z\otimes E\otimes W)\]
	given in \autoref{Kunneth_map_construction_main} is an isomorphism of abelian groups.
\end{proposition}

\subsection{The dual \texorpdfstring{$E$}{E}-Steenrod algebra}

In \Cref{monoid_objects_subsection}, we showed that given a monoid object $(E,\mu,e)$ in $\cSH$, that $E_*(E)$ is canonically an $A$-graded bimodule over the ring $\pi_*(E)$. In this subsection, we will outline some additional structure carried by $E_*(E)$. In particular, we will show that if $(E,\mu,e)$ is a flat (\autoref{flat}) commutative monoid object, then the pair $(E_*(E),\pi_*(E))$ is canonically an $A$-graded commutative Hopf algebroid over the stable homotopy ring $\pi_*(S)$ (\autoref{hopf_algebroid_defn}), called the \emph{dual $E$-Steenrod algebra}. To start with, we outline some structure maps relating $E_*(E)$ and $\pi_*(E)$.

\begin{proposition}\label{structure_maps_are_monoid_homos_main}
    Let $(E,\mu,e)$ be a commutative monoid object in $\cSH$. Then the maps\begin{enumerate}
        \item $E\xr\cong E\otimes S\xr{E\otimes e}E\otimes E$,
        \item $E\xr\cong S\otimes E\xr{e\otimes E}E\otimes E$,
        \item $E\otimes E\xr\cong E\otimes S\otimes E\xr{E\otimes e\otimes E}E\otimes E\otimes E$,
        \item $E\otimes E\xr\mu E$, and
        \item $E\otimes E\xr{\tau_{E,E}}E\otimes E$
    \end{enumerate}
    are homomorphisms of monoid objects in $\cSH$ (where here $E\otimes E$ and $E\otimes E\otimes E$ are considered as monoid objects in $\cSH$ by \autoref{product_of_monoids_is_monoid} and \autoref{product_of_3+_monoids_no_ambiguity}, respectively), so that by \autoref{pi_*:CMon_SH-->pi_*(S)-GrCAlg_main}, under $\pi_*$ they induce morphisms in $\GrCAlg{\pi_*(S)}$:
    \begin{enumerate}
        \item $\eta_L:\pi_*(E)\to E_*(E)$,
        \item $\eta_R:\pi_*(E)\to E_*(E)$,
        \item $h:E_*(E)\to E_*(E\otimes E)$,
        \item $\vare:E_*(E)\to \pi_*(E)$, and
        \item $c:E_*(E)\to E_*(E)$.
    \end{enumerate}
\end{proposition}

\begin{lemma}[\autoref{eta_L_left_module/eta_R_right_module_coincide_appendix}]\label{eta_L_left_module/eta_R_right_module_coincide}
    Let $(E,\mu,e)$ be a commutative monoid object in $\cSH$. Then the left (resp.\ right) $\pi_*(E)$-module structure induced on $E_*(E)$ by the ring homomorphism $\eta_L$ (resp.\ $\eta_R$)\footnote{Recall that given a homomorphism of rings $\varphi:R\to S$, that $S$ canonically inherits the structure of a left (resp.\ right) $R$-module by defining $r\cdot s:=\varphi(r)s$ (resp.\ $s\cdot r:=s\varphi(r)$).} coincides with the canonical left (resp.\ right) $\pi_*(E)$-module structure on $E_*(E)$ given in \autoref{module_main}.
\end{lemma}

\begin{corollary}[\autoref{E*E_ox_E*E_is_A-graded_pi*S-commutative_ring_appendix}]\label{E*E_ox_E*E_is_A-graded_pi*S-commutative_ring}
    Given a %flat (\autoref{flat}) and cellular (\autoref{cellular}) 
	commutative monoid object $(E,\mu,e)$ in $\cSH$, the domain of the %isomorphism 
	homomorphism
    \[\Phi_{E,E}:E_*(E)\otimes_{\pi_*(E)}E_*(E)\to E_*(E\otimes E)\]
    constructed in \autoref{Kunneth_map_iso_main} is canonically an $A$-graded $\pi_*(S)$-ring, and sits in the following pushout diagram in $\GrCAlg{\pi_*(S)}$:
    % https://q.uiver.app/#q=WzAsNCxbMCwwLCJcXHBpXyooRSkiXSxbMSwwLCJFXyooRSkiXSxbMCwxLCJFXyooRSkiXSxbMSwxLCJFXyooRSlcXG90aW1lc197XFxwaV8qKEUpfUVfKihFKSJdLFswLDEsIlxcZXRhX0wiXSxbMCwyLCJcXGV0YV9SIiwyXSxbMiwzLCJ4XFxtYXBzdG8geFxcb3RpbWVzIDEiLDJdLFsxLDMsInhcXG1hcHN0bzFcXG90aW1lcyB4Il1d
    \[\begin{tikzcd}
        {\pi_*(E)} & {E_*(E)} \\
        {E_*(E)} & {E_*(E)\otimes_{\pi_*(E)}E_*(E)}
        \arrow["{\eta_L}", from=1-1, to=1-2]
        \arrow["{\eta_R}"', from=1-1, to=2-1]
        \arrow["{x\mapsto x\otimes 1}"', from=2-1, to=2-2]
        \arrow["{x\mapsto1\otimes x}", from=1-2, to=2-2]
    \end{tikzcd}\]
\end{corollary}

\begin{lemma}[\autoref{Phi_E_is_homo_of_A-graded_pi_*S-commutative_rings_appendix}]\label{Phi_E_is_homo_of_A-graded_pi_*S-commutative_rings_main}
    Let $(E,\mu,e)$ be a commutative monoid object in $\cSH$. Then the homomorphism
    \[\Phi_{E,E}:E_*(E)\otimes_{\pi_*(E)}E_*(E)\to E_*(E\otimes E)\]
    constructed in \autoref{Kunneth_map_construction_main} is a homomorphism of $A$-graded $\pi_*(S)$-commutative rings, i.e.\ a morphism in $\GrCAlg{\pi_*(S)}$, where here $E_*(E)\otimes_{\pi_*(E)}E_*(E)$ is considered as an object in $\GrCAlg{\pi_*(S)}$ by \autoref{E*E_ox_E*E_is_A-graded_pi*S-commutative_ring}, and $E_*(E\otimes E)=\pi_*(E\otimes (E\otimes E))$ is considered as an object in $\GrCAlg{\pi_*(S)}$ by \autoref{pi_*:CMon_SH-->pi_*(S)-GrCAlg_main}, since $E\otimes(E\otimes E)$ is a monoid object in $\cSH$ by \autoref{product_of_monoids_is_monoid}.
\end{lemma}

\begin{definition}\label{dual_E-Steenrod_algebra_defn}
    Let $(E,\mu,e)$ be a \emph{commutative} monoid object (\autoref{monoid_object}) which is flat (\autoref{flat}) and cellular (\autoref{cellular}). Then the \emph{dual $E$-Steenrod algebra} is the pair of $A$-graded abelian groups $(E_*(E),\pi_*(E))$ equipped with the following structure:\begin{enumerate}[label={\arabic*.}]
        \item The $A$-graded $\pi_*(S)$-commutative ring structure on $\pi_*(E)$
        induced from $E$ being a commutative monoid object in $\cSH$ (\autoref{pi_*:CMon_SH-->pi_*(S)-GrCAlg_main}).
        \item The $A$-graded $\pi_*(S)$-commutative ring structure on $E_*(E)$ induced from the fact that $E\otimes E$ is canonically a commutative monoid object in $\cSH$ (\autoref{product_of_monoids_is_monoid}), so that also $E_*(E)=\pi_*(E\otimes E)$ is an $A$-graded $\pi_*(S)$-commutative ring (\autoref{pi_*:CMon_SH-->pi_*(S)-GrCAlg_main}).
        \item The homomorphisms of $A$-graded $\pi_*(S)$-commutative rings
        \[\eta_L:\pi_*(E)\to E_*(E)\]
        and
        \[\eta_R:\pi_*(E)\to E_*(E)\]
        induced under $\pi_*$ by the monoid object homomorphisms
        \[E\xr\cong E\otimes S\xr{E\otimes e}E\otimes E\]
        and
        \[E\xr\cong S\otimes E\xr{e\otimes E}E\otimes E.\]
        \item The homomorphism of $A$-graded $\pi_*(S)$-commutative rings
        \[\Psi:E_*(E)\to E_*(E)\otimes_{\pi_*(E)}E_*(E)\]
        given by the composition
        \[E_*(E)\xr{h}E_*(E\otimes E)\xr{\Phi_{E,E}^{-1}}E_*(E)\otimes_{\pi_*(E)}E_*(E),\]
        where $h$ is a homomorphism of $A$-graded $\pi_*(S)$-commutative rings induced under $\pi_*$ by the monoid object homomorphism
        \[E\otimes E\xr\cong E\otimes S\otimes E\xr{E\otimes e\otimes E}E\otimes E\otimes E,\]
        and $\Phi_{E,E}$ is morphism constructed in \autoref{Kunneth_map_construction_main}, which is proven to be an isomorphism in \autoref{Kunneth_map_iso_main} and a morphism in $\GrCAlg{\pi_*(S)}$ in \autoref{Phi_E_is_homo_of_A-graded_pi_*S-commutative_rings_main}.
        \item The homomorphism of $A$-graded $\pi_*(S)$-commutative rings
        \[\vare:E_*(E)\to\pi_*(E)\]
        induced under $\pi_*$ by the monoid object homomorphism
        \[E\otimes E\xr\mu E.\]
        \item The homomorphism of $A$-graded $\pi_*(S)$-commutative rings
        \[c:E_*(E)\to E_*(E)\]
        induced under $\pi_*$ from the monoid object homomorphism
        \[E\otimes E\xr\tau E\otimes E.\]
    \end{enumerate}
\end{definition}

\begin{proposition}[\autoref{dual_E-Steenrod_algebra_is_a_Hopf_algebroid_appendix}]\label{dual_E-Steenrod_algebra_is_a_Hopf_algebroid_main}
    Let $(E,\mu,e)$ be a commutative monoid object in $\cSH$ which is flat (\autoref{flat}) and cellular (\autoref{cellular}). Then the dual $E$-Steenrod algebra $(E_*(E),\pi_*(E))$ with the structure maps $(\eta_L,\eta_R,\Psi,\vare,c)$ from \autoref{dual_E-Steenrod_algebra_defn} is an $A$-graded commutative Hopf algebroid over $\pi_*(S)$ (\autoref{hopf_algebroid_defn}), i.e., a co-groupoid object in the category $\GrCAlg{\pi_*(S)}$.
\end{proposition}

\subsection{Comodules over the dual \texorpdfstring{$E$}{E}-Steenrod algebra}

\begin{lemma}
    Let $(E,\mu,e)$ be a monoid object in $\cSH$. Then for all objects $X$ in $\cSH$, the $A$-graded homomorphism
    \[E_*(E)\otimes_{\pi_*(E)}E_*(X)\xr{\Phi_{E,X}}E_*(E\otimes X)\]
    is a homomorphism of left $A$-graded $\pi_*(E)$-module objects, where here we are considering the left $E$-module structure on $E_*(E)\otimes_{\pi_*(E)}E_*(X)$ induced by the canonical $\pi_*(E)$-bimodule structure on $E_*(E)$ (\autoref{module_main}).
\end{lemma}

\begin{proposition}
    Let $(E,\mu,e)$ be a flat (\autoref{flat}) and cellular (\autoref{cellular}) commutative monoid object in $\cSH$. Then $E_*(-)$ is a functor from $\cSH$ to the category $E_*(E)\text-\CoMod$ of left $A$-graded comodules (\autoref{left_comodule_defn}) over the dual $E$-Steenrod algebra, which is an $A$-graded commutative Hopf algebroid over $\pi_*(S)$, by \autoref{dual_E-Steenrod_algebra_is_a_Hopf_algebroid_main}.

    In particular, given an object $X$ in $\cSH$, we are viewing $E_*(X)$ with its canonical left $\pi_*(E)$-module structure (\autoref{module_main}), and the action map 
    \[\Psi_X:E_*(X)\to E_*(E)\otimes_{\pi_*(E)}E_*(X)\]
    is given by the composition
    \[\Psi_X:E_*(X)\xr{E_*(e\otimes X)}E_*(E\otimes X)\xr{\Phi_{E,X}^{-1}}E_*(E)\otimes_{\pi_*(E)}E_*(X).\]
\end{proposition}
\begin{proof}
    It is straightforward to check that the action map is a homomorphism of left $\pi_*(E)$-modules. Now, we need to show that it makes the two diagrams in \autoref{left_comodule_defn} commute.
\end{proof}

\end{document}
