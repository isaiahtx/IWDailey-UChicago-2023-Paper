\documentclass[../main.tex]{subfiles}
\begin{document}

In this section, we aim to provide a nicer characterization of the $E_1$ page. Here we will often work in a symmetric strict monoidal category by the coherence theorem for symmetric monoidal categories, and we will do so without comment. Recall that by how the Adams spectral sequence for the computation of $[X,Y]_*$ is constructed (\autoref{ASS}), that the $E_1$ page is given by
\[E_1^{s,a}(X,Y)=[X,W_s]_a=[S^a\otimes X,{\ol E}^s\otimes Y],\]
where $\ol E$ is the fiber (\autoref{fiber}) of the unit map $e:S\to E$.  In this section, we will show that under suitable conditions, these groups may alternatively be computed as hom-groups of morphisms of comodules over the dual $E$-Steenrod algebra.

\subsection{Flat monoid objects in \texorpdfstring{$\cSH$}{SH}}

\begin{definition}\label{flat}
    Call a monoid object $(E,\mu,e)$ in $\cSH$ (\autoref{monoid_object}) \emph{flat} if the canonical right $\pi_*(E)$-module structure on $E_*(E)$ (\autoref{module_main}) is that of a flat module.
\end{definition}

The key consequence of the assumption that $E$ is flat in the sense of \autoref{flat} is the following:

\begin{proposition}\label{Kunneth_map_iso_main}
    Let $(E,\mu,e)$ be a monoid object (\autoref{monoid_object}) and $X$ an object in $\cSH$. Then there is a homomorphism of left $A$-graded $\pi_*(E)$-modules
    \[\Phi_X:E_*(E)\otimes_{\pi_*(E)}E_*(X)\to E_*(E\otimes X)\]
    which given elements $x:S^a\to E\otimes E$ in $E_a(E)$  and $y:S^b\to E\otimes X$ in $E_b(X)$, sends the homogeneous pure tensor $x\otimes y$ in $E_*(E)\otimes_{\pi_*(E)}E_*(X)$ to the composition
    \[S^{a+b}\xr{\phi_{a,b}}S^a\otimes S^b\xr{x\otimes y}E\otimes E\otimes E\otimes X\xr{E\otimes\mu\otimes X}E\otimes E\otimes X\]
    (where here \autoref{module_main} makes $E_*(E)$ an $A$-graded $\pi_*(E)$-bimodule and $E_*(X)$ and $E_*(E\otimes X)$ $A$-graded left $\pi_*(E)$-modules, so that $E_*(E)\otimes_{\pi_*(E)}E_*(X)$ is a left $A$-graded $\pi_*(E)$-module by \autoref{tensor_of_A_graded_is_A_graded}).

    Furthermore, this homomorphism is natural in $X$, and if $E$ is flat (\autoref{flat}) and $X$ is a cellular object (\autoref{cellular}), then this homomorphism is an isomorphism of left $\pi_*(E)$-modules.
\end{proposition}
\begin{proof}
    The homomorphism is constructed and proven to be natural in \autoref{Kunneth_map}. In \autoref{Kunneth_iso_for_cellular_objects}, it is shown that this homomorphism is an isomorphism in the case that $(E,\mu,e)$ is a flat monoid object and $X$ is cellular.
\end{proof}

\subsection{The dual \texorpdfstring{$E$}{E}-Steenrod algebra}

In \Cref{monoid_objects_subsection}, we showed that given a monoid object $(E,\mu,e)$ in $\cSH$, that $E_*(E)$ is canonically an $A$-graded bimodule over the ring $\pi_*(E)$. In this subsection, we will explore some additional structure carried by $E_*(E)$. In particular, we will show that if $(E,\mu,e)$ is a flat (\autoref{flat}) commutative monoid object, then the pair $(E_*(E),\pi_*(E))$ is canonically an $A$-graded commutative Hopf algebroid over the stable homotopy ring $\pi_*(S)$ (\autoref{hopf_algebroid_defn}). 

\begin{definition}\label{dual_E-Steenrod_algebra_defn}
    Let $(E,\mu,e)$ be a \emph{commutative} monoid object (\autoref{monoid_object}) which is flat (\autoref{flat}) and cellular (\autoref{cellular}). Then the \emph{dual $E$-Steenrod algebra} is the pair of $A$-graded abelian groups $(E_*(E),\pi_*(E))$ equipped with the following structure:\begin{enumerate}[label={\arabic*.}]
        \item The $A$-graded $\pi_*(S)$-commutative ring structure on $\pi_*(E)$
        induced from $E$ being a commutative monoid object in $\cSH$ (\autoref{pi_*:CMon_SH-->pi_*(S)-GrCAlg_main}).
        \item The $A$-graded $\pi_*(S)$-commutativite ring structure on $E_*(E)$ induced from the fact that $E\otimes E$ is canonically a commutative monoid object in $\cSH$ (\autoref{product_of_monoids_is_monoid}), so that also $E_*(E)=\pi_*(E\otimes E)$ is an $A$-graded $\pi_*(S)$-commutative ring (\autoref{pi_*:CMon_SH-->pi_*(S)-GrCAlg_main}).
        \item (\autoref{(E,mu,e):eta_L,eta_R}) The homomorphisms of $A$-graded $\pi_*(S)$-commutative rings
        \[\eta_L:\pi_*(E)\to E_*(E)\]
        and
        \[\eta_R:\pi_*(E)\to E_*(E)\]
        induced under $\pi_*$ by the monoid object homomorphisms
        \[E\xr\cong E\otimes S\xr{E\otimes e}E\otimes E\]
        and
        \[E\xr\cong S\otimes E\xr{e\otimes E}E\otimes E.\]
        \item (\autoref{(E,mu,e):Psi_defn}) The homomorphism of $A$-graded $\pi_*(S)$-commutative rings
        \[\Psi:E_*(E)\to E_*(E)\otimes_{\pi_*(E)}E_*(E)\]
        induced under $\pi_*$ from
        \[E\otimes E\xr\cong E\otimes S\otimes E\xr{E\otimes e\otimes E}E\otimes E\otimes E\]
        via the isomorphism given in \autoref{Kunneth_map_iso_main}, where here $E_*(E)\otimes_{\pi_*(E)}E_*(E)$ is considered as the pushout in $\GrCAlg{\pi_*(S)}$ of $\eta_R$ and $\eta_L$ by \autoref{E*E_ox_E*E_is_A-graded_pi*S-commutative_ring}.
        \item (\autoref{(E,mu,e):vare}) The homomorphism of $A$-graded $\pi_*(S)$-commutativite rings
        \[\vare:E_*(E)\to\pi_*(E)\]
        induced under $\pi_*$ by the monoid object homomorphism
        \[E\otimes E\xr\mu E.\]
        \item (\autoref{(E,mu,e):c}) The homomorphism of $A$-graded $\pi_*(S)$-commutative rings
        \[c:E_*(E)\to E_*(E)\]
        induced under $\pi_*$ from the monoid object homomorphism
        \[E\otimes E\xr\tau E\otimes E.\]
    \end{enumerate}
\end{definition}

\begin{proposition}[\autoref{(E,mu,e):eta_L,eta_R_appendix}]\label{(E,mu,e):eta_L,eta_R}
    Let $(E,\mu,e)$ be a commutativite monoid object in $\cSH$. Then the maps
	\[E\xr\cong E\otimes S\xr{E\otimes e}E\otimes E\]
    and
    \[E\xr\cong S\otimes E\xr{e\otimes E}E\otimes E\]
	are homomorphisms of monoid objects (where here $E\otimes E$ is considered a monoid object by \autoref{product_of_monoids_is_monoid}), so that by \autoref{pi_*:CMon_SH-->pi_*(S)-GrCAlg} they induce morphisms in $\GrCAlg{\pi_*(S)}$:
	\[\eta_L:\pi_*(E)\to E_*(E)\]
	and
	\[\eta_R:\pi_*(E)\to E_*(E),\]
	respectively.
\end{proposition}

\begin{lemma}[\autoref{eta_L_left_module/eta_R_right_module_coincide_appendix}]\label{eta_L_left_module/eta_R_right_module_coincide}
    Let $(E,\mu,e)$ be a commutative monoid object in $\cSH$. Then the left (resp.\ right) $\pi_*(E)$-module structure induced on $E_*(E)$ by the ring homomorphism $\eta_L$ (resp.\ $\eta_R$) coincides with the canonical left (resp.\ right) $\pi_*(E)$-module structure on $E_*(E)$ given in \autoref{module_main}.
\end{lemma}

\begin{corollary}[\autoref{E*E_ox_E*E_is_A-graded_pi*S-commutative_ring_appendix}]\label{E*E_ox_E*E_is_A-graded_pi*S-commutative_ring}
    Given a %flat (\autoref{flat}) and cellular (\autoref{cellular}) 
	commutative monoid object $(E,\mu,e)$ in $\cSH$, the domain of the %isomorphism 
	homomorphism
    \[\Phi_E:E_*(E)\otimes_{\pi_*(E)}E_*(E)\to E_*(E\otimes E)\]
    constructed in \autoref{Kunneth_map_iso_main} is canonically an $A$-graded $\pi_*(S)$-ring, and sits in the following pushout diagram in $\GrCAlg{\pi_*(S)}$:
    % https://q.uiver.app/#q=WzAsNCxbMCwwLCJcXHBpXyooRSkiXSxbMSwwLCJFXyooRSkiXSxbMCwxLCJFXyooRSkiXSxbMSwxLCJFXyooRSlcXG90aW1lc197XFxwaV8qKEUpfUVfKihFKSJdLFswLDEsIlxcZXRhX0wiXSxbMCwyLCJcXGV0YV9SIiwyXSxbMiwzLCJ4XFxtYXBzdG8geFxcb3RpbWVzIDEiLDJdLFsxLDMsInhcXG1hcHN0bzFcXG90aW1lcyB4Il1d
    \[\begin{tikzcd}
        {\pi_*(E)} & {E_*(E)} \\
        {E_*(E)} & {E_*(E)\otimes_{\pi_*(E)}E_*(E)}
        \arrow["{\eta_L}", from=1-1, to=1-2]
        \arrow["{\eta_R}"', from=1-1, to=2-1]
        \arrow["{x\mapsto x\otimes 1}"', from=2-1, to=2-2]
        \arrow["{x\mapsto1\otimes x}", from=1-2, to=2-2]
    \end{tikzcd}\]
\end{corollary}

\begin{proposition}[\autoref{(E,mu,e):Psi_defn_appendix}]\label{(E,mu,e):Psi_defn}
    Let $(E,\mu,e)$ be a flat (\autoref{flat}) and cellular (\autoref{cellular}) commutative monoid object in $\cSH$. Then consider the map
    \[\Psi:E_*(E)\xr{\pi_*(E\otimes e\otimes E)}E_*(E\otimes E)\xr{\Phi_E^{-1}}E_*(E)\otimes_{\pi_*(E)}E_*(E),\]
    where $\Phi_E$ is the isomorphism given in \autoref{Kunneth_map_iso_main}. Then $\Psi$ is a homomorphism of $A$-graded $\pi_*(S)$-commutative rings, where here the object $E_*(E)\otimes_{\pi_*(E)}E_*(E)$ is considered an $A$-graded $\pi_*(S)$-commutative ring by \autoref{E*E_ox_E*E_is_A-graded_pi*S-commutative_ring}.
\end{proposition}

\begin{proposition}[\autoref{(E,mu,e):vare_appendix}]\label{(E,mu,e):vare}
    Let $(E,\mu,e)$ be a commutative monoid object in $\cSH$. Then the morphism 
    \[\mu:E\otimes E\to E\] 
    is a homomorphism of monoid objects (where $E\otimes E$ is considered a monoid object by \autoref{product_of_monoids_is_monoid}), so that by \autoref{pi_*:CMon_SH-->pi_*(S)-GrCAlg_main}, under $\pi_*$ it induces a homomorphism of $A$-graded $\pi_*(S)$-commutative rings 
    \[\vare:E_*(E)\to\pi_*(E).\]
\end{proposition}

\begin{proposition}[\autoref{(E,mu,e):c_appendix}]\label{(E,mu,e):c}
    Let $(E,\mu,e)$ be a commutative monoid object in $\cSH$. Then the morphism
    \[\tau_{E,E}:E\otimes E\to E\otimes E\]
    is a homomorphism of monoid objects (where $E\otimes E$ is considered a monoid object by \autoref{product_of_monoids_is_monoid}), so that by \autoref{pi_*:CMon_SH-->pi_*(S)-GrCAlg_main}, under $\pi_*$ it induces a homomorphism of $A$-graded $\pi_*(S)$-commutative rings
    \[c:E_*(E)\to E_*(E).\]
\end{proposition}

\begin{proposition}[\autoref{dual_E-Steenrod_algebra_is_a_Hopf_algebroid_appendix}]\label{dual_E-Steenrod_algebra_is_a_Hopf_algebroid_main}
    Let $(E,\mu,e)$ be a commutative monoid object in $\cSH$ which is flat (\autoref{flat}) and cellular (\autoref{cellular}). Then the dual $E$-Steenrod algebra $(E_*(E),\pi_*(E))$ with the structure maps $(\eta_L,\eta_R,\Psi,\vare,c)$ from \autoref{dual_E-Steenrod_algebra_defn} is an $A$-graded commutative Hopf algebroid over $\pi_*(S)$ (\autoref{hopf_algebroid_defn}), i.e., a co-groupoid object in the category $\GrCAlg{\pi_*(S)}$.
\end{proposition}

\end{document}
