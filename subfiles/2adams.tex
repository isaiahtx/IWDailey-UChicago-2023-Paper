\documentclass[../main.tex]{subfiles}
\begin{document}

\subsection{Setup}\label{setup}

In order to construct an abstract version of the Adams spectral sequence, we need to work in some axiomatic version of a stable homotopy category $\cSH$ which acts like the familiar classical stable homotopy category $\hoSp$ (\Cref{classical}) or the motivic stable homotopy category $\SH\scS$ over some base scheme $\scS$ (\Cref{motivic}):

\begin{definition}\label{stable_homotopy_cat_defn}
	A \emph{stable homotopy category} $\cSH$ is the following data:
	\begin{itemize}
		\item A closed tensor triangulated category $(\cSH,\otimes,S,\Sigma,\cD)$ with arbitrary (small) (co)products.
		\item A pointed abelian group $(A,\1)$ along with a homomorphism of pointed groups $h:(A,\1)\to(\Pic\cSH,\Sigma S)$.
		\item For each $a\in A$, a chosen representative $S^a$ in the isomorphism class $h(a)$ such that $S^0=S$.
		\item For each $a,b\in A$, an isomorphism $\phi_{a,b}:S^{a+b}\to S^a\otimes S^b$. This family of isomorphisms is required to be \emph{coherent}, in the following sense:
		\begin{itemize}
			\item For all $a\in A$, we must have that $\phi_{a,0}$ coincides with the right unitor $S^{a}\xrightarrow\cong S^a\otimes S$ and $\phi_{0,a}$ coincides the left unitor $S^a\xrightarrow\cong S\otimes S^a$.
			\item For all $a,b,c\in A$, the following diagram must commute:
			\[\begin{tikzcd}
				{S^{a+b}\otimes S^c} & {S^{a+b+c}} & {S^a\otimes S^{b+c}} \\
				{(S^a\otimes S^b)\otimes S^c} && {S^a\otimes(S^b\otimes S^c)}
				\arrow["{\phi_{a+b,c}}"', from=1-2, to=1-1]
				\arrow["{\phi_{a,b+c}}", from=1-2, to=1-3]
				\arrow["{S^a\otimes\phi_{b,c}}", from=1-3, to=2-3]
				\arrow["{\phi_{a,b}\otimes S^c}"', from=1-1, to=2-1]
				\arrow["\cong", from=2-1, to=2-3]
			\end{tikzcd}\]
		\end{itemize}
	\end{itemize}
\end{definition}

From now on we fix the data given in the above definition, and we establish some conventions. First of all, given objects $X_1,\ldots,X_n$ in $\cSH$, we write $X_1\otimes\cdots\otimes X_n$ to denote the object
\[X_1\otimes(X_2\otimes\cdots(X_{n-1}\otimes X_n)).\]
In particular, given an object $X$ and a natural number $n>0$, we write
\[X^n:=\overbrace{X\otimes\cdots\otimes X}^\text{$n$ times}\qquad\text{and}\qquad X^0:=S.\]
We denote the associator, symmetry, left unitor, and right unitor isomorphisms in $\cSH$ by
\[\begin{split}
	\alpha_{X,Y,Z}:(X\otimes Y)\otimes Z&\xrightarrow\cong X\otimes(Y\otimes Z) \\
	\lambda_X:S\otimes  X&\xrightarrow\cong X 
\end{split}\qquad\qquad\begin{split}
	\tau_{X,Y}:X\otimes Y &\xrightarrow\cong Y\otimes X\\
	\rho_X:X\otimes S&\xrightarrow\cong X.
\end{split}\]

Note that since $S^\1$ belongs to the isomorphism class of $\Sigma S$, there exists some isomorphism $t:\Sigma S\xrightarrow\cong S^\1$, which we can use to construct a natural isomorphism $S^\1\otimes-\cong\Sigma$:
\[S^\1\otimes X\xrightarrow{t\otimes X}\Sigma S\otimes X\xrightarrow{e_{S,X}}\Sigma(S\otimes X)\xrightarrow{\Sigma\lambda_X}\Sigma X.\]
The last two arrows are natural in $X$ by definition. The first arrow is natural in $X$ by functoriality of $-\otimes-$. Furthermore, under this isomorphism $e_{X,Y}:\Sigma X\otimes Y\xrightarrow\cong \Sigma(X\otimes Y)$ corresponds to the associator, by commutativity of the following diagram:
% https://q.uiver.app/#q=WzAsOSxbMCwwLCIoU15cXDFcXG90aW1lcyBYKVxcb3RpbWVzIFkiXSxbMCwyLCJTXlxcMVxcb3RpbWVzKFhcXG90aW1lcyBZKSJdLFsxLDAsIihcXFNpZ21hIFNcXG90aW1lcyBYKVxcb3RpbWVzIFkiXSxbMiwwLCJcXFNpZ21hKFNcXG90aW1lcyBYKVxcb3RpbWVzIFkiXSxbMywwLCJcXFNpZ21hIFhcXG90aW1lcyBZIl0sWzMsMiwiXFxTaWdtYShYXFxvdGltZXMgWSkiXSxbMSwyLCJcXFNpZ21hIFNcXG90aW1lcyAoWFxcb3RpbWVzIFkpIl0sWzIsMiwiXFxTaWdtYShTXFxvdGltZXMgKFhcXG90aW1lcyBZKSkiXSxbMiwxLCJcXFNpZ21hKChTXFxvdGltZXMgWClcXG90aW1lcyBZKSJdLFswLDEsIlxcYWxwaGEiLDJdLFswLDIsIih0XFxvdGltZXMgWClcXG90aW1lcyBZIl0sWzIsMywiZV97UyxYfVxcb3RpbWVzIFkiXSxbMyw0LCJcXFNpZ21hXFxsYW1iZGFfWFxcb3RpbWVzIFkiXSxbNCw1LCJlX3tYLFl9Il0sWzEsNiwidFxcb3RpbWVzIChYXFxvdGltZXMgWSkiLDJdLFs2LDcsImVfe1MsWFxcb3RpbWVzIFl9IiwyXSxbNyw1LCJcXFNpZ21hXFxsYW1iZGFfe1hcXG90aW1lcyBZfSIsMl0sWzIsNiwiXFxhbHBoYSJdLFszLDgsImVfe1NcXG90aW1lcyBYLFl9IiwyXSxbOCw3LCJcXFNpZ21hXFxhbHBoYSIsMl0sWzgsNSwiXFxTaWdtYShcXGxhbWJkYV9YXFxvdGltZXMgWSkiLDFdXQ==
\[\begin{tikzcd}
	{(S^\1\otimes X)\otimes Y} & {(\Sigma S\otimes X)\otimes Y} & {\Sigma(S\otimes X)\otimes Y} & {\Sigma X\otimes Y} \\
	&& {\Sigma((S\otimes X)\otimes Y)} \\
	{S^\1\otimes(X\otimes Y)} & {\Sigma S\otimes (X\otimes Y)} & {\Sigma(S\otimes (X\otimes Y))} & {\Sigma(X\otimes Y)}
	\arrow["\alpha"', from=1-1, to=3-1]
	\arrow["{(t\otimes X)\otimes Y}", from=1-1, to=1-2]
	\arrow["{e_{S,X}\otimes Y}", from=1-2, to=1-3]
	\arrow["{\Sigma\lambda_X\otimes Y}", from=1-3, to=1-4]
	\arrow["{e_{X,Y}}", from=1-4, to=3-4]
	\arrow["{t\otimes (X\otimes Y)}"', from=3-1, to=3-2]
	\arrow["{e_{S,X\otimes Y}}"', from=3-2, to=3-3]
	\arrow["{\Sigma\lambda_{X\otimes Y}}"', from=3-3, to=3-4]
	\arrow["\alpha", from=1-2, to=3-2]
	\arrow["{e_{S\otimes X,Y}}"', from=1-3, to=2-3]
	\arrow["\Sigma\alpha"', from=2-3, to=3-3]
	\arrow["{\Sigma(\lambda_X\otimes Y)}"{description}, from=2-3, to=3-4]
\end{tikzcd}\]
The left square commutes by naturality of $\alpha$. Commutativity of the middle square is axiom TT4 for a tensor triangulated category. Commutativity of the right trapezoid is naturality of $e$. Finally the bottom triangle commutes by coherence for monoidal categories and functoriality of $\Sigma$.

\begin{remark}
	In light of the above discussion, from now on we will always assume that $\Sigma=S^\1\otimes-$ and $e_{X,Y}:\Sigma X\otimes Y\xrightarrow\cong\Sigma(X\otimes Y)$ is the associator $\alpha:(S^\1\otimes X)\otimes Y\xrightarrow\alpha S^\1\otimes(X\otimes Y)$.
\end{remark}

Given some $a\in A$, we define $\Sigma^a:=S^a\otimes-$ and $\Omega^a:=\Sigma^{-a}=S^{-a}\otimes-$. We specifically define $\Omega:=\Omega^\1$, and we are assuming $\Sigma=\Sigma^\1$. Then it turns out that $\Omega^a$ and $\Sigma^a$ form an adjoint equivalence of $\cSH$:

\begin{proposition}\label{Sigma^a,Sigma^-a_adjoint_equiv}
	For each $a\in A$, the isomorphisms $\eta^a_X:X\to\Sigma^{a}\Omega^a X$ and $\vare^a_X:\Omega^a\Sigma^a X\to X$ defined in \autoref{stable_homotopy_cat_defn} exhibit an adjoint autoequivalence $(\Omega^a,\Sigma^a,\eta^a,\vare^a)$ of $\cSH$.
\end{proposition}
\begin{proof}
	In this proof, we will freely employ the coherence theorem for monoidal categories (see \cite{MLCoherence}), which essentially tells us that we may assume we are working in a strict monoidal category (i.e., that the associators and unitors and are identities). Then $\eta^a_X$ and $\vare^a_X$ become simply the maps
	\[\eta^a_X:X\xrightarrow{\phi_{a,-a}\otimes X}S^{a}\otimes S^{-a}\otimes X\qquad\text{and}\qquad\vare_X^a:S^{-a}\otimes S^{a}\otimes X\xrightarrow{\phi_{-a,a}^{-1}\otimes X}X.\]
	That these maps are natural in $X$ follows by functoriality of $-\otimes-$. Now, recall that in order to show that these natural isomorphisms form an \emph{adjoint} equivalence, it suffices to show that the natural isomorphisms $\eta^a:\Id_\cSH\Rightarrow\Omega^a\Sigma^a$ and $\vare^a:\Sigma^a\Omega^a\Rightarrow\Id_\cSH$ satisfy one of the two zig-zag identities:
	% https://q.uiver.app/#q=WzAsNixbMCwwLCJcXE9tZWdhXmEiXSxbMSwwLCJcXE9tZWdhXmFcXFNpZ21hXmFcXE9tZWdhXmEiXSxbMSwxLCJcXE9tZWdhXmEiXSxbMiwwLCJcXFNpZ21hXmFcXE9tZWdhXmFcXFNpZ21hXmEiXSxbMywwLCJcXFNpZ21hXmEiXSxbMiwxLCJcXFNpZ21hXmEiXSxbMCwxLCJcXE9tZWdhXmFcXGV0YV5hIl0sWzEsMiwiXFx2YXJlcHNpbG9uXmFcXE9tZWdhXmEiXSxbMyw1LCJcXFNpZ21hXmFcXHZhcmVwc2lsb25eYSIsMl0sWzQsNSwiIiwyLHsibGV2ZWwiOjIsInN0eWxlIjp7ImhlYWQiOnsibmFtZSI6Im5vbmUifX19XSxbMCwyLCIiLDIseyJsZXZlbCI6Miwic3R5bGUiOnsiaGVhZCI6eyJuYW1lIjoibm9uZSJ9fX1dLFs0LDMsIlxcZXRhXmFcXFNpZ21hXmEiLDJdXQ==
	\[\begin{tikzcd}
		{\Omega^a} & {\Omega^a\Sigma^a\Omega^a} & {\Sigma^a\Omega^a\Sigma^a} & {\Sigma^a} \\
		& {\Omega^a} & {\Sigma^a}
		\arrow["{\Omega^a\eta^a}", from=1-1, to=1-2]
		\arrow["{\varepsilon^a\Omega^a}", from=1-2, to=2-2]
		\arrow["{\Sigma^a\varepsilon^a}"', from=1-3, to=2-3]
		\arrow[Rightarrow, no head, from=1-4, to=2-3]
		\arrow[Rightarrow, no head, from=1-1, to=2-2]
		\arrow["{\eta^a\Sigma^a}"', from=1-4, to=1-3]
	\end{tikzcd}\]
	(see \cite[Lemma 3.2]{nlab:adjoint_equivalence}). We will show that the left is satisfied. Unravelling definitions, we simply wish to show that the following diagram commutes for all $X$ in $\cSH$:
	% https://q.uiver.app/#q=WzAsMyxbMCwwLCJTXnstYX1cXG90aW1lcyBYIl0sWzEsMCwiU157LWF9XFxvdGltZXMgU15hXFxvdGltZXMgU157LWF9XFxvdGltZXMgWCJdLFsxLDEsIlNeey1hfVxcb3RpbWVzIFgiXSxbMCwxLCJTXnstYX1cXG90aW1lcyBcXHBoaV97YSwtYX1cXG90aW1lcyBYIl0sWzEsMiwiXFxwaGlfey1hLGF9XnstMX1cXG90aW1lcyBTXnstYX1cXG90aW1lcyBYIl0sWzAsMiwiIiwyLHsibGV2ZWwiOjIsInN0eWxlIjp7ImhlYWQiOnsibmFtZSI6Im5vbmUifX19XV0=
	\[\begin{tikzcd}
		{S^{-a}\otimes X} & {S^{-a}\otimes S^a\otimes S^{-a}\otimes X} \\
		& {S^{-a}\otimes X}
		\arrow["{S^{-a}\otimes \phi_{a,-a}\otimes X}", from=1-1, to=1-2]
		\arrow["{\phi_{-a,a}^{-1}\otimes S^{-a}\otimes X}", from=1-2, to=2-2]
		\arrow[Rightarrow, no head, from=1-1, to=2-2]
	\end{tikzcd}\]
	Yet this is simply the diagram obtained by applying $-\otimes X$ to the associativity coherence diagram for the $\phi_{a,b}$'s, so it does commute, as desired.
\end{proof}

Given two objects $X$ and $Y$ in $\cSH$, we extend the abelian group $[X,Y]$ into an $A$-graded abelian group $[X,Y]_*$ by defining $[X,Y]_a:=[S^a\otimes X,Y]$.

\begin{itemize}
	\item Given some $a\in A$, we will define $\Sigma^a:=S^a\otimes-$ and $\Omega^a:=\Sigma^{-a}=S^{-a}\otimes-$, so that in particular $\Sigma=\Sigma^\1$.
	\item Given two objects $X$ and $Y$, we denote the hom-abelian group of morphisms from $X$ to $Y$ in $\cSH$ by $[X,Y]$, and we denote the internal hom object by $F(X,Y)$. We will often refer to morphisms in $\cSH$ as \emph{classes}, as we will think of them as representing homotopy classes of maps.
	\item Given two objects $X$ and $Y$ in $\cSH$, we may extend the abelian group $[X,Y]$ to an $A$-graded abelian group $[X,Y]_*$ defined by
	\[[X,Y]_a:=[\Sigma^aX,Y]=[S^a\otimes X,Y].\]
	(See \autoref{(co)algebra} for a review of the theory of $A$-graded abelian groups, rings, modules, etc.)
	\item Given an object $X$ in $\cSH$ and some $a\in A$, define the abelian group 
	\[\pi_a(X):=[S^a,X],\]
	and write $\pi_*(X)$ for the associated $A$-graded abelian group $\bigoplus_{a\in A}\pi_a(X)$. We call $\pi_a(X)$ the \emph{$a^\text{th}$ stable homotopy group of $X$}.
	\item Given two objects $E$ and $X$ in $\cSH$, we define the $A$-graded abelian groups $E_*(X)$ and $E^*(X)$ by
	\[E_a(X):=\pi_a(E\otimes X)=[S^a,E\otimes X]\qquad\text{and}\qquad E^a(X):=[X,S^a\otimes E].\]
	We refer to the functor $E_*(-)$ as the \emph{homology theory represented by $E$}, or just $E$-homology, and we refer to $E^*(-)$ as the \emph{cohomology theory represented by $E$}, or just $E$-cohomology.
\end{itemize}

From now on, we fix the data of a stable homotopy category $\cSH$ given above once and for all. We first would like to make some remarks on the above definition. To start with, note that 

%Whether or not this happens is essentially the entire discussion of Dugger's paper \cite{Dugger_2014}, and as it turns out, $\pi_\ast(E)$ is in fact a graded ring provided we can choose these morphisms to be \emph{coherent}, in the following sense:
%
%\begin{theorem}[{\cite[Proposition 7.1]{Dugger_2014}}]\label{coherent_existence}
	%There exists a coherent family of isomorphisms
	%\[\phi_{a,b}:S^{a+b}\xrightarrow\cong S^a\otimes S^b\]
	%in the sense of \autoref{coherent_isos},
	%and in particular, the set of such coherent families is in bijective correspondence with the set of normalized $2$-cocycles $Z^2(A;\mathrm{Aut}(S))_\mathit{norm}$, i.e., the set of functions $\alpha:A\times A\to\mathrm{Aut}(S)$ such that $\alpha(0,0)=\id_S$ and for all $a,b,c\in A$, $\alpha(a+b,c)\cdot\alpha(a,b)=\alpha(b,c)\cdot\alpha(a,b+c)$. 
%\end{theorem}
%
%Thus, from now on we will suppose once and for all we have fixed a coherent family $\{\phi_{a,b}\}_{a,b\in A}$. Such a coherent family has very nice properties, in particular:

\begin{remark}\label{unique_comp_Sas}
	Note that by induction the coherence conditions say that given any $a_1,\ldots,a_n\in A$ and $b_1,\ldots,b_m\in A$ such that $a_1+\cdots+a_n=b_1+\cdots+b_m$ and any fixed parenthesizations of $X=S^{a_1}\otimes\cdots\otimes S^{a_b}$ and $Y=S^{b_1}\otimes\cdots\otimes S^{b_m}$, there is a \emph{unique} isomorphism $X\to Y$ that can be obtained by forming formal compositions of products of $\phi_{a,b}$, identities, associators, and their inverses.
\end{remark}

Of course, we get our desired result: $\pi_*(E)$ is indeed an $A$-graded ring if $E$ is a monoid object.

\begin{proposition}\label{pi_*E_is_ring_for_E_monoid}
	Let $(E,\mu,e)$ be a commutative monoid object in $\cSH$, and consider the multiplication map $\pi_*(E)\times\pi_*(E)\to\pi_*(E)$ which sends classes $x:S^a\to E$ and $y:S^b\to E$ to the composition
	\[S^{a+b}\xrightarrow{\phi_{a,b}}S^a\otimes S^b\xrightarrow{x\otimes y}E\otimes E\xrightarrow\mu E.\]
	Then this endows $\pi_*(E)$ with the structure of an $A$-graded ring with unit $e\in\pi_0(E)=[S,E]$.
\end{proposition}
\begin{proof}
	See \autoref{pi_*E_is_ring_for_E_monoid_appendix}. 
\end{proof}

Furthermore, it turns out that if $E$ is a \emph{commutative} monoid object in $\cSH$, then $\pi_\ast(E)$ is ``$A$-graded commutative,'' in the following sense:

\begin{proposition}
	For all $a,b\in A$ there exists an element $\theta_{a,b}\in\pi_0(S)=[S,S]$ (determined by choice of coherent family $\{\phi_{a,b}\}$) such that given any commutative monoid object $(E,\mu,e)$ in $\cSH$, the $A$-graded ring structure on $\pi_\ast(E)$ (\autoref{pi_*E_is_ring_for_E_monoid}) has a commutativity formula given by
	\[x\cdot y=y\cdot x\cdot (e\circ\theta_{a,b})\]
	for all $x\in\pi_a(E)$ and $y\in\pi_b(E)$.
	
	Furthermore, $\theta_{0,a}=\theta_{a,0}=\id_S$ for all $a\in A$, so that if either $x$ or $y$ has degree $0$, $x\cdot y=y\cdot x$.
\end{proposition}
\begin{proof}
	See \autoref{pi_*(E)_is_A-graded_commutative_if_E_is_commutative} and \autoref{theta_a,0=theta_0,a=id_S}.
\end{proof}

The last ingredient in order to develop the Adams spectral sequence abstractly is a notion of \emph{cellularity} in $\cSH$:

\begin{definition}\label{cellular}
	Define the class of \emph{cellular} objects in $\cSH$ to be the smallest class of objects such that:
	\begin{enumerate}
		\item For all $a\in A$, $S^a$ is cellular.
		\item If we have a distinguished triangle
		\[X\to Y\to Z\to\Sigma X(=S^\1\otimes X)\]
		such that two of the three objects $X$, $Y$, and $Z$ are cellular, than the third object is also cellular.
		\item Given a collection of cellular objects $X_i$ indexed by some small set $I$, $\bigoplus_{i\in I} X_i$ is cellular.
	\end{enumerate}
\end{definition}

\subsection{Construction of the Adams spectral sequence}

In what follows, let $E$ be a commutative monoid object in $\cSH$.

\begin{definition}\label{mASS}
	Let $\ol E$ be the fiber of the unit map $e:S\to E$ (\autoref{fiber}), and for $s\geq0$ define
	\[Y_s:=\ol E^s\otimes Y,\qquad W_s = E\otimes Y_s=E\otimes(\ol E^s\otimes Y),\]
	where recall for $s>0$, $\ol E^s$ denotes the $s$-fold product parenthesized as $\ol E\otimes(\ol E\otimes\cdots(\ol E\otimes\ol E))$, and $\ol E^0:=S$. Then we get fiber sequences
	\[Y_{s+1}\xrightarrow{i_s}Y_s\xrightarrow{j_s} W_s\xrightarrow{k_s}\Sigma Y_{s+1}(=S^\1\otimes Y_{s+1})\]
	obtained by applying $-\otimes Y_s$ to the sequence
	\[\ol E\to S\xrightarrow eE\to\Sigma\ol E\]
	(and applying the necessary associator and unitor isomorphisms). These sequences can be spliced together to form the \emph{(canonical) Adams filtration of $Y$}:
	\[\begin{tikzcd}
		\cdots & {Y_3} & {Y_2} & {Y_1} & {Y_0=Y} \\
		& {W_3} & {W_2} & {W_1} & {W_0}
		\arrow[from=1-1, to=1-2]
		\arrow["{i_2}", from=1-2, to=1-3]
		\arrow["{i_1}", from=1-3, to=1-4]
		\arrow["{i_0}", from=1-4, to=1-5]
		\arrow["{j_0}", from=1-5, to=2-5]
		\arrow["{k_0}"', dashed, from=2-5, to=1-4]
		\arrow["{j_1}", from=1-4, to=2-4]
		\arrow["{k_1}"', dashed, from=2-4, to=1-3]
		\arrow["{j_3}", from=1-2, to=2-2]
		\arrow["{j_2}", from=1-3, to=2-3]
		\arrow["{k_2}"', dashed, from=2-3, to=1-2]
	\end{tikzcd}\]
	where the diagonal dashed arrows are of degree $-\1$ (note these triangles do NOT commute in any sense). Now we may apply the functor $[X,-]_\ast$, and by \autoref{dist_tri_LES} we obtain an exact couple of $\bN\times A$-graded abelian groups:
	\[\begin{tikzcd}
		{[X,Y_\ast]_\ast} && {[X,Y_\ast]_\ast} \\
		\\
		&& {[X,W_\ast]_\ast}
		\arrow["{i_\aast}", from=1-1, to=1-3]
		\arrow["{j_\aast}", from=1-3, to=3-3]
		\arrow["{k_\aast}", from=3-3, to=1-1]
	\end{tikzcd}\]
	where $i_\aast$, $j_\aast$, and $k_\aast$ have $\bZ\times A$-degree $(-1,0)$, $(0,0)$, and $(1,-\1)$, respectively\footnote{Explicitly, the map $k_{s,a}:[X,W_s]_a\to [X,Y_{s+1}]_{a-\1}$ sends a map $f:S^a\otimes X\to W_s$ to the map $S^{a-\1}\otimes X\to Y_{s+1}$ corresponding under the isomorphism $[X,\Sigma Y_{s+1}]_\ast\cong[X,Y_{s+1}]_{\ast-\1}$ to the composition $k_s\circ f:S^a\otimes X\to\Sigma Y_{s+1}$.}. The standard argument yields an $\bN\times A$-graded spectral sequence called from this exact couple (cf.\ Section 5.9 of \cite{Weibel_1994}) with $E_1$ page given by 
	\[E_1^{s,a}=[X,W_s]_{a}\]
	and $r^\text{th}$ differential of $\bZ\times A$-degree $(r,-\1)$:
	\[d_r:E_r^{s,a}\to E_r^{s+r,a-\1}.\]
	A priori, this is all $\bN\times A$-graded, but we regard it as being $\bZ\times A$-graded by setting $E_r^{s,a}:=0$ for $s<0$ and trivially extending the definition of the differentials to these zero groups. This spectral sequence is called the \emph{$E$-Adams spectral sequence} for the computation of $[X,Y]_\ast$. The index $s$ is called the \emph{Adams filtration} and $a$ is the \emph{stem}.
\end{definition}

\subsection{Monoid objects in \texorpdfstring{$\cSH$}{TEXT}}

We have constructed an Adams spectral sequence, but as it currently stands we do not yet know why it is useful. To start with, we'd like to provide a characterization of its $E_1$ and $E_2$ pages in terms of something more algebraic. To start, we first need to develop some theory of the algebra of monoid objects in $\cSH$. Much of this work is entirely straightforward although tedious to verify, so we relegate most of the proofs in this section to \Cref{monoid_objects}.

\begin{proposition}
	Let $(E,\mu,e)$ be a monoid object in $\cSH$. Then $E_*(-)$ is a functor from $\cSH$ to left $A$-graded $\pi_*(E)$-modules, where given some $X$ in $\cSH$, $E_*(X)$ may be endowed with the structure of a left $A$-graded $\pi_*(E)$-module via the map 
	\[\pi_*(E)\times E_*(X)\to E_*(X)\]
	which given $a,b\in A$, sends $x:S^a\to E$ and $y:S^b\to E\otimes X$ to the composition
	\[x\cdot y:S^{a+b}\cong S^a\otimes S^b\xrightarrow{x\otimes y}E\otimes (E\otimes X)\cong (E\otimes E)\otimes X\xrightarrow{\mu\otimes X}E\otimes X.\]
	Similarly, the assignment $X\mapsto X_*(E)$ is a functor from $\cSH$ to right $A$-graded $\pi_*(E)$-modules, where the structure map
	\[X_*(E)\times\pi_*(E)\to X_*(E)\]
	sends $x:S^a\to X\otimes E$ and $y:S^b\to E$ to the composition
	\[x\cdot y:S^{a+b}\cong S^a\otimes S^b\xrightarrow{x\otimes y}(X\otimes E)\otimes E\cong X\otimes(E\otimes E)\xrightarrow{X\otimes\mu}X\otimes E.\]
	Finally, $E_*(E)$ is a $\pi_*(E)$-bimodule, in the sense that the left and right actions of $\pi_*(E)$ are compatible, so that given $y, z\in\pi_*(E)$ and $x\in E_*(E)$, $y\cdot(x\cdot z)=(y\cdot x)\cdot z$.
\end{proposition}
\begin{proof}
	See \autoref{module}.
\end{proof}

\begin{definition}\label{flat}
	Given a monoid object $E$ in $\cSH$, we say $E$ is \emph{flat} if the canonical right $\pi_*(E)$-module structure on $E_*(E)$ (see the above proposition) is that of a flat module.
\end{definition}

\subsection{The \texorpdfstring{$E_1$}{TEXT} page}

The goal of this subsection is to provide the following characterization for the $E_1$ page of the Adams spectral sequence:

\begin{theorem}
	Let $E$ be a flat commutative monoid object in $\cSH$, and let $X$ and $Y$ be two objects in $\cSH$ such that $E_\ast(X)$ is a projective module over $\pi_\ast(E)$. Then for all $s\geq0$ and $a\in A$, we have isomorphisms in the associated $E$-Adams spectral sequence
	\[E_1^{s,a}\cong\Hom_{E_\ast(E)}^{a}(E_\ast(X),E_\ast(W_s))\]
	Furthermore, under these isomorphisms, the differential $d_1:E_1^{s,a}\to E_1^{s+1,a-\1}$ corresponds to the map
	\[\Hom_{E_\ast(E)}^{a}(E_\ast(X),E_\ast(W_s))\to\Hom_{E_\ast(E)}^{a-\1}(E_\ast(X),E_\ast(X\otimes W_{s+1}))\]
	which sends a map $f:E_\ast(X)\to E_{\ast+a}(W_s)$ to the composition 
	\[E_\ast(X)\xrightarrow f E_{\ast+a}(W_s)\xrightarrow{(X\otimes h_s)_\ast}E_{\ast+a-\1}(X\otimes Y_{s+1})\xrightarrow{(X\otimes j_{s+1})_\ast}E_{\ast+a-\1}(X\otimes W_{s+1}).\]
\end{theorem}
\begin{proof}
	By \autoref{2.30_2.33}, for all $s\geq0$ and $t,w\in\bZ$, we have isomorphisms
	\[[X,E\otimes Y_s]_{t,w}\cong\Hom_{E_\ast(E)}^{t,w}(E_\ast(X),E_\ast(E\otimes Y_s)).\]
	since $W_s=E\otimes Y_s$, we have that
	\[E_1^{s,(t,w)}=[X,W_s]_{t,w}\cong\Hom_{E_\ast(E)}^{t,w}(E_\ast(X),E_\ast(W_s)),\]
	as desired. 
\end{proof}

\begin{definition}
	Let $(E,\mu,e)$ be a monoid object in $\cSH$. We say $E$ is \emph{flat} if the canonical right $\pi_\ast(E)$-module structure on $E_\ast(E)$ is that of a flat module.
\end{definition}

\subsection{The \texorpdfstring{$E_2$}{E\_2} page}

\subsection{Convergence}

\href{https://www.uio.no/studier/emner/matnat/math/MAT9580/v12/undervisningsmateriale/boardman-conditionally-1999.pdf}{convergence of spectral sequences}


% https://q.uiver.app/#q=WzAsMjIsWzEsMCwiU157YStifSIsWzAsNjAsNjAsMV1dLFszLDIsIlNeYVxcb3RpbWVzIFNeey0oMSwwKX1cXG90aW1lcyBTXntiKygxLDApfSIsWzAsNjAsNjAsMV1dLFszLDQsIkVcXG90aW1lcyBFXFxvdGltZXMgU157LSgxLDApfVxcb3RpbWVzIEVcXG90aW1lcyBaIl0sWzMsNSwiRVxcb3RpbWVzIEVcXG90aW1lcyBTXnstKDEsMCl9XFxvdGltZXMgRVxcb3RpbWVzIFNeeygxLDApfVxcb3RpbWVzIFgiXSxbMyw2LCJFXFxvdGltZXMgRVxcb3RpbWVzIFNeey0oMSwwKX1cXG90aW1lcyBTXnsoMSwwKX1cXG90aW1lcyBFXFxvdGltZXMgWCJdLFszLDcsIkVcXG90aW1lcyBFXFxvdGltZXMgRVxcb3RpbWVzIFgiXSxbMyw4LCJFXFxvdGltZXMgRVxcb3RpbWVzIFgiXSxbMCwyLCJTXnstKDEsMCl9XFxvdGltZXMgU15hXFxvdGltZXMgU157YisoMSwwKX0iLFswLDYwLDYwLDFdXSxbMCw0LCJTXnstKDEsMCl9XFxvdGltZXMgRVxcb3RpbWVzIEVcXG90aW1lcyBFXFxvdGltZXMgWiJdLFswLDUsIlNeey0oMSwwKX1cXG90aW1lcyBFXFxvdGltZXMgRVxcb3RpbWVzIFoiXSxbMCw2LCJTXnstKDEsMCl9XFxvdGltZXMgRVxcb3RpbWVzIEVcXG90aW1lcyBTXnsoMSwwKX1cXG90aW1lcyBYIl0sWzEsOCwiU157LSgxLDApfVxcb3RpbWVzIFNeeygxLDApfVxcb3RpbWVzIEVcXG90aW1lcyBFXFxvdGltZXMgWCJdLFswLDgsIlNeey0oMSwwKX1cXG90aW1lcyBFXFxvdGltZXMgU157KDEsMCl9XFxvdGltZXMgRVxcb3RpbWVzIFgiXSxbMCwwLCJTXnstKDEsMCl9XFxvdGltZXMgU157YStiKygxLDApfSJdLFszLDAsIlNeYVxcb3RpbWVzIFNeYiJdLFsxLDMsIlNeey0oMSwwKX1cXG90aW1lcyBTXmFcXG90aW1lcyBFXFxvdGltZXMgWiJdLFsyLDMsIlNeYVxcb3RpbWVzIFNeey0oMSwwKX1cXG90aW1lcyBFXFxvdGltZXMgWiJdLFsxLDEsIlNeey0oMSwwKSthfVxcb3RpbWVzIFNee2IrKDEsMCl9IixbMCw2MCw2MCwxXV0sWzIsMSwiU157LSgxLDApK2F9XFxvdGltZXMgU157YisoMSwwKX0iLFswLDYwLDYwLDFdXSxbMiwwLCJTXnthK2J9IixbMCw2MCw2MCwxXV0sWzEsNSwiU157LSgxLDApfVxcb3RpbWVzIEVcXG90aW1lcyBFXFxvdGltZXMgRVxcb3RpbWVzIFNeeygxLDApfVxcb3RpbWVzIFgiXSxbMSw2LCJTXnstKDEsMCl9XFxvdGltZXMgU157KDEsMCl9XFxvdGltZXMgRVxcb3RpbWVzIEVcXG90aW1lcyBFXFxvdGltZXMgWCJdLFsxLDIsInhcXG90aW1lcyBTXnstKDEsMCl9XFxvdGltZXMgeSJdLFsyLDMsIkVcXG90aW1lcyBFXFxvdGltZXMgU157LSgxLDApfVxcb3RpbWVzIEVcXG90aW1lcyBoIl0sWzMsNCwiRVxcb3RpbWVzIEVcXG90aW1lcyBTXnstKDEsMCl9XFxvdGltZXMgXFx0YXVfe0UsU15cXDF9XFxvdGltZXMgWCJdLFs0LDUsIkVcXG90aW1lcyBFXFxvdGltZXNcXHBoaV97LSgxLDApLCgxLDApfV57LTF9XFxvdGltZXMgRVxcb3RpbWVzIFgiXSxbNSw2LCJFXFxvdGltZXMgXFxtdVxcb3RpbWVzIFgiXSxbOCw5LCJTXnstKDEsMCl9XFxvdGltZXMgRVxcb3RpbWVzIFxcbXVcXG90aW1lcyBaIiwyXSxbOSwxMCwiU157LSgxLDApfVxcb3RpbWVzIEVcXG90aW1lcyBFXFxvdGltZXMgaCIsMl0sWzEwLDExLCJTXnstKDEsMCl9XFxvdGltZXMgXFx0YXVfe0VcXG90aW1lcyBFLFNeeygxLDApfX1cXG90aW1lcyBYIiwxXSxbMTEsNiwiXFxwaGlfey0oMSwwKSwoMSwwKX1eey0xfVxcb3RpbWVzIEVcXG90aW1lcyBFXFxvdGltZXMgWCIsMl0sWzEwLDEyLCJTXnstKDEsMCl9XFxvdGltZXMgRVxcb3RpbWVzIFxcdGF1X3tFLFNeeygxLDApfX1cXG90aW1lcyBYIiwyXSxbMTIsMTEsIlNeey0oMSwwKX1cXG90aW1lcyBcXHRhdV97RSxTXnsoMSwwKX19XFxvdGltZXMgRVxcb3RpbWVzIFgiLDJdLFs4LDIsIlxcdGF1X3tTXnstKDEsMCl9LEVcXG90aW1lcyBFfVxcb3RpbWVzIEVcXG90aW1lcyBaIl0sWzAsMTMsIlxcc2lnbWFfey0oMSwwKSxhK2IrKDEsMCl9IiwyXSxbMTMsNywiU157LSgxLDApfVxcb3RpbWVzIFxcc2lnbWFfe2EsYisoMSwwKX0iLDJdLFsxNCwxLCJTXmFcXG90aW1lcyBcXHNpZ21hX3stKDEsMCksYisoMSwwKX0iXSxbNyw4LCJTXnstKDEsMCl9XFxvdGltZXMgeFxcb3RpbWVzIHkiLDJdLFs3LDEsIlxcdGF1X3tTXnstKDEsMCl9LFNeYX1cXG90aW1lcyBTXntiKygxLDApfSIsMCx7ImNvbG91ciI6WzAsNjAsNjBdfSxbMCw2MCw2MCwxXV0sWzcsMTUsIlNeey0oMSwwKX1cXG90aW1lcyBTXmFcXG90aW1lcyB5IiwxXSxbMTUsOCwiU157LSgxLDApfVxcb3RpbWVzIHhcXG90aW1lcyBFXFxvdGltZXMgWiIsMV0sWzE1LDE2LCJcXHRhdV97U157LSgxLDApfSxTXmF9XFxvdGltZXMgRVxcb3RpbWVzIFoiXSxbMTYsMiwieFxcb3RpbWVzIFNeey0oMSwwKX1cXG90aW1lcyBFXFxvdGltZXMgWiIsMV0sWzEsMTYsIlNeYVxcb3RpbWVzIFNeey0oMSwwKX1cXG90aW1lcyB5IiwxXSxbMTcsNywiXFxzaWdtYV97LSgxLDApLGF9XFxvdGltZXMgU157YisoMSwwKX0iLDIseyJjb2xvdXIiOlswLDYwLDYwXX0sWzAsNjAsNjAsMV1dLFswLDE3LCJcXHNpZ21hX3stKDEsMCkrYSxiKygxLDApfSIsMix7ImNvbG91ciI6WzAsNjAsNjBdfSxbMCw2MCw2MCwxXV0sWzE4LDEsIlxcc2lnbWFfe2EsLSgxLDApfVxcb3RpbWVzIFNee2IrKDEsMCl9IiwwLHsiY29sb3VyIjpbMCw2MCw2MF19LFswLDYwLDYwLDFdXSxbMTksMTQsIlxcc2lnbWFfe2EsYn0iXSxbMTksMTgsIlxcc2lnbWFfey0oMSwwKSthLGIrKDEsMCl9IiwwLHsiY29sb3VyIjpbMCw2MCw2MF19LFswLDYwLDYwLDFdXSxbOCwyMCwiU157LSgxLDApfVxcb3RpbWVzIEVcXG90aW1lcyBFXFxvdGltZXMgRVxcb3RpbWVzIGgiLDFdLFsyMCwxMCwiU157LSgxLDApfVxcb3RpbWVzIEVcXG90aW1lcyBcXG11XFxvdGltZXMgU157KDEsMCl9XFxvdGltZXMgWCIsMV0sWzIwLDMsIlxcdGF1X3tTXnstKDEsMCl9LEVcXG90aW1lcyBFfVxcb3RpbWVzIEVcXG90aW1lcyBTXnsoMSwwKX1cXG90aW1lcyBYIl0sWzAsMTksIiIsMCx7ImxldmVsIjoyLCJjb2xvdXIiOlswLDYwLDYwXSwic3R5bGUiOnsiaGVhZCI6eyJuYW1lIjoibm9uZSJ9fX1dLFsyMCwyMSwiU157LSgxLDApfVxcb3RpbWVzIFxcdGF1X3tFXFxvdGltZXMgRVxcb3RpbWVzIEVcXG90aW1lcyAsU157KDEsMCl9fVxcb3RpbWVzIFgiXSxbMjEsMTEsIlNeey0oMSwwKX1cXG90aW1lcyBTXnsoMSwwKX1cXG90aW1lcyBFXFxvdGltZXMgXFxtdVxcb3RpbWVzIFgiXSxbMjEsNCwiXFx0YXVfe1Neey0oMSwwKX1cXG90aW1lcyBTXnsoMSwwKX0sRVxcb3RpbWVzIEV9XFxvdGltZXMgRVxcb3RpbWVzIFgiXSxbMjEsNSwiXFxwaGleey0xfV97LSgxLDApLCgxLDApfVxcb3RpbWVzIEVcXG90aW1lcyBFXFxvdGltZXMgRVxcb3RpbWVzIFgiLDFdXQ==
\[\begin{tikzcd}[scale cd=0.75]
	{S^{-(1,0)}\otimes S^{a+b+(1,0)}} & \textcolor{rgb,255:red,214;green,92;blue,92}{S^{a+b}} & \textcolor{rgb,255:red,214;green,92;blue,92}{S^{a+b}} & {S^a\otimes S^b} \\
	& \textcolor{rgb,255:red,214;green,92;blue,92}{S^{-(1,0)+a}\otimes S^{b+(1,0)}} & \textcolor{rgb,255:red,214;green,92;blue,92}{S^{-(1,0)+a}\otimes S^{b+(1,0)}} \\
	\textcolor{rgb,255:red,214;green,92;blue,92}{S^{-(1,0)}\otimes S^a\otimes S^{b+(1,0)}} &&& \textcolor{rgb,255:red,214;green,92;blue,92}{S^a\otimes S^{-(1,0)}\otimes S^{b+(1,0)}} \\
	& {S^{-(1,0)}\otimes S^a\otimes E\otimes Z} & {S^a\otimes S^{-(1,0)}\otimes E\otimes Z} \\
	{S^{-(1,0)}\otimes E\otimes E\otimes E\otimes Z} &&& {E\otimes E\otimes S^{-(1,0)}\otimes E\otimes Z} \\
	{S^{-(1,0)}\otimes E\otimes E\otimes Z} & {S^{-(1,0)}\otimes E\otimes E\otimes E\otimes S^{(1,0)}\otimes X} && {E\otimes E\otimes S^{-(1,0)}\otimes E\otimes S^{(1,0)}\otimes X} \\
	{S^{-(1,0)}\otimes E\otimes E\otimes S^{(1,0)}\otimes X} & {S^{-(1,0)}\otimes S^{(1,0)}\otimes E\otimes E\otimes E\otimes X} && {E\otimes E\otimes S^{-(1,0)}\otimes S^{(1,0)}\otimes E\otimes X} \\
	&&& {E\otimes E\otimes E\otimes X} \\
	{S^{-(1,0)}\otimes E\otimes S^{(1,0)}\otimes E\otimes X} & {S^{-(1,0)}\otimes S^{(1,0)}\otimes E\otimes E\otimes X} && {E\otimes E\otimes X}
	\arrow["{x\otimes S^{-(1,0)}\otimes y}", from=3-4, to=5-4]
	\arrow["{E\otimes E\otimes S^{-(1,0)}\otimes E\otimes h}", from=5-4, to=6-4]
	\arrow["{E\otimes E\otimes S^{-(1,0)}\otimes \tau_{E,S^\1}\otimes X}", from=6-4, to=7-4]
	\arrow["{E\otimes E\otimes\sigma_{-(1,0),(1,0)}^{-1}\otimes E\otimes X}", from=7-4, to=8-4]
	\arrow["{E\otimes \mu\otimes X}", from=8-4, to=9-4]
	\arrow["{S^{-(1,0)}\otimes E\otimes \mu\otimes Z}"', from=5-1, to=6-1]
	\arrow["{S^{-(1,0)}\otimes E\otimes E\otimes h}"', from=6-1, to=7-1]
	\arrow["{S^{-(1,0)}\otimes \tau_{E\otimes E,S^{(1,0)}}\otimes X}"{description}, from=7-1, to=9-2]
	\arrow["{\sigma_{-(1,0),(1,0)}^{-1}\otimes E\otimes E\otimes X}"', from=9-2, to=9-4]
	\arrow["{S^{-(1,0)}\otimes E\otimes \tau_{E,S^{(1,0)}}\otimes X}"', from=7-1, to=9-1]
	\arrow["{S^{-(1,0)}\otimes \tau_{E,S^{(1,0)}}\otimes E\otimes X}"', from=9-1, to=9-2]
	\arrow["{\tau_{S^{-(1,0)},E\otimes E}\otimes E\otimes Z}", from=5-1, to=5-4]
	\arrow["{\sigma_{-(1,0),a+b+(1,0)}}"', from=1-2, to=1-1]
	\arrow["{S^{-(1,0)}\otimes \sigma_{a,b+(1,0)}}"', from=1-1, to=3-1]
	\arrow["{S^a\otimes \sigma_{-(1,0),b+(1,0)}}", from=1-4, to=3-4]
	\arrow["{S^{-(1,0)}\otimes x\otimes y}"', from=3-1, to=5-1]
	\arrow["{\tau_{S^{-(1,0)},S^a}\otimes S^{b+(1,0)}}", color={rgb,255:red,214;green,92;blue,92}, from=3-1, to=3-4]
	\arrow["{S^{-(1,0)}\otimes S^a\otimes y}"{description}, from=3-1, to=4-2]
	\arrow["{S^{-(1,0)}\otimes x\otimes E\otimes Z}"{description}, from=4-2, to=5-1]
	\arrow["{\tau_{S^{-(1,0)},S^a}\otimes E\otimes Z}", from=4-2, to=4-3]
	\arrow["{x\otimes S^{-(1,0)}\otimes E\otimes Z}"{description}, from=4-3, to=5-4]
	\arrow["{S^a\otimes S^{-(1,0)}\otimes y}"{description}, from=3-4, to=4-3]
	\arrow["{\sigma_{-(1,0),a}\otimes S^{b+(1,0)}}"', color={rgb,255:red,214;green,92;blue,92}, from=2-2, to=3-1]
	\arrow["{\sigma_{-(1,0)+a,b+(1,0)}}"', color={rgb,255:red,214;green,92;blue,92}, from=1-2, to=2-2]
	\arrow["{\sigma_{a,-(1,0)}\otimes S^{b+(1,0)}}", color={rgb,255:red,214;green,92;blue,92}, from=2-3, to=3-4]
	\arrow["{\sigma_{a,b}}", from=1-3, to=1-4]
	\arrow["{\sigma_{-(1,0)+a,b+(1,0)}}", color={rgb,255:red,214;green,92;blue,92}, from=1-3, to=2-3]
	\arrow["{S^{-(1,0)}\otimes E\otimes E\otimes E\otimes h}"{description}, from=5-1, to=6-2]
	\arrow["{S^{-(1,0)}\otimes E\otimes \mu\otimes S^{(1,0)}\otimes X}"{description}, from=6-2, to=7-1]
	\arrow["{\tau_{S^{-(1,0)},E\otimes E}\otimes E\otimes S^{(1,0)}\otimes X}", from=6-2, to=6-4]
	\arrow[draw={rgb,255:red,214;green,92;blue,92}, Rightarrow, no head, from=1-2, to=1-3]
	\arrow["{S^{-(1,0)}\otimes \tau_{E\otimes E\otimes E\otimes ,S^{(1,0)}}\otimes X}", from=6-2, to=7-2]
	\arrow["{S^{-(1,0)}\otimes S^{(1,0)}\otimes E\otimes \mu\otimes X}", from=7-2, to=9-2]
	\arrow["{\tau_{S^{-(1,0)}\otimes S^{(1,0)},E\otimes E}\otimes E\otimes X}", from=7-2, to=7-4]
	\arrow["{\sigma^{-1}_{-(1,0),(1,0)}\otimes E\otimes E\otimes E\otimes X}"{description}, from=7-2, to=8-4]
\end{tikzcd}\]

\end{document}