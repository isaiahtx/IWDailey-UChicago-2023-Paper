\documentclass[../main.tex]{subfiles}
\begin{document}


\subsection{Construction of the Adams spectral sequence}

The last thing we need before we can construct the Adams spectral sequence in $\cSH$ is the following:

\begin{proposition}\label{X,Y*_LES}
	Suppose we are given a distinguished triangle
	\[X\xr fY\xr gZ\xr h\Sigma X\]
	and some object $W$ in $\cSH$. Then there is an infinite long exact sequence of $A$-graded abelian groups:
	\[\cdots\to{[W,Z]}_{*+\n+\1}\xr\partial{[W,X]}_{*+\n}\xr{f_*}{[W,Y]}_{*+\n}\xr{g_*}{[W,Z]}_{*+\n}\xr{\partial}{[W,Z]}_{*+\n-\1}\to\cdots,\]
	where $\partial:{[W,Z]}_{*+\n+\1}\to{[W,X]}_{*+\n}$ sends a class $x:S^{a+\n+\1}\otimes W\to Z$ to the composition 
	\[S^{a+\n}\otimes W\xr{\phi_{-\1,a+\n+\1}}S^{-\1}\otimes S^{a+\n+\1}\otimes W\xrightarrow{S^{-\1}\otimes x}S^{-\1}\otimes Z\xr{\wt h}X,\]
	where here we are suppressing the associator from the notation, and $\wt h:\Omega Z=S^{-\1}\otimes Z\to X$ is the adjoint (\autoref{Sigma^a,Sigma^-a_adjoint_equiv}) of $h:Z\to\Sigma X$.
\end{proposition}
\begin{proof}
	In this proof, we will freely employ the coherence theorem for a symmetric monoidal category, which tells us we may assume associativity and unitality of $-\otimes-$ holds up to strict equality. Furthermore, we will simply write $\phi$ to refer to any isomorphism that can be constructed by composing copies of products of $\phi_{a,b}$'s, unitors, identities, associators, and their inverses (see \autoref{unique_comp_Sas}). Finally, given $n>0$, we will write $\Sigma^{-n}$ to denote the functor $\Omega^n=(S^{-\1})^n\otimes-$.
	
	For all $n>0$, the $\phi_{a,b}$'s yield natural isomorphisms
	\[s^{-n}_X:\Sigma^{-n}X=(S^{-\1})^n\otimes X\xr{\phi\otimes X}S^{-\n}\otimes X=\Omega^\n X.\]
	and
	\[s^n_X:\Sigma^nX\xrightarrow{\nu^n_X}(S^\1)^n\otimes X\xrightarrow{\phi\otimes X}S^\n\otimes X=\Sigma^\n X,\]
	where we recursively define $\nu^1:=\nu$ and $\nu^{n+1}$ is given by the composition
	\[\nu^{n+1}_X:\Sigma^{n+1}X=\Sigma^n\Sigma X\xrightarrow{\nu^n_{\Sigma X}}(S^\1)^n\otimes\Sigma X\xrightarrow{(S^\1)^n\otimes\nu_X}(S^\1)^{n}\otimes S^\1\otimes X=(S^\1)^{n+1}\otimes X.\]
	Finally, we define $s^0$ to be the identity natural transformation on $\cSH$. Then we get the following natural isomorphisms of $A$-graded abelian groups for all $n\in\bZ$
	\[\ell^{n}_{V}:{[W,\Sigma^n V]}_*\xrightarrow{{(s^n_V)}_*}{[W,\Sigma^\n V]}_*\xr{r^\n_{W,V}}{[W,V]}_{*-\n},\]
	where $r^\n_{W,V}$ is the natural isomorphism given as the composition
	\[[W,\Sigma^\n V]_*\xr\cong [S^{-\n}\otimes S^*\otimes W,V]\xr{{(\phi\otimes W)}^*}[S^{*-\n}\otimes W,V]=[W,V]_{*-\n},\]
	where the first isomorphism is the adjunction $\Omega^\n\dashv \Sigma^\n$ (\autoref{Sigma^a,Sigma^-a_adjoint_equiv}).
	Now, given $n\in\bZ$, consider the following diagram
	% https://q.uiver.app/#q=WzAsMTAsWzEsMCwie1tXLFxcU2lnbWFeblhdfV8qIl0sWzIsMCwie1tXLFxcU2lnbWFeblldfV8qIl0sWzMsMCwie1tXLFxcU2lnbWFeblpdfV8qIl0sWzEsMSwie1tXLFhdfV97Ki1cXG59Il0sWzIsMSwie1tXLFldfV97Ki1cXG59Il0sWzMsMSwie1tXLFpdfV97Ki1cXG59Il0sWzQsMCwie1tXLFxcU2lnbWFee24rMX1YXX1fKiJdLFs0LDEsIntbVyxYXX1feyotXFxuLVxcMX0iXSxbMCwwLCJ7W1csXFxTaWdtYV57bi0xfVpdfV8qIl0sWzAsMSwie1tXLFpdfV97Ki1cXG4rXFwxfSJdLFswLDEsIlxcU2lnbWFebmZfKiJdLFsxLDIsIlxcU2lnbWFebmdfKiJdLFszLDQsImZfKiIsMl0sWzQsNSwiZ18qIiwyXSxbMSw0LCJcXGVsbF5uX1kiLDJdLFsyLDUsIlxcZWxsXm5fWiIsMl0sWzIsNiwiaF9uIl0sWzUsNywiXFxwYXJ0aWFsIiwyXSxbNiw3LCJcXGVsbF57bisxfV9YIl0sWzAsMywiXFxlbGxebl9YIiwyXSxbOCw5LCJcXGVsbF57bi0xfV9aIiwyXSxbOSwzLCJcXHBhcnRpYWwiLDJdLFs4LDAsImhfe24tMX0iXV0=
	\begin{equation}\label{ladder_diagram_ell}\begin{tikzcd}
		{{[W,\Sigma^{n-1}Z]}_*} & {{[W,\Sigma^nX]}_*} & {{[W,\Sigma^nY]}_*} & {{[W,\Sigma^nZ]}_*} & {{[W,\Sigma^{n+1}X]}_*} \\
		{{[W,Z]}_{*-\n+\1}} & {{[W,X]}_{*-\n}} & {{[W,Y]}_{*-\n}} & {{[W,Z]}_{*-\n}} & {{[W,X]}_{*-\n-\1}}
		\arrow["{\Sigma^nf_*}", from=1-2, to=1-3]
		\arrow["{\Sigma^ng_*}", from=1-3, to=1-4]
		\arrow["{f_*}"', from=2-2, to=2-3]
		\arrow["{g_*}"', from=2-3, to=2-4]
		\arrow["{\ell^n_Y}"', from=1-3, to=2-3]
		\arrow["{\ell^n_Z}"', from=1-4, to=2-4]
		\arrow["{h_n}", from=1-4, to=1-5]
		\arrow["\partial"', from=2-4, to=2-5]
		\arrow["{\ell^{n+1}_X}", from=1-5, to=2-5]
		\arrow["{\ell^n_X}"', from=1-2, to=2-2]
		\arrow["{\ell^{n-1}_Z}"', from=1-1, to=2-1]
		\arrow["\partial"', from=2-1, to=2-2]
		\arrow["{h_{n-1}}", from=1-1, to=1-2]
	\end{tikzcd}\end{equation}
	where for $n\geq0$, $h_n=\Sigma^nh$, and for $n>0$, $h_{-n}=\Omega^{n-1}\wt h$ (where $\wt h:\Omega Z\to X$ is the adjoint of $h:Z\to\Sigma X$). We would like to show the bottom row is exact. The top row is exact since it is obtained by applying $[W,-]_*$ to a fiber sequence (see \autoref{dist_tri_LES} for full details), and we have constructed the vertical arrows to be isomorphisms. Thus it suffices to show each square commutes. The inner two squares commute by naturality of $\ell^n$. Thus, it further suffices to show the outermost squares commute. Since our choice of $n\in\bZ$ is arbitrary, we can just show the right square commutes. First consider the case that $n\geq0$, and consider the following diagram:
	% https://q.uiver.app/#q=WzAsNixbMCwwLCJbVyxcXFNpZ21hXm5aXV8qIl0sWzMsMCwiW1csXFxTaWdtYV57bisxfVhdXyoiXSxbMywzLCJbVyxYXV97Ki1cXG4tXFwxfSJdLFswLDMsIltXLFpdX3sqLVxcbn0iXSxbMSwxLCJbVyxcXFNpZ21hIFhdX3sqLVxcbn0iXSxbMiwyLCJbVyxcXFNpZ21hXlxcMVhdX3sqLVxcbn0iXSxbMCwxLCJcXFNpZ21hXm5oXyoiXSxbMSwyLCJcXGVsbF9YXntuKzF9Il0sWzAsMywiXFxlbGxfWl5uIiwyXSxbMywyLCJcXHBhcnRpYWwiLDJdLFs0LDUsInsoXFxudV9YKX1fKiJdLFs1LDIsInJeXFwxX3tXLFh9Il0sWzEsNCwiXFxlbGxebl97XFxTaWdtYSBYfSJdLFszLDQsImhfKiJdXQ==
	\[\begin{tikzcd}
		{[W,\Sigma^nZ]_*} &&& {[W,\Sigma^{n+1}X]_*} \\
		& {[W,\Sigma X]_{*-\n}} \\
		&& {[W,\Sigma^\1X]_{*-\n}} \\
		{[W,Z]_{*-\n}} &&& {[W,X]_{*-\n-\1}}
		\arrow["{\Sigma^nh_*}", from=1-1, to=1-4]
		\arrow["{\ell_X^{n+1}}", from=1-4, to=4-4]
		\arrow["{\ell_Z^n}"', from=1-1, to=4-1]
		\arrow["\partial"', from=4-1, to=4-4]
		\arrow["{{(\nu_X)}_*}", from=2-2, to=3-3]
		\arrow["{r^\1_{W,X}}", from=3-3, to=4-4]
		\arrow["{\ell^n_{\Sigma X}}", from=1-4, to=2-2]
		\arrow["{h_*}", from=4-1, to=2-2]
	\end{tikzcd}\]
	The leftmost region commutes by naturality of $\ell$. By unravelling how $r^\1_{W,X}$ and the adjoint $\wt h$ used in the definition of $\partial$ are defined, a simple diagram chase yields that the bottom triangle commutes. Thus, it remains to show the rightmost triangle in the above diagram commutes. To see this, note that by unravelling how $\ell$ and $r$ are defined, the rightmost square becomes
	% https://q.uiver.app/#q=WzAsMTMsWzIsMCwiW1csXFxTaWdtYV57bisxfVhdXyoiXSxbMCwyLCJbVyxcXFNpZ21hXntcXG59XFxTaWdtYSBYXV8qIl0sWzAsMywiW1Neey1cXG59XFxvdGltZXMgU14qXFxvdGltZXMgVyxcXFNpZ21hIFhdIl0sWzAsNCwiW1csXFxTaWdtYSBYXV97Ki1cXG59Il0sWzAsNSwiW1csXFxTaWdtYV5cXDFYXV97Ki1cXG59Il0sWzEsNSwiW1Neey1cXDF9XFxvdGltZXMgU157Ki1cXG59XFxvdGltZXMgVyxYXSJdLFsyLDUsIltXLFhdX3sqLVxcbi1cXDF9Il0sWzIsMSwiW1csKFNeXFwxKV57bisxfVxcb3RpbWVzIFhdXyoiXSxbMCwwLCJbVywoU15cXDEpXm5cXG90aW1lc1xcU2lnbWEgWF0iXSxbMiwyLCJbVyxcXFNpZ21hXntcXG4rXFwxfVhdXyoiXSxbMiwzLCJbU157LVxcbi1cXDF9XFxvdGltZXMgU14qXFxvdGltZXMgVyxYXSJdLFsxLDIsIltXLFxcU2lnbWFeXFxuXFxTaWdtYV5cXDEgWF1fKiJdLFsxLDMsIltTXnstXFxufVxcb3RpbWVzIFNeKlxcb3RpbWVzIFcsU15cXDFcXG90aW1lcyBYXSJdLFsxLDIsIlxcdGV4dHthZGp9IiwyXSxbMiwzLCJ7KFxccGhpXFxvdGltZXMgVyl9XioiLDJdLFs0LDUsIlxcdGV4dHthZGp9IiwyXSxbNSw2LCJ7KFxccGhpXFxvdGltZXMgVyl9XioiLDJdLFswLDcsInsoXFxudV9YXntuKzF9KX1fKiJdLFswLDgsInsoXFxudV57bn1fe1xcU2lnbWEgWH0pfV8qIiwyXSxbOCwxLCJ7KFxccGhpXFxvdGltZXNcXFNpZ21hIFgpfV8qIiwyXSxbOCw3LCJ7KChTXlxcMSleblxcb3RpbWVzXFxudV9YKX1fKiIsMl0sWzMsNCwieyhcXG51X1gpfV8qIiwyXSxbNyw5LCJ7KFxccGhpXFxvdGltZXMgWCl9XyoiXSxbOSwxMCwiXFx0ZXh0e2Fkan0iXSxbMTAsNiwieyhcXHBoaVxcb3RpbWVzIFcpfV4qIl0sWzEsMTEsInsoXFxTaWdtYV5cXG5cXG51X1gpfV8qIl0sWzExLDksInsoXFxwaGkgXFxvdGltZXMgWCl9XyoiLDJdLFs3LDExLCJ7KFxccGhpXFxvdGltZXMgWCl9XyoiLDFdLFsyLDEyLCJ7KFxcbnVfWCl9XyoiXSxbMTEsMTIsIlxcdGV4dHthZGp9Il0sWzEyLDQsInsoXFxwaGlcXG90aW1lcyBXKX1eKiIsMV0sWzUsMTAsInsoXFxwaGlcXG90aW1lcyBXKX1eKiIsMV1d
	\[\begin{tikzcd}
		{[W,(S^\1)^n\otimes\Sigma X]} && {[W,\Sigma^{n+1}X]_*} \\
		&& {[W,(S^\1)^{n+1}\otimes X]_*} \\
		{[W,\Sigma^{\n}\Sigma X]_*} & {[W,\Sigma^\n\Sigma^\1 X]_*} & {[W,\Sigma^{\n+\1}X]_*} \\
		{[S^{-\n}\otimes S^*\otimes W,\Sigma X]} & {[S^{-\n}\otimes S^*\otimes W,S^\1\otimes X]} & {[S^{-\n-\1}\otimes S^*\otimes W,X]} \\
		{[W,\Sigma X]_{*-\n}} \\
		{[W,\Sigma^\1X]_{*-\n}} & {[S^{-\1}\otimes S^{*-\n}\otimes W,X]} & {[W,X]_{*-\n-\1}}
		\arrow["{\text{adj}}"', from=3-1, to=4-1]
		\arrow["{{(\phi\otimes W)}^*}"', from=4-1, to=5-1]
		\arrow["{\text{adj}}"', from=6-1, to=6-2]
		\arrow["{{(\phi\otimes W)}^*}"', from=6-2, to=6-3]
		\arrow["{{(\nu_X^{n+1})}_*}", from=1-3, to=2-3]
		\arrow["{{(\nu^{n}_{\Sigma X})}_*}"', from=1-3, to=1-1]
		\arrow["{{(\phi\otimes\Sigma X)}_*}"', from=1-1, to=3-1]
		\arrow["{{((S^\1)^n\otimes\nu_X)}_*}"', from=1-1, to=2-3]
		\arrow["{{(\nu_X)}_*}"', from=5-1, to=6-1]
		\arrow["{{(\phi\otimes X)}_*}", from=2-3, to=3-3]
		\arrow["{\text{adj}}", from=3-3, to=4-3]
		\arrow["{{(\phi\otimes W)}^*}", from=4-3, to=6-3]
		\arrow["{{(\Sigma^\n\nu_X)}_*}", from=3-1, to=3-2]
		\arrow["{{(\phi \otimes X)}_*}"', from=3-2, to=3-3]
		\arrow["{{(\phi\otimes X)}_*}"{description}, from=2-3, to=3-2]
		\arrow["{{(\nu_X)}_*}", from=4-1, to=4-2]
		\arrow["{\text{adj}}", from=3-2, to=4-2]
		\arrow["{{(\phi\otimes W)}^*}"{description}, from=4-2, to=6-1]
		\arrow["{{(\phi\otimes W)}^*}"{description}, from=6-2, to=4-3]
	\end{tikzcd}\]
	The top right triangle commutes by how $\nu^{n+1}$ was defined. The top left oddly-shaped region commutes by functoriality of $-\otimes-$.  The middle right triangle commutes by coherence for the $\phi$'s. The middle left rectangle commutes by naturality of the adjunction isomorphism. Commutativity of the bottom left triangle is clear (do a diagram chase). Commutativity of the bottom right triangle is coherence for the $\phi$'s. Finally, commutativity of the remaining region is again coherence of the $\phi$'s, since the adjunction isomorphism are constructed using them (\autoref{Sigma^a,Sigma^-a_adjoint_equiv}).

	Now we consider the negative case: Unravelling definitions, given $n>0$, the rightmost square in diagram (\ref{ladder_diagram_ell}) for $-n$ becomes
	% https://q.uiver.app/#q=WzAsMTIsWzIsNSwiW1csWF1feyorXFxuLVxcMX0iXSxbMiwwLCJbVyxcXE9tZWdhXntuLTF9WF1fKiJdLFswLDAsIltXLFxcT21lZ2Fee259IFpdXyoiXSxbMSwzLCJbVyxcXE9tZWdhIFpdX3sqK1xcbi1cXDF9Il0sWzEsMSwiW1csXFxPbWVnYV57XFxuLVxcMX1cXE9tZWdhIFpdXyoiXSxbMiwxLCJbVyxcXE9tZWdhXntcXG4tXFwxfVhdXyoiXSxbMSwyLCJbU157XFxuLVxcMX1cXG90aW1lcyBTXipcXG90aW1lcyBXLFxcT21lZ2EgWl0iXSxbMiwyLCJbU157XFxuLVxcMX1cXG90aW1lcyBTXipcXG90aW1lcyBXLFhdIl0sWzEsNCwiW1NeXFwxXFxvdGltZXMgU157KitcXG4tXFwxfVxcb3RpbWVzIFcsWl0iXSxbMCwxLCJbVyxcXE9tZWdhXlxcbiBaXV8qIl0sWzAsNCwiW1NeXFxuXFxvdGltZXMgU14qXFxvdGltZXMgVyxaXSJdLFswLDUsIltXLFpdX3sqK1xcbn0iXSxbMiwxLCJcXE9tZWdhXntuLTF9XFx3dCBoXyoiXSxbMywwLCJcXHd0IGhfKiJdLFsyLDQsIihcXHBoaVxcb3RpbWVzXFxPbWVnYSBaKV8qIiwxXSxbMSw1LCIoXFxwaGlcXG90aW1lcyBYKV8qIl0sWzQsNSwiXFxPbWVnYV57XFxuLVxcMX1cXHd0IGhfKiJdLFs0LDYsIlxcdGV4dHthZGp9IiwyXSxbNSw3LCJcXHRleHR7YWRqfSJdLFs2LDcsIlxcd3QgaF8qIl0sWzYsMywiKFxccGhpXFxvdGltZXMgVyleKiIsMl0sWzcsMCwiKFxccGhpXFxvdGltZXMgVyleKiJdLFszLDgsIlxcdGV4dHthZGp9IiwyXSxbMiw5LCIoXFxwaGlcXG90aW1lcyBaKV8qIiwyXSxbOSwxMCwiXFx0ZXh0e2Fkan0iLDJdLFs4LDExLCIoXFxwaGlcXG90aW1lcyBXKV4qIiwxXSxbMTEsMCwiXFxwYXJ0aWFsIiwyXSxbMTAsMTEsIihcXHBoaVxcb3RpbWVzIFcpXioiLDJdLFs5LDQsIihcXHBoaVxcb3RpbWVzIFopXyoiLDJdLFsxMCw4LCIoXFxwaGlcXG90aW1lcyBXKV4qIl1d
	\begin{equation}\label{n_le_0diagram_LES}\begin{tikzcd}
		{[W,\Omega^{n} Z]_*} && {[W,\Omega^{n-1}X]_*} \\
		{[W,\Omega^\n Z]_*} & {[W,\Omega^{\n-\1}\Omega Z]_*} & {[W,\Omega^{\n-\1}X]_*} \\
		& {[S^{\n-\1}\otimes S^*\otimes W,\Omega Z]} & {[S^{\n-\1}\otimes S^*\otimes W,X]} \\
		& {[W,\Omega Z]_{*+\n-\1}} \\
		{[S^\n\otimes S^*\otimes W,Z]} & {[S^\1\otimes S^{*+\n-\1}\otimes W,Z]} \\
		{[W,Z]_{*+\n}} && {[W,X]_{*+\n-\1}}
		\arrow["{\Omega^{n-1}\wt h_*}", from=1-1, to=1-3]
		\arrow["{\wt h_*}", from=4-2, to=6-3]
		\arrow["{(\phi\otimes\Omega Z)_*}"{description}, from=1-1, to=2-2]
		\arrow["{(\phi\otimes X)_*}", from=1-3, to=2-3]
		\arrow["{\Omega^{\n-\1}\wt h_*}", from=2-2, to=2-3]
		\arrow["{\text{adj}}"', from=2-2, to=3-2]
		\arrow["{\text{adj}}", from=2-3, to=3-3]
		\arrow["{\wt h_*}", from=3-2, to=3-3]
		\arrow["{(\phi\otimes W)^*}"', from=3-2, to=4-2]
		\arrow["{(\phi\otimes W)^*}", from=3-3, to=6-3]
		\arrow["{\text{adj}}"', from=4-2, to=5-2]
		\arrow["{(\phi\otimes Z)_*}"', from=1-1, to=2-1]
		\arrow["{\text{adj}}"', from=2-1, to=5-1]
		\arrow["{(\phi\otimes W)^*}"{description}, from=5-2, to=6-1]
		\arrow["\partial"', from=6-1, to=6-3]
		\arrow["{(\phi\otimes W)^*}"', from=5-1, to=6-1]
		\arrow["{(\phi\otimes Z)_*}"', from=2-1, to=2-2]
		\arrow["{(\phi\otimes W)^*}", from=5-1, to=5-2]
	\end{tikzcd}\end{equation}
	The top right trapezoid commutes by functoriality of $-\otimes-$. The top left triangle commutes by coherence for the $\phi$'s. The middle right rectangle commutes by naturality of the adjunction. The right trapezoid below that commutes obviously. The bottom left triangle commutes by coherence of the $\phi$'s. The large middle left rectangle commutes by coherence for the $\phi$'s, again since the adjunction $\Sigma^{\n}\dashv\Omega^\n$ is constructed using the $\phi$'s. Finally, to see the bottom diagram commutes, we will chase some homogeneous element $f:S^{b+\n-\1}\otimes W\to\Omega Z$ around the region. Consider the following diagram:
	% https://q.uiver.app/#q=WzAsOCxbMSwwLCJTXntiK1xcbi1cXDF9XFxvdGltZXMgVyJdLFs0LDIsIlgiXSxbMCwwLCJTXnstXFwxfVxcb3RpbWVzIFNee2IrXFxufVxcb3RpbWVzIFciXSxbMCwxLCJTXnstXFwxfVxcb3RpbWVzIFNeXFwxXFxvdGltZXMgU157YitcXG4rXFwxfVxcb3RpbWVzIFciXSxbMCwyLCJTXnstXFwxfVxcb3RpbWVzIFNeXFwxXFxvdGltZXMgU157LVxcMX1cXG90aW1lcyBaIl0sWzEsMiwiU157LTF9XFxvdGltZXMgWiJdLFsyLDIsIlNeey1cXDF9XFxvdGltZXMgXFxTaWdtYSBYIl0sWzMsMiwiU157LVxcMX1cXG90aW1lcyBTXlxcMVxcb3RpbWVzIFgiXSxbMCwyLCJcXHBoaVxcb3RpbWVzIFciLDJdLFsyLDMsIlxccGhpXFxvdGltZXMgVyIsMl0sWzMsNCwiU157LVxcMX1cXG90aW1lcyBTXlxcMVxcb3RpbWVzIGYiLDJdLFs0LDUsIlxccGhpXFxvdGltZXMgWiIsMl0sWzUsNiwiU157LVxcMX1cXG90aW1lcyBoIl0sWzYsNywiU157LVxcMX1cXG90aW1lcyBcXG51X1giXSxbNywxLCJcXHBoaVxcb3RpbWVzIFgiXSxbMCw1LCJmIl0sWzMsMCwiXFxwaGlcXG90aW1lcyBXIiwxXV0=
	\[\begin{tikzcd}
		{S^{-\1}\otimes S^{b+\n}\otimes W} & {S^{b+\n-\1}\otimes W} \\
		{S^{-\1}\otimes S^\1\otimes S^{b+\n+\1}\otimes W} \\
		{S^{-\1}\otimes S^\1\otimes S^{-\1}\otimes Z} & {S^{-1}\otimes Z} & {S^{-\1}\otimes \Sigma X} & {S^{-\1}\otimes S^\1\otimes X} & X
		\arrow["{\phi\otimes W}"', from=1-2, to=1-1]
		\arrow["{\phi\otimes W}"', from=1-1, to=2-1]
		\arrow["{S^{-\1}\otimes S^\1\otimes f}"', from=2-1, to=3-1]
		\arrow["{\phi\otimes Z}"', from=3-1, to=3-2]
		\arrow["{S^{-\1}\otimes h}", from=3-2, to=3-3]
		\arrow["{S^{-\1}\otimes \nu_X}", from=3-3, to=3-4]
		\arrow["{\phi\otimes X}", from=3-4, to=3-5]
		\arrow["f", from=1-2, to=3-2]
		\arrow["{\phi\otimes W}"{description}, from=2-1, to=1-2]
	\end{tikzcd}\]
	By unravelling how the adjunction and $\wt h$ are defined, the two compositions around the outside of this diagram are the two morphisms obtained by chasing $f$ around the bottom region in diagram (\ref{n_le_0diagram_LES}). The top left triangle commutes by coherence of the $\phi$'s (\autoref{unique_comp_Sas}), while the bottom region commutes by functoriality of $-\otimes-$ and coherence of the $\phi$'s. Thus we've shown diagram (\ref{ladder_diagram_ell}) commutes, so the bottom row is exact, as desired.
\end{proof}

\begin{remark}\label{X,Y*_LES_compact}
	Expressed more compactly, the above proposition says that for each object $W$ and distinguished triangle
	\[X\xr fY\xr gZ\xr h\Sigma X\]
	in $\cSH$ gives rise to the following diagram of $A$-graded abelian groups
	% https://q.uiver.app/#q=WzAsMyxbMCwwLCJ7W1csWF19XyoiXSxbMSwwLCJ7W1csWV19XyoiXSxbMSwxLCJ7W1csWl19XyoiXSxbMCwxLCJmXyoiXSxbMSwyLCJnXyoiXSxbMiwwLCJcXHBhcnRpYWwiXV0=
	\[\begin{tikzcd}
		{{[W,X]}_*} & {{[W,Y]}_*} \\
		& {{[W,Z]}_*}
		\arrow["{f_*}", from=1-1, to=1-2]
		\arrow["{g_*}", from=1-2, to=2-2]
		\arrow["\partial", from=2-2, to=1-1]
	\end{tikzcd}\]
	which is exact at each vertex, and where $f_*$, $g_*$, and $\partial$ are $A$-graded homomorphisms of degree $0$, $0$, and $-\1$, respectively. Explicitly, $\partial$ sends a class $x:S^a\otimes W\to Z$ to the composition
	\[S^{a-\1}\otimes W\cong S^{-\1}\otimes S^a\otimes W\xr{S^{-\1}\otimes x}S^{-\1}\otimes Z\xr{S^{-\1}\otimes h}S^{-\1}\otimes\Sigma X\xr{S^{-\1}\otimes\nu_X}S^{-\1}\otimes S^\1\otimes X\xr{\phi_{-\1,\1}^{-1}\otimes X}X.\]
\end{remark}

In what follows, let $(E,\mu,e)$ be a commutative monoid object in $\cSH$. In this section we will freely use the coherence theorem for symmetric monoidal categories without comment, in particular, we will assume unitality and associativity hold up to strict equality.

\begin{definition}\label{mASS}
	Let $\ol E$ be the fiber of the unit map $e:S\to E$ (\autoref{fiber}). Let $Y_0:=Y$ and $W_0:= E\otimes Y$. Then for $s>0$, define
	\[Y_s:=\ol E^s\otimes Y,\qquad W_s := E\otimes Y_s=E\otimes\ol E^s\otimes Y,\]
	where $\ol E^s$ denotes the $s$-fold tensor product $\ol E\otimes\cdots\otimes\ol E$. Then we get fiber sequences
	\[Y_{s+1}\xrightarrow{i_s}Y_s\xrightarrow{j_s} W_s\xrightarrow{k_s}\Sigma Y_{s+1}\]
	obtained by applying $-\otimes Y_s$ to the fiber sequence
	\[\ol E\to S\xrightarrow eE\to\Sigma\ol E.\]
	We can splice these sequences together to get the \emph{(canonical) Adams filtration of $Y$}:
	% https://q.uiver.app/#q=WzAsOSxbMSwwLCJZXzMiXSxbMiwwLCJZXzIiXSxbMywwLCJZXzEiXSxbNCwwLCJZXzA9WSJdLFswLDAsIlxcY2RvdHMiXSxbMSwxLCJXXzMiXSxbMiwxLCJXXzIiXSxbMywxLCJXXzEiXSxbNCwxLCJXXzAiXSxbMCwxLCJpXzIiXSxbMSwyLCJpXzEiXSxbMiwzLCJpXzAiXSxbNCwwXSxbMCw1LCJqXzMiXSxbMSw2LCJqXzIiXSxbMiw3LCJqXzEiXSxbMyw4LCJqXzAiXSxbNiwwLCJrXzIiLDAseyJsYWJlbF9wb3NpdGlvbiI6MzAsInN0eWxlIjp7ImJvZHkiOnsibmFtZSI6ImRhc2hlZCJ9fX1dLFs3LDEsImtfMSIsMCx7ImxhYmVsX3Bvc2l0aW9uIjozMCwic3R5bGUiOnsiYm9keSI6eyJuYW1lIjoiZGFzaGVkIn19fV0sWzgsMiwia18wIiwwLHsibGFiZWxfcG9zaXRpb24iOjMwLCJzdHlsZSI6eyJib2R5Ijp7Im5hbWUiOiJkYXNoZWQifX19XV0=
	\[\begin{tikzcd}
		\cdots & {Y_3} & {Y_2} & {Y_1} & {Y_0=Y} \\
		& {W_3} & {W_2} & {W_1} & {W_0}
		\arrow["{i_2}", from=1-2, to=1-3]
		\arrow["{i_1}", from=1-3, to=1-4]
		\arrow["{i_0}", from=1-4, to=1-5]
		\arrow[from=1-1, to=1-2]
		\arrow["{j_3}", from=1-2, to=2-2]
		\arrow["{j_2}", from=1-3, to=2-3]
		\arrow["{j_1}", from=1-4, to=2-4]
		\arrow["{j_0}", from=1-5, to=2-5]
		\arrow["{k_2}"{pos=0.3}, dashed, from=2-3, to=1-2]
		\arrow["{k_1}"{pos=0.3}, dashed, from=2-4, to=1-3]
		\arrow["{k_0}"{pos=0.3}, dashed, from=2-5, to=1-4]
	\end{tikzcd}\]
	where here each $k_s$ is of degree $-\1$ (in particular, the above diagram does not commute in any sense), and each $i_s$ and $j_s$ have degree $0$. We can extend this diagram to the right by setting $Y_s=Y$, $W_s=0$, and $i_s=\id_Y$ for $s<0$. Then we may apply the functor ${[X,-]}_\ast$, and by \autoref{X,Y*_LES_compact}, we obtain the following $A$-graded unrolled exact couple (\autoref{unrolled_exact_couple}):
	% https://q.uiver.app/#q=WzAsMTAsWzAsMCwiXFxjZG90cyJdLFsxLDAsIntbWCxZX3tzKzJ9XX1fKiJdLFsyLDAsIntbWCxZX3tzKzF9XX1fKiJdLFszLDAsIntbWCxZX3tzfV19XyoiXSxbNCwwLCJ7W1gsWV97cy0xfV19XyoiXSxbNSwwLCJcXGNkb3RzIl0sWzIsMSwie1tYLFdfe3MrMX1dfV8qIl0sWzEsMSwie1tYLFdfe3MrMn1dfV8qIl0sWzQsMSwie1tYLFdfe3MtMX1dfV8qIl0sWzMsMSwie1tYLFdfe3N9XX1fKiJdLFswLDFdLFsxLDIsImlfe3MrMX0iXSxbMiwzLCJpX3MiXSxbMyw0LCJpX3tzLTF9Il0sWzQsNV0sWzIsNiwial97cysxfSJdLFs2LDEsIlxccGFydGlhbF97cysxfSJdLFsxLDcsImpfe3MrMn0iXSxbNCw4LCJqX3tzLTF9Il0sWzgsMywiXFxwYXJ0aWFsX3tzLTF9Il0sWzMsOSwial97c30iXSxbOSwyLCJcXHBhcnRpYWxfcyJdXQ==
	\[\begin{tikzcd}
		\cdots & {{[X,Y_{s+2}]}_*} & {{[X,Y_{s+1}]}_*} & {{[X,Y_{s}]}_*} & {{[X,Y_{s-1}]}_*} & \cdots \\
		& {{[X,W_{s+2}]}_*} & {{[X,W_{s+1}]}_*} & {{[X,W_{s}]}_*} & {{[X,W_{s-1}]}_*}
		\arrow[from=1-1, to=1-2]
		\arrow["{i_{s+1}}", from=1-2, to=1-3]
		\arrow["{i_s}", from=1-3, to=1-4]
		\arrow["{i_{s-1}}", from=1-4, to=1-5]
		\arrow[from=1-5, to=1-6]
		\arrow["{j_{s+1}}", from=1-3, to=2-3]
		\arrow["{\partial_{s+1}}", from=2-3, to=1-2]
		\arrow["{j_{s+2}}", from=1-2, to=2-2]
		\arrow["{j_{s-1}}", from=1-5, to=2-5]
		\arrow["{\partial_{s-1}}", from=2-5, to=1-4]
		\arrow["{j_{s}}", from=1-4, to=2-4]
		\arrow["{\partial_s}", from=2-4, to=1-3]
	\end{tikzcd}\]
	where here we are being abusive and writing $i_s:{[X,Y_{s+1}]}_*\to{[X,Y_{s}]}_*$ and $j_s:{[X,Y_s]}_*\to{[X,W_s]}_*$ to denote the pushforwards of $i_s:Y_{s+1}\to Y_s$ and $j_s:Y_s\to W_s$, respectively. Each $i_s$, $j_s$, and $\partial_s$ are $A$-graded homomorphisms of degrees $0$, $0$, and $-\1$, respectively. 

	By \autoref{SSeq_assoc_to_unrolled_EC}, we may associate a $\bZ\times A$-graded spectral sequence $r\mapsto(E_r^{*,*}(X,Y),d_r)$ to the above $A$-graded unrolled exact couple, where $d_r$ has $\bZ\times A$-degree $(r,-\1)$. We call this spectral sequence the \emph{$E$-Adams spectral sequence for the computation of ${[X,Y]}_*$}.
\end{definition}

\subsection{The \texorpdfstring{$E_1$}{E1} page}

The goal of this subsection is to provide a nicer characterization of the $E_1$ page of the $E$-Adams spectral sequence for the computation of ${[X,Y]}_*$. Given a monoid object $(E,\mu,e)$ in $\cSH$.

\begin{theorem}
	Let $(E,\mu,e)$ be a flat commutative monoid object in $\cSH$, and let $X$ and $Y$ be two objects in $\cSH$. Further suppose at least one of the following hold:\begin{enumerate}
		\item $E$ and $X$ are cellular objects (\autoref{cellular}) and $E_*(X)$ is a graded projective (\autoref{graded_projective_module}) left $\pi_*(E)$-module (via \autoref{module_main}).
  		\item There exists a collection of $a_i\in A$ indexed by some set $I$ such that $E\otimes X$ is a retract of $\bigoplus_i\Sigma^{a_i}E$
	\end{enumerate}
	
	Then for all $s\geq0$ and $a\in A$, we have isomorphisms in the associated $E$-Adams spectral sequence
	\[E_1^{s,a}(X,Y)\cong\Hom_{E_\ast(E)}^{a}(E_\ast(X),E_\ast(W_s))\]
	Furthermore, under these isomorphisms, the differential $d_1:E_1^{s,a}\to E_1^{s+1,a-\1}$ corresponds to the map
	\[\Hom_{E_\ast(E)}^{a}(E_\ast(X),E_\ast(W_s))\to\Hom_{E_\ast(E)}^{a-\1}(E_\ast(X),E_\ast(X\otimes W_{s+1}))\]
	which sends a map $f:E_\ast(X)\to E_{\ast+a}(W_s)$ to the composition 
	\[E_\ast(X)\xrightarrow f E_{\ast+a}(W_s)\xrightarrow{(X\otimes h_s)_\ast}E_{\ast+a-\1}(X\otimes Y_{s+1})\xrightarrow{(X\otimes j_{s+1})_\ast}E_{\ast+a-\1}(X\otimes W_{s+1}).\]
\end{theorem}
\begin{proof}
	By \autoref{2.30_2.33}, for all $s\geq0$ and $t,w\in\bZ$, we have isomorphisms
	\[{[X,E\otimes Y_s]}_{t,w}\cong\Hom_{E_\ast(E)}^{t,w}(E_\ast(X),E_\ast(E\otimes Y_s)).\]
	since $W_s=E\otimes Y_s$, we have that
	\[E_1^{s,(t,w)}={[X,W_s]}_{t,w}\cong\Hom_{E_\ast(E)}^{t,w}(E_\ast(X),E_\ast(W_s)),\]
	as desired. 
\end{proof}

\begin{definition}
	Let $(E,\mu,e)$ be a monoid object in $\cSH$. We say $E$ is \emph{flat} if the canonical right $\pi_\ast(E)$-module structure on $E_\ast(E)$ is that of a flat module.
\end{definition}

\subsection{The \texorpdfstring{$E_2$}{E\_2} page}

\subsection{Convergence}

\href{https://www.uio.no/studier/emner/matnat/math/MAT9580/v12/undervisningsmateriale/boardman-conditionally-1999.pdf}{convergence of spectral sequences}


\end{document}