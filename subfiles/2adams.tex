\documentclass[../main.tex]{subfiles}
\begin{document}

\subsection{Setup}\label{setup}

In order to construct an abstract version of the Adams spectral sequence, we need to work in some axiomatic version of a stable homotopy category $\cSH$ which acts like the familiar classical stable homotopy category $\hoSp$ (\Cref{classical}) or the motivic stable homotopy category $\SH\scS$ over some base scheme $\scS$ (\Cref{motivic}). As it turns out, practically all the data we need is the following:

\begin{definition}\label{stable_homotopy_cat}
	A \emph{stable homotopy category} is the following data:
	\begin{itemize}
		\item A closed tensor triangulated category $(\cSH,\otimes,S,\Sigma,\Omega)$ with arbitrary small (co)products.
		\item A pointed abelian group $(A,\1)$ and a homomorphism $h:(A,\1)\to(\Pic(\cSH),\Sigma S)$ of pointed groups (i.e., $\1$ is sent to the isomorphism class of $\Sigma S$), where $\Pic(\cSH)$ is the group of isomorphism classes of invertible objects in $\cSH$\footnote{Recall an object $X$ in a symmetric monoidal category is \textit{invertible} if there exists some object $Y$ in $\cSH$ and an isomorphism $S\cong Y\otimes X$. To see $\Sigma S$ is invertible, note that we have isomorphisms
		\[\Sigma S\otimes \Omega S\cong\Sigma(S\otimes\Omega S)\cong \Sigma(\Omega S\otimes S)\cong\Sigma\Omega S\otimes S\cong S\otimes S\cong S,\]
		where the first isomorphism is axiom TT1 for a tensor triangulated category (\autoref{tentri}), the second isomorphism is given by the symmetry in $\cSH$, the third isomorphism is again axiom TT1, the fourth isomorphism is the fact that $\Sigma$ and $\Omega$ for an adjoint equivalence, and finally the last isomorphism follows by the fact that $S$ is the monoidal unit in $\cSH$.}.
		\item For each $a\in A$, a chosen object $S^a$ in the isomorphism class $h(a)$.
	\end{itemize}
\end{definition}

Given an abstract stable homotopy category as above, we will always assume without loss of generality that $S^0=S$ and $\Sigma=S^\1\otimes-$ (by \autoref{tentri_characterize_shift}). we establish the following conventions:
\begin{itemize}
	\item Given objects $X_1,\ldots,X_n$ in $\cSH$, we write $X_1\otimes\cdots\otimes X_n$ to denote the object
	\[X_1\otimes(X_2\otimes\cdots(X_{n-1}\otimes X_n)).\]
	In particular, given an object $X$ and a natural number $n>0$, we write
	\[X^n:=\overbrace{X\otimes\cdots\otimes X}^\text{$n$ times}\qquad\text{and}\qquad X^0:=S.\]
	\item We denote the associator, symmetry, left unitor, and right unitor isomorphisms in $\cSH$ by
	\[\begin{split}
		\alpha_{X,Y,Z}:(X\otimes Y)\otimes Z&\xrightarrow\cong X\otimes(Y\otimes Z) \\
		\lambda_X:S\otimes  X&\xrightarrow\cong X 
	\end{split}\qquad\qquad\begin{split}
		\tau_{X,Y}:X\otimes Y &\xrightarrow\cong Y\otimes X\\
		\rho_X:X\otimes S&\xrightarrow\cong X.
	\end{split}\]
	Often we will suppress these isomorphisms from the notation (particularly the associators), choosing instead to denote them without their subscripts or simply with the symbol $\cong$.
	\item Given some $a\in A$, we define the functor $\Sigma^a:=S^a\otimes-$, so that in particular $\Sigma^\1=\Sigma$.
	\item Given two objects $X$ and $Y$, we denote the hom-abelian group of morphisms from $X$ to $Y$ in $\cSH$ by $[X,Y]$, and we denote the internal hom object by $F(X,Y)$. We will often refer to morphisms in $\cSH$ as \textit{classes}, as we will think of them as representing homotopy classes of maps.
	\item Given two objects $X$ and $Y$ in $\cSH$, we may extend the abelian group $[X,Y]$ to an $A$-graded abelian group $[X,Y]_*$ defined by
	\[[X,Y]_a:=[\Sigma^aX,Y]=[S^a\otimes X,Y].\]
	(See \autoref{(co)algebra} for a review of the theory of $A$-graded abelian groups, rings, modules, etc.)
	\item Given an object $X$ in $\cSH$ and some $a\in A$, define the abelian group 
	\[\pi_a(X):=[S^a,X],\]
	and write $\pi_*(X)$ for the associated $A$-graded abelian group $\bigoplus_{a\in A}\pi_a(X)$. We call $\pi_a(X)$ the \emph{$a^\text{th}$ stable homotopy group of $X$}.
	\item Given two objects $E$ and $X$ in $\cSH$, we define the $A$-graded abelian groups $E_*(X)$ and $E^*(X)$ by
	\[E_a(X):=\pi_a(E\otimes X)=[S^a,E\otimes X]\qquad\text{and}\qquad E^a(X):=[X,S^a\otimes E].\]
	We refer to the functor $E_*(-)$ as the \emph{homology theory represented by $E$}, or just $E$-homology, and we refer to $E^*(-)$ as the \emph{cohomology theory represented by $E$}, or just $E$-cohomology.
\end{itemize}

From now on, we fix the data of a stable homotopy category $\cSH$ given above once and for all. Observe that for all $a,b\in A$, the objects $S^{a+b}$ and $S^a\otimes S^b$ are isomorphic, since $h:A\to\Pic(\cSH)$ is a group homomorphism. Hence given a monoid object $(E,\mu,e)$ in $\cSH$ (\autoref{monoid_object}), supposing we had fixed isomorphisms $S^{a+b}\cong S^a\otimes S^b$ for all $a,b\in A$, we get a multiplication map $\pi_\ast(E)\times\pi_\ast(E)\to\pi_*(E)$ which sends classes $x:S^a\to E$ and $y:S^b\to E$ to the product
\[S^{a+b}\cong S^a\otimes S^b\xrightarrow{x\otimes y}E\otimes E\xrightarrow\mu E.\]
Naturally, we would like this product to make $\pi_\ast(E)$ into an $A$-graded ring (with unit $e\in\pi_0(E)=[S,E]$), rather than just an $A$-graded abelian group. This is essentially the entire discussion of Dugger's paper \cite{Dugger_2014}, and as it turns out, $\pi_\ast(E)$ is in fact a graded ring provided we can choose these morphisms to be \emph{coherent}, in the following sense:

\begin{definition}\label{coherent_isos}
	Suppose we have a family of isomorphisms
	\[\phi_{a,b}:S^{a+b}\xrightarrow\cong S^a\otimes S^b\]
	for all $a,b\in A$. We say this family is \emph{coherent} if:
	\begin{enumerate}
		\item For all $a\in A$, we have equalities $\phi_{a,0}=\rho_{S^a}^{-1}:S^a\to S^a\otimes S$ and $\phi_{0,a}=\lambda_{S^a}^{-1}:S^a\to S\otimes S^a$.
		\item For all $a,b,c\in A$, the following diagram commutes:
		\[\begin{tikzcd}
			{S^{a+b}\otimes S^c} & {S^{a+b+c}} & {S^a\otimes S^{b+c}} \\
			{(S^a\otimes S^b)\otimes S^c} && {S^a\otimes(S^b\otimes S^c)}
			\arrow["{\phi_{a+b,c}}"', from=1-2, to=1-1]
			\arrow["{\phi_{a,b+c}}", from=1-2, to=1-3]
			\arrow["{S^a\otimes\phi_{b,c}}", from=1-3, to=2-3]
			\arrow["{\phi_{a,b}\otimes S^c}"', from=1-1, to=2-1]
			\arrow["\cong", from=2-1, to=2-3]
		\end{tikzcd}\]
	\end{enumerate}
\end{definition}

Furthermore, Dugger gaurantees that we can always find such a coherent family:

\begin{theorem}[{\cite[Proposition 7.1]{Dugger_2014}}]\label{coherent_existence}
	There exists a coherent family of isomorphisms
	\[\phi_{a,b}:S^{a+b}\xrightarrow\cong S^a\otimes S^b\]
	in the sense of \autoref{coherent_isos},
	and in particular, the set of such coherent families is in bijective correspondence with the set of normalized $2$-cocycles $Z^2(A;\mathrm{Aut}(S))_\mathit{norm}$, i.e., the set of functions $\alpha:A\times A\to\mathrm{Aut}(S)$ such that $\alpha(0,0)=\id_S$ and for all $a,b,c\in A$, $\alpha(a+b,c)\cdot\alpha(a,b)=\alpha(b,c)\cdot\alpha(a,b+c)$. 
\end{theorem}

Thus, from now on we will suppose once and for all we have fixed a coherent family $\{\phi_{a,b}\}_{a,b\in A}$. Such a coherent family has very nice properties, in particular:

\begin{remark}\label{unique_comp_Sas}
	Note that by induction the coherence conditions say that given any $a_1,\ldots,a_n\in A$ and $b_1,\ldots,b_m\in A$ such that $a_1+\cdots+a_n=b_1+\cdots+b_m$ and any fixed parenthesizations of $X=S^{a_1}\otimes\cdots\otimes S^{a_b}$ and $Y=S^{b_1}\otimes\cdots\otimes S^{b_m}$, there is a \emph{unique} isomorphism $X\to Y$ that can be obtained by forming formal compositions of tensor products of $\phi_{a,b}$, associators, and their inverses.
\end{remark}

Of course, we get our desired result: $\pi_*(E)$ is indeed an $A$-graded ring if $E$ is a monoid object.

\begin{proposition}\label{pi_*E_is_ring_for_E_monoid}
	Let $(E,\mu,e)$ be a commutative monoid object in $\cSH$, and consider the multiplication map $\pi_*(E)\times\pi_*(E)\to\pi_*(E)$ which sends classes $x:S^a\to E$ and $y:S^b\to E$ to the composition
	\[S^{a+b}\xrightarrow{\phi_{a,b}}S^a\otimes S^b\xrightarrow{x\otimes y}E\otimes E\xrightarrow\mu E.\]
	Then this endows $\pi_*(E)$ with the structure of an $A$-graded ring with unit $e\in\pi_0(E)=[S,E]$.
\end{proposition}
\begin{proof}
	See \autoref{pi_*E_is_ring_for_E_monoid_appendix}.
\end{proof}

Furthermore, it turns out that if $E$ is a \emph{commutative} monoid object in $\cSH$, then $\pi_\ast(E)$ is ``$A$-graded commutative,'' in the following sense:

\begin{proposition}
	For all $a,b\in A$ there exists an element $\theta_{a,b}\in\pi_0(S)=[S,S]$ (determined by choice of coherent family $\{\phi_{a,b}\}$) such that given any commutative monoid object $(E,\mu,e)$ in $\cSH$, the $A$-graded ring structure on $\pi_\ast(E)$ (\autoref{pi_*E_is_ring_for_E_monoid}) has a commutativity formula given by
	\[x\cdot y=y\cdot x\cdot (e\circ\theta_{a,b})\]
	for all $x\in\pi_a(E)$ and $y\in\pi_b(E)$.
	
	Furthermore, $\theta_{0,a}=\theta_{a,0}=\id_S$ for all $a\in A$, so that if either $x$ or $y$ has degree $0$, $x\cdot y=y\cdot x$.
\end{proposition}
\begin{proof}
	See \autoref{pi_*(E)_is_A-graded_commutative_if_E_is_commutative} and \autoref{theta_a,0=theta_0,a=id_S}.
\end{proof}

We also have the following result:

\begin{proposition}
	Given some $a\in A$, the functors $\Sigma^a$ and $\Sigma^{-a}$ canonically form an adjoint equivalence of $\cSH$.
\end{proposition}
\begin{proof}
	See \autoref{Sigma^a,Sigma^-a_adjoint_equiv}.
\end{proof}

In particular, note that this tells us that given objects $E$ and $X$ in $\cSH$, we have isomorphisms
\[E^\ast(X)=[X,S^*\otimes X]\cong[S^{-\ast}\otimes X,E]\cong[S^{-*},F(X,E)]=\pi_{-*}(F(X,E)).\]
Similarly, given any objects $X$ and $Y$ in $\cSH$, we have isomorphisms of $A$-graded abelian groups
\[[X,\Sigma Y]_*=[S^*\otimes X,S^\1\otimes Y]\cong[S^{-\1}\otimes S^\ast\otimes X,Y]\cong[S^{*-\1}\otimes X,Y]=[X,Y]_{*-\1},\]
where the first isomorphism is the adjunction specified by the above proposition, and the second isomorphism is induced by the isomorphism
\[S^{*-\1}\otimes X\xrightarrow{\phi_{-\1,\ast}\otimes X}S^{-\1}\otimes S^*\otimes X.\]

The last ingredient in order to develop the Adams spectral sequence abstractly is a notion of \emph{cellularity} in $\cSH$:

\begin{definition}\label{cellular}
	Define the class of \emph{cellular} objects in $\cSH$ to be the smallest class of objects such that:
	\begin{enumerate}
		\item For all $a\in A$, $S^a$ is cellular.
		\item If we have a distinguished triangle
		\[X\to Y\to Z\to\Sigma X(=S^\1\otimes X)\]
		such that two of the three objects $X$, $Y$, and $Z$ are cellular, than the third object is also cellular.
		\item Given a collection of cellular objects $X_i$ indexed by some small set $I$, $\bigoplus_{i\in I} X_i$ is cellular.
	\end{enumerate}
\end{definition}

\subsection{Monoid objects in \texorpdfstring{$\cSH$}{TEXT}}

We have constructed an Adams spectral sequence, but as it currently stands we do not yet know why it is useful. To start with, we'd like to provide a characterization of its $E_1$ and $E_2$ pages in terms of something more algebraic. To start, we first need to develop some theory of the algebra of monoid objects in $\cSH$. Much of this work is entirely straightforward although tedious to verify, so we relegate most of the proofs in this section to \Cref{monoid_objects}.

\begin{proposition}
	Let $(E,\mu,e)$ be a monoid object in $\cSH$. Then for any object $X$ in $\cSH$, $E_*(X)$ canonically inherits the structure of a left $A$-graded module over $\pi_*(E)$ (which recall is an $A$-graded ring by \autoref{pi_*E_is_ring_for_E_monoid}) via the map
	\[\pi_*(E)\times E_*(X)\to E_*(X)\]
	which given $a,b\in A$, sends $x:S^a\to E$ and $y:S^b\to E$ to the composition
	\[x\cdot y:S^{a+b}\cong S^a\otimes S^b\xrightarrow{x\otimes y}E\otimes (E\otimes X)\cong(E\otimes E)\otimes X\xrightarrow{\mu\otimes X}E\otimes X.\]
	Similarly, $X_*(E)$ canonically inherits the structure of a right graded $\pi_*(E)$-module via the map
	\[X_*(E)\times\pi_*(E)\to X_*(E)\]
	which given $a,b\in A$, sends $x:S^a\to X\otimes E$ and $y:S^b\to E$ to the composition
	\[x\cdot y:S^{a+b}\cong S^a\otimes S^b\xrightarrow{x\otimes y}(X\otimes E)\otimes E\cong X\otimes(E\otimes E)\xrightarrow{X\otimes\mu}X\otimes E.\]
\end{proposition}
\begin{proof}
	See \autoref{module}.
\end{proof}

\begin{definition}\label{flat}
	Given a monoid object $E$ in $\cSH$, we say $E$ is \emph{flat} if the canonical right $\pi_*(E)$-module structure on $E_*(E)$ (see the above proposition) is that of a flat module.
\end{definition}

\subsection{Construction of the Adams spectral sequence}

In what follows, let $E$ be a commutative monoid object in $\cSH$.

\begin{definition}\label{mASS}
	Let $\ol E$ be the fiber of the unit map $e:S\to E$ (\autoref{fiber}), and for $s\geq0$ define
	\[Y_s:=\ol E^s\otimes Y,\qquad W_s = E\otimes Y_s=E\otimes(\ol E^s\otimes Y),\]
	where recall for $s>0$, $\ol E^s$ denotes the $s$-fold product parenthesized as $\ol E\otimes(\ol E\otimes\cdots(\ol E\otimes\ol E))$, and $\ol E^0:=S$. Then we get fiber sequences
	\[Y_{s+1}\xrightarrow{i_s}Y_s\xrightarrow{j_s} W_s\xrightarrow{k_s}\Sigma Y_{s+1}(=S^\1\otimes Y_{s+1})\]
	obtained by applying $-\otimes Y_s$ to the sequence
	\[\ol E\to S\xrightarrow eE\to\Sigma\ol E\]
	(and applying the necessary associator isomorphisms). These sequences can be spliced together to form the \emph{(canonical) Adams filtration of $Y$}:
	\[\begin{tikzcd}
		\cdots & {Y_3} & {Y_2} & {Y_1} & {Y_0=Y} \\
		& {W_3} & {W_2} & {W_1} & {W_0}
		\arrow[from=1-1, to=1-2]
		\arrow["{i_2}", from=1-2, to=1-3]
		\arrow["{i_1}", from=1-3, to=1-4]
		\arrow["{i_0}", from=1-4, to=1-5]
		\arrow["{j_0}", from=1-5, to=2-5]
		\arrow["{k_0}"', dashed, from=2-5, to=1-4]
		\arrow["{j_1}", from=1-4, to=2-4]
		\arrow["{k_1}"', dashed, from=2-4, to=1-3]
		\arrow["{j_3}", from=1-2, to=2-2]
		\arrow["{j_2}", from=1-3, to=2-3]
		\arrow["{k_2}"', dashed, from=2-3, to=1-2]
	\end{tikzcd}\]
	where the diagonal dashed arrows are of degree $-\1$ (note these triangles do NOT commute in any sense). Now we may apply the functor $[X,-]_\ast$, and by \autoref{dist_tri_LES} we obtain an exact couple of $\bN\times A$-graded abelian groups:
	\[\begin{tikzcd}
		{[X,Y_\ast]_\ast} && {[X,Y_\ast]_\ast} \\
		\\
		&& {[X,W_\ast]_\ast}
		\arrow["{i_\aast}", from=1-1, to=1-3]
		\arrow["{j_\aast}", from=1-3, to=3-3]
		\arrow["{k_\aast}", from=3-3, to=1-1]
	\end{tikzcd}\]
	where $i_\aast$, $j_\aast$, and $k_\aast$ have $\bZ\times A$-degree $(-1,0)$, $(0,0)$, and $(1,-\1)$, respectively\footnote{Explicitly, the map $k_{s,a}:[X,W_s]_a\to [X,Y_{s+1}]_{a-\1}$ sends a map $f:S^a\otimes X\to W_s$ to the map $S^{a-\1}\otimes X\to Y_{s+1}$ corresponding under the isomorphism $[X,\Sigma Y_{s+1}]_\ast\cong[X,Y_{s+1}]_{\ast-\1}$ to the composition $k_s\circ f:S^a\otimes X\to\Sigma Y_{s+1}$.}. The standard argument yields a $\bN\times A$-graded spectral sequence called from this exact couple (cf.\ Section 5.9 of \cite{Weibel_1994}) with $E_1$ page given by 
	\[E_1^{s,a}=[X,W_s]_{a}\]
	and $r^\text{th}$ differential of $\bZ\times A$-degree $(r,-\1)$:
	\[d_r:E_r^{s,a}\to E_r^{s+r,a-\1}.\]
	A priori, this is all $\bN\times A$-graded, but we regard it as being $\bZ\times A$-graded by setting $E_r^{s,a}:=0$ for $s<0$ and trivially extending the definition of the differentials to these zero groups. This spectral sequence is called the \emph{$E$-Adams spectral sequence} for the computation of $[X,Y]_\ast$. The index $s$ is called the \emph{Adams filtration} and $a$ is the \emph{stem}.
\end{definition}

\subsection{The \texorpdfstring{$E_1$}{TEXT} page}

The goal of this subsection is to provide the following characterization for the $E_1$ page of the Adams spectral sequence:

\begin{theorem}
	Let $E$ be a flat commutative monoid object in $\cSH$, and let $X$ and $Y$ be two objects in $\cSH$ such that $E_\ast(X)$ is a projective module over $\pi_\ast(E)$. Then for all $s\geq0$ and $a\in A$, we have isomorphisms in the associated $E$-Adams spectral sequence
	\[E_1^{s,a}\cong\Hom_{E_\ast(E)}^{a}(E_\ast(X),E_\ast(W_s))\]
	Furthermore, under these isomorphisms, the differential $d_1:E_1^{s,a}\to E_1^{s+1,a-\1}$ corresponds to the map
	\[\Hom_{E_\ast(E)}^{a}(E_\ast(X),E_\ast(W_s))\to\Hom_{E_\ast(E)}^{a-\1}(E_\ast(X),E_\ast(X\otimes W_{s+1}))\]
	which sends a map $f:E_\ast(X)\to E_{\ast+a}(W_s)$ to the composition 
	\[E_\ast(X)\xrightarrow f E_{\ast+a}(W_s)\xrightarrow{(X\otimes h_s)_\ast}E_{\ast+a-\1}(X\otimes Y_{s+1})\xrightarrow{(X\otimes j_{s+1})_\ast}E_{\ast+a-\1}(X\otimes W_{s+1}).\]
\end{theorem}
\begin{proof}
	By \autoref{2.30_2.33}, for all $s\geq0$ and $t,w\in\bZ$, we have isomorphisms
	\[[X,E\otimes Y_s]_{t,w}\cong\Hom_{E_\ast(E)}^{t,w}(E_\ast(X),E_\ast(E\otimes Y_s)).\]
	since $W_s=E\otimes Y_s$, we have that
	\[E_1^{s,(t,w)}=[X,W_s]_{t,w}\cong\Hom_{E_\ast(E)}^{t,w}(E_\ast(X),E_\ast(W_s)),\]
	as desired. 
\end{proof}

\begin{definition}
	Let $(E,\mu,e)$ be a monoid object in $\cSH$. We say $E$ is \emph{flat} if the canonical right $\pi_\ast(E)$-module structure on $E_\ast(E)$ is that of a flat module.
\end{definition}

\subsection{The \texorpdfstring{$E_2$}{TEXT} page}

\subsection{Convergence}

\href{https://www.uio.no/studier/emner/matnat/math/MAT9580/v12/undervisningsmateriale/boardman-conditionally-1999.pdf}{convergence of spectral sequences}

\end{document}