\documentclass[../main.tex]{subfiles}

\begin{document}

\subsection{The category \texorpdfstring{$\GrCAlg{R}$}{R-GrCAlg} of \texorpdfstring{$A$}{A}-graded \texorpdfstring{$R$}{R}-commutative rings}

We will freely use the results of \Cref{graded_stuff} in this section. In what follows, fix an $A$-graded ring $R$. Further suppose that for all $a,b\in A$, there exists units $\theta_{a,b}\in R_0$ such that:\begin{itemize}
	\item For all $a\in A$, $\theta_{a,0}=\theta_{0,a}=1$,
 	\item For all $a,b\in A$, $\theta_{a,b}^{-1}=\theta_{b,a}$,
  	\item For all $a,b,c\in A$, $\theta_{a,b}\cdot\theta_{a,c}=\theta_{a,b+c}$ and $\theta_{b,a}\cdot\theta_{c,a}=\theta_{b+c,a}$. 
	\item For all $x\in R_a$ and $y\in R_b$,
	\[x\cdot y=y\cdot x\cdot\theta_{a,b}.\]
\end{itemize}

\begin{definition}\label{R-GrCAlg_defn}
	Let $\GrCAlg R$ denote the following category:\begin{itemize}
		\item The objects are pairs $(S,\varphi)$ called $A$-graded \emph{$R$-commutative} rings, where $S$ is an $A$-graded ring and $\varphi:R\to S$ is an $A$-graded ring homomorphism such that for all $x\in S_a$ and $y\in S_b$, we have
		\[x\cdot y=y\cdot x\cdot\varphi(\theta_{a,b}),\]
		\item The morphisms $(S,\varphi)\to(S',\varphi')$ are $A$-graded ring homomorphisms $f:S\to S'$ such that $f\circ\varphi=\varphi'$.
	\end{itemize}
\end{definition}

Note that our notation for the category $\GrCAlg{R}$ is somewhat deficient, as there may be multiple choices of families of units $\theta_{a,b}\in R_0$ satisfying the required properties which give rise to strictly different categories, as the following example illustrates. Despite this, for our purposes it will always be clear from context which collection of $\theta_{a,b}$'s we're working with.

\begin{example}
	Let $A=\bZ$, let $R$ be the ring $\bZ$, viewed as a $\bZ$-graded ring concentrated in degree $0$, and let $\theta_{n,m}:=(-1)^{n\cdot m}$ for all $n,m\in\bZ$. Then the category $\GrCAlg R$ is simply the category of graded anticommutative rings, i.e., $\bZ$-graded rings $S$ such that for all homogeneous $x,y\in S$, $x\cdot y=y\cdot x\cdot(-1)^{|x||y|}$. On the other hand, if we take the same $A$ and $R$, but instead we define $\theta_{n,m}=1$ for all $n,m\in\bZ$, then the category $\GrCAlg{R}$ becomes the category of strictly commutative $\bZ$-graded rings.
\end{example}

\begin{proposition}\label{B-tensor_product_in_R-GrCAlg}
	Suppose we have two morphisms $f:(B,\varphi_B)\to(C,\varphi_C)$ and $g:(B,\varphi_B)\to(D,\varphi_D)$ of $A$-graded $R$-commutative rings in $\GrCAlg{R}$. Then $f$ and $g$ make $C$ and $D$ both $B$-bimodules, respectively,\footnote{Explicitly, it is a standard fact that given a ring homomorphism $\varphi:R\to S$ that $S$ canonically becomes an $R$-bimodule with left action $r\cdot s:=\varphi(r)s$ and right action $s\cdot r:=s\varphi(r)$, so that in particular if $\varphi$ is an $A$-graded homomorphism of $A$-graded rings, then $\varphi$ makes $S$ an $A$-graded $R$-bimodule.} so we may form their tensor product $C\otimes_BD$, which is itself an $A$-graded $B$-bimodule (\autoref{tensor_of_A_graded_is_A_graded}). Then $C\otimes_BD$ canonically inherits the structure of an $A$-graded $R$-commutative ring with unit $1_C\otimes 1_D$ via a product
	\[(C\otimes_BD)\times(C\otimes_BD)\to C\otimes_BD\]
	which sends a pair $(x\otimes y,x'\otimes y')$ of homogeneous pure tensors to the element
	\[\varphi_B(\theta_{|x|,|y'|})\cdot(xx'\otimes yy')=\varphi_C(\theta_{|x|,|y'|})xx'\otimes yy',\]
	(where here $\cdot$ denotes the left module action of $B$ on $C\otimes_BD$), and with structure map
	\begin{align*}
		\varphi:R&\to C\otimes_BD \\ 
		r&\mapsto \varphi_B(r)\cdot(1_C\otimes 1_D)=(\varphi_C(r)\otimes 1_D)=(1_C\otimes\varphi_D(r)).
	\end{align*}
\end{proposition}
\begin{proof}[Proof sketch]
	We simply lay out everything that needs to be shown, and we leave it to the reader to fill in the details. First to show that the indicated product is actually well-defined and distributive, by \autoref{tensor_lift_of_A_graded_is_A_graded} it suffices to show that for all homogeneous $c,c',c''\in C$, $d,d',d''\in D$, and $b\in B$ with $|c'|=|c''|$ and $|d'|=|d''|$, that
	\begin{align*}
		\varphi_B(\theta_{|d|,|c'+c''|})\cdot(c(c'+c'')\otimes dd')&=\varphi_B(\theta_{|d|,|c'|})\cdot(cc'\otimes dd')+\varphi_B(\theta_{|d|,|c''|})\cdot (cc''\otimes dd') \\
		\varphi_B(\theta_{|d|,|c'|})\cdot (cc'\otimes d(d'+d''))&=\varphi_B(\theta_{|d|,|c'|})\cdot(cc'\otimes dd')+\varphi_B(\theta_{|d|,|c'|})\cdot(cc'\otimes dd'') \\
		\varphi_B(\theta_{|d|,|c'\cdot b|})\cdot (c(c'\cdot b)\otimes dd')&=\varphi_B(\theta_{|d|,|c'|})\cdot(cc'\otimes d(b\cdot d')) \\
		\varphi_B(\theta_{|d'|,|c|})\cdot((c'+c'')c\otimes d'd)&=\varphi_B(\theta_{|d'|,|c|})\cdot(c'c\otimes d'd)+\varphi_B(\theta_{|d'|,|c|})\cdot(c''c\otimes d'd) \\
		\varphi_B(\theta_{|d'+d''|,|c|})\cdot(c'c\otimes (d'+d'')d)&=\varphi_B(\theta_{|d'|,|c|})\cdot(c'c\otimes d'd)+\varphi_B(\theta_{|d''|,|c|})\cdot(c'c\otimes d''d) \\
		\varphi_B(\theta_{|d'|,|c|})((c'\cdot b)c\otimes d'd)&=\varphi_B(\theta_{|c|,|b\cdot d'|})\cdot (c'c\otimes (b\cdot d')d).
	\end{align*}
	These tell us that for all $x\in C\otimes_BD$ that the maps $C\otimes_BD\to C\otimes_BD$ sending $y\mapsto xy$ and $y\mapsto yx$ are well-defined $A$-graded homomorphisms of abelian groups, so we have a distributive product $(x,y)\mapsto xy$. Then to show that this product makes $C\otimes_BD$ an $A$-graded ring, by \autoref{A_graded_ring}, it suffices to show that for all homogeneous $x,y,z\in C\otimes_BD$ that $(xy)z=x(yz)$ and $x(1_C\otimes 1_D)=x=(1_C\otimes 1_D)x$. By distributivity, it further suffices to consider the case that $x$, $y$, and $z$ are homogeneous \emph{pure tensors} in $C\otimes_BD$, i.e., it suffices to show that for all homogeneous $c,c',c''\in C$ and $d,d',d''\in D$ that
	\[((c\otimes d)(c'\otimes d'))(c''\otimes d'')=(c\otimes d)((c'\otimes d')(c''\otimes d''))\]
	and
	\[(c\otimes d)(1_C\otimes 1_D)=(c\otimes d)=(1_C\otimes 1_D)(c\otimes d).\]
	Thus, we have that the given product endows $C\otimes_BD$ with the structure of an $A$-graded ring, as desired. Now, we wish to show that the given map $\varphi:R\to C\otimes_BD$ is a ring homomorphism. Clearly it sends $1$ to $1_C\otimes 1_D$, and again by linearity, it suffices to show that given \emph{homogeneous} $r,s\in R$ that
	\[\varphi(r+s)=\varphi_B(r+s)(1_C\otimes 1_D)=\varphi_B(r)(1_C\otimes 1_D)+\varphi_B(s)(1_C\otimes 1_D)=\varphi(r)+\varphi(s)\]
	and
	\[\varphi(rs)=\varphi_B(rs)(1_C\otimes 1_D)=(\varphi_B(r)(1_C\otimes 1_D))(\varphi_B(s)(1_C\otimes 1_D))=\varphi(r)\varphi(s).\]
	Finally, we need to show that $C\otimes_BD$ satisfies the graded commutativity condition, for which again by linearity it suffices to show that given homogeneous $c,c'\in C$ and $d,d'\in D$ that
	\[(c\otimes d)(c'\otimes d')=\varphi(\theta_{|c\otimes d|,|c'\otimes d'|})(c'\otimes d')(c\otimes d)=\varphi(\theta_{|c|+|d|,|c'|+|d'|})(c'\otimes d)(c\otimes d).\]
	Showing all of these is relatively straightforward.
\end{proof}

\begin{proposition}\label{R-GrCAlg_has_pushouts_and_binary_coproducts}
	The category $\GrCAlg{R}$ has pushouts, where given $f:(B,\varphi_B)\to (C,\varphi_C)$ and $g:(B,\varphi_B)\to(D,\varphi_D)$, their pushout is the object $(C\otimes_BD,\varphi)$ constructed in \autoref{B-tensor_product_in_R-GrCAlg}, along with the canonical maps $(C,\varphi_C)\to(C\otimes_BD,\varphi)$ sending $c\mapsto c\otimes1_D$ and $(D,\varphi_D)\to(C\otimes_BD,\varphi)$ sending $d\mapsto  1_C\otimes d$. In particular, since $(R,\id_R)$ is initial, $\GrCAlg{R}$ has binary coproducts.
\end{proposition}
\begin{proof}[Proof sketch]
	First, we need to show that the given maps $i_C:(C,\varphi_C)\to(C\otimes_BD,\varphi)$ and $i_D:(D,\varphi_D)\to(C\otimes_BD,\varphi)$ are actually morphisms in $\GrCAlg{R}$, i.e., that they are ring homomorphisms and that the following diagram commutes:
	% https://q.uiver.app/#q=WzAsNCxbMSwxLCJDXFxvdGltZXNfQkQiXSxbMCwxLCJDIl0sWzIsMSwiRCJdLFsxLDAsIlIiXSxbMSwwLCJpX0MiLDJdLFsyLDAsImlfRCJdLFszLDEsIlxcdmFycGhpX0MiLDJdLFszLDIsIlxcdmFycGhpX0QiXSxbMywwLCJcXHZhcnBoaSIsMV1d
	\[\begin{tikzcd}
		& R \\
		C & {C\otimes_BD} & D
		\arrow["{i_C}"', from=2-1, to=2-2]
		\arrow["{i_D}", from=2-3, to=2-2]
		\arrow["{\varphi_C}"', from=1-2, to=2-1]
		\arrow["{\varphi_D}", from=1-2, to=2-3]
		\arrow["\varphi"{description}, from=1-2, to=2-2]
	\end{tikzcd}\]
	Showing this is entirely straightforward. Furthermore, $i_C$ and $i_D$ clearly make the following diagram commute:
	% https://q.uiver.app/#q=WzAsNCxbMCwwLCJCIl0sWzEsMCwiRCJdLFsxLDEsIkNcXG90aW1lc19CRCJdLFswLDEsIkMiXSxbMCwxLCJnIl0sWzEsMiwiaV9EIl0sWzAsMywiZiIsMl0sWzMsMiwiaV9DIiwyXV0=
	\[\begin{tikzcd}
		B & D \\
		C & {C\otimes_BD}
		\arrow["g", from=1-1, to=1-2]
		\arrow["{i_D}", from=1-2, to=2-2]
		\arrow["f"', from=1-1, to=2-1]
		\arrow["{i_C}"', from=2-1, to=2-2]
	\end{tikzcd}\]
	It remains to show that $i_C$ and $i_D$ are the universal such arrows. Suppose we have some object $(E,\varphi_E)$ in $\GrCAlg{R}$ and a commuting diagram
	% https://q.uiver.app/#q=WzAsNCxbMCwwLCJCIl0sWzAsMSwiQyJdLFsxLDAsIkQiXSxbMSwxLCJFIl0sWzAsMSwiZiIsMl0sWzAsMiwiZyJdLFsxLDMsImgiXSxbMiwzLCJrIl1d
	\[\begin{tikzcd}
		B & D \\
		C & E
		\arrow["f"', from=1-1, to=2-1]
		\arrow["g", from=1-1, to=1-2]
		\arrow["h", from=2-1, to=2-2]
		\arrow["k", from=1-2, to=2-2]
	\end{tikzcd}\]
	of morphisms in $\GrCAlg{R}$. Then we'd like to show there exists a unique morphism $\ell:C\otimes_BD\to E$ in $\GrCAlg{R}$ which makes the following diagram commute:
	% https://q.uiver.app/#q=WzAsNSxbMCwwLCJCIl0sWzAsMSwiQyJdLFsxLDAsIkQiXSxbMiwyLCJFIl0sWzEsMSwiQ1xcb3RpbWVzX0JEIl0sWzAsMSwiZiIsMl0sWzAsMiwiZyJdLFsxLDMsImgiLDIseyJjdXJ2ZSI6M31dLFsyLDMsImsiLDAseyJjdXJ2ZSI6LTN9XSxbMSw0LCJpX0MiXSxbMiw0LCJpX0QiLDJdLFs0LDMsIlxcZWxsIiwxLHsic3R5bGUiOnsiYm9keSI6eyJuYW1lIjoiZGFzaGVkIn19fV1d
	\[\begin{tikzcd}
		B & D \\
		C & {C\otimes_BD} \\
		&& E
		\arrow["f"', from=1-1, to=2-1]
		\arrow["g", from=1-1, to=1-2]
		\arrow["h"', curve={height=18pt}, from=2-1, to=3-3]
		\arrow["k", curve={height=-18pt}, from=1-2, to=3-3]
		\arrow["{i_C}", from=2-1, to=2-2]
		\arrow["{i_D}"', from=1-2, to=2-2]
		\arrow["\ell"{description}, dashed, from=2-2, to=3-3]
	\end{tikzcd}\]
	First we show uniqueness. Supposing such an arrow $\ell$ existed, given elements $c\in C$ and $d\in D$, we must have
	\[\ell(c\otimes d)=\ell((c\otimes 1_D)(1_C\otimes d))=\ell(c\otimes 1_D)\ell(1_C\otimes d)=\ell(i_C(c))\ell(i_D(d))=h(c)k(d).\]
	Since pure tensors generate $C\otimes_BD$, we have uniquely determined $\ell$, and clearly it makes the above diagram commute. Now, it remains to show that as defined $\ell$ is a morphism in $\GrCAlg{R}$, i.e., that it is an $A$-graded ring homomorphism and that the following diagram commutes:
	% https://q.uiver.app/#q=WzAsMyxbMSwwLCJSIl0sWzAsMSwiQ1xcb3RpbWVzX0JEIl0sWzIsMSwiRSJdLFswLDEsIlxcdmFycGhpIiwyXSxbMSwyLCJcXGVsbCJdLFswLDIsIlxcdmFycGhpX0UiXV0=
	\[\begin{tikzcd}
		& R \\
		{C\otimes_BD} && E
		\arrow["\varphi"', from=1-2, to=2-1]
		\arrow["\ell", from=2-1, to=2-3]
		\arrow["{\varphi_E}", from=1-2, to=2-3]
	\end{tikzcd}\]
	This is all entirely straightforward to show.
\end{proof}


\subsection{\texorpdfstring{$A$}{A}-graded commutative Hopf algebroids over \texorpdfstring{$R$}{R}}

\begin{definition}
	Let $\cC$ be a category admitting pullbacks. A \emph{groupoid object} in $\cC$ consists of a pair of objects $(M,O)$ together with five morphisms
	\begin{enumerate}
		\item \emph{Source and target}: $s,t:M\to O$,
		\item \emph{Identity}: $e:O\to M$,
		\item \emph{Composition}: $c:M\times_{O}M\to M$,
		\item \emph{Inverse}: $i:M\to M$
	\end{enumerate}
	Explicitly, $M\times_OM$ fits into the following pullback diagram:
	\[\begin{tikzcd}
		{M\times_OM} & M \\
		M & O
		\arrow["{p_2}", from=1-1, to=1-2]
		\arrow["{p_1}"', from=1-1, to=2-1]
		\arrow["s"', from=2-1, to=2-2]
		\arrow["t", from=1-2, to=2-2]
		\arrow["\lrcorner"{anchor=center, pos=0.125}, draw=none, from=1-1, to=2-2]
	\end{tikzcd}\]
	so if we're working with sets, the composition map sends a pair $(g,f)$ such that the codomain of $f$ is the domain of $g$ to $g\circ f$.
	These data must satisfy the following diagrams:
	\begin{enumerate}
		\item Composition works correctly:
		\[\begin{tikzcd}
			{M\times_OM} & M & M & O & M & {M\times_OM} & M \\
			M & O && O && M & O
			\arrow["e"', from=1-4, to=1-3]
			\arrow["s"', from=1-3, to=2-4]
			\arrow["e", from=1-4, to=1-5]
			\arrow["t", from=1-5, to=2-4]
			\arrow[Rightarrow, no head, from=1-4, to=2-4]
			\arrow["c", from=1-1, to=1-2]
			\arrow["t", from=1-2, to=2-2]
			\arrow["{p_1}"', from=1-1, to=2-1]
			\arrow["t", from=2-1, to=2-2]
			\arrow["c"', from=1-6, to=2-6]
			\arrow["s", from=2-6, to=2-7]
			\arrow["{p_2}", from=1-6, to=1-7]
			\arrow["s", from=1-7, to=2-7]
		\end{tikzcd}\]
		The first diagram says that the codomain of $g\circ f$ is the codomain of $g$. The second diagram says that the domain and codomain of the identity on some object $x$ is $x$. The third diagram says that the domain of $g\circ f$ is the domain of $f$.
		\item Associativity of composition: Write $M\times_O(M\times_OM)$ and $(M\times_OM)\times_OM$ for the pullbacks of $(s,t\circ c)$ and $(s\circ c,t)$, respectively, so we have commuting diagrams
		% https://q.uiver.app/#q=WzAsMTQsWzQsMCwiTVxcdGltZXNfTyhNXFx0aW1lc19PTSkiXSxbNiwwLCJNXFx0aW1lc19PTSJdLFs1LDEsIk1cXHRpbWVzX09NIl0sWzYsMSwiTSJdLFs2LDIsIk8iXSxbNSwyLCJNIl0sWzQsMiwiTSJdLFswLDAsIihNXFx0aW1lc19PTSlcXHRpbWVzX09NIl0sWzEsMSwiTVxcdGltZXNfT00iXSxbMCwyLCJNXFx0aW1lc19PTSJdLFsxLDIsIk0iXSxbMiwyLCJPIl0sWzIsMSwiTSJdLFsyLDAsIk0iXSxbMCwxLCJwXzInJyJdLFswLDIsIk1cXHRpbWVzIGMiLDIseyJzdHlsZSI6eyJib2R5Ijp7Im5hbWUiOiJkYXNoZWQifX19XSxbMyw0LCJ0Il0sWzIsNSwicF8xIiwyXSxbNSw0LCJzIl0sWzIsMywicF8yIl0sWzEsMywiYyJdLFswLDYsInBfMScnIiwyXSxbNiw1LCIiLDAseyJsZXZlbCI6Miwic3R5bGUiOnsiaGVhZCI6eyJuYW1lIjoibm9uZSJ9fX1dLFs3LDgsImNcXHRpbWVzIE0iLDIseyJzdHlsZSI6eyJib2R5Ijp7Im5hbWUiOiJkYXNoZWQifX19XSxbNyw5LCJwXzEnIiwyXSxbOSwxMCwiYyJdLFsxMCwxMSwicyJdLFs4LDEwLCJwXzEiLDJdLFs4LDEyLCJwXzIiXSxbMTIsMTEsInQiXSxbNywxMywicF8yJyJdLFsxMywxMiwiIiwyLHsibGV2ZWwiOjIsInN0eWxlIjp7ImhlYWQiOnsibmFtZSI6Im5vbmUifX19XV0=
		\[\begin{tikzcd}
			{(M\times_OM)\times_OM} && M && {M\times_O(M\times_OM)} && {M\times_OM} \\
			& {M\times_OM} & M &&& {M\times_OM} & M \\
			{M\times_OM} & M & O && M & M & O
			\arrow["{p_2''}", from=1-5, to=1-7]
			\arrow["{M\times c}"', dashed, from=1-5, to=2-6]
			\arrow["t", from=2-7, to=3-7]
			\arrow["{p_1}"', from=2-6, to=3-6]
			\arrow["s", from=3-6, to=3-7]
			\arrow["{p_2}", from=2-6, to=2-7]
			\arrow["c", from=1-7, to=2-7]
			\arrow["{p_1''}"', from=1-5, to=3-5]
			\arrow[Rightarrow, no head, from=3-5, to=3-6]
			\arrow["{c\times M}"', dashed, from=1-1, to=2-2]
			\arrow["{p_1'}"', from=1-1, to=3-1]
			\arrow["c", from=3-1, to=3-2]
			\arrow["s", from=3-2, to=3-3]
			\arrow["{p_1}"', from=2-2, to=3-2]
			\arrow["{p_2}", from=2-2, to=2-3]
			\arrow["t", from=2-3, to=3-3]
			\arrow["{p_2'}", from=1-1, to=1-3]
			\arrow[Rightarrow, no head, from=1-3, to=2-3]
		\end{tikzcd}\]
		where the inner and outer squares in both diagrams are pullback squares. Furthermore, assuming the diagrams in condition (1) above are satisfied, we have that $t\circ p_1\circ p_2''=t\circ c\circ p_2''=s\circ p_1''$, so that by the universal property of the pullback we have a map $M\times p_1:M\times_O(M\times_OM)\to M\times_OM$ like so:
		% https://q.uiver.app/#q=WzAsNSxbMCwwLCJNXFx0aW1lc19PKE1cXHRpbWVzX09NKSJdLFsxLDEsIk1cXHRpbWVzX09NIl0sWzIsMSwiTSJdLFsyLDIsIk8iXSxbMSwyLCJNIl0sWzAsMSwiTVxcdGltZXMgcF8xIiwxLHsic3R5bGUiOnsiYm9keSI6eyJuYW1lIjoiZGFzaGVkIn19fV0sWzEsMiwicF8yIl0sWzIsMywidCJdLFsxLDQsInBfMSIsMl0sWzQsMywicyJdLFswLDQsInBfMScnIiwyLHsiY3VydmUiOjR9XSxbMCwyLCJwXzFcXGNpcmMgcF8yJyciLDAseyJjdXJ2ZSI6LTR9XV0=
		\[\begin{tikzcd}
			{M\times_O(M\times_OM)} \\
			& {M\times_OM} & M \\
			& M & O
			\arrow["{M\times p_1}"{description}, dashed, from=1-1, to=2-2]
			\arrow["{p_2}", from=2-2, to=2-3]
			\arrow["t", from=2-3, to=3-3]
			\arrow["{p_1}"', from=2-2, to=3-2]
			\arrow["s", from=3-2, to=3-3]
			\arrow["{p_1''}"', curve={height=24pt}, from=1-1, to=3-2]
			\arrow["{p_1\circ p_2''}", curve={height=-24pt}, from=1-1, to=2-3]
		\end{tikzcd}\]
		Now note that again assuming composition works correctly, so $s\circ c=s\circ p_2$, we have 
		\[s\circ c\circ (M\times p_1)=s\circ p_2\circ(M\times p_1)=s\circ p_1\circ p_2''=t\circ p_2\circ p_2'',\] 
		so that by the unviersal property of the pullback we get a map $a:M\times_O(M\times_OM)\to (M\times_OM)\times_OM$ like so:
		% https://q.uiver.app/#q=WzAsNyxbMCwwLCJNXFx0aW1lc19PKE1cXHRpbWVzX09NKSJdLFsxLDEsIihNXFx0aW1lc19PTSlcXHRpbWVzX09NIl0sWzEsMywiTVxcdGltZXNfT00iXSxbMiwzLCJNIl0sWzMsMywiTyJdLFszLDIsIk0iXSxbMywxLCJNIl0sWzEsMiwicF8xJyIsMl0sWzIsMywiYyJdLFswLDEsImEiLDAseyJzdHlsZSI6eyJib2R5Ijp7Im5hbWUiOiJkYXNoZWQifX19XSxbMCwyLCJNXFx0aW1lcyBwXzEiLDIseyJjdXJ2ZSI6NH1dLFszLDQsInMiXSxbNSw0LCJ0Il0sWzEsNiwicF8yJyJdLFs2LDUsIiIsMCx7ImxldmVsIjoyLCJzdHlsZSI6eyJoZWFkIjp7Im5hbWUiOiJub25lIn19fV0sWzAsNiwicF8yXFxjaXJjIHBfMicnIiwwLHsiY3VydmUiOi00fV1d
		\[\begin{tikzcd}
			{M\times_O(M\times_OM)} \\
			& {(M\times_OM)\times_OM} && M \\
			&&& M \\
			& {M\times_OM} & M & O
			\arrow["{p_1'}"', from=2-2, to=4-2]
			\arrow["c", from=4-2, to=4-3]
			\arrow["a", dashed, from=1-1, to=2-2]
			\arrow["{M\times p_1}"', curve={height=24pt}, from=1-1, to=4-2]
			\arrow["s", from=4-3, to=4-4]
			\arrow["t", from=3-4, to=4-4]
			\arrow["{p_2'}", from=2-2, to=2-4]
			\arrow[Rightarrow, no head, from=2-4, to=3-4]
			\arrow["{p_2\circ p_2''}", curve={height=-24pt}, from=1-1, to=2-4]
		\end{tikzcd}\]
		Then we require that the following diagram commutes:
		% https://q.uiver.app/#q=WzAsNSxbMCwwLCJNXFx0aW1lc19PKE1cXHRpbWVzX09NKSJdLFsyLDAsIihNXFx0aW1lc19PTSlcXHRpbWVzX09NIl0sWzAsMSwiTVxcdGltZXNfT00iXSxbMSwxLCJNIl0sWzIsMSwiTVxcdGltZXNfT00iXSxbMCwxLCJhIl0sWzAsMiwiTVxcdGltZXMgYyIsMl0sWzIsMywiYyJdLFsxLDQsImNcXHRpbWVzIE0iXSxbNCwzLCJjIiwyXV0=
		\[\begin{tikzcd}
			{M\times_O(M\times_OM)} && {(M\times_OM)\times_OM} \\
			{M\times_OM} & M & {M\times_OM}
			\arrow["a", from=1-1, to=1-3]
			\arrow["{M\times c}"', from=1-1, to=2-1]
			\arrow["c", from=2-1, to=2-2]
			\arrow["{c\times M}", from=1-3, to=2-3]
			\arrow["c"', from=2-3, to=2-2]
		\end{tikzcd}\]
		This diagram says $h\circ(g\circ f)=(h\circ g)\circ f$.
		\item Unitality of composition: Given the maps $(\id_M,e\circ t),(e\circ s,\id_M):M\to M\times_OM$ defined by the universal property of $M\times_OM$:
		\[\begin{tikzcd}
			M &&& M \\
			& {M\times_OM} & M && {M\times_OM} & M \\
			& M & O && M & O
			\arrow["{(\id_M,e\circ s)}", dashed, from=1-1, to=2-2]
			\arrow["{p_2}", from=2-2, to=2-3]
			\arrow["t", from=2-3, to=3-3]
			\arrow["{p_1}"', from=2-2, to=3-2]
			\arrow["s", from=3-2, to=3-3]
			\arrow["{e\circ s}", curve={height=-18pt}, from=1-1, to=2-3]
			\arrow[curve={height=18pt}, Rightarrow, no head, from=1-1, to=3-2]
			\arrow["{(e\circ t,\id_M)}", dashed, from=1-4, to=2-5]
			\arrow["{p_2}", from=2-5, to=2-6]
			\arrow["t", from=2-6, to=3-6]
			\arrow["{p_1}"', from=2-5, to=3-5]
			\arrow["s", from=3-5, to=3-6]
			\arrow[curve={height=-18pt}, Rightarrow, no head, from=1-4, to=2-6]
			\arrow["{e\circ t}"', curve={height=18pt}, from=1-4, to=3-5]
			\arrow["\lrcorner"{anchor=center, pos=0.125}, draw=none, from=2-2, to=3-3]
			\arrow["\lrcorner"{anchor=center, pos=0.125}, draw=none, from=2-5, to=3-6]
		\end{tikzcd}\]
		the following diagram commutes:
		% https://q.uiver.app/#q=WzAsNCxbMCwwLCJNIl0sWzIsMiwiTSJdLFsyLDAsIk1cXHRpbWVzX09NIl0sWzAsMiwiTVxcdGltZXNfT00iXSxbMCwxLCIiLDAseyJsZXZlbCI6Miwic3R5bGUiOnsiaGVhZCI6eyJuYW1lIjoibm9uZSJ9fX1dLFswLDIsIihlXFxjaXJjIHQsXFxpZF9NKSJdLFsyLDEsImMiXSxbMCwzLCIoXFxpZF9NLGVcXGNpcmMgcykiLDJdLFszLDEsImMiLDJdXQ==
		\[\begin{tikzcd}
			M && {M\times_OM} \\
			\\
			{M\times_OM} && M
			\arrow[Rightarrow, no head, from=1-1, to=3-3]
			\arrow["{(e\circ t,\id_M)}", from=1-1, to=1-3]
			\arrow["c", from=1-3, to=3-3]
			\arrow["{(\id_M,e\circ s)}"', from=1-1, to=3-1]
			\arrow["c"', from=3-1, to=3-3]
		\end{tikzcd}\]
		This diagram says that given $f:x\to y$ in $M$, that $f\circ\id_x=f$ and $\id_y\circ f=f$.
		\item Inverse: The following diagrams must commute:
		\[\begin{tikzcd}
			& M & M & {M\times_OM} & M && M \\
			M & M & O & M & O & O & M & O
			\arrow["{(\id_M,i)}", from=1-3, to=1-4]
			\arrow["t"', from=1-3, to=2-3]
			\arrow["e", from=2-3, to=2-4]
			\arrow["c", from=1-4, to=2-4]
			\arrow["{(i,\id_M)}"', from=1-5, to=1-4]
			\arrow["s", from=1-5, to=2-5]
			\arrow["e"', from=2-5, to=2-4]
			\arrow["i", from=1-2, to=2-2]
			\arrow["i", from=1-7, to=2-7]
			\arrow["s"', from=1-7, to=2-6]
			\arrow["t"', from=2-7, to=2-6]
			\arrow["t", from=1-7, to=2-8]
			\arrow["s", from=2-7, to=2-8]
			\arrow["i"', from=2-2, to=2-1]
			\arrow[Rightarrow, no head, from=1-2, to=2-1]
		\end{tikzcd}\]
		where the arrows $(\id_M,i)$ and $(i,\id_M)$ are determined by the universal property of $M\times_OM$ like so:
		\[\begin{tikzcd}
			M &&& M \\
			& {M\times_OM} & M && {M\times_OM} & M \\
			& M & O && M & O
			\arrow["{(i,\id_M)}", dashed, from=1-4, to=2-5]
			\arrow["{p_2}", from=2-5, to=2-6]
			\arrow["t", from=2-6, to=3-6]
			\arrow["{p_1}"', from=2-5, to=3-5]
			\arrow["s", from=3-5, to=3-6]
			\arrow[curve={height=-18pt}, Rightarrow, no head, from=1-4, to=2-6]
			\arrow["i"', curve={height=18pt}, from=1-4, to=3-5]
			\arrow["{(\id_M,i)}", dashed, from=1-1, to=2-2]
			\arrow["{p_2}", from=2-2, to=2-3]
			\arrow["t", from=2-3, to=3-3]
			\arrow["{p_1}"', from=2-2, to=3-2]
			\arrow["s", from=3-2, to=3-3]
			\arrow["i", curve={height=-18pt}, from=1-1, to=2-3]
			\arrow[curve={height=18pt}, Rightarrow, no head, from=1-1, to=3-2]
		\end{tikzcd}\]
		Given $f:x\to y$ in $M$, the first diagram says that ${(f^{-1})}^{-1}=f$. The second says that $f\circ f^{-1}=\id_y$ and $f^{-1}\circ f=\id_x$. The last diagram says that the domain and codomain of $f^{-1}$ are $y$ and $x$, respectively.
	\end{enumerate}
\end{definition}

\begin{definition}\label{hopf_algebroid_defn}
	An \emph{$A$-graded commutative Hopf algebroid over $R$} is a co-groupoid object in $\GrCAlg{R}$, i.e., a groupoid object in $\GrCAlg{R}^\op$. Explicitly, an $A$-graded commutative Hopf algebroid over $E$ is a pair $(B,\Gamma)$ of objects in $\GrCAlg{R}$ along with morphisms
	\begin{enumerate}
		\item \emph{left unit}: $\eta_L:B\to\Gamma$ (corresponding to $t$),
		\item \emph{right unit}: $\eta_R:B\to\Gamma$ (corresponding to $s$),
		\item \emph{comultiplication}: $\Psi:\Gamma\to\Gamma\otimes_B\Gamma$ (corresponding to $c$),
		\item \emph{counit}: $\vare:\Gamma\to B$ (corresponding to $e$),
		\item \emph{conjugation}: $c:\Gamma\to\Gamma$ (corresponding to $i$),
	\end{enumerate}
	where here $\Gamma$ may be viewed as a $B$-bimodule with left $B$-module structure induced by $\eta_L$ and right $B$-module structure induced by $\eta_R$, so we may form the tensor product of bimodules $\Gamma\otimes_B\Gamma$, which further may be given the structure of an $A$-graded $R$-commutative ring (by \autoref{B-tensor_product_in_R-GrCAlg}), and fits into the following pushout diagram in $\GrCAlg{R}$ (\autoref{R-GrCAlg_has_pushouts_and_binary_coproducts}):
	% https://q.uiver.app/#q=WzAsNCxbMCwwLCJCIl0sWzAsMSwiXFxHYW1tYSJdLFsxLDEsIlxcR2FtbWFcXG90aW1lc19CXFxHYW1tYSJdLFsxLDAsIlxcR2FtbWEiXSxbMCwxLCJcXGV0YV9SIiwyXSxbMSwyLCJnXFxtYXBzdG8gZ1xcb3RpbWVzIDEiLDJdLFswLDMsIlxcZXRhX0wiXSxbMywyLCJnXFxtYXBzdG8gMVxcb3RpbWVzIGciXV0=
	\[\begin{tikzcd}
		B & \Gamma \\
		\Gamma & {\Gamma\otimes_B\Gamma}
		\arrow["{\eta_R}"', from=1-1, to=2-1]
		\arrow["{g\mapsto g\otimes 1}"', from=2-1, to=2-2]
		\arrow["{\eta_L}", from=1-1, to=1-2]
		\arrow["{g\mapsto 1\otimes g}", from=1-2, to=2-2]
	\end{tikzcd}\]
	These data must satisfy the following
	\begin{enumerate}
		\item The following diagrams must commute:
		% https://q.uiver.app/#q=WzAsMTIsWzgsMSwiXFxHYW1tYVxcb3RpbWVzX0JcXEdhbW1hIl0sWzcsMCwiQiJdLFs4LDAsIlxcR2FtbWEiXSxbNywxLCJcXEdhbW1hIl0sWzQsMSwiQiJdLFs0LDAsIkIiXSxbNSwxLCJcXEdhbW1hIl0sWzMsMSwiXFxHYW1tYSJdLFswLDAsIkIiXSxbMCwxLCJcXEdhbW1hIl0sWzEsMSwiXFxHYW1tYVxcb3RpbWVzX0JcXEdhbW1hIl0sWzEsMCwiXFxHYW1tYSJdLFs0LDUsIiIsMCx7ImxldmVsIjoyLCJzdHlsZSI6eyJoZWFkIjp7Im5hbWUiOiJub25lIn19fV0sWzgsOSwiXFxldGFfTCIsMl0sWzksMTAsImdcXG1hcHN0byBnXFxvdGltZXMgMSIsMl0sWzgsMTEsIlxcZXRhX0wiXSxbMTEsMTAsIlxcUHNpIl0sWzUsNywiXFxldGFfUiIsMl0sWzcsNCwiXFx2YXJlIiwyXSxbNSw2LCJcXGV0YV9MIl0sWzYsNCwiXFx2YXJlIl0sWzEsMiwiXFxldGFfUiJdLFsyLDAsImdcXG1hcHN0bzFcXG90aW1lcyBnIl0sWzEsMywiXFxldGFfUiIsMl0sWzMsMCwiXFxQc2kiXV0=
		\[\begin{tikzcd}
			B & \Gamma &&& B &&& B & \Gamma \\
			\Gamma & {\Gamma\otimes_B\Gamma} && \Gamma & B & \Gamma && \Gamma & {\Gamma\otimes_B\Gamma}
			\arrow[Rightarrow, no head, from=2-5, to=1-5]
			\arrow["{\eta_L}"', from=1-1, to=2-1]
			\arrow["{g\mapsto g\otimes 1}"', from=2-1, to=2-2]
			\arrow["{\eta_L}", from=1-1, to=1-2]
			\arrow["\Psi", from=1-2, to=2-2]
			\arrow["{\eta_R}"', from=1-5, to=2-4]
			\arrow["\vare"', from=2-4, to=2-5]
			\arrow["{\eta_L}", from=1-5, to=2-6]
			\arrow["\vare", from=2-6, to=2-5]
			\arrow["{\eta_R}", from=1-8, to=1-9]
			\arrow["{g\mapsto1\otimes g}", from=1-9, to=2-9]
			\arrow["{\eta_R}"', from=1-8, to=2-8]
			\arrow["\Psi", from=2-8, to=2-9]
		\end{tikzcd}\]
		\item (Coassociativity) The following diagram must commute
		% https://q.uiver.app/#q=WzAsNSxbMiwwLCJcXEdhbW1hXFxvdGltZXNfQlxcR2FtbWEiXSxbMSwwLCJcXEdhbW1hIl0sWzAsMCwiXFxHYW1tYVxcb3RpbWVzX0JcXEdhbW1hIl0sWzAsMSwiKFxcR2FtbWFcXG90aW1lc19CXFxHYW1tYSlcXG90aW1lc19CXFxHYW1tYSJdLFsyLDEsIlxcR2FtbWFcXG90aW1lc19CKFxcR2FtbWFcXG90aW1lc19CXFxHYW1tYSkiXSxbMSwyLCJcXFBzaSIsMl0sWzIsMywiXFxQc2lcXG90aW1lc19CXFxHYW1tYSIsMl0sWzAsNCwiXFxHYW1tYVxcb3RpbWVzX0JcXFBzaSJdLFsxLDAsIlxcUHNpIl0sWzMsNF1d
		\[\begin{tikzcd}
			{\Gamma\otimes_B\Gamma} & \Gamma & {\Gamma\otimes_B\Gamma} \\
			{(\Gamma\otimes_B\Gamma)\otimes_B\Gamma} && {\Gamma\otimes_B(\Gamma\otimes_B\Gamma)}
			\arrow["\Psi"', from=1-2, to=1-1]
			\arrow["{\Psi\otimes_B\Gamma}"', from=1-1, to=2-1]
			\arrow["{\Gamma\otimes_B\Psi}", from=1-3, to=2-3]
			\arrow["\Psi", from=1-2, to=1-3]
			\arrow[from=2-1, to=2-3]
		\end{tikzcd}\]
		where the bottom arrow sends $(g\otimes g')\otimes g''$ to $g\otimes(g'\otimes g'')$ and $\Psi\otimes\Gamma$ and $\Gamma\otimes\Psi$ fit into the following commutative diagrams, where both outer and inner squares in both diagrams are pushout squares in $\GrCAlg{R}$:
		% https://q.uiver.app/#q=WzAsMTQsWzIsMiwiXFxHYW1tYVxcb3RpbWVzX0IoXFxHYW1tYVxcb3RpbWVzX0JcXEdhbW1hKSJdLFsyLDAsIlxcR2FtbWFcXG90aW1lc19CXFxHYW1tYSJdLFsxLDAsIlxcR2FtbWEiXSxbMCwwLCJCIl0sWzAsMiwiXFxHYW1tYSJdLFswLDEsIlxcR2FtbWEiXSxbMSwxLCJcXEdhbW1hXFxvdGltZXNfQlxcR2FtbWEiXSxbNCwwLCJCIl0sWzUsMCwiXFxHYW1tYSJdLFs2LDAsIlxcR2FtbWEiXSxbNiwyLCIoXFxHYW1tYVxcb3RpbWVzX0JcXEdhbW1hKVxcb3RpbWVzX0JcXEdhbW1hIl0sWzQsMSwiXFxHYW1tYSAiXSxbNCwyLCJcXEdhbW1hXFxvdGltZXNfQlxcR2FtbWEiXSxbNSwxLCJcXEdhbW1hXFxvdGltZXNfQlxcR2FtbWEiXSxbMSwwLCJ4XFxtYXBzdG8gMVxcb3RpbWVzIHgiXSxbMiwxLCJcXFBzaSJdLFszLDIsIlxcZXRhX0wiXSxbNCwwLCJnXFxtYXBzdG8gZ1xcb3RpbWVzIDEiLDJdLFs1LDQsIiIsMCx7ImxldmVsIjoyLCJzdHlsZSI6eyJoZWFkIjp7Im5hbWUiOiJub25lIn19fV0sWzMsNSwiXFxldGFfUiIsMl0sWzIsNiwiZ1xcbWFwc3RvMVxcb3RpbWVzIGciXSxbNSw2LCJnXFxtYXBzdG8gZ1xcb3RpbWVzIDEiLDJdLFs2LDAsIlxcR2FtbWFcXG90aW1lc1xcUHNpIiwxLHsic3R5bGUiOnsiYm9keSI6eyJuYW1lIjoiZGFzaGVkIn19fV0sWzcsOCwiXFxldGFfTCJdLFs4LDksIiIsMCx7ImxldmVsIjoyLCJzdHlsZSI6eyJoZWFkIjp7Im5hbWUiOiJub25lIn19fV0sWzksMTAsImdcXG1hcHN0byAxXFxvdGltZXMgZyJdLFs3LDExLCJcXGV0YV9SIiwyXSxbMTEsMTIsIlxcUHNpIiwyXSxbMTIsMTAsInhcXG1hcHN0byB4XFxvdGltZXMgMSIsMl0sWzExLDEzLCJnXFxtYXBzdG8gZ1xcb3RpbWVzIDEiLDJdLFs4LDEzLCJnXFxtYXBzdG8xXFxvdGltZXMgZyJdLFsxMywxMCwiXFxQc2lcXG90aW1lc1xcR2FtbWEiLDEseyJzdHlsZSI6eyJib2R5Ijp7Im5hbWUiOiJkYXNoZWQifX19XV0=
		\[\begin{tikzcd}
			B & \Gamma & {\Gamma\otimes_B\Gamma} && B & \Gamma & \Gamma \\
			\Gamma & {\Gamma\otimes_B\Gamma} &&& {\Gamma } & {\Gamma\otimes_B\Gamma} \\
			\Gamma && {\Gamma\otimes_B(\Gamma\otimes_B\Gamma)} && {\Gamma\otimes_B\Gamma} && {(\Gamma\otimes_B\Gamma)\otimes_B\Gamma}
			\arrow["{x\mapsto 1\otimes x}", from=1-3, to=3-3]
			\arrow["\Psi", from=1-2, to=1-3]
			\arrow["{\eta_L}", from=1-1, to=1-2]
			\arrow["{g\mapsto g\otimes 1}"', from=3-1, to=3-3]
			\arrow[Rightarrow, no head, from=2-1, to=3-1]
			\arrow["{\eta_R}"', from=1-1, to=2-1]
			\arrow["{g\mapsto1\otimes g}", from=1-2, to=2-2]
			\arrow["{g\mapsto g\otimes 1}"', from=2-1, to=2-2]
			\arrow["\Gamma\otimes\Psi"{description}, dashed, from=2-2, to=3-3]
			\arrow["{\eta_L}", from=1-5, to=1-6]
			\arrow[Rightarrow, no head, from=1-6, to=1-7]
			\arrow["{g\mapsto 1\otimes g}", from=1-7, to=3-7]
			\arrow["{\eta_R}"', from=1-5, to=2-5]
			\arrow["\Psi"', from=2-5, to=3-5]
			\arrow["{x\mapsto x\otimes 1}"', from=3-5, to=3-7]
			\arrow["{g\mapsto g\otimes 1}"', from=2-5, to=2-6]
			\arrow["{g\mapsto1\otimes g}", from=1-6, to=2-6]
			\arrow["\Psi\otimes\Gamma"{description}, dashed, from=2-6, to=3-7]
		\end{tikzcd}\]
		\item The following diagram must commute:
		% https://q.uiver.app/#q=WzAsNCxbMiwyLCJcXEdhbW1hIl0sWzAsMCwiXFxHYW1tYSJdLFsyLDAsIlxcR2FtbWFcXG90aW1lc19CXFxHYW1tYSJdLFswLDIsIlxcR2FtbWFcXG90aW1lc19CXFxHYW1tYSJdLFswLDEsIiIsMCx7ImxldmVsIjoyLCJzdHlsZSI6eyJoZWFkIjp7Im5hbWUiOiJub25lIn19fV0sWzIsMCwiKFxcZXRhX0xcXGNpcmNcXHZhcmUpXFxjZG90XFxpZF9cXEdhbW1hIl0sWzEsMiwiXFxQc2kiXSxbMywwLCJcXGlkX1xcR2FtbWFcXGNkb3QoXFxldGFfUlxcY2lyY1xcdmFyZSkiLDJdLFsxLDMsIlxcUHNpIiwyXV0=
		\[\begin{tikzcd}
			\Gamma && {\Gamma\otimes_B\Gamma} \\
			\\
			{\Gamma\otimes_B\Gamma} && \Gamma
			\arrow[Rightarrow, no head, from=3-3, to=1-1]
			\arrow["{(\eta_L\circ\vare)\cdot\id_\Gamma}", from=1-3, to=3-3]
			\arrow["\Psi", from=1-1, to=1-3]
			\arrow["{\id_\Gamma\cdot(\eta_R\circ\vare)}"', from=3-1, to=3-3]
			\arrow["\Psi"', from=1-1, to=3-1]
		\end{tikzcd}\]
		where the right vertical arrow sends $g\otimes g'$ to $\eta_L(\vare(g))g'$ and the bottom horizontal arrow sends $g\otimes g'$ to $g\eta_R(\vare(g'))$.
		\item The following diagrams must commute:
		% https://q.uiver.app/#q=WzAsMTMsWzEsMCwiXFxHYW1tYSJdLFsxLDEsIlxcR2FtbWEiXSxbMCwxLCJcXEdhbW1hIl0sWzMsMSwiXFxHYW1tYSJdLFs0LDEsIlxcR2FtbWFcXG90aW1lc19CXFxHYW1tYSJdLFs0LDAsIlxcR2FtbWEiXSxbMywwLCJCIl0sWzUsMCwiQiJdLFs1LDEsIlxcR2FtbWEiXSxbOCwxLCJcXEdhbW1hIl0sWzgsMCwiXFxHYW1tYSJdLFs3LDAsIkIiXSxbOSwwLCJCIl0sWzAsMSwiYyJdLFsxLDIsImMiXSxbMCwyLCIiLDIseyJsZXZlbCI6Miwic3R5bGUiOnsiaGVhZCI6eyJuYW1lIjoibm9uZSJ9fX1dLFs0LDMsIlxcaWRfXFxHYW1tYVxcY2RvdCBjIl0sWzUsNiwiXFx2YXJlIiwyXSxbNiwzLCJcXGV0YV9MIiwyXSxbNSw0LCJpIiwyXSxbNSw3LCJcXHZhcmUiXSxbNyw4LCJcXGV0YV9SIl0sWzQsOCwiY1xcY2RvdFxcaWRfXFxHYW1tYSIsMl0sWzExLDEwLCJcXGV0YV9MIl0sWzEyLDEwLCJcXGV0YV9SIiwyXSxbMTAsOSwiYyIsMl0sWzExLDksIlxcZXRhX1IiLDJdLFsxMiw5LCJcXGV0YV9MIl1d
		\[\begin{tikzcd}
			& \Gamma && B & \Gamma & B && B & \Gamma & B \\
			\Gamma & \Gamma && \Gamma & {\Gamma\otimes_B\Gamma} & \Gamma &&& \Gamma
			\arrow["c", from=1-2, to=2-2]
			\arrow["c", from=2-2, to=2-1]
			\arrow[Rightarrow, no head, from=1-2, to=2-1]
			\arrow["{\id_\Gamma\cdot c}", from=2-5, to=2-4]
			\arrow["\vare"', from=1-5, to=1-4]
			\arrow["{\eta_L}"', from=1-4, to=2-4]
			\arrow["i"', from=1-5, to=2-5]
			\arrow["\vare", from=1-5, to=1-6]
			\arrow["{\eta_R}", from=1-6, to=2-6]
			\arrow["{c\cdot\id_\Gamma}"', from=2-5, to=2-6]
			\arrow["{\eta_L}", from=1-8, to=1-9]
			\arrow["{\eta_R}"', from=1-10, to=1-9]
			\arrow["c"', from=1-9, to=2-9]
			\arrow["{\eta_R}"', from=1-8, to=2-9]
			\arrow["{\eta_L}", from=1-10, to=2-9]
		\end{tikzcd}\]
		where the bottom left arrow in the middle diagram sends $g\otimes g'$ to $gc(g')$ and the bottom right arrow in the middle diagram sends $g\otimes g'$ to $c(g)g'$.
	\end{enumerate}
\end{definition}

\begin{proposition}\label{G_ox_B_G_has_one_interpretation_as_B-bimodule}
	Suppose we have an $A$-graded commutative Hopf algebroid $(B,\Gamma)$ over $R$ with structure maps $\eta_L$, $\eta_R$, $\Psi$, $\vare$, and $c$. Recall in \autoref{hopf_algebroid_defn} we considered $\Gamma\otimes_B\Gamma$ to be the $A$-graded $R$-commutative ring whose underlying abelian group was given by the tensor product of $B$-bimodules, where $\Gamma$ has left $B$-module structure induced by $\eta_L$ and right $B$-module structure induced by $\eta_R$. Then this left (resp.\ right) $B$-module structure on $\Gamma\otimes_B\Gamma$ coincides with that induced by the ring homomorphism $\Psi\circ\eta_L$ (resp.\ $\Psi\circ\eta_R$).
\end{proposition}
\begin{proof}
	First we show the left module structures coincide. By additivity, in order to show the module structures coincide, it suffices to show that given a homogeneous pure tensor $g\otimes g'$ in $\Gamma\otimes_B\Gamma$ and some $b\in B$ that $\Psi(\eta_L(b))\cdot(g\otimes g')=(\eta_L(b)\cdot g)\otimes g'$, where $\cdot$ on the left denotes the product in $\Gamma\otimes_B\Gamma$ and the $\cdot$ on the right denotes the product in $\Gamma$. By the axioms for a Hopf algebroid, we have that $\Psi(\eta_L(b))=\eta_L(b)\otimes 1$. Thus by how the product in $\Gamma\otimes_B\Gamma$ is defined (\autoref{B-tensor_product_in_R-GrCAlg}), we have that
	\[\Psi(\eta_L(b))\cdot(g\otimes g')=(\eta_L(b)\otimes 1)\cdot(g\otimes g')=(\varphi_\Gamma(\theta_{0,|g|})\cdot \eta_L(b)\cdot g)\otimes (g'\cdot 1)=(\eta_L(b)\cdot g)\otimes g',\]
	where $\varphi_\Gamma:R\to \Gamma$ is the structure map, and the last equality follows by the fact that $\theta_{0,|g|}=1$. An entirely analagous argument yields that the canonical right module structure on $\Gamma\otimes_B\Gamma$ coincides with that induced by $\Psi\circ\eta_R$.
\end{proof}

\begin{remark}
	By \autoref{G_ox_B_G_has_one_interpretation_as_B-bimodule}, given an $A$-graded commutative Hopf algebroid $(B,\Gamma)$ over $R$, there is no ambiguity when discussing the objects $\Gamma\otimes_B(\Gamma\otimes_B\Gamma)$ and $(\Gamma\otimes_B\Gamma)\otimes_B\Gamma$ --- they may both be considered as the threefold tensor product of the $B$-bimodule $\Gamma$ with itself. In particular, we have a canonical isomorphism of $B$-bimodules
	\[(\Gamma\otimes_B\Gamma)\otimes_B\Gamma\to\Gamma\otimes_B(\Gamma\otimes_B\Gamma)\]
	sending $(g\otimes g')\otimes g''$ to $g\otimes(g'\otimes g'')$, and this is precisely the isomorphism in the coassociativity diagram in the definition of a Hopf algebroid (\autoref{hopf_algebroid_defn}).
\end{remark}

\begin{proposition}
	Suppose we have an $A$-graded commutative Hopf algebroid $(B,\Gamma)$ over $R$ with structure maps $\eta_L$, $\eta_R$, $\Psi$, $\vare$, and $c$. Then $\eta_L$ is a homomorphism of left $B$-modules, $\eta_R$ is a homomorphism of right $B$-modules, and $\Psi$ and $\vare$ are homomorphisms of $B$-bimodules.
\end{proposition}
\begin{proof}
	Since the left (resp.\ right) $B$-module structure on $\Gamma$ is induced by $\eta_L$ (resp.\ $\eta_R$), the map $\eta_L$ (resp.\ $\eta_R$) is a homomorphism of left (resp.\ right) $B$-modules by definition.

	Next, we want to show $\Psi$ is a homomorphism of $B$-bimodules. Note that given $b,b'\in B$ and $g\in\Gamma$, since $\Psi$ is a ring homomorphism, we have that
	\[\Psi(\eta_L(b)g\eta_R(b'))=\Psi(\eta_L(b))\Psi(g)\Psi(\eta_R(b')).\]
	By \autoref{G_ox_B_G_has_one_interpretation_as_B-bimodule}, we know that $\Psi(\eta_L(b))\Psi(g)\Psi(\eta_R(b'))=b\cdot\Psi(g)\cdot b'$, where the first and second $\cdot$ denotes the left and right action of $B$ on $\Gamma\otimes_B\Gamma$, respectively.

	Lastly, we claim that $\vare:\Gamma\to B$ is a homomorphism of $B$-bimodules. We need to show that given $g\in\Gamma$ and $b,b'\in B$ that $\vare(\eta_L(b)g\eta_R(g'))=b\vare(g)b'$. This follows from the fact that $\vare\circ\eta_L=\vare\circ\eta_R=\id_B$. 
\end{proof}

\subsection{Comodules over a Hopf algebroid}

In what follows, fix an $A$-graded commutative Hopf algebroid $(B,\Gamma)$ over $R$ with structure maps $\eta_L$, $\eta_R$, $\Psi$, $\vare$, and $c$. We will always view $\Gamma$ as a $B$-bimodule, with left $B$-module structure induced by $\eta_L$, and right $B$-module structure induced by $\eta_R$.

\begin{lemma}\label{triple_gamma_tensor_iso}
	Let $N$ be an $A$-graded left $B$-module. Then we have an $A$-graded isomorphism of left $B$-modules
	\[(\Gamma\otimes_B\Gamma)\otimes_BN\xr\cong\Gamma\otimes_B(\Gamma\otimes_BN)\]
	sending a pure tensor $(g\otimes g')\otimes n$ to $g\otimes(g'\otimes n)$.
\end{lemma}
\begin{proof}
	\todo{aghh i hate this}
\end{proof}

\begin{definition}\label{left_comodule_defn}
	A \emph{left comodule over $\Gamma$} is a pair $(N,\Psi_N)$, where $N$ is a left $A$-graded $B$-module and $\Psi_N:N\to\Gamma\otimes_BN$ is an $A$-graded homomorphism of left $A$-graded $B$-modules (where here we view $\Gamma$ as a $B$-bimodule with its left module structure induced by $\eta_L$, and its right module structure induced by $\eta_R$). These data are required to make the following diagrams commute
	% https://q.uiver.app/#q=WzAsOCxbMCwwLCJOIl0sWzEsMCwiXFxHYW1tYVxcb3RpbWVzX0JOIl0sWzEsMSwiQlxcb3RpbWVzX0JOIl0sWzQsMCwiTiJdLFszLDAsIlxcR2FtbWFcXG90aW1lc19CTiJdLFszLDEsIihcXEdhbW1hXFxvdGltZXNfQlxcR2FtbWEpXFxvdGltZXNfQk4iXSxbNSwwLCJcXEdhbW1hXFxvdGltZXNfQk4iXSxbNSwxLCJcXEdhbW1hXFxvdGltZXNfQihcXEdhbW1hXFxvdGltZXNfQk4pIl0sWzAsMSwiXFxQc2lfTiJdLFsxLDIsIlxcdmFyZVxcb3RpbWVzIE4iXSxbMCwyLCJcXGNvbmciLDJdLFszLDQsIlxcUHNpX04iLDJdLFs0LDUsIlxcUHNpXFxvdGltZXMgTiIsMl0sWzMsNiwiXFxQc2lfTiJdLFs2LDcsIlxcR2FtbWFcXG90aW1lc1xcUHNpX04iXSxbNSw3LCJcXGNvbmciXV0=
	\[\begin{tikzcd}
		N & {\Gamma\otimes_BN} && {\Gamma\otimes_BN} & N & {\Gamma\otimes_BN} \\
		& {B\otimes_BN} && {(\Gamma\otimes_B\Gamma)\otimes_BN} && {\Gamma\otimes_B(\Gamma\otimes_BN)}
		\arrow["{\Psi_N}", from=1-1, to=1-2]
		\arrow["{\vare\otimes N}", from=1-2, to=2-2]
		\arrow["\cong"', from=1-1, to=2-2]
		\arrow["{\Psi_N}"', from=1-5, to=1-4]
		\arrow["{\Psi\otimes N}"', from=1-4, to=2-4]
		\arrow["{\Psi_N}", from=1-5, to=1-6]
		\arrow["{\Gamma\otimes\Psi_N}", from=1-6, to=2-6]
		\arrow["\cong", from=2-4, to=2-6]
	\end{tikzcd}\]
	The maps $\vare\otimes N$ and $\Psi\otimes N$ are well-defined by \autoref{hopf_algebroid_structure_maps_are_module_homos}, and the bottom isomorphism in the right diagram is that given in \autoref{triple_gamma_tensor_iso}.

	Given two left $A$-graded $\Gamma$-comodules $(N_1,\Psi_{N_1})$ and $(N_2,\Psi_{N_2})$, a homomorphism of left $A$-graded comodules $f:N_1\to N_2$ is an $A$-graded homomorphism of the underlying left $B$-modules such that the following diagram commutes:
	% https://q.uiver.app/#q=WzAsNCxbMCwwLCJOXzEiXSxbMSwwLCJOXzIiXSxbMSwxLCJcXEdhbW1hXFxvdGltZXNfQk5fMiJdLFswLDEsIlxcR2FtbWFcXG90aW1lc19CTl8xIl0sWzAsMSwiZiJdLFsxLDIsIlxcUHNpX3tOXzJ9Il0sWzAsMywiXFxQc2lfe05fMX0iLDJdLFszLDIsIlxcR2FtbWFcXG90aW1lcyBmIl1d
	\[\begin{tikzcd}
		{N_1} & {N_2} \\
		{\Gamma\otimes_BN_1} & {\Gamma\otimes_BN_2}
		\arrow["f", from=1-1, to=1-2]
		\arrow["{\Psi_{N_2}}", from=1-2, to=2-2]
		\arrow["{\Psi_{N_1}}"', from=1-1, to=2-1]
		\arrow["{\Gamma\otimes f}", from=2-1, to=2-2]
	\end{tikzcd}\]

	We write $\Gamma\text-\CoMod$ for the resulting category of left $A$-graded comodules over $\Gamma$. In the above definition, we required $A$-graded left $\Gamma$-comodule homomorphisms to strictly preserve the grading, but we could have instead considered left $\Gamma$-comodule homomorphisms which are of degree $d$ for some $d\in A$, or equivalently, the set of degree zero $A$-graded $\Gamma$-comodule homomorphisms from $N_1$ to the shifted comodule ${(N_2)}_{*+d}$. We denote the hom-set of degree-$d$ $A$-graded left $\Gamma$-comodule homomorphisms from $(N_1,\Psi_{N_1})$ to $(N_2,\Psi_{N_2})$ by
	\[\Hom_{\Gamma\text-\CoMod}^d(N_1,N_2)\qquad\text{or usually just}\qquad\Hom_{\Gamma}^d(N_1,N_2).\]
	In particular, we simply write $\Hom_{\Gamma\text-\CoMod}(N_1,N_2)$ or $\Hom_{\Gamma}(N_1,N_2)$ for the set of degree $0$ $A$-graded left $\Gamma$-comodule homomorphisms from $(N_1,\Psi_{N_1})$ to $(N_2,\Psi_{N_2})$.
\end{definition}

\begin{proposition}
	The category $\Gamma\text-\CoMod$ is an additive category.
\end{proposition}
\begin{proof}
	First, we show the category is $\Ab$-enriched. As a subcategory of $B\text-\Mod$, it suffices to show that hom-sets in $\Gamma\text-\CoMod$ are closed under addition and taking inverses. To that end, suppose we have two $A$-graded left $\Gamma$-comodule homomorphisms $f,g:(N_1,\Psi_{N_1})\to(N_2,\Psi_{N_2})$, then we have
	\begin{align*}
		\Psi_{N_2}\circ(f+g)&=(\Psi_{N_2}\circ f)+(\Psi_{N_2}\circ g) \\
		&=((\Gamma\otimes_Bf)\circ\Psi_{N_1})+((\Gamma\otimes_Bg)\circ\Psi_{N_1}) \\
		&=((\Gamma\otimes_Bf)+(\Gamma\otimes_Bg))\circ\Psi_{N_1} \\
		&=(\Gamma\otimes_B(f+g))\circ\Psi_{N_1},
	\end{align*}
	where the first equality follows since $\Psi_{N_2}$ is a homomorphism of modules, the second follows since $f$ and $g$ are left $\Gamma$-comodule homomorphisms, the third follows since $\Psi_{N_1}$ is a homomorphism of modules, and the last equality follows by definition of the tensor product of modules. Hence $f+g$ is indeed an $A$-graded left $\Gamma$-comodule homomorphism, as desired. Now, we also claim $-f$ is an $A$-graded left $\Gamma$-comodule homomorphism. To that end, note that
	\[\Psi_{N_2}\circ(-f)=-\Psi_{N_2}\circ f=-(\Gamma\otimes_Bf)\circ\Psi_{N_1}=(\Gamma\otimes_B(-f))\circ\Psi_{N_1},\]
	where the first equality follows since $\Psi_{N_2}$ is a module homomorphism, the second follows since $f$ is an $A$-graded left $\Gamma$-comodule homomorphism, and the third equality follows by definition of the tensor product.

	Thus, we've shown that the hom-sets in $\Gamma\text-\CoMod$ are abelian groups, and composition is clearly bilinear, so that $\Gamma\text-\CoMod$ is indeed $\Ab$-enriched.

	Now, in order to show $\Gamma\text-\CoMod$ is additive, it suffices to show that it contains a zero object and has binary coproducts. First, note that the zero left $B$-module is clearly an $A$-graded left $\Gamma$-comodule with structure map the unique map $0\to\Gamma\otimes_B0\cong 0$, and that given any other $A$-graded left $\Gamma$-comodule $(N,\Psi_N)$, the unique homomorphisms of left $B$-modules $0\to N$ and $N\to 0$ are left comodule homomorphisms.

	Now, suppose we have two $A$-graded left $\Gamma$-comodules $(N_1,\Psi_{N_1})$ and $(N_2,\Psi_{N_2})$. First, we claim their direct sum as left $B$-modules $N_1\oplus N_2$ is canonically an $A$-graded left $\Gamma$-comodule. We know that $N_1\oplus N_2$ is an $A$-graded left $B$-mod
\end{proof}

\begin{proposition}
	Suppose that $\Gamma$ is flat as a right $B$-module (with its canonical right $B$-module structure induced by $\eta_R$). Then the category $\Gamma\text-\CoMod$ is an abelian category. 
\end{proposition}

\subsection{The dual \texorpdfstring{$E$}{E}-Steenrod algebra is a Hopf algebroid}

In this subsection, we fix a monoidal closed tensor triangulated category $\cSH$ with arbitrary (small) (co)products and sub-Picard grading $(A,\1,\{S^a\},\{\phi_{a,b}\})$ (\autoref{sub_Picard_grading_defn}), and we adopt the conventions outlined in \Cref{section:tri_cat_with_sub-picard_grading}.

\begin{proposition}\label{pi_*:CMon_SH-->pi_*(S)-GrCAlg_appendix}
    The assignment $(E,\mu,e)\mapsto(\pi_*(E),\pi_*(e))$ yields a functor 
    \[\pi_*:\CMon_\cSH\to\GrCAlg{\pi_*(S)}\]
    from the category of commutative monoid objects in $\cSH$ (\autoref{Mon_C,CMon_C}) to the category of $A$-graded $\pi_*(S)$-commutative rings (\autoref{R-GrCAlg_defn}).
\end{proposition}
\begin{proof}
    By \autoref{pi_*E_is_ring_for_E_monoid}, we know that $\pi_*$ yields a homomorphism from $\CMon_\cSH$ to $A$-graded commutative rings.  Furthermore, by \autoref{pi_*(E)_is_A-graded_commutative_if_E_is_commutative}, we know that for all homogeneous $x,y\in\pi_*(E)$ that
    \[x\cdot y=y\cdot x\cdot(e\circ\theta_{|x|,|y|})=y\cdot x\cdot\pi_*(e)(\theta_{|x|,|y|}),\]
    as desired. Thus, it remains to show that $\pi_*(e):\pi_*(S)\to\pi_*(E)$ is an $A$-graded ring homomorphism for any (commutative) monoid object $(E,\mu,e)$ in $\cSH$, and that given a monoid homomorphism $f:(E_1,\mu_1,e_1)\to(E_2,\mu_2,e_2)$ in $\CMon_\cSH$, that $\pi_*(f)$ satisfies $\pi_*(f)\circ\pi_*(e_1)=\pi_*(e_2)$. The latter clearly holds, as since $f$ is a monoid homomorphism, we have $f\circ e_1=e_2$, so that 
    \[\pi_*(f)\circ\pi_*(e_1)=\pi_*(f\circ e_1)=\pi_*(e_2).\]
    To see that $\pi_*(e):\pi_*(S)\to\pi_*(E)$ is an $A$-graded ring homomorphism if $(E,\mu,e)$ is a monoid object, it suffices to show that $e:S\to E$ is a monoid homomorphism, since we already know $\pi_*$ takes monoid homomorphisms to $A$-graded ring homomorphisms. Consider the following diagrams:
    % https://q.uiver.app/#q=WzAsOCxbMCwwLCJTXFxvdGltZXMgUyJdLFsyLDAsIkVcXG90aW1lcyBFIl0sWzIsMiwiRSJdLFswLDIsIlMiXSxbNCwwLCJTIl0sWzMsMiwiUyJdLFs1LDIsIkUiXSxbMSwxLCJTXFxvdGltZXMgRSJdLFswLDEsImVcXG90aW1lcyBlIl0sWzEsMiwiXFxtdSJdLFswLDMsIlxcY29uZyIsMl0sWzMsMiwiZSJdLFs0LDUsIiIsMCx7ImxldmVsIjoyLCJzdHlsZSI6eyJoZWFkIjp7Im5hbWUiOiJub25lIn19fV0sWzUsNiwiZSJdLFs0LDYsImUiXSxbMCw3LCJTXFxvdGltZXMgZSIsMV0sWzcsMSwiZVxcb3RpbWVzIEUiLDFdLFs3LDIsIlxcY29uZyIsMV1d
    \[\begin{tikzcd}
        {S\otimes S} && {E\otimes E} && S \\
        & {S\otimes E} \\
        S && E & S && E
        \arrow["{e\otimes e}", from=1-1, to=1-3]
        \arrow["\mu", from=1-3, to=3-3]
        \arrow["\cong"', from=1-1, to=3-1]
        \arrow["e", from=3-1, to=3-3]
        \arrow[Rightarrow, no head, from=1-5, to=3-4]
        \arrow["e", from=3-4, to=3-6]
        \arrow["e", from=1-5, to=3-6]
        \arrow["{S\otimes e}"{description}, from=1-1, to=2-2]
        \arrow["{e\otimes E}"{description}, from=2-2, to=1-3]
        \arrow["\cong"{description}, from=2-2, to=3-3]
    \end{tikzcd}\]
    The right diagram commutes by definition. The top triangle in the left diagram commutes by functoriality of $-\otimes-$. The right triangle in the left diagram commutes by unitality of $\mu$. Finally, the left triangle in the left diagram commutes by naturality of the unitors. Thus, we have shown $e$ is a monoid object homomorphism, as desired.
\end{proof}

\begin{proposition}\label{structure_maps_are_monoid_homos_appendix}
    Let $(E,\mu,e)$ be a commutative monoid object in $\cSH$. Then the maps\begin{enumerate}
        \item $E\xr\cong E\otimes S\xr{E\otimes e}E\otimes E$,
        \item $E\xr\cong S\otimes E\xr{e\otimes E}E\otimes E$,
        \item $E\otimes E\xr\cong E\otimes S\otimes E\xr{E\otimes e\otimes E}E\otimes E\otimes E$,
        \item $E\otimes E\xr\mu E$, and
        \item $E\otimes E\xr{\tau_{E,E}}E\otimes E$
    \end{enumerate}
    are homomorphisms of monoid objects in $\cSH$ (where here $E\otimes E$ and $E\otimes E\otimes E$ are considered as monoid objects in $\cSH$ by \autoref{product_of_monoids_is_monoid} and \autoref{product_of_3+_monoids_no_ambiguity}, respectively), so that by \autoref{pi_*:CMon_SH-->pi_*(S)-GrCAlg_main}, under $\pi_*$ they induce morphisms in $\GrCAlg{\pi_*(S)}$:
    \begin{enumerate}
        \item $\eta_L:\pi_*(E)\to E_*(E)$,
        \item $\eta_R:\pi_*(E)\to E_*(E)$,
        \item $h:E_*(E)\to E_*(E\otimes E)$,
        \item $\vare:E_*(E)\to \pi_*(E)$, and
        \item $c:E_*(E)\to E_*(E)$.
    \end{enumerate}
\end{proposition}
\begin{proof}
	To start with, we will show $E\xr\cong E\otimes S\xr{E\otimes e}E\otimes E$ is a monoid object homomorphism. First, consider the following diagram:
    % https://q.uiver.app/#q=WzAsOSxbMCwwLCJFXzFcXG90aW1lcyBFXzIiXSxbMiwwLCJFXzFcXG90aW1lcyBFXFxvdGltZXMgRV8yXFxvdGltZXMgRSJdLFswLDUsIkVfezEsMn0iXSxbMiwzLCJFXzFcXG90aW1lcyBFXzJcXG90aW1lcyBFXFxvdGltZXMgRSJdLFsyLDUsIkVfezEsMn1cXG90aW1lcyBFIl0sWzEsMiwiRV8xXFxvdGltZXMgRV8yXFxvdGltZXMgRSJdLFsxLDEsIkVfMVxcb3RpbWVzIEVfMlxcb3RpbWVzIEUiXSxbMSw0LCJFXzFcXG90aW1lcyBFXzJcXG90aW1lcyBFIl0sWzEsMywiRV8xXFxvdGltZXMgRV8yXFxvdGltZXMgRSJdLFswLDEsIkVcXG90aW1lcyBlXFxvdGltZXMgRVxcb3RpbWVzIGUiXSxbMCwyLCJcXG11IiwyXSxbMSwzLCJFXFxvdGltZXNcXHRhdVxcb3RpbWVzIEUiXSxbMiw0LCJFXFxvdGltZXMgZSIsMl0sWzMsNCwiXFxtdVxcb3RpbWVzIFxcbXUiXSxbMSw1LCJFXFxvdGltZXNcXG11XFxvdGltZXMgRSJdLFszLDUsIkVcXG90aW1lc1xcbXVcXG90aW1lcyBFIiwyXSxbNiwxLCJFXFxvdGltZXMgZVxcb3RpbWVzIEVcXG90aW1lcyBFIiwxXSxbNiw1LCIiLDIseyJsZXZlbCI6Miwic3R5bGUiOnsiaGVhZCI6eyJuYW1lIjoibm9uZSJ9fX1dLFswLDYsIkVcXG90aW1lcyBFXFxvdGltZXMgZSIsMV0sWzcsNCwiXFxtdVxcb3RpbWVzIEUiLDJdLFszLDcsIkVcXG90aW1lcyBFXFxvdGltZXMgXFxtdSJdLFs1LDgsIiIsMSx7ImxldmVsIjoyLCJzdHlsZSI6eyJoZWFkIjp7Im5hbWUiOiJub25lIn19fV0sWzgsNywiIiwxLHsibGV2ZWwiOjIsInN0eWxlIjp7ImhlYWQiOnsibmFtZSI6Im5vbmUifX19XSxbOCwzLCJFXFxvdGltZXMgRVxcb3RpbWVzIGVcXG90aW1lcyBFIiwyXV0=
    \[\begin{tikzcd}
        {E_1\otimes E_2} && {E_1\otimes E\otimes E_2\otimes E} \\
        & {E_1\otimes E_2\otimes E} \\
        & {E_1\otimes E_2\otimes E} \\
        & {E_1\otimes E_2\otimes E} & {E_1\otimes E_2\otimes E\otimes E} \\
        & {E_1\otimes E_2\otimes E} \\
        {E_{1,2}} && {E_{1,2}\otimes E}
        \arrow["{E\otimes e\otimes E\otimes e}", from=1-1, to=1-3]
        \arrow["\mu"', from=1-1, to=6-1]
        \arrow["{E\otimes\tau\otimes E}", from=1-3, to=4-3]
        \arrow["{E\otimes e}"', from=6-1, to=6-3]
        \arrow["{\mu\otimes \mu}", from=4-3, to=6-3]
        \arrow["{E\otimes\mu\otimes E}", from=1-3, to=3-2]
        \arrow["{E\otimes\mu\otimes E}"', from=4-3, to=3-2]
        \arrow["{E\otimes e\otimes E\otimes E}"{description}, from=2-2, to=1-3]
        \arrow[Rightarrow, no head, from=2-2, to=3-2]
        \arrow["{E\otimes E\otimes e}"{description}, from=1-1, to=2-2]
        \arrow["{\mu\otimes E}"', from=5-2, to=6-3]
        \arrow["{E\otimes E\otimes \mu}", from=4-3, to=5-2]
        \arrow[Rightarrow, no head, from=3-2, to=4-2]
        \arrow[Rightarrow, no head, from=4-2, to=5-2]
        \arrow["{E\otimes E\otimes e\otimes E}"', from=4-2, to=4-3]
    \end{tikzcd}\]
    The leftmost region commutes by functoriality of $-\otimes-$. The top triangle also commutes by functoriality of $-\otimes-$. The triangle below that commutes by unitality of $\mu$. The triangle below that commutes by commutativity of $\mu$. The next two triangles below that commutes by unitality of $\mu$. Finally, the bottom right triangle commutes by functoriality of $-\otimes-$. Next, consider the following diagram:
    % https://q.uiver.app/#q=WzAsNSxbMiwwLCJTIl0sWzAsMiwiRSJdLFs0LDIsIkVcXG90aW1lcyBFIl0sWzMsMSwiU1xcb3RpbWVzIFMiXSxbMiwyLCJFXFxvdGltZXMgUyJdLFswLDEsImUiLDJdLFswLDMsIlxcY29uZyJdLFszLDIsImVcXG90aW1lcyBlIl0sWzEsNCwiXFxjb25nIl0sWzQsMiwiRVxcb3RpbWVzIGUiLDJdLFszLDQsImVcXG90aW1lcyBTIiwxXV0=
    \[\begin{tikzcd}
        && S \\
        &&& {S\otimes S} \\
        E && {E\otimes S} && {E\otimes E}
        \arrow["e"', from=1-3, to=3-1]
        \arrow["\cong", from=1-3, to=2-4]
        \arrow["{e\otimes e}", from=2-4, to=3-5]
        \arrow["\cong", from=3-1, to=3-3]
        \arrow["{E\otimes e}"', from=3-3, to=3-5]
        \arrow["{e\otimes S}"{description}, from=2-4, to=3-3]
    \end{tikzcd}\]
    The leftmost region commutes by naturality of the unitors, while the rightmost region commutes by functoriality of $-\otimes-$. Hence, we have shown $E\xr\cong E\otimes S\xr{E\otimes e}E\otimes E$ is indeed a monoid homomorphism, as desired. Showing that $E\xr\cong S\otimes E\xr{e\otimes E}E\otimes E$ is a monoid object homomorphism is entirely analagous.

	Next, we will show that $E\otimes E\xr\cong E\otimes S\otimes E\xr{E\otimes e\otimes E}$ is a monoid object homomorphism.
\end{proof}

\begin{lemma}\label{eta_L_left_module/eta_R_right_module_coincide_appendix}
    Let $(E,\mu,e)$ be a commutative monoid object in $\cSH$. Then the left (resp.\ right) $\pi_*(E)$-module structure induced on $E_*(E)$ by the ring homomorphism $\eta_L$ (resp.\ $\eta_R$) coincides with the canonical left (resp.\ right) $\pi_*(E)$-module structure on $E_*(E)$ given in \autoref{module_main}.
\end{lemma}
\begin{proof}
    What's going on here is a bit subtle, so we're going to be really explicit. In \autoref{module_main}, it was shown that $E_*(E)$ is a left $\pi_*(E)$-module via the assignment
    \[\pi_*(E)\times E_*(E)\to E_*(E)\]
    which sends homogeneous elements $r:S^a\to E$ and $x:S^b\to E\otimes E$ to the composition
    \[S^{a+b}\xr\cong S^a\otimes S^b\xr{r\otimes x}E\otimes E\otimes E\xr{\mu\otimes E}E\otimes E.\]
    We'd like to show that this is the same thing as the assignment $\pi_*(E)\times E_*(E)\to E_*(E)$ sending $(r,x)\mapsto \eta_L(r)x$, where $\eta_L(r)x$ denotes the product of $\eta_L(r)$ and $x$ taken in the ring $E_*(E)$. Explicitly, the product structure on $E_*(E)=\pi_*(E\otimes E)$ is that induced by the fact that $E\otimes E$ is a monoid object in $\cSH$ by \autoref{product_of_monoids_is_monoid}, with product
    \[E\otimes E\otimes E\otimes E\xr{E\otimes\tau\otimes E}E\otimes E\otimes E\otimes E\xr{\mu\otimes\mu}E\otimes E\]
    (note the middle two factors are swapped). It is a standard fact from algebra that given a ring homomorphism $\varphi:R\to R'$, that $R'$ is canonically a left $R$-module via the rule $(r,r')\mapsto \varphi(r)r'$, and a right $R$-module via the rule $(r',r)\mapsto r'\varphi(r)$. Thus, we can be sure that we actually have two left module actions. Furthermore, these are both clearly $A$-graded left module actions, so in order to show they're the same it suffices to show they agree on homogeneous elements (\autoref{A-graded_module}). Now, suppose we have homogeneous elements $r:S^a\to E$ in $\pi_*(E)$ and $x:S^b\to E\otimes E$ in $E_*(E)$. Then consider the following diagram, where we've passed to a symmetric strict monoidal category:
    % https://q.uiver.app/#q=WzAsMTAsWzAsMCwiU157YStifSJdLFswLDEsIlNeYVxcb3RpbWVzIFNeYiJdLFswLDIsIkVfMVxcb3RpbWVzIEVfMlxcb3RpbWVzIEVfMyJdLFs0LDIsIkVfezEsMn1cXG90aW1lcyBFXzMiXSxbMCw1LCJFXzFcXG90aW1lcyBFXFxvdGltZXMgRV8yXFxvdGltZXMgRV8zIl0sWzIsNSwiRV8xXFxvdGltZXMgRV8yXFxvdGltZXMgRVxcb3RpbWVzIEVfMyJdLFs0LDUsIkVfezEsMn1cXG90aW1lcyBFXzMiXSxbMSwzLCJFXzFcXG90aW1lcyBFXzJcXG90aW1lcyBFXzMiXSxbMiwzLCJFXzFcXG90aW1lcyBFXzJcXG90aW1lcyBFXzMiXSxbMywzLCJFXzFcXG90aW1lcyBFXzJcXG90aW1lcyBFXzMiXSxbMCwxLCJcXHBoaV97YSxifSJdLFsxLDIsInJcXG90aW1lcyB4Il0sWzIsMywiXFxtdVxcb3RpbWVzIEUiXSxbMiw0LCJFXFxvdGltZXMgZVxcb3RpbWVzIEUiLDJdLFs0LDUsIkVcXG90aW1lc1xcdGF1XFxvdGltZXMgRSIsMl0sWzUsNiwiXFxtdVxcb3RpbWVzXFxtdSIsMl0sWzMsNiwiIiwxLHsibGV2ZWwiOjIsInN0eWxlIjp7ImhlYWQiOnsibmFtZSI6Im5vbmUifX19XSxbNCw3LCJFXFxvdGltZXMgXFxtdVxcb3RpbWVzIEUiLDFdLFsyLDcsIiIsMSx7ImxldmVsIjoyLCJzdHlsZSI6eyJoZWFkIjp7Im5hbWUiOiJub25lIn19fV0sWzUsNywiRVxcb3RpbWVzIFxcbXVcXG90aW1lcyBFIl0sWzgsNSwiRVxcb3RpbWVzIEVcXG90aW1lcyBlXFxvdGltZXMgRSIsMV0sWzgsOSwiIiwxLHsibGV2ZWwiOjIsInN0eWxlIjp7ImhlYWQiOnsibmFtZSI6Im5vbmUifX19XSxbNSw5LCJFXFxvdGltZXMgRVxcb3RpbWVzIFxcbXUiLDJdLFs5LDYsIlxcbXVcXG90aW1lcyBFIiwxXSxbNyw4LCIiLDEseyJsZXZlbCI6Miwic3R5bGUiOnsiaGVhZCI6eyJuYW1lIjoibm9uZSJ9fX1dXQ==
    \[\begin{tikzcd}[column sep=small]
        {S^{a+b}} \\
        {S^a\otimes S^b} \\
        {E_1\otimes E_2\otimes E_3} &&&& {E_{1,2}\otimes E_3} \\
        & {E_1\otimes E_2\otimes E_3} & {E_1\otimes E_2\otimes E_3} & {E_1\otimes E_2\otimes E_3} \\
        \\
        {E_1\otimes E\otimes E_2\otimes E_3} && {E_1\otimes E_2\otimes E\otimes E_3} && {E_{1,2}\otimes E_3}
        \arrow["{\phi_{a,b}}", from=1-1, to=2-1]
        \arrow["{r\otimes x}", from=2-1, to=3-1]
        \arrow["{\mu\otimes E}", from=3-1, to=3-5]
        \arrow["{E\otimes e\otimes E}"', from=3-1, to=6-1]
        \arrow["{E\otimes\tau\otimes E}"', from=6-1, to=6-3]
        \arrow["\mu\otimes\mu"', from=6-3, to=6-5]
        \arrow[Rightarrow, no head, from=3-5, to=6-5]
        \arrow["{E\otimes \mu\otimes E}"{description}, from=6-1, to=4-2]
        \arrow[Rightarrow, no head, from=3-1, to=4-2]
        \arrow["{E\otimes \mu\otimes E}", from=6-3, to=4-2]
        \arrow["{E\otimes E\otimes e\otimes E}"{description}, from=4-3, to=6-3]
        \arrow[Rightarrow, no head, from=4-3, to=4-4]
        \arrow["{E\otimes E\otimes \mu}"', from=6-3, to=4-4]
        \arrow["{\mu\otimes E}"{description}, from=4-4, to=6-5]
        \arrow[Rightarrow, no head, from=4-2, to=4-3]
    \end{tikzcd}\]
    Here we've numbered the $E$'s to make it clear what's going on. The bottom composition is $\eta_L(r)x$, while the top composition is the canonical left action of $r$ on $x$ given in \autoref{module_main}. The leftmost triangle commutes by unitality of $\mu$. The triangle to the right of that commutes by commutativity of $\mu$. The triangle to the right of that commutes by unitality of $\mu$, as does the next triangle. The remaining triangle on the right commutes by functoriality of $-\otimes-$. Finally, the top region commutes by definition. Thus, we've shown that the left $\pi_*(E)$-module structure induced on $E_*(E)$ by $\eta_L$ is in fact the canonical one. 
    On the other hand, showing that the right $\pi_*(E)$-module structure induced on $E_*(E)$ by $\eta_R$ is the canonical one is entirely analagous, and we leave it as an exercise for the reader.
%
%    On the other hand, we'd like to show that the right $\pi_*(E)$-module structure induced by $\eta_R$ on $E_*(E)$ is the canonical one. By the same arguments as above, it suffices to show the action maps agree for homogeneous elements $x:S^a\to E\otimes E$ in $E_*(E)$ and $r:S^b\to E$ in $\pi_*(E)$. Indeed, given such elements, consider the following diagram, where again we've passed to a symmetric strict monoidal category:
%    % https://q.uiver.app/#q=WzAsMTAsWzAsMCwiU157YStifSJdLFswLDEsIlNeYVxcb3RpbWVzIFNeYiJdLFswLDIsIkVfMVxcb3RpbWVzIEVfMlxcb3RpbWVzIEVfMyJdLFs0LDIsIkVfMVxcb3RpbWVzIEVfezIsM30iXSxbMCw1LCJFXzFcXG90aW1lcyBFXzJcXG90aW1lcyBFXFxvdGltZXMgRV8zIl0sWzIsNSwiRV8xXFxvdGltZXMgRVxcb3RpbWVzIEVfMlxcb3RpbWVzIEVfMyJdLFs0LDUsIkVfMVxcb3RpbWVzIEVfezIsM30iXSxbMSwzLCJFXzFcXG90aW1lcyBFXzJcXG90aW1lcyBFXzMiXSxbMiwzLCJFXzFcXG90aW1lcyBFXzJcXG90aW1lcyBFXzMiXSxbMywzLCJFXzFcXG90aW1lcyBFXzJcXG90aW1lcyBFXzMiXSxbMCwxLCJcXHBoaV97YSxifSJdLFsxLDIsInhcXG90aW1lcyByIl0sWzIsMywiRVxcb3RpbWVzXFwsXFxtdSJdLFsyLDQsIkVcXG90aW1lcyBFXFxvdGltZXMgZVxcb3RpbWVzIEUiLDJdLFs0LDUsIkVcXG90aW1lcyBcXHRhdVxcb3RpbWVzIEUiLDJdLFs1LDYsIlxcbXVcXG90aW1lcyBcXG11IiwyXSxbMyw2LCIiLDEseyJsZXZlbCI6Miwic3R5bGUiOnsiaGVhZCI6eyJuYW1lIjoibm9uZSJ9fX1dLFsyLDcsIiIsMCx7ImxldmVsIjoyLCJzdHlsZSI6eyJoZWFkIjp7Im5hbWUiOiJub25lIn19fV0sWzcsOCwiIiwwLHsibGV2ZWwiOjIsInN0eWxlIjp7ImhlYWQiOnsibmFtZSI6Im5vbmUifX19XSxbOCw5LCIiLDAseyJsZXZlbCI6Miwic3R5bGUiOnsiaGVhZCI6eyJuYW1lIjoibm9uZSJ9fX1dLFs5LDYsIkVcXG90aW1lcyBcXG11IiwxXSxbNCw3LCJFXFxvdGltZXMgXFxtdVxcb3RpbWVzIEUiXSxbOCw1LCJFXFxvdGltZXMgZVxcb3RpbWVzIEVcXG90aW1lcyBFIiwxXSxbNSw5LCJcXG11XFxvdGltZXMgRVxcb3RpbWVzIEUiLDJdLFs1LDcsIkVcXG90aW1lcyBcXG11XFxvdGltZXMgRSJdXQ==
%    \[\begin{tikzcd}[column sep=small]
%        {S^{a+b}} \\
%        {S^a\otimes S^b} \\
%        {E_1\otimes E_2\otimes E_3} &&&& {E_1\otimes E_{2,3}} \\
%        & {E_1\otimes E_2\otimes E_3} & {E_1\otimes E_2\otimes E_3} & {E_1\otimes E_2\otimes E_3} \\
%        \\
%        {E_1\otimes E_2\otimes E\otimes E_3} && {E_1\otimes E\otimes E_2\otimes E_3} && {E_1\otimes E_{2,3}}
%        \arrow["{\phi_{a,b}}", from=1-1, to=2-1]
%        \arrow["{x\otimes r}", from=2-1, to=3-1]
%        \arrow["{E\otimes\,\mu}", from=3-1, to=3-5]
%        \arrow["{E\otimes E\otimes e\otimes E}"', from=3-1, to=6-1]
%        \arrow["{E\otimes \tau\otimes E}"', from=6-1, to=6-3]
%        \arrow["{\mu\otimes \mu}"', from=6-3, to=6-5]
%        \arrow[Rightarrow, no head, from=3-5, to=6-5]
%        \arrow[Rightarrow, no head, from=3-1, to=4-2]
%        \arrow[Rightarrow, no head, from=4-2, to=4-3]
%        \arrow[Rightarrow, no head, from=4-3, to=4-4]
%        \arrow["{E\otimes \mu}"{description}, from=4-4, to=6-5]
%        \arrow["{E\otimes \mu\otimes E}", from=6-1, to=4-2]
%        \arrow["{E\otimes e\otimes E\otimes E}"{description}, from=4-3, to=6-3]
%        \arrow["{\mu\otimes E\otimes E}"', from=6-3, to=4-4]
%        \arrow["{E\otimes \mu\otimes E}", from=6-3, to=4-2]
%    \end{tikzcd}\]
%    Again we've numbered the $E$'s to make it clear what's going on. The bottom composition is $x\eta_R(r)$, while the top composition the canonical right action of $r$ on $x$ given in \autoref{module_main}. The leftmost triangle commutes by unitality of $\mu$. The triangle to the right of that commutes by commutativity of $\mu$. The triangle to the right of that commutes by unitality of $\mu$, as does the next triangle. The remaining triangle on the right commutes by functoriality of $-\otimes-$. Finally, the top region commutes by definition. Thus, we've shown that the right $\pi_*(E)$-module structure induced on $E_*(E)$ by $\eta_R$ is in fact the canonical one, as desired.
\end{proof}

\begin{corollary}\label{E*E_ox_E*E_is_A-graded_pi*S-commutative_ring_appendix}
    Given a %flat (\autoref{flat}) and cellular (\autoref{cellular}) 
	commutative monoid object $(E,\mu,e)$ in $\cSH$, the domain of the %isomorphism 
	homomorphism
    \[\Phi_E:E_*(E)\otimes_{\pi_*(E)}E_*(E)\to E_*(E\otimes E)\]
    constructed in \autoref{Kunneth_map_iso_main} is canonically an $A$-graded $\pi_*(S)$-ring, and sits in the following pushout diagram in $\GrCAlg{\pi_*(S)}$:
    % https://q.uiver.app/#q=WzAsNCxbMCwwLCJcXHBpXyooRSkiXSxbMSwwLCJFXyooRSkiXSxbMCwxLCJFXyooRSkiXSxbMSwxLCJFXyooRSlcXG90aW1lc197XFxwaV8qKEUpfUVfKihFKSJdLFswLDEsIlxcZXRhX0wiXSxbMCwyLCJcXGV0YV9SIiwyXSxbMiwzLCJ4XFxtYXBzdG8geFxcb3RpbWVzIDEiLDJdLFsxLDMsInhcXG1hcHN0bzFcXG90aW1lcyB4Il1d
    \[\begin{tikzcd}
        {\pi_*(E)} & {E_*(E)} \\
        {E_*(E)} & {E_*(E)\otimes_{\pi_*(E)}E_*(E)}
        \arrow["{\eta_L}", from=1-1, to=1-2]
        \arrow["{\eta_R}"', from=1-1, to=2-1]
        \arrow["{x\mapsto x\otimes 1}"', from=2-1, to=2-2]
        \arrow["{x\mapsto1\otimes x}", from=1-2, to=2-2]
    \end{tikzcd}\]
\end{corollary}
\begin{proof}
	By \autoref{R-GrCAlg_has_pushouts_and_binary_coproducts}, we have a pushout diagram
	% https://q.uiver.app/#q=WzAsNCxbMCwwLCJcXHBpXyooRSkiXSxbMSwwLCJFXyooRSkiXSxbMCwxLCJFXyooRSkiXSxbMSwxLCJSIl0sWzAsMSwiXFxldGFfTCJdLFswLDIsIlxcZXRhX1IiLDJdLFsyLDMsInhcXG1hcHN0byB4XFxvdGltZXMgMSIsMl0sWzEsMywieFxcbWFwc3RvMVxcb3RpbWVzIHgiXV0=
	\[\begin{tikzcd}
		{\pi_*(E)} & {E_*(E)} \\
		{E_*(E)} & R
		\arrow["{\eta_L}", from=1-1, to=1-2]
		\arrow["{\eta_R}"', from=1-1, to=2-1]
		\arrow["{x\mapsto x\otimes 1}"', from=2-1, to=2-2]
		\arrow["{x\mapsto1\otimes x}", from=1-2, to=2-2]
	\end{tikzcd}\]
	where the underlying $A$-graded abelian group of $R$ is the tensor product over $\pi_*(E)$ of $E_*(E)$ considered as right $\pi_*(E)$-module via $\eta_R$ and $E_*(E)$ considered as a left $\pi_*(E)$-module via $\eta_L$. By \autoref{eta_L_left_module/eta_R_right_module_coincide_appendix}, as an $A$-graded abelian group, $R$ is precisely the domain of $\Phi_E$, as desired.
\end{proof}

\begin{lemma}\label{Phi_E_is_homo_of_A-graded_pi_*S-commutative_rings_appendix}
    Let $(E,\mu,e)$ be a commutative monoid object in $\cSH$. Then the homomorphism
    \[\Phi_{E,E}:E_*(E)\otimes_{\pi_*(E)}E_*(E)\to E_*(E\otimes E)\]
    constructed in \autoref{Kunneth_map_construction_main} is a homomorphism of $A$-graded $\pi_*(S)$-commutative rings, i.e.\ a morphism in $\GrCAlg{\pi_*(S)}$, where here $E_*(E)\otimes_{\pi_*(E)}E_*(E)$ is considered as an object in $\GrCAlg{\pi_*(S)}$ by \autoref{E*E_ox_E*E_is_A-graded_pi*S-commutative_ring}, and $E_*(E\otimes E)=\pi_*(E\otimes (E\otimes E))$ is considered as an object in $\GrCAlg{\pi_*(S)}$ by \autoref{pi_*:CMon_SH-->pi_*(S)-GrCAlg_main}, since $E\otimes(E\otimes E)$ is a monoid object in $\cSH$ by \autoref{product_of_monoids_is_monoid}.
\end{lemma}
\begin{proof}
	Consider the maps
	\[f:E\otimes E\xr{e\otimes E\otimes E}E\otimes E\otimes E\]
	and
	\[g:E\otimes E\xr{E\otimes E\otimes e}E\otimes E\otimes E.\]
	We know that the maps
	\[E\xr{e\otimes E}E\otimes E\qquad\text{and}\qquad E\xr{E\otimes e}E\otimes E\]
	are monoid homomorphisms by \autoref{(E,mu,e):eta_L,eta_R_appendix}, so that $f$ and $g$ are monoid homomorphisms by \autoref{E_ox_f,f_ox_E_are_monoid_homos}. Furthermore, by \autoref{E_ox_E_ox_E_is_uniquely_monoid}, they are monoid homomorphisms between the same monoid objects in $\cSH$. Finally, note that we have the following commutative diagram
	% https://q.uiver.app/#q=WzAsNCxbMCwwLCJFIl0sWzEsMCwiRVxcb3RpbWVzIEUiXSxbMSwxLCJFXFxvdGltZXMgRVxcb3RpbWVzIEUiXSxbMCwxLCJFXFxvdGltZXMgRSJdLFswLDEsIkVcXG90aW1lcyBlIl0sWzEsMiwiZVxcb3RpbWVzIEVcXG90aW1lcyBFIl0sWzAsMywiZVxcb3RpbWVzIEUiLDJdLFszLDIsIkVcXG90aW1lcyBFXFxvdGltZXMgZSIsMl0sWzAsMiwiZVxcb3RpbWVzIEVcXG90aW1lcyBlIiwxXV0=
	\[\begin{tikzcd}
		E & {E\otimes E} \\
		{E\otimes E} & {E\otimes E\otimes E}
		\arrow["{E\otimes e}", from=1-1, to=1-2]
		\arrow["{e\otimes E\otimes E}", from=1-2, to=2-2]
		\arrow["{e\otimes E}"', from=1-1, to=2-1]
		\arrow["{E\otimes E\otimes e}"', from=2-1, to=2-2]
		\arrow["{e\otimes E\otimes e}"{description}, from=1-1, to=2-2]
	\end{tikzcd}\]
	where the outer arrows are monoid object homomorphisms, thus, we may apply $\pi_*$, which yields the following commutative diagram in $\GrCAlg{\pi_*(S)}$ (\autoref{pi_*:CMon_SH-->pi_*(S)-GrCAlg_appendix}):
	% https://q.uiver.app/#q=WzAsNCxbMCwwLCJcXHBpXyooRSkiXSxbMSwwLCJFXyooRSkiXSxbMCwxLCJFXyooRSkiXSxbMSwxLCJFXyooRVxcb3RpbWVzIEUpIl0sWzAsMSwiXFxldGFfTCJdLFswLDIsIlxcZXRhX1IiLDJdLFsxLDMsIlxccGlfKihmKSJdLFsyLDMsIlxccGlfKihnKSIsMl1d
	\[\begin{tikzcd}
		{\pi_*(E)} & {E_*(E)} \\
		{E_*(E)} & {E_*(E\otimes E)}
		\arrow["{\eta_L}", from=1-1, to=1-2]
		\arrow["{\eta_R}"', from=1-1, to=2-1]
		\arrow["{\pi_*(f)}", from=1-2, to=2-2]
		\arrow["{\pi_*(g)}"', from=2-1, to=2-2]
	\end{tikzcd}\]
	Hence by \autoref{Phi_E_is_iso_of_A-graded_pi_*S-commutative_rings} and the universal property of the pushout, there exists some unique morphism $\ell:E_*(E)\otimes_{\pi_*(E)}E_*(E)\to E_*(E\otimes E)$ in $\GrCAlg{\pi_*(S)}$ which makes the following diagram commute:
	% https://q.uiver.app/#q=WzAsNSxbMCwwLCJcXHBpXyooRSkiXSxbMSwwLCJFXyooRSkiXSxbMCwxLCJFXyooRSkiXSxbMSwxLCJFXyooRSlcXG90aW1lc197XFxwaV8qKEUpfUVfKihFKSJdLFsyLDIsIkVfKihFXFxvdGltZXMgRSkiXSxbMCwxLCJcXGV0YV9MIl0sWzAsMiwiXFxldGFfUiIsMl0sWzIsMywieFxcbWFwc3RvIHhcXG90aW1lcyAxIl0sWzEsMywieFxcbWFwc3RvMVxcb3RpbWVzIHgiLDJdLFszLDQsIlxcZWxsIiwxLHsic3R5bGUiOnsiYm9keSI6eyJuYW1lIjoiZGFzaGVkIn19fV0sWzEsNCwiXFxwaV8qKGYpIiwwLHsiY3VydmUiOi0zfV0sWzIsNCwiXFxwaV8qKGcpIiwyLHsiY3VydmUiOjN9XV0=
	\[\begin{tikzcd}
		{\pi_*(E)} & {E_*(E)} \\
		{E_*(E)} & {E_*(E)\otimes_{\pi_*(E)}E_*(E)} \\
		&& {E_*(E\otimes E)}
		\arrow["{\eta_L}", from=1-1, to=1-2]
		\arrow["{\eta_R}"', from=1-1, to=2-1]
		\arrow["{x\mapsto x\otimes 1}", from=2-1, to=2-2]
		\arrow["{x\mapsto1\otimes x}"', from=1-2, to=2-2]
		\arrow["\ell"{description}, dashed, from=2-2, to=3-3]
		\arrow["{\pi_*(f)}", curve={height=-18pt}, from=1-2, to=3-3]
		\arrow["{\pi_*(g)}"', curve={height=18pt}, from=2-1, to=3-3]
	\end{tikzcd}\]
	Thus in order to show $\Phi_E$ is a morphism in $\GrCAlg{\pi_*(S)}$, it suffices to show that $\Phi_E$ and $\ell$ are the same map, since we know $\ell$ is a homomorphism of $A$-graded $\pi_*(S)$-commutative rings. Since $\Phi_E$ and $\ell$ are both abelian group homomorphisms, it further suffices to show they agree on homogeneous pure tensors, which generate $E_*(E)\otimes_{\pi_*(E)}E_*(E)$. Given homogeneous elements $x:S^a\to E\otimes E$ and $y:S^b\to E\otimes E$ in $E_*(E)$, unravelling how pushouts in $\GrCAlg{\pi_*(S)}$ are defined (\autoref{R-GrCAlg_has_pushouts_and_binary_coproducts}), $\ell$ sends the pure homogeneous tensor $x\otimes y$ to the element $\pi_*(g)(x)\cdot\pi_*(f)(y)$, where here $\cdot$ denotes the product taken in $E_*(E\otimes E)=\pi_*(E\otimes E\otimes E)$. Now, consider the following diagram:
	% https://q.uiver.app/#q=WzAsOSxbMCwwLCJTXnthK2J9Il0sWzAsMSwiU157YX1cXG90aW1lcyBTXmIiXSxbMCwyLCJFXzFcXG90aW1lcyBFXzJcXG90aW1lcyBFXzNcXG90aW1lcyBFXzQiXSxbMCw1LCJFXzFcXG90aW1lcyBFX3syLDN9XFxvdGltZXMgRV80Il0sWzIsMiwiRV8xXFxvdGltZXMgRV8yXFxvdGltZXMgRV9hXFxvdGltZXMgRV9iXFxvdGltZXMgRV8zXFxvdGltZXMgRV80Il0sWzIsMywiRV8xXFxvdGltZXMgRV9iXFxvdGltZXMgRV8yXFxvdGltZXMgRV9hXFxvdGltZXMgRV8zXFxvdGltZXMgRV80Il0sWzIsNCwiRV8xXFxvdGltZXMgRV8yXFxvdGltZXMgRV8zXFxvdGltZXMgRV9hXFxvdGltZXMgRV80Il0sWzIsNSwiRV8xXFxvdGltZXMgRV97MiwzfVxcb3RpbWVzIEVfNCJdLFsxLDQsIkVfMVxcb3RpbWVzIEVfMlxcb3RpbWVzIEVfM1xcb3RpbWVzIEVfNCJdLFswLDEsIlxccGhpX3thLGJ9Il0sWzEsMiwieFxcb3RpbWVzIHkiXSxbMiwzLCJFXFxvdGltZXMgXFxtdVxcb3RpbWVzIEUiLDJdLFsyLDQsImdcXG90aW1lcyBmPUVcXG90aW1lcyBFXFxvdGltZXMgZVxcb3RpbWVzIGVcXG90aW1lcyBFXFxvdGltZXMgRSJdLFs0LDUsIkVcXG90aW1lcyBcXHRhdV97RVxcb3RpbWVzIEUsRX1cXG90aW1lcyBFXFxvdGltZXMgRSJdLFs1LDYsIlxcbXVcXG90aW1lcyBFXFxvdGltZXMgXFx0YXVcXG90aW1lcyBFIl0sWzMsNywiIiwwLHsibGV2ZWwiOjIsInN0eWxlIjp7ImhlYWQiOnsibmFtZSI6Im5vbmUifX19XSxbNiw3LCJFXFxvdGltZXMgXFxtdVxcb3RpbWVzIFxcbXUiXSxbMiw1LCJFXFxvdGltZXMgZVxcb3RpbWVzIEVcXG90aW1lcyBlXFxvdGltZXMgRVxcb3RpbWVzIEUiLDFdLFsyLDgsIiIsMix7ImxldmVsIjoyLCJzdHlsZSI6eyJoZWFkIjp7Im5hbWUiOiJub25lIn19fV0sWzgsNiwiRVxcb3RpbWVzIEVcXG90aW1lcyBFXFxvdGltZXMgZVxcb3RpbWVzIEUiLDJdLFs4LDUsIkVcXG90aW1lcyBlXFxvdGltZXMgRVxcb3RpbWVzIGVcXG90aW1lcyBFXFxvdGltZXMgRSIsMV0sWzgsMywiRVxcb3RpbWVzXFxtdVxcb3RpbWVzIEUiLDFdXQ==
	\[\begin{tikzcd}
		{S^{a+b}} \\
		{S^{a}\otimes S^b} \\
		{E_1\otimes E_2\otimes E_3\otimes E_4} && {E_1\otimes E_2\otimes E_a\otimes E_b\otimes E_3\otimes E_4} \\
		&& {E_1\otimes E_b\otimes E_2\otimes E_a\otimes E_3\otimes E_4} \\
		& {E_1\otimes E_2\otimes E_3\otimes E_4} & {E_1\otimes E_2\otimes E_3\otimes E_a\otimes E_4} \\
		{E_1\otimes E_{2,3}\otimes E_4} && {E_1\otimes E_{2,3}\otimes E_4}
		\arrow["{\phi_{a,b}}", from=1-1, to=2-1]
		\arrow["{x\otimes y}", from=2-1, to=3-1]
		\arrow["{E\otimes \mu\otimes E}"', from=3-1, to=6-1]
		\arrow["{g\otimes f=E\otimes E\otimes e\otimes e\otimes E\otimes E}", from=3-1, to=3-3]
		\arrow["{E\otimes \tau_{E\otimes E,E}\otimes E\otimes E}", from=3-3, to=4-3]
		\arrow["{\mu\otimes E\otimes \tau\otimes E}", from=4-3, to=5-3]
		\arrow[Rightarrow, no head, from=6-1, to=6-3]
		\arrow["{E\otimes \mu\otimes \mu}", from=5-3, to=6-3]
		\arrow["{E\otimes e\otimes E\otimes e\otimes E\otimes E}"{description}, from=3-1, to=4-3]
		\arrow[Rightarrow, no head, from=3-1, to=5-2]
		\arrow["{E\otimes E\otimes E\otimes e\otimes E}"', from=5-2, to=5-3]
		\arrow["{E\otimes e\otimes E\otimes e\otimes E\otimes E}"{description}, from=5-2, to=4-3]
		\arrow["{E\otimes\mu\otimes E}"{description}, from=5-2, to=6-1]
	\end{tikzcd}\]
	Here we have labelled the $E$'s to make things clearer. The left outside composition is $\Phi_E(x\otimes y)$, while the right composition is $\pi_*(g)(x)\cdot\pi_*(f)(y)$. The top right triangle commutes by coherence for a symmetric monoidal category. The middle tright triangle commutes by unitality of $\mu$ and coherence for a symmetric monoidal category. The bottom trapezoid commutes by unitality of $\mu$. The rest of the diagram commutes by definition. Thus we have $\Phi_E(x\otimes y)=\pi_*(g)(x)\cdot\pi_*(f)(y)$, so that $\Phi_E=\ell$ is not just an isomorphism of left $\pi_*(E)$-modules, but an isomorphism of $A$-graded $\pi_*(S)$-commutative rings, as desired.
\end{proof}

\begin{proposition}\label{(E,mu,e):Psi_defn_appendix}
    Let $(E,\mu,e)$ be a flat (\autoref{flat}) and cellular (\autoref{cellular}) commutative monoid object in $\cSH$. Then consider the map
    \[\Psi:E_*(E)\xr{\pi_*(E\otimes e\otimes E)}E_*(E\otimes E)\xr{\Phi_E^{-1}}E_*(E)\otimes_{\pi_*(E)}E_*(E),\]
    where $\Phi_E$ is the isomorphism given in \autoref{Kunneth_map_iso_main}. Then $\Psi$ is a homomorphism of $A$-graded $\pi_*(S)$-commutative rings, where here the object $E_*(E)\otimes_{\pi_*(E)}E_*(E)$ is considered an $A$-graded $\pi_*(S)$-commutative ring by \autoref{E*E_ox_E*E_is_A-graded_pi*S-commutative_ring_appendix}.
\end{proposition}
\begin{proof}
	By \autoref{Phi_E_is_homo_of_A-graded_pi_*S-commutative_rings}, we know $\Phi_E:E_*(E)\otimes_{\pi_*(E)}E_*(E)\to E_*(E\otimes E)$ is a bijective homomorphism of $A$-graded $\pi_*(S)$-commutative rings, thus it is clearly an isomorphism in $\GrCAlg{\pi_*(S)}$, so that its inverse $\Phi_E^{-1}$ is also a homomorphism in $\GrCAlg{\pi_*(S)}$. Thus, it suffices to show that $\pi_*(E\otimes e\otimes E)$ is as well. By \autoref{pi_*:CMon_SH-->pi_*(S)-GrCAlg_appendix}, it suffices to show $E\otimes e\otimes E:E\otimes E\to E\otimes E\otimes E$ is a homomorphism of monoid objects in $\cSH$. Yet, we know this is true, as $e\otimes E:E\to E\otimes E$ is a homomorphism of monoid objects by \autoref{(E,mu,e):eta_L,eta_R_appendix}, so that by \autoref{E_ox_f,f_ox_E_are_monoid_homos} we have $E\otimes e\otimes E$ is also a homomorphism of monoid objects, as desired.
\end{proof}

\begin{proposition}\label{(E,mu,e):vare_appendix}
    Let $(E,\mu,e)$ be a commutative monoid object in $\cSH$. Then the morphism 
	\[\mu:E\otimes E\to E\] 
	is a homomorphism of monoid objects (where $E\otimes E$ is considered a monoid object by \autoref{product_of_monoids_is_monoid}), so that by \autoref{pi_*:CMon_SH-->pi_*(S)-GrCAlg_appendix}, under $\pi_*$ it induces a homomorphism of $A$-graded $\pi_*(S)$-commutative rings 
    \[\vare:E_*(E)\to\pi_*(E).\]
\end{proposition}
\begin{proof}
	Consider the following diagram:
	% https://q.uiver.app/#q=WzAsMTAsWzAsMCwiRVxcb3RpbWVzIEVcXG90aW1lcyBFXFxvdGltZXMgRSJdLFszLDAsIkVcXG90aW1lcyBFIl0sWzMsNCwiRSJdLFswLDIsIkVcXG90aW1lcyBFXFxvdGltZXMgRVxcb3RpbWVzIEUiXSxbMCw0LCJFXFxvdGltZXMgRSJdLFsxLDIsIkVcXG90aW1lcyBFXFxvdGltZXMgRSJdLFsyLDIsIkVcXG90aW1lcyBFIl0sWzIsMywiRVxcb3RpbWVzIEUiXSxbMiwxLCJFXFxvdGltZXMgRVxcb3RpbWVzIEUiXSxbMSwzLCJFXFxvdGltZXMgRVxcb3RpbWVzIEUiXSxbMCwxLCJcXG11XFxvdGltZXNcXG11Il0sWzEsMiwiXFxtdSJdLFswLDMsIkVcXG90aW1lc1xcdGF1XFxvdGltZXMgRSIsMl0sWzMsNCwiXFxtdVxcb3RpbWVzXFxtdSIsMl0sWzQsMiwiXFxtdSIsMl0sWzAsNSwiRVxcb3RpbWVzXFxtdVxcb3RpbWVzIEUiLDFdLFszLDUsIkVcXG90aW1lc1xcbXVcXG90aW1lcyBFIiwyXSxbNSw2LCJcXG11XFxvdGltZXMgRSJdLFs1LDcsIkVcXG90aW1lc1xcbXUiLDFdLFs3LDIsIlxcbXUiLDJdLFs2LDIsIlxcbXUiXSxbMCw4LCJcXG11XFxvdGltZXMgRVxcb3RpbWVzIEUiLDFdLFs4LDEsIkVcXG90aW1lc1xcbXUiLDFdLFs4LDYsIlxcbXVcXG90aW1lcyBFIiwxXSxbMyw5LCJFXFxvdGltZXMgRVxcb3RpbWVzIFxcbXUiLDFdLFs5LDcsIkVcXG90aW1lc1xcbXUiLDJdLFs5LDQsIlxcbXVcXG90aW1lcyBFIiwxXV0=
	\[\begin{tikzcd}
		{E\otimes E\otimes E\otimes E} &&& {E\otimes E} \\
		&& {E\otimes E\otimes E} \\
		{E\otimes E\otimes E\otimes E} & {E\otimes E\otimes E} & {E\otimes E} \\
		& {E\otimes E\otimes E} & {E\otimes E} \\
		{E\otimes E} &&& E
		\arrow["\mu\otimes\mu", from=1-1, to=1-4]
		\arrow["\mu", from=1-4, to=5-4]
		\arrow["{E\otimes\tau\otimes E}"', from=1-1, to=3-1]
		\arrow["\mu\otimes\mu"', from=3-1, to=5-1]
		\arrow["\mu"', from=5-1, to=5-4]
		\arrow["{E\otimes\mu\otimes E}"{description}, from=1-1, to=3-2]
		\arrow["{E\otimes\mu\otimes E}"', from=3-1, to=3-2]
		\arrow["{\mu\otimes E}", from=3-2, to=3-3]
		\arrow["E\otimes\mu"{description}, from=3-2, to=4-3]
		\arrow["\mu"', from=4-3, to=5-4]
		\arrow["\mu", from=3-3, to=5-4]
		\arrow["{\mu\otimes E\otimes E}"{description}, from=1-1, to=2-3]
		\arrow["E\otimes\mu"{description}, from=2-3, to=1-4]
		\arrow["{\mu\otimes E}"{description}, from=2-3, to=3-3]
		\arrow["{E\otimes E\otimes \mu}"{description}, from=3-1, to=4-2]
		\arrow["E\otimes\mu"', from=4-2, to=4-3]
		\arrow["{\mu\otimes E}"{description}, from=4-2, to=5-1]
	\end{tikzcd}\]
	The top left triangle commutes by commutativity of $\mu$. Every other region commutes by functoriality of $-\otimes-$ and/or associativity of $\mu$. Thus, we have shown $\mu$ satisfies the first diagram in \autoref{Mon_C,CMon_C} required for it to be a monoid homomorphism. To see it satisfies the second condition, consider the following diagram:
	% https://q.uiver.app/#q=WzAsNCxbMiwwLCJTIl0sWzAsMiwiRVxcb3RpbWVzIEUiXSxbNCwyLCJFIl0sWzIsMSwiRSJdLFswLDEsImVcXG90aW1lcyBlIiwyXSxbMSwyLCJcXG11Il0sWzAsMiwiZSJdLFswLDMsImUiXSxbMywxLCJFXFxvdGltZXMgZSIsMV0sWzMsMiwiIiwxLHsibGV2ZWwiOjIsInN0eWxlIjp7ImhlYWQiOnsibmFtZSI6Im5vbmUifX19XV0=
	\[\begin{tikzcd}
		&& S \\
		&& E \\
		{E\otimes E} &&&& E
		\arrow["{e\otimes e}"', from=1-3, to=3-1]
		\arrow["\mu", from=3-1, to=3-5]
		\arrow["e", from=1-3, to=3-5]
		\arrow["e", from=1-3, to=2-3]
		\arrow["{E\otimes e}"{description}, from=2-3, to=3-1]
		\arrow[Rightarrow, no head, from=2-3, to=3-5]
	\end{tikzcd}\]
	The top left region commutes by functoriality of $-\otimes-$. The top right region commutes by definition. Finally, the bottom region commutes by unitality of $\mu$. Thus we have shown $\mu$ is a monoid object homomorphism, as desired.
\end{proof}

\begin{proposition}\label{(E,mu,e):c_appendix}
    Let $(E,\mu,e)$ be a commutative monoid object in $\cSH$. Then the morphism
    \[\tau_{E,E}:E\otimes E\to E\otimes E\]
    is a homomorphism of monoid objects (where $E\otimes E$ is considered a monoid object by \autoref{product_of_monoids_is_monoid}), so that by \autoref{pi_*:CMon_SH-->pi_*(S)-GrCAlg_appendix}, under $\pi_*$ it induces a homomorphism of $A$-graded $\pi_*(S)$-commutative rings
    \[c:E_*(E)\to E_*(E).\]
\end{proposition}
\begin{proof}
	Consider the following diagram:
	% https://q.uiver.app/#q=WzAsNixbMCwwLCJFXzFcXG90aW1lcyBFXzJcXG90aW1lcyBFXzNcXG90aW1lcyBFXzQiXSxbMSwwLCJFXzJcXG90aW1lcyBFXzFcXG90aW1lcyBFXzRcXG90aW1lcyBFXzMiXSxbMCwxLCJFXzFcXG90aW1lcyBFXzNcXG90aW1lcyBFXzJcXG90aW1lcyBFXzQiXSxbMCwyLCJFX3sxLDN9XFxvdGltZXMgRV97Miw0fSJdLFsxLDIsIkVfezIsNH1cXG90aW1lcyBFX3sxLDN9Il0sWzEsMSwiRV8yXFxvdGltZXMgRV80XFxvdGltZXMgRV8xXFxvdGltZXMgRV8zIl0sWzAsMSwiXFx0YXVcXG90aW1lcyBcXHRhdSJdLFswLDIsIkVcXG90aW1lcyBcXHRhdVxcb3RpbWVzIEUiLDJdLFsyLDMsIlxcbXVcXG90aW1lcyBcXG11IiwyXSxbMyw0LCJcXHRhdSIsMl0sWzEsNSwiRVxcb3RpbWVzIFxcdGF1XFxvdGltZXMgRSJdLFs1LDQsIlxcbXVcXG90aW1lcyBcXG11Il0sWzIsNSwiXFx0YXVfe0VcXG90aW1lcyBFLEVcXG90aW1lcyBFfSJdXQ==
	\[\begin{tikzcd}
		{E_1\otimes E_2\otimes E_3\otimes E_4} & {E_2\otimes E_1\otimes E_4\otimes E_3} \\
		{E_1\otimes E_3\otimes E_2\otimes E_4} & {E_2\otimes E_4\otimes E_1\otimes E_3} \\
		{E_{1,3}\otimes E_{2,4}} & {E_{2,4}\otimes E_{1,3}}
		\arrow["{\tau\otimes \tau}", from=1-1, to=1-2]
		\arrow["{E\otimes \tau\otimes E}"', from=1-1, to=2-1]
		\arrow["{\mu\otimes \mu}"', from=2-1, to=3-1]
		\arrow["\tau"', from=3-1, to=3-2]
		\arrow["{E\otimes \tau\otimes E}", from=1-2, to=2-2]
		\arrow["{\mu\otimes \mu}", from=2-2, to=3-2]
		\arrow["{\tau_{E\otimes E,E\otimes E}}", from=2-1, to=2-2]
	\end{tikzcd}\]
	The top region commutes by coherence for the symmetries in a symmetric monoidal category, while the bottom region commutes by naturality of $\tau$. Now, consider the following diagram:
	% https://q.uiver.app/#q=WzAsNSxbMiwwLCJTIl0sWzAsMiwiRVxcb3RpbWVzIEUiXSxbNCwyLCJFXFxvdGltZXMgRSJdLFsxLDEsIlNcXG90aW1lcyBTIl0sWzMsMSwiU1xcb3RpbWVzIFMiXSxbMSwyLCJcXHRhdSJdLFszLDQsIlxcdGF1Il0sWzAsMywiXFxjb25nIiwyXSxbMCw0LCJcXGNvbmciXSxbMywxLCJlXFxvdGltZXMgZSIsMV0sWzQsMiwiZVxcb3RpbWVzIGUiLDFdXQ==
	\[\begin{tikzcd}
		&& S \\
		& {S\otimes S} && {S\otimes S} \\
		{E\otimes E} &&&& {E\otimes E}
		\arrow["\tau", from=3-1, to=3-5]
		\arrow["\tau", from=2-2, to=2-4]
		\arrow["\cong"', from=1-3, to=2-2]
		\arrow["\cong", from=1-3, to=2-4]
		\arrow["{e\otimes e}"{description}, from=2-2, to=3-1]
		\arrow["{e\otimes e}"{description}, from=2-4, to=3-5]
	\end{tikzcd}\]
	The top triangle commutes by coherence for a symmetric monoidal category, while the bottom region commutes by naturality of $\tau$. Thus, we have shown $\tau_{E,E}$ is a homomorphism of monoid objects, as desired.
\end{proof}

\begin{proposition}\label{dual_E-Steenrod_algebra_is_a_Hopf_algebroid_appendix}
    Let $(E,\mu,e)$ be a commutative monoid object in $\cSH$ which is flat (\autoref{flat}) and cellular (\autoref{cellular}). Then the dual $E$-Steenrod algebra $(E_*(E),\pi_*(E))$ with the structure maps $(\eta_L,\eta_R,\Psi,\vare,c)$ constructed above is an $A$-graded commutative Hopf algebroid over $\pi_*(S)$ (\autoref{hopf_algebroid_defn}), i.e., a co-groupoid object in the category $\GrCAlg{\pi_*(S)}$.
\end{proposition}
\begin{proof}
	We need to show all the diagrams in \autoref{hopf_algebroid_defn} commute. Since we are dealing with $A$-graded homomorphisms, when showing these diagrams commute, it always suffices to chase homogeneous elements around. To that end, we fix homogeneous elements $x:S^a\to E$ in $\pi_*(E)$ and $y:S^b\to E\otimes E$ in $E_*(E\otimes E)$ now.

	First, we wish to show the outside of the following diagram commutes:
	% https://q.uiver.app/#q=WzAsNSxbMCwwLCJcXHBpXyooRSkiXSxbMiwwLCJFXyooRSkiXSxbMCwyLCJFXyooRSkiXSxbMiwyLCJFXyooRSlcXG90aW1lc197XFxwaV8qKEUpfUVfKihFKSJdLFsxLDEsIkVfKihFXFxvdGltZXMgRSkiXSxbMCwxLCJcXGV0YV9SIl0sWzAsMiwiXFxldGFfUiIsMl0sWzIsMywieFxcbWFwc3RvMVxcb3RpbWVzIHgiLDJdLFsxLDQsIlxccGlfKihFXFxvdGltZXMgZVxcb3RpbWVzIEUpIiwxXSxbMyw0LCJcXFBoaV97RSxFfSIsMV0sWzEsMywiXFxQc2kiXV0=
	\[\begin{tikzcd}
		{\pi_*(E)} && {E_*(E)} \\
		& {E_*(E\otimes E)} \\
		{E_*(E)} && {E_*(E)\otimes_{\pi_*(E)}E_*(E)}
		\arrow["{\eta_R}", from=1-1, to=1-3]
		\arrow["{\eta_R}"', from=1-1, to=3-1]
		\arrow["{x\mapsto1\otimes x}"', from=3-1, to=3-3]
		\arrow["{\pi_*(E\otimes e\otimes E)}"{description}, from=1-3, to=2-2]
		\arrow["{\Phi_{E,E}}"{description}, from=3-3, to=2-2]
		\arrow["\Psi", from=1-3, to=3-3]
	\end{tikzcd}\]
	The right region commutes by how $\Psi$ is defined (\autoref{(E,mu,e):Psi_defn_appendix}), so it suffices to show the left region commutes. To that end, consider the following diagram:
	% https://q.uiver.app/#q=WzAsNyxbMCwwLCJTXmEiXSxbMSwwLCJFIl0sWzIsMCwiRVxcb3RpbWVzIEUiXSxbMiwzLCJFXFxvdGltZXMgRVxcb3RpbWVzIEUiXSxbMCwxLCJTXFxvdGltZXMgU15hIl0sWzAsMiwiRVxcb3RpbWVzIEVcXG90aW1lcyBFIl0sWzAsMywiRVxcb3RpbWVzIEVcXG90aW1lcyBFXFxvdGltZXMgRSJdLFswLDEsIngiXSxbMSwyLCJlXFxvdGltZXMgRSJdLFsyLDMsIkVcXG90aW1lcyBlXFxvdGltZXMgRSJdLFswLDQsIlxccGhpX3swLGF9PVxcbGFtYmRhX3tTXmF9XnstMX0iLDIseyJsZXZlbCI6Miwic3R5bGUiOnsiaGVhZCI6eyJuYW1lIjoibm9uZSJ9fX1dLFs0LDUsImVcXG90aW1lcyBlXFxvdGltZXMgeCIsMl0sWzUsNiwiRVxcb3RpbWVzIEVcXG90aW1lcyBlXFxvdGltZXMgRSIsMl0sWzYsMywiRVxcb3RpbWVzIFxcbXVcXG90aW1lcyBFIiwyXSxbNSwzLCIiLDEseyJsZXZlbCI6Miwic3R5bGUiOnsiaGVhZCI6eyJuYW1lIjoibm9uZSJ9fX1dLFswLDMsImVcXG90aW1lcyBlXFxvdGltZXMgeCIsMV1d
	\[\begin{tikzcd}
		{S^a} & E & {E\otimes E} \\
		{S\otimes S^a} \\
		{E\otimes E\otimes E} \\
		{E\otimes E\otimes E\otimes E} && {E\otimes E\otimes E}
		\arrow["x", from=1-1, to=1-2]
		\arrow["{e\otimes E}", from=1-2, to=1-3]
		\arrow["{E\otimes e\otimes E}", from=1-3, to=4-3]
		\arrow["{\phi_{0,a}=\lambda_{S^a}^{-1}}"', Rightarrow, no head, from=1-1, to=2-1]
		\arrow["{e\otimes e\otimes x}"', from=2-1, to=3-1]
		\arrow["{E\otimes E\otimes e\otimes E}"', from=3-1, to=4-1]
		\arrow["{E\otimes \mu\otimes E}"', from=4-1, to=4-3]
		\arrow[Rightarrow, no head, from=3-1, to=4-3]
		\arrow["{e\otimes e\otimes x}"{description}, from=1-1, to=4-3]
	\end{tikzcd}\]
	The top composition is $\pi_*(E\otimes e\otimes E)(\eta_R(x))$, while the bottom composition is $\Phi_{E,E}(1\otimes\eta_R(x))$. The top right region commutes by functoriality of $-\otimes-$. The bottom left triangle commutes by unitality of $\mu$. Finally, the middle triangle commutes by definition.

	Now, we wish to show the following diagram commutes
	% https://q.uiver.app/#q=WzAsNCxbMSwwLCJcXHBpXyooRSkiXSxbMSwxLCJcXHBpXyooRSkiXSxbMiwwLCJFXyooRSkiXSxbMCwwLCJFXyooRSkiXSxbMCwxLCIiLDAseyJsZXZlbCI6Miwic3R5bGUiOnsiaGVhZCI6eyJuYW1lIjoibm9uZSJ9fX1dLFswLDIsIlxcZXRhX1IiXSxbMCwzLCJcXGV0YV9MIiwyXSxbMywxLCJcXHZhcmUiLDJdLFsyLDEsIlxcdmFyZSJdXQ==
	\[\begin{tikzcd}
		{E_*(E)} & {\pi_*(E)} & {E_*(E)} \\
		& {\pi_*(E)}
		\arrow[Rightarrow, no head, from=1-2, to=2-2]
		\arrow["{\eta_R}", from=1-2, to=1-3]
		\arrow["{\eta_L}"', from=1-2, to=1-1]
		\arrow["\vare"', from=1-1, to=2-2]
		\arrow["\vare", from=1-3, to=2-2]
	\end{tikzcd}\]
	Unravelling how $\eta_L$, $\eta_R$, and $\vare$ are defined, this is the diagram obtained by applying $\pi_*$ to the following diagram:
	% https://q.uiver.app/#q=WzAsNCxbMCwwLCJFXFxvdGltZXMgRSJdLFsxLDAsIkUiXSxbMiwwLCJFXFxvdGltZXMgRSJdLFsxLDEsIkUiXSxbMSwyLCJlXFxvdGltZXMgRSJdLFsyLDMsIlxcbXUiXSxbMCwzLCJcXG11IiwyXSxbMSwwLCJFXFxvdGltZXMgZSIsMl0sWzEsMywiIiwxLHsibGV2ZWwiOjIsInN0eWxlIjp7ImhlYWQiOnsibmFtZSI6Im5vbmUifX19XV0=
	\[\begin{tikzcd}
		{E\otimes E} & E & {E\otimes E} \\
		& E
		\arrow["{e\otimes E}", from=1-2, to=1-3]
		\arrow["\mu", from=1-3, to=2-2]
		\arrow["\mu"', from=1-1, to=2-2]
		\arrow["{E\otimes e}"', from=1-2, to=1-1]
		\arrow[Rightarrow, no head, from=1-2, to=2-2]
	\end{tikzcd}\]
	This commutes by unitality of $\mu$.

	Showing that the third diagram in item (1) in \autoref{hopf_algebroid_defn} is entirely analagous to how we showed the first diagram commutes.

	Now, we'd like to show the following diagram commutes:
\end{proof}

\end{document}