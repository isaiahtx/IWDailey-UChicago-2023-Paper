\documentclass[../main.tex]{subfiles}
\begin{document}

In what follows, we fix an abelian group $A$. We assume the reader is familiar with the basic theory of modules over not-necessarily-commutative rings.

\begin{definition}\label{graded_abgrp}
	An \emph{$A$-graded abelian group} is an abelian group $B$ along with a subgroup $B_a\leq B$ for each $a\in A$ such that the canonical map
	\[\bigoplus_{a\in A}B_a\to B\]
	sending $(x_a)_{a\in A}$ to $\sum_{a\in A}x_a$ is an isomorphism. Given two $A$-graded abelian groups $B$ and $C$, a homomorphism $f:B\to C$ is a \textit{homomorphism of $A$-graded abelian groups}, or just an \emph{$A$-graded homomorphism}, if it preserves the grading, i.e., if it restricts to a map $B_a\to C_a$ for all $a\in A$. 
\end{definition}

It is easy to see that an $A$-graded abelian group $B$ is generated by its \emph{homogeneous} elements, that is, nonzero elements $x\in B$ such that there exists some $a\in A$ with $x\in B_a$.

\begin{remark}
	Clearly the condition that the canonical map $\bigoplus_{a\in A}B_a\to B$ is an isomorphism requires that $B_a\cap B_b=0$ if $a\neq b$. In particular, given a homogeneous element $x\in B$, there exists precisely one $a\in A$ such that $x\in B_a$. We call this $a$ the \emph{degree} of $x$, and we write $|x|=a$.
\end{remark}

\begin{definition}
	An \emph{$A$-graded ring} is a ring $R$ such that its underlying abelian group $R$ is $A$-graded and the multiplication map $R\times R\to R$ restricts to $R_a\times R_b\to R_{a+b}$ for all $a,b\in A$. A morphism of $A$-graded rings is a ring homomorphism whose underlying homomorphism of abelian groups is $A$-graded.
\end{definition}

Explicitly, given an $A$-graded ring $R$ and homogeneous elements $x,y\in R$, we must have $|xy|=|x|+|y|$. For example, given some field $k$, the ring $R=k[x,y]$ is $\bZ^2$-graded, where given $(n,m)\in\bZ^2$, $R_{n,m}$ is the subgroup of those monomials of the form $ax^ny^m$ for some $a\in k$. 

\begin{definition}
	Let $R$ be an $A$-graded ring. A \emph{left $A$-graded $R$-module} $M$ is a left $R$-module $M$ such that $M$ is an $A$-graded abelian group and the action map $R\times M\to M$ restricts to a map $R_a\times M_b\to M_{a+b}$ for all $a,b\in A$. Right $A$-graded $R$-modules are defined similarly. Finally, an $A$-graded $R$-bimodule is an $A$-graded abelian group $M$ which has the structure of both an $A$-graded left and right $R$-module such that given $r,s\in R$ and $m\in M$, $r\cdot(m\cdot s)=(r\cdot m)\cdot s$. 
	
	Morphisms between $A$-graded $R$-modules are precisely $A$-graded $R$-module homomorphisms. We write $R\text-\GrMod$ for the category of left $A$-graded $R$-modules and $\GrMod\text-R$ for the category of right $A$-graded $R$-modules.
\end{definition}

\begin{remark}
	It is straightforward to see that an $A$-graded abelian group is equivalently an $A$-graded $\bZ$-module, where here we are considering $\bZ$ as an $A$-graded ring concentrated in degree $0$. Thus any result below about $A$-graded modules applies equally to $A$-graded abelian groups.
\end{remark}

\begin{lemma}\label{product_of_A_graded}
	Given an $A$-graded ring $R$ and two left (resp.\ right) $A$-graded $R$-modules $M$ and $N$, their direct sum $M\oplus N$ is naturally a left (resp.\ right) $A$-graded $R$-module group by defining
	\[(M\oplus N)_a:=M_a\oplus N_a.\]
\end{lemma}
\begin{proof}
	The canonical map $\bigoplus_{a\in A}(M_a\oplus N_a)\to M\oplus N$ factors as
	\[\bigoplus_{a\in A}(M_a\oplus N_a)\xr\cong\bigoplus_{a\in A}M_a\oplus\bigoplus_{a\in A}N_a\xr\cong M\oplus N.\qedhere\]
\end{proof}

Oftentimes when constructing $A$-graded rings, we do so only by defining the product of homogeneous elements, like so:

\begin{lemma}\label{A_graded_ring}
	Suppose we have an $A$-graded abelian group $R$, a distinguished element $1\in R_0$, and $\bZ$-bilinear maps $m_{a,b}:R_a\times R_b\to R_{a+b}$ for all $a,b\in A$. Further suppose that for all $x\in R_a$, $y\in R_b$, and $z\in R_c$, we have
	\[m_{a+b,c}(m_{a,b}(x,y),z)=m_{a,b+c}(x,m_{b,c}(y,z))\qquad\text{and}\qquad m_{a,0}(x,1)=m_{0,a}(1,x)=x.\]
	Then there exists a unique multiplication map $m:R\times R\to R$ which endows $R$ with the structure of an $A$-graded ring and restricts to $m_{a,b}$ for all $a,b\in A$.
\end{lemma}
\begin{proof}
	Given $r,s\in R$, since $R\cong\bigoplus_{a\in A}R_a$, we may uniquely decompose $r$ and $s$ into homogeneous elements as $r=\sum_{a\in A}r_a$ and $s=\sum_{a\in A}s_a$ with each $r_a,s_a\in R_a$ such that only finitely many of the $r_a$'s and $s_a$'s are nonzero. Then in order to define a distributive product $R\times R\to R$ which restricts to $m_{a,b}:R_a\times R_b\to R_{a+b}$, note we \emph{must} define
	\[r\cdot s=\(\sum_{a\in A}r_a\)\cdot\(\sum_{b\in A}s_b\)=\sum_{a,b\in A}r_a\cdot s_b=\sum_{a,b\in A}m_{a,b}(r_a,s_b).\]
	Thus, we have shown uniqueness. It remains to show this product actually gives $R$ the structure of a ring.  First we claim that the sum on the right is actually finite. Note there exists only finitely many nonzero $r_a$'s and $s_b$'s, and if $s_b=0$ then 
	\[m_{a,b}(r_a,0)=m_{a,b}(r_a,0+0)\overset{(*)}=m_{a,b}(r_a,0)+m_{a,b}(r_a,0)\implies m_{a,b}(r_a,0)=0,\]
	where $(*)$ follows from bilinearity of $m_{a,b}$. A similar argument yields that $m_{a,b}(0,s_b)=0$ for all $a,b\in A$. Hence indeed $m_{a,b}(r_a,s_b)$ is zero for all but finitely many pairs $(a,b)\in A^2$, as desired. Observe that in particular
	\[(r\cdot s)_a=\sum_{b+c=a}m_{b,c}(r_b,s_c)=\sum_{b\in A}m_{b,a-b}(r_b,s_{a-b})=\sum_{c\in A}m_{a-c,c}(r_{a-c},s_{c}).\]
	Now we claim this multiplication is associative. Given $t=\sum_{a\in A}t_a\in R$, we have
	\begin{align*}
		(r\cdot s)\cdot t&=\sum_{a,b\in A}m_{a,b}((r\cdot s)_a,t_b) \\
		&=\sum_{a,b\in A}m_{a,b}\(\sum_{c\in A}m_{a-c,c}(r_{a-c},s_{c}),t_b\) \\
		&\overset{(1)}=\sum_{a,b,c\in A}m_{a,b}(m_{a-c,c}(r_{a-c},s_{c}),t_b) \\
		&\overset{(2)}=\sum_{a,b,c\in A}m_{c,a+b-c}(r_c,m_{a-c,b}(s_{a-c},t_b)) \\
		&\overset{(3)}=\sum_{a,b,c\in A}m_{a,c}(r_a,m_{b,c-b}(s_b,t_{c-b})) \\
		&\overset{(1)}=\sum_{a,c\in A}m_{a,c}\left(r_a,\sum_{b\in A}m_{b,c-b}(s_b,t_{c-b})\right) \\
		&=\sum_{a,c\in A}m_{a,c}(r_a,(s\cdot t)_c)=r\cdot(s\cdot t),
	\end{align*}
	where each occurrence of $(1)$ follows by bilinearity of the $m_{a,b}$'s, each occurrence of $(2)$ is associativity of the $m_{a,b}$'s, and $(3)$ is obtained by re-indexing by re-defining $a:=c$, $b:=a-c$, and $c:=a+b-c$. Next, we wish to show that the distinguished element $1\in R_0$ is a unit with respect to this multiplication. Indeed, we have
	\[1\cdot r\overset{(1)}=\sum_{a\in A}m_{0,a}(1,r_a)\overset{(2)}=\sum_{a\in A}r_a=r\qquad\text{and}\qquad r\cdot 1\overset{(1)}=\sum_{a\in A}m_{a,0}(r_a,1)\overset{(2)}=\sum_{a\in A}r_a=r,\]
	where $(1)$ follows by the fact that $m_{a,b}(0,-)=m_{a,b}(-,0)=0$, which we have shown above, and $(2)$ follows by unitality of the $m_{0,a}$'s and $m_{0,a}$'s, respectively. Finally, we wish to show that this product is distributive. Indeed, we have
	\begin{align*}
		r\cdot(s+t)&=\sum_{a,b\in A}m_{a,b}(r_a,(s+t)_b) \\
		&=\sum_{a,b\in A}m_{a,b}(r_a,s_b+t_b) \\
		&\overset{(*)}=\sum_{a,b\in A}m_{a,b}(r_a,s_b)+\sum_{a,b\in A}m_{a,b}(r_a,t_b)=(r\cdot s)+(r\cdot t),
	\end{align*}
	where $(*)$ follows by bilinearity of $m_{a,b}$. An entirely analagous argument yields that $(r+s)\cdot t=(r\cdot t)+(s\cdot t)$.
\end{proof}

\begin{lemma}\label{A-graded_module}
	Let $R$ be an $A$-graded ring, $M$ an $A$-graded abelian group, and suppose there exists $\bZ$-bilinear maps $\kappa_{a,b}:R_a\times M_b\to M_{a+b}$ for all $a,b\in A$. Further suppose that for all $r\in R_a$, $r'\in R_b$, and $m\in M_c$ that
	\[\kappa_{a+b,c}(r\cdot r',m)=\kappa_{a,b+c}(r,\kappa_{b,c}(r',m))\qquad\text{and}\qquad\kappa_{0,c}(1,m)=m.\]
	Then there is a unique map $\kappa:R\times M\to M$ which endows $M$ with the structure of a left $A$-graded $R$-module and restricts to $\kappa_{a,b}$ for all $a,b\in A$.

	On the other hand, suppose there exists $\bZ$-bilinear maps $\kappa_{a,b}:M_a\times R_b\to M_{a+b}$ for all $a,b\in A$. Further suppose that for all $r\in R_a$, $r'\in R_b$, and $m\in M_c$ that
	\[\kappa_{c,a+b}(m,r\cdot r')=\kappa_{c+a,b}(\kappa_{c,a}(m,r),r')\qquad\text{and}\qquad\kappa_{c,0}(m,1)=m.\]
	Then there is a unique map $\kappa:M\times R\to M$ which endows $M$ with the structure of a right $A$-graded $R$-module and restricts to $\kappa_{a,b}$ for all $a,b\in A$.
\end{lemma}
\begin{proof}
	We show the left module case, as the right module case is entirely analagous. Supposing for each $a,b\in A$ we have a map $\kappa_{a,b}:R_a\times M_b\to M_{a+b}$ satisfying the above conditions, in order to extend these to a map $R\times M\to M$, by additivity we \emph{must} define
	\[\kappa:R\times M\to M\]
	to be the map sending $r=\sum_ar_a$ and $m=\sum_am_a$ to $\sum_{a,b\in A}\kappa_{a,b}(r_a, m_b)$. Now, we need to check that for all $r,s\in R$, $x,y\in M$ that\begin{enumerate}
		\item $r\cdot(x+y)=r\cdot x+r\cdot y$
  		\item $(r+s)\cdot x=r\cdot x+s\cdot x$
		\item $(rs)\cdot x=r\cdot(s\cdot x)$
		\item $1\cdot x=x$,
	\end{enumerate}
	where above we are written $-\cdot-$ for $\kappa(-,-)$. To see the first, note
	\begin{align*}
		\kappa(r,x+y)&=\sum_{a,b\in A}\kappa_{a,b}(r_a,(x+y)_b) \\
		&=\sum_{a,b\in A}\kappa_{a,b}(r_a,x_b+y_b) \\
		&=\sum_{a,b\in A}(\kappa_{a,b}(r_a,x_b)+\kappa_{a,b}(r_a,y_b)) \\
		&=\sum_{a,b\in A}\kappa_{a,b}(r_a,x_b)+\sum_{a,b\in A}\kappa_{a,b}(r_a,y_b) \\
		&=\kappa(r,x)+\kappa(r,y).
	\end{align*}
	To see the second, note
	\begin{align*}
		\kappa(r+s,x)&=\sum_{a,b\in A}\kappa_{a,b}((r+s)_a,x_b) \\
		&=\sum_{a,b\in A}\kappa_{a,b}(r_a+s_a,x_b) \\
		&=\sum_{a,b\in A}(\kappa_{a,b}(r_a,x_b)+\kappa_{a,b}(s_a,x_b)) \\
		&=\sum_{a,b\in A}\kappa_{a,b}(r_a,x_b)+\sum_{a,b\in A}\kappa_{a,b}(s_a,x_b) \\
		&=\kappa(r,x)+\kappa(s,x).
	\end{align*}
	To see the third, note
	\begin{align*}
		\kappa(rs,x)&=\sum_{a,b\in A}\kappa_{a,b}((rs)_a,x_b) \\
		&=\sum_{a,b\in A}\kappa_{a,b}\(\sum_{c\in A}r_cs_{a-c},x_b\) \\
		&=\sum_{a,b,c\in A}\kappa_{a,b}(r_cs_{a-c},x_b) \\
		&=\sum_{a,b,c\in A}\kappa_{a,b}(r_c,\kappa_{a-c,b}(s_{a-c},x_b)) \\
		&=
	\end{align*}
	\todo{FINISH}
\end{proof}

When working with $A$-graded rings and modules, we will often freely use the above propositions without comment. In what follows, fix an $A$-graded ring $R$. We will simply say ``$A$-graded $R$-module'' when we are freely considering either left or right $A$-graded $R$-modules.

\begin{remark}
	We often will denote an $A$-Braded $R$-module $M$ by $M_*$. Given some $a\in A$, we can define the shifted $A$-graded abelian group $M_{*+a}$ whose $b^\text{th}$ component is $M_{b+a}$.
\end{remark}

\begin{definition}
    More generally, given two $A$-graded $R$-modules $M$ and $N$ and some $d\in A$, an $R$-module homomorphism $f:M\to N$ is \emph{an $A$-graded homomorphism of degree $d$} if it restricts to a map $M_a\to N_{a+d}$ for all $a\in A$. Thus, an $A$-graded homomorphism of degree $d$ from $M$ to $N$ is equivalently an $A$-graded homomorphism $M_*\to N_{*+d}$ or an $A$-graded homomorphism $M_{*-d}\to N$. Given some $a\in A$ and left (resp.\ right) $R$-modules $M$ and $N$, we will write 
	\[\Hom_R^d(M,N)=\Hom_R(M_*,N_{*+d})=\Hom_R(M_{*-d},N_*)\]
	to denote the set of $A$-graded homomorphisms of degree $d$ from $M$ to $N$, and simply
	\[\Hom_R(M,N)\] 
	to denote the set of degree-$0$ $A$-graded homomorphisms from $M$ to $N$. Clearly $A$-graded homomorphisms may be added and subtracted, so these are further abelian groups. Thus we have an $A$-graded abelian group
	\[\Hom_R^*(M,N).\]
\end{definition}

Unless stated otherwise, an ``$A$-graded homomorphism'' will always refer to an $A$-graded homomorphism of degree $0$. 

\begin{lemma}\label{ev_at_1_is_iso}
	Let $R$ be an $A$-graded ring and $M$ an $A$-graded left (resp.\ right) $R$-module. Then for all $d\in A$, the evaluation map
	\begin{align*}
		\mathrm{ev_1}:\Hom_{R}^d(R,M)&\to M_d \\
		\varphi&\mapsto\varphi(1)
	\end{align*}
	is an isomorphism of abelian groups.
\end{lemma}
\begin{proof}
	We consider the case that $M$ is a left $A$-graded $R$-module, as showing it when $M$ is a right module is entirely analagous. First of all, this map is clearly a homomorphism, as given degree $d$ $A$-graded homomorphisms $\varphi,\psi:R\to M$, we have 
	\[\mathrm{ev}_1(\varphi+\psi)=(\varphi+\psi)(1)=\varphi(1)+\psi(1)=\mathrm{ev}_1(\varphi)+\mathrm{ev}_1(\psi).\]
	Now, to see it is surjective, let $m\in M_d$, and define $\varphi_m:R\to M$ to send $r\mapsto r\cdot m$. First of all, $\varphi_m$ is a module homomorphism, as given $r,s\in R$, 
	\[\varphi_m(r+s)=(r+s)\cdot m=r\cdot m+s\cdot m=\varphi_m(r)+\varphi_m(s)\quad\text{and}\quad\varphi_m(r\cdot s)=r\cdot s\cdot m=r\cdot\varphi_m(s).\]
	Furthermore, it is clearly $A$-graded of degree $d$, as given a homogeneous element $r\in R_a$ for some $a\in A$, we have $\varphi_m(r)=r\cdot m\in R_{a+d}$, since $m$ is homogeneous of degree $d$. Finally, clearly 
	\[\mathrm{ev}_1(\varphi_m)=\varphi_m(1)=1\cdot m=m,\]
	so indeed $\mathrm{ev}_1$ is surjective. On the other hand, to see it is injective, suppose we are given $\varphi,\psi\in\Hom_R^d(R,M)$ such that $\varphi(1)=\psi(1)$. Then given $r\in R$, we must have
	\[\varphi(r)=\varphi(r\cdot 1)=r\cdot\varphi(1)=r\cdot\psi(1)=\psi(r\cdot 1)=\psi(r),\]
	so $\varphi$ and $\psi$ are exactly the same map. Thus, $\mathrm{ev}_1$ is injective, as desired.
\end{proof}

Recall that given a ring $R$, a left (resp.\ right) module $P$ is \emph{projective} if, for all diagrams of $R$-module homomorphisms of the form
% https://q.uiver.app/#q=WzAsMyxbMCwxLCJQIl0sWzEsMSwiTiJdLFsxLDAsIk0iXSxbMCwxLCJmIiwyXSxbMiwxLCJnIiwwLHsic3R5bGUiOnsiaGVhZCI6eyJuYW1lIjoiZXBpIn19fV1d
\[\begin{tikzcd}
	& M \\
	P & N
	\arrow["f"', from=2-1, to=2-2]
	\arrow["g", two heads, from=1-2, to=2-2]
\end{tikzcd}\]
with $g$ an epimorphism, there exists a lift $h:P\to M$ satisfying $g\circ h=f$
% https://q.uiver.app/#q=WzAsMyxbMCwxLCJQIl0sWzEsMSwiTiJdLFsxLDAsIk0iXSxbMCwxLCJmIiwyXSxbMiwxLCJnIiwwLHsic3R5bGUiOnsiaGVhZCI6eyJuYW1lIjoiZXBpIn19fV0sWzAsMiwiaCIsMCx7InN0eWxlIjp7ImJvZHkiOnsibmFtZSI6ImRhc2hlZCJ9fX1dXQ==
\[\begin{tikzcd}
	& M \\
	P & N
	\arrow["f"', from=2-1, to=2-2]
	\arrow["g", two heads, from=1-2, to=2-2]
	\arrow["h", dashed, from=2-1, to=1-2]
\end{tikzcd}\]
(Note $h$ is not required to be unique.)

\begin{definition}\label{graded_projective_module}
	Let $R$ be an $A$-graded ring, and let $P$ be a left (resp.\ right) $A$-graded $R$-module. Then $P$ is a \emph{graded projective} module if, for all diagrams of $A$-graded $R$-module homomorphisms of the form 
	% https://q.uiver.app/#q=WzAsMyxbMCwxLCJQIl0sWzEsMSwiTiJdLFsxLDAsIk0iXSxbMCwxLCJmIiwyXSxbMiwxLCJnIiwwLHsic3R5bGUiOnsiaGVhZCI6eyJuYW1lIjoiZXBpIn19fV1d
	\[\begin{tikzcd}
		& M \\
		P & N
		\arrow["f"', from=2-1, to=2-2]
		\arrow["g", two heads, from=1-2, to=2-2]
	\end{tikzcd}\]
	with $g$ an epimorphism, there exists an $A$-graded homomorphism $h:P\to M$ satisfying $g\circ h=f$.
	% https://q.uiver.app/#q=WzAsMyxbMCwxLCJQIl0sWzEsMSwiTiJdLFsxLDAsIk0iXSxbMCwxLCJmIiwyXSxbMiwxLCJnIiwwLHsic3R5bGUiOnsiaGVhZCI6eyJuYW1lIjoiZXBpIn19fV0sWzAsMiwiaCIsMCx7InN0eWxlIjp7ImJvZHkiOnsibmFtZSI6ImRhc2hlZCJ9fX1dXQ==
	\[\begin{tikzcd}
		& M \\
		P & N
		\arrow["f"', from=2-1, to=2-2]
		\arrow["g", two heads, from=1-2, to=2-2]
		\arrow["h", dashed, from=2-1, to=1-2]
	\end{tikzcd}\]
	(Note $h$ is not required to be unique.)
\end{definition}

\begin{definition}
	Let $M$ be an $A$-graded $R$-module. Then an \emph{$A$-graded $R$-submodule} is an $A$-graded $R$-module $N$ which is a subset of $M$ and for which the inclusion $N\into M$ is an $A$-graded homomorphism of $R$-modules. Equivalently, it is a submodule $N$ one for which the canonical map
	\[\bigoplus N\cap M_a\to N\]
	is an isomorphism.
\end{definition}

\begin{lemma}\label{submodule_lemma}
	Let $M$ be an $A$-graded $R$-module. Then an $R$-submodule $N\leq M$ is an $A$-graded submodule if and only if it is generated as an $R$-module by homogeneous elements of $M$.
\end{lemma}
\begin{proof}
	If $N\leq M$ is a $A$-graded submodule, it is generated by the set of all its homogeneous elements, which are also homogeneous elements in $M$, by definition.

	Conversely, suppose $N\leq M$ is a submodule which is generated by homogeneous elements of $M$. Then define $N_a:=N\cap M_a$, and consider the canonical map
	\[\Phi:\bigoplus_{a\in A}N_a\to N.\]
	First of all, it is surjective, as each generator of $N$ belongs to some $N_a$, by definition. To see it is injective, consider the following commutative diagram:
	% https://q.uiver.app/#q=WzAsNCxbMCwwLCJcXGJpZ29wbHVzX3thXFxpbiBBfU5fYSJdLFsxLDAsIlxcYmlnb3BsdXNfe2FcXGluIEF9TV9hIl0sWzEsMSwiTSJdLFswLDEsIk4iXSxbMCwxLCIiLDAseyJzdHlsZSI6eyJ0YWlsIjp7Im5hbWUiOiJob29rIiwic2lkZSI6InRvcCJ9fX1dLFsxLDIsIlxcY29uZyJdLFswLDMsIlxcUGhpIiwyXSxbMywyLCIiLDIseyJzdHlsZSI6eyJ0YWlsIjp7Im5hbWUiOiJob29rIiwic2lkZSI6InRvcCJ9fX1dXQ==
	\[\begin{tikzcd}
		{\bigoplus_{a\in A}N_a} & {\bigoplus_{a\in A}M_a} \\
		N & M
		\arrow[hook, from=1-1, to=1-2]
		\arrow["\cong", from=1-2, to=2-2]
		\arrow["\Phi"', from=1-1, to=2-1]
		\arrow[hook, from=2-1, to=2-2]
	\end{tikzcd}\]
	Since $\Phi$ composes with an injection to get an injection, clearly $\Phi$ must be injective itself. We have the desired result.
\end{proof}

\begin{proposition}\label{image_and_kernel_of_A_graded_map_is_A_graded}
	Given two left (resp.\ right) $A$-graded $R$-modules $M$ and $N$ and an $A$-graded $R$-module homomorphism $\varphi:M\to N$ (of possibly nonzero degree), the kernel and images of $\varphi$ are $A$-graded submodules of $M$ and $N$, respectively.
\end{proposition}
\begin{proof}
	First recall that a degree $d$ $A$-graded homomorphism $M\to N$ is simply an $A$-graded homomorphism $M_*\to N_{*+d}$, so it suffices to consider the case $\varphi$ is of degree $0$. Next, note that since the forgetful functor from $R$-modules to abelian groups preserves kernels and images, it suffices to consider the case that $\varphi$ is a homomorphism of $A$-graded abelian groups. Finally, by \autoref{submodule_lemma}, it suffices to show that $\ker\varphi$ and $\imm\varphi$ are generated by homogeneous elements of $M$ and $N$, respectively.

	Note that by the universal property of the coproduct in $\Ab$, the data of an $A$-graded homomorphism of abelian groups $\varphi:M\to N$ is precisely the data of an $A$-indexed collection of abelian group homomorphisms $\varphi_a:M_a\to N_a$, in which case the following diagram commutes:
	% https://q.uiver.app/#q=WzAsNCxbMCwwLCJcXGJpZ29wbHVzX2FNX2EiXSxbMSwwLCJcXGJpZ29wbHVzX2FOX2EiXSxbMCwxLCJNIl0sWzEsMSwiTiJdLFswLDEsIlxcYmlnb3BsdXNfYVxcdmFycGhpX2EiXSxbMCwyLCJcXGNvbmciLDJdLFsyLDMsIlxcdmFycGhpIl0sWzEsMywiXFxjb25nIl1d
	\[\begin{tikzcd}
		{\bigoplus_aM_a} & {\bigoplus_aN_a} \\
		M & N
		\arrow["{\bigoplus_a\varphi_a}", from=1-1, to=1-2]
		\arrow["\cong"', from=1-1, to=2-1]
		\arrow["\varphi", from=2-1, to=2-2]
		\arrow["\cong", from=1-2, to=2-2]
	\end{tikzcd}\]
	Finally, the desired result follows by the purely formal fact that taking images and kernels commutes with arbitrary direct sums.
\end{proof}

\begin{proposition}\label{preimage_of_A_graded_is_A_graded}
	Given two left (resp.\ right) $A$-graded $R$-modules $M$ and $N$, an $A$-graded submodule $K\leq N$, and an $A$-graded $R$-module homomorphism $\varphi:M\to N$ (of possibly nonzero degree), the submodule $\varphi^{-1}(K)$ of $M$ is $A$-graded.
\end{proposition}
\begin{proof}
	Recall that a degree $d$ $A$-graded homomorphism $M\to N$ is simply an $A$-graded homomorphism $M_*\to N_{*+d}$, so it suffices to consider the case $\varphi$ is of degree $0$. Now, let $x\in L:=\varphi^{-1}(K)$. As an element of $M$, we may uniquely write $x=\sum_{a\in A}x_a$ where each $x_a\in M_a$. Similarly, if we set $y:=\varphi(x)$, then we may uniquely write $y=\sum_{a\in A}y_a$ where each $y_a\in N_a$. Then since $K$ is an $A$-graded submodule of $N$ and $y\in K$, by definition, we have that $y_a\in K$ for each $a$. Finally, note that
	\[\sum_{a\in A}y_a=y=\varphi(x)=\sum_{a\in A}\varphi(x_a),\]
	so that $\varphi(x_a)=y_a\in K$ for all $a\in A$, so that $x_a\in L$ for all $a\in A$. Thus we have shown that each element in $L$ can be written as a sum of homogeneous elements in $M$, as desired.
\end{proof}

\begin{proposition}\label{quotient_of_A_graded_is_A_graded}
	Given an $A$-graded $R$-module $M$ and an $A$-graded subgroup $N\leq M$, the quotient $M/N$ is canonically $A$-graded by defining $(M/N)_a$ to be the subgroup generated by cosets represented by homogeneous elements of degree $a$ in $M$. Furthermore, the canonical maps $M_a/N_a\to (M/N)_a$ taking a coset $m+N_a$ to $m+N$ are isomorphisms.
\end{proposition}
\begin{proof}
	Consider the canonical map
	\[\Phi:\bigoplus_a (M/N)_a\to M/N.\]
	First of all, surjectivity of $\Phi$ follows by commutativity of the following diagram:
	% https://q.uiver.app/#q=WzAsNCxbMCwwLCJcXGJpZ29wbHVzX2EgTV9hIl0sWzEsMCwiTSJdLFsxLDEsIk0vTiJdLFswLDEsIlxcYmlnb3BsdXNfYShNL04pX2EiXSxbMCwxLCJcXGNvbmciXSxbMSwyLCIiLDAseyJzdHlsZSI6eyJoZWFkIjp7Im5hbWUiOiJlcGkifX19XSxbMCwzLCIiLDIseyJzdHlsZSI6eyJoZWFkIjp7Im5hbWUiOiJlcGkifX19XSxbMywyLCJcXFBoaSJdXQ==
	\[\begin{tikzcd}
		{\bigoplus_a M_a} & M \\
		{\bigoplus_a(M/N)_a} & {M/N}
		\arrow["\cong", from=1-1, to=1-2]
		\arrow[two heads, from=1-2, to=2-2]
		\arrow[two heads, from=1-1, to=2-1]
		\arrow["\Phi", from=2-1, to=2-2]
	\end{tikzcd}\]
	where the vertical left map sends a generator $m\in M_a$ to the coset $m+N$ in $(M/N)_a\sseq M/N$. To see $\Phi$ is injective, suppose we are given some element $(m_a+N)_{a\in A}$ in $\bigoplus_a(M/C)_a$ such that $\sum_{a\in A}(m_a+N)=0$ in $M/N$. Thus $\sum_{a\in A}m_a\in N$, and since $N$ is $A$-graded this implies that each $m_a$ belongs to $N\cap M_a=N_a$, so that in particular $m_a+N$ is zero in $(M/N)_a\sseq M/N$, so that $(m_a+N)_{a\in A}=0$ in $\bigoplus_{a}(M/N)_a$, as desired.

	It remains to show that the canonical map
	\[\varphi_a:M_a/N_a\to (M/N)_a\]
	is an isomorphism. It is clearly surjective, as $(M/N)_a$ is generated by elements $m+N$ for $m\in M_a$, and these elements make up precisely the image of $\varphi_a$. Thus $\varphi_a$ hits every generator of $(M/N)_a$, so $\varphi_a$ is surjective. On the other hand, suppose we are given some $m\in M_a$ such that $\varphi(m+N_a)=m+N=0$. Thus $m\in N$, and $m\in M_a$, so that $m\in M_a\cap N=N_a$, meaning $m+N_a=0$ in $M_a/N_a$, as desired.
\end{proof}

Recall that given a ring $R$, a left $R$-module $M$, a right $R$-module $N$, and an abelian group $A$, an \emph{$R$-balanced map} $\varphi:M\times N\to B$ is one which satisifies
\begin{align*}
	\varphi(m,n+n')&=\varphi(m,n)+\varphi(m,n') \\
	\varphi(m+m',n)&=\varphi(m,n)+\varphi(m',n) \\
	\varphi(m\cdot r,n)&=\varphi(m,r\cdot n).
\end{align*}
for all $m,m'\in M$, $n,n'\in N$, and $r\in R$. Then the tensor product $M\otimes_RN$ is the universal abelian group equipped with an $R$-balanced map $\otimes:M\times N\to M\otimes_RN$ such that for every abelian group $B$ and every $R$-balanced map $\varphi:M\times N\to B$, there is a \emph{unique} group homomorphism $\wt\varphi:M\otimes_RN\to B$ such that $\wt f\circ\otimes=f$. We call elements in the image of $\otimes:M\times N\to M\otimes_RN$ \emph{pure tensors}. It is a standard fact that $M\otimes_RN$ is generated as an abelian group by its pure tensors.

\begin{definition}
	Suppose we have a right $A$-graded $R$-module $M$, a left $A$-graded $R$-module $N$, and an $A$-graded abelian group $B$. Then an \emph{$A$-graded $R$-balanced map} $\varphi:M\times N\to B$ is an $R$-balanced map which restricts to $M_a\times N_b\to B_{a+b}$ for all $a,b\in A$.
\end{definition}

\begin{proposition}\label{tensor_of_A_graded_is_A_graded}
	Suppose we have a right $A$-graded $R$-module $M$ and a left $A$-graded $R$-module $N$. Then the tensor product
	\[M\otimes_RN\]
	is naturally an $A$-graded abelian group by defining $(M\otimes_RN)_a$ to be the subgroup generated by \emph{homogeneous} pure tensors $m\otimes n$ with $m\in M_b$ and $n\in N_c$ such that $b+c=a$. Furthermore, if either $M$ (resp.\ $N$) is an $A$-graded bimodule, then this decomposition makes $M\otimes_RN$ into a left (resp.\ right) $A$-graded $R$-module
\end{proposition}
\begin{proof}
	By definition, since $M$ and $N$ are $A$-graded abelian groups, they are generated (as abelian groups) by their homogeneous elements. Thus it follows that $M\otimes_RN$ is generated by \textit{homogeneous pure tensors}, that is, elements of the form $m\otimes n$ with $m\in M$ and $n\in N$ homogeneous. Now, given a homogeneous pure tensor $m\otimes n$, we define its \textit{degree} by the formula $|m\otimes n|:=|m|+|n|$. It follows this formula is well-defined by checking that given homogeneous elements $m\in M$, $n\in N$, and $r\in R$ that
	\[|(m\cdot r)\otimes n|=|m\cdot r|+|n|=|m|+|r|+|n|=|m|+|r\cdot n|=|m\otimes(r\cdot n)|.\]
	Thus, we may define $(M\otimes_RN)_a$ to be the subgroup of $M\otimes_RN$ generated by those pure homogeneous tensors of degree $a$. Now, consider the map
	\[\Psi:M\times N\to\bigoplus_{a\in A}(M\otimes_RN)_a\]
	which takes a pair $(m,n)=\sum_{a\in A}(m_a,n_a)$ to the element $\Psi(m,n)$ whose $a^\text{th}$ component is
	\[(\Psi(m,n))_a:=\sum_{b+c=a}m_b\otimes n_c.\]
	It is straightforward to see that this map is $R$-balanced, in the sense that it is additive in each argument and $\Psi(m\cdot r,n)=\Psi(m,r\cdot n)$ for all $m\in M$, $n\in N$, and $r\in R$. Thus by the universal property of $M\otimes_RN$, we get a homomorphism of abelian groups $\wt\Psi:M\otimes_RN\to\bigoplus_{a\in A}(M\otimes_RN)_a$ lifting $\Psi$ along the canonical map $M\times N\to M\otimes_RN$. Now, also consider the canonical map
	\[\Phi:\bigoplus_{a\in A}(M\otimes_RN)_a\to M\otimes_RN.\]
	We would like to show $\wt\Psi$ and $\Phi$ are inverses of eah other. Since $\wt\Psi$ and $\Phi$ are both homomorphisms, it suffices to show this on generators. Let $m\otimes n$ be a homogeneous pure tensor with $m=m_a\in M_a$ and $n=n_b\in N_b$. Then we have
	\[\Phi(\wt\Psi(m\otimes n))=\Phi\(\bigoplus_{a\in A}\sum_{b+c=a}m_b\otimes n_c\)\overset{(*)}=\Phi(m\otimes n)=m\otimes n,\]
	and
	\[\wt\Psi(\Phi(m\otimes n))=\wt\Psi(m\otimes n)=\bigoplus_{a\in A}\sum_{b+c=a}m_b\otimes n_c\overset{(*)}=m\otimes n,\]
	where both occurrences of $(\ast)$ follow by the fact that $m_b\otimes n_c=0$ unless $b=c=a$, in which case $m_a\otimes n_a=m\otimes n$. Thus since $\Phi$ is an isomorphism, $M\otimes_RN$ is indeed an $A$-graded abelian group, as desired.

	Now, suppose that $M$ is an $A$-graded $R$-bimodule, so there exists left and right $A$-graded actions of $R$ on $M$ such that given $r,s\in R$ and $m\in M$ we have $r\cdot(m\cdot s)=(r\cdot m)\cdot s$. Then we would like to show that given a left $A$-graded $R$-module $N$ that $M\otimes_RN$ is canonically a left $A$-graded $R$-module. Indeed, define the action of $R$ on $M\otimes_RN$ on pure tensors by the formula
	\[r\cdot(m\otimes n)=(r\cdot m)\otimes n.\]
	First of all, clearly this map is $A$-graded, as if $r\in R_a$, $m\in M_b$, and $n\in N_c$ then $(r\cdot m)\otimes n$, by definition, has degree $|r\cdot m|+|n|=|r|+|m|+|n|$ (the last equality follows since the left action of $R$ on $M$ is $A$-graded). In order to show the above map defines a left module structure, it suffices to show that given pure tensors $m\otimes n,m'\otimes n'\in M\otimes_RN$ and elements $r,r'\in R$ that
	\begin{enumerate}
		\item $r\cdot(m\otimes n+m'\otimes n')=r\cdot(m\otimes n)+r\cdot( m'\otimes n')$,
		\item $(r+r')\cdot(m\otimes n)=r\cdot(m\otimes n)+r'\cdot(m'\otimes n')$,
		\item $(rr')\cdot(m\otimes n)=r\cdot(r'\cdot(m\otimes n))$, and
		\item $1\cdot (m\otimes n)=m\otimes n$.
	\end{enumerate}
	Axiom $(1)$ holds by definition. To see $(2)$, note that by the fact that $R$ acts on $M$ on the left that
	\[(r+r')\cdot(m\otimes n)=((r+r')\cdot m)\otimes n=(r\cdot m+r'\cdot m)\otimes n=r\cdot m\otimes n+r'\cdot m\otimes n.\]
	That $(3)$ and $(4)$ hold follows similarly by the fact that $(rr')\cdot m=r\cdot(r'\cdot m)$ and $1\cdot m=m$.

	Conversely, if $N$ is an $A$-graded $R$-bimodule, then showing $M\otimes_RN$ is canonically a right $A$-graded $R$-module via the rule
	\[(m\otimes n)\cdot r=m\otimes(n\cdot r)\]
	is entirely analagous.
\end{proof}

\begin{lemma}\label{tensor_lift_of_A_graded_is_A_graded}
	Let $R$ be an $A$-graded ring, and suppose we have a right $A$-graded $R$-module $M$ and a left $A$-graded $R$-module $N$. Then given an $A$-graded abelian group $B$ and an $A$-graded $R$-balanced map
	\[\varphi:M\times N\to B,\]
	the lift
	\[\wt\varphi:M\otimes_RN\to B\]
	determined by the universal property of $M\otimes_RN$ is an $A$-graded homomorphism.
\end{lemma}
\begin{proof}
	This simply amounts to unravelling definitions. Recall that the subgroup of homogeneous elements of degree $a$ in $M\otimes_RN$ is that generated by pure tensors $m\otimes n$ with $m$ and $n$ homogeneous satisfying $|m|+|n|=a$. Thus, in order to show $\wt\varphi$ is an $A$-graded homomorphism, it suffices to show that given homogeneous $m\in M$ and $n\in N$ that $\wt\varphi(m\otimes n)$ is homogeneous and that
	\[|\wt\varphi(m\otimes n)|=|m\otimes n|=|m|+|n|.\]
	Indeed, given two such elements $m$ and $n$, consider the following diagram
	% https://q.uiver.app/#q=WzAsMyxbMCwxLCJNXFx0aW1lcyBOIl0sWzAsMCwiTVxcb3RpbWVzX1JOIl0sWzEsMSwiQiJdLFswLDFdLFsxLDIsIlxcd3RcXHZhcnBoaSJdLFswLDIsIlxcdmFycGhpIiwyXV0=
	\[\begin{tikzcd}
		{M\otimes_RN} \\
		{M\times N} & B
		\arrow[from=2-1, to=1-1]
		\arrow["\wt\varphi", from=1-1, to=2-2]
		\arrow["\varphi"', from=2-1, to=2-2]
	\end{tikzcd}\]
	This diagram commutes by universal property of $-\otimes_R-$. Note that the element $m\otimes n$ is mapped to by the pair $(m,n)$ along the left vertical map. Hence by commutativity, we necessarily have
	\[|\wt\varphi(m\otimes n)|=|\varphi(m,n)|\overset{(*)}=|m|+|n|,\]
	where $(*)$ follows by the fact that $\varphi$ is an $A$-graded $R$-balanced map.
\end{proof}

%In this section, fix the following data:
%\begin{itemize}
	%\item A pointed symmetric closed monoidal stable model category $(\cC,\otimes,S)$
	%\item An abelian group $A$
	%\item A homomorphism $h:A\to\mathrm{Pic}(\cC)$
	%\item For each $a\in A$, an object $S^a$ in $\cC$ belonging to the isomorphism class of $h(a)$. WLOG, we may assume $S^0=S$. 
%\end{itemize}
%We write $\ho\cC$ for the homotopy category of $\cC$, and given two objects $X,Y\in\cC$, we write $[X,Y]$ for the homset $\ho\cC(X,Y)$, and we write $F(X,Y)$ for the internal-hom object in $\ho\cC$. By \cite{Dugger_2014}, in the homotopy category $\ho\cC$ there exists a \textit{coherent} family of isomorphisms
%\[\phi_{a,b}:S^{a+b}\xrightarrow\cong S^a\otimes S^b,\]
%in the sense that $\phi_{a,0}$ and $\phi_{0,a}$ may be identified with the unitors $S^a\cong S^a\otimes S$ and $S^a\cong S\otimes S^a$, and for all $a,b,c\in A$ the following diagram commutes in $\ho\cC$:
%\[\begin{tikzcd}
	%{S^{a+b}\otimes S^c} & {S^{a+b+c}} & {S^a\otimes S^{b+c}} \\
	%{(S^a\otimes S^b)\otimes S^c} && {S^a\otimes(S^b\otimes S^c)}
	%\arrow["{\phi_{a+b,c}}"', from=1-2, to=1-1]
	%\arrow["{\phi_{a,b+c}}", from=1-2, to=1-3]
	%\arrow["{S^a\otimes\phi_{b,c}}", from=1-3, to=2-3]
	%\arrow["{\phi_{a,b}\otimes S^c}"', from=1-1, to=2-1]
	%\arrow["\cong", from=2-1, to=2-3]
%\end{tikzcd}\]
%Given an object $X$ in $\cC$ and some $a\in A$, we may define $\pi_a(X):=[S^a,X]$, and we write $\pi_\ast(X)$ for the $A$-graded abelian group $\bigoplus_{a\in A}\pi_a(X)$. More generally, given two objects $E$ and $X$ in $\cC$ and some $a\in A$, we define
%\[E_a(X):=\pi_a(E\wedge X)=[S^a,E\wedge X]\qquad\text{and}\qquad E^a(X):=[X,S^a\otimes E]\cong\pi_{-a}(F(X,E)).\]
%As described in \cite{Dugger_2014}, if $E$ is a 

%\begin{definition}
%	Let $E$ be a flat cellular commutative ring spectrum. Then a \emph{left $E_\aast(E)$-comodule} is the data of
%	\begin{enumerate}
%		\item A $\bZ^2$-graded left $\pi_\aast(E)$-module $M$;
%		\item A homomorphism of left-graded $\pi_\aast(E)$-modules:
%		\[\Psi_M:M\to E_\aast(E)\otimes_{\pi_\aast(E)}M.\]
%		(Note $E_\aast(E)\otimes_{\pi_\aast(E)}M$ is canonically a left $\pi_\aast(E)$-module by \autoref{tensor_of_A_graded_is_A_graded}.)
%	\end{enumerate}
%	These data must make the following diagrams commute:
%	\begin{enumerate}
%		\item (Co-unitality)
%		\[\begin{tikzcd}
%			{M} & {E_\aast(E)\otimes_{\pi_\aast(E)}M} \\
%			& {\pi_\aast(E)\otimes_{\pi_\aast(E)}M}
%			\arrow["{\Psi_M}", from=1-1, to=1-2]
%			\arrow["{\vare\otimes M}", from=1-2, to=2-2]
%			\arrow["\cong"', from=1-1, to=2-2]
%		\end{tikzcd}\]
%		\item (Co-action property)
%		\[\begin{tikzcd}
%			M && {E_\aast(E)\otimes_{\pi_\aast(E)}M} \\
%			\\
%			{E_\aast(E)\otimes_{\pi_\aast(E)}M} && {E_\aast(E)\otimes_{\pi_\aast(E)}E_\aast(E)\otimes_{\pi_\aast(E)}M}
%			\arrow["{\Psi_M}", from=1-1, to=1-3]
%			\arrow["{\Psi\otimes M}", from=1-3, to=3-3]
%			\arrow["{\Psi_M}"', from=1-1, to=3-1]
%			\arrow["{E_\aast(E)\otimes\Psi_M}", from=3-1, to=3-3]
%		\end{tikzcd}\]
%	\end{enumerate}
%	
%	Given two left $E_\aast(E)$-comodules $M$ and $N$, a \emph{homomorphism of $E_\aast(E)$-comodules} is a homomorphism $f:M\to N$ of the underlying graded left $\pi_\aast(E)$-modules such that the following diagram commutes:
%	\[\begin{tikzcd}
%		M & N \\
%		{E_\aast(E)\otimes_{\pi_\aast(E)}M} & {E_\aast(E)\otimes_{\pi_\aast(E)}N}
%		\arrow["f", from=1-1, to=1-2]
%		\arrow["{\Psi_N}", from=1-2, to=2-2]
%		\arrow["{\Psi_M}"', from=1-1, to=2-1]
%		\arrow["{E\otimes f}", from=2-1, to=2-2]
%	\end{tikzcd}\]
%	
%	We write $E_\aast(E)\text-\CoMod$ for the resulting category of left $E_\aast(E)$-comodules. The notation for the hom-sets in this category is usually abbreviated to
%	\[\Hom_{E_\aast(E)}(-,-):=\Hom_{E_\aast(E)\text-\CoMod}(-,-).\]
%\end{definition}
%
%
%Recall when working with the classical Adams spectral sequence, one usually develops the theory of \textit{graded commutative Hopf algebroids}, i.e., internal groupoids in the opposite category $\mbf{gCRing}^\op$ of $\bZ$-graded commutative rings, regarded with its cartesian monoidal category structure. Then, one goes on to show that given a commutative ring spectrum $E$ in $\hoSp$ that $E_\ast(E)$ is a commutative Hopf algebroid over $\pi_\ast(E)$.
%
%Now, in the motivic setting, things become a bit more subtle. Namely, given a commutative ring spectrum $E$ in $\SH\scS$, we would like $E_\aast(E)$ to be a ``bigraded commutative Hopf algebroid over $\pi_\aast(E)$''. To define such a thing, we would like to find some category $\cC$ containing objects such as $E_\aast(E)$ and $\pi_\aast(E)$ so that the pair $(E_\aast(E),\pi_\aast(E))$ forms a groupoid object in $\cC^\op$. The na\"ive answer would be to consider $E_\aast(E)$ and $\pi_\aast(E)$ as objects in some category of ``bigraded commutative rings'', in the same way we considered $E_\ast(E)$ and $\pi_\ast(E)$ as ojects in the category of graded commutative rings in the classical case. Yet, we run into difficulty here, as the commutative law for $E_\aast(E)=\pi_\aast(E\wedge E)$ and $\pi_\aast(E)$ (\autoref{good_product}) depends on their structure as $\pi_\aast(S)$-algebras. Thus, we instead are led to the category of commutative $\pi_\aast(S)$-algebras:
%
%\begin{definition}
%	Let $\CStabRing\scS$ denote the full subcategory of $\bZ^2$-graded algebras over $\pi_\aast^\scS(S)$ (the stable homotopy groups of the sphere spectrum $S$ in $\SH\scS$) containing those objects satisfying the commutativity condition given in \autoref{good_product}. Explicitly, an object in $\CStabRing\scS$ is a $\bZ^2$-graded ring $C_\aast$ together with a ring morphism $e:\pi_\aast^\scS(S)\to C_\aast$ such that given $x\in C_{a_1,a_2}$ and $y\in C_{b_1,b_2}$, we have
%	\[x\cdot y=y\cdot x\cdot(-1)^{a_1b_1}\cdot e(-\epsilon)^{a_2b_2}.\]
%	A morphism in $\CStabRing\scS$ is simply a morphism of $\pi_\aast^\scS(S)$-algebras.
%\end{definition}
%
%\begin{definition}\label{BCHA}
%	For our purposes, a \textit{bigraded commutative Hopf algebroid} $(\Gamma,A)$ is an internal groupoid in the opposite category $(\CStabRing\scS)^\op$.
%\end{definition}
%
%Let's unravel what this definition means. First, recall the definition of a groupoid object in a category:
%
%\begin{definition}
%	Let $\cC$ be a category admitting pullbacks. A \emph{groupoid object} in $\cC$ consists of a pair of objects $(M,O)$ together with five morphisms
%	\begin{enumerate}
%		\item \emph{Source and target}: $s,t:M\to O$,
%		\item \emph{Identity}: $e:O\to M$,
%		\item \emph{Composition}: $c:M\times_{O}M\to M$,
%		\item \emph{Inverse}: $i:M\to M$
%	\end{enumerate}
%	Explicitly, $M\times_OM$ fits into the following pullback diagram:
%	\[\begin{tikzcd}
%		{M\times_OM} & M \\
%		M & O
%		\arrow["{p_2}", from=1-1, to=1-2]
%		\arrow["{p_1}"', from=1-1, to=2-1]
%		\arrow["s"', from=2-1, to=2-2]
%		\arrow["t", from=1-2, to=2-2]
%		\arrow["\lrcorner"{anchor=center, pos=0.125}, draw=none, from=1-1, to=2-2]
%	\end{tikzcd}\]
%	These data must satisfy the following diagrams:
%	\begin{enumerate}
%		\item Composition works correctly:
%		\[\begin{tikzcd}
%			{M\times_OM} & M & M & O & M & {M\times_OM} & M \\
%			M & O && O && M & O
%			\arrow["e"', from=1-4, to=1-3]
%			\arrow["s"', from=1-3, to=2-4]
%			\arrow["e", from=1-4, to=1-5]
%			\arrow["t", from=1-5, to=2-4]
%			\arrow[Rightarrow, no head, from=1-4, to=2-4]
%			\arrow["c", from=1-1, to=1-2]
%			\arrow["t", from=1-2, to=2-2]
%			\arrow["{p_1}"', from=1-1, to=2-1]
%			\arrow["t", from=2-1, to=2-2]
%			\arrow["c"', from=1-6, to=2-6]
%			\arrow["s", from=2-6, to=2-7]
%			\arrow["{p_2}", from=1-6, to=1-7]
%			\arrow["s", from=1-7, to=2-7]
%		\end{tikzcd}\]
%		\item Associativity of composition:
%		\[\begin{tikzcd}
%			{M\times_O(M\times_OM)} && {M\times_O(M\times_OM)} \\
%			{M\times_OM} & M & {M\times_OM}
%			\arrow["\cong", from=1-1, to=1-3]
%			\arrow["{M\times c}"', from=1-1, to=2-1]
%			\arrow["{M\times c}", from=1-3, to=2-3]
%			\arrow["c", from=2-1, to=2-2]
%			\arrow["c"', from=2-3, to=2-2]
%		\end{tikzcd}\]
%		where the top objects and the maps $M\times c$, $c\times M$ are determined like so, where both outer and inner squares in the following diagram are pullback squares:
%		\[\begin{tikzcd}
%			{M\times_O(M\times_OM)} && {M\times_OM} & {(M\times_OM)\times_OM} && M \\
%			& {M\times_OM} & M && {M\times_OM} & M \\
%			M & M & O & {M\times_OM} & M & O
%			\arrow["{p_2}", from=1-1, to=1-3]
%			\arrow["{p_2}", from=1-4, to=1-6]
%			\arrow["{p_1}"', from=1-4, to=3-4]
%			\arrow["c", from=3-4, to=3-5]
%			\arrow["s", from=3-5, to=3-6]
%			\arrow["c", from=1-3, to=2-3]
%			\arrow["t", from=2-3, to=3-3]
%			\arrow["{M\times c}", dashed, from=1-1, to=2-2]
%			\arrow["{c\times M}", dashed, from=1-4, to=2-5]
%			\arrow[from=2-5, to=3-5]
%			\arrow["{p_2}", from=2-2, to=2-3]
%			\arrow[Rightarrow, no head, from=1-6, to=2-6]
%			\arrow["t", from=2-6, to=3-6]
%			\arrow[from=2-5, to=2-6]
%			\arrow["{p_1}"', from=1-1, to=3-1]
%			\arrow[Rightarrow, no head, from=3-1, to=3-2]
%			\arrow["s", from=3-2, to=3-3]
%			\arrow["{p_1}"', from=2-2, to=3-2]
%		\end{tikzcd}\]
%		\item Unitality of composition: Given the maps $(\id_M,e\circ t),(e\circ s,\id_M):M\to M\times_OM$ defined by the universal property of $M\times_OM$:
%		\[\begin{tikzcd}
%			M &&& M \\
%			& {M\times_OM} & M && {M\times_OM} & M \\
%			& M & O && M & O
%			\arrow["{(\id_M,e\circ s)}", dashed, from=1-1, to=2-2]
%			\arrow["{p_2}", from=2-2, to=2-3]
%			\arrow["t", from=2-3, to=3-3]
%			\arrow["{p_1}"', from=2-2, to=3-2]
%			\arrow["s", from=3-2, to=3-3]
%			\arrow["{e\circ s}", curve={height=-18pt}, from=1-1, to=2-3]
%			\arrow[curve={height=18pt}, Rightarrow, no head, from=1-1, to=3-2]
%			\arrow["{(e\circ t,\id_M)}", dashed, from=1-4, to=2-5]
%			\arrow["{p_2}", from=2-5, to=2-6]
%			\arrow["t", from=2-6, to=3-6]
%			\arrow["{p_1}"', from=2-5, to=3-5]
%			\arrow["s", from=3-5, to=3-6]
%			\arrow[curve={height=-18pt}, Rightarrow, no head, from=1-4, to=2-6]
%			\arrow["{e\circ t}"', curve={height=18pt}, from=1-4, to=3-5]
%			\arrow["\lrcorner"{anchor=center, pos=0.125}, draw=none, from=2-2, to=3-3]
%			\arrow["\lrcorner"{anchor=center, pos=0.125}, draw=none, from=2-5, to=3-6]
%		\end{tikzcd}\]
%		the following diagram commutes:
%		\[\begin{tikzcd}
%			M & {M\times_O M} \\
%			{M\times_OM} & M
%			\arrow["c", from=1-2, to=2-2]
%			\arrow["{(e\circ t,\id_M)}", from=1-1, to=1-2]
%			\arrow["{(\id_M,e\circ s)}"', from=1-1, to=2-1]
%			\arrow["c"', from=2-1, to=2-2]
%		\end{tikzcd}\]
%		\item Inverse: The following diagrams must commute:
%		\[\begin{tikzcd}
%			& M & M & {M\times_OM} & M && M \\
%			M & M & O & M & O & O & M & O
%			\arrow["{(\id_M,i)}", from=1-3, to=1-4]
%			\arrow["t"', from=1-3, to=2-3]
%			\arrow["e", from=2-3, to=2-4]
%			\arrow["c", from=1-4, to=2-4]
%			\arrow["{(i,\id_M)}"', from=1-5, to=1-4]
%			\arrow["s", from=1-5, to=2-5]
%			\arrow["e"', from=2-5, to=2-4]
%			\arrow["i", from=1-2, to=2-2]
%			\arrow["i", from=1-7, to=2-7]
%			\arrow["s"', from=1-7, to=2-6]
%			\arrow["t"', from=2-7, to=2-6]
%			\arrow["t", from=1-7, to=2-8]
%			\arrow["s", from=2-7, to=2-8]
%			\arrow["i"', from=2-2, to=2-1]
%			\arrow[Rightarrow, no head, from=1-2, to=2-1]
%		\end{tikzcd}\]
%		where the arrows $(\id_M,i)$ and $(i,\id_M)$ are determined by the universal property of $M\times_OM$ like so:
%		\[\begin{tikzcd}
%			M &&& M \\
%			& {M\times_OM} & M && {M\times_OM} & M \\
%			& M & O && M & O
%			\arrow["{(i,\id_M)}", dashed, from=1-4, to=2-5]
%			\arrow["{p_2}", from=2-5, to=2-6]
%			\arrow["t", from=2-6, to=3-6]
%			\arrow["{p_1}"', from=2-5, to=3-5]
%			\arrow["s", from=3-5, to=3-6]
%			\arrow[curve={height=-18pt}, Rightarrow, no head, from=1-4, to=2-6]
%			\arrow["i"', curve={height=18pt}, from=1-4, to=3-5]
%			\arrow["{(\id_M,i)}", dashed, from=1-1, to=2-2]
%			\arrow["{p_2}", from=2-2, to=2-3]
%			\arrow["t", from=2-3, to=3-3]
%			\arrow["{p_1}"', from=2-2, to=3-2]
%			\arrow["s", from=3-2, to=3-3]
%			\arrow["i", curve={height=-18pt}, from=1-1, to=2-3]
%			\arrow[curve={height=18pt}, Rightarrow, no head, from=1-1, to=3-2]
%		\end{tikzcd}\]
%	\end{enumerate}
%\end{definition}
%
%\begin{proposition}[{\cite[Proposition 2.3, Proposition 2.12]{nlab:introduction_to_the_adams_spectral_sequence}}]
%	Let $E$ be a flat cellular commutative ring spectrum (\autoref{cellular}, \autoref{flat}). Consider the following data:
%	\begin{enumerate}
%		\item The maps $\eta_L,\eta_R:\pi_\aast(E)\to E_\aast(E)$ which send an element $\alpha:S^a\to E$ to the compositions
%		\[S^a\xrightarrow\alpha E\xrightarrow{\cong}E\wedge S\xrightarrow{E\wedge e}E\wedge E\]
%		and
%		\[S^a\xrightarrow\alpha E\xrightarrow\cong S\wedge E\xrightarrow{e\wedge E}E\wedge E,\]
%		respectively.
%		\item The map $\vare:E_\aast(E)\to\pi_\aast(E)$ sending a class $\alpha:S^a\to E\wedge E$ to the composition
%		\[S^a\xrightarrow\alpha E\wedge E\xrightarrow\mu E\]
%		\item The map $\Psi:E_\aast(E)\to E_\aast(E)\otimes_{\pi_\aast(E)}E_\aast(E)$ which factors as
%		\[E_\aast(E)\to E_\aast(E\wedge E)\xrightarrow\cong E_\aast(E)\otimes_{\pi_\aast(E)}E_\aast(E)\]
%		where the second arrow is the isomorphism prescribed by \autoref{2.2}, and the first arrow sends a class $\alpha:S^a\to E\wedge E$ to the composition
%		\[S^a\xrightarrow\alpha E\wedge E\cong E\wedge S\wedge E\xrightarrow{E\wedge e\wedge E}E\wedge E\wedge E.\]
%		\item The map $c:E_\aast(E)\to E_\aast(E)$ sending a map $\alpha:S^a\to E\wedge E$ to the composition
%		\[S^a\xrightarrow\alpha E\wedge E\xrightarrow\tau E\wedge E,\]
%		where $\tau$ is the symmetry map prescribed by the symmetric monoidal structure on $\SH\scS$.
%	\end{enumerate}
%	Then all of these maps are homomorphisms of $\pi_\aast(S)$-algebras, and they furthermore endow the pair $(E_\aast(E),\pi_\aast(E))$ with the structure of a bigraded commutative Hopf algebroid (\autoref{BCHA})
%\end{proposition}
%\begin{proof}
%	\todo{TODO}
%\end{proof}

%\begin{lemma}\label{tensor_shift_A_graded}
	%Let $R$ be an $A$-graded ring, and suppose we have an $A$-graded $R$-bimodule $M$. Then for all $a\in A$, we have an $A$-graded isomorphism of left $A$-graded $R$-modules
	%\[M\otimes_R R_{*+a}\cong M_{*+a}\]
	%induced by the assignment
	%\[M\times R_{*+a}\to M_{*+a}\]
	%sending $m\in M_b$ and $r\in R_{c+a}$ to $m\cdot r\in M_{b+c+a}$ (where here $M\otimes_RR$ has the structure of a left $A$-graded $R$-module by \autoref{tensor_of_A_graded_is_A_graded}, and $m\cdot r$ denotes the right action of $r$ on $m$).
%\end{lemma}
%\begin{proof}
	%First of all, note that if you ignore the grading then the map $M\times R_{*+a}\to M_{*+a}$ is simply the structure map for the right action of $R$ on $M$. In particular, by the module axioms this map is $R$-balanced, so it does indeed induce an $A$-graded homomorphism of $A$-graded abelian groups $\varphi:M\otimes_RR_{*+a}\to M_{*+a}$. Furthermore, note this map is actually a homomorphism of left $A$-graded $R$-modules, as given $m\in M$ and $r,r'\in R$, we have $r\cdot(m\cdot r')=(r\cdot m)\cdot r'$, since $M$ is a bimodule. Now, to see this map is an isomorphism, it suffices to construct an inverse. Indeed, define the map
	%\[\psi:M_{*+a}\to M\otimes_RR_{*+a}\]
	%to send $m\mapsto m\otimes 1$. First of all note this map is $A$-graded, as given $m\in M_{b+a}$, we have $\psi(m)=m\otimes 1$ has degree $|m|+|1|=|m|=b+a$, by definition of the graded structure on $M\otimes_RR_{*+a}$. Note that it is a homomorphism of left $R$-modules, as given $m,m'\in M$ and $r,r'\in R$ we have
	%\[\psi(rm+r'm')=(rm+r'm')\otimes 1=r(m\otimes1)+r'(m'\otimes1)=r\psi(m)+r'\psi(m').\]
	%Now, to see $\psi$ and $\varphi$ are inverses, note first that given $m\in M_{*+a}$ that
	%\[\varphi(\psi(m))=\varphi(m\otimes 1)=m\cdot1=m,\]
	%and given $m\otimes r\in M\otimes_RR_{*+a}$, 
	%\[\psi(\varphi(m\otimes r))=\psi(m\cdot r)=(m\cdot r)\otimes 1=m\otimes(r\cdot 1)=m\otimes r.\]
%\end{proof}

\end{document}
