\documentclass[../main.tex]{subfiles}

\makeatletter
\renewcommand{\xRightarrow}[2][]{\ext@arrow 0359\Rightarrowfill@{#1}{#2}}
\makeatother

\begin{document}

\subsection{Setup of \texorpdfstring{$\cSH$}{SH}}\label{setup}

In order to construct an abstract version of the Adams spectral sequence, we need to work in some axiomatic version of a stable homotopy category $\cSH$ which acts like the familiar classical stable homotopy category $\hoSp$ (\Cref{classical}) or the motivic stable homotopy category $\SH\scS$ over some base scheme $\scS$ (\Cref{motivic}). 

\begin{definition}\label{defn_compact}
	Let $\cC$ be an additive category with arbitrary (small) coproducts. Then an object $X$ in $\cC$ is \emph{compact} if, for any collection of objects $Y_i$ in $\cC$ indexed by some (small) set $I$, the canonical map
	\[\bigoplus_i\cC(X,Y_i)\to\cC(X,\bigoplus_iY_i)\]
	is an isomorphism of abelian groups. (Explicitly, the above map takes a generator $x\in\cC(X,Y_i)$ to the composition $X\xr xY_i\into\bigoplus_iY_i$.)
\end{definition}

\begin{definition}\label{sub_Picard_grading_defn}
	Given a tensor triangulated category $(\cC,\otimes,S,\Sigma,e,\cD)$ (\autoref{tentri}), a \emph{sub-Picard grading} on $\cC$ is the following data:
	\begin{itemize}
		\item A pointed abelian group $(A,\1)$ along with a homomorphism of pointed groups $h:(A,\1)\to(\Pic\cC,\Sigma S)$, where $\Pic\cC$ is the \emph{Picard group} of isomorphism classes of invertible objects in $\cC$.\footnote{Recall an object $X$ is a symmetric monoidal category is \emph{invertible} if there exists some object $Y$ and an isomorphism $S\cong X\otimes Y$.}
		\item For each $a\in A$, a chosen representative $S^a$ in the isomorphism class $h(a)$ such that each $S^a$ is a compact object (\autoref{defn_compact}) and $S^0=S$.
		\item For each $a,b\in A$, an isomorphism $\phi_{a,b}:S^{a+b}\to S^a\otimes S^b$. This family of isomorphisms is required to be \emph{coherent}, in the following sense:
		\begin{itemize}
			\item For all $a\in A$, we must have that $\phi_{a,0}$ coincides with the right unitor $S^{a}\xrightarrow\cong S^a\otimes S$ and $\phi_{0,a}$ coincides the left unitor $S^a\xrightarrow\cong S\otimes S^a$.
			\item For all $a,b,c\in A$, the following ``associativity diagram'' must commute:
			\[\begin{tikzcd}
				{S^{a+b}\otimes S^c} & {S^{a+b+c}} & {S^a\otimes S^{b+c}} \\
				{(S^a\otimes S^b)\otimes S^c} && {S^a\otimes(S^b\otimes S^c)}
				\arrow["{\phi_{a+b,c}}"', from=1-2, to=1-1]
				\arrow["{\phi_{a,b+c}}", from=1-2, to=1-3]
				\arrow["{S^a\otimes\phi_{b,c}}", from=1-3, to=2-3]
				\arrow["{\phi_{a,b}\otimes S^c}"', from=1-1, to=2-1]
				\arrow["\cong", from=2-1, to=2-3]
			\end{tikzcd}\]
		\end{itemize}
	\end{itemize}
\end{definition}

From now on we fix a monoidal closed tensor triangulated category $(\cSH,\otimes,S,\Sigma,e,\cD)$ with arbitrary (small) (co)products and sub-Picard grading $(A,\1,h,\{S^a\},\{\phi_{a,b}\})$. We also fix an isomorphism $\nu:\Sigma S\xr\cong S^\1$ once and for all. We establish conventions. First, observe the following remark:

\begin{remark}\label{unique_comp_Sas}
	Note that by induction the coherence conditions for the $\phi_{a,b}$'s in the above definition say that given any $a_1,\ldots,a_n\in A$ and $b_1,\ldots,b_m\in A$ such that $a_1+\cdots+a_n=b_1+\cdots+b_m$ and any fixed parenthesizations of $X=S^{a_1}\otimes\cdots\otimes S^{a_b}$ and $Y=S^{b_1}\otimes\cdots\otimes S^{b_m}$, there is a \emph{unique} isomorphism $X\to Y$ that can be obtained by forming formal compositions of products of $\phi_{a,b}$, identities, associators, unitors, and their inverses (but not symmetries).
\end{remark}

In light of this remark, we will usually simply write $\phi$ or even just $\cong$ for any isomorphism that is built by taking compositions of products of $\phi_{a,b}$'s, unitors, associators, identities, and their inverses. Given 
%\[X_1\otimes(X_2\otimes\cdots(X_{n-1}\otimes X_n)).\]
%In particular, given 
an object $X$ and a natural number $n>0$, we write
\[X^n:=\overbrace{X\otimes\cdots\otimes X}^\text{$n$ times}\qquad\text{and}\qquad X^0:=S.\]
We denote the associator, symmetry, left unitor, and right unitor isomorphisms in $\cSH$ by
\[\begin{split}
	\alpha_{X,Y,Z}:(X\otimes Y)\otimes Z&\xrightarrow\cong X\otimes(Y\otimes Z) \\
	\lambda_X:S\otimes  X&\xrightarrow\cong X 
\end{split}\qquad\qquad\begin{split}
	\tau_{X,Y}:X\otimes Y &\xrightarrow\cong Y\otimes X\\
	\rho_X:X\otimes S&\xrightarrow\cong X.
\end{split}\]
Often we will drop the subscripts. Furthermore, by the coherence theorem for symmetric monoidal categories (\cite{MLCoherence}), we will often assume $\alpha$, $\rho$, and $\lambda$ are actual equalities. 

Given some integer $n\in\bZ$, we will write a bold $\n$ to denote the element $n\cdot\1$ in $A$. Note that we can use the isomorphism $\nu:\Sigma S\xr\cong S^\1$ to construct a natural isomorphism $\Sigma\cong S^\1\otimes-$:
\[\Sigma X\xr{\Sigma\lambda_X^{-1}}\Sigma(S\otimes X)\xr{e_{S,X}^{-1}}\Sigma S\otimes X\xr{\nu\otimes X}S^\1\otimes X.\]
The first two arrows are natural in $X$ by definition. The last arrow is natural in $X$ by functoriality of $-\otimes-$. By abuse of notation, we will also use $\nu$ to denote this natural isomorphism. 
%\todo{is this characterization of $\nu$ and $e$ needed?}Furthermore, under this isomorphism, $e_{X,Y}:\Sigma X\otimes Y\xrightarrow\cong \Sigma(X\otimes Y)$ corresponds to the associator, by commutativity of the following diagram:
%% https://q.uiver.app/#q=WzAsOSxbMywwLCIoU15cXDFcXG90aW1lcyBYKVxcb3RpbWVzIFkiXSxbMywyLCJTXlxcMVxcb3RpbWVzKFhcXG90aW1lcyBZKSJdLFsyLDAsIihcXFNpZ21hIFNcXG90aW1lcyBYKVxcb3RpbWVzIFkiXSxbMSwwLCJcXFNpZ21hKFNcXG90aW1lcyBYKVxcb3RpbWVzIFkiXSxbMCwwLCJcXFNpZ21hIFhcXG90aW1lcyBZIl0sWzAsMiwiXFxTaWdtYShYXFxvdGltZXMgWSkiXSxbMiwyLCJcXFNpZ21hIFNcXG90aW1lcyAoWFxcb3RpbWVzIFkpIl0sWzEsMiwiXFxTaWdtYShTXFxvdGltZXMgKFhcXG90aW1lcyBZKSkiXSxbMSwxLCJcXFNpZ21hKChTXFxvdGltZXMgWClcXG90aW1lcyBZKSJdLFswLDEsIlxcYWxwaGEiLDJdLFswLDIsIihcXG51XFxvdGltZXMgWClcXG90aW1lcyBZIiwyXSxbMywyLCJlX3tTLFh9XnstMX1cXG90aW1lcyBZIl0sWzQsMywiXFxTaWdtYVxcbGFtYmRhX1heey0xfVxcb3RpbWVzIFkiXSxbNCw1LCJlX3tYLFl9IiwyXSxbNiwxLCJcXG51XFxvdGltZXMgKFhcXG90aW1lcyBZKSIsMl0sWzcsNiwiZV97UyxYXFxvdGltZXMgWX1eey0xfSIsMl0sWzUsNywiXFxTaWdtYVxcbGFtYmRhX3tYXFxvdGltZXMgWX1eey0xfSIsMl0sWzIsNiwiXFxhbHBoYSJdLFszLDgsImVfe1NcXG90aW1lcyBYLFl9Il0sWzgsNywiXFxTaWdtYVxcYWxwaGEiXSxbOCw1LCJcXFNpZ21hKFxcbGFtYmRhX1hcXG90aW1lcyBZKSIsMV1d
%\[\begin{tikzcd}
%	{\Sigma X\otimes Y} & {\Sigma(S\otimes X)\otimes Y} & {(\Sigma S\otimes X)\otimes Y} & {(S^\1\otimes X)\otimes Y} \\
%	& {\Sigma((S\otimes X)\otimes Y)} \\
%	{\Sigma(X\otimes Y)} & {\Sigma(S\otimes (X\otimes Y))} & {\Sigma S\otimes (X\otimes Y)} & {S^\1\otimes(X\otimes Y)}
%	\arrow["\alpha"', from=1-4, to=3-4]
%	\arrow["{(\nu\otimes X)\otimes Y}"', from=1-4, to=1-3]
%	\arrow["{e_{S,X}^{-1}\otimes Y}", from=1-2, to=1-3]
%	\arrow["{\Sigma\lambda_X^{-1}\otimes Y}", from=1-1, to=1-2]
%	\arrow["{e_{X,Y}}"', from=1-1, to=3-1]
%	\arrow["{\nu\otimes (X\otimes Y)}"', from=3-3, to=3-4]
%	\arrow["{e_{S,X\otimes Y}^{-1}}"', from=3-2, to=3-3]
%	\arrow["{\Sigma\lambda_{X\otimes Y}^{-1}}"', from=3-1, to=3-2]
%	\arrow["\alpha", from=1-3, to=3-3]
%	\arrow["{e_{S\otimes X,Y}}", from=1-2, to=2-2]
%	\arrow["\Sigma\alpha", from=2-2, to=3-2]
%	\arrow["{\Sigma(\lambda_X\otimes Y)}"{description}, from=2-2, to=3-1]
%\end{tikzcd}\]
%Commutativity of the left trapezoid is naturality of $e$. The bottom left triangle commutes by coherence for monoidal categories and functoriality of $\Sigma$. Commutativity of the middle square is axiom TT4 for a tensor triangulated category. Finally, the right square commutes by naturality of $\alpha$.

Given some $a\in A$, we define $\Sigma^a:=S^a\otimes-$ and $\Omega^a:=\Sigma^{-a}=S^{-a}\otimes-$. We specifically define $\Omega:=\Omega^\1$. We say ``the $a^\text{th}$ suspension of $X$'' to denote $\Sigma^aX$. It turns out that $\Sigma^a$ is an autoequivalence of $\cSH$ for each $a\in A$, and furthermore, $\Omega^a$ and $\Sigma^a$ form an adjoint equivalence of $\cSH$ for all $a$ in $A$:

\begin{proposition}\label{Sigma^a,Sigma^-a_adjoint_equiv}
	For each $a\in A$, the isomorphisms
	\[\eta^a_X:X\xr{\lambda_X^{-1}}S\otimes X\xr{\phi_{a,-a}\otimes X}(S^{a}\otimes S^{-a})\otimes X\xr\alpha S^a\otimes(S^{-a}\otimes X)=\Sigma^{a}\Omega^a X\]
	and 
	\[\vare^a_X:\Omega^a\Sigma^a X=S^{-a}\otimes(S^a\otimes X)\xrightarrow{\alpha^{-1}}(S^{-a}\otimes S^a)\otimes X\xrightarrow{\phi_{-a,a}^{-1}\otimes X}S\otimes X\xrightarrow{\lambda_X}X\]
	are natural in $X$, and furthermore, they are the unit and counit respectively of the adjoint autoequivalence $(\Omega^a,\Sigma^a,\eta^a,\vare^a)$ of $\cSH$. In particular, since $\Sigma\cong\Sigma^\1$, $\Omega:=\Omega^\1$ is a left adjoint for $\Sigma$, so that $(\cSH,\Omega,\Sigma,\eta,\vare,\cD)$ is an \emph{adjointly} triangulated category (\autoref{adjointly_triangulated_defn}), where $\eta$ and $\vare$ are the compositions 
	\[\eta:\Id_\cSH\xRightarrow{\eta^\1}\Sigma^\1\Omega\xRightarrow{\nu^{-1}\Omega}\Sigma\Omega\qquad\text{and}\qquad\vare:\Omega\Sigma\xRightarrow{\Omega\nu}\Omega\Sigma^\1\xRightarrow{\vare^\1}\Id_\cSH.\]
\end{proposition}
\begin{proof}
	In this proof, we will freely employ the coherence theorem for monoidal categories (see \cite{MLCoherence}), which essentially tells us that we may assume we are working in a strict monoidal category (i.e., that the associators and unitors and are identities). Then $\eta^a_X$ and $\vare^a_X$ become simply the maps
	\[\eta^a_X:X\xrightarrow{\phi_{a,-a}\otimes X}S^{a}\otimes S^{-a}\otimes X\qquad\text{and}\qquad\vare_X^a:S^{-a}\otimes S^{a}\otimes X\xrightarrow{\phi_{-a,a}^{-1}\otimes X}X.\]
	That these maps are natural in $X$ follows by functoriality of $-\otimes-$. Now, recall that in order to show that these natural isomorphisms form an \emph{adjoint} equivalence, it suffices to show that the natural isomorphisms $\eta^a:\Id_\cSH\Rightarrow\Omega^a\Sigma^a$ and $\vare^a:\Sigma^a\Omega^a\Rightarrow\Id_\cSH$ satisfy one of the two zig-zag identities:
	% https://q.uiver.app/#q=WzAsNixbMCwwLCJcXE9tZWdhXmEiXSxbMSwwLCJcXE9tZWdhXmFcXFNpZ21hXmFcXE9tZWdhXmEiXSxbMSwxLCJcXE9tZWdhXmEiXSxbMiwwLCJcXFNpZ21hXmFcXE9tZWdhXmFcXFNpZ21hXmEiXSxbMywwLCJcXFNpZ21hXmEiXSxbMiwxLCJcXFNpZ21hXmEiXSxbMCwxLCJcXE9tZWdhXmFcXGV0YV5hIl0sWzEsMiwiXFx2YXJlcHNpbG9uXmFcXE9tZWdhXmEiXSxbMyw1LCJcXFNpZ21hXmFcXHZhcmVwc2lsb25eYSIsMl0sWzQsNSwiIiwyLHsibGV2ZWwiOjIsInN0eWxlIjp7ImhlYWQiOnsibmFtZSI6Im5vbmUifX19XSxbMCwyLCIiLDIseyJsZXZlbCI6Miwic3R5bGUiOnsiaGVhZCI6eyJuYW1lIjoibm9uZSJ9fX1dLFs0LDMsIlxcZXRhXmFcXFNpZ21hXmEiLDJdXQ==
	\[\begin{tikzcd}
		{\Omega^a} & {\Omega^a\Sigma^a\Omega^a} & {\Sigma^a\Omega^a\Sigma^a} & {\Sigma^a} \\
		& {\Omega^a} & {\Sigma^a}
		\arrow["{\Omega^a\eta^a}", from=1-1, to=1-2]
		\arrow["{\varepsilon^a\Omega^a}", from=1-2, to=2-2]
		\arrow["{\Sigma^a\varepsilon^a}"', from=1-3, to=2-3]
		\arrow[Rightarrow, no head, from=1-4, to=2-3]
		\arrow[Rightarrow, no head, from=1-1, to=2-2]
		\arrow["{\eta^a\Sigma^a}"', from=1-4, to=1-3]
	\end{tikzcd}\]
	(that it suffices to show only one is~\cite[Lemma~3.2]{nlab:adjoint_equivalence}). We will show that the left is satisfied. Unravelling definitions, we simply wish to show that the following diagram commutes for all $X$ in $\cSH$:
	% https://q.uiver.app/#q=WzAsMyxbMCwwLCJTXnstYX1cXG90aW1lcyBYIl0sWzEsMCwiU157LWF9XFxvdGltZXMgU15hXFxvdGltZXMgU157LWF9XFxvdGltZXMgWCJdLFsxLDEsIlNeey1hfVxcb3RpbWVzIFgiXSxbMCwxLCJTXnstYX1cXG90aW1lcyBcXHBoaV97YSwtYX1cXG90aW1lcyBYIl0sWzEsMiwiXFxwaGlfey1hLGF9XnstMX1cXG90aW1lcyBTXnstYX1cXG90aW1lcyBYIl0sWzAsMiwiIiwyLHsibGV2ZWwiOjIsInN0eWxlIjp7ImhlYWQiOnsibmFtZSI6Im5vbmUifX19XV0=
	\[\begin{tikzcd}
		{S^{-a}\otimes X} & {S^{-a}\otimes S^a\otimes S^{-a}\otimes X} \\
		& {S^{-a}\otimes X}
		\arrow["{S^{-a}\otimes \phi_{a,-a}\otimes X}", from=1-1, to=1-2]
		\arrow["{\phi_{-a,a}^{-1}\otimes S^{-a}\otimes X}", from=1-2, to=2-2]
		\arrow[Rightarrow, no head, from=1-1, to=2-2]
	\end{tikzcd}\]
	Yet this is simply the diagram obtained by applying $-\otimes X$ to the associativity coherence diagram for the $\phi_{a,b}$'s (since $\phi_{a,0}$ and $\phi_{0,a}$ coincide with the unitors, and here we are taking the unitors and associators to be equalities), so it does commute, as desired.
\end{proof}

Given two objects $X$ and $Y$ in $\cSH$, we will denote the hom-abelian group of morphisms from $X$ to $Y$ in $\cSH$ by $[X,Y]$, and the internal hom object by $F(X,Y)$. We can extend the abelian group $[X,Y]$ into an $A$-graded abelian group ${[X,Y]}_*$ by defining ${[X,Y]}_a:=[S^a\otimes X,Y]$. Given an object $X$ in $\cSH$ and some $a\in A$, we can define the abelian group
\[\pi_a(X):=[S^a,X],\]
which we call the \emph{$a^\text{th}$ (stable) homotopy group of $X$}. We write $\pi_*(X)$ for the $A$-graded abelian group $\bigoplus_{a\in A}\pi_a(X)$, so that in particular we have a canonical isomorphism
\[\pi_*(X)=[S^*,X]\cong{[S,X]}_*.\]
Given some other object $E$, we can define the $A$-graded abelian groups $E_*(X)$ and $E^*(X)$ by the formulas
\[E_a(X):=\pi_a(E\otimes X)=[S^a,E\otimes X]\qquad\text{and}\qquad E^a(X):=[X,S^a\otimes E].\]
We refer to the functor $E_*(-)$ as the \emph{homology theory represented by $E$}, or just $E$-homology, and we refer to $E^*(-)$ as the \emph{cohomology theory represented by $E$}, or just $E$-cohomology. 


\subsection{Cellular objects in \texorpdfstring{$\cSH$}{SH}}

One very important class of objects in $\cSH$ are the \emph{cellular} objects. Intuitively, these are the objects that can be built out of spheres via taking coproducts and (co)fibers.

\begin{definition}\label{cellular}
	Define the class of \emph{cellular} objects in $\cSH$ to be the smallest class of objects such that:
	\begin{enumerate}
		\item For all $a\in A$, the $a$-sphere $S^a$ is cellular.
		\item If we have a distinguished triangle
		\[X\to Y\to Z\to\Sigma X\]
		such that two of the three objects $X$, $Y$, and $Z$ are cellular, than the third object is also cellular.
		\item Given a collection of cellular objects $X_i$ indexed by some (small) set $I$, the object $\bigoplus_{i\in I} X_i$ is cellular (recall we have chosen $\cSH$ to have arbitrary coproducts).
	\end{enumerate}
\end{definition}

\begin{lemma}\label{cellular_closed_under_iso}
	Let $X$ and $Y$ be two isomorphic objects in $\cSH$. Then $X$ is cellular iff $Y$ is cellular.
\end{lemma}
\begin{proof}
	Assume we have an isomorphism $f:X\xr\cong Y$ and that $X$ is cellular. Then consider the following commutative diagram
	% https://q.uiver.app/#q=WzAsOCxbMCwwLCJYIl0sWzEsMCwiWSJdLFsyLDAsIjAiXSxbMywwLCJcXFNpZ21hIFgiXSxbMSwxLCJYIl0sWzAsMSwiWCJdLFsyLDEsIjAiXSxbMywxLCJcXFNpZ21hIFgiXSxbMCwxLCJmIl0sWzEsMl0sWzIsM10sWzEsNCwiZl57LTF9Il0sWzAsNSwiIiwyLHsibGV2ZWwiOjIsInN0eWxlIjp7ImhlYWQiOnsibmFtZSI6Im5vbmUifX19XSxbNSw0LCIiLDIseyJsZXZlbCI6Miwic3R5bGUiOnsiaGVhZCI6eyJuYW1lIjoibm9uZSJ9fX1dLFs0LDZdLFs2LDddLFszLDcsIiIsMSx7ImxldmVsIjoyLCJzdHlsZSI6eyJoZWFkIjp7Im5hbWUiOiJub25lIn19fV0sWzIsNiwiIiwxLHsibGV2ZWwiOjIsInN0eWxlIjp7ImhlYWQiOnsibmFtZSI6Im5vbmUifX19XV0=
	\[\begin{tikzcd}
		X & Y & 0 & {\Sigma X} \\
		X & X & 0 & {\Sigma X}
		\arrow["f", from=1-1, to=1-2]
		\arrow[from=1-2, to=1-3]
		\arrow[from=1-3, to=1-4]
		\arrow["{f^{-1}}", from=1-2, to=2-2]
		\arrow[Rightarrow, no head, from=1-1, to=2-1]
		\arrow[Rightarrow, no head, from=2-1, to=2-2]
		\arrow[from=2-2, to=2-3]
		\arrow[from=2-3, to=2-4]
		\arrow[Rightarrow, no head, from=1-4, to=2-4]
		\arrow[Rightarrow, no head, from=1-3, to=2-3]
	\end{tikzcd}\]
	The bottom row is distinguished by axiom TR1 for a triangulated category. Hence since $X$ is cellular, $0$ is also cellular, since the class of cellular objects satisfies two-of-three for distinguished triangles. Furthermore, since the vertical arrows are all isomorphisms, the top row is distinguished as well, by axiom TR0. Thus again by two-of-three, since $X$ and $0$ are cellular, so is $Y$, as desired.
\end{proof}

\begin{lemma}\label{cellular_closed_under_tensor}
	Let $X$ and $Y$ be cellular objects in $\cSH$. Then $X\otimes Y$ is cellular.
\end{lemma}
\begin{proof}
	Let $E$ be a cellular object in $\cSH$, and let $\cE$ be the collection of objects $X$ in $\cSH$ such that $E\otimes X$ is cellular. First of all, suppose we have a distinguished triangle
	\[X\to Y\to Z\to\Sigma X\]
	such that two of three of $X$, $Y$, and $Z$ belong to $\cE$. Then since $\cSH$ is tensor triangulated, we have a distinguished triangle
	\[E\otimes X\to E\otimes Y\to E\otimes Z\to \Sigma(E\otimes X).\]
	Per our assumptions, two of three of $E\otimes X$, $E\otimes Y$, and $E\otimes Z$ are cellular, so that the third is by definition. Thus, all three of $X$, $Y$, and $Z$ belong to $\cE$ if two of them do.

	Second of all, suppose we have a family $X_i$ of objects in $\cE$ indexed by some (small) set $I$, and set $X:=\bigoplus_iX_i$. Then we'd like to show $X$ belongs to $\cE$, i.e., that $E\otimes X$ is cellular. Indeed,
	\[E\otimes X=E\otimes\(\bigoplus_iX_i\)\cong\bigoplus_i(E\otimes X_i),\]
	where the isomorphism is given by the fact that $\cSH$ is monoidal closed, so $E\otimes-$ preserves arbitrary colimits as it is a left adjoint. Per our assumption, since each $E\otimes X_i$ is cellular, the rightmost object is cellular, since the class of cellular objects is closed under taking arbitrary coproducts, by definition. Hence $E\otimes X$ is cellular by \autoref{cellular_closed_under_iso}.

	Finally, we would like to show that each $S^a$ belongs to $\cE$, i.e., that $S^a\otimes E$ is cellular for all $a\in A$. When $E=S^b$ for some $b\in A$, this is clearly true, since $S^b\otimes S^a\cong S^{a+b}$, which is cellular by definition, so that $S^b\otimes S^a$ is cellular by \autoref{cellular_closed_under_iso}. Thus by what we have shown, the class of objects $X$ for which $S^a\otimes X$ is cellular contains every cellular object. Hence in particular $E\otimes S^a\cong S^a\otimes E$ is cellular for all $a\in A$, as desired.
\end{proof}

\begin{lemma}\label{cellular_pi*=0_implies_contractible}
	Let $W$ be a cellular object in $\cSH$ such that $\pi_*(W)=0$. Then $W\cong 0$.
\end{lemma}
\begin{proof}
	Let $\cE$ be the collection of all $X$ in $\cSH$ such that $[\Sigma^nX,W]=0$ for all $n\in\bZ$ (where for $n>0$, $\Sigma^{-n}:=\Omega^n=(S^{-\1})^n\otimes-$). We claim $\cE$ contains every cellular object in $\cSH$. First of all, each $S^a$ belongs to $\cE$, as 
	\[[\Sigma^nS^a,W]\cong[S^\n\otimes S^a,W]\cong[S^{a+\n},W]\leq\pi_*(W)=0.\] 
	Furthermore, suppose we are given a distinguished triangle
	\[X\to Y\to Z\to\Sigma X\]
	such that two of three of $X$, $Y$, and $Z$ belong to $\cE$. By \autoref{dist_tri_LES}, for all $n\in\bZ$ we get an exact sequence
	\[[\Sigma^{n+1}X,W]\to[\Sigma^nZ,W]\to[\Sigma^nY,W]\to[\Sigma^nX,W]\to[\Sigma^{n-1}Z,W].\]
	Clearly if any two of three of $X$, $Y$, and $Z$ belong to $\cE$, then by exactness of the above sequence all three of the middle terms will be zero, so that the third object will belong to $\cE$ as well. Finally, suppose we have a collection of objects $X_i$ in $\cE$ indexed by some small set $I$. Then
	\[\left[\Sigma^n\bigoplus_iX_i,W\right]\cong\left[\bigoplus_i\Sigma^nX_i,W\right]\cong\prod_i[\Sigma^nX_i,W]=\prod_i0=0,\]
	where the first isomorphism follows by the fact that $\Sigma^n$ is apart of an adjoint equivalence (\autoref{Sigma^a,Sigma^-a_adjoint_equiv}), so it preserves arbitrary colimits.

	Thus, by definition of cellularity, $\cE$ contains every cellular object. In particular, $\cE$ contains $W$, so that $[W,W]=0$, meaning in particular that $\id_W=0$, so we have a commutative diagram
	% https://q.uiver.app/#q=WzAsNCxbMCwxLCJXIl0sWzEsMCwiMCJdLFsyLDEsIlciXSxbMywwLCIwIl0sWzAsMV0sWzEsMl0sWzIsM10sWzEsMywiIiwwLHsibGV2ZWwiOjIsInN0eWxlIjp7ImhlYWQiOnsibmFtZSI6Im5vbmUifX19XSxbMCwyLCIiLDAseyJsZXZlbCI6Miwic3R5bGUiOnsiaGVhZCI6eyJuYW1lIjoibm9uZSJ9fX1dXQ==
	\[\begin{tikzcd}
		& 0 && 0 \\
		W && W
		\arrow[from=2-1, to=1-2]
		\arrow[from=1-2, to=2-3]
		\arrow[from=2-3, to=1-4]
		\arrow[Rightarrow, no head, from=1-2, to=1-4]
		\arrow[Rightarrow, no head, from=2-1, to=2-3]
	\end{tikzcd}\]
	Hence the diagonals exhibit isomorphisms between $0$ and $W$, as desired.
\end{proof}

\begin{theorem}\label{whitehead}
	Let $X$ and $Y$ be cellular objects in $\cSH$, and suppose $f:X\to Y$ is a morphism such that $f_*:\pi_*(X)\to\pi_*(Y)$ is an isomorphism. Then $f$ is an isomorphism.
\end{theorem}
\begin{proof}
	By axiom TR2 for a triangulated category (\autoref{triangulated_defn}), we have a distinguished triangle
	\[X\xr fY\xr gC_f\xr h\Sigma X.\]
	First of all, note that by definition since $X$ and $Y$ are cellular, so is $C_f$. Now, we claim $\pi_*(C_f)=0$. Indeed, given $a\in A$, by \autoref{dist_tri_LES} we have the following exact sequence:
	\[[S^a,X]\xr{f_*}[S^a,Y]\xr{g_*}[S^a,C_f]\xr{h_*}[S^a,\Sigma X]\xr{-(\Sigma f)_*}[S^a,\Sigma Y],\]
	where the first arrow is an isomorphism, per our assumption that $f_*$ is an isomorphism. To see the last arrow is an isomorphism, consider the following diagram:
	% https://q.uiver.app/#q=WzAsOCxbMCwwLCJbU15hLFxcU2lnbWEgWF0iXSxbMiwwLCJbU15hLFxcU2lnbWEgWV0iXSxbMCwxLCJbU15hLFNee1xcMX1cXG90aW1lcyBYXSJdLFsyLDEsIltTXmEsU15cXDFcXG90aW1lcyBZXSJdLFswLDIsIltTXnstXFwxfVxcb3RpbWVzIFNee2F9LFhdIl0sWzIsMiwiW1Neey1cXDF9XFxvdGltZXMgU157YX0sWV0iXSxbMCwzLCJbU157YS1cXDF9LFhdIl0sWzIsMywiW1Nee2EtXFwxfSxZXSJdLFswLDEsIihcXFNpZ21hIGYpXyoiXSxbMCwyLCIoXFxudV9YKV8qIiwyXSxbMSwzLCIoXFxudV9ZKV8qIl0sWzIsMywiKFNeXFwxXFxvdGltZXMgZilfKiJdLFs0LDUsImZfKiJdLFsyLDQsIlxcY29uZyIsMl0sWzMsNSwiXFxjb25nIl0sWzQsNiwieyhcXHBoaV97LVxcMSxhfSl9KiIsMl0sWzYsNywiZl8qIl0sWzUsNywieyhcXHBoaV97LVxcMSxhfSl9KiJdXQ==
	\[\begin{tikzcd}
		{[S^a,\Sigma X]} && {[S^a,\Sigma Y]} \\
		{[S^a,S^{\1}\otimes X]} && {[S^a,S^\1\otimes Y]} \\
		{[S^{-\1}\otimes S^{a},X]} && {[S^{-\1}\otimes S^{a},Y]} \\
		{[S^{a-\1},X]} && {[S^{a-\1},Y]}
		\arrow["{(\Sigma f)_*}", from=1-1, to=1-3]
		\arrow["{(\nu_X)_*}"', from=1-1, to=2-1]
		\arrow["{(\nu_Y)_*}", from=1-3, to=2-3]
		\arrow["{(S^\1\otimes f)_*}", from=2-1, to=2-3]
		\arrow["{f_*}", from=3-1, to=3-3]
		\arrow["\cong"', from=2-1, to=3-1]
		\arrow["\cong", from=2-3, to=3-3]
		\arrow["{{(\phi_{-\1,a})}*}"', from=3-1, to=4-1]
		\arrow["{f_*}", from=4-1, to=4-3]
		\arrow["{{(\phi_{-\1,a})}*}", from=3-3, to=4-3]
	\end{tikzcd}\]
	where the middle vertical arrows are the adjunction natural isomorphisms specified by \autoref{Sigma^a,Sigma^-a_adjoint_equiv}. The bottom arrow is an isomorphism per our assumptions, so the top arrow is likewise an isomorphism, as desired. Thus $\imm h_*=\ker -(\Sigma f)_*=0$, and $\ker g_*=\imm f_*=[S^a,Y]$, so that $\ker h_*=\imm g_*=0$. It is only possible that $\ker h_*=\imm h_*=0$ if $[S^a,C_f]=0$. Thus, we have shown $\pi_*(C_f)=0$, and $C_f$ is cellular, so by \autoref{cellular_pi*=0_implies_contractible} there is an isomorphism $C_f\cong 0$. Now consider the following commuting diagram:
	% https://q.uiver.app/#q=WzAsOCxbMCwwLCJYIl0sWzAsMSwiWSJdLFsxLDEsIlkiXSxbMSwwLCJZICJdLFsyLDEsIjAiXSxbMiwwLCJDX2YiXSxbMywwLCJcXFNpZ21hIFgiXSxbMywxLCJcXFNpZ21hIFkiXSxbMCwxLCJmIl0sWzEsMiwiIiwwLHsibGV2ZWwiOjIsInN0eWxlIjp7ImhlYWQiOnsibmFtZSI6Im5vbmUifX19XSxbMCwzLCJmIl0sWzMsMiwiIiwyLHsibGV2ZWwiOjIsInN0eWxlIjp7ImhlYWQiOnsibmFtZSI6Im5vbmUifX19XSxbMiw0XSxbMyw1XSxbNSw2XSxbNiw3LCJcXFNpZ21hIGYiXSxbNCw3XSxbNSw0LCJcXGNvbmciXV0=
	\[\begin{tikzcd}
		X & {Y } & {C_f} & {\Sigma X} \\
		Y & Y & 0 & {\Sigma Y}
		\arrow["f", from=1-1, to=2-1]
		\arrow[Rightarrow, no head, from=2-1, to=2-2]
		\arrow["f", from=1-1, to=1-2]
		\arrow[Rightarrow, no head, from=1-2, to=2-2]
		\arrow[from=2-2, to=2-3]
		\arrow[from=1-2, to=1-3]
		\arrow[from=1-3, to=1-4]
		\arrow["{\Sigma f}", from=1-4, to=2-4]
		\arrow[from=2-3, to=2-4]
		\arrow["\cong", from=1-3, to=2-3]
	\end{tikzcd}\]
	The top row is distinguished by assumption. The bottom row is distinguished by axiom TR2. Then since the middle two vertical arrows are isomorphisms, by \autoref{2-of-3-dist_tri-lemma}, $f$ is an isomorphism as well, as desired.
\end{proof}

\begin{lemma}\label{cellular_idempotent_splits_cellularly}
	Let $e:X\to X$ be an idempotent morphism in $\cSH$, so $e\circ e=e$. Then since $\cSH$ is a triangulated category with arbitrary coproducts, this idempotent splits (\autoref{idempotent_splits_in_tri_cat_with_countable_coproducts}), meaning $e$ factors as
	\[X\xr rY\xr\iota X\]
	for some object $Y$ and morphisms $r$ and $\iota$ with $r\circ\iota=\id_Y$. Then $Y$ is cellular if $X$ is.
\end{lemma}
\begin{proof}
	It is a general categorical fact that the splitting of an idempotent, if it exists, is unique up to unique isomorphism,\footnote{In particular, given an idempotent $e:X\to X$ which splits as $X\xr rY\xr\iota X$, $r$ and $\iota$ are the coequalizer and equalizer, respectively, of $e$ and $\id_X$.} so by \autoref{cellular_closed_under_iso}, it suffices to show that $e$ has some cellular splitting. In \autoref{idempotent_splits_in_tri_cat_with_countable_coproducts}, it is shown that we may take $Y$ to be the homotopy colimit (\autoref{homotopy_colimit_defn}) of the sequence
	\[X\xr eX\xr eX\xr eX\xr e\cdots,\]
	so there is a distinguished triangle
	\[\bigoplus_{i=0}^\infty X\to\bigoplus_{i=0}^\infty X\to Y\to\Sigma\(\bigoplus_{i=0}^\infty X\).\]
	Since $X$ is cellular, by definition $\bigoplus_{i=0}^\infty X$ is as well. Thus by 2-of-3 for distinguished triangles for cellular objects, $Y$ is cellular as desired.
\end{proof}

%\subsection{Miscellaneous facts about \texorpdfstring{$\cSH$}{SH}}\todo{what do I call this subsection?}
%
%%Whether or not this happens is essentially the entire discussion of Dugger's paper \cite{Dugger_2014}, and as it turns out, $\pi_\ast(E)$ is in fact a graded ring provided we can choose these morphisms to be \emph{coherent}, in the following sense:
%%
%%\begin{theorem}[{\cite[Proposition 7.1]{Dugger_2014}}]\label{coherent_existence}
%	%There exists a coherent family of isomorphisms
%	%\[\phi_{a,b}:S^{a+b}\xrightarrow\cong S^a\otimes S^b\]
%	%in the sense of \autoref{coherent_isos},
%	%and in particular, the set of such coherent families is in bijective correspondence with the set of normalized $2$-cocycles $Z^2(A;\mathrm{Aut}(S))_\mathit{norm}$, i.e., the set of functions $\alpha:A\times A\to\mathrm{Aut}(S)$ such that $\alpha(0,0)=\id_S$ and for all $a,b,c\in A$, $\alpha(a+b,c)\cdot\alpha(a,b)=\alpha(b,c)\cdot\alpha(a,b+c)$. 
%%\end{theorem}
%%
%%Thus, from now on we will suppose once and for all we have fixed a coherent family $\{\phi_{a,b}\}_{a,b\in A}$. Such a coherent family has very nice properties, in particular:
%
%%Of course, we get our desired result: $\pi_*(E)$ is indeed an $A$-graded ring if $E$ is a monoid object.
%%
%%\begin{proposition}\label{pi_*E_is_ring_for_E_monoid}
%	%Let $(E,\mu,e)$ be a commutative monoid object in $\cSH$, and consider the multiplication map $\pi_*(E)\times\pi_*(E)\to\pi_*(E)$ which sends classes $x:S^a\to E$ and $y:S^b\to E$ to the composition
%	%\[S^{a+b}\xrightarrow{\phi_{a,b}}S^a\otimes S^b\xrightarrow{x\otimes y}E\otimes E\xrightarrow\mu E.\]
%	%Then this endows $\pi_*(E)$ with the structure of an $A$-graded ring with unit $e\in\pi_0(E)=[S,E]$.
%%\end{proposition}
%%\begin{proof}
%	%See \autoref{pi_*E_is_ring_for_E_monoid_appendix}. 
%%\end{proof}
%%
%%Furthermore, it turns out that if $E$ is a \emph{commutative} monoid object in $\cSH$, then $\pi_\ast(E)$ is ``$A$-graded commutative,'' in the following sense:
%%
%%\begin{proposition}
%	%For all $a,b\in A$ there exists an element $\theta_{a,b}\in\pi_0(S)=[S,S]$ (determined by choice of coherent family $\{\phi_{a,b}\}$) such that given any commutative monoid object $(E,\mu,e)$ in $\cSH$, the $A$-graded ring structure on $\pi_\ast(E)$ (\autoref{pi_*E_is_ring_for_E_monoid}) has a commutativity formula given by
%	%\[x\cdot y=y\cdot x\cdot (e\circ\theta_{a,b})\]
%	%for all $x\in\pi_a(E)$ and $y\in\pi_b(E)$.
%%	
%	%Furthermore, $\theta_{0,a}=\theta_{a,0}=\id_S$ for all $a\in A$, so that if either $x$ or $y$ has degree $0$, $x\cdot y=y\cdot x$.
%%\end{proposition}
%%\begin{proof}
%	%See \autoref{pi_*(E)_is_A-graded_commutative_if_E_is_commutative} and \autoref{theta_a,0=theta_0,a=id_S}.
%%\end{proof}
%%
%%The last ingredient in order to develop the Adams spectral sequence abstractly is a notion of \emph{cellularity} in $\cSH$:
%
%\begin{proposition}
%	Let $(E,\mu,e)$ be a monoid object in $\cSH$ (\autoref{monoid_object}). Then $\pi_*(E)$ is canonically an $A$-graded ring via the assignment $\pi_*(E)\times\pi_*(E)\to\pi_*(E)$ which takes classes $x:S^a\to E$ and $y:S^b\to E$ to the composition
%	\[xy:S^{a+b}\xr{\phi_{a,b}}S^a\otimes S^b\xr{x\otimes y}E\otimes E\xr\mu E.\]
%	In particular, the unit for this ring is the element $e\in[S,E]=\pi_0(E)$.
%\end{proposition}
%\begin{proof}
%	See \autoref{pi_*E_is_ring_for_E_monoid_appendix}.
%\end{proof}
%
%\begin{proposition}\label{module_main}
%	Let $(E,\mu,e)$ be a monoid object in $\cSH$. Then $E_*(-)$ is a functor from $\cSH$ to left $A$-graded $\pi_*(E)$-modules, where given some $X$ in $\cSH$, $E_*(X)$ may be endowed with the structure of a left $A$-graded $\pi_*(E)$-module via the map 
%	\[\pi_*(E)\times E_*(X)\to E_*(X)\]
%	which given $a,b\in A$, sends $x:S^a\to E$ and $y:S^b\to E\otimes X$ to the composition
%	\[x\cdot y:S^{a+b}\cong S^a\otimes S^b\xrightarrow{x\otimes y}E\otimes (E\otimes X)\cong (E\otimes E)\otimes X\xrightarrow{\mu\otimes X}E\otimes X.\]
%	Similarly, the assignment $X\mapsto X_*(E)$ is a functor from $\cSH$ to right $A$-graded $\pi_*(E)$-modules, where the structure map
%	\[X_*(E)\times\pi_*(E)\to X_*(E)\]
%	sends $x:S^a\to X\otimes E$ and $y:S^b\to E$ to the composition
%	\[x\cdot y:S^{a+b}\cong S^a\otimes S^b\xrightarrow{x\otimes y}(X\otimes E)\otimes E\cong X\otimes(E\otimes E)\xrightarrow{X\otimes\mu}X\otimes E.\]
%	Finally, $E_*(E)$ is a $\pi_*(E)$-bimodule, in the sense that the left and right actions of $\pi_*(E)$ are compatible, so that given $y, z\in\pi_*(E)$ and $x\in E_*(E)$, $y\cdot(x\cdot z)=(y\cdot x)\cdot z$.
%\end{proposition}
%\begin{proof}
%	See \autoref{module}.
%\end{proof}
%
%%\begin{lemma}\label{E_homology_suspension_iso_t^a's}
%%	Let $E$ and $X$ be objects in $\cSH$. Then for all $a\in A$, there is an $A$-graded isomorphism of $A$-graded abelian groups
%%	\[t^a_X:E_*(\Sigma^aX)\cong E_{*-a}(X)\]
%%	Furthermore this isomorphism is natural in $X$, and if $E$ is a monoid object in $\cSH$ then it is a natural isomorphism of left $\pi_*(E)$-modules.
%%\end{lemma}
%%\begin{proof}
%%	See \autoref{E_homology_suspension_iso_t^a's_appendix}.
%%\end{proof}
%
%\begin{definition}\label{flat}
%	Given a monoid object $E$ in $\cSH$, we say $E$ is \emph{flat} if the canonical right $\pi_*(E)$-module structure on $E_*(E)$ (see the above proposition) is that of a flat module.
%\end{definition}
%
%%\begin{proposition}
%    %Suppose we have a distinguished triangle
%    %\[X\xr fY\xr gZ\xr h\Sigma X\]
%    %and an object $E$ in $\cSH$. Then there is an associated \emph{$E$-homology long exact sequence} of $A$-graded abelian groups:
%    %% https://q.uiver.app/#q=WzAsOSxbMiwwLCJFX3sqK1xcbisxfShaKSJdLFswLDEsIkVfeyorXFxufShYKSJdLFsxLDEsIkVfeyorXFxufShZKSJdLFsyLDEsIkVfeyorXFxufShaKSJdLFswLDIsIkVfeyorXFxuLVxcMX0oWCkiXSxbMSwyLCJFX3sqK1xcbi1cXDF9KFkpIl0sWzEsMCwiRV97KitcXG4rXFwxfShZKSJdLFswLDAsIlxcY2RvdHMiXSxbMiwyLCJcXGNkb3RzIl0sWzAsMSwiXFxwYXJ0aWFsIiwxXSxbMSwyLCJFX3sqK1xcbn0oZikiLDJdLFsyLDMsIkVfeyorXFxufShnKSJdLFszLDQsIlxccGFydGlhbCIsMV0sWzQsNSwiRV97KitcXG4tXFwxfShmKSIsMl0sWzYsMCwiRV97KitcXG4rXFwxfShnKSJdLFs3LDZdLFs1LDhdXQ==
%    %\[\begin{tikzcd}
%        %\cdots & {E_{*+\n+\1}(Y)} & {E_{*+\n+1}(Z)} \\
%        %{E_{*+\n}(X)} & {E_{*+\n}(Y)} & {E_{*+\n}(Z)} \\
%        %{E_{*+\n-\1}(X)} & {E_{*+\n-\1}(Y)} & \cdots
%        %\arrow["\partial"{description}, from=1-3, to=2-1]
%        %\arrow["{E_{*+\n}(f)}"', from=2-1, to=2-2]
%        %\arrow["{E_{*+\n}(g)}", from=2-2, to=2-3]
%        %\arrow["\partial"{description}, from=2-3, to=3-1]
%        %\arrow["{E_{*+\n-\1}(f)}"', from=3-1, to=3-2]
%        %\arrow["{E_{*+\n+\1}(g)}", from=1-2, to=1-3]
%        %\arrow[from=1-1, to=1-2]
%        %\arrow[from=3-2, to=3-3]
%    %\end{tikzcd}\]
%    %extending infinitely in either direction, where $\partial:E_{*+\n+\1}(Z)\to E_{*+\n}(X)$ takes a class $x:S^{a+\n+\1}\to E\otimes Z$ to the composition
%    %\[S^{a+\n}\cong S^{a+\n+\1}S^{-\1}\xr{x\otimes S^{-\1}} E\otimes Z\otimes S^{-\1}\xrightarrow{E\otimes h\otimes S^{-\1}}E\otimes S^\1\otimes X\otimes S^{-\1}\xr{E\otimes S^\1\otimes\tau}E\otimes S^\1\otimes S^{-\1}\otimes X\xr{E\otimes\phi_{\1,-\1}^{-1}\otimes X}E\otimes X\]
%    %(where here we are ignoring associators and unitors).  Furthermore, if $(E,\mu,e)$ is a monoid object in $\cSH$, then the above sequence is a long exact sequence of left $A$-graded $\pi_*(E)$-modules.
%%\end{proposition}
%%\begin{proof}
%	%First of all, note that by axiom TT3 for a tensor triangulated category, the following triangle is distinguished:
%	%\[E\otimes X\xr{E\otimes f}E\otimes Y\xr{E\otimes g}E\otimes Z\xr{E\otimes'h} \Sigma(E\otimes X)\]
%	%where $E\otimes' h$ is the composition
%	%\[E\otimes Z\xr{E\otimes h}E\otimes\Sigma X\xr{e_{E,X}'}\Sigma(E\otimes X),\]
%	%where $e'_{X,Y}:X\otimes\Sigma Y$ is the natural isomorphism given by the composition
%	%\[X\otimes\Sigma Y\xr{\tau_{X,\Sigma Y}}\Sigma Y\otimes X\xr{e_{Y,X}}\Sigma(Y\otimes X)\xr{\Sigma\tau_{Y,X}}\Sigma(X\otimes Y).\]
%	%(see \autoref{e'_defn}).
%	%Since $\cSH$ is adjointly tensor triangulated (\autoref{Sigma^a,Sigma^-a_adjoint_equiv}), we can apply \autoref{dist_tri_LES} to this distinguished triangle to get a long exact sequence in $\cSH$, and applying $\pi_*=[S^*,-]$ yields the following long exact sequence of $A$-graded abelian groups:
%    %% https://q.uiver.app/#q=WzAsMTksWzQsMCwiXFxwaV8qKFxcT21lZ2Fee24rMX0oRVxcb3RpbWVzIFopKSJdLFswLDIsIlxccGlfKihcXE9tZWdhXm4oRVxcb3RpbWVzIFgpKSJdLFsyLDIsIlxccGlfKihcXE9tZWdhXm4oRVxcb3RpbWVzIFkpKSJdLFs0LDIsIlxccGlfKihcXE9tZWdhXm4oRVxcb3RpbWVzIFopKSJdLFswLDQsIlxccGlfKihcXE9tZWdhXntuLTF9KEVcXG90aW1lcyBYKSkiXSxbMiw0LCJcXGNkb3RzIl0sWzIsMCwiXFxjZG90cyJdLFs0LDQsIlxccGlfKihcXE9tZWdhKEVcXG90aW1lcyBaKSkiXSxbMCw2LCJcXHBpXyooRVxcb3RpbWVzIFgpIl0sWzIsNiwiXFxwaV8qKEVcXG90aW1lcyBZKSJdLFs0LDYsIlxccGlfKihFXFxvdGltZXMgWikiXSxbMCw4LCJcXHBpXyooXFxTaWdtYShFXFxvdGltZXMgWCkpIl0sWzIsOCwiXFxjZG90cyJdLFs0LDgsIlxccGlfKihcXFNpZ21hXntuLTF9KEVcXG90aW1lcyBaKSkiXSxbMCwxMCwiXFxwaV8qKFxcU2lnbWFee259KEVcXG90aW1lcyBYKSkiXSxbMiwxMCwiXFxwaV8qKFxcU2lnbWFee259KEVcXG90aW1lcyBZKSkiXSxbNCwxMCwiXFxwaV8qKFxcU2lnbWFee259KEVcXG90aW1lcyBaKSkiXSxbMCwxMiwiXFxwaV8qKFxcU2lnbWFee24rMX0oRVxcb3RpbWVzIFgpKSJdLFsyLDEyLCJcXGNkb3RzIl0sWzAsMSwiXFxwaV8qKFxcT21lZ2FebihcXHdpZGV0aWxkZXtFXFxvdGltZXMnaH0pKSIsMV0sWzEsMiwiXFxwaV8qKFxcT21lZ2FebihFXFxvdGltZXMgZikpIiwyXSxbMiwzLCJcXHBpXyooXFxPbWVnYV5uKEVcXG90aW1lcyBnKSkiXSxbMyw0LCJcXHBpXyooXFxPbWVnYV57bi0xfShcXHdpZGV0aWxkZXtFXFxvdGltZXMnaH0pKSIsMV0sWzQsNV0sWzYsMF0sWzUsN10sWzcsOCwiXFxwaV8qKFxcd2lkZXRpbGRle0VcXG90aW1lcydofSkiLDFdLFs4LDksIlxccGlfKihFXFxvdGltZXMgZikiLDJdLFs5LDEwLCJcXHBpXyooRVxcb3RpbWVzIGcpIl0sWzEwLDExLCJcXHBpXyooRVxcb3RpbWVzJ2gpIiwxXSxbMTEsMTJdLFsxMiwxM10sWzEzLDE0LCJcXHBpXyooXFxTaWdtYV57bi0xfShFXFxvdGltZXMnaCkpIiwxXSxbMTQsMTUsIlxccGlfKihcXFNpZ21hXm4oRVxcb3RpbWVzIGYpKSIsMl0sWzE1LDE2LCJcXHBpXyooXFxTaWdtYV5uKEVcXG90aW1lcyBnKSkiXSxbMTYsMTcsIlxccGlfKihcXFNpZ21hXntufShFXFxvdGltZXMnaCkpIiwxXSxbMTcsMThdXQ==
%    %\begin{equation}\label{E_LES_diag_1}\begin{tikzcd}
%        %&& \cdots && {\pi_*(\Omega^{n+1}(E\otimes Z))} \\
%        %\\
%        %{\pi_*(\Omega^n(E\otimes X))} && {\pi_*(\Omega^n(E\otimes Y))} && {\pi_*(\Omega^n(E\otimes Z))} \\
%        %\\
%        %{\pi_*(\Omega^{n-1}(E\otimes X))} && \cdots && {\pi_*(\Omega(E\otimes Z))} \\
%        %\\
%        %{\pi_*(E\otimes X)} && {\pi_*(E\otimes Y)} && {\pi_*(E\otimes Z)} \\
%        %\\
%        %{\pi_*(\Sigma(E\otimes X))} && \cdots && {\pi_*(\Sigma^{n-1}(E\otimes Z))} \\
%        %\\
%        %{\pi_*(\Sigma^{n}(E\otimes X))} && {\pi_*(\Sigma^{n}(E\otimes Y))} && {\pi_*(\Sigma^{n}(E\otimes Z))} \\
%        %\\
%        %{\pi_*(\Sigma^{n+1}(E\otimes X))} && \cdots
%        %\arrow["{\pi_*(\Omega^n(\widetilde{E\otimes'h}))}"{description}, from=1-5, to=3-1]
%        %\arrow["{\pi_*(\Omega^n(E\otimes f))}"', from=3-1, to=3-3]
%        %\arrow["{\pi_*(\Omega^n(E\otimes g))}", from=3-3, to=3-5]
%        %\arrow["{\pi_*(\Omega^{n-1}(\widetilde{E\otimes'h}))}"{description}, from=3-5, to=5-1]
%        %\arrow[from=5-1, to=5-3]
%        %\arrow[from=1-3, to=1-5]
%        %\arrow[from=5-3, to=5-5]
%        %\arrow["{\pi_*(\widetilde{E\otimes'h})}"{description}, from=5-5, to=7-1]
%        %\arrow["{\pi_*(E\otimes f)}"', from=7-1, to=7-3]
%        %\arrow["{\pi_*(E\otimes g)}", from=7-3, to=7-5]
%        %\arrow["{\pi_*(E\otimes'h)}"{description}, from=7-5, to=9-1]
%        %\arrow[from=9-1, to=9-3]
%        %\arrow[from=9-3, to=9-5]
%        %\arrow["{\pi_*(\Sigma^{n-1}(E\otimes'h))}"{description}, from=9-5, to=11-1]
%        %\arrow["{\pi_*(\Sigma^n(E\otimes f))}"', from=11-1, to=11-3]
%        %\arrow["{\pi_*(\Sigma^n(E\otimes g))}", from=11-3, to=11-5]
%        %\arrow["{\pi_*(\Sigma^{n}(E\otimes'h))}"{description}, from=11-5, to=13-1]
%        %\arrow[from=13-1, to=13-3]
%    %\end{tikzcd}\end{equation}
%    %where $\wt{E\otimes'h}:\Omega(E\otimes Z)\to E\otimes X$ is the left adjoint of $E\otimes'h:E\otimes Z\to\Sigma(E\otimes X)$ (note here $\Sigma^n$ and $\Omega^n$ denote $n$ applications of $\Sigma=\Sigma^\1$ and $\Omega=\Omega^\1$, respectively. They are not the same as $\Sigma^\n$ or $\Omega^\n$). Note that for all $a\in A$ and $X$ in $\cSH$, we have an isomorphism $s_X^{a,n}:\pi_*((\Sigma^a)^nX)\cong\pi_{*-n\cdot a}(X)$ given by the composition
%    %\[[S^*,\overbrace{S^a\otimes\cdots\otimes S^a}^\text{$n$ times}\otimes X]\xrightarrow{(\phi\otimes X)_*}[S^*,S^{n\cdot a}\otimes X]\cong[S^{-n\cdot a}\otimes S^*,X]\xrightarrow{(\phi_{-n\cdot a,*})^*}[S^{*-n\cdot a},X],\]
%    %where the $\phi$ in the first arrow stands for the unique isomorphism $(S^a)^n\cong S^{n\cdot a}$ that can be obtained by composing copies of products of $\phi_{a,b}$'s, associators, unitors, and their inverses, and the middle isomorphism is the adjunction given by \autoref{Sigma^a,Sigma^-a_adjoint_equiv}. It is straightforward to see that each of the above arrows is natural in $X$ (the first is natural by functoriality of $-\otimes-$, the middle is natural as it is the isomorphism given by an adjunction, and the last is clearly natural simply by unravelling definitions). Thus for all $n>0$ we have isomorphisms of exact sequences
%    %% https://q.uiver.app/#q=WzAsMTMsWzAsMCwiXFxwaV8qKFxcT21lZ2FebihFXFxvdGltZXMgWCkpIl0sWzEsMCwiXFxwaV8qKFxcT21lZ2FebihFXFxvdGltZXMgWSkpIl0sWzIsMCwiXFxwaV8qKFxcT21lZ2FebihFXFxvdGltZXMgWikpIl0sWzAsMSwiRV97KitcXG59KFgpIl0sWzEsMSwiRV97KitcXG59KFkpIl0sWzIsMSwiRV97KitcXG59KFopIl0sWzEsMiwiXFx0ZXh0e2FuZH0iXSxbMCwzLCJcXHBpXyooXFxTaWdtYV5uKEVcXG90aW1lcyBYKSkiXSxbMSwzLCJcXHBpXyooXFxTaWdtYV5uKEVcXG90aW1lcyBZKSkiXSxbMiwzLCJcXHBpXyooXFxTaWdtYV5uKEVcXG90aW1lcyBaKSkiXSxbMCw0LCJFX3sqLVxcbn0oWCkiXSxbMSw0LCJFX3sqLVxcbn0oWSkiXSxbMiw0LCJFX3sqLVxcbn0oWikiXSxbMCwxLCJcXHBpXyooXFxPbWVnYV5uKEVcXG90aW1lcyBmKSkiXSxbMSwyLCJcXHBpXyooXFxPbWVnYV5uKEVcXG90aW1lcyBnKSkiXSxbMCwzLCJzXnstXFwxLG59IiwyXSxbMyw0LCJFX3srXFxufShmKSIsMl0sWzQsNSwiRV97KitcXG59KGcpIiwyXSxbMSw0LCJzXnstXFwxLG59IiwyXSxbMiw1LCJzXnstXFwxLG59IiwyXSxbNyw4LCJcXHBpXyooXFxTaWdtYV5uKEVcXG90aW1lcyBmKSkiXSxbOCw5LCJcXHBpXyooXFxTaWdtYV5uKEVcXG90aW1lcyBnKSkiXSxbNywxMCwic157XFwxLG59IiwyXSxbMTAsMTEsIkVfeyotXFxufShmKSIsMl0sWzExLDEyLCJFX3sqLVxcbn0oZykiLDJdLFs4LDExLCJzXntcXDEsbn0iLDJdLFs5LDEyLCJzXntcXDEsbn0iLDJdXQ==
%    %\[\begin{tikzcd}
%        %{\pi_*(\Omega^n(E\otimes X))} & {\pi_*(\Omega^n(E\otimes Y))} & {\pi_*(\Omega^n(E\otimes Z))} \\
%        %{E_{*+\n}(X)} & {E_{*+\n}(Y)} & {E_{*+\n}(Z)} \\
%        %& {\text{and}} \\
%        %{\pi_*(\Sigma^n(E\otimes X))} & {\pi_*(\Sigma^n(E\otimes Y))} & {\pi_*(\Sigma^n(E\otimes Z))} \\
%        %{E_{*-\n}(X)} & {E_{*-\n}(Y)} & {E_{*-\n}(Z)}
%        %\arrow["{\pi_*(\Omega^n(E\otimes f))}", from=1-1, to=1-2]
%        %\arrow["{\pi_*(\Omega^n(E\otimes g))}", from=1-2, to=1-3]
%        %\arrow["{s^{-\1,n}}"', from=1-1, to=2-1]
%        %\arrow["{E_{+\n}(f)}"', from=2-1, to=2-2]
%        %\arrow["{E_{*+\n}(g)}"', from=2-2, to=2-3]
%        %\arrow["{s^{-\1,n}}"', from=1-2, to=2-2]
%        %\arrow["{s^{-\1,n}}"', from=1-3, to=2-3]
%        %\arrow["{\pi_*(\Sigma^n(E\otimes f))}", from=4-1, to=4-2]
%        %\arrow["{\pi_*(\Sigma^n(E\otimes g))}", from=4-2, to=4-3]
%        %\arrow["{s^{\1,n}}"', from=4-1, to=5-1]
%        %\arrow["{E_{*-\n}(f)}"', from=5-1, to=5-2]
%        %\arrow["{E_{*-\n}(g)}"', from=5-2, to=5-3]
%        %\arrow["{s^{\1,n}}"', from=4-2, to=5-2]
%        %\arrow["{s^{\1,n}}"', from=4-3, to=5-3]
%    %\end{tikzcd}\]
%    %It remains to show that the following diagram commutes:
%%\end{proof}

\subsection{Monoid objects in \texorpdfstring{$\cSH$}{SH}}

Many of the proofs in this section are quite technical and not very euclidiating, so we direct the reader to the appendix for most proofs.
To start with, recall the following definition:

\begin{definition}\label{monoid_object}
    Let $(\cC,\otimes,S)$ be a symmetric monoidal category with left unitor, right unitor, associator, and symmetry isomorphisms $\lambda$, $\rho$, $\alpha$, and $\tau$, respectively. A \emph{monoid object} $(E,\mu,e)$ is an object $E$ in $\cC$ along with a multiplication morphism $\mu:E\otimes E\to E$ and a unit map $e:S\to E$ such that the following diagrams commute:
	% https://q.uiver.app/#q=WzAsOSxbMSwwLCJFXFxvdGltZXMgRSJdLFsxLDEsIkUiXSxbMiwwLCJTXFxvdGltZXMgRSJdLFswLDAsIkVcXG90aW1lcyBTIl0sWzMsMCwiKEVcXG90aW1lcyBFKVxcb3RpbWVzIEUiXSxbMywxLCJFXFxvdGltZXMoRVxcb3RpbWVzIEUpIl0sWzQsMSwiRVxcb3RpbWVzIEUiXSxbNSwxLCJFIl0sWzUsMCwiRVxcb3RpbWVzIEUiXSxbMCwxLCJcXG11Il0sWzIsMCwiZVxcb3RpbWVzIEUiLDJdLFszLDAsIkVcXG90aW1lcyBlIl0sWzIsMSwiXFxsYW1iZGFfRSJdLFszLDEsIlxccmhvX0UiLDJdLFs0LDUsIlxcYWxwaGEiLDJdLFs1LDYsIkVcXG90aW1lc1xcbXUiXSxbNiw3LCJcXG11Il0sWzQsOCwiXFxtdVxcb3RpbWVzIEUiXSxbOCw3LCJcXG11Il1d
	\[\begin{tikzcd}
		{E\otimes S} & {E\otimes E} & {S\otimes E} & {(E\otimes E)\otimes E} && {E\otimes E} \\
		& E && {E\otimes(E\otimes E)} & {E\otimes E} & E
		\arrow["\mu", from=1-2, to=2-2]
		\arrow["{e\otimes E}"', from=1-3, to=1-2]
		\arrow["{E\otimes e}", from=1-1, to=1-2]
		\arrow["{\lambda_E}", from=1-3, to=2-2]
		\arrow["{\rho_E}"', from=1-1, to=2-2]
		\arrow["\alpha"', from=1-4, to=2-4]
		\arrow["E\otimes\mu", from=2-4, to=2-5]
		\arrow["\mu", from=2-5, to=2-6]
		\arrow["{\mu\otimes E}", from=1-4, to=1-6]
		\arrow["\mu", from=1-6, to=2-6]
	\end{tikzcd}\]
    The first diagram expresses unitality, while the second expressed associativity. If in addition the following diagram commutes, 
    % https://q.uiver.app/#q=WzAsMyxbMCwwLCJFXFxvdGltZXMgRSJdLFsyLDAsIkVcXG90aW1lcyBFIl0sWzEsMSwiRSJdLFswLDEsIlxcdGF1Il0sWzEsMiwiXFxtdSJdLFswLDIsIlxcbXUiLDJdXQ==
    \[\begin{tikzcd}
        {E\otimes E} && {E\otimes E} \\
        & E
        \arrow["\tau", from=1-1, to=1-3]
        \arrow["\mu", from=1-3, to=2-2]
        \arrow["\mu"', from=1-1, to=2-2]
    \end{tikzcd}\]
    then we say $(E,\mu,e)$ is a \emph{commutative} monoid object.
\end{definition}

Monoid objects in $\cSH$ will be the focus of the rest of this paper. The most important example of a monoid object in $\cSH$ is the unit $S$, which has multiplication map $\phi_{0,0}^{-1}=\lambda_S=\rho_S:S\otimes S\to S$ and unit map $\id_S:S\to S$.

\begin{proposition}[\autoref{pi_*E_is_ring_for_E_monoid}]\label{pi_*E_is_ring_for_E_monoid_main}
	Let $(E,\mu,e)$ be a monoid object in $\cSH$, then $\pi_*(E)$ is a ring under the multiplication map $\pi_*(E)\times\pi_*(E)\to\pi_*(E)$ which sends classes $x:S^a\to E$ and $y:S^b\to E$ to the composition
	\[xy:S^{a+b}\xr{\phi_{a,b}}S^a\otimes S^b\xr{x\otimes y}E\otimes E\xr\mu E.\]
	In particular, the unit of this ring is $e\in\pi_0(E)=[S,E]$.
\end{proposition}
 
We call the ring $\pi_*(S)$ the \emph{stable homotopy ring}.

\begin{proposition}[\autoref{pi_*(E)_is_A-graded_commutative_if_E_is_commutative}]\label{pi_*(E)_is_A-graded_commutative_if_E_is_commutative_main}
	For all $a,b\in A$ there exists an element $\theta_{a,b}\in\pi_0(S)=[S,S]$ such that given any commutative monoid object $(E,\mu,e)$ in $\cSH$, the $A$-graded ring structure on $\pi_\ast(E)$ (\autoref{pi_*E_is_ring_for_E_monoid_main}) has a commutativity formula given by
	\[x\cdot y=y\cdot x\cdot (e\circ\theta_{a,b})\]
	for all $x\in\pi_a(E)$ and $y\in\pi_b(E)$. In particular, $\theta_{a,b}\in\mathrm{Aut}(S)$ is the composition
	\[S\xr{\cong}S^{-a-b}\otimes S^a\otimes S^b\xr{S^{-a-b}\otimes\tau}S^{-a-b}\otimes S^b\otimes S^a\xr\cong S,\]
	where the outermost maps are the unique maps specified by \autoref{unique_comp_Sas}.
\end{proposition}

\begin{proposition}
	The $\theta_{a,b}$'s described in \autoref{pi_*(E)_is_A-graded_commutative_if_E_is_commutative_main} satisfy the following properties for all $a,b,c\in A$:
	\begin{enumerate}
		\item $\theta_{a,b}\circ\theta_{c,d}=\theta_{c,d}\cdot\theta_{a,b}$ (where $\cdot$ denotes the product in $\pi_*(S)$ given in \autoref{pi_*E_is_ring_for_E_monoid_main}),
		\item $\theta_{a,0}=\theta_{0,a}=\id_S$,
		\item $\theta_{a,b}\cdot\theta_{b,a}=\id_S$,
		\item $\theta_{a,b}\cdot\theta_{a,c}=\theta_{a,b+c}$ and $\theta_{b,a}\cdot\theta_{c,a}=\theta_{b+c,a}$,
	\end{enumerate}
\end{proposition}
\begin{proof}
	\todo{FINISH}
\end{proof}

\end{document}
