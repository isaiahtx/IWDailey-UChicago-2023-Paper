\documentclass[../main.tex]{subfiles}
\begin{document}

In this section, we briefly touch on some future directions in which one could carry on this work in our general setting.

\begin{itemize}
    \item One could weaken the cellularity conditions required for the characterization of the $E_2$ page of the $E$-Adams spectral sequence for ${[X,Y]}_*$ (\autoref{E_2_page_characterization}) by instead proving a version of \autoref{Kunneth_theorem} for \emph{$E$-cellular} objects, in the sense of the definition given on pg.\ 21 in the paper \cite{DugIsak}.
    \item One could set up a cohomological version of the $E$-Adams spectral sequence in $\cSH$, as in \cite[\S3]{DugIsak}. In order to show that it agrees with the homological $E$-Adams spectral sequence we have constructed, one would need to develop some sort anaologue in $\cSH$ of the finiteness condition given in \cite[Definitions 2.11 \& 2.12]{DugIsak}.
    \item Much more could be said about properties of the Adams spectral sequence in $\cSH$, particularly convergence. Are there connectivity conditions we can place on $\pi_{*}(Y)$ which guarantee convergence of the $E$-Adams spectral sequence for ${[X,Y]}_*$ to ${[X,Y^\wedge_E]}_*$?
    \item Under what conditions does a morphisms in $\cSH$ induce a homomorphism of $E$-Adams spectral sequences, as in \cite[Proposition 11.4.1]{Rognes_SSeq}? Similarly, how should one define an $E$-Adams resolution in $\cSH$ in a way such that the $E$-Adams spectral sequence is independent of the choice of resolution?
    \item One could define products on the $E$-Adams spectral sequence in $\cSH$ as in \cite[Sections 11.7 \& 11.8]{Rognes_SSeq}.
    \item Given a prime $p$, one could define the  ``mod-$p$ Moore object'' in $\cSH$ to be the cofiber of the multiplication-by-$p$ map $S\xr pS$. Using this object, one could define the $p$-completion $Y_p^\wedge$ of an object $Y$ in $\cSH$, as in \cite[Definition 11.5.4]{Rognes_SSeq}. Under which conditions is the canonical map $Y\to Y_p^\wedge$ an isomorphism in $\cSH$, and how are $\pi_*(Y)$ and and $\pi_*(Y^\wedge_p)$ related?
    \item There is a symmetric monoidal realization functor from the motivic stable homotopy category $\SH\bC$ over $\Spec\bC$ to the classical stable homotopy category $\hoSp$ given by Betti realization. Under the standard grading anticommutativity conventions for these categories (see Examples \ref{classical_SH_grading_conventions} and \ref{motivic_SH_grading_conventions}), it further induces a ring homomorphism from the motivic stable homotopy ring (which is $\bZ^2$-graded) to the classical stable homotopy ring (which is $\bZ$-graded) which sends homogeneous elements of degree $(p,q)$ to elements of degree $p$. Furthermore, this functor induces a homomorphism of spectral sequence from the $\bC$-motivic Adams spectral sequence to the classical Adams spectral sequence. We refer the reader to the Levine's paper \cite{Levine_2013} for a more in-depth exposition of this story. Similarly, there are realization functors from $\SH\bC$ to the $C_2$-equivariant stable homotopy category with similar properties (see \cite[Remarks 3 \& 4]{DDIO}). This motivates the idea of a ``homomorphism of tensor-triangulated categories with sub-Picard grading'', which should be functors which are in some sense compatible with the structure described in \autoref{sub_Picard_grading_defn}. Furthermore, these functors should induce homomorphisms of homotopy groups and of Adams spectral sequences.
    \item More work could be done to examine the graded anticommutativity properties of homotopy rings in the $G$-equvariant stable homotopy category, using the language and methods of \Cref{section:monoid_in_SH} and \cite{DDIO}.
    \item In the classical, equivariant, and motivic stable homotopy categories, given an abelian group $G$, there is an associated Eilenberg-MacLane spectrum $HG$, and this assignment yields a monoidal functor from abelian groups to the stable homotopy category. Given a monoidal functor $\Ab\to\cSH$ sending $G$ to $HG$, what can we say about the $HG$-Adams spectral sequence? when is the $H\bF_p$-completion.
    \item In the classical, equivariant, and motivic stable homotopy categories, the grading comes from an abelian group $A$ which also happens to be a ring ($A=\bZ,\bZ^2,RO(G)$). Furthermore, in the classical and motivic stable homotopy categories, this additional ring structure comes into play with regards to the graded anticommutativity properties of $\pi_*(S)$ (e.g., in the classical case, we have a commutativity formula on $\pi_*(S)$ given by $x\cdot y=y\cdot x\cdot(-1)^{|x|\cdot|y|}$). What can be said about a tensor triangulated category with sub-Picard grading coming from a ring, and can we impose any additional conditions on the sub-Picard grading which allows us to say something about the graded anticommutativity properties of $\pi_*(S)$?
    \item What conditions can we impose on a flat, cellular commtative monoid object $(E,\mu,e)$ in $\cSH$ which give us a filtration that allows us to create a May spectral sequence for computing the $E_2$ page of the $E$-Adams spectral sequence?
\end{itemize}

As the above exhibits, we have really only scratched the surface of what is possible in this setting, and much more could be done to develop the theory of tensor triangulated categories with sub-Picard grading.

\end{document}
