\documentclass[../main.tex]{subfiles}
\begin{document}

In this appendix, we fix a symmetric monoidal category $(\cC,\otimes,S)$ with left unitor, right unitor, associator, and symmetry isomorphisms $\lambda$, $\rho$, $\alpha$, and $\tau$, respectively.

\subsection{Monoid objects in a symmetric monoidal category}\label{monoid_objects_subsection_appendix}

\begin{definition}\label{monoid_object}
    A \emph{monoid object} $(E,\mu,e)$ is an object $E$ in $\cC$ along with a multiplication morphism $\mu:E\otimes E\to E$ and a unit map $e:S\to E$ such that the following diagrams commute:
	% https://q.uiver.app/#q=WzAsOSxbMSwwLCJFXFxvdGltZXMgRSJdLFsxLDEsIkUiXSxbMiwwLCJTXFxvdGltZXMgRSJdLFswLDAsIkVcXG90aW1lcyBTIl0sWzMsMCwiKEVcXG90aW1lcyBFKVxcb3RpbWVzIEUiXSxbMywxLCJFXFxvdGltZXMoRVxcb3RpbWVzIEUpIl0sWzQsMSwiRVxcb3RpbWVzIEUiXSxbNSwxLCJFIl0sWzUsMCwiRVxcb3RpbWVzIEUiXSxbMCwxLCJcXG11Il0sWzIsMCwiZVxcb3RpbWVzIEUiLDJdLFszLDAsIkVcXG90aW1lcyBlIl0sWzIsMSwiXFxsYW1iZGFfRSJdLFszLDEsIlxccmhvX0UiLDJdLFs0LDUsIlxcYWxwaGEiLDJdLFs1LDYsIkVcXG90aW1lc1xcbXUiXSxbNiw3LCJcXG11Il0sWzQsOCwiXFxtdVxcb3RpbWVzIEUiXSxbOCw3LCJcXG11Il1d
	\[\begin{tikzcd}
		{E\otimes S} & {E\otimes E} & {S\otimes E} & {(E\otimes E)\otimes E} && {E\otimes E} \\
		& E && {E\otimes(E\otimes E)} & {E\otimes E} & E
		\arrow["\mu", from=1-2, to=2-2]
		\arrow["{e\otimes E}"', from=1-3, to=1-2]
		\arrow["{E\otimes e}", from=1-1, to=1-2]
		\arrow["{\lambda_E}", from=1-3, to=2-2]
		\arrow["{\rho_E}"', from=1-1, to=2-2]
		\arrow["\alpha"', from=1-4, to=2-4]
		\arrow["E\otimes\mu", from=2-4, to=2-5]
		\arrow["\mu", from=2-5, to=2-6]
		\arrow["{\mu\otimes E}", from=1-4, to=1-6]
		\arrow["\mu", from=1-6, to=2-6]
	\end{tikzcd}\]
    The first diagram expresses unitality, while the second expressed associativity. If in addition the following diagram commutes, 
    % https://q.uiver.app/#q=WzAsMyxbMCwwLCJFXFxvdGltZXMgRSJdLFsyLDAsIkVcXG90aW1lcyBFIl0sWzEsMSwiRSJdLFswLDEsIlxcdGF1Il0sWzEsMiwiXFxtdSJdLFswLDIsIlxcbXUiLDJdXQ==
    \[\begin{tikzcd}
        {E\otimes E} && {E\otimes E} \\
        & E
        \arrow["\tau", from=1-1, to=1-3]
        \arrow["\mu", from=1-3, to=2-2]
        \arrow["\mu"', from=1-1, to=2-2]
    \end{tikzcd}\]
    then we say $(E,\mu,e)$ is a \emph{commutative} monoid object.
\end{definition}

\begin{example}\label{unit_is_monoid}
    The object $S$ is a monoid object, with multiplication map $\rho_S=\lambda_S:S\otimes S\to S$ and unit $\id_S:S\to S$.
\end{example}

\begin{definition}\label{Mon_C,CMon_C}
	Given two monoid objects $(E_1,\mu_1,e_1)$ and $(E_2,\mu_2,e_2)$ in a symmetric monoidal category $(\cC,\otimes,S)$, a \emph{monoid homomorphism} from $E_1$ to $E_2$ is a morphism $f:E_1\to E_2$ in $\cC$ such that the following diagrams commute:
	% https://q.uiver.app/#q=WzAsNyxbMCwwLCJFXzFcXG90aW1lcyBFXzEiXSxbMSwwLCJFXzJcXG90aW1lcyBFXzIiXSxbMSwxLCJFXzIiXSxbMCwxLCJFXzEiXSxbMywwLCJTIl0sWzIsMSwiRV8xIl0sWzQsMSwiRV8yIl0sWzAsMSwiZlxcb3RpbWVzIGYiXSxbMSwyLCJcXG11XzIiXSxbMCwzLCJcXG11XzEiLDJdLFszLDIsImYiXSxbNCw1LCJlXzEiLDJdLFs1LDYsImYiXSxbNCw2LCJlXzIiXV0=
	\[\begin{tikzcd}
		{E_1\otimes E_1} & {E_2\otimes E_2} && S \\
		{E_1} & {E_2} & {E_1} && {E_2}
		\arrow["{f\otimes f}", from=1-1, to=1-2]
		\arrow["{\mu_2}", from=1-2, to=2-2]
		\arrow["{\mu_1}"', from=1-1, to=2-1]
		\arrow["f", from=2-1, to=2-2]
		\arrow["{e_1}"', from=1-4, to=2-3]
		\arrow["f", from=2-3, to=2-5]
		\arrow["{e_2}", from=1-4, to=2-5]
	\end{tikzcd}\]
	It is straightforward to show that $\id_{E_1}$ is a homomorphism of monoid objects from $E_1$ to itself, and that the composition of monoid homomorphisms is still a monoid homomorphism. Thus, we have categories $\Mon_\cC$ and $\CMon_\cC$ of monoid objects and commutative monoid objects in $\cC$, respectively, with monoid homomorphisms between them.
\end{definition}

\begin{lemma}\label{product_of_monoids_is_monoid}
	Given two monoid objects $(E_1,\mu_1,e_1)$ and $(E_2,\mu_2,e_2)$ in a symmetric monoidal category $(\cC,\otimes,S)$, their tensor product $E_1\otimes E_2$ canonically becomes a monoid object in $\cC$ with unit map
	\[e:S\xr\cong S\otimes S\xr{e_1\otimes e_2}E_1\otimes E_2\]
	and multiplication map
	\[\mu:E_1\otimes E_2\otimes E_1\otimes E_2\xr{E_1\otimes\tau\otimes E_2}E_1\otimes E_1\otimes E_2\otimes E_2\xr{\mu_1\otimes\mu_2}E_1\otimes E_2\]
	(where here we are suppressing the associators from the notation). If in addition $(E_1,\mu_1,e_1)$ and $(E_2,\mu_2,e_2)$ are \emph{commutative} monoid objects, then $(E_1\otimes E_2,\mu,e)$ is as well.
\end{lemma}
\begin{proof}
    \todo{todo}
\end{proof}

\begin{lemma}\label{product_of_3+_monoids_no_ambiguity}
	Given monoid objects $(E_i,\mu_i,e_i)$ for $i=1,2,3$ in a symmetric monoidal category $\cC$, the associator $(E_1\otimes E_2)\otimes E_3\xr\cong E_1\otimes(E_2\otimes E_3)$ is an isomorphism of monoid objects. In other words, up to associativity, given a collection of monoid objects $E_1,\ldots,E_n$ in $\cC$, there is no ambiguity when talking about their tensor product $E_1\otimes\cdots\otimes E_n$ as a monoid object.
\end{lemma}
\begin{proof}
	Clearly, up to associativity, $(E_1\otimes E_2)\otimes E_3$ and $E_1\otimes (E_2\otimes E_3)$ have the same unit map $S\xr{e_1\otimes e_2\otimes e_3}E_1\otimes E_2\otimes E_3$. Thus, it remains to show that they have the same product map, up to associativity. To see this, consider the following diagram, where we've passed to a symmetric strict monoidal category:
	% https://q.uiver.app/#q=WzAsOSxbMCwwLCJFXzFcXG90aW1lcyhFXzJcXG90aW1lcyBFXzMpXFxvdGltZXMgRV8xXFxvdGltZXMoRV8yXFxvdGltZXMgRV8zKSJdLFswLDEsIkVfMVxcb3RpbWVzIEVfMVxcb3RpbWVzIEVfMlxcb3RpbWVzIEVfM1xcb3RpbWVzIEVfMlxcb3RpbWVzIEVfMyJdLFsyLDEsIkVfMVxcb3RpbWVzIEVfMlxcb3RpbWVzIEVfMVxcb3RpbWVzIEVfMlxcb3RpbWVzIEVfM1xcb3RpbWVzIEVfMyJdLFswLDIsIkVfezF9XFxvdGltZXMgRV8yXFxvdGltZXMgRV8yXFxvdGltZXMgRV8zXFxvdGltZXMgRV8zIl0sWzAsMywiRV97MX1cXG90aW1lcyBFX3syfVxcb3RpbWVzIEVfezN9Il0sWzIsMiwiRV8xXFxvdGltZXMgRV8xXFxvdGltZXMgRV8yXFxvdGltZXMgRV8yXFxvdGltZXMgRV97M30iXSxbMiwzLCJFX3sxfVxcb3RpbWVzIEVfezJ9XFxvdGltZXMgRV97M30iXSxbMSwyLCJFXzFcXG90aW1lcyBFXzFcXG90aW1lcyBFXzJcXG90aW1lcyBFXzJcXG90aW1lcyBFXzNcXG90aW1lcyBFXzMiXSxbMiwwLCIoRV8xXFxvdGltZXMgRV8yKVxcb3RpbWVzIEVfM1xcb3RpbWVzIChFXzFcXG90aW1lcyBFXzIpXFxvdGltZXMgRV8zIl0sWzAsMSwiRV8xXFxvdGltZXMgXFx0YXVfe0VfMlxcb3RpbWVzIEVfMyxFXzF9XFxvdGltZXMgRV8yXFxvdGltZXMgRV8zIiwyXSxbMSwzLCJcXG11XzFcXG90aW1lcyBFXzJcXG90aW1lcyBcXHRhdVxcb3RpbWVzIEVfMyIsMl0sWzMsNCwiRV8xXFxvdGltZXMgXFxtdV8yXFxvdGltZXMgXFxtdV8zIiwyXSxbMiw1LCJFXzFcXG90aW1lcyBcXHRhdVxcb3RpbWVzIEVfMlxcb3RpbWVzIFxcbXVfMyJdLFs1LDYsIlxcbXVfMVxcb3RpbWVzIFxcbXVfMlxcb3RpbWVzIEVfMyJdLFs0LDYsIlxcYWxwaGEiLDAseyJsZXZlbCI6Miwic3R5bGUiOnsiaGVhZCI6eyJuYW1lIjoibm9uZSJ9fX1dLFsxLDcsIkVfMVxcb3RpbWVzIEVfMVxcb3RpbWVzIEVfMlxcb3RpbWVzIFxcdGF1XFxvdGltZXMgRV8zIiwxXSxbMiw3LCJFXzFcXG90aW1lcyBcXHRhdVxcb3RpbWVzIEVfMlxcb3RpbWVzIEVfM1xcb3RpbWVzIEVfMyIsMV0sWzcsNSwiRV8xXFxvdGltZXMgRV8yXFxvdGltZXMgRV8yXFxvdGltZXMgRV8yXFxvdGltZXMgXFxtdV8zIiwyXSxbNywzLCJcXG11XzFcXG90aW1lcyBFXzJcXG90aW1lcyBFXzJcXG90aW1lcyBFXzNcXG90aW1lcyBFXzMiXSxbMCw4LCJcXGFscGhhIiwwLHsibGV2ZWwiOjIsInN0eWxlIjp7ImhlYWQiOnsibmFtZSI6Im5vbmUifX19XSxbOCwyLCJFXzFcXG90aW1lcyBFXzJcXG90aW1lcyBcXHRhdV97RV8zLEVfMVxcb3RpbWVzIEVfMn1cXG90aW1lcyBFXzMiXSxbNyw0LCJcXG11XzFcXG90aW1lcyBcXG11XzJcXG90aW1lcyBcXG11XzMiLDFdLFs3LDYsIlxcbXVfMVxcb3RpbWVzIFxcbXVfMlxcb3RpbWVzIFxcbXVfMyIsMV1d
	\[\begin{tikzcd}[column sep=tiny]
		{E_1\otimes(E_2\otimes E_3)\otimes E_1\otimes(E_2\otimes E_3)} && {(E_1\otimes E_2)\otimes E_3\otimes (E_1\otimes E_2)\otimes E_3} \\
		{E_1\otimes E_1\otimes E_2\otimes E_3\otimes E_2\otimes E_3} && {E_1\otimes E_2\otimes E_1\otimes E_2\otimes E_3\otimes E_3} \\
		{E_{1}\otimes E_2\otimes E_2\otimes E_3\otimes E_3} & {E_1\otimes E_1\otimes E_2\otimes E_2\otimes E_3\otimes E_3} & {E_1\otimes E_1\otimes E_2\otimes E_2\otimes E_{3}} \\
		{E_{1}\otimes E_{2}\otimes E_{3}} && {E_{1}\otimes E_{2}\otimes E_{3}}
		\arrow["{E_1\otimes \tau_{E_2\otimes E_3,E_1}\otimes E_2\otimes E_3}"', from=1-1, to=2-1]
		\arrow["{\mu_1\otimes E_2\otimes \tau\otimes E_3}"', from=2-1, to=3-1]
		\arrow["{E_1\otimes \mu_2\otimes \mu_3}"', from=3-1, to=4-1]
		\arrow["{E_1\otimes \tau\otimes E_2\otimes \mu_3}", from=2-3, to=3-3]
		\arrow["{\mu_1\otimes \mu_2\otimes E_3}", from=3-3, to=4-3]
		\arrow["\alpha", Rightarrow, no head, from=4-1, to=4-3]
		\arrow["{E_1\otimes E_1\otimes E_2\otimes \tau\otimes E_3}"{description}, from=2-1, to=3-2]
		\arrow["{E_1\otimes \tau\otimes E_2\otimes E_3\otimes E_3}"{description}, from=2-3, to=3-2]
		\arrow["{E_1\otimes E_2\otimes E_2\otimes E_2\otimes \mu_3}"', from=3-2, to=3-3]
		\arrow["{\mu_1\otimes E_2\otimes E_2\otimes E_3\otimes E_3}", from=3-2, to=3-1]
		\arrow["\alpha", Rightarrow, no head, from=1-1, to=1-3]
		\arrow["{E_1\otimes E_2\otimes \tau_{E_3,E_1\otimes E_2}\otimes E_3}", from=1-3, to=2-3]
		\arrow["{\mu_1\otimes \mu_2\otimes \mu_3}"{description}, from=3-2, to=4-1]
		\arrow["{\mu_1\otimes \mu_2\otimes \mu_3}"{description}, from=3-2, to=4-3]
	\end{tikzcd}\]
	The top pentagonal region commutes by coherence for the $\tau$'s in a symmetric monoidal category. The bottom triangle commutes by definition. The remaining four triangles commute by functoriality of $-\otimes-$. On the left is the product for $E_1\otimes(E_2\otimes E_3)$, while on the right is the product for $(E_1\otimes E_2)\otimes E_3$. Thus they are equal up to associativity, as desired.
\end{proof}

\begin{lemma}\label{e_and_mu_are_monoid_homos}
    Let $(E,\mu,e)$ be a monoid object in $\cSH$. Then the maps $e:S\to E$ and $\mu:E\otimes E\to E$ are monoid object homomorphisms (where here $S$ and $E\otimes E$ are considered to be monoid objects by \autoref{unit_is_monoid} and \autoref{product_of_monoids_is_monoid}, respectively).
\end{lemma}
\begin{proof}
    \todo{todo}
\end{proof}

\begin{lemma}\label{E_ox_f,f_ox_E_are_monoid_homos}
	Suppose we have some monoid object $(E,\mu,e)$ in $\cC$ and some homomorphism of monoid objects $f:(E_1,\mu_1,e_1)\to (E_2,\mu_2,e_2)$ in $\Mon_\cC$. Then $E\otimes f:E\otimes E_1\to E\otimes E_2$ and $f\otimes E:E_1\otimes E\to E_2\otimes E$ are monoid homomorphisms, where here we are considering $E\otimes E_1$, $E\otimes E_2$, $E_1\otimes E$, and $E_2\otimes E$ to be monoid objects by \autoref{product_of_monoids_is_monoid}.
\end{lemma}
\begin{proof}
	We will show that $E\otimes f$ is a monoid object homomorphism, as showing $f\otimes E$ is a monoid homomorphism is entirely analagous. First consider the following diagram:
	% https://q.uiver.app/#q=WzAsOCxbMCwwLCJFXFxvdGltZXMgRV8xXFxvdGltZXMgRVxcb3RpbWVzIEVfMSJdLFszLDAsIkVcXG90aW1lcyBFXzJcXG90aW1lcyBFXFxvdGltZXMgRV8yIl0sWzAsMSwiRVxcb3RpbWVzIEVcXG90aW1lcyBFXzFcXG90aW1lcyBFXzEiXSxbMywxLCJFXFxvdGltZXMgRVxcb3RpbWVzIEVfMlxcb3RpbWVzIEVfMiJdLFswLDMsIkVcXG90aW1lcyBFXzEiXSxbMywzLCJFXFxvdGltZXMgRV8yIl0sWzEsMiwiRVxcb3RpbWVzIEVfMVxcb3RpbWVzIEVfMSJdLFsyLDIsIkVcXG90aW1lcyBFXzJcXG90aW1lcyBFXzIiXSxbMCwxLCJFXFxvdGltZXMgZlxcb3RpbWVzIEVcXG90aW1lcyBmIl0sWzAsMiwiRVxcb3RpbWVzIFxcdGF1XFxvdGltZXMgRV8xIiwyXSxbMSwzLCJFXFxvdGltZXMgXFx0YXVcXG90aW1lcyBFXzIiXSxbMiw0LCJcXG11XFxvdGltZXMgXFxtdV8xIiwyXSxbNCw1LCJFXFxvdGltZXMgZiJdLFszLDUsIlxcbXVcXG90aW1lcyBcXG11XzIiXSxbMiwzLCJFXFxvdGltZXMgRVxcb3RpbWVzIGZcXG90aW1lcyBmIl0sWzIsNiwiXFxtdVxcb3RpbWVzIEVfMVxcb3RpbWVzIEVfMiIsMV0sWzYsNCwiRVxcb3RpbWVzIFxcbXVfMSIsMV0sWzYsNywiRVxcb3RpbWVzIGZcXG90aW1lcyBmIl0sWzcsNSwiRVxcb3RpbWVzXFxtdV8yIiwxXSxbMyw3LCJcXG11XFxvdGltZXMgRV8yXFxvdGltZXMgRV8yIiwxXV0=
	\[\begin{tikzcd}
		{E\otimes E_1\otimes E\otimes E_1} &&& {E\otimes E_2\otimes E\otimes E_2} \\
		{E\otimes E\otimes E_1\otimes E_1} &&& {E\otimes E\otimes E_2\otimes E_2} \\
		& {E\otimes E_1\otimes E_1} & {E\otimes E_2\otimes E_2} \\
		{E\otimes E_1} &&& {E\otimes E_2}
		\arrow["{E\otimes f\otimes E\otimes f}", from=1-1, to=1-4]
		\arrow["{E\otimes \tau\otimes E_1}"', from=1-1, to=2-1]
		\arrow["{E\otimes \tau\otimes E_2}", from=1-4, to=2-4]
		\arrow["{\mu\otimes \mu_1}"', from=2-1, to=4-1]
		\arrow["{E\otimes f}", from=4-1, to=4-4]
		\arrow["{\mu\otimes \mu_2}", from=2-4, to=4-4]
		\arrow["{E\otimes E\otimes f\otimes f}", from=2-1, to=2-4]
		\arrow["{\mu\otimes E_1\otimes E_2}"{description}, from=2-1, to=3-2]
		\arrow["{E\otimes \mu_1}"{description}, from=3-2, to=4-1]
		\arrow["{E\otimes f\otimes f}", from=3-2, to=3-3]
		\arrow["{E\otimes\mu_2}"{description}, from=3-3, to=4-4]
		\arrow["{\mu\otimes E_2\otimes E_2}"{description}, from=2-4, to=3-3]
	\end{tikzcd}\]
	The top region commutes by naturality of $\tau$. The bottom trapezoid commutes since $f$ is a monoid homomorphism. The remaining three regions commute by functoriality of $-\otimes-$. Now, consider the following diagram:
	% https://q.uiver.app/#q=WzAsNCxbMiwwLCJTIl0sWzAsMywiRVxcb3RpbWVzIEVfMSJdLFs0LDMsIkVcXG90aW1lcyBFXzIiXSxbMiwyLCJFIl0sWzAsMSwiZVxcb3RpbWVzIGVfMSIsMl0sWzEsMiwiRVxcb3RpbWVzIGYiXSxbMCwyLCJlXFxvdGltZXMgZV8yIl0sWzAsMywiZSIsMV0sWzMsMSwiRVxcb3RpbWVzIGVfMSIsMV0sWzMsMiwiRVxcb3RpbWVzIGVfMiIsMV1d
	\[\begin{tikzcd}
		&& S \\
		\\
		&& E \\
		{E\otimes E_1} &&&& {E\otimes E_2}
		\arrow["{e\otimes e_1}"', from=1-3, to=4-1]
		\arrow["{E\otimes f}", from=4-1, to=4-5]
		\arrow["{e\otimes e_2}", from=1-3, to=4-5]
		\arrow["e"{description}, from=1-3, to=3-3]
		\arrow["{E\otimes e_1}"{description}, from=3-3, to=4-1]
		\arrow["{E\otimes e_2}"{description}, from=3-3, to=4-5]
	\end{tikzcd}\]
	The bottom region commutes since $f$ is a monoid homomorphism. The top two regions commute by functoriality of $-\otimes-$. Thus, we've shown $E\otimes f$ is a monoid object homomorphism, as desired.
\end{proof}

\subsection{Modules over monoid objects in a symmetric monoidal category}\label{subsection:modules_over_monoids}

\begin{definition}\label{left_module_object}
	Let $(E,\mu,e)$ be a monoid object in $\cC$. Then a \emph{(left) module object} $(N,\kappa)$ over $(E,\mu,e)$ is the data of an object $N$ in $\cC$ and a morphism $\kappa:E\otimes N\to N$ such that the following two diagrams commute in $\cC$:
	% https://q.uiver.app/#q=WzAsOCxbMCwwLCJTXFxvdGltZXMgTiJdLFsxLDAsIkVcXG90aW1lcyBOIl0sWzEsMSwiTiJdLFsyLDAsIihFXFxvdGltZXMgRSlcXG90aW1lcyBOIl0sWzQsMCwiRVxcb3RpbWVzIE4iXSxbNCwxLCJOIl0sWzIsMSwiRVxcb3RpbWVzKEVcXG90aW1lcyBOKSJdLFszLDEsIkVcXG90aW1lcyBOIl0sWzAsMSwiZVxcb3RpbWVzIE4iXSxbMSwyLCJcXGthcHBhIl0sWzAsMiwiXFxsYW1iZGFfTiIsMl0sWzMsNCwiXFxtdVxcb3RpbWVzIE4iXSxbNCw1LCJcXGthcHBhIl0sWzMsNiwiXFxhbHBoYSIsMl0sWzYsNywiRVxcb3RpbWVzIFxca2FwcGEiXSxbNyw1LCJcXGthcHBhIl1d
	\[\begin{tikzcd}
		{S\otimes N} & {E\otimes N} & {(E\otimes E)\otimes N} && {E\otimes N} \\
		& N & {E\otimes(E\otimes N)} & {E\otimes N} & N
		\arrow["{e\otimes N}", from=1-1, to=1-2]
		\arrow["\kappa", from=1-2, to=2-2]
		\arrow["{\lambda_N}"', from=1-1, to=2-2]
		\arrow["{\mu\otimes N}", from=1-3, to=1-5]
		\arrow["\kappa", from=1-5, to=2-5]
		\arrow["\alpha"', from=1-3, to=2-3]
		\arrow["{E\otimes \kappa}", from=2-3, to=2-4]
		\arrow["\kappa", from=2-4, to=2-5]
	\end{tikzcd}\]
\end{definition}

\begin{definition}\label{homomorphism_of_left_module_objects}
	Let $(E,\mu,e)$ be a monoid object in $\cC$, and suppose we have two (left) module objects $(N,\kappa)$ and $(N',\kappa')$ over $(E,\mu,e)$. Then a morphism $f:N\to N'$ is a \emph{(left) $E$-module homomorphism} if the following diagram commutes in $\cC$:
	% https://q.uiver.app/#q=WzAsNCxbMCwwLCJFXFxvdGltZXMgTiJdLFsxLDAsIkVcXG90aW1lcyBOJyJdLFswLDEsIk4iXSxbMSwxLCJOJyJdLFswLDEsIkVcXG90aW1lcyBmIl0sWzAsMiwiXFxrYXBwYSIsMl0sWzIsMywiZiJdLFsxLDMsIlxca2FwcGEnIl1d
	\[\begin{tikzcd}
		{E\otimes N} & {E\otimes N'} \\
		N & {N'}
		\arrow["{E\otimes f}", from=1-1, to=1-2]
		\arrow["\kappa"', from=1-1, to=2-1]
		\arrow["f", from=2-1, to=2-2]
		\arrow["{\kappa'}", from=1-2, to=2-2]
	\end{tikzcd}\]
\end{definition}

\begin{definition}
	Given a monoid object $(E,\mu,e)$ in $\cC$, we write $E\text-\Mod$ to denote the category of (left) module objects over $E$ and $E$-module homomorphisms between them. We denote the homset in $E\text-\Mod$ by
	\[\Hom_{E\text-\Mod}(M,N),\qquad\text{or simply}\qquad\Hom_E(M,N).\]
\end{definition}

For our purposes, we will only consider left module objects, so we will usually drop the quanitfier ``left'' and just refer to them as ``module objects''.

\begin{lemma}\label{module_if_iso_to_module}
	Let $(E,\mu,e)$ be a monoid object in $\cC$ and let $(N,\kappa)$ be an $E$ module object. Then given some object $X$ in $\cC$ and an isomorphism $\phi:N\xr\cong X$, $X$ inherits the structure of an $E$-module via the action map
	\[\kappa_\phi:E\otimes X\xr{E\otimes\phi^{-1}}E\otimes N\xr\kappa N\xr\phi X.\]
\end{lemma}
\begin{proof}
	We need to show the two coherence diagrams in \autoref{left_module_object} commute. To see the former commutes, consider the following diagram:
	% https://q.uiver.app/#q=WzAsNixbMCwwLCJYIl0sWzMsMywiWCJdLFszLDAsIkVcXG90aW1lcyBYIl0sWzMsMSwiRVxcb3RpbWVzIE4iXSxbMywyLCJOIl0sWzIsMSwiTiJdLFswLDEsIiIsMCx7ImxldmVsIjoyLCJzdHlsZSI6eyJoZWFkIjp7Im5hbWUiOiJub25lIn19fV0sWzAsMiwiZVxcb3RpbWVzIFgiXSxbMiwzLCJFXFxvdGltZXNcXHBoaV57LTF9Il0sWzMsNCwiXFxrYXBwYSJdLFs0LDEsIlxccGhpIl0sWzUsNCwiIiwwLHsibGV2ZWwiOjIsInN0eWxlIjp7ImhlYWQiOnsibmFtZSI6Im5vbmUifX19XSxbNSwzLCJlXFxvdGltZXMgTiJdLFswLDUsIlxccGhpXnstMX0iXV0=
	\[\begin{tikzcd}
		X &&& {E\otimes X} \\
		&& N & {E\otimes N} \\
		&&& N \\
		&&& X
		\arrow[Rightarrow, no head, from=1-1, to=4-4]
		\arrow["{e\otimes X}", from=1-1, to=1-4]
		\arrow["{E\otimes\phi^{-1}}", from=1-4, to=2-4]
		\arrow["\kappa", from=2-4, to=3-4]
		\arrow["\phi", from=3-4, to=4-4]
		\arrow[Rightarrow, no head, from=2-3, to=3-4]
		\arrow["{e\otimes N}", from=2-3, to=2-4]
		\arrow["{\phi^{-1}}", from=1-1, to=2-3]
	\end{tikzcd}\]
	The top trapezoid commutes by functoriality of $-\otimes-$. The middle small triangle commutes by unitality of $\kappa$. The remaining region commutes by definition. To see the second coherence diagram commutes, consider the following diagram:
	% https://q.uiver.app/#q=WzAsMTAsWzAsMCwiRVxcb3RpbWVzIEVcXG90aW1lcyBYIl0sWzMsMCwiRVxcb3RpbWVzIFgiXSxbMywxLCJFXFxvdGltZXMgTiJdLFszLDIsIk4iXSxbMywzLCJYIl0sWzAsMSwiRVxcb3RpbWVzIEVcXG90aW1lcyBOIl0sWzAsMiwiRVxcb3RpbWVzIE4iXSxbMCwzLCJFXFxvdGltZXMgWCJdLFsxLDMsIkVcXG90aW1lcyBOIl0sWzIsMywiTiJdLFswLDEsIlxcbXVcXG90aW1lcyBYIl0sWzEsMiwiRVxcb3RpbWVzXFxwaGleey0xfSJdLFsyLDMsIlxca2FwcGEiXSxbMyw0LCJcXHBoaSJdLFswLDUsIkVcXG90aW1lcyBFXFxvdGltZXNcXHBoaV57LTF9IiwyXSxbNSw2LCJFXFxvdGltZXNcXGthcHBhIiwyXSxbNiw3LCJFXFxvdGltZXNcXHBoaSIsMl0sWzcsOCwiRVxcb3RpbWVzXFxwaGleezEtfSIsMl0sWzgsOSwiXFxrYXBwYSIsMl0sWzksNCwiXFxwaGkiLDJdLFs1LDIsIlxcbXVcXG90aW1lcyBOIl0sWzYsMywiXFxrYXBwYSJdLFs2LDgsIiIsMCx7ImxldmVsIjoyLCJzdHlsZSI6eyJoZWFkIjp7Im5hbWUiOiJub25lIn19fV1d
	\[\begin{tikzcd}
		{E\otimes E\otimes X} &&& {E\otimes X} \\
		{E\otimes E\otimes N} &&& {E\otimes N} \\
		{E\otimes N} &&& N \\
		{E\otimes X} & {E\otimes N} & N & X
		\arrow["{\mu\otimes X}", from=1-1, to=1-4]
		\arrow["{E\otimes\phi^{-1}}", from=1-4, to=2-4]
		\arrow["\kappa", from=2-4, to=3-4]
		\arrow["\phi", from=3-4, to=4-4]
		\arrow["{E\otimes E\otimes\phi^{-1}}"', from=1-1, to=2-1]
		\arrow["E\otimes\kappa"', from=2-1, to=3-1]
		\arrow["E\otimes\phi"', from=3-1, to=4-1]
		\arrow["{E\otimes\phi^{1-}}"', from=4-1, to=4-2]
		\arrow["\kappa"', from=4-2, to=4-3]
		\arrow["\phi"', from=4-3, to=4-4]
		\arrow["{\mu\otimes N}", from=2-1, to=2-4]
		\arrow["\kappa", from=3-1, to=3-4]
		\arrow[Rightarrow, no head, from=3-1, to=4-2]
	\end{tikzcd}\]
	The top rectangle commutes by functoriality of $-\otimes-$. The middle rectangle commutes by coherence for $\kappa$. The bottom two regions commute by definition.
\end{proof}

\begin{proposition}\label{free_forgetful_E-Mod}
	Given a monoid object $(E,\mu,e)$ in $\cC$, the forgetful functor $E\text-\Mod\to\cC$ has a left adjoint $\cC\to E\text-\Mod$ sending an object $X$ in $\cC$ to $(E\otimes X,\kappa_X)$ where $\kappa_X$ is the composition
	\[E\otimes(E\otimes X)\xr{\alpha^{-1}}(E\otimes E)\otimes X\xr{\mu\otimes X}E\otimes X,\]
	and sending a morphism $f:X\to Y$ to $E\otimes f:E\otimes X\to E\otimes Y$.

	We call this functor $E\otimes-:\cC\to E\text-\Mod$ the \emph{free} functor, and we call $E$-modules in the image of the free functor \emph{free modules}.
\end{proposition}
\begin{proof}
	In this proof, we work in a symmetric strict monoidal category. First, we wish to show that $E\otimes-:\cC\to E\text-\Mod$ as constructed is well-defined. First, to see that $(X,\kappa_X)$ is actually a $E$-module, we need to show the two diagrams in \autoref{left_module_object} commute. Indeed, consider the following diagrams:
	% https://q.uiver.app/#q=WzAsNyxbMCwwLCJFXFxvdGltZXMgWCJdLFsxLDAsIkVcXG90aW1lcyBFXFxvdGltZXMgWCJdLFsxLDEsIkVcXG90aW1lcyBYIl0sWzIsMCwiRVxcb3RpbWVzIEVcXG90aW1lcyBFXFxvdGltZXMgWCJdLFszLDAsIkVcXG90aW1lcyBFXFxvdGltZXMgWCJdLFszLDEsIkVcXG90aW1lcyBYIl0sWzIsMSwiRVxcb3RpbWVzIEVcXG90aW1lcyBYIl0sWzAsMSwiZVxcb3RpbWVzIEVcXG90aW1lcyBYIl0sWzEsMiwiXFxtdVxcb3RpbWVzIFgiXSxbMCwyLCIiLDIseyJsZXZlbCI6Miwic3R5bGUiOnsiaGVhZCI6eyJuYW1lIjoibm9uZSJ9fX1dLFszLDQsIlxcbXVcXG90aW1lcyBFXFxvdGltZXMgWCJdLFs0LDUsIlxcbXVcXG90aW1lcyBYIl0sWzMsNiwiRVxcb3RpbWVzIFxcbXVcXG90aW1lcyBYIiwyXSxbNiw1LCJcXG11XFxvdGltZXMgWCIsMl1d
	\[\begin{tikzcd}
		{E\otimes X} & {E\otimes E\otimes X} & {E\otimes E\otimes E\otimes X} & {E\otimes E\otimes X} \\
		& {E\otimes X} & {E\otimes E\otimes X} & {E\otimes X}
		\arrow["{e\otimes E\otimes X}", from=1-1, to=1-2]
		\arrow["{\mu\otimes X}", from=1-2, to=2-2]
		\arrow[Rightarrow, no head, from=1-1, to=2-2]
		\arrow["{\mu\otimes E\otimes X}", from=1-3, to=1-4]
		\arrow["{\mu\otimes X}", from=1-4, to=2-4]
		\arrow["{E\otimes \mu\otimes X}"', from=1-3, to=2-3]
		\arrow["{\mu\otimes X}"', from=2-3, to=2-4]
	\end{tikzcd}\]
	These are precisely the diagrams obtained by applying $X\otimes-$ to the coherence diagrams for $\mu$, so that they commute as desired. Now, suppose $f:X\to Y$ is a morphism in $\cC$, then we would like to show that $E\otimes f:E\otimes X\to E\otimes Y$ is a morphism of $E$-module objects. Indeed, consider the following diagram:
	% https://q.uiver.app/#q=WzAsNCxbMCwwLCJFXFxvdGltZXMgRVxcb3RpbWVzIFgiXSxbMSwwLCJFXFxvdGltZXMgRVxcb3RpbWVzIFkiXSxbMSwxLCJFXFxvdGltZXMgWSJdLFswLDEsIkVcXG90aW1lcyBYIl0sWzAsMSwiRVxcb3RpbWVzIEVcXG90aW1lcyBmIl0sWzEsMiwiXFxtdVxcb3RpbWVzIFkiXSxbMCwzLCJcXG11XFxvdGltZXMgWCIsMl0sWzMsMiwiRVxcb3RpbWVzIGYiXV0=
	\[\begin{tikzcd}
		{E\otimes E\otimes X} & {E\otimes E\otimes Y} \\
		{E\otimes X} & {E\otimes Y}
		\arrow["{E\otimes E\otimes f}", from=1-1, to=1-2]
		\arrow["{\mu\otimes Y}", from=1-2, to=2-2]
		\arrow["{\mu\otimes X}"', from=1-1, to=2-1]
		\arrow["{E\otimes f}", from=2-1, to=2-2]
	\end{tikzcd}\]
	It commutes by functoriality of $-\otimes-$, so $E\otimes f$ is indeed an $E$-module homomorphism as desired.

	Now, in order to see that $E\otimes-$ is left adjoint to the forgetful functor, it suffices to construct a unit and counit for the adjunction and show they satisfy the zig-zag identities. Given $X$ in $\cC$ and $(N,\kappa)$ in $E\text-\Mod$, define $\eta_X:=e\otimes X:X\to E\otimes X$ and $\vare_{(N,\kappa)}:=\kappa:E\otimes N\to N$. $\eta_X$ is clearly natural in $X$ by functoriality of $-\otimes-$, and $\vare_{(N,\kappa)}$ is natural in $(N,\kappa)$ by how morphisms in $E\text-\Mod$ are defined. Now, to see these are actually the unit and counit of an adjunction, we need to show that the following diagrams commute for all $X$ in $\cC$ and $(N,\kappa)$ in $E\text-\Mod$:
	% https://q.uiver.app/#q=WzAsNixbMCwwLCJFXFxvdGltZXMgWCJdLFsyLDAsIkVcXG90aW1lcyBFXFxvdGltZXMgWCJdLFsyLDIsIkVcXG90aW1lcyBYIl0sWzUsMCwiRVxcb3RpbWVzIE4iXSxbNSwyLCJOIl0sWzcsMCwiTiJdLFswLDEsIkVcXG90aW1lc1xcZXRhX1g9RVxcb3RpbWVzIGVcXG90aW1lcyBYIl0sWzEsMiwiXFx2YXJlX3soRVxcb3RpbWVzIFgsXFxrYXBwYV9YKX09XFxtdVxcb3RpbWVzIFgiXSxbMCwyLCIiLDIseyJsZXZlbCI6Miwic3R5bGUiOnsiaGVhZCI6eyJuYW1lIjoibm9uZSJ9fX1dLFszLDQsIlxcdmFyZV97KE4sXFxrYXBwYSl9PVxca2FwcGEiLDJdLFs1LDMsIlxcZXRhX049ZVxcb3RpbWVzIE4iLDJdLFs1LDQsIiIsMCx7ImxldmVsIjoyLCJzdHlsZSI6eyJoZWFkIjp7Im5hbWUiOiJub25lIn19fV1d
	\[\begin{tikzcd}
		{E\otimes X} && {E\otimes E\otimes X} &&& {E\otimes N} && N \\
		\\
		&& {E\otimes X} &&& N
		\arrow["{E\otimes\eta_X=E\otimes e\otimes X}", from=1-1, to=1-3]
		\arrow["{\vare_{(E\otimes X,\kappa_X)}=\mu\otimes X}", from=1-3, to=3-3]
		\arrow[Rightarrow, no head, from=1-1, to=3-3]
		\arrow["{\vare_{(N,\kappa)}=\kappa}"', from=1-6, to=3-6]
		\arrow["{\eta_N=e\otimes N}"', from=1-8, to=1-6]
		\arrow[Rightarrow, no head, from=1-8, to=3-6]
	\end{tikzcd}\]
	Commutativity of the left diagram is unitality of $\mu$, while commutativity of the right diagram is unitality of $\kappa$. Thus indeed $E\otimes-:\cC\to E\text-\Mod$ is a left adjoint of the forgetful functor $E\text-\Mod\to\cC$, as desired.
\end{proof}

\begin{lemma}\label{retract_of_module_whose_idempotent_is_module_homomorphism_is_module}
	Let $(E,\mu,e)$ be a monoid object in $\cC$. Further suppose we have some object $X$ in $\cC$ and an $E$-module object $(N,\kappa)$, along with a commuting diagram in $\cC$
	% https://q.uiver.app/#q=WzAsMyxbMCwwLCJYIl0sWzEsMCwiTiJdLFsyLDAsIlgiXSxbMCwxLCJcXGlvdGEiLDJdLFsxLDIsInIiLDJdLFswLDIsIiIsMix7ImN1cnZlIjotMywibGV2ZWwiOjIsInN0eWxlIjp7ImhlYWQiOnsibmFtZSI6Im5vbmUifX19XV0=
	\[\begin{tikzcd}
		X & N & X
		\arrow["\iota"', from=1-1, to=1-2]
		\arrow["r"', from=1-2, to=1-3]
		\arrow[curve={height=-18pt}, Rightarrow, no head, from=1-1, to=1-3]
	\end{tikzcd}\]
	Then if $\ell:=\iota\circ r:N\to N$ is an $E$-module homomorphism, then $X$ is canonically an $E$-module object with structure map
	\[\kappa_X:E\otimes X\xr{E\otimes\iota}E\otimes N\xr{\kappa}N\xr{r}X,\]
	and furthermore, the maps $\iota:X\to N$ and $r:N\to X$ are $E$-module homomorphisms.
\end{lemma}
\begin{proof}
	First, in order to show $(X,\kappa_X)$ is an $E$-module, we need to show the two diagrams in \autoref{left_module_object} commute. To see the unitality diagram holds, consider the following diagram:
	% https://q.uiver.app/#q=WzAsNyxbMCwwLCJTXFxvdGltZXMgWCJdLFsyLDAsIkVcXG90aW1lcyBYIl0sWzIsMSwiRVxcb3RpbWVzIE4iXSxbMiwyLCJOIl0sWzIsMywiWCJdLFswLDMsIlgiXSxbMSwxLCJTXFxvdGltZXMgTiJdLFswLDEsImVcXG90aW1lcyBYIl0sWzEsMiwiRVxcb3RpbWVzXFxpb3RhIl0sWzIsMywiXFxrYXBwYSJdLFszLDQsInIiXSxbMCw1LCJcXGxhbWJkYV9YIiwyXSxbNSw0LCIiLDIseyJsZXZlbCI6Miwic3R5bGUiOnsiaGVhZCI6eyJuYW1lIjoibm9uZSJ9fX1dLFswLDYsIlNcXG90aW1lc1xcaW90YSIsMV0sWzYsMiwiZVxcb3RpbWVzIE4iXSxbNiwzLCJcXGxhbWJkYV9OIiwxXSxbNSwzLCJcXGlvdGEiXV0=
	\[\begin{tikzcd}
		{S\otimes X} && {E\otimes X} \\
		& {S\otimes N} & {E\otimes N} \\
		&& N \\
		X && X
		\arrow["{e\otimes X}", from=1-1, to=1-3]
		\arrow["E\otimes\iota", from=1-3, to=2-3]
		\arrow["\kappa", from=2-3, to=3-3]
		\arrow["r", from=3-3, to=4-3]
		\arrow["{\lambda_X}"', from=1-1, to=4-1]
		\arrow[Rightarrow, no head, from=4-1, to=4-3]
		\arrow["S\otimes\iota"{description}, from=1-1, to=2-2]
		\arrow["{e\otimes N}", from=2-2, to=2-3]
		\arrow["{\lambda_N}"{description}, from=2-2, to=3-3]
		\arrow["\iota", from=4-1, to=3-3]
	\end{tikzcd}\]
	The large left triangle commutes by naturality of $\lambda$. The top trapezoid commutes by functoriality of $-\otimes-$. The small middle right triangle commutes by unitality of $\kappa$. Finally, the bottom triangle commutes by definition, since we are assuming $r\circ\iota=\id_X$. Now the right composition is $\kappa_X$, so we have shown $\kappa_X\circ(e\otimes X)=\lambda_X$, as desired. Now, consider the following diagram:
	% https://q.uiver.app/#q=WzAsMTEsWzAsMCwiRVxcb3RpbWVzIEVcXG90aW1lcyBYIl0sWzMsMCwiRVxcb3RpbWVzIFgiXSxbMywxLCJFXFxvdGltZXMgTiJdLFszLDIsIk4iXSxbMywzLCJYIl0sWzAsMSwiRVxcb3RpbWVzIEVcXG90aW1lcyBOIl0sWzAsMiwiRVxcb3RpbWVzIE4iXSxbMCwzLCJFXFxvdGltZXMgWCJdLFsxLDMsIkVcXG90aW1lcyBOIl0sWzIsMywiTiJdLFsxLDEsIkVcXG90aW1lcyBFXFxvdGltZXMgTiJdLFswLDEsIlxcbXVcXG90aW1lcyBYIl0sWzEsMiwiRVxcb3RpbWVzXFxpb3RhIl0sWzIsMywiXFxrYXBwYSJdLFszLDQsInIiXSxbMCw1LCJFXFxvdGltZXMgRVxcb3RpbWVzXFxpb3RhIiwyXSxbNSw2LCJFXFxvdGltZXNcXGthcHBhIiwyXSxbNiw3LCJFXFxvdGltZXMgciIsMl0sWzcsOCwiRVxcb3RpbWVzXFxpb3RhIiwyXSxbOCw5LCJcXGthcHBhIiwyXSxbOSw0LCJyIiwyXSxbNSwxMCwiRVxcb3RpbWVzIEVcXG90aW1lc1xcZWxsIiwyXSxbMCwxMCwiRVxcb3RpbWVzIEVcXG90aW1lc1xcaW90YSIsMV0sWzEwLDgsIkVcXG90aW1lc1xca2FwcGEiXSxbNiw4LCJFXFxvdGltZXNcXGVsbCJdLFsxMCwyLCJcXG11XFxvdGltZXMgTiJdLFszLDksIiIsMSx7ImxldmVsIjoyLCJzdHlsZSI6eyJoZWFkIjp7Im5hbWUiOiJub25lIn19fV1d
	\[\begin{tikzcd}
		{E\otimes E\otimes X} &&& {E\otimes X} \\
		{E\otimes E\otimes N} & {E\otimes E\otimes N} && {E\otimes N} \\
		{E\otimes N} &&& N \\
		{E\otimes X} & {E\otimes N} & N & X
		\arrow["{\mu\otimes X}", from=1-1, to=1-4]
		\arrow["E\otimes\iota", from=1-4, to=2-4]
		\arrow["\kappa", from=2-4, to=3-4]
		\arrow["r", from=3-4, to=4-4]
		\arrow["{E\otimes E\otimes\iota}"', from=1-1, to=2-1]
		\arrow["E\otimes\kappa"', from=2-1, to=3-1]
		\arrow["{E\otimes r}"', from=3-1, to=4-1]
		\arrow["E\otimes\iota"', from=4-1, to=4-2]
		\arrow["\kappa"', from=4-2, to=4-3]
		\arrow["r"', from=4-3, to=4-4]
		\arrow["{E\otimes E\otimes\ell}"', from=2-1, to=2-2]
		\arrow["{E\otimes E\otimes\iota}"{description}, from=1-1, to=2-2]
		\arrow["E\otimes\kappa", from=2-2, to=4-2]
		\arrow["E\otimes\ell", from=3-1, to=4-2]
		\arrow["{\mu\otimes N}", from=2-2, to=2-4]
		\arrow[Rightarrow, no head, from=3-4, to=4-3]
	\end{tikzcd}\]
	The top trapezoid commutes by funtoriality of $-\otimes-$. The top left triangle commutes by functoriality of $-\otimes-$ and the fact that $\ell\circ\iota=\iota\circ r\circ\iota=\iota\circ\id_X=\iota$.  The middle left trapezoid commutes by since $\ell$ is an $E$-module homomorphism, by assumption. The bottom left triangle commutes by functoriality of $-\otimes-$ and the fact that $\iota\circ r=\ell$. Thus, we have shown that $(X,\kappa_X)$ is an $E$-module object, as desired.

	Now, it remains to show that $\iota:X\to N$ and $r:N\to X$ are $E$-module homomorphisms. To that end, consider the following two diagrams:
	% https://q.uiver.app/#q=WzAsMTIsWzAsMCwiRVxcb3RpbWVzIFgiXSxbMiwwLCJFXFxvdGltZXMgTiJdLFsyLDMsIk4iXSxbMCwxLCJFXFxvdGltZXMgTiJdLFswLDIsIk4iXSxbMCwzLCJYIl0sWzMsMCwiRVxcb3RpbWVzIFgiXSxbMywzLCJYIl0sWzMsMSwiRVxcb3RpbWVzIE4iXSxbMywyLCJOIl0sWzEsMCwiRVxcb3RpbWVzIE4iXSxbMSwzLCJOIl0sWzEsMiwiXFxrYXBwYSIsMl0sWzAsMywiRVxcb3RpbWVzXFxpb3RhIiwyXSxbMyw0LCJcXGthcHBhIiwyXSxbNCw1LCJyIiwyXSxbMSw2LCJFXFxvdGltZXMgciJdLFsyLDcsInIiXSxbNiw4LCJFXFxvdGltZXNcXGlvdGEiXSxbOCw5LCJcXGthcHBhIl0sWzksNywiciJdLFsxLDgsIkVcXG90aW1lc1xcZWxsIiwxXSxbMiw5LCJcXGVsbCIsMV0sWzEwLDExLCJcXGthcHBhIl0sWzAsMTAsIkVcXG90aW1lc1xcaW90YSJdLFs1LDExLCJcXGlvdGEiLDJdLFs0LDExLCJcXGVsbCIsMV0sWzMsMTAsIkVcXG90aW1lcyBcXGVsbCIsMV1d
	\[\begin{tikzcd}
		{E\otimes X} & {E\otimes N} & {E\otimes N} & {E\otimes X} \\
		{E\otimes N} &&& {E\otimes N} \\
		N &&& N \\
		X & N & N & X
		\arrow["\kappa"', from=1-3, to=4-3]
		\arrow["E\otimes\iota"', from=1-1, to=2-1]
		\arrow["\kappa"', from=2-1, to=3-1]
		\arrow["r"', from=3-1, to=4-1]
		\arrow["{E\otimes r}", from=1-3, to=1-4]
		\arrow["r", from=4-3, to=4-4]
		\arrow["E\otimes\iota", from=1-4, to=2-4]
		\arrow["\kappa", from=2-4, to=3-4]
		\arrow["r", from=3-4, to=4-4]
		\arrow["E\otimes\ell"{description}, from=1-3, to=2-4]
		\arrow["\ell"{description}, from=4-3, to=3-4]
		\arrow["\kappa", from=1-2, to=4-2]
		\arrow["E\otimes\iota", from=1-1, to=1-2]
		\arrow["\iota"', from=4-1, to=4-2]
		\arrow["\ell"{description}, from=3-1, to=4-2]
		\arrow["{E\otimes \ell}"{description}, from=2-1, to=1-2]
	\end{tikzcd}\]
	The trapezoids in each diagram commute since we are assuming $\ell$ is a $E$-module homomorphism. The four triangles commute since $\ell\circ\iota=\iota$ and $r\circ\ell=r$. Thus, we have shown that $\kappa_X\circ(E\otimes r)=r\circ\kappa$ and $\kappa\circ(E\otimes\iota)=\iota\circ\kappa_X$, so we indeed have that $\iota$ and $r$ are $E$-module homomorphisms, as desired.
\end{proof}

\begin{proposition}\label{coproduct_of_E_modules_is_coproduct_in_E_mod}
	Suppose that $\cC$ is an additive symmetric monoidal closed category. Let $(E,\mu,e)$ be a monoid object in $\cC$, and suppose we have a family of $E$-module objects $(N_i,\kappa_i)$ indexed by some small set $I$. Then $N:=\bigoplus_{i\in I}N_i$ is canonically an $E$-module, with action map given by the composition
	\[\kappa:E\otimes\bigoplus_iN_i\xr\cong\bigoplus_i(E\otimes N_i)\xrightarrow{\bigoplus_i\kappa_i}\bigoplus_iN_i,\]
	where the first isomorphism is given by the fact that $E\otimes-$ preserves coproducts, since it is a left adjoint. Furthermore, $N$ is the coproduct of all the $N_i$'s in $E\text-\Mod$, so that $E\text-\Mod$ has arbitrary coproducts.
\end{proposition}
\begin{proof}
	We need to show the action map $\kappa$ makes the diagrams in \autoref{left_module_object} commute. To see the first (unitality) diagram commutes, consider the following diagram:
	% https://q.uiver.app/#q=WzAsNCxbMCwwLCJcXGJpZ29wbHVzX2lOX2kiXSxbNCwwLCJFXFxvdGltZXNcXGJpZ29wbHVzX2lOX2kiXSxbNCwyLCJcXGJpZ29wbHVzX2koRVxcb3RpbWVzIE5faSkiXSxbNCw0LCJcXGJpZ29wbHVzX2lOX2kiXSxbMCwxLCJlXFxvdGltZXNcXGJpZ29wbHVzX2lOX2kiXSxbMSwyLCJcXGNvbmciXSxbMiwzLCJcXGJpZ29wbHVzX2lcXGthcHBhX2kiXSxbMCwzLCIiLDIseyJsZXZlbCI6Miwic3R5bGUiOnsiaGVhZCI6eyJuYW1lIjoibm9uZSJ9fX1dLFswLDIsIlxcYmlnb3BsdXNfaShlXFxvdGltZXMgTl9pKSIsMV1d
	\[\begin{tikzcd}
		{\bigoplus_iN_i} &&&& {E\otimes\bigoplus_iN_i} \\
		\\
		&&&& {\bigoplus_i(E\otimes N_i)} \\
		\\
		&&&& {\bigoplus_iN_i}
		\arrow["{e\otimes\bigoplus_iN_i}", from=1-1, to=1-5]
		\arrow["\cong", from=1-5, to=3-5]
		\arrow["{\bigoplus_i\kappa_i}", from=3-5, to=5-5]
		\arrow[Rightarrow, no head, from=1-1, to=5-5]
		\arrow["{\bigoplus_i(e\otimes N_i)}"{description}, from=1-1, to=3-5]
	\end{tikzcd}\]
	The top triangle commutes since $E\otimes-$ preserves coproducts, as it is a left adjoint. The bottom triangle commutes by unitality of each of the $\kappa_i$'s. To see the second coherence diagram commutes, consider the following diagram:
	% https://q.uiver.app/#q=WzAsOCxbMCwwLCJFXFxvdGltZXMgRVxcb3RpbWVzXFxiaWdvcGx1c19pTl9pIl0sWzIsMCwiRVxcb3RpbWVzIFxcYmlnb3BsdXNfaU5faSJdLFswLDEsIkVcXG90aW1lc1xcYmlnb3BsdXNfaShFXFxvdGltZXMgTl9pKSJdLFswLDIsIkVcXG90aW1lc1xcYmlnb3BsdXNfaU5faSJdLFsxLDIsIlxcYmlnb3BsdXNfaShFXFxvdGltZXMgTl9pKSJdLFsyLDIsIlxcYmlnb3BsdXNfaU5faSJdLFsyLDEsIlxcYmlnb3BsdXNfaShFXFxvdGltZXMgTl9pKSJdLFsxLDEsIlxcYmlnb3BsdXNfaShFXFxvdGltZXMgRVxcb3RpbWVzIE5faSkiXSxbMCwxLCJcXG11XFxvcGx1c1xcYmlnb3BsdXNfaU5faSJdLFswLDIsIkVcXG90aW1lc1xcY29uZyIsMl0sWzIsMywiRVxcb3RpbWVzXFxiaWdvcGx1c19pXFxrYXBwYV9pIiwyXSxbMyw0LCJcXGNvbmciLDJdLFs0LDUsIlxcYmlnb3BsdXNfaVxca2FwcGFfaSIsMl0sWzEsNiwiXFxjb25nIl0sWzYsNSwiXFxiaWdvcGx1c19pXFxrYXBwYV9pIl0sWzAsNywiXFxjb25nIl0sWzIsNywiXFxjb25nIl0sWzcsNiwiXFxiaWdvcGx1c19pKFxcbXVcXG90aW1lcyBOX2kpIl0sWzcsNCwiXFxiaWdvcGx1c19pKEVcXG90aW1lc1xca2FwcGFfaSkiLDJdXQ==
	\[\begin{tikzcd}
		{E\otimes E\otimes\bigoplus_iN_i} && {E\otimes \bigoplus_iN_i} \\
		{E\otimes\bigoplus_i(E\otimes N_i)} & {\bigoplus_i(E\otimes E\otimes N_i)} & {\bigoplus_i(E\otimes N_i)} \\
		{E\otimes\bigoplus_iN_i} & {\bigoplus_i(E\otimes N_i)} & {\bigoplus_iN_i}
		\arrow["{\mu\oplus\bigoplus_iN_i}", from=1-1, to=1-3]
		\arrow["E\otimes\cong"', from=1-1, to=2-1]
		\arrow["{E\otimes\bigoplus_i\kappa_i}"', from=2-1, to=3-1]
		\arrow["\cong"', from=3-1, to=3-2]
		\arrow["{\bigoplus_i\kappa_i}"', from=3-2, to=3-3]
		\arrow["\cong", from=1-3, to=2-3]
		\arrow["{\bigoplus_i\kappa_i}", from=2-3, to=3-3]
		\arrow["\cong", from=1-1, to=2-2]
		\arrow["\cong", from=2-1, to=2-2]
		\arrow["{\bigoplus_i(\mu\otimes N_i)}", from=2-2, to=2-3]
		\arrow["{\bigoplus_i(E\otimes\kappa_i)}"', from=2-2, to=3-2]
	\end{tikzcd}\]
	The bottom right square commutes by coherence for the $\kappa_i$'s. Every other region commutes since $-\otimes-$ preserves colimits in each variable. Thus $N=\bigoplus_iN_i$ is indeed an $E$-module object, as desired.

	Now, we claim that $(N,\kappa)$ is the coproduct of the $(N_i,\kappa_i)$'s in $E\text-\Mod$. First, we need to show that the canonical maps $\iota_i:N_i\into N$ are morphisms in $E\text-\Mod$ for all $i\in I$. To see $\iota_i$ is a homomorphism of $E$-module objects, consider the following diagram:
	% https://q.uiver.app/#q=WzAsNSxbMCwwLCJFXFxvdGltZXMgTl9pIl0sWzIsMCwiRVxcb3RpbWVzXFxiaWdvcGx1c19pTl9pIl0sWzAsMiwiTl9pIl0sWzIsMiwiXFxiaWdvcGx1c19pTl9pIl0sWzIsMSwiXFxiaWdvcGx1c19pKEVcXG90aW1lcyBOX2kpIl0sWzAsMSwiRVxcb3RpbWVzXFxpb3RhX2kiXSxbMCwyLCJcXGthcHBhX2kiLDJdLFsyLDMsIlxcaW90YV9pIiwyLHsic3R5bGUiOnsidGFpbCI6eyJuYW1lIjoiaG9vayIsInNpZGUiOiJ0b3AifX19XSxbMSw0LCJcXGNvbmciXSxbNCwzLCJcXGJpZ29wbHVzX2lcXGthcHBhX2kiXSxbMCw0LCJcXGlvdGFfe0VcXG90aW1lcyBOX2l9IiwxLHsic3R5bGUiOnsidGFpbCI6eyJuYW1lIjoiaG9vayIsInNpZGUiOiJ0b3AifX19XV0=
	\[\begin{tikzcd}
		{E\otimes N_i} && {E\otimes\bigoplus_iN_i} \\
		&& {\bigoplus_i(E\otimes N_i)} \\
		{N_i} && {\bigoplus_iN_i}
		\arrow["{E\otimes\iota_i}", from=1-1, to=1-3]
		\arrow["{\kappa_i}"', from=1-1, to=3-1]
		\arrow["{\iota_i}"', hook, from=3-1, to=3-3]
		\arrow["\cong", from=1-3, to=2-3]
		\arrow["{\bigoplus_i\kappa_i}", from=2-3, to=3-3]
		\arrow["{\iota_{E\otimes N_i}}"{description}, hook, from=1-1, to=2-3]
	\end{tikzcd}\]
	The top triangle commutes by additivity of $E\otimes-$. The bottom trapezoid commutes since, by univeral property of the coproduct, $\bigoplus_i\kappa_i$ is the unique arrow which makes the trapezoid commute for all $i\in I$. Now, it remains to show that given an $E$-module object $(N',\kappa')$ and homomorphisms $f_i:N_i\to N'$ of $E$-module objects for all $i\in I$, that the unique arrow $f:N\to N'$ in $\cSH$ satisfying $f\circ\iota_i=f_i$ for all $i\in I$ is a homomorphism of $E$-module objects, so that $N$ is actually the coproduct of the $N_i$'s. To see this, first let $h:\bigoplus_i(E\otimes N_i)\to E\otimes N'$ be the arrow determined by the maps $E\otimes N_i\xrightarrow{E\otimes f_i}E\otimes N'$. Then consider the following diagram:
	% https://q.uiver.app/#q=WzAsNyxbMCwwLCJFXFxvdGltZXNcXGJpZ29wbHVzX2lOX2kiXSxbMiwwLCJFXFxvdGltZXMgTiciXSxbMCwxLCJcXGJpZ29wbHVzX2koRVxcb3RpbWVzIE5faSkiXSxbMCwzLCJcXGJpZ29wbHVzX2lOX2kiXSxbMiwzLCJOJyJdLFsxLDEsIlxcYmlnb3BsdXNfaShFXFxvdGltZXMgTicpIl0sWzEsMiwiXFxiaWdvcGx1c19pTiciXSxbMCwxLCJFXFxvdGltZXMgZiJdLFswLDIsIlxcY29uZyIsMl0sWzIsMywiXFxiaWdvcGx1c19pXFxrYXBwYV9pIiwyXSxbMyw0LCJmIiwyXSxbMSw0LCJcXGthcHBhJyJdLFsyLDUsIlxcYmlnb3BsdXNfaShFXFxvdGltZXMgZl9pKSIsMl0sWzIsMSwiaCJdLFs1LDEsIlxcbmFibGEiLDJdLFs1LDYsIlxcYmlnb3BsdXNfaVxca2FwcGEnIl0sWzYsNCwiXFxuYWJsYSJdLFszLDYsIlxcYmlnb3BsdXNfaWZfaSIsMV1d
	\[\begin{tikzcd}
		{E\otimes\bigoplus_iN_i} && {E\otimes N'} \\
		{\bigoplus_i(E\otimes N_i)} & {\bigoplus_i(E\otimes N')} \\
		& {\bigoplus_iN'} \\
		{\bigoplus_iN_i} && {N'}
		\arrow["{E\otimes f}", from=1-1, to=1-3]
		\arrow["\cong"', from=1-1, to=2-1]
		\arrow["{\bigoplus_i\kappa_i}"', from=2-1, to=4-1]
		\arrow["f"', from=4-1, to=4-3]
		\arrow["{\kappa'}", from=1-3, to=4-3]
		\arrow["{\bigoplus_i(E\otimes f_i)}"', from=2-1, to=2-2]
		\arrow["h", from=2-1, to=1-3]
		\arrow["\nabla"', from=2-2, to=1-3]
		\arrow["{\bigoplus_i\kappa'}", from=2-2, to=3-2]
		\arrow["\nabla", from=3-2, to=4-3]
		\arrow["{\bigoplus_if_i}"{description}, from=4-1, to=3-2]
	\end{tikzcd}\]
	The top triangle commutes by additivity of $E\otimes-$. The triangle below that commutes by the universal property of the coproduct, since it is straightforward to check that $\nabla\circ\bigoplus_i(E\otimes f_i)$ and $h$ both satisfy the universal property of the colimit. The left trapezoid commutes by functoriality of $-\oplus-$ and the fact that $f_i$ is a homomorphism of $E$-module objects for all $i$ in $I$. The right trapezoid commutes by naturality of $\nabla$. Finally, the bottom triangle commutes by the universal product of the coproduct, by showing that $\nabla\circ\bigoplus_if_i$ in place of $f$ also satisfies the universal property of the colimit. Hence $f$ is inded a homomorphism of $E$-module objects, as desired.

	To recap, we have shown that given a set of $E$-module objects $\{(N_i,\kappa_i)\}_{i\in I}$, the inclusion maps $\iota_i:N_i\into \bigoplus_iN_i$ are morphisms in $E\text-\Mod$, and that given morphisms $f_i:(N_i,\kappa_i)\to(N',\kappa')$ for all $i\in I$, the unique induced map $\bigoplus_iN_i\to N'$ is a morphism in $E\text-\Mod$. Thus, $E\text-\Mod$ does indeed have arbitrary coproducts, and the forgetful functor $E\text-\Mod\to\cSH$ preserves them.
\end{proof}

\begin{proposition}\label{E-Mod,free,forgetful_are_additive}
	Suppose that $\cC$ is an additive closed symmetric monoidal category, and let $(E,\mu,e)$ be a monoid object in $\cC$. Then $E\text-\Mod$ is itself an additive category, so that in particular the forgetful functor $E\text-\Mod\to\cC$ and the free functor $\cC\to E\text-\Mod$ (\autoref{free_forgetful_E-Mod}) are additive.
\end{proposition}
\begin{proof}
	It is a general fact that adjoint functors between additive categories are necessarily additive. In order to show $E\text-\Mod$ is an additive category, it suffices to show it has finite coproducts, that $\Hom_{E\text-\Mod}(N,N')$ is an abelian group for all $E$-modules $N$ and $N'$, and that composition is bilinear. We know that $E\text-\Mod$ has coproducts which are preserved by the forgetful functor $E\text-\Mod\to\cC$ by \autoref{coproduct_of_E_modules_is_coproduct_in_E_mod} (which is clearly faithful). Thus, because $\cC$ is $\Ab$-enriched and $\Hom_{E\text-\Mod}(N,N')\sseq\cC(N,N')$, it suffices to show that $\Hom_{E\text-\Mod}(N,N')$ is closed under addition and taking inverses. To see the former, let $f,g:N\to N'$ be $E$-module homomorphisms, and consider the following diagram:
	% https://q.uiver.app/#q=WzAsMTAsWzAsMCwiRVxcb3RpbWVzIE4iXSxbMSwwLCJFXFxvdGltZXMgKE5cXG9wbHVzIE4pIl0sWzIsMCwiRVxcb3RpbWVzIChOJ1xcb3BsdXMgTicpIl0sWzMsMCwiRVxcb3RpbWVzIE4nIl0sWzMsMiwiTiciXSxbMCwyLCJOIl0sWzEsMSwiKEVcXG90aW1lcyBOKVxcb3BsdXMoRVxcb3RpbWVzIE4pIl0sWzIsMSwiKEVcXG90aW1lcyBOJylcXG90aW1lcyAoRVxcb3RpbWVzIE4nKSJdLFsxLDIsIk5cXG9wbHVzIE4iXSxbMiwyLCJOJ1xcb3BsdXMgTiciXSxbMCwxLCJFXFxvdGltZXNcXERlbHRhX04iXSxbMSwyLCJFXFxvdGltZXMgKGZcXG9wbHVzIGcpIl0sWzIsMywiRVxcb3RpbWVzIFxcbmFibGFfe04nfSJdLFszLDQsIlxca2FwcGEnIl0sWzAsNSwiXFxrYXBwYSIsMl0sWzAsNiwiXFxEZWx0YV97RVxcb3RpbWVzIE59IiwyXSxbNiw3LCIoRVxcb3RpbWVzIGYpXFxvcGx1cyhFXFxvdGltZXMgZykiXSxbMiw3LCJcXGNvbmciXSxbMSw2LCJcXGNvbmciLDJdLFs3LDMsIlxcbmFibGFfe0VcXG90aW1lcyBOJ30iLDJdLFs1LDgsIlxcRGVsdGFfTiJdLFs4LDksImZcXG9wbHVzIGciXSxbOSw0LCJcXG5hYmxhX3tOJ30iXSxbNiw4LCJcXGthcHBhXFxvcGx1c1xca2FwcGEiLDJdLFs3LDksIlxca2FwcGEnXFxvcGx1c1xca2FwcGEnIl1d
	\[\begin{tikzcd}
		{E\otimes N} & {E\otimes (N\oplus N)} & {E\otimes (N'\oplus N')} & {E\otimes N'} \\
		& {(E\otimes N)\oplus(E\otimes N)} & {(E\otimes N')\otimes (E\otimes N')} \\
		N & {N\oplus N} & {N'\oplus N'} & {N'}
		\arrow["{E\otimes\Delta_N}", from=1-1, to=1-2]
		\arrow["{E\otimes (f\oplus g)}", from=1-2, to=1-3]
		\arrow["{E\otimes \nabla_{N'}}", from=1-3, to=1-4]
		\arrow["{\kappa'}", from=1-4, to=3-4]
		\arrow["\kappa"', from=1-1, to=3-1]
		\arrow["{\Delta_{E\otimes N}}"', from=1-1, to=2-2]
		\arrow["{(E\otimes f)\oplus(E\otimes g)}", from=2-2, to=2-3]
		\arrow["\cong", from=1-3, to=2-3]
		\arrow["\cong"', from=1-2, to=2-2]
		\arrow["{\nabla_{E\otimes N'}}"', from=2-3, to=1-4]
		\arrow["{\Delta_N}", from=3-1, to=3-2]
		\arrow["{f\oplus g}", from=3-2, to=3-3]
		\arrow["{\nabla_{N'}}", from=3-3, to=3-4]
		\arrow["\kappa\oplus\kappa"', from=2-2, to=3-2]
		\arrow["{\kappa'\oplus\kappa'}", from=2-3, to=3-3]
	\end{tikzcd}\]
	The outermost trapezoids commute by naturality of $\Delta$ and $\nabla$. The triangles in the top corners and the top middle rectangle commute by additivity of $E\otimes-$. Finally, the middle bottom rectangle commutes by functoriality of $-\oplus-$ and $-\otimes-$, and the fact that $f$ and $g$ are $E$-module homomorphisms. Commutativity of the above diagram shows that $f+g$ is a homomorphism of $E$-modules as desired. Finally, to see $-f$ is a $E$-module homomorphism if $f$ is, we would like to show that $\kappa'\circ(E\otimes(-f))=(-f)\circ\kappa$. This follows by the fact that $\kappa'\circ(E\otimes f)=f\circ\kappa$ and additivity of $-\otimes-$ and composition.
\end{proof}

\end{document}
