\documentclass[../main.tex]{subfiles}
\begin{document}

One very important class of objects in $\cSH$ are the \emph{cellular} objects. Intuitively, these are the objects that can be built out of spheres via taking coproducts and (co)fibers.

\begin{definition}\label{cellular}
	Define the class of \emph{cellular} objects in $\cSH$ to be the smallest class of objects such that:
	\begin{enumerate}
		\item For all $a\in A$, the $a$-sphere $S^a$ is cellular.
		\item If we have a distinguished triangle
		\[X\to Y\to Z\to\Sigma X\]
		such that two of the three objects $X$, $Y$, and $Z$ are cellular, than the third object is also cellular.
		\item Given a collection of cellular objects $X_i$ indexed by some (small) set $I$, the object $\bigoplus_{i\in I} X_i$ is cellular (recall we have chosen $\cSH$ to have arbitrary coproducts).
	\end{enumerate}
	We write $\cSH\text-\Cell$ to denote the full subcategory of $\cSH$ on the cellular objects.
\end{definition}

We devote the rest of the section to proving some important facts about cellular objects. These should be familiar to anyone acquainted with the usual notion of cellular spaces (CW complexes).

\begin{lemma}\label{cellular_closed_under_iso}
	Let $X$ and $Y$ be two isomorphic objects in $\cSH$. Then $X$ is cellular iff $Y$ is cellular.
\end{lemma}
\begin{proof}
	Assume we have an isomorphism $f:X\xr\cong Y$ and that $X$ is cellular. Then consider the following commutative diagram
	% https://q.uiver.app/#q=WzAsOCxbMCwwLCJYIl0sWzEsMCwiWSJdLFsyLDAsIjAiXSxbMywwLCJcXFNpZ21hIFgiXSxbMSwxLCJYIl0sWzAsMSwiWCJdLFsyLDEsIjAiXSxbMywxLCJcXFNpZ21hIFgiXSxbMCwxLCJmIl0sWzEsMl0sWzIsM10sWzEsNCwiZl57LTF9Il0sWzAsNSwiIiwyLHsibGV2ZWwiOjIsInN0eWxlIjp7ImhlYWQiOnsibmFtZSI6Im5vbmUifX19XSxbNSw0LCIiLDIseyJsZXZlbCI6Miwic3R5bGUiOnsiaGVhZCI6eyJuYW1lIjoibm9uZSJ9fX1dLFs0LDZdLFs2LDddLFszLDcsIiIsMSx7ImxldmVsIjoyLCJzdHlsZSI6eyJoZWFkIjp7Im5hbWUiOiJub25lIn19fV0sWzIsNiwiIiwxLHsibGV2ZWwiOjIsInN0eWxlIjp7ImhlYWQiOnsibmFtZSI6Im5vbmUifX19XV0=
	\[\begin{tikzcd}
		X & Y & 0 & {\Sigma X} \\
		X & X & 0 & {\Sigma X}
		\arrow["f", from=1-1, to=1-2]
		\arrow[from=1-2, to=1-3]
		\arrow[from=1-3, to=1-4]
		\arrow["{f^{-1}}", from=1-2, to=2-2]
		\arrow[Rightarrow, no head, from=1-1, to=2-1]
		\arrow[Rightarrow, no head, from=2-1, to=2-2]
		\arrow[from=2-2, to=2-3]
		\arrow[from=2-3, to=2-4]
		\arrow[Rightarrow, no head, from=1-4, to=2-4]
		\arrow[Rightarrow, no head, from=1-3, to=2-3]
	\end{tikzcd}\]
	The bottom row is distinguished by axiom TR1 for a triangulated category. Hence since $X$ is cellular, $0$ is also cellular, since the class of cellular objects satisfies two-of-three for distinguished triangles. Furthermore, since the vertical arrows are all isomorphisms, the top row is distinguished as well, by axiom TR0. Thus again by two-of-three, since $X$ and $0$ are cellular, so is $Y$, as desired.
\end{proof}

\begin{lemma}\label{cellular_closed_under_tensor}
	Let $X$ and $Y$ be cellular objects in $\cSH$. Then $X\otimes Y$ is cellular.
\end{lemma}
\begin{proof}
	Let $E$ be a cellular object in $\cSH$, and let $\cE$ be the collection of objects $X$ in $\cSH$ such that $E\otimes X$ is cellular. First of all, suppose we have a distinguished triangle
	\[X\to Y\to Z\to\Sigma X\]
	such that two of three of $X$, $Y$, and $Z$ belong to $\cE$. Then since $\cSH$ is tensor triangulated, we have a distinguished triangle
	\[E\otimes X\to E\otimes Y\to E\otimes Z\to \Sigma(E\otimes X).\]
	Per our assumptions, two of three of $E\otimes X$, $E\otimes Y$, and $E\otimes Z$ are cellular, so that the third is by definition. Thus, all three of $X$, $Y$, and $Z$ belong to $\cE$ if two of them do.

	Second of all, suppose we have a family $X_i$ of objects in $\cE$ indexed by some (small) set $I$, and set $X:=\bigoplus_iX_i$. Then we'd like to show $X$ belongs to $\cE$, i.e., that $E\otimes X$ is cellular. Indeed,
	\[E\otimes X=E\otimes\(\bigoplus_iX_i\)\cong\bigoplus_i(E\otimes X_i),\]
	where the isomorphism is given by the fact that $\cSH$ is monoidal closed, so $E\otimes-$ preserves arbitrary colimits as it is a left adjoint. Per our assumption, since each $E\otimes X_i$ is cellular, the rightmost object is cellular, since the class of cellular objects is closed under taking arbitrary coproducts, by definition. Hence $E\otimes X$ is cellular by \autoref{cellular_closed_under_iso}.

	Finally, we would like to show that each $S^a$ belongs to $\cE$, i.e., that $S^a\otimes E$ is cellular for all $a\in A$. When $E=S^b$ for some $b\in A$, this is clearly true, since $S^b\otimes S^a\cong S^{a+b}$, which is cellular by definition, so that $S^b\otimes S^a$ is cellular by \autoref{cellular_closed_under_iso}. Thus by what we have shown, the class of objects $X$ for which $S^a\otimes X$ is cellular contains every cellular object. Hence in particular $E\otimes S^a\cong S^a\otimes E$ is cellular for all $a\in A$, as desired.
\end{proof}

\begin{lemma}\label{cellular_pi*=0_implies_contractible}
	Let $W$ be a cellular object in $\cSH$ such that $\pi_*(W)=0$. Then $W\cong 0$.
\end{lemma}
\begin{proof}
	Let $\cE$ be the collection of all $X$ in $\cSH$ such that and $[\Sigma^nX,W]=0$ for all $n\in\bZ$ (where for $n>0$ we define $\Sigma^{-n}:=\Omega^{n}=(S^{-\1}\otimes-)^n$). We claim $\cE$ contains every cellular object in $\cSH$. First of all, each $S^a$ belongs to $\cE$, as 
	\[[\Sigma^nS^a,W]\cong[S^\n\otimes S^a,W]\cong[S^{a+\n},W]\leq\pi_*(W)=0.\] 
	Furthermore, suppose we are given a distinguished triangle
	\[X\to Y\to Z\to\Sigma X\]
	such that two of three of $X$, $Y$, and $Z$ belong to $\cE$. By \autoref{dist_tri_LES}, for all $n\in\bZ$ we get an exact sequence of abelian groups
	\[[\Sigma^{n+1}X,W]\to[\Sigma^nZ,W]\to[\Sigma^nY,W]\to[\Sigma^nX,W]\to[\Sigma^{n-1}Z,W].\]
	Clearly if any two of three of $X$, $Y$, and $Z$ belong to $\cE$, then by exactness of the above sequence all three of the middle terms will be zero, so that the third object will belong to $\cE$ as well. Finally, suppose we have a collection of objects $X_i$ in $\cE$ indexed by some small set $I$. Then
	\[\left[\Sigma^n\bigoplus_iX_i,W\right]\cong\left[\bigoplus_i\Sigma^nX_i,W\right]\cong\prod_i[\Sigma^nX_i,W]=\prod_i0=0,\]
	where the first isomorphism follows by the fact that $\Sigma^n$ is apart of an adjoint equivalence (\autoref{Sigma^a,Sigma^-a_adjoint_equiv}), so it preserves arbitrary colimits.

	Thus, by definition of cellularity, $\cE$ contains every cellular object. In particular, $\cE$ contains $W$, so that $[W,W]=0$, meaning $\id_W=0$, so we have a commutative diagram
	% https://q.uiver.app/#q=WzAsNCxbMCwxLCJXIl0sWzEsMCwiMCJdLFsyLDEsIlciXSxbMywwLCIwIl0sWzAsMV0sWzEsMl0sWzIsM10sWzEsMywiIiwwLHsibGV2ZWwiOjIsInN0eWxlIjp7ImhlYWQiOnsibmFtZSI6Im5vbmUifX19XSxbMCwyLCIiLDAseyJsZXZlbCI6Miwic3R5bGUiOnsiaGVhZCI6eyJuYW1lIjoibm9uZSJ9fX1dXQ==
	\[\begin{tikzcd}
		& 0 && 0 \\
		W && W
		\arrow[from=2-1, to=1-2]
		\arrow[from=1-2, to=2-3]
		\arrow[from=2-3, to=1-4]
		\arrow[Rightarrow, no head, from=1-2, to=1-4]
		\arrow[Rightarrow, no head, from=2-1, to=2-3]
	\end{tikzcd}\]
	Hence the diagonals exhibit isomorphisms between $0$ and $W$, as desired.
\end{proof}

\begin{theorem}\label{whitehead}
	Let $X$ and $Y$ be cellular objects in $\cSH$, and suppose $f:X\to Y$ is a morphism such that $f_*:\pi_*(X)\to\pi_*(Y)$ is an isomorphism. Then $f$ is an isomorphism.
\end{theorem}
\begin{proof}
	By axiom TR2 for a triangulated category, we have a distinguished triangle
	\[X\xr fY\xr gC_f\xr h\Sigma X.\]
	First of all, note that by definition since $X$ and $Y$ are cellular, so is $C_f$. We claim $\pi_*(C_f)=0$. Indeed, given $a\in A$, by axiom TR4 for a triangulated category and the fact that distinguished triangles are exact, the following sequence of abelian groups is exact:
	\[[S^a,X]\xr{f_*}[S^a,Y]\xr{g_*}[S^a,C_f]\xr{h_*}[S^a,\Sigma X]\xr{{\Sigma f}_*}[S^a,\Sigma Y].\]
	where the first arrow is and last arrows are isomorphisms, per our assumption that $f$ is an isomorphism. Then by exactness we have $\imm h_*=\ker({\Sigma f}_*)=0$. Yet we also have $\ker g_*=\imm f_*=[S^a,Y]$, so that $\ker h_*=\imm g_*=0$. It is only possible that $\ker h_*=\imm h_*=0$ if $[S^a,C_f]=0$. Thus, we have shown $\pi_*(C_f)=0$, and $C_f$ is cellular, so by \autoref{cellular_pi*=0_implies_contractible} there is an isomorphism $C_f\cong 0$. Now consider the following diagram:
	% https://q.uiver.app/#q=WzAsOCxbMCwwLCJYIl0sWzAsMSwiWSJdLFsxLDEsIlkiXSxbMSwwLCJZICJdLFsyLDEsIjAiXSxbMiwwLCJDX2YiXSxbMywwLCJcXFNpZ21hIFgiXSxbMywxLCJcXFNpZ21hIFkiXSxbMCwxLCJmIl0sWzEsMiwiIiwwLHsibGV2ZWwiOjIsInN0eWxlIjp7ImhlYWQiOnsibmFtZSI6Im5vbmUifX19XSxbMCwzLCJmIl0sWzMsMiwiIiwyLHsibGV2ZWwiOjIsInN0eWxlIjp7ImhlYWQiOnsibmFtZSI6Im5vbmUifX19XSxbMiw0XSxbMyw1XSxbNSw2XSxbNiw3LCJcXFNpZ21hIGYiXSxbNCw3XSxbNSw0LCJcXGNvbmciXV0=
	\[\begin{tikzcd}
		X & {Y } & {C_f} & {\Sigma X} \\
		Y & Y & 0 & {\Sigma Y}
		\arrow["f", from=1-1, to=2-1]
		\arrow[Rightarrow, no head, from=2-1, to=2-2]
		\arrow["f", from=1-1, to=1-2]
		\arrow[Rightarrow, no head, from=1-2, to=2-2]
		\arrow[from=2-2, to=2-3]
		\arrow[from=1-2, to=1-3]
		\arrow[from=1-3, to=1-4]
		\arrow["{\Sigma f}", from=1-4, to=2-4]
		\arrow[from=2-3, to=2-4]
		\arrow["\cong", from=1-3, to=2-3]
	\end{tikzcd}\]
	The middle square commutes since $0$ is terminal, while the right square commutes since $C_f\cong0$ is initial. The top row is distinguished by assumption. The bottom row is distinguished by axiom TR2. Then since the middle two vertical arrows are isomorphisms, by \autoref{2-of-3-dist_tri-lemma}, $f$ is an isomorphism as well, as desired.
\end{proof}

\begin{lemma}\label{cellular_idempotent_splits_cellularly}
	Let $e:X\to X$ be an idempotent morphism in $\cSH$, i.e., $e\circ e=e$. Then this idempotent splits, meaning $e$ factors as
	\[X\xr rY\xr\iota X\]
	for some object $Y$ and morphisms $r$ and $\iota$ with $r\circ\iota=\id_Y$. Furthermore, if $X$ is cellular than so is $Y$.
\end{lemma}
\begin{proof}
	In \cite[Proposition 1.6.8]{Neeman_2001}, it is shown that idempotents split in triangulated categories with countable coproducts, and in particular, the object $Y$ through which the splitting factors may be taken as the homotopy colimit of the sequence
	\[X\xr eX\xr eX\xr eX\xr e X\to\cdots.\]
	Thus since $\cSH$ is triangulated and has arbitrary coproducts, given an idempotent $e:X\to X$ in $\cSH$, $e$ splits as desired. Furthermore, the splitting factors through the homotopy limit $Y$ of the above sequence, so we have a distinguished triangle in $\cSH$
	\[\bigoplus_{i=0}^\infty X\to\bigoplus_{i=0}^\infty X\to Y\to\Sigma(\bigoplus_{i=0}^\infty X).\]
	Then if $X$ is cellular, by definition $\bigoplus_{i=0}^\infty X$ is as well. Thus by 2-of-3 for distinguished triangles for cellular objects, we would have that $Y$ is cellular as desired.
\end{proof}

\end{document}
