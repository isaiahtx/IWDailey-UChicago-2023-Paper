\documentclass[../main.tex]{subfiles}
\begin{document}

So far, we have already identified a good amount of structure on the objects $\pi_*(E)$, $E_*(E)=\pi_*(E\otimes E)$, and $E_*(X)$ for $E$ a (commutative) monoid object and $X$ an object in $\cSH$. Namely, we have shown that $\pi_*(E)$ and $E_*(E)$ are canonically $A$-graded anticommutative algebras over the stable homotopy ring (\autoref{pi_*:CMon_SH-->pi_*(S)-GrCAlg}), and that $E_*(X)$ is canonically an $A$-graded left $\pi_*(E)$-module (\autoref{module}). We would like to identify even more structure on these objects, namely, in \Cref{section:dual_E-Steenrod_algebra}, we will show that the pair $(E_*(E),\pi_*(E))$ is an \emph{$A$-graded anticommutative Hopf algebroid}, over which $E_*(X)$ is an $A$-graded left comodule. To that end, we need two important theorems, namely, we need analogs of the K\"unneth isomorphism and the universal coefficient theorem from algebraic topology. This section is dedicated to formulating and proving these theorems. The proofs of these theorems are arguably the most technical and difficult in this paper, so we will be especially careful to give them in their full and explicit detail.

\subsection{A K\"unneth isomorphism}

The goal of this subsection will be to prove the following theorem, which, given a monoid object $(E,\mu,e)$ and objects $Z$ and $W$ in $\cSH$, relates the $Z$-homology of $E$ with the $E$-homology of $W$ to $\pi_*(Z\otimes E\otimes W)$:

\begin{theorem}[The K\"unneth isomorphism]\label{Kunneth_theorem}
    Let $(E,\mu,e)$ be a monoid object and $Z$ and $W$ objects in $\cSH$. Then if\begin{itemize}
        \item $Z_*(E)$ is a flat right $\pi_*(E)$-module (via \autoref{module}) and $W$ is cellular (\autoref{cellular}), \emph{or}
        \item $E_*(W)$ is a flat left $\pi_*(E)$-module (via \autoref{module}) and $Z$ is cellular,
    \end{itemize}
    then there is a natural $A$-graded isomorphism of $A$-graded abelian groups, called the \emph{K\"unneth isomorphism}:
    \[\Phi_{Z,W}:Z_*(E)\otimes_{\pi_*(E)}E_*(W)\xr\cong\pi_*(Z\otimes E\otimes W).\]
\end{theorem}

There is much work to be done. First, we construct the map and show it is natural:

\begin{proposition}\label{Kunneth_map}
    Let $(E,\mu,e)$ be a monoid object and $Z$ and $W$ be objects in $\cSH$. Then there is an $A$-graded homomorphism of abelian groups
    \[\Phi_{Z,W}:Z_*(E)\otimes_{\pi_*(E)}E_*(W)\to\pi_*(Z\otimes E\otimes W)\]
    which given homogeneous elements $x:S^a\to Z\otimes E$ in $Z_*(E)=\pi_*(Z\otimes E)$  and $y:S^b\to E\otimes W$ in $E_*(W)=\pi_*(E\otimes W)$, sends the homogeneous pure tensor $x\otimes y$ in $Z_*(E)\otimes_{\pi_*(E)}E_*(W)$ to the composition
    \[\Phi_{Z,W}(x\otimes y):S^{a+b}\xr{\phi_{a,b}}S^a\otimes S^b\xr{x\otimes y}Z\otimes E\otimes E\otimes W\xr{Z\otimes\mu\otimes W}Z\otimes E\otimes W\]
    Furthermore, this homomorphism is natural in both $Z$ and $W$.
\end{proposition}
\begin{proof}
	By \autoref{tensor_lift_of_A_graded_is_A_graded}, in order to get an $A$-graded homomorphism
	\[\Phi_{Z,W}:Z_*(E)\otimes_{\pi_*(E)}E_*(W)\to\pi_*(Z\otimes E\otimes W),\]
	it suffices to define an assignment $P:Z_*(E)\times E_*(W)\to\pi_*(Z\otimes E\otimes W)$ on homogeneous elements (which we have), and show that it is additive in each argument for homogeneous elements of the same degree, and that for all homogeneous $z\in Z_*(E)$, $r\in\pi_*(E)$, and $w\in E_*(W)$ that $P(zr,w)=P(z,rw)$, where concatenation denotes the module action.
	
	First, note that by \autoref{bilinear} it is straightforward to see that the assignment commutes with addition of maps of the same degree in each argument. Now, let $a,b,c\in A$, $z:S^a\to Z\otimes E$, $w:S^b\to E\otimes W$, and $r:S^c\to E$. Then we wish to show $P(zr,w)=P(z,rw)$. Consider the following diagram (where here we are passing to a symmetric strict monoidal category):
	% https://q.uiver.app/#q=WzAsNixbMCwxLCJTXnthK2IrY30iXSxbMSwxLCJTXmFcXG90aW1lcyBTXmNcXG90aW1lcyBTXmIiXSxbMiwxLCJaXFxvdGltZXMgRVxcb3RpbWVzIEVcXG90aW1lcyBFXFxvdGltZXMgVyJdLFszLDAsIlpcXG90aW1lcyBFXFxvdGltZXMgRVxcb3RpbWVzIFciXSxbMywxLCJaXFxvdGltZXMgRVxcb3RpbWVzIFciXSxbMywyLCJaXFxvdGltZXMgRVxcb3RpbWVzIEVcXG90aW1lcyBXIl0sWzAsMSwiXFxjb25nIl0sWzEsMiwielxcb3RpbWVzIHJcXG90aW1lcyB3Il0sWzIsMywiWlxcb3RpbWVzIFxcbXVcXG90aW1lcyBFXFxvdGltZXMgVyJdLFszLDQsIlpcXG90aW1lcyBcXG11XFxvdGltZXMgVyJdLFsyLDUsIlpcXG90aW1lcyBFXFxvdGltZXMgXFxtdVxcb3RpbWVzIFciLDJdLFs1LDQsIlpcXG90aW1lcyBcXG11XFxvdGltZXMgVyIsMl1d
	\[\begin{tikzcd}
		&&& {Z\otimes E\otimes E\otimes W} \\
		{S^{a+b+c}} & {S^a\otimes S^c\otimes S^b} & {Z\otimes E\otimes E\otimes E\otimes W} & {Z\otimes E\otimes W} \\
		&&& {Z\otimes E\otimes E\otimes W}
		\arrow["\cong", from=2-1, to=2-2]
		\arrow["{z\otimes r\otimes w}", from=2-2, to=2-3]
		\arrow["{Z\otimes \mu\otimes E\otimes W}", from=2-3, to=1-4]
		\arrow["{Z\otimes \mu\otimes W}", from=1-4, to=2-4]
		\arrow["{Z\otimes E\otimes \mu\otimes W}"', from=2-3, to=3-4]
		\arrow["{Z\otimes \mu\otimes W}"', from=3-4, to=2-4]
	\end{tikzcd}\]
	It commutes by associativity of $\mu$. By functoriality of $-\otimes-$, the top composition is given by $P(zr,w)$ and the bottom composition is $P(z,rw)$, so they are equal as desired. Thus, by \autoref{tensor_lift_of_A_graded_is_A_graded} we get the desired $A$-graded homomorphism $\pi_*(Z\otimes E)\otimes_{\pi_*(E)}\pi_*(E\otimes W)\to\pi_*(Z\otimes E\otimes W)$.
	
	%In order to see this map is a homomorphism of left $\pi_*(E)$-modules, we must show that $z(x\cdot y)=zx\cdot y$, where $x$, $y$, and $z$ are defined as above. Now consider the following diagram:
	%% https://q.uiver.app/#q=WzAsNixbMCwxLCJTXnthK2IrY30iXSxbMSwxLCJTXmNcXG90aW1lcyBTXmFcXG90aW1lcyBTXmIiXSxbMiwxLCJFXFxvdGltZXMgRVxcb3RpbWVzIEVcXG90aW1lcyBFXFxvdGltZXMgWCJdLFszLDAsIkVcXG90aW1lcyBFXFxvdGltZXMgRVxcb3RpbWVzIFgiXSxbMywxLCJFXFxvdGltZXMgRVxcb3RpbWVzIFgiXSxbMywyLCJFXFxvdGltZXMgRVxcb3RpbWVzIEVcXG90aW1lcyBYIl0sWzAsMSwiXFxjb25nIl0sWzEsMiwielxcb3RpbWVzIHhcXG90aW1lcyB5Il0sWzIsMywiXFxtdVxcb3RpbWVzIEVcXG90aW1lcyBFXFxvdGltZXMgWCJdLFszLDQsIkVcXG90aW1lcyBcXG11XFxvdGltZXMgWCJdLFsyLDUsIkVcXG90aW1lcyBFXFxvdGltZXMgXFxtdVxcb3RpbWVzIFgiLDJdLFs1LDQsIlxcbXVcXG90aW1lcyBFXFxvdGltZXMgWCIsMl0sWzIsNCwiXFxtdVxcb3RpbWVzXFxtdVxcb3RpbWVzIFgiXV0=
	%\[\begin{tikzcd}
		%&&& {E\otimes E\otimes E\otimes X} \\
		%{S^{a+b+c}} & {S^c\otimes S^a\otimes S^b} & {E\otimes E\otimes E\otimes E\otimes X} & {E\otimes E\otimes X} \\
		%&&& {E\otimes E\otimes E\otimes X}
		%\arrow["\cong", from=2-1, to=2-2]
		%\arrow["{z\otimes x\otimes y}", from=2-2, to=2-3]
		%\arrow["{\mu\otimes E\otimes E\otimes X}", from=2-3, to=1-4]
		%\arrow["{E\otimes \mu\otimes X}", from=1-4, to=2-4]
		%\arrow["{E\otimes E\otimes \mu\otimes X}"', from=2-3, to=3-4]
		%\arrow["{\mu\otimes E\otimes X}"', from=3-4, to=2-4]
		%\arrow["{\mu\otimes\mu\otimes X}", from=2-3, to=2-4]
	%\end{tikzcd}\]
	%Commutativity of the triangles is functoriality of $-\otimes-$. By functoriality of $-\otimes-$, the top composition is $zx\cdot y$, and the bottom composition is $z(x\cdot y)$. Hence they are equal, as desired, so that the map we have constructed
	%\[E_*(E)\otimes_{\pi_*(E)}E_*(X)\to E_*(E\otimes X)\]
	%is indeed an $A$-graded homomorphism of left $A$-graded $\pi_*(E)$-modules.

	Next, we would like to show that this homomorphism is natural in $Z$. Let $f:Z\to Z'$ in $\cSH$. Then we would like to show the following diagram commutes:
	% https://q.uiver.app/#q=WzAsNCxbMCwwLCJcXHBpXyooWlxcb3RpbWVzIEUpXFxvdGltZXNfe1xccGlfKihFKX1cXHBpXyooRVxcb3RpbWVzIFcpIl0sWzAsMSwiXFxwaV8qKFonXFxvdGltZXMgRSlcXG90aW1lc197XFxwaV8qKEUpfVxccGlfKihFXFxvdGltZXMgVykiXSxbMSwxLCJcXHBpXyooWidcXG90aW1lcyBFXFxvdGltZXMgVykiXSxbMSwwLCJcXHBpXyooWlxcb3RpbWVzIEVcXG90aW1lcyBXKSJdLFswLDEsIlxccGlfKihmXFxvdGltZXMgRSlcXG90aW1lc1xccGlfKihFXFxvdGltZXMgVykiLDJdLFsxLDIsIlxcUGhpX3taJyxXfSJdLFszLDIsIlxccGlfKihmXFxvdGltZXMgRVxcb3RpbWVzIFcpIl0sWzAsMywiXFxQaGlfe1osV30iXV0=
	\begin{equation}\label{naturality_diagram_for_E*EoxE_*X-->E_*(EoxX)}\begin{tikzcd}
		{\pi_*(Z\otimes E)\otimes_{\pi_*(E)}\pi_*(E\otimes W)} & {\pi_*(Z\otimes E\otimes W)} \\
		{\pi_*(Z'\otimes E)\otimes_{\pi_*(E)}\pi_*(E\otimes W)} & {\pi_*(Z'\otimes E\otimes W)}
		\arrow["{\pi_*(f\otimes E)\otimes\pi_*(E\otimes W)}"', from=1-1, to=2-1]
		\arrow["{\Phi_{Z',W}}", from=2-1, to=2-2]
		\arrow["{\pi_*(f\otimes E\otimes W)}", from=1-2, to=2-2]
		\arrow["{\Phi_{Z,W}}", from=1-1, to=1-2]
	\end{tikzcd}\end{equation}
	As all the maps here are homomorphisms, in order to show it commutes, it suffices to chase generators around the diagram. In particular, suppose we are given $z:S^a\to Z\otimes E$ and $w:S^b\to E\otimes W$, and consider the following diagram exhibiting the two possible ways to chase $z\otimes w$ around the diagram (as usual, we are passing to a symmetric strict monoidal category):
	% https://q.uiver.app/#q=WzAsNixbMCwwLCJTXnthK2J9Il0sWzEsMCwiU15hXFxvdGltZXMgU15iIl0sWzIsMCwiWlxcb3RpbWVzIEVcXG90aW1lcyBFXFxvdGltZXMgVyJdLFszLDAsIlpcXG90aW1lcyBFXFxvdGltZXMgVyJdLFszLDEsIlonXFxvdGltZXMgRVxcb3RpbWVzIFciXSxbMiwxLCJaXFxvdGltZXMgRVxcb3RpbWVzIEVcXG90aW1lcyBXIl0sWzAsMSwiXFxwaGlfe2EsYn0iXSxbMSwyLCJ6XFxvdGltZXMgdyJdLFsyLDMsIlpcXG90aW1lcyBcXG11XFxvdGltZXMgVyJdLFszLDQsImZcXG90aW1lcyBFXFxvdGltZXMgVyJdLFsyLDUsImZcXG90aW1lcyBFXFxvdGltZXMgRVxcb3RpbWVzIFciLDJdLFs1LDQsIlpcXG90aW1lcyBcXG11XFxvdGltZXMgVyJdXQ==
	\[\begin{tikzcd}
		{S^{a+b}} & {S^a\otimes S^b} & {Z\otimes E\otimes E\otimes W} & {Z\otimes E\otimes W} \\
		&& {Z\otimes E\otimes E\otimes W} & {Z'\otimes E\otimes W}
		\arrow["{\phi_{a,b}}", from=1-1, to=1-2]
		\arrow["{z\otimes w}", from=1-2, to=1-3]
		\arrow["{Z\otimes \mu\otimes W}", from=1-3, to=1-4]
		\arrow["{f\otimes E\otimes W}", from=1-4, to=2-4]
		\arrow["{f\otimes E\otimes E\otimes W}"', from=1-3, to=2-3]
		\arrow["{Z\otimes \mu\otimes W}", from=2-3, to=2-4]
	\end{tikzcd}\]
	This diagram commutes by functoriality of $-\otimes-$. Thus we have that diagram (\ref{naturality_diagram_for_E*EoxE_*X-->E_*(EoxX)}) does indeed commute, so that $\Phi_{Z,W}$ is natural in $Z$ as desired. Showing that $\Phi_{Z,W}$ is natural in $W$ is entirely analagous.
\end{proof}

Now, before proving the K\"unneth map is an isomorphism under the conditions given in \autoref{Kunneth_theorem}, we prove the following lemmas:


\begin{lemma}\label{t_a's_commute_with_Kunneth_map}
	Let $(E,\mu,e)$ be a monoid object and $Z$ and $W$ be objects in $\cSH$. Then for all $a\in A$, the following diagram commutes
	% https://q.uiver.app/#q=WzAsNixbMCwwLCJaXyooRSlcXG90aW1lc197XFxwaV8qKEUpfUVfKihcXFNpZ21hXmFXKSJdLFsyLDAsIlpfKihFKVxcb3RpbWVzX3tcXHBpXyooRSl9RV97Ki1hfShXKSJdLFsyLDEsIlxccGlfeyotYX0oWlxcb3RpbWVzIEVcXG90aW1lcyBXKSJdLFswLDEsIlxccGlfKihaXFxvdGltZXMgRVxcb3RpbWVzXFxTaWdtYV5hVykiXSxbMCwyLCIoWlxcb3RpbWVzIEUpXyooXFxTaWdtYV5hVykiXSxbMiwyLCIoWlxcb3RpbWVzIEUpX3sqLWF9KFcpIl0sWzAsMSwiWl8qKEUpXFxvdGltZXNfe1xccGlfKihFKX10X2FeVyJdLFsxLDIsIlxcUGhpX3taLFd9Il0sWzAsMywiXFxQaGlfe1osXFxTaWdtYV5hV30iLDJdLFszLDQsIiIsMix7ImxldmVsIjoyLCJzdHlsZSI6eyJoZWFkIjp7Im5hbWUiOiJub25lIn19fV0sWzQsNSwidF9hXlciXSxbNSwyLCIiLDIseyJsZXZlbCI6Miwic3R5bGUiOnsiaGVhZCI6eyJuYW1lIjoibm9uZSJ9fX1dXQ==
	\[\begin{tikzcd}
		{Z_*(E)\otimes_{\pi_*(E)}E_*(\Sigma^aW)} && {Z_*(E)\otimes_{\pi_*(E)}E_{*-a}(W)} \\
		{\pi_*(Z\otimes E\otimes\Sigma^aW)} && {\pi_{*-a}(Z\otimes E\otimes W)} \\
		{(Z\otimes E)_*(\Sigma^aW)} && {(Z\otimes E)_{*-a}(W)}
		\arrow["{Z_*(E)\otimes_{\pi_*(E)}t_a^W}", from=1-1, to=1-3]
		\arrow["{\Phi_{Z,W}}", from=1-3, to=2-3]
		\arrow["{\Phi_{Z,\Sigma^aW}}"', from=1-1, to=2-1]
		\arrow[Rightarrow, no head, from=2-1, to=3-1]
		\arrow["{t_a^W}", from=3-1, to=3-3]
		\arrow[Rightarrow, no head, from=3-3, to=2-3]
	\end{tikzcd}\]
	where the maps $t_a$ are constructed and proven to be $A$-graded isomorphisms of abelian groups in \autoref{E_homology_suspension_iso_t^a's_appendix}.
\end{lemma}
\begin{proof}
	Note that in \autoref{E_homology_suspension_iso_t^a's_appendix}, it is shown that $t_a^W:E_*(\Sigma^aW)\to E_{*-a}(W)$ is not just an $A$-graded isomorphism of abelian groups, but it is furthermore a left $\pi_*(E)$-module isomorphism. Thus, the top arrow in the above diagram is well-defined. Since all the arrows involved are $A$-graded homomorphisms, in order to show the diagram commutes it suffices to chase a pure homogeneous tensor around, as they generate the top left object. To that end, let $x:S^b\to Z\otimes E$ in $Z_*(E)$ and $y:S^c\to E\otimes S^a\otimes W$ in $E_*(\Sigma^aW)$, and consider the following diagram exhibiting the two ways to chase $x\otimes y$ around:
	% https://q.uiver.app/#q=WzAsOCxbMCwwLCJTXntiK2MtYX0iXSxbMCwyLCJTXmJcXG90aW1lcyBTXmNcXG90aW1lcyBTXnstYX0iXSxbMiwyLCJaXFxvdGltZXMgRVxcb3RpbWVzIEVcXG90aW1lcyBTXmFcXG90aW1lcyBXXFxvdGltZXMgU157LWF9Il0sWzIsMCwiWlxcb3RpbWVzIEVcXG90aW1lcyBFXFxvdGltZXMgV1xcb3RpbWVzIFNeYVxcb3RpbWVzIFNeey1hfSJdLFs0LDAsIlpcXG90aW1lcyBFXFxvdGltZXMgRVxcb3RpbWVzIFciXSxbNCwyLCJaXFxvdGltZXMgRVxcb3RpbWVzIFciXSxbMiw0LCJaXFxvdGltZXMgRVxcb3RpbWVzIFNeYVxcb3RpbWVzIFdcXG90aW1lcyBTXnstYX0iXSxbNCw0LCJaXFxvdGltZXMgRVxcb3RpbWVzIFdcXG90aW1lcyBTXmFcXG90aW1lcyBTXnstYX0iXSxbMCwxLCJcXHBoaSJdLFsxLDIsInhcXG90aW1lcyB5XFxvdGltZXMgU157LWF9Il0sWzIsMywiWlxcb3RpbWVzIEVcXG90aW1lcyBFXFxvdGltZXMgXFx0YXVcXG90aW1lcyBTXnstYX0iXSxbMyw0LCJaXFxvdGltZXMgRVxcb3RpbWVzIEVcXG90aW1lcyBXXFxvdGltZXMgXFxwaGlfe2EsLWF9XnstMX0iXSxbNCw1LCJaXFxvdGltZXMgXFxtdVxcb3RpbWVzIFciXSxbMiw2LCJaXFxvdGltZXMgXFxtdVxcb3RpbWVzIFNeYVxcb3RpbWVzIFdcXG90aW1lcyBTXnstYX0iLDJdLFs2LDcsIlpcXG90aW1lcyBFXFxvdGltZXMgXFx0YXVcXG90aW1lcyBTXnstYX0iLDJdLFs3LDUsIlpcXG90aW1lcyBFXFxvdGltZXMgV1xcb3RpbWVzIFxccGhpX3thLC1hfV57LTF9IiwyXSxbMyw3LCJaXFxvdGltZXMgXFxtdVxcb3RpbWVzIFdcXG90aW1lcyBTXmFcXG90aW1lcyBTXnstYX0iXV0=
	\[\begin{tikzcd}
		{S^{b+c-a}} && {Z\otimes E\otimes E\otimes W\otimes S^a\otimes S^{-a}} && {Z\otimes E\otimes E\otimes W} \\
		\\
		{S^b\otimes S^c\otimes S^{-a}} && {Z\otimes E\otimes E\otimes S^a\otimes W\otimes S^{-a}} && {Z\otimes E\otimes W} \\
		\\
		&& {Z\otimes E\otimes S^a\otimes W\otimes S^{-a}} && {Z\otimes E\otimes W\otimes S^a\otimes S^{-a}}
		\arrow["\phi", from=1-1, to=3-1]
		\arrow["{x\otimes y\otimes S^{-a}}", from=3-1, to=3-3]
		\arrow["{Z\otimes E\otimes E\otimes \tau\otimes S^{-a}}", from=3-3, to=1-3]
		\arrow["{Z\otimes E\otimes E\otimes W\otimes \phi_{a,-a}^{-1}}", from=1-3, to=1-5]
		\arrow["{Z\otimes \mu\otimes W}", from=1-5, to=3-5]
		\arrow["{Z\otimes \mu\otimes S^a\otimes W\otimes S^{-a}}"', from=3-3, to=5-3]
		\arrow["{Z\otimes E\otimes \tau\otimes S^{-a}}"', from=5-3, to=5-5]
		\arrow["{Z\otimes E\otimes W\otimes \phi_{a,-a}^{-1}}"', from=5-5, to=3-5]
		\arrow["{Z\otimes \mu\otimes W\otimes S^a\otimes S^{-a}}", from=1-3, to=5-5]
	\end{tikzcd}\]
	Each triangle commutes by functoriality of $-\otimes-$, so the diagram commutes as desired.
\end{proof}

\begin{lemma}\label{s^a_isos_are_right_pi_*(E)-module_homos}
    Given a monoid object $(E,\mu,e)$ and an object $X$ in $\cSH$, for all $a\in A$ the $A$-graded isomorphisms
    \[s^a_{X\otimes E}:\pi_*(\Sigma^aX\otimes E)\to\pi_{*-a}(X\otimes E)\]
    from \autoref{s^a_isos} are isomorphisms of right $\pi_*(E)$-modules, where here $\pi_*(\Sigma^aX\otimes E)$ and $\pi_*(X\otimes E)=X_*(E)$ are considered with their canonical right $\pi_*(E)$-module structure given in \autoref{module}.
\end{lemma}
\begin{proof}
    By additivity, in order to show $s^a_{X\otimes E}$ is a homomorphism of right $\pi_*(E)$-modules, it suffices to show that for all homogeneous  $x:S^b\to S^a\otimes X\otimes E$ in $\pi_*(\Sigma^a X\otimes E)$ and $r:S^b\to E$ in $\pi_*(E)$ that $s^a_{X\otimes E}(x\cdot r)=s^a_{X\otimes E}(x)\cdot r$. To that end, consider the following diagram:
    % https://q.uiver.app/#q=WzAsNixbMCwwLCJTXntiK2MtYX0iXSxbMSwwLCJTXnstYX1cXG90aW1lcyBTXmJcXG90aW1lcyBTXmMiXSxbMiwwLCJTXnstYX1cXG90aW1lcyBTXmFcXG90aW1lcyBYXFxvdGltZXMgRVxcb3RpbWVzIEUiXSxbMiwxLCJYXFxvdGltZXMgRVxcb3RpbWVzIEUiXSxbMywxLCJYXFxvdGltZXMgRSJdLFszLDAsIlNeey1hfVxcb3RpbWVzIFNeYVxcb3RpbWVzIFhcXG90aW1lcyBFIl0sWzAsMSwiXFxwaGkiXSxbMSwyLCJTXnstYX1cXG90aW1lcyB4XFxvdGltZXMgciJdLFsyLDMsIlxccGhpXnstMX1fey1hLGF9XFxvdGltZXMgWFxcb3RpbWVzIEVcXG90aW1lcyBFIiwyXSxbMyw0LCJYXFxvdGltZXMgXFxtdSJdLFsyLDUsIlNeey1hfVxcb3RpbWVzIFNeYVxcb3RpbWVzIFhcXG90aW1lcyBcXG11Il0sWzUsNCwiXFxwaGleey0xfV97LWEsYX1cXG90aW1lcyBYXFxvdGltZXMgRSJdXQ==
    \[\begin{tikzcd}
        {S^{b+c-a}} & {S^{-a}\otimes S^b\otimes S^c} & {S^{-a}\otimes S^a\otimes X\otimes E\otimes E} & {S^{-a}\otimes S^a\otimes X\otimes E} \\
        && {X\otimes E\otimes E} & {X\otimes E}
        \arrow["\phi", from=1-1, to=1-2]
        \arrow["{S^{-a}\otimes x\otimes r}", from=1-2, to=1-3]
        \arrow["{\phi^{-1}_{-a,a}\otimes X\otimes E\otimes E}"', from=1-3, to=2-3]
        \arrow["{X\otimes \mu}", from=2-3, to=2-4]
        \arrow["{S^{-a}\otimes S^a\otimes X\otimes \mu}", from=1-3, to=1-4]
        \arrow["{\phi^{-1}_{-a,a}\otimes X\otimes E}", from=1-4, to=2-4]
    \end{tikzcd}\]
    The top composition is $s^a_{X\otimes E}(x\cdot r)$, while the bottom composition is $s^a_{X\otimes E}(x)\cdot r$. The diagram commutes by functoriality of $-\otimes-$, so that $s^a_{X\otimes E}(x\cdot r)=s^a_{X\otimes E}(x)\cdot r$ as desired, meaning $s^a_{X\otimes E}$ is indeed a right $\pi_*(E)$-module homomorphism.
\end{proof}

\begin{lemma}\label{Phi_Z,W_iso_implies_Phi_S^aZ,W_and_Phi_Z,S^aW_are_isos}
	Let $(E,\mu,e)$ be a monoid object and $Z$ and $W$ objects in $\cSH$, and suppose the K\"unneth map $\Phi_{Z,W}$ is an isomorphism. Then $\Phi_{\Sigma^a Z,W}$ and $\Phi_{Z,\Sigma^aW}$ are isomorphisms for all $a\in A$, and so are $\Phi_{\Sigma Z,W}$ an $\Phi_{Z,\Sigma W}$.
\end{lemma}
\begin{proof}
	If $\Phi_{Z,W}$ is an isomorphism, it follows that $\Phi_{Z,\Sigma^aW}$ is an isomorphism by \autoref{t_a's_commute_with_Kunneth_map}. On the other hand, in order to see $\Phi_{\Sigma^aZ,W}$ is an isomorphism, consider the following diagram:
	% https://q.uiver.app/#q=WzAsNCxbMCwwLCJcXHBpXyooXFxTaWdtYV5hWlxcb3RpbWVzIEUpXFxvdGltZXNfe1xccGlfKihFKX1cXHBpXyooRVxcb3RpbWVzIFcpIl0sWzEsMCwiXFxwaV8qKFxcU2lnbWFeYSBaXFxvdGltZXMgRVxcb3RpbWVzIFcpIl0sWzAsMSwiXFxwaV97Ki1hfShaXFxvdGltZXMgRSlcXG90aW1lc197XFxwaV8qKEUpfVxccGlfKihFXFxvdGltZXMgVykiXSxbMSwxLCJcXHBpX3sqLWF9KFpcXG90aW1lcyBFXFxvdGltZXMgVykiXSxbMCwxLCJcXFBoaV97XFxTaWdtYV5hWixXfSJdLFswLDIsIlxcbWF0aHJte2Fkan1cXG90aW1lc197XFxwaV8qKEUpfVxccGlfKihFXFxvdGltZXMgVykiLDJdLFsxLDMsIlxcbWF0aHJte2Fkan0iXSxbMiwzLCJcXFBoaV97WixXfSJdXQ==
	\begin{equation}\label{Phi_Sigma^aZ,W_diagram}\begin{tikzcd}
        {\pi_*(\Sigma^aZ\otimes E)\otimes_{\pi_*(E)}\pi_*(E\otimes W)} & {\pi_*(\Sigma^a Z\otimes E\otimes W)} \\
        {\pi_{*-a}(Z\otimes E)\otimes_{\pi_*(E)}\pi_*(E\otimes W)} & {\pi_{*-a}(Z\otimes E\otimes W)}
        \arrow["{\Phi_{\Sigma^aZ,W}}", from=1-1, to=1-2]
        \arrow["{s^a_{Z\otimes E}\otimes_{\pi_*(E)}\pi_*(E\otimes W)}"', from=1-1, to=2-1]
        \arrow["{s^a_{Z\otimes E\otimes W}}", from=1-2, to=2-2]
        \arrow["{\Phi_{Z,W}}", from=2-1, to=2-2]
    \end{tikzcd}\end{equation}
	Here the vertical arrows are induced via the isomorphisms constructed in \autoref{s^a_isos}, and the left vertical arrow is well-defined since $s^a_{Z\otimes E}$ is a right $\pi_*(E)$-module homomorphism by \autoref{s^a_isos_are_right_pi_*(E)-module_homos}. Since every arrow in diagram (\ref{Phi_Sigma^aZ,W_diagram}) is an isomorphism of abelian groups except the top arrow, in order to show $\Phi_{\Sigma^aZ,W}$ is an isomorphism, it suffices to show the diagram commutes. To that end, since all the arrows are homomorphisms, it suffices to chase a pure homogeneous tensor around. So let $x:S^b\to\Sigma^aZ\otimes E$ and $y:S^c\to E\otimes W$, and consider the following diagram whose outside compositions exhibit the two ways to chase the pure tensor $x\otimes y$ around diagrama (\ref{Phi_Sigma^aZ,W_diagram}):
	% https://q.uiver.app/#q=WzAsNixbMCwwLCJTXntiK2MtYX0iXSxbMSwwLCJTXnstYX1cXG90aW1lcyBTXmJcXG90aW1lcyBTXmMiXSxbMiwwLCJTXnstYX1cXG90aW1lcyBTXmFcXG90aW1lcyBaXFxvdGltZXMgRVxcb3RpbWVzIEVcXG90aW1lcyBXIl0sWzMsMCwiU157LWF9XFxvdGltZXMgU15hXFxvdGltZXMgWlxcb3RpbWVzIEVcXG90aW1lcyBXIl0sWzMsMSwiWlxcb3RpbWVzIEVcXG90aW1lcyBXIl0sWzIsMSwiWlxcb3RpbWVzIEVcXG90aW1lcyBFXFxvdGltZXMgVyJdLFswLDEsIlxccGhpIl0sWzEsMiwiU157LWF9XFxvdGltZXMgeFxcb3RpbWVzIHkiXSxbMiwzLCJTXnstYX1cXG90aW1lcyBTXmFcXG90aW1lcyBaXFxvdGltZXMgXFxtdVxcb3RpbWVzIFciXSxbMyw0LCJcXHBoaV97LWEsYX1eey0xfVxcb3RpbWVzIFpcXG90aW1lcyBFXFxvdGltZXMgVyJdLFsyLDUsIlxccGhpX3stYSxhfV57LTF9XFxvdGltZXMgWlxcb3RpbWVzIEVcXG90aW1lcyBFXFxvdGltZXMgVyIsMl0sWzUsNCwiWlxcb3RpbWVzIFxcbXVcXG90aW1lcyBXIl1d
	\[\begin{tikzcd}
		{S^{b+c-a}} & {S^{-a}\otimes S^b\otimes S^c} & {S^{-a}\otimes S^a\otimes Z\otimes E\otimes E\otimes W} & {S^{-a}\otimes S^a\otimes Z\otimes E\otimes W} \\
		&& {Z\otimes E\otimes E\otimes W} & {Z\otimes E\otimes W}
		\arrow["\phi", from=1-1, to=1-2]
		\arrow["{S^{-a}\otimes x\otimes y}", from=1-2, to=1-3]
		\arrow["{S^{-a}\otimes S^a\otimes Z\otimes \mu\otimes W}", from=1-3, to=1-4]
		\arrow["{\phi_{-a,a}^{-1}\otimes Z\otimes E\otimes W}", from=1-4, to=2-4]
		\arrow["{\phi_{-a,a}^{-1}\otimes Z\otimes E\otimes E\otimes W}"', from=1-3, to=2-3]
		\arrow["{Z\otimes \mu\otimes W}", from=2-3, to=2-4]
	\end{tikzcd}\]
	The diagram clearly commutes by functoriality of $-\otimes-$, so that indeed diagram (\ref{Phi_Sigma^aZ,W_diagram}) commutes, so that $\Phi_{\Sigma^aZ,W}$ is indeed an isomorphism as desired. 

	Now, it remains to show that $\Phi_{Z,\Sigma W}$ and $\Phi_{\Sigma Z,W}$ are isomorphisms. To that end, consider the following diagram:
	% https://q.uiver.app/#q=WzAsNCxbMCwwLCJcXHBpXyooWlxcb3RpbWVzIEUpXFxvdGltZXNfe1xccGlfKihFKX1cXHBpXyooRVxcb3RpbWVzIFxcU2lnbWEgVykiXSxbMCwyLCJcXHBpXyooWlxcb3RpbWVzIEUpXFxvdGltZXNfe1xccGlfKihFKX1cXHBpXyooRVxcb3RpbWVzXFxTaWdtYV5cXDFXKSJdLFsxLDIsIlxccGlfKihaXFxvdGltZXMgRVxcb3RpbWVzXFxTaWdtYV5cXDFXKSJdLFsxLDAsIlxccGlfKihaXFxvdGltZXMgRVxcb3RpbWVzXFxTaWdtYSBXKSJdLFswLDEsIlxccGlfKihaXFxvdGltZXMgRSlcXG90aW1lc197XFxwaV8qKEUpfVxccGlfKihFXFxvdGltZXNcXG51X1cpIiwyXSxbMSwyLCJcXFBoaV97WixcXFNpZ21hXlxcMVd9IiwyXSxbMCwzLCJcXFBoaV97WixcXFNpZ21hIFd9Il0sWzMsMiwiXFxwaV8qKFpcXG90aW1lcyBFXFxvdGltZXMgXFxudV9XKSJdXQ==
	\[\begin{tikzcd}
		{\pi_*(Z\otimes E)\otimes_{\pi_*(E)}\pi_*(E\otimes \Sigma W)} & {\pi_*(Z\otimes E\otimes\Sigma W)} \\
		\\
		{\pi_*(Z\otimes E)\otimes_{\pi_*(E)}\pi_*(E\otimes\Sigma^\1W)} & {\pi_*(Z\otimes E\otimes\Sigma^\1W)}
		\arrow["{\pi_*(Z\otimes E)\otimes_{\pi_*(E)}\pi_*(E\otimes\nu_W)}"', from=1-1, to=3-1]
		\arrow["{\Phi_{Z,\Sigma^\1W}}"', from=3-1, to=3-2]
		\arrow["{\Phi_{Z,\Sigma W}}", from=1-1, to=1-2]
		\arrow["{\pi_*(Z\otimes E\otimes \nu_W)}", from=1-2, to=3-2]
	\end{tikzcd}\]
	It commutes by naturality of $\Phi$. Furthermore, assuming $\Phi_{Z,W}$ is an isomorphism, by what we have shown above we know that $\Phi_{Z,\Sigma^\1W}$ is an isomorphism, and since $\nu_W$ is an isomorphism, it follows that the above diagram commutes and all arrows except $\Phi_{Z,\Sigma W}$ are isomorphisms, so that $\Phi_{Z,\Sigma W}$ must be an isomorphism itself. Finally, an entirely analagous argument using naturality of $\Phi$ with respect to $\nu_Z$ yields that $\Phi_{\Sigma Z,W}$ is an isomorphism as well.
\end{proof}

Now, we can finally prove the desired theorem:

\begin{proposition}\label{Kunneth_map_iso}
	Let $(E,\mu,e)$ be a monoid object and $Z$ and $W$ objects in $\cSH$. Then if either:\begin{enumerate}
		\item $Z_*(E)$ is a flat right $\pi_*(E)$-module (via \autoref{module}) and $W$ is cellular (\autoref{cellular}), or
		\item $E_*(W)$ is a flat left $\pi_*(E)$-module (via \autoref{module}) and $Z$ is cellular,
	\end{enumerate} 
	then the natural homomorphism
	\[\Phi_{Z,W}:Z_*(E)\otimes_{\pi_*(E)}E_*(W)\to \pi_*(Z\otimes E\otimes W)\]
	given in \autoref{Kunneth_map} is an isomorphism of abelian groups.
\end{proposition}
\begin{proof}
	In this proof, we will freely employ the coherence theorem for symmetric monoidal categories, and we will assume that associativity and unitality of $-\otimes-$ holds up to strict equality. First we will consider the case that $\pi_*(Z\otimes E)=Z_*(E)$ is a flat right $\pi_*(E)$-module and $W$ is cellular. To start, let $\cE$ be the collection of objects $W$ in $\cSH$ for which $\Phi_{Z,W}$ is an isomorphism. Then in order to show $\cE$ contains every cellular object, it suffices to show that $\cE$ satisfies the three conditions given for the class of cellular objects in \autoref{cellular}. First, we need to show that $\Phi_{Z,W}$ is an isomorphism when $W=S^a$ for some $a\in A$.
%	Note that
%	\[E_*(S^a)=[S^*,E\otimes S^a]\cong[S^{-a}\otimes S^*,E]\cong[S^{*-a},E]=\pi_{*-a}(E),\]
%	where the first isomorphism follows by the adunction between $S^{-a}\otimes-$ and $-\otimes S^a\cong S^a\otimes-$ (\autoref{Sigma^a,Sigma^-a_adjoint_equiv}). Similarly, we have
%	\[E_*(E\otimes S^a)=[S^*,E\otimes E\otimes S^a]\cong[S^{*-a},E\otimes E]=E_{*-a}(E).\]
%	Hence by \autoref{tensor_shift_A_graded} we have isomorphisms
%	\[E_*(E)\otimes_{\pi_*(E)}E_*(S^a)\cong E_*(E)\otimes_{\pi_*(E)}\pi_{*-a}(E)\cong E_{*-a}(E)\cong E_*(E\otimes S^a).\]
	Indeed, consider the $A$-graded homomorphism
	\begin{align*}
		\Psi:\pi_*(Z\otimes E\otimes S^a)&\to \pi_*(Z\otimes E)\otimes_{\pi_*(E)}\pi_*(E\otimes S^a)
	\end{align*}
	which sends a class $x:S^b\to Z\otimes E\otimes S^a$ in $\pi_b(Z\otimes E\otimes S^a)$ to the pure tensor $\wt x\otimes\wt e$, where $\wt x\in \pi_{b-a}(Z\otimes E)$ is the composition
	\[S^{b-a}\xr{\phi_{b,-a}}S^b\otimes S^{-a}\xr{x\otimes S^{-a}}Z\otimes E\otimes S^a\otimes S^{-a}\xr{Z\otimes E\otimes\phi_{a,-a}^{-1}}Z\otimes E\]
	and $\wt e\in \pi_a(E\otimes S^a)$ is the composition
	\[S^a\xr{e\otimes S^a}E\otimes S^a.\]
	In order to see $\Psi$ is an ($A$-graded) homomorphism of abelian groups: Given $x,x'\in \pi_b(Z\otimes E\otimes S^a)$, we would like to show that $\wt x\otimes\wt e+\wt x'\otimes\wt e=\wt{x+x'}\otimes\wt e$. It suffices to show that $\wt x+\wt x'=\wt{x+x'}$. To see this, consider the following diagram (again, we are passing to a symmetric strict monoidal category):
	% https://q.uiver.app/#q=WzAsMTAsWzAsMCwiU157Yi1hfSJdLFsxLDAsIlNee2ItYX1cXG9wbHVzIFNee2ItYX0iXSxbMSwxLCIoU15iXFxvdGltZXMgU157LWF9KVxcb3BsdXMoU15iXFxvdGltZXMgU157LWF9KSJdLFsxLDIsIihaXFxvdGltZXMgRVxcb3RpbWVzIFNeYVxcb3RpbWVzIFNeey1hfSlcXG9wbHVzKFpcXG90aW1lcyBFXFxvdGltZXMgU15hXFxvdGltZXMgU157LWF9KSJdLFsxLDMsIihaXFxvdGltZXMgRSlcXG9wbHVzKFpcXG90aW1lcyBFKSJdLFsxLDQsIlpcXG90aW1lcyBFIl0sWzAsMSwiU15iXFxvdGltZXMgU157LWF9Il0sWzAsMiwiKFNeYlxcb3BsdXMgU15iKVxcb3RpbWVzIFNeey1hfSJdLFswLDMsIigoWlxcb3RpbWVzIEVcXG90aW1lcyBTXmEpXFxvcGx1cyAoWlxcb3RpbWVzIEVcXG90aW1lcyBTXmEpKVxcb3RpbWVzIFNeey1hfSJdLFswLDQsIlpcXG90aW1lcyBFXFxvdGltZXMgU15hXFxvdGltZXMgU157LWF9Il0sWzAsMSwiXFxEZWx0YSJdLFsxLDIsIlxccGhpX3tiLC1hfVxcb3BsdXNcXHBoaV97YiwtYX0iXSxbMiwzLCIoeFxcb3RpbWVzIFNeey1hfSlcXG9wbHVzKHgnXFxvdGltZXMgU157LWF9KSJdLFszLDQsIihaXFxvdGltZXMgRVxcb3RpbWVzXFxwaGlfe2EsLWF9XnstMX0pXFxvcGx1cyhaXFxvdGltZXMgRVxcb3RpbWVzXFxwaGlfe2EsLWF9XnstMX0pIl0sWzQsNSwiXFxuYWJsYSJdLFswLDYsIlxccGhpX3tiLWF9IiwyXSxbNiw3LCJcXERlbHRhXFxvdGltZXMgU157LWF9IiwyXSxbNyw4LCIoeFxcb3BsdXMgeCcpXFxvdGltZXMgU157LWF9IiwyXSxbOCw5LCJcXG5hYmxhXFxvdGltZXMgU157LWF9IiwyXSxbOSw1LCJaXFxvdGltZXMgRVxcb3RpbWVzXFxwaGlfe2EsLWF9XnstMX0iLDJdLFs3LDIsIlxcY29uZyJdLFs4LDMsIlxcY29uZyJdLFszLDksIlxcbmFibGEiXSxbNiwyLCJcXERlbHRhIl1d
	\[\begin{tikzcd}
		{S^{b-a}} & {S^{b-a}\oplus S^{b-a}} \\
		{S^b\otimes S^{-a}} & {(S^b\otimes S^{-a})\oplus(S^b\otimes S^{-a})} \\
		{(S^b\oplus S^b)\otimes S^{-a}} & {(Z\otimes E\otimes S^a\otimes S^{-a})\oplus(Z\otimes E\otimes S^a\otimes S^{-a})} \\
		{((Z\otimes E\otimes S^a)\oplus (Z\otimes E\otimes S^a))\otimes S^{-a}} & {(Z\otimes E)\oplus(Z\otimes E)} \\
		{Z\otimes E\otimes S^a\otimes S^{-a}} & {Z\otimes E}
		\arrow["\Delta", from=1-1, to=1-2]
		\arrow["{\phi_{b,-a}\oplus\phi_{b,-a}}", from=1-2, to=2-2]
		\arrow["{(x\otimes S^{-a})\oplus(x'\otimes S^{-a})}", from=2-2, to=3-2]
		\arrow["{(Z\otimes E\otimes\phi_{a,-a}^{-1})\oplus(Z\otimes E\otimes\phi_{a,-a}^{-1})}", from=3-2, to=4-2]
		\arrow["\nabla", from=4-2, to=5-2]
		\arrow["{\phi_{b-a}}"', from=1-1, to=2-1]
		\arrow["{\Delta\otimes S^{-a}}"', from=2-1, to=3-1]
		\arrow["{(x\oplus x')\otimes S^{-a}}"', from=3-1, to=4-1]
		\arrow["{\nabla\otimes S^{-a}}"', from=4-1, to=5-1]
		\arrow["{Z\otimes E\otimes\phi_{a,-a}^{-1}}"', from=5-1, to=5-2]
		\arrow["\cong", from=3-1, to=2-2]
		\arrow["\cong", from=4-1, to=3-2]
		\arrow["\nabla", from=3-2, to=5-1]
		\arrow["\Delta", from=2-1, to=2-2]
	\end{tikzcd}\]
	The top rectangle commutes by naturality of $\Delta$ in an additive category. The bottom triangle commutes by naturality of $\nabla$ in an additive category. Finally, the remaining regions of the diagram commute by additivity of $-\otimes-$. By functoriality of $-\otimes-$, it follows that the left composition is $\wt{x+x'}$ and the right composition is $\wt x+\wt x'$, so they are equal as desired. Thus $\Psi$ is a homomorphism of abelian groups, as desired.

	Now, we claim that $\Psi$ is an inverse to $\Phi_{Z,S^a}$. Since $\Phi_{Z,S^a}$ and $\Psi$ are homomorphisms it suffices to check that they are inverses on generators. First, let $x:S^b\to Z\otimes E\otimes S^a$ in $\pi_b(Z\otimes E\otimes S^a)$. We would like to show that $\Phi_{Z,S^a}(\Psi(x))=x$. Consider the following diagram, where here we are passing to a symmetric strict monoidal category:
	% https://q.uiver.app/#q=WzAsNyxbMCwwLCJTXmIiXSxbMiwwLCJTXmJcXG90aW1lcyBTXnstYX1cXG90aW1lcyBTXmEiXSxbNCwxLCJaXFxvdGltZXMgRVxcb3RpbWVzIFNeYVxcb3RpbWVzIFNeey1hfVxcb3RpbWVzIEVcXG90aW1lcyBTXmEiXSxbMCwyLCJaXFxvdGltZXMgRVxcb3RpbWVzIFNeYSJdLFsyLDMsIlpcXG90aW1lcyBFXFxvdGltZXMgRVxcb3RpbWVzIFNeYSJdLFsyLDEsIlpcXG90aW1lcyBFXFxvdGltZXMgU15hIFxcb3RpbWVzIFNeey1hfVxcb3RpbWVzIFNeYSJdLFsyLDIsIlpcXG90aW1lcyBFXFxvdGltZXMgU15hIl0sWzAsMSwiXFxjb25nIl0sWzEsMiwieFxcb3RpbWVzIFNeey1hfVxcb3RpbWVzIGVcXG90aW1lcyBTXmEiXSxbMCwzLCJ4IiwyXSxbMSw1LCJ4XFxvdGltZXMgU157LWF9XFxvdGltZXMgU15hIiwxXSxbMyw1LCJaXFxvdGltZXMgRVxcb3RpbWVzIFNeYVxcb3RpbWVzIFxccGhpX3stYSxhfSJdLFs1LDIsIlpcXG90aW1lcyBFXFxvdGltZXMgU15hXFxvdGltZXMgU157LWF9XFxvdGltZXMgZVxcb3RpbWVzIFNeYSIsMl0sWzYsNSwiWlxcb3RpbWVzIEVcXG90aW1lcyBcXHBoaV97YSwtYX1cXG90aW1lcyBTXmEiLDFdLFs2LDQsIlpcXG90aW1lcyBFXFxvdGltZXMgZVxcb3RpbWVzIFNeYSIsMV0sWzYsMywiIiwyLHsibGV2ZWwiOjIsInN0eWxlIjp7ImhlYWQiOnsibmFtZSI6Im5vbmUifX19XSxbMiw0LCJaXFxvdGltZXMgRVxcb3RpbWVzIFxccGhpX3thLC1hfV57LTF9XFxvdGltZXMgRVxcb3RpbWVzIFNeYSJdLFs0LDMsIlpcXG90aW1lcyBcXG11XFxvdGltZXMgU15hIl1d
	\[\begin{tikzcd}
		{S^b} && {S^b\otimes S^{-a}\otimes S^a} \\
		&& {Z\otimes E\otimes S^a \otimes S^{-a}\otimes S^a} && {Z\otimes E\otimes S^a\otimes S^{-a}\otimes E\otimes S^a} \\
		{Z\otimes E\otimes S^a} && {Z\otimes E\otimes S^a} \\
		&& {Z\otimes E\otimes E\otimes S^a}
		\arrow["\cong", from=1-1, to=1-3]
		\arrow["{x\otimes S^{-a}\otimes e\otimes S^a}", from=1-3, to=2-5]
		\arrow["x"', from=1-1, to=3-1]
		\arrow["{x\otimes S^{-a}\otimes S^a}"{description}, from=1-3, to=2-3]
		\arrow["{Z\otimes E\otimes S^a\otimes \phi_{-a,a}}", from=3-1, to=2-3]
		\arrow["{Z\otimes E\otimes S^a\otimes S^{-a}\otimes e\otimes S^a}"', from=2-3, to=2-5]
		\arrow["{Z\otimes E\otimes \phi_{a,-a}\otimes S^a}"{description}, from=3-3, to=2-3]
		\arrow["{Z\otimes E\otimes e\otimes S^a}"{description}, from=3-3, to=4-3]
		\arrow[Rightarrow, no head, from=3-3, to=3-1]
		\arrow["{Z\otimes E\otimes \phi_{a,-a}^{-1}\otimes E\otimes S^a}", from=2-5, to=4-3]
		\arrow["{Z\otimes \mu\otimes S^a}", from=4-3, to=3-1]
	\end{tikzcd}\]
	The top left trapezoid commutes since the isomorphism $S^b\xr\cong S^b\otimes S^{-a}\otimes S^a$ may be given as $S^b\otimes\phi_{-a,a}$ (see \autoref{unique_comp_Sas}), in which case the trapezoid commmutes by functoriality of $-\otimes-$. The triangle below that commutes by coherence for the $\phi_{a,b}$'s. The bottom left triangle commutes by unitality for $\mu$. The top right triangle commutes by functoriality of $-\otimes-$. Finally, the bottom right triangle commutes by functoriality of $-\otimes-$. It follows by unravelling definitions that the two outside compositions are $x$ and $\Phi_{Z,S^a}(\Psi(x))$, so indeed we have $\Phi_{Z,S^a}(\Psi(x))=x$ since the diagram commutes.

	On the other hand, suppose we are given a homogeneous pure tensor $x\otimes y$ in $\pi_*(Z\otimes E)\otimes_{\pi_*(E)}\pi_*(E\otimes S^a)$, so $x:S^b\to Z\otimes E$ and $y:S^c\to E\otimes S^a$ for some $b,c\in A$. Then we would like to show that $\Psi(\Phi_{Z,S^a}(x\otimes y))=x\otimes y$. Unravelling definitions, $\Psi(\Phi_{Z,S^a}(x\otimes y))$ is the homogeneous pure tensor $\wt{x y}\otimes\wt e$, where $\wt e$ is the map $e\otimes S^a:S^{a}\to E\otimes S^a$ is defined above, and by functoriality of $-\otimes-$, $\wt{xy}:S^{b+c-a}\to Z\otimes E$ is the composition
	% https://q.uiver.app/#q=WzAsNSxbMCwwLCJTXntiK2MtYX0iXSxbMCwxLCJTXmJcXG90aW1lcyBTXmNcXG90aW1lcyBTXnstYX0iXSxbMCwyLCJaXFxvdGltZXMgRVxcb3RpbWVzIEVcXG90aW1lcyBTXmFcXG90aW1lcyBTXnstYX0iXSxbMCwzLCJaXFxvdGltZXMgRVxcb3RpbWVzIFNeYVxcb3RpbWVzIFNeey1hfSJdLFswLDQsIlpcXG90aW1lcyBFIl0sWzAsMSwiXFxjb25nIl0sWzEsMiwieFxcb3RpbWVzIHlcXG90aW1lcyBTXnstYX0iXSxbMiwzLCJaXFxvdGltZXNcXG11XFxvdGltZXMgU15hXFxvdGltZXMgU157LWF9Il0sWzMsNCwiWlxcb3RpbWVzIEVcXG90aW1lc1xccGhpX3thLC1hfV57LTF9Il1d
	\[\begin{tikzcd}
		{S^{b+c-a}} \\
		{S^b\otimes S^c\otimes S^{-a}} \\
		{Z\otimes E\otimes E\otimes S^a\otimes S^{-a}} \\
		{Z\otimes E\otimes S^a\otimes S^{-a}} \\
		{Z\otimes E}
		\arrow["\cong", from=1-1, to=2-1]
		\arrow["{x\otimes y\otimes S^{-a}}", from=2-1, to=3-1]
		\arrow["{Z\otimes\mu\otimes S^a\otimes S^{-a}}", from=3-1, to=4-1]
		\arrow["{Z\otimes E\otimes\phi_{a,-a}^{-1}}", from=4-1, to=5-1]
	\end{tikzcd}\]
	Now, define $r\in\pi_{c-a}(E)$ to be the composition
	\[S^{c-a}\cong S^c\otimes S^{-a}\xr{y\otimes S^{-a}}E\otimes S^a\otimes S^{-a}\xr{E\otimes\phi_{a,-a}^{-1}}E.\]
	First, we claim that $x\cdot r=\wt{xy}$. To that end, consider the following diagram, where here we are again passing to a symmetric strict monoidal category:
	% https://q.uiver.app/#q=WzAsNixbMCwwLCJTXntiK2MtYX0iXSxbMSwwLCJTXntifVxcb3RpbWVzIFNeY1xcb3RpbWVzIFNeey1hfSJdLFsyLDAsIlpcXG90aW1lcyBFXFxvdGltZXMgRVxcb3RpbWVzIFNeYVxcb3RpbWVzIFNeey1hfSJdLFszLDAsIlpcXG90aW1lcyBFXFxvdGltZXMgU15hXFxvdGltZXMgU157LWF9Il0sWzMsMSwiWlxcb3RpbWVzIEUiXSxbMiwxLCJaXFxvdGltZXMgRVxcb3RpbWVzIEUiXSxbMCwxLCJcXGNvbmciXSxbMSwyLCJ4XFxvdGltZXMgeVxcb3RpbWVzIFNeey1hfSJdLFsyLDMsIlpcXG90aW1lcyBcXG11XFxvdGltZXMgU15hXFxvdGltZXMgU157LWF9Il0sWzIsNSwiWlxcb3RpbWVzIEVcXG90aW1lcyBFXFxvdGltZXMgXFxwaGlfe2EsLWF9XnstMX0iLDJdLFs1LDQsIlpcXG90aW1lcyBcXG11IiwyXSxbMyw0LCJaXFxvdGltZXMgRVxcb3RpbWVzIFxccGhpX3thLC1hfV57LTF9Il1d
	\[\begin{tikzcd}
		{S^{b+c-a}} & {S^{b}\otimes S^c\otimes S^{-a}} & {Z\otimes E\otimes E\otimes S^a\otimes S^{-a}} & {Z\otimes E\otimes S^a\otimes S^{-a}} \\
		&& {Z\otimes E\otimes E} & {Z\otimes E}
		\arrow["\cong", from=1-1, to=1-2]
		\arrow["{x\otimes y\otimes S^{-a}}", from=1-2, to=1-3]
		\arrow["{Z\otimes \mu\otimes S^a\otimes S^{-a}}", from=1-3, to=1-4]
		\arrow["{Z\otimes E\otimes E\otimes \phi_{a,-a}^{-1}}"', from=1-3, to=2-3]
		\arrow["{Z\otimes \mu}"', from=2-3, to=2-4]
		\arrow["{Z\otimes E\otimes \phi_{a,-a}^{-1}}", from=1-4, to=2-4]
	\end{tikzcd}\]
	Commutativity is functoriality of $-\otimes-$, which also tells us that the two outside compositions are $\wt{xy}$ (on top) and $x\cdot r$ (on the bottom), so they are equal as desired. On the other hand, we claim that $r\cdot\wt e=y$. To see this, consider the following diagram:
	% https://q.uiver.app/#q=WzAsOCxbMCwwLCJTXmMiXSxbMywwLCJTXmNcXG90aW1lcyBTXnstYX1cXG90aW1lcyBTXmEiXSxbMywxLCJFXFxvdGltZXMgU15hXFxvdGltZXMgU157LWF9XFxvdGltZXMgRVxcb3RpbWVzIFNeYSJdLFswLDMsIkVcXG90aW1lcyBFXFxvdGltZXMgU15hIl0sWzAsMiwiRVxcb3RpbWVzIFNeYSJdLFsxLDEsIkVcXG90aW1lcyBTXmFcXG90aW1lcyBTXnstYX1cXG90aW1lcyBTXmEiXSxbMywzLCJFXFxvdGltZXMgRVxcb3RpbWVzIFNeYSJdLFsxLDIsIkVcXG90aW1lcyBTXmEiXSxbMCwxLCJcXGNvbmciXSxbMSwyLCJ5XFxvdGltZXMgU157LWF9XFxvdGltZXMgZVxcb3RpbWVzIFNeYSJdLFszLDQsIlxcbXVcXG90aW1lcyBTXmEiXSxbMCw0LCJ5IiwyXSxbNSwyLCJFXFxvdGltZXMgU157YX1cXG90aW1lcyBTXnstYX1cXG90aW1lcyBlXFxvdGltZXMgU15hIiwyXSxbNSw0LCJFXFxvdGltZXMgU15hXFxvdGltZXMgXFxwaGlfey1hLGF9XnstMX0iLDFdLFs2LDMsIiIsMCx7ImxldmVsIjoyLCJzdHlsZSI6eyJoZWFkIjp7Im5hbWUiOiJub25lIn19fV0sWzIsNiwiRVxcb3RpbWVzIFxccGhpX3thLC1hfV57LTF9XFxvdGltZXMgRVxcb3RpbWVzIFNee2F9Il0sWzEsNSwieVxcb3RpbWVzIFNeey1hfVxcb3RpbWVzIFNeYSIsMV0sWzUsNywiRVxcb3RpbWVzIFxccGhpX3thLC1hfV57LTF9XFxvdGltZXMgU15hIl0sWzQsNywiIiwxLHsibGV2ZWwiOjIsInN0eWxlIjp7ImhlYWQiOnsibmFtZSI6Im5vbmUifX19XSxbNyw2LCJFXFxvdGltZXMgZVxcb3RpbWVzIFNeYSIsMV1d
	\[\begin{tikzcd}
		{S^c} &&& {S^c\otimes S^{-a}\otimes S^a} \\
		& {E\otimes S^a\otimes S^{-a}\otimes S^a} && {E\otimes S^a\otimes S^{-a}\otimes E\otimes S^a} \\
		{E\otimes S^a} & {E\otimes S^a} \\
		{E\otimes E\otimes S^a} &&& {E\otimes E\otimes S^a}
		\arrow["\cong", from=1-1, to=1-4]
		\arrow["{y\otimes S^{-a}\otimes e\otimes S^a}", from=1-4, to=2-4]
		\arrow["{\mu\otimes S^a}", from=4-1, to=3-1]
		\arrow["y"', from=1-1, to=3-1]
		\arrow["{E\otimes S^{a}\otimes S^{-a}\otimes e\otimes S^a}"', from=2-2, to=2-4]
		\arrow["{E\otimes S^a\otimes \phi_{-a,a}^{-1}}"{description}, from=2-2, to=3-1]
		\arrow[Rightarrow, no head, from=4-4, to=4-1]
		\arrow["{E\otimes \phi_{a,-a}^{-1}\otimes E\otimes S^{a}}", from=2-4, to=4-4]
		\arrow["{y\otimes S^{-a}\otimes S^a}"{description}, from=1-4, to=2-2]
		\arrow["{E\otimes \phi_{a,-a}^{-1}\otimes S^a}", from=2-2, to=3-2]
		\arrow[Rightarrow, no head, from=3-1, to=3-2]
		\arrow["{E\otimes e\otimes S^a}"{description}, from=3-2, to=4-4]
	\end{tikzcd}\]
	By \autoref{unique_comp_Sas}, we may take the top arrow to be $S^c\otimes \phi_{-a,a}$, in which case the top left triangle commutes by functoriality of $-\otimes-$. The bottom trapezoid commutes by unitality of $\mu$. Every other region commutes either by definition or by functoriality of $-\otimes-$. The top composition is $r\cdot\wt e$, so we have shown $r\cdot\wt e=y$ as desired. Thus, we have that
	\[\Psi(\Phi_{Z,S^a}(x\otimes y))=\wt{xy}\otimes\wt e=x\cdot r\otimes\wt e=x\otimes r\cdot\wt e=x\otimes y,\]
	as desired. Hence we have shown $\Psi$ is both a left and right inverse for $\Phi_{Z,S^a}$, so that indeed $S^a$ belongs to $\cE$ as desired.

	Now, we would like to show that given a distinguished triangle in $\cSH$
	\[X\xr fY\xr gW\xr h\Sigma X,\]
	if two of three of the objects $X$, $Y$, and $W$ belong to $\cE$, then so does the third. From now on, write $L^E_*$ to denote the functor from $\cSH$ to $A$-graded abelian groups sending $X\mapsto \pi_*(Z\otimes E)\otimes_{\pi_*(E)}\pi_*(E\otimes X)$. Then $\Phi_{Z,-}$ is a natural transformation $L_*^E\Rightarrow \pi_*(Z\otimes E\otimes-)=Z_*(E\otimes-)$. First, recall that it follows generally that in an adjointly triangulated category (\autoref{adjointly_triangulated_defn}), which $\cSH$ is by \autoref{Sigma^a,Sigma^-a_adjoint_equiv}, given a distinguished triangle $(f,g,h)$ we have a long exact sequence (see \autoref{defn_exact} for the definition of an exact sequence in an additive category, and see \autoref{dist_tri_LES} for the explicit contruction of the LES associated to a distinguished triangle in an adjointly triangulated category):
	\[\Omega Y\xr{\Omega g}\Omega W\xr{\wt h}X\xr fY\xr gW\xr h\Sigma X\xr{\Sigma f}\Sigma Y,\]
	where $\wt h:\Omega W\to X$ is the adjoint of $h:W\to\Sigma X$. Then since $\cSH$ is further a tensor triangulated category (\autoref{tentri}), we have that the above sequence remains exact even after tensoring by $E$ on the left (see \autoref{LES_remains_exact_after_tensor} for details), so we have the following exact sequence in $\cSH$:
	\[E\otimes \Omega Y\xr{E\otimes \Omega g}E\otimes \Omega W\xr{E\otimes \wt h}E\otimes X\xr{E\otimes f}E\otimes Y\xr{E\otimes g}E\otimes W\xr{E\otimes h}E\otimes \Sigma X\xr{E\otimes \Sigma f}E\otimes \Sigma Y.\]
	We can then apply $[S^*,-]=\pi_*(-)$ to it, which yields the following exact sequence of $A$-graded abelian groups:
	\[E_*(\Omega Y)\xr{E_*(\Omega g)}E_*(\Omega W)\xr{E_*(\wt h)}E_*(X)\xr{E_*(f)}E_*(Y)\xr{E_*(g)}E_*(W)\xr{E_*(h)}E_*(\Sigma X)\xr{E_*(f)}E_*(\Sigma Y).\]
	Now, we can tensor this sequence with $\pi_*(Z\otimes E)$ on the left over $\pi_*(E)$, and since $\pi_*(Z\otimes E)$ is a flat right $\pi_*(E)$ module, we get that the top row in the following diagram is exact:
	% https://q.uiver.app/#q=WzAsMTQsWzAsMCwiTF8qXkUoXFxPbWVnYSBZKSJdLFsxLDAsIkxfKl5FKFxcT21lZ2EgVykiXSxbMiwwLCJMXypeRShYKSJdLFszLDAsIkxfKl5FKFkpIl0sWzQsMCwiTF8qXkUoVykiXSxbNSwwLCJMXypeRShcXFNpZ21hIFgpIl0sWzYsMCwiTF8qXkUoXFxTaWdtYSBZKSJdLFswLDEsIlpfKihFXFxvdGltZXNcXE9tZWdhIFkpIl0sWzEsMSwiWl8qKEVcXG90aW1lc1xcT21lZ2EgVykiXSxbMiwxLCJaXyooRVxcb3RpbWVzIFgpIl0sWzMsMSwiWl8qKEVcXG90aW1lcyBZKSJdLFs0LDEsIlpfKihFXFxvdGltZXMgVykiXSxbNSwxLCJaXyooRVxcb3RpbWVzIFxcU2lnbWEgWCkiXSxbNiwxLCJaXyooRVxcb3RpbWVzIFxcU2lnbWEgWSkiXSxbMCwxLCJMXypeRShcXE9tZWdhIGcpIl0sWzEsMiwiTF8qXkUoXFx3dCBoKSJdLFsyLDMsIkxfKl5FKGYpIl0sWzMsNCwiTF8qXkUoZykiXSxbNCw1LCJMXypeRShoKSJdLFs1LDYsIkxfKl5FKFxcU2lnbWEgZikiXSxbMCw3LCJcXFBoaV97WixcXE9tZWdhIFl9IiwyXSxbNyw4LCJaXyooRVxcb3RpbWVzIFxcT21lZ2EgZykiLDJdLFsxMSwxMiwiWl8qKEVcXG90aW1lcyBoKSIsMl0sWzEyLDEzLCJaXyooRVxcb3RpbWVzXFxTaWdtYSBmKSIsMl0sWzEsOCwiXFxQaGlfe1osXFxPbWVnYSBXfSIsMl0sWzksMTAsIlpfKihFXFxvdGltZXMgZikiLDJdLFsyLDksIlxcUGhpX3taLFh9IiwyXSxbMywxMCwiXFxQaGlfe1osWX0iLDJdLFsxMCwxMSwiWl8qKEVcXG90aW1lcyBnKSIsMl0sWzQsMTEsIlxcUGhpX3taLFd9IiwyXSxbNSwxMiwiXFxQaGlfe1osXFxTaWdtYSBYfSIsMl0sWzYsMTMsIlxcUGhpX3taLFxcU2lnbWEgWX0iLDJdLFs4LDksIlpfKihFXFxvdGltZXNcXHd0IGgpIiwyXV0=
	\[\begin{tikzcd}[column sep=tiny]
		{L_*^E(\Omega Y)} & {L_*^E(\Omega W)} & {L_*^E(X)} & {L_*^E(Y)} & {L_*^E(W)} & {L_*^E(\Sigma X)} & {L_*^E(\Sigma Y)} \\
		{Z_*(E\otimes\Omega Y)} & {Z_*(E\otimes\Omega W)} & {Z_*(E\otimes X)} & {Z_*(E\otimes Y)} & {Z_*(E\otimes W)} & {Z_*(E\otimes \Sigma X)} & {Z_*(E\otimes \Sigma Y)}
		\arrow["{L_*^E(\Omega g)}", from=1-1, to=1-2]
		\arrow["{L_*^E(\wt h)}", from=1-2, to=1-3]
		\arrow["{L_*^E(f)}", from=1-3, to=1-4]
		\arrow["{L_*^E(g)}", from=1-4, to=1-5]
		\arrow["{L_*^E(h)}", from=1-5, to=1-6]
		\arrow["{L_*^E(\Sigma f)}", from=1-6, to=1-7]
		\arrow["{\Phi_{Z,\Omega Y}}"', from=1-1, to=2-1]
		\arrow["{Z_*(E\otimes \Omega g)}"', from=2-1, to=2-2]
		\arrow["{Z_*(E\otimes h)}"', from=2-5, to=2-6]
		\arrow["{Z_*(E\otimes\Sigma f)}"', from=2-6, to=2-7]
		\arrow["{\Phi_{Z,\Omega W}}"', from=1-2, to=2-2]
		\arrow["{Z_*(E\otimes f)}"', from=2-3, to=2-4]
		\arrow["{\Phi_{Z,X}}"', from=1-3, to=2-3]
		\arrow["{\Phi_{Z,Y}}"', from=1-4, to=2-4]
		\arrow["{Z_*(E\otimes g)}"', from=2-4, to=2-5]
		\arrow["{\Phi_{Z,W}}"', from=1-5, to=2-5]
		\arrow["{\Phi_{Z,\Sigma X}}"', from=1-6, to=2-6]
		\arrow["{\Phi_{Z,\Sigma Y}}"', from=1-7, to=2-7]
		\arrow["{Z_*(E\otimes\wt h)}"', from=2-2, to=2-3]
	\end{tikzcd}\]
	This diagram further commutes by naturality of $\Phi_{Z,-}$. Now, supposing that two of three of $X$, $Y$, and $W$ belong to $\cE$, by  \autoref{Phi_Z,W_iso_implies_Phi_S^aZ,W_and_Phi_Z,S^aW_are_isos}, if $\Phi_{Z,V}$ is an isomorphism for some object $V$ in $\cSH$ then $\Phi_{Z,\Omega V}$ and $\Phi_{Z,\Sigma V}$ are. Thus by the five lemma, it follows that the middle three vertical arrows in the above diagram are necessarily all isomorphisms if any two of them are, so we have shown that $\cE$ is closed under two-of-three for exact triangles, as desired.

	Finally, it remains to show that $\cE$ is closed under arbitrary coproducts. Let $\{W_i\}_{i\in I}$ be a collection of objects in $\cE$ indexed by some (small) set $I$. Then we'd like to show that $W:=\bigoplus_iW_i$ belongs to $\cE$. First of all, note that $-\otimes-$ preserves arbitrary coproducts in each argument, as it has a right adjoint $F(-,-)$. Thus without loss of generality, given any object $X$ in $\cSH$, we may take $\bigoplus_iX\otimes W_i=X\otimes\bigoplus_iW_i$ (as $X\otimes\bigoplus_iW_i$ \emph{is} a coproduct of all the $X\otimes W_i$'s). Now, recall that we have chosen each $S^a$ to be a compact object (\autoref{defn_compact}), so that given any object $X$ and collection of objects $\{Y_i\}_{i\in I}$ in $\cSH$, if $Y:=\bigoplus_{i\in I}Y_i$, then the canonical map
	\[q_{X,Y_i}:\bigoplus_i X_*(Y_i)=\bigoplus_i[S^*,X\otimes Y_i]\to[S^*,\bigoplus_iX\otimes Y_i]=[S^*,X\otimes Y]=X_*(Y)\]
	is an isomorphism, natural in $Y_i$ for each $i$. Note in particular that $q_{E,W_i}$ is an isomorphism of left $\pi_*(E)$-modules. To see this, first note by additivity of $q_{E,W_i}$, it suffices to check that $q_{E,W_i}(r\cdot x)=r\cdot q_{E,W_i}(x)$ for each homogeneous $r\in\pi_*(E)$ and homogeneous $x\in E_*(W_i)$ for some $i$, as such $x$ generate $\bigoplus_i E_*(W_i)$ by definition. Then given $r:S^a\to E$ and $x:S^b\to E\otimes W_i$, consider the following diagram	
	% https://q.uiver.app/#q=WzAsOCxbMCwwLCJTXnthK2J9Il0sWzEsMCwiU15hXFxvdGltZXMgU15iIl0sWzIsMCwiRVxcb3RpbWVzIEVcXG90aW1lcyBXX2kiXSxbMywwLCJFXFxvdGltZXMgXFxiaWdvcGx1c19pKEVcXG90aW1lcyBXX2kpIl0sWzMsMSwiRVxcb3RpbWVzIEVcXG90aW1lcyBXIl0sWzMsMiwiRVxcb3RpbWVzIFciXSxbMiwzLCJFXFxvdGltZXMgV19pIl0sWzMsMywiXFxiaWdvcGx1c19pKEVcXG90aW1lcyBXX2kpIl0sWzAsMSwiXFxwaGlfe2EsYn0iXSxbMSwyLCJ4XFxvdGltZXMgeSJdLFsyLDMsIkVcXG90aW1lcyBcXGlvdGFfe0VcXG90aW1lcyBXX2l9Il0sWzMsNCwiIiwwLHsibGV2ZWwiOjIsInN0eWxlIjp7ImhlYWQiOnsibmFtZSI6Im5vbmUifX19XSxbMiw2LCJcXG11XFxvdGltZXMgV19pIiwyXSxbNiw3LCJcXGlvdGFfe0VcXG90aW1lcyBXX2l9IiwyXSxbNyw1LCIiLDEseyJsZXZlbCI6Miwic3R5bGUiOnsiaGVhZCI6eyJuYW1lIjoibm9uZSJ9fX1dLFs2LDUsIkVcXG90aW1lcyBcXGlvdGFfe1dfaX0iLDFdLFsyLDQsIkVcXG90aW1lcyBFXFxvdGltZXMgXFxpb3RhX3tXX2l9IiwxXSxbNCw1LCJcXG11XFxvdGltZXMgVyJdXQ==
	\[\begin{tikzcd}
		{S^{a+b}} & {S^a\otimes S^b} & {E\otimes E\otimes W_i} & {E\otimes \bigoplus_i(E\otimes W_i)} \\
		&&& {E\otimes E\otimes W} \\
		&&& {E\otimes W} \\
		&& {E\otimes W_i} & {\bigoplus_i(E\otimes W_i)}
		\arrow["{\phi_{a,b}}", from=1-1, to=1-2]
		\arrow["{x\otimes y}", from=1-2, to=1-3]
		\arrow["{E\otimes \iota_{E\otimes W_i}}", from=1-3, to=1-4]
		\arrow[Rightarrow, no head, from=1-4, to=2-4]
		\arrow["{\mu\otimes W_i}"', from=1-3, to=4-3]
		\arrow["{\iota_{E\otimes W_i}}"', from=4-3, to=4-4]
		\arrow[Rightarrow, no head, from=4-4, to=3-4]
		\arrow["{E\otimes \iota_{W_i}}"{description}, from=4-3, to=3-4]
		\arrow["{E\otimes E\otimes \iota_{W_i}}"{description}, from=1-3, to=2-4]
		\arrow["{\mu\otimes W}", from=2-4, to=3-4]
	\end{tikzcd}\]
	where $\iota_{E\otimes W_i}:E\otimes W_i\into\bigoplus_i(E\otimes W_i)$ and $\iota_{W_i}:W_i\into\bigoplus_iW_i$ are the maps determined by the definition of the coproduct. Commutativity of the two triangles is by the fact that $E\otimes-$ is colimit preserving. Commutativity of the trapezoid is functoriality of $-\otimes-$. Thus, since $q_{E,W_i}$ is a homomorphism of left $A$-graded $\pi_*(E)$-modules, the top right arrow in the following diagram is well-defined:
	% https://q.uiver.app/#q=WzAsNixbMSwwLCJaXyooRSlcXG90aW1lc197XFxwaV8qKEUpfVxcYmlnb3BsdXNfaUVfKihXX2kpIl0sWzAsMiwiXFxiaWdvcGx1c19pWl8qKEVcXG90aW1lcyBXX2kpIl0sWzIsMiwiWl8qKEVcXG90aW1lcyBXKSJdLFsyLDAsIlpfKihFKVxcb3RpbWVzX3tcXHBpXyooRSl9IEVfKihXKSJdLFswLDAsIlxcYmlnb3BsdXNfaVpfKihFKVxcb3RpbWVzX3tcXHBpXyooRSl9RV8qKFdfaSkiXSxbMSwyLCJaXyooXFxiaWdvcGx1c19pRVxcb3RpbWVzIFdfaSkiXSxbMCwzLCJaXyooRSlcXG90aW1lc197XFxwaV8qKEUpfSBzX3tFLFdfaX0iXSxbMywyLCJcXFBoaV97WixXfSJdLFs0LDAsIiIsMCx7ImxldmVsIjoyLCJzdHlsZSI6eyJoZWFkIjp7Im5hbWUiOiJub25lIn19fV0sWzEsNSwic197WixFXFxvdGltZXMgV19pfSJdLFs1LDIsIiIsMCx7ImxldmVsIjoyLCJzdHlsZSI6eyJoZWFkIjp7Im5hbWUiOiJub25lIn19fV0sWzQsMSwiXFxiaWdvcGx1c19pXFxQaGlfe1osV19pfSIsMl1d
	\begin{equation}\label{kunneth_iso_pt_3_diag}\begin{tikzcd}
        {\bigoplus_iZ_*(E)\otimes_{\pi_*(E)}E_*(W_i)} & {Z_*(E)\otimes_{\pi_*(E)}\bigoplus_iE_*(W_i)} & {Z_*(E)\otimes_{\pi_*(E)} E_*(W)} \\
        \\
        {\bigoplus_iZ_*(E\otimes W_i)} & {Z_*(\bigoplus_iE\otimes W_i)} & {Z_*(E\otimes W)}
        \arrow["{Z_*(E)\otimes_{\pi_*(E)} q_{E,W_i}}", from=1-2, to=1-3]
        \arrow["{\Phi_{Z,W}}", from=1-3, to=3-3]
        \arrow[Rightarrow, no head, from=1-1, to=1-2]
        \arrow["{q_{Z,E\otimes W_i}}", from=3-1, to=3-2]
        \arrow[Rightarrow, no head, from=3-2, to=3-3]
        \arrow["{\bigoplus_i\Phi_{Z,W_i}}"', from=1-1, to=3-1]
    \end{tikzcd}\end{equation}
	We wish to show this diagram commutes. Again, since each map here is a homomorphism, it suffices to chase generators. By definition, a generator of the top left element is a homogeneous pure tensor in $E_*(E)\otimes_{\pi_{*}(E)}E_*(W_i)$ for some $i$ in $I$. Given classes $x:S^a\to Z\otimes E$ in $Z_*(E)$ and $y:S^b\to E\otimes W_i$ in $E_*(W_i)$, consider the following diagram:
	% https://q.uiver.app/#q=WzAsOCxbMCwwLCJTXnthK2J9Il0sWzEsMCwiU15hXFxvdGltZXMgU15iIl0sWzIsMCwiWlxcb3RpbWVzIEVcXG90aW1lcyBFXFxvdGltZXMgV19pIl0sWzMsMCwiWlxcb3RpbWVzIEVcXG90aW1lcyBcXGJpZ29wbHVzX2lFXFxvdGltZXMgV19pIl0sWzMsMiwiWlxcb3RpbWVzIEVcXG90aW1lcyBXIl0sWzIsMSwiWlxcb3RpbWVzIEVcXG90aW1lcyBXX2kiXSxbMywxLCJaXFxvdGltZXMgRVxcb3RpbWVzIEVcXG90aW1lcyBXIl0sWzIsMiwiXFxiaWdvcGx1c19pWlxcb3RpbWVzIEVcXG90aW1lcyBXX2kiXSxbMCwxLCJcXHBoaV97YSxifSJdLFsxLDIsInhcXG90aW1lcyB5Il0sWzIsMywiWlxcb3RpbWVzIEVcXG90aW1lcyBcXGlvdGFfe0VcXG90aW1lcyBXX2l9Il0sWzIsNSwiWlxcb3RpbWVzIFxcbXVcXG90aW1lcyBXX2kiLDJdLFszLDYsIiIsMCx7ImxldmVsIjoyLCJzdHlsZSI6eyJoZWFkIjp7Im5hbWUiOiJub25lIn19fV0sWzYsNCwiWlxcb3RpbWVzIFxcbXVcXG90aW1lcyBXIl0sWzUsNywiXFxpb3RhX3taXFxvdGltZXMgRVxcb3RpbWVzIFdfaX0iLDJdLFs3LDQsIiIsMSx7ImxldmVsIjoyLCJzdHlsZSI6eyJoZWFkIjp7Im5hbWUiOiJub25lIn19fV0sWzUsNCwiWlxcb3RpbWVzIEVcXG90aW1lcyBcXGlvdGFfe1dfaX0iLDFdLFsyLDYsIlpcXG90aW1lcyBFXFxvdGltZXMgRVxcb3RpbWVzIFxcaW90YV97V19pfSIsMV1d
	\[\begin{tikzcd}
		{S^{a+b}} & {S^a\otimes S^b} & {Z\otimes E\otimes E\otimes W_i} & {Z\otimes E\otimes \bigoplus_iE\otimes W_i} \\
		&& {Z\otimes E\otimes W_i} & {Z\otimes E\otimes E\otimes W} \\
		&& {\bigoplus_iZ\otimes E\otimes W_i} & {Z\otimes E\otimes W}
		\arrow["{\phi_{a,b}}", from=1-1, to=1-2]
		\arrow["{x\otimes y}", from=1-2, to=1-3]
		\arrow["{Z\otimes E\otimes \iota_{E\otimes W_i}}", from=1-3, to=1-4]
		\arrow["{Z\otimes \mu\otimes W_i}"', from=1-3, to=2-3]
		\arrow[Rightarrow, no head, from=1-4, to=2-4]
		\arrow["{Z\otimes \mu\otimes W}", from=2-4, to=3-4]
		\arrow["{\iota_{Z\otimes E\otimes W_i}}"', from=2-3, to=3-3]
		\arrow[Rightarrow, no head, from=3-3, to=3-4]
		\arrow["{Z\otimes E\otimes \iota_{W_i}}"{description}, from=2-3, to=3-4]
		\arrow["{Z\otimes E\otimes E\otimes \iota_{W_i}}"{description}, from=1-3, to=2-4]
	\end{tikzcd}\]
	Unravelling definitions, the two outside compositions are the two ways to chase $x\otimes y$ around diagram (\ref{kunneth_iso_pt_3_diag}). The two triangles commute again by the fact that $-\otimes-$ preserves colimits in each argument. Commutativity of the inner parallelogram is functoriality of $-\otimes-$. Thus diagram (\ref{kunneth_iso_pt_3_diag}) tells us $\Phi_{Z,W}$ is an isomorphism, since $q_{E,W_i}$ and $q_{Z,E\otimes W_i}$ are isomorphisms, and $\Phi_{Z,W_i}$ is an isomorphism for each $i$ in $I$, meaning $\bigoplus_i\Phi_{W_i}$ is as well.

	Thus, we've shown the class $\cE$ of objects $W$ for which $\Phi_{Z,W}$ is an isomorphism contains the $S^a$'s, is closed under two-of-three for distinguished triangles, and is closed under arbitrary coproducts. Thus, it follows that $\cE$ contains the class of all cellular objects in $\cSH$, as desired.

	Now, suppose that $\pi_*(E\otimes W)$ is a flat left $\pi_*(E)$-module, then we'd like to show $\Phi_{Z,W}$ is an isomorphism for all cellular $Z$ in $\cSH$. Showing this is entirely analagous to above, so we only outline the argument. Let $\cE$ be the class of $Z$ in $\cSH$ such that $\Phi_{Z,W}$ is an isomorphism. Then in order to show $\cE$ contains every cellular object, it suffices to show it contains the $S^a$'s, is closed under two-of-three for distinguished triangles, and is closed under arbitrary coproducts. 
	
	To see $\cE$ contains the $S^a$'s, consider the map
	\[\Psi:\pi_*(S^a\otimes E\otimes W)\to \pi_*(S^a\otimes E)\otimes_{\pi_*(E)}\pi_*(E\otimes W)\]
	sending $x:S^b\to S^a\otimes E\otimes W$ to $\wt e\otimes\wt x$, where $\wt e\in\pi_a(S^a\otimes E)$ is the map $S^a\otimes e:S^a\to S^a\otimes E$, and $\wt x\in\pi_{b-a}(E\otimes W)$ is the map
	\[\wt x:S^{b-a}\xr{\phi_{-a,b}}S^{-a}\otimes S^b\xr{S^{-a}\otimes x}S^{-a}\otimes S^a\otimes E\otimes W\xr{\phi_{-a,a}^{-1}\otimes E\otimes W}E\otimes W.\]
	Then checking that $\Psi$ is a left and right inverse to $\Phi_{S^a,W}$ is entirely analagous, so that $S^a$ belongs to $\cE$ as desired.

	To see $\cE$ is closed under two-of-three for distinguished triangles, let
	\[X\xr fY\xr gZ\xr h\Sigma X\]
	be a distinguished triangle in $\cSH$. Then an analagous argument as above (using \autoref{dist_tri_LES} and \autoref{LES_remains_exact_after_tensor}) yields a long exact sequence of $A$-graded abelian groups
	% https://q.uiver.app/#q=WzAsNyxbMSwwLCJcXHBpXyooXFxPbWVnYSBZXFxvdGltZXMgRSkiXSxbMiwwLCJcXHBpXyooXFxPbWVnYSBaXFxvdGltZXMgRSkiXSxbMCwxLCJcXHBpXyooWFxcb3RpbWVzIEUpIl0sWzEsMSwiXFxwaV8qKFlcXG90aW1lcyBFKSJdLFsyLDEsIlxccGlfKihaXFxvdGltZXMgRSkiXSxbMCwyLCJcXHBpXyooXFxTaWdtYSBYXFxvdGltZXMgRSkiXSxbMSwyLCJcXHBpXyooXFxTaWdtYSBZXFxvdGltZXMgRSkiXSxbMCwxLCJcXHBpXyooXFxPbWVnYSBnXFxvdGltZXMgRSkiXSxbMSwyLCJcXHBpXyooXFx3dCBoXFxvdGltZXMgRSkiLDFdLFsyLDMsIlxccGlfKihmXFxvdGltZXMgRSkiLDJdLFszLDQsIlxccGlfKihnXFxvdGltZXMgRSkiXSxbNCw1LCJcXHBpXyooaFxcb3RpbWVzIEUpIiwxXSxbNSw2LCJcXHBpXyooXFxTaWdtYSBmXFxvdGltZXMgRSkiLDJdXQ==
	\[\begin{tikzcd}
		& {\pi_*(\Omega Y\otimes E)} & {\pi_*(\Omega Z\otimes E)} \\
		{\pi_*(X\otimes E)} & {\pi_*(Y\otimes E)} & {\pi_*(Z\otimes E)} \\
		{\pi_*(\Sigma X\otimes E)} & {\pi_*(\Sigma Y\otimes E)}
		\arrow["{\pi_*(\Omega g\otimes E)}", from=1-2, to=1-3]
		\arrow["{\pi_*(\wt h\otimes E)}"{description}, from=1-3, to=2-1]
		\arrow["{\pi_*(f\otimes E)}"', from=2-1, to=2-2]
		\arrow["{\pi_*(g\otimes E)}", from=2-2, to=2-3]
		\arrow["{\pi_*(h\otimes E)}"{description}, from=2-3, to=3-1]
		\arrow["{\pi_*(\Sigma f\otimes E)}"', from=3-1, to=3-2]
	\end{tikzcd}\]
	Then since $\pi_*(E\otimes W)$ is a flat left $\pi_*(E)$-module, we can tensor the above long exact sequence with $\pi_*(E\otimes W)$ on the right to obtain a long exact sequence which fits in the left column of the following commuting diagram:
	% https://q.uiver.app/#q=WzAsMTQsWzAsMCwiUl5FXyooXFxPbWVnYSBZKSJdLFswLDEsIlJeRV8qKFxcT21lZ2EgWikiXSxbMCwyLCJSXkVfKihYKSJdLFswLDMsIlJeRV8qKFkpIl0sWzAsNCwiUl5FXyooWikiXSxbMCw1LCJSXkVfKihcXFNpZ21hIFgpIl0sWzAsNiwiUl5FXyooXFxTaWdtYSBZKSJdLFsxLDAsIlxccGlfKihcXE9tZWdhIFlcXG90aW1lcyBFXFxvdGltZXMgVykiXSxbMSwxLCJcXHBpXyooXFxPbWVnYSBaXFxvdGltZXMgRVxcb3RpbWVzIFcpIl0sWzEsMiwiXFxwaV8qKFhcXG90aW1lcyBFXFxvdGltZXMgVykiXSxbMSwzLCJcXHBpXyooWVxcb3RpbWVzIEVcXG90aW1lcyBXKSJdLFsxLDQsIlxccGlfKihaXFxvdGltZXMgRVxcb3RpbWVzIFcpIl0sWzEsNSwiXFxwaV8qKFxcU2lnbWEgWFxcb3RpbWVzIEVcXG90aW1lcyBXKSJdLFsxLDYsIlxccGlfKihcXFNpZ21hIFlcXG90aW1lcyBFXFxvdGltZXMgVykiXSxbMCwxLCJSXypeRShcXE9tZWdhIGcpIiwyXSxbMSwyLCJSXypeRShcXHd0IGgpIiwyXSxbMiwzLCJSXypeRShmKSIsMl0sWzMsNCwiUl8qXkUoZykiLDJdLFs0LDUsIlJfKl5FKGgpIiwyXSxbNSw2LCJSXypeRShcXFNpZ21hIGYpIiwyXSxbMCw3LCJcXFBoaV97XFxPbWVnYSBZLFd9Il0sWzcsOCwiXFxwaV8qKFxcT21lZ2EgZ1xcb3RpbWVzIEVcXG90aW1lcyBXKSJdLFs4LDksIlxccGlfKihcXHd0IGhcXG90aW1lcyBFXFxvdGltZXMgVykiXSxbOSwxMCwiXFxwaV8qKGZcXG90aW1lcyBFXFxvdGltZXMgVykiXSxbMTAsMTEsIlxccGlfKihnXFxvdGltZXMgRVxcb3RpbWVzIFcpIl0sWzExLDEyLCJcXHBpXyooaFxcb3RpbWVzIEVcXG90aW1lcyBXKSJdLFsxMiwxMywiXFxwaV8qKFxcU2lnbWEgZlxcb3RpbWVzIEVcXG90aW1lcyBXKSJdLFsxLDgsIlxcUGhpX3tcXE9tZWdhIFosV30iXSxbMiw5LCJcXFBoaV97WCxXfSJdLFszLDEwLCJcXFBoaV97WSxXfSJdLFs0LDExLCJcXFBoaV97WixXfSJdLFs1LDEyLCJcXFBoaV97XFxTaWdtYSBYLFd9Il0sWzYsMTMsIlxcUGhpX3tcXFNpZ21hIFksV30iXV0=
	\[\begin{tikzcd}[row sep=small]
		{R^E_*(\Omega Y)} & {\pi_*(\Omega Y\otimes E\otimes W)} \\
		{R^E_*(\Omega Z)} & {\pi_*(\Omega Z\otimes E\otimes W)} \\
		{R^E_*(X)} & {\pi_*(X\otimes E\otimes W)} \\
		{R^E_*(Y)} & {\pi_*(Y\otimes E\otimes W)} \\
		{R^E_*(Z)} & {\pi_*(Z\otimes E\otimes W)} \\
		{R^E_*(\Sigma X)} & {\pi_*(\Sigma X\otimes E\otimes W)} \\
		{R^E_*(\Sigma Y)} & {\pi_*(\Sigma Y\otimes E\otimes W)}
		\arrow["{R_*^E(\Omega g)}"', from=1-1, to=2-1]
		\arrow["{R_*^E(\wt h)}"', from=2-1, to=3-1]
		\arrow["{R_*^E(f)}"', from=3-1, to=4-1]
		\arrow["{R_*^E(g)}"', from=4-1, to=5-1]
		\arrow["{R_*^E(h)}"', from=5-1, to=6-1]
		\arrow["{R_*^E(\Sigma f)}"', from=6-1, to=7-1]
		\arrow["{\Phi_{\Omega Y,W}}", from=1-1, to=1-2]
		\arrow["{\pi_*(\Omega g\otimes E\otimes W)}", from=1-2, to=2-2]
		\arrow["{\pi_*(\wt h\otimes E\otimes W)}", from=2-2, to=3-2]
		\arrow["{\pi_*(f\otimes E\otimes W)}", from=3-2, to=4-2]
		\arrow["{\pi_*(g\otimes E\otimes W)}", from=4-2, to=5-2]
		\arrow["{\pi_*(h\otimes E\otimes W)}", from=5-2, to=6-2]
		\arrow["{\pi_*(\Sigma f\otimes E\otimes W)}", from=6-2, to=7-2]
		\arrow["{\Phi_{\Omega Z,W}}", from=2-1, to=2-2]
		\arrow["{\Phi_{X,W}}", from=3-1, to=3-2]
		\arrow["{\Phi_{Y,W}}", from=4-1, to=4-2]
		\arrow["{\Phi_{Z,W}}", from=5-1, to=5-2]
		\arrow["{\Phi_{\Sigma X,W}}", from=6-1, to=6-2]
		\arrow["{\Phi_{\Sigma Y,W}}", from=7-1, to=7-2]
	\end{tikzcd}\]
	where $R_*^E$ denotes the functor from $\cSH$ to $A$-graded abelian groups sending $X\mapsto\pi_*(X\otimes E)\otimes_{\pi_*(E)}\pi_*(E\otimes W)$, so that $\Phi_{-,W}$ is a natural homomorphism $R_*^E(-)\Rightarrow \pi_*(-\otimes E\otimes W)$. Then finally by \autoref{Phi_Z,W_iso_implies_Phi_S^aZ,W_and_Phi_Z,S^aW_are_isos} and the five lemma, if any two of three of the middle three horizontal arrows are isomorphisms, then all three of the horizontal arrows are isomorphisms, as desired.

	Finally, in order to show $\cE$ is closed under arbitrary coproducts, suppose we have a collection of objects $\{Z_i\}_{i\in I}$ in $\cE$ indexed by some (small) set $\cE$. Then we'd like to show $Z:=\bigoplus_{i\in I}Z_i$ also belongs to $\cE$. First note that since the $S^a$'s are compact, for any object $Y$ we have isomorphisms
	\[q_{Z_i,Y}:\bigoplus_{i}{Z_i}_*(Y)=\bigoplus_i[S^*,Z_i\otimes Y]\to[S^*,\bigoplus_i(Z_i\otimes Y)]=[S^*,Z\otimes Y]=Z_*(Y).\]
	It is straightforward to verify that $q_{Z_i,E}:\bigoplus_i{Z_i}_*(E)\to Z_*(E)$ is not only an isomorphism of abelian groups, but an isomorphism of right $A$-graded $\pi_*(E)$-modules, so that the top arrow in the following diagram is well-defined:
    % https://q.uiver.app/#q=WzAsNSxbMSwwLCJcXGJpZ29wbHVzX2lcXGxlZnQoe1pfaX1fKihFKVxccmlnaHQpXFxvdGltZXNfe1xccGlfKihFKX1FXyooVykiXSxbMCwyLCJcXGJpZ29wbHVzX2l7Wl9pfV8qKEVcXG90aW1lcyBXKSJdLFszLDIsIlpfKihFXFxvdGltZXMgVykiXSxbMywwLCJaXyooRSlcXG90aW1lc197XFxwaV8qKEUpfSBFXyooVykiXSxbMCwwLCJcXGJpZ29wbHVzX2lcXGxlZnQoe1pfaX1fKihFKVxcb3RpbWVzX3tcXHBpXyooRSl9RV8qKFcpXFxyaWdodCkiXSxbMCwzLCJxX3taX2ksRX1cXG90aW1lcyBFXyooV19pKSJdLFszLDIsIlxcUGhpX3taLFd9Il0sWzQsMCwiIiwwLHsibGV2ZWwiOjIsInN0eWxlIjp7ImhlYWQiOnsibmFtZSI6Im5vbmUifX19XSxbNCwxLCJcXGJpZ29wbHVzX2lcXFBoaV97Wl9pLFd9IiwyXSxbMSwyLCJxX3taX2ksRVxcb3RpbWVzIFd9Il1d
    \[\begin{tikzcd}
        {\bigoplus_i\left({Z_i}_*(E)\otimes_{\pi_*(E)}E_*(W)\right)} & {\bigoplus_i\left({Z_i}_*(E)\right)\otimes_{\pi_*(E)}E_*(W)} && {Z_*(E)\otimes_{\pi_*(E)} E_*(W)} \\
        \\
        {\bigoplus_i{Z_i}_*(E\otimes W)} &&& {Z_*(E\otimes W)}
        \arrow["{q_{Z_i,E}\otimes E_*(W_i)}", from=1-2, to=1-4]
        \arrow["{\Phi_{Z,W}}", from=1-4, to=3-4]
        \arrow[Rightarrow, no head, from=1-1, to=1-2]
        \arrow["{\bigoplus_i\Phi_{Z_i,W}}"', from=1-1, to=3-1]
        \arrow["{q_{Z_i,E\otimes W}}", from=3-1, to=3-4]
    \end{tikzcd}\]
	Then a simple diagram chase yields the diagram commutes, so that $\Phi_{Z,W}$ is an isomorphism, assuming all the $\Phi_{Z_i,W}$'s are.
\end{proof}

\subsection{Modules over monoid objects in $\cSH$}

Now, before we prove our next theorem (an analog of the universal coefficient theorem in $\cSH$), we need to develop some of the theory of (left) module objects over monoid objects in $\cSH$. For a review of the basic definitions and properties of module objects over monoid objects in symmetric monoidal categories, see \Cref{subsection:modules_over_monoids}. Recall specifically that given a monoid object $(E,\mu,e)$ in $\cSH$, the category $E\text-\Mod$ of (left) $E$-module objects is additive (\autoref{E-Mod,free,forgetful_are_additive}), and the forgetful functor $E\text-\Mod\to\cSH$ preserves arbitrary coproducts and has a right adjoint $\cSH\to E\text-\Mod$ taking an object $X$ in $\cSH$ to the \emph{free $E$-module} $E\otimes X$ (\autoref{free_forgetful_E-Mod}).

\begin{proposition}\label{E-module_N_implies_pi*N_is_pi*E_module}
	Let $(E,\mu,e)$ be a monoid object in $\cSH$. Then the assignment $\pi_*:(N,\kappa)\mapsto\pi_*(N)$ yields an additive functor from $E\text-\Mod$ to the category $\pi_*(E)\text-\Mod^A$ of $A$-graded left $\pi_*(E)$-modules and degree-preserving homomorphisms between them. In particular, if $(N,\kappa)$ is an $E$-module object in $\cSH$, then we view it with its \emph{canonical} $A$-graded left $\pi_*(E)$-module structure given by the graded map
	\[\pi_*(E)\times\pi_*(N)\to\pi_*(N)\]
	sending a class $r:S^a\to E$ and $x:S^b\to N$ to the composition
	\[r\cdot x:S^{a+b}\xr{\phi_{a,b}} S^a\otimes S^b\xr{r\otimes x}E\otimes N\xr\kappa N.\]
	%Furthermore, this functor is lax $A$-graded (\autoref{A-graded_functor_defn}), with structure map $\pi_*(\Sigma^aN)\cong\pi_{*-a}(N)$ which sends $x:S^b\to S^a\otimes N$ to the composition
	%\[S^{b-a}\xr{\phi_{b,-a}}S^b\otimes S^{-a}\xr{x\otimes S^{-a}}S^a\otimes N\otimes S^{-a}\xr{\tau\otimes S^{-a}}N\otimes S^a\otimes S^{-a}\xr{N\otimes\phi_{a,-a}^{-1}}N.\]
\end{proposition}
\begin{proof}
	First let $(N,\kappa)$ be an $E$-module object. Let $a,b,c\in A$ and $x,x':S^a\to N$, $y:S^b\to E$, and $z, z'\in S^c\to E$. Then by \autoref{A-graded_module}, it suffices to show that
	\begin{enumerate}
		\item $y\cdot( x+ x')= y\cdot x+ y\cdot x'$, 
		\item $( z+ z')\cdot x= z\cdot x+ z'\cdot x$,
		\item $(zy)\cdot  x= z\cdot( y\cdot x)$,
		\item $e\cdot x= x$.
	\end{enumerate}
	The first two axioms follow by \autoref{bilinear}. To see $(3)$, consider the diagram:
	% https://q.uiver.app/#q=WzAsNixbMCwxLCJTXnthK2IrY30iXSxbMSwxLCJTXmNcXG90aW1lcyBTXmJcXG90aW1lcyBTXmEiXSxbMiwxLCJFXFxvdGltZXMgRVxcb3RpbWVzIE4iXSxbMywwLCJFXFxvdGltZXMgTiJdLFszLDEsIk4iXSxbMywyLCJFXFxvdGltZXMgTiJdLFswLDEsIlxcY29uZyJdLFsxLDIsInpcXG90aW1lcyB5XFxvdGltZXMgeCJdLFsyLDMsIkVcXG90aW1lc1xca2FwcGEiXSxbMyw0LCJcXGthcHBhIl0sWzIsNSwiXFxtdVxcb3RpbWVzIE4iLDJdLFs1LDQsIlxca2FwcGEiLDJdXQ==
	\[\begin{tikzcd}
		&&& {E\otimes N} \\
		{S^{a+b+c}} & {S^c\otimes S^b\otimes S^a} & {E\otimes E\otimes N} & N \\
		&&& {E\otimes N}
		\arrow["\cong", from=2-1, to=2-2]
		\arrow["{z\otimes y\otimes x}", from=2-2, to=2-3]
		\arrow["E\otimes\kappa", from=2-3, to=1-4]
		\arrow["\kappa", from=1-4, to=2-4]
		\arrow["{\mu\otimes N}"', from=2-3, to=3-4]
		\arrow["\kappa"', from=3-4, to=2-4]
	\end{tikzcd}\]
	It commutes by coherence for $\kappa$. By functoriality of $-\otimes-$, the two outside compositions equal $z\cdot(y\cdot x)$ on the top and $(z\cdot y)\cdot x$ on the bottom. Hence, they are equal, as desired.

	Next, to see $(4)$, consider the following diagram:
	% https://q.uiver.app/#q=WzAsNCxbMCwwLCJTXmEiXSxbMSwxLCJOIl0sWzIsMCwiTiJdLFsxLDIsIkVcXG90aW1lcyBOIl0sWzEsMiwiIiwxLHsibGV2ZWwiOjIsInN0eWxlIjp7ImhlYWQiOnsibmFtZSI6Im5vbmUifX19XSxbMSwzLCJlXFxvdGltZXMgTiIsMV0sWzAsMiwieCJdLFszLDIsIlxca2FwcGEiLDIseyJjdXJ2ZSI6M31dLFswLDEsIngiLDJdLFswLDMsImVcXG90aW1lcyB4IiwyLHsiY3VydmUiOjN9XV0=
	\[\begin{tikzcd}
		{S^a} && N \\
		& N \\
		& {E\otimes N}
		\arrow[Rightarrow, no head, from=2-2, to=1-3]
		\arrow["{e\otimes N}"{description}, from=2-2, to=3-2]
		\arrow["x", from=1-1, to=1-3]
		\arrow["\kappa"', curve={height=18pt}, from=3-2, to=1-3]
		\arrow["x"', from=1-1, to=2-2]
		\arrow["{e\otimes x}"', curve={height=18pt}, from=1-1, to=3-2]
	\end{tikzcd}\]
	The top triangle commutes by definition. The left triangle commutes by functoriality of $-\otimes-$. The right triangle commutes by unitality of $\kappa$. The top composition is $ x$ while the bottom is $e\cdot x$, thus they are necessarily equal since the diagram commutes.

	Now, we'd like to show that if $f:(N,\kappa)\to(N',\kappa)$ is a homomorphism of $E$-module objects, then $\pi_*(f):\pi_*(N)\to\pi_*(N')$ is a homomorphism of left $\pi_*(E)$-modules. To see this, let $r:S^a\to E$ in $\pi_a(E)$ and $x,x:S^b\to N$ in $\pi_b(N)$. We'd like to show that $\pi_*(f)(x+x')=\pi_*(f)(x)+\pi_*(f)(x')$ and $\pi_*(f)(r\cdot x)=r\cdot\pi_*(f)(x)$. To see the former, consider the following diagram:
	% https://q.uiver.app/#q=WzAsNixbMCwxLCJTXmEiXSxbMSwxLCJTXmFcXG9wbHVzIFNeYSJdLFsyLDEsIk5cXG9wbHVzIE4iXSxbMywwLCJOJ1xcb3BsdXMgTiciXSxbMywxLCJOJyJdLFszLDIsIk4iXSxbMCwxLCJcXERlbHRhIl0sWzEsMiwieFxcb3BsdXMgeCciXSxbMiwzLCJmXFxvcGx1cyBmIl0sWzMsNCwiXFxuYWJsYSJdLFsyLDUsIlxcbmFibGEiLDJdLFs1LDQsImYiLDJdXQ==
	\[\begin{tikzcd}
		&&& {N'\oplus N'} \\
		{S^a} & {S^a\oplus S^a} & {N\oplus N} & {N'} \\
		&&& N
		\arrow["\Delta", from=2-1, to=2-2]
		\arrow["{x\oplus x'}", from=2-2, to=2-3]
		\arrow["{f\oplus f}", from=2-3, to=1-4]
		\arrow["\nabla", from=1-4, to=2-4]
		\arrow["\nabla"', from=2-3, to=3-4]
		\arrow["f"', from=3-4, to=2-4]
	\end{tikzcd}\]
	It commutes by naturality of $\nabla$ in an additive category. The top composition is $\pi_*(f)(x)+\pi_*(f)(x')$, while the bottom is $\pi_*(f)(x+x')$, so they are equal as desired. To see that $\pi_*(f)(r\cdot x)=r\cdot \pi_*(f)(x)$, consider the following diagram:
	% https://q.uiver.app/#q=WzAsNixbMCwxLCJTXnthK2J9Il0sWzEsMSwiU15iXFxvdGltZXMgU15hIl0sWzIsMSwiRVxcb3RpbWVzIE4iXSxbMywwLCJFXFxvdGltZXMgTiciXSxbMywxLCJOJyJdLFszLDIsIk4iXSxbMCwxLCJcXHBoaV97YixhfSJdLFsxLDIsInJcXG90aW1lcyB4Il0sWzIsMywiRVxcb3RpbWVzIGYiXSxbMyw0LCJcXGthcHBhJyJdLFsyLDUsIlxca2FwcGEiLDJdLFs1LDQsImYiLDJdXQ==
	\[\begin{tikzcd}
		&&& {E\otimes N'} \\
		{S^{a+b}} & {S^b\otimes S^a} & {E\otimes N} & {N'} \\
		&&& N
		\arrow["{\phi_{b,a}}", from=2-1, to=2-2]
		\arrow["{r\otimes x}", from=2-2, to=2-3]
		\arrow["{E\otimes f}", from=2-3, to=1-4]
		\arrow["{\kappa'}", from=1-4, to=2-4]
		\arrow["\kappa"', from=2-3, to=3-4]
		\arrow["f"', from=3-4, to=2-4]
	\end{tikzcd}\]
	It commutes by the fact that $f$ is a homomorphism of $E$-module objects. The bottom composition is $\pi_*(f)(r\cdot x)$, while the top composition is $r\cdot \pi_*(f)(x)$, so they are equal, as desired.
 
	Next we claim this functor is additive. It suffices to show it preserves the zero object and preserves coproducts. To see the former, note that $\pi_*(0)=[S^*,0]=0$ by definition, since $0$ is terminal. To see the latter, we need to show that given $(N,\kappa),(N',\kappa')\in E\text-\Mod$ that $\pi_*(N)\oplus\pi_*(N')\cong\pi_*(N\oplus N')$, and that the following diagram commutes:
	% https://q.uiver.app/#q=WzAsMyxbMSwxLCJcXHBpXyooTlxcb3BsdXMgTicpIl0sWzAsMCwiXFxwaV8qKE4pIl0sWzAsMSwiXFxwaV8qKE4pXFxvcGx1c1xccGlfKihOJykiXSxbMSwwLCJcXHBpXyooXFxpb3RhX04pIl0sWzIsMCwiXFxjb25nIiwyXSxbMSwyLCJcXGlvdGFfe1xccGlfKihOKX0iLDJdXQ==
	\[\begin{tikzcd}
		{\pi_*(N)} \\
		{\pi_*(N)\oplus\pi_*(N')} & {\pi_*(N\oplus N')}
		\arrow["{\pi_*(\iota_N)}", from=1-1, to=2-2]
		\arrow["\cong"', from=2-1, to=2-2]
		\arrow["{\iota_{\pi_*(N)}}"', from=1-1, to=2-1]
	\end{tikzcd}\]
	Since each $S^a$ is compact, for all $a,b\in A$ we have isomorphisms
	\[\pi_a(N)\oplus\pi_a(N')=[S^a,N]\oplus[S^a,N']\cong[S^a,N\oplus N']=\pi_a(N\oplus N'),\]
	and these combine together to yield $A$-graded isomorphisms $\pi_*(N)\oplus\pi_*(N')\xr\cong\pi_*(N\oplus N')$. To see the above diagram commutes, note that since everything is an $A$-graded homomorphism of $A$-graded abelian groups, it suffices to chase homogeneous elements around to show it commutes. Indeed, it is entirely straightforward, by unravelling definitions, that both compositions around the diagram take a generator $x:S^a\to N$ in $\pi_a(N)$ to the composition
	\[S^a\xr xN\xr{\iota_N}N\oplus N'.\]
	Thus, we have shown that $\pi_*$ preserves all finite coproducts, so it is additive.
%	Finally, we claim this functor is lax $A$-graded. To that end, we need to show the given map
%	\[t^a_{(N,\kappa)}\pi_*(\Sigma^aN)\to\pi_{*-a}(N),\]
%	is an isomorphism of left $\pi_*(E)$-modules for all $(N,\kappa)$ in $E\text-\Mod$. First, note that by unravelling definitions it factors as
%	% https://q.uiver.app/#q=WzAsNyxbMCwwLCJcXHBpXyooXFxTaWdtYV5hTikiXSxbMSwwLCJbU14qLFNeYVxcb3RpbWVzIE5dIl0sWzEsMSwiW1NeKlxcb3RpbWVzIFNeey1hfSxTXmFcXG90aW1lcyBOXFxvdGltZXMgU157LWF9XSJdLFsxLDIsIltTXipcXG90aW1lcyBTXnstYX0sTlxcb3RpbWVzIFNeYVxcb3RpbWVzIFNeey1hfV0iXSxbMSwzLCJbU14qXFxvdGltZXMgU157LWF9LE5dIl0sWzEsNCwiW1NeeyotYX0sTl0iXSxbMiw0LCJcXHBpX3sqLWF9KE4pIl0sWzAsMSwiIiwwLHsibGV2ZWwiOjIsInN0eWxlIjp7ImhlYWQiOnsibmFtZSI6Im5vbmUifX19XSxbMSwyLCItXFxvdGltZXMgU157LWF9Il0sWzIsMywieyhcXHRhdVxcb3RpbWVzIFNeey1hfSl9XyoiXSxbMyw0LCJ7KE5cXG90aW1lcyBcXHBoaV97YSwtYX1eey0xfSl9XyoiXSxbNCw1LCJ7KFxccGhpX3sqLC1hfSl9XioiXSxbNSw2LCIiLDAseyJsZXZlbCI6Miwic3R5bGUiOnsiaGVhZCI6eyJuYW1lIjoibm9uZSJ9fX1dXQ==
%	\[\begin{tikzcd}
%		{\pi_*(\Sigma^aN)} & {[S^*,S^a\otimes N]} \\
%		& {[S^*\otimes S^{-a},S^a\otimes N\otimes S^{-a}]} \\
%		& {[S^*\otimes S^{-a},N\otimes S^a\otimes S^{-a}]} \\
%		& {[S^*\otimes S^{-a},N]} \\
%		& {[S^{*-a},N]} & {\pi_{*-a}(N)}
%		\arrow[Rightarrow, no head, from=1-1, to=1-2]
%		\arrow["{-\otimes S^{-a}}", from=1-2, to=2-2]
%		\arrow["{{(\tau\otimes S^{-a})}_*}", from=2-2, to=3-2]
%		\arrow["{{(N\otimes \phi_{a,-a}^{-1})}_*}", from=3-2, to=4-2]
%		\arrow["{{(\phi_{*,-a})}^*}", from=4-2, to=5-2]
%		\arrow[Rightarrow, no head, from=5-2, to=5-3]
%	\end{tikzcd}\]
%	The first vertical arrow is an isomorphism since $-\otimes S^a$ is an additive equivalence of $\cSH$ (\autoref{Sigma^a,Sigma^-a_adjoint_equiv}), and the other three arrows are given by composing with isomorphisms in an additive category. Thus, we have constructed $A$-graded isomorphisms of abelian groups $\pi_*(\Sigma^aX)\to\pi_{*-a}(X)$. It remains to show that this map is a homomorphism of left $\pi_*(E)$-modules. To that end, let $r:S^b\to E$ in $\pi_*(E)$ and $x:S^c\to S^a\otimes N$ in $\pi_*(\Sigma^aN)$, and consider the following diagram:
%	% https://q.uiver.app/#q=WzAsOSxbMCwwLCJTXntiK2MtYX0iXSxbMCwxLCJTXmJcXG90aW1lcyBTXmNcXG90aW1lcyBTXnstYX0iXSxbMCwyLCJFXFxvdGltZXMgU15hXFxvdGltZXMgTlxcb3RpbWVzIFNeey1hfSJdLFsxLDIsIkVcXG90aW1lcyBOXFxvdGltZXMgU15hXFxvdGltZXMgU157LWF9Il0sWzIsMiwiRU4iXSxbMCw0LCJTXmFcXG90aW1lcyBFXFxvdGltZXMgTlxcb3RpbWVzIFNeey1hfSJdLFsxLDQsIlNeYVxcb3RpbWVzIE5cXG90aW1lcyBTXnstYX0iXSxbMiw0LCJOXFxvdGltZXMgU15hXFxvdGltZXMgU157LWF9Il0sWzIsMywiTiJdLFswLDEsIlxccGhpIiwyXSxbMSwyLCJyXFxvdGltZXMgeFxcb3RpbWVzIFNeey1hfSIsMl0sWzIsMywiRVxcb3RpbWVzIFxcdGF1XFxvdGltZXMgIFNeey1hfSJdLFszLDQsIkVcXG90aW1lcyBOXFxvdGltZXMgXFxwaGlfe2EsLWF9XnstMX0iXSxbMiw1LCJcXHRhdVxcb3RpbWVzICBOXFxvdGltZXMgU157LWF9IiwyXSxbNSw2LCJTXmFcXG90aW1lcyBcXGthcHBhXFxvdGltZXMgIFNeey1hfSJdLFs2LDcsIlxcdGF1XFxvdGltZXMgU157LWF9Il0sWzQsOCwiXFxrYXBwYSJdLFs3LDgsIk5cXG90aW1lcyBcXHBoaV97YSwtYX1ee18xfSIsMV0sWzMsNywiXFxrYXBwYVxcb3RpbWVzIFNeYVxcb3RpbWVzIFNeey1hfSIsMV0sWzUsMywiXFx0YXVfe1NeYSxFXFxvdGltZXMgTn1cXG90aW1lcyBTXnstYX0iLDFdXQ==
%	\[\begin{tikzcd}
%		{S^{b+c-a}} \\
%		{S^b\otimes S^c\otimes S^{-a}} \\
%		{E\otimes S^a\otimes N\otimes S^{-a}} & {E\otimes N\otimes S^a\otimes S^{-a}} & EN \\
%		&& N \\
%		{S^a\otimes E\otimes N\otimes S^{-a}} & {S^a\otimes N\otimes S^{-a}} & {N\otimes S^a\otimes S^{-a}}
%		\arrow["\phi"', from=1-1, to=2-1]
%		\arrow["{r\otimes x\otimes S^{-a}}"', from=2-1, to=3-1]
%		\arrow["{E\otimes \tau\otimes  S^{-a}}", from=3-1, to=3-2]
%		\arrow["{E\otimes N\otimes \phi_{a,-a}^{-1}}", from=3-2, to=3-3]
%		\arrow["{\tau\otimes  N\otimes S^{-a}}"', from=3-1, to=5-1]
%		\arrow["{S^a\otimes \kappa\otimes  S^{-a}}", from=5-1, to=5-2]
%		\arrow["{\tau\otimes S^{-a}}", from=5-2, to=5-3]
%		\arrow["\kappa", from=3-3, to=4-3]
%		\arrow["{N\otimes \phi_{a,-a}^{_1}}"{description}, from=5-3, to=4-3]
%		\arrow["{\kappa\otimes S^a\otimes S^{-a}}"{description}, from=3-2, to=5-3]
%		\arrow["{\tau_{S^a,E\otimes N}\otimes S^{-a}}"{description}, from=5-1, to=3-2]
%	\end{tikzcd}\]
%	The top composition is $r\cdot t^a_{(N,\kappa)}(x)$, while the bottom composition is $t^a_{(N,\kappa)}(r\cdot x)$. The left triangle commutes by coherence for the symmetries. The bottom triangle commutes by naturality of $\tau$. The right triangle commutes by functoriality of $-\otimes-$. Thus, indeed $\pi_*:E\text-\Mod\to\pi_*(E)\text-\Mod(A)$ is lax $A$-graded, as desired.
\end{proof}

\begin{remark}
    In the above proposition, we have shown that given an $E$-module object $(N,\kappa)$ in $\cSH$, $\pi_*(N)$ is canonically an $A$-graded left $\pi_*(E)$-module. In particular, we may apply this proposition to the free $E$-module $E\otimes X$ (\autoref{free_forgetful_E-Mod}). It is straightforward to see, and we leave it to the reader to check, that the $A$-graded left $\pi_*(E)$-module structure on $E_*(X)=\pi_*(E\otimes X)$ induced by the above proposition is precisely the canonical module structure from \autoref{module}. In fact, the above proposition entirely subsumes the first half of \autoref{module} (although we give the two separate statements for the sake of clarity). Thus, there continues to be no ambiguity when talking about the left $\pi_*(E)$-module structure on $E_*(X)$.
\end{remark}

\begin{lemma}\label{suspension_of_module_object_is_module_object}
	Let $(E,\mu,e)$ be a monoid object in $\cSH$, and suppose $(N,\kappa)$ is a module object over $E$ (\autoref{left_module_object}). Then for all $a\in A$, the $a^\text{th}$ suspension $\Sigma^a N$ of $N$ is canonically an $E$-module object, with action map given by
	\[\kappa^a:E\otimes\Sigma^aN=E\otimes S^a\otimes N\xr{\tau\otimes N}S^a\otimes E\otimes N\xr{S^a\otimes\kappa}S^a\otimes N=\Sigma^aN.\]
	Furthermore, given an $E$-module homomorphism $f:(N,\kappa)\to (N',\kappa')$, $\Sigma^af:\Sigma^aN\to\Sigma^aN'$ is likewise an $E$-module homomorphism.
\end{lemma}
\begin{proof}
	In this proof, we are assuming that unitality and associativity hold up to strict equality, by the coherence theorem for monoidal categories. In order to show $(\Sigma^a N,\kappa^a)$ is a module object over $E$, we need to show $\kappa^a$ makes the two coherence diagrams in \autoref{left_module_object} commute. First, to see the first diagram commutes, consider the following diagram:
	% https://q.uiver.app/#q=WzAsNCxbMCwwLCJTXmFcXG90aW1lcyBOIl0sWzIsMCwiRVxcb3RpbWVzIFNeYVxcb3RpbWVzIE4iXSxbMiwyLCJTXmFcXG90aW1lcyBOIl0sWzIsMSwiU15hXFxvdGltZXMgRVxcb3RpbWVzIE4iXSxbMCwxLCJlXFxvdGltZXMgU15hXFxvdGltZXMgTiJdLFsxLDMsIlxcdGF1XFxvdGltZXMgTiJdLFszLDIsIlNeYVxcb3RpbWVzIFxca2FwcGEiXSxbMCwyLCIiLDIseyJsZXZlbCI6Miwic3R5bGUiOnsiaGVhZCI6eyJuYW1lIjoibm9uZSJ9fX1dLFswLDMsIlNeYVxcb3RpbWVzIGVcXG90aW1lcyBOIiwxXV0=
	\[\begin{tikzcd}
		{S^a\otimes N} && {E\otimes S^a\otimes N} \\
		&& {S^a\otimes E\otimes N} \\
		&& {S^a\otimes N}
		\arrow["{e\otimes S^a\otimes N}", from=1-1, to=1-3]
		\arrow["{\tau\otimes N}", from=1-3, to=2-3]
		\arrow["{S^a\otimes \kappa}", from=2-3, to=3-3]
		\arrow[Rightarrow, no head, from=1-1, to=3-3]
		\arrow["{S^a\otimes e\otimes N}"{description}, from=1-1, to=2-3]
	\end{tikzcd}\]
	The top inner triangle commutes by coherence for a symmetric monoidal category, and the bottom inner triangle commutes by the coherence condition for $\kappa$. To see the other module condition for $\wt\kappa$, consider the following diagram:
	% https://q.uiver.app/#q=WzAsOCxbMCwwLCJFXFxvdGltZXMgRVxcb3RpbWVzIFNeYVxcb3RpbWVzIE4iXSxbMiwwLCJFXFxvdGltZXMgU15hXFxvdGltZXMgTiJdLFsyLDEsIlNeYVxcb3RpbWVzIEVcXG90aW1lcyBOIl0sWzIsMiwiU15hXFxvdGltZXMgTiJdLFswLDEsIkVcXG90aW1lcyBTXmFcXG90aW1lcyBFXFxvdGltZXMgTiJdLFswLDIsIkVcXG90aW1lcyBTXmFcXG90aW1lcyBOIl0sWzEsMiwiU15hXFxvdGltZXMgRVxcb3RpbWVzIE4iXSxbMSwxLCJTXmFcXG90aW1lcyBFXFxvdGltZXMgRVxcb3RpbWVzIE4iXSxbMCwxLCJcXG11XFxvdGltZXMgU15hXFxvdGltZXMgTiJdLFsxLDIsIlxcdGF1XFxvdGltZXMgTiJdLFsyLDMsIlNeYVxcb3RpbWVzIFxca2FwcGEiXSxbMCw0LCJFXFxvdGltZXMgXFx0YXVcXG90aW1lcyBOIiwyXSxbNCw1LCJFXFxvdGltZXMgU15hXFxvdGltZXMgXFxrYXBwYSIsMl0sWzUsNiwiXFx0YXVcXG90aW1lcyBOIl0sWzYsMywiU15hXFxvdGltZXMgXFxrYXBwYSJdLFswLDcsIlxcdGF1X3tFXFxvdGltZXMgRSxTXmF9XFxvdGltZXMgTiIsMV0sWzcsMiwiU15hXFxvdGltZXMgXFxtdVxcb3RpbWVzIE4iXSxbNyw2LCJTXmFcXG90aW1lcyBFXFxvdGltZXMgXFxrYXBwYSIsMV0sWzQsNywiXFx0YXVcXG90aW1lcyBFXFxvdGltZXMgTiIsMl1d
	\[\begin{tikzcd}
		{E\otimes E\otimes S^a\otimes N} && {E\otimes S^a\otimes N} \\
		{E\otimes S^a\otimes E\otimes N} & {S^a\otimes E\otimes E\otimes N} & {S^a\otimes E\otimes N} \\
		{E\otimes S^a\otimes N} & {S^a\otimes E\otimes N} & {S^a\otimes N}
		\arrow["{\mu\otimes S^a\otimes N}", from=1-1, to=1-3]
		\arrow["{\tau\otimes N}", from=1-3, to=2-3]
		\arrow["{S^a\otimes \kappa}", from=2-3, to=3-3]
		\arrow["{E\otimes \tau\otimes N}"', from=1-1, to=2-1]
		\arrow["{E\otimes S^a\otimes \kappa}"', from=2-1, to=3-1]
		\arrow["{\tau\otimes N}", from=3-1, to=3-2]
		\arrow["{S^a\otimes \kappa}", from=3-2, to=3-3]
		\arrow["{\tau_{E\otimes E,S^a}\otimes N}"{description}, from=1-1, to=2-2]
		\arrow["{S^a\otimes \mu\otimes N}", from=2-2, to=2-3]
		\arrow["{S^a\otimes E\otimes \kappa}"{description}, from=2-2, to=3-2]
		\arrow["{\tau\otimes E\otimes N}"', from=2-1, to=2-2]
	\end{tikzcd}\]
	The top left triangle commutes by coherence for a symmetric monoidal category. The bottom left rectangle and top right trapezoid commute by naturality of $\tau$. Finally, the bottom right square commutes by the coherence condition for $\kappa$.

	Thus, we have shown that $\Sigma^aN$ is indeed an object in $E\text-\Mod$, as desired. Now let $f:(N,\kappa)\to(N',\kappa')$ be a morphism in $E\text-\Mod$, we would like to show $\Sigma^af:\Sigma^aN\to\Sigma^aN'$ is also a homomorphism of $E$-modules. To that end, consider the following diagram:
	% https://q.uiver.app/#q=WzAsNixbMCwwLCJFXFxvdGltZXMgU15hXFxvdGltZXMgTiJdLFsxLDAsIkVcXG90aW1lcyBTXmFcXG90aW1lcyBOJyJdLFswLDEsIlNeYVxcb3RpbWVzIEVcXG90aW1lcyBOIl0sWzEsMSwiU15hXFxvdGltZXMgRVxcb3RpbWVzIE4nIl0sWzAsMiwiU15hXFxvdGltZXMgTiJdLFsxLDIsIlNeYVxcb3RpbWVzIE4nIl0sWzAsMSwiRVxcb3RpbWVzIFNeYVxcb3RpbWVzIGYiXSxbMCwyLCJcXHRhdVxcb3RpbWVzIE4iLDJdLFsyLDMsIlNeYVxcb3RpbWVzIEVcXG90aW1lcyBmIl0sWzEsMywiXFx0YXVcXG90aW1lcyBOJyJdLFsyLDQsIlNeYVxcb3RpbWVzIFxca2FwcGEiLDJdLFszLDUsIlNeYVxcb3RpbWVzIFxca2FwcGEnIl0sWzQsNSwiU15hXFxvdGltZXMgZiIsMl1d
	\[\begin{tikzcd}
		{E\otimes S^a\otimes N} & {E\otimes S^a\otimes N'} \\
		{S^a\otimes E\otimes N} & {S^a\otimes E\otimes N'} \\
		{S^a\otimes N} & {S^a\otimes N'}
		\arrow["{E\otimes S^a\otimes f}", from=1-1, to=1-2]
		\arrow["{\tau\otimes N}"', from=1-1, to=2-1]
		\arrow["{S^a\otimes E\otimes f}", from=2-1, to=2-2]
		\arrow["{\tau\otimes N'}", from=1-2, to=2-2]
		\arrow["{S^a\otimes \kappa}"', from=2-1, to=3-1]
		\arrow["{S^a\otimes \kappa'}", from=2-2, to=3-2]
		\arrow["{S^a\otimes f}"', from=3-1, to=3-2]
	\end{tikzcd}\]
	The top rectangle commutes by functoriality of $-\otimes-$, while the bottom commutes since $f$ is an $E$-module homomorphism. Thus, $S^a\otimes f=\Sigma^af$ is an $E$-module homomorphism, as desired.
\end{proof}

\begin{definition}
    We can extend the hom-groups in $E\text-\Mod$ (which is additive by \autoref{E-Mod,free,forgetful_are_additive}) to $A$-graded abelian groups $\Hom_{E\text-\Mod}^*(N,N')$ defined by
    \[\Hom_{E\text-\Mod}^a(N,N'):=\Hom_E(\Sigma^aN,N'),\]
    where $\Sigma^aN$ is considered as an $E$-module object by the above lemma.
\end{definition}

\begin{lemma}\label{free_susp_is_susp_of_free}
	Given a monoid object $(E,\mu,e)$ in $\cSH$, an object $X$ in $\cSH$, and some $a\in A$, the suspension of the free module $\Sigma^a(E\otimes X)$ is naturally isomorphic as an $E$-module object to the free $E$-module $E\otimes\Sigma^aX$.
\end{lemma}
\begin{proof}
	It suffices to show the map $S^a\otimes E\otimes X\xr{\tau\otimes X}E\otimes S^a\otimes X$ is a homomorphism of $E$-module objects, as we know it is an isomorphism and natural in $X$. To that end, consider the following diagram:
	% https://q.uiver.app/#q=WzAsNSxbMCwwLCJFXFxvdGltZXMgU15hXFxvdGltZXMgRVxcb3RpbWVzIFgiXSxbMCwxLCJTXmFcXG90aW1lcyBFXFxvdGltZXMgRVxcb3RpbWVzIFgiXSxbMCwyLCJTXmFcXG90aW1lcyBFXFxvdGltZXMgWCJdLFsyLDIsIkVcXG90aW1lcyBTXmFcXG90aW1lcyBYIl0sWzIsMCwiRVxcb3RpbWVzIEVcXG90aW1lcyBTXmFcXG90aW1lcyBYIl0sWzAsMSwiXFx0YXVcXG90aW1lcyBFXFxvdGltZXMgWCIsMl0sWzEsMiwiU15hXFxvdGltZXMgXFxtdVxcb3RpbWVzIFgiLDJdLFsyLDMsIlxcdGF1XFxvdGltZXMgWCJdLFswLDQsIkVcXG90aW1lcyBcXHRhdVxcb3RpbWVzIFgiXSxbNCwzLCJcXG11XFxvdGltZXMgU15hXFxvdGltZXMgWCJdLFsxLDQsIlxcdGF1X3tTXmEsRVxcb3RpbWVzIEV9XFxvdGltZXMgWCIsMl1d
	\[\begin{tikzcd}
		{E\otimes S^a\otimes E\otimes X} && {E\otimes E\otimes S^a\otimes X} \\
		{S^a\otimes E\otimes E\otimes X} \\
		{S^a\otimes E\otimes X} && {E\otimes S^a\otimes X}
		\arrow["{\tau\otimes E\otimes X}"', from=1-1, to=2-1]
		\arrow["{S^a\otimes \mu\otimes X}"', from=2-1, to=3-1]
		\arrow["{\tau\otimes X}", from=3-1, to=3-3]
		\arrow["{E\otimes \tau\otimes X}", from=1-1, to=1-3]
		\arrow["{\mu\otimes S^a\otimes X}", from=1-3, to=3-3]
		\arrow["{\tau_{S^a,E\otimes E}\otimes X}"', from=2-1, to=1-3]
	\end{tikzcd}\]
	The top triangle commutes by coherence for a symmetric monoidal category. The bottom trapezoid commutes by naturality of $\tau$. 
\end{proof}

\begin{lemma}\label{susp_of_coprod_is_iso_in_E-Mod_to_coprod_of_sus}
	Let $(E,\mu,e)$ be a monoid object in $\cSH$, and suppose we have a collection of objects $(N_i,\kappa_i)$ in $E\text-\Mod$. Then for all $a\in A$, since $\Sigma^a$ has a right adjoint $\Sigma^{-a}$ (\autoref{Sigma^a,Sigma^-a_adjoint_equiv}), it preserves coproducts in $\cSH$, (which are coproducts in $E\text-\Mod$ by \autoref{coproduct_of_E_modules_is_coproduct_in_E_mod}), so we have an isomorphism
	\[\Sigma^a\bigoplus_iN_i\cong\bigoplus_i\Sigma^aN_i.\]
	Then this isomorphism is an $E$-module homomorphism.
\end{lemma}
\begin{proof}
	\todo{TODO}
\end{proof}

%\begin{lemma}\label{q^a_N_isos}
%    Let $(E,\mu,e)$ be a monoid object in $\cSH$, and $(N,\kappa)$ an $E$-module object. Then given $a\in A$, we have an $A$-graded isomorphism of left $\pi_*(E)$-modules
%    \[q^a_N:\pi_{*-a}(N)\to\pi_*(\Sigma^aN)\]
%    taking a class $x:S^{b-a}\to N$ in $\pi_{*-a}(N)$ to the composition
%    \[S^b\xr{\phi_{b-a,a}}S^{b-a}\otimes S^a\xr{x\otimes S^a}N\otimes S^a\xr{\tau}S^a\otimes N.\]
%    (Where here $\Sigma^aN$ is considered an $E$-module object by \autoref{suspension_of_module_object_is_module_object}, so that $\pi_*(\Sigma^aN)$ and $\pi_*(N)$ may be considered with their canonical left $\pi_*(E)$-module structures from \autoref{E-module_N_implies_pi*N_is_pi*E_module}.)
%\end{lemma}
%\begin{proof}
%    First, note that by unravelling definitions, the assignment factors as the composition 
%    \[\pi_{*-a}(N)=[S^{*-a},N]\xr{-\otimes S^a}[S^{*-a}\otimes S^a,N\otimes S^a]\xr{{(\phi_{*-a,a})}^*}[S^*,N\otimes S^a]\xr{\tau_*}[S^*,S^a\otimes N]=\pi_*(\Sigma^aN).\]
%    The first arrow is an isomorphism as $-\otimes S^{a}$ is an additive equivalence of $\cSH$. The remaining two arrows are likewise $A$-graded isomorphisms of abelian groups, as they are given by composing with an isomorphism is an additive category. Thus, we've shown $q^a_N$ is an $A$-graded isomorphism. It remains to show that it is a homomorphism of left $\pi_*(E)$-modules. By additivity, it suffices to show that for homogeneous $r:S^b\to E$ and $x:S^{c-a}\to N$ that $q^a_N(r\cdot x)=r\cdot q^a_N(x)$. To that end, consider the following diagram:
%    % https://q.uiver.app/#q=WzAsNyxbMCwwLCJTXnthK2J9Il0sWzEsMCwiU157Yn1cXG90aW1lcyBTXntjLWF9XFxvdGltZXMgU15hIl0sWzIsMCwiRVxcb3RpbWVzIE5cXG90aW1lcyBTXmEiXSxbMiwyLCJOXFxvdGltZXMgU15hIl0sWzMsMiwiU15hXFxvdGltZXMgTiJdLFszLDAsIkVcXG90aW1lcyBTXmFcXG90aW1lcyBOIl0sWzMsMSwiU15hXFxvdGltZXMgRVxcb3RpbWVzIE4iXSxbMCwxLCJcXHBoaSJdLFsxLDIsInJcXG90aW1lcyB4XFxvdGltZXMgU15hIl0sWzIsMywiXFxrYXBwYVxcb3RpbWVzIFNeYSIsMl0sWzMsNCwiXFx0YXUiXSxbMiw1LCJFXFxvdGltZXMgXFx0YXUiXSxbNSw2LCJcXHRhdVxcb3RpbWVzIE4iXSxbNiw0LCJTXmFcXG90aW1lcyBcXGthcHBhIl0sWzIsNiwiXFx0YXVfe0VcXG90aW1lcyBOLFNeYX0iLDFdXQ==
%    \[\begin{tikzcd}
%        {S^{a+b}} & {S^{b}\otimes S^{c-a}\otimes S^a} & {E\otimes N\otimes S^a} & {E\otimes S^a\otimes N} \\
%        &&& {S^a\otimes E\otimes N} \\
%        && {N\otimes S^a} & {S^a\otimes N}
%        \arrow["\phi", from=1-1, to=1-2]
%        \arrow["{r\otimes x\otimes S^a}", from=1-2, to=1-3]
%        \arrow["{\kappa\otimes S^a}"', from=1-3, to=3-3]
%        \arrow["\tau", from=3-3, to=3-4]
%        \arrow["{E\otimes \tau}", from=1-3, to=1-4]
%        \arrow["{\tau\otimes N}", from=1-4, to=2-4]
%        \arrow["{S^a\otimes \kappa}", from=2-4, to=3-4]
%        \arrow["{\tau_{E\otimes N,S^a}}"{description}, from=1-3, to=2-4]
%    \end{tikzcd}\]
%    The top composition is $r\cdot q^a_N(x)$, while the bottom is $q^a_N(r\cdot x)$. The top triangle commutes by coherence for the symmetries in a symmetric monoidal category, while the trapezoid commutes by naturality of $\tau$. Thus indeed $r\cdot q^a_N(x)=q^a_N(r\cdot x)$, so that $q^a_N$ is an $A$-graded isomorphism of left $\pi_*(E)$-modules, as desired.
%\end{proof}

\subsection{A universal coefficient theorem}

Finally, we have the ingredients required to state and prove the following universal coefficient theorem:

\begin{theorem}
    Let $(E,\mu,e)$ be a monoid object and let $X$ and $Y$ be objects in $\cSH$. Then if $E$ and $X$ are cellular and $E_*(X)$ is a graded projective (\autoref{graded_projective_module}) left $\pi_*(E)$-module (via \autoref{module}), then the map
    \[[X,E\otimes Y]\to\Hom_{\pi_*(E)}(E_*(X),E_*(Y)),\qquad[X\xr fE\otimes Y]\mapsto [\pi_*(\mu\otimes Y)\circ \pi_*(E\otimes f)]\]
    is an isomorphism, and extends to an $A$-graded isomorphism
    \[{[X,E\otimes Y]}_*\to\Hom^*_{\pi_*(E)}(E_*(X),E_*(Y)).\]
\end{theorem}
\begin{proof}
    Since: (1) $E\otimes X$ is a free $E$-module object (\autoref{free_forgetful_E-Mod}), (2) $E_*(X)=\pi_*(E\otimes X)$ is a graded projective left $\pi_*(E)$-module, and (3) $E$ and $E\otimes X$ are cellular (by \autoref{cellular_closed_under_tensor}), by \autoref{if_pi_*N_graded_proj_then_retract_of_wedge_of_susps} below it follows that $E\otimes X$ is a retract of $\bigoplus_i(E\otimes S^{a_i})$ in $E\text-\Mod$ for some collection of $a_i\in A$ indexed by some set $I$. Thus the desired result follows by \autoref{UCT_for_retract} below with $N=E\otimes Y$ (which is considered as a free $E$-module by \autoref{free_forgetful_E-Mod}).
\end{proof}

In the case that $Y=S$, this theorem becomes the more familiar statement:
\[E^*(X)\cong[X,E]_{-*}\cong[X,E\otimes S]_{-*}\cong\Hom^{-*}_{\pi_*(E)}(E_*(X),\pi_*(E)),\]
i.e., the $E$-cohomology of $X$ is isomorphic to the dual of the $E$-homology of $X$ when $E_*(X)$ is a graded projective module. Hence why we call it the universal coefficient theorem.

\begin{proposition}\label{pi_*_iso_when_N_retract_of_wedge_of_susps}
	Let $(E,\mu,e)$ be a monoid object and $(N,\kappa)$ an $E$-module object in $\cSH$. Then given a collection of $a_i\in A$ indexed by some set $I$, if $(N,\kappa)$ is a retract of $\bigoplus_i(E\otimes S^{a_i})$ in $E\text-\Mod$,\footnote{Here $\bigoplus_i(E\otimes S^{a_i})$ is a coproduct (\autoref{coproduct_of_E_modules_is_coproduct_in_E_mod}) of a bunch of free $E$-module objects (\autoref{free_forgetful_E-Mod}), so it is itself an $E$-module object.} then for all $E$-module objects $(N',\kappa')$, the functor $\pi_*:E\text-\Mod\to\pi_*(E)\text-\Mod(A)$ (\autoref{E-module_N_implies_pi*N_is_pi*E_module}) induces an isomorphism of abelian groups
	\[\pi_*:\Hom_{E\text-\Mod}(N,N')\to\Hom_{\pi_*(E)}(\pi_*(N),\pi_*(N')).\]
    %sending an $E$-module homomorphism $f:\Sigma^aN\to N'$ to the assignment $\pi_{*-a}(N)\to\pi_*(N')$ which sends a class $x:S^{b-a}\to N$ to the composition
    %\[S^b\xr{\phi_{b-a,a}}S^{b-a}\otimes S^a\xr{x\otimes S^a}N\otimes S^a\xr{\tau}S^a\otimes N\xr{f}N'.\]
\end{proposition}
\begin{proof}
	%First, we show that if $N$ is a retract of $\bigoplus_i(E\otimes S^{a_i})$ in $E\text-\Mod$, then
	%\[\pi_*:\Hom_E(N,N')\to\Hom_{\pi_*(E)}(\pi_*(N),\pi_*(N'))\]
	%is an isomorphism. 
    To start, we consider the case $N=\bigoplus_i(E\otimes S^{a_i})$. Consider the following diagram:
    % https://q.uiver.app/#q=WzAsOSxbMCwwLCJcXEhvbV97RX0oXFxiaWdvcGx1c19pIChFXFxvdGltZXMgU157YV9pfSksTicpIl0sWzEsMCwiXFxIb21fe1xccGlfKihFKX0oXFxwaV8qKFxcYmlnb3BsdXNfaSAoRVxcb3RpbWVzIFNee2FfaX0pKSxcXHBpXyooTicpKSJdLFswLDEsIlxccHJvZF9pXFxIb21fe0V9KEVcXG90aW1lcyBTXnthX2l9LE4nKSJdLFswLDIsIlxccHJvZF9pW1Nee2FfaX0sTiddIl0sWzEsMSwiXFxIb21fe1xccGlfKihFKX0oXFxiaWdvcGx1c19pXFxwaV8qKEVcXG90aW1lcyBTXnthX2l9KSxcXHBpXyooTicpKSJdLFsxLDIsIlxccHJvZF9pXFxIb21fe1xccGlfKihFKX0oXFxwaV8qKEVcXG90aW1lcyBTXnthX2l9KSxcXHBpXyooTicpKSJdLFsxLDMsIlxccHJvZF9pXFxIb21fe1xccGlfKihFKX0oXFxwaV97Ki1hX2l9KEUpLFxccGlfKihOJykpIl0sWzEsNCwiXFxwcm9kX2lcXEhvbV57YV9pfV97XFxwaV8qKEUpfShcXHBpX3sqfShFKSxcXHBpXyooTicpKSJdLFswLDQsIlxccHJvZF9pXFxwaV97YV9pfShOJykiXSxbMCwxLCJcXHBpXyoiXSxbMCwyLCJcXGNvbmciLDJdLFsyLDMsIlxcY29uZyIsMl0sWzEsNCwiXFxjb25nIl0sWzQsNSwiXFxjb25nIl0sWzUsNiwiXFxjb25nIl0sWzYsNywiIiwwLHsibGV2ZWwiOjIsInN0eWxlIjp7ImhlYWQiOnsibmFtZSI6Im5vbmUifX19XSxbNyw4LCJcXHByb2RfaVxcbWF0aHJte2V2fV8xIiwyXSxbMyw4LCIiLDIseyJsZXZlbCI6Miwic3R5bGUiOnsiaGVhZCI6eyJuYW1lIjoibm9uZSJ9fX1dXQ==
    \[\begin{tikzcd}
        {\Hom_{E}(\bigoplus_i (E\otimes S^{a_i}),N')} & {\Hom_{\pi_*(E)}(\pi_*(\bigoplus_i (E\otimes S^{a_i})),\pi_*(N'))} \\
        {\prod_i\Hom_{E}(E\otimes S^{a_i},N')} & {\Hom_{\pi_*(E)}(\bigoplus_i\pi_*(E\otimes S^{a_i}),\pi_*(N'))} \\
        {\prod_i[S^{a_i},N']} & {\prod_i\Hom_{\pi_*(E)}(\pi_*(E\otimes S^{a_i}),\pi_*(N'))} \\
        & {\prod_i\Hom_{\pi_*(E)}(\pi_{*-a_i}(E),\pi_*(N'))} \\
        {\prod_i\pi_{a_i}(N')} & {\prod_i\Hom^{a_i}_{\pi_*(E)}(\pi_{*}(E),\pi_*(N'))}
        \arrow["{\pi_*}", from=1-1, to=1-2]
        \arrow["\cong"', from=1-1, to=2-1]
        \arrow["\cong"', from=2-1, to=3-1]
        \arrow["\cong", from=1-2, to=2-2]
        \arrow["\cong", from=2-2, to=3-2]
        \arrow["\cong", from=3-2, to=4-2]
        \arrow[Rightarrow, no head, from=4-2, to=5-2]
        \arrow["{\prod_i\mathrm{ev}_1}"', from=5-2, to=5-1]
        \arrow[Rightarrow, no head, from=3-1, to=5-1]
    \end{tikzcd}\]
	Here the top left vertical isomorphism exibits the universal property of the coproduct in $E\text-\Mod$, and middle left vertical isomorphism below that is the free-forgetful adjunction for $E$-modules (\autoref{free_forgetful_E-Mod}). The bottom horizontal isomorphism is the product of the evaluation-at-$1$ isomorphisms (\autoref{ev_at_1_is_iso}). On the other side, the top right vertical isomorphism is given by the fact that $S^a$ is compact for each $a\in A$, so we have isomorphisms
	\[\bigoplus_i\pi_*(E\otimes S^{a_i})=\bigoplus_{a\in A}\bigoplus_{i}[S^a,E\otimes S^{a_i}]\cong\bigoplus_{a\in A}[S^a,\bigoplus_{i}(E\otimes S^{a_i})]=\pi_*(\bigoplus_i(E\otimes S^{a_i})),\]
	where the middle isomorphism takes a generator $x:S^a\xr E\otimes S^{a_i}$ to the composition $S^a\xr xE\otimes S^{a_i}\into\bigoplus_i(E\otimes S^{a_i})$. The middle right vertical isomorphism exhibits the universal property of the coproduct of modules. Finally the bottom right vertical isomorphism is given by the isomorphisms
	\[\pi_{*-a_i}(E\otimes S^{a_i})=[S^{*-a_i},E\otimes S^{a_i}]\xr{-\otimes S^{a_i}}[S^{*-a_i}\otimes S^{a_i},E\otimes S^{a_i}]\xr{\phi^*}[S^*,E\otimes S^{a_i}]=\pi_*(E\otimes S^{a_i}),\]
	where $-\otimes S^{a_i}\cong\Sigma^{a_i}$ is an isomorphism by \autoref{Sigma^a,Sigma^-a_adjoint_equiv}. Now, we claim this diagram commutes. This really simply amounts to unravelling definitions, and chasing a homomorphism $f:\bigoplus_i(E\otimes S^{a_i})\to N'$ of $E$-module objects both ways around the diagram yields the composition
	\[\prod_i(S^{a_i}\xr{e\otimes S^{a_i}}E\otimes S^{a_i}\into\bigoplus_i(E\otimes S^{a_i})\xr fN').\]
	Thus, since the diagram commutes, we have that 
	\[\pi_*:\Hom_E(\bigoplus_i(E\otimes S^{a_i}),N')\to\Hom_{\pi_*(E)}(\pi_*(\bigoplus_i(E\otimes S^{a_i})),\pi_*(N'))\] is an isomorphism, as desired.

	Now, consider the case that $N$ is a retract of $\bigoplus_i(E\otimes S^{a_i})$ in $E\text-\Mod$, so there exists a commuting diagram of $E$-module object homomorphisms:
	% https://q.uiver.app/#q=WzAsMyxbMCwwLCJOIl0sWzEsMCwiXFxiaWdvcGx1c19pKEVcXG90aW1lcyBTXnthX2l9KSJdLFsyLDAsIk4iXSxbMCwxLCJcXGlvdGEiLDJdLFsxLDIsInIiLDJdLFswLDIsIiIsMix7ImN1cnZlIjotNCwibGV2ZWwiOjIsInN0eWxlIjp7ImhlYWQiOnsibmFtZSI6Im5vbmUifX19XV0=
	\[\begin{tikzcd}
		N & {\bigoplus_i(E\otimes S^{a_i})} & N
		\arrow["\iota"', from=1-1, to=1-2]
		\arrow["r"', from=1-2, to=1-3]
		\arrow[curve={height=-24pt}, Rightarrow, no head, from=1-1, to=1-3]
	\end{tikzcd}\]
	Now consider the following diagram:
    % https://q.uiver.app/#q=WzAsNixbMiwwLCJcXEhvbV97RX0oTixOJykiXSxbMSwwLCJcXEhvbV97RX0oXFxiaWdvcGx1c19pKEVcXG90aW1lcyBTXnthX2l9KSxOJykiXSxbMCwwLCJcXEhvbV97RX0oTixOJykiXSxbMSwxLCJcXEhvbV97XFxwaV8qKEUpfShcXHBpXyooXFxiaWdvcGx1c19pKEVcXG90aW1lcyBTXnthX2l9KSksXFxwaV8qKE4nKSkiXSxbMCwxLCJcXEhvbV97XFxwaV8qKEUpfShcXHBpXyooTiksXFxwaV8qKE4nKSkiXSxbMiwxLCJcXEhvbV97XFxwaV8qKEUpfShcXHBpXyooTiksXFxwaV8qKE4nKSkiXSxbMiwxLCJyXioiXSxbMSwwLCJcXGlvdGFeKiJdLFsyLDAsIiIsMix7ImN1cnZlIjotNCwibGV2ZWwiOjIsInN0eWxlIjp7ImhlYWQiOnsibmFtZSI6Im5vbmUifX19XSxbMSwzLCJcXHBpXyoiXSxbMiw0LCJcXHBpXyoiLDJdLFs0LDMsIihcXHBpXyoocikpXioiXSxbMyw1LCIoXFxwaV8qKFxcaW90YSkpXioiXSxbMCw1LCJcXHBpXyoiLDJdLFs0LDUsIiIsMix7ImN1cnZlIjo0LCJsZXZlbCI6Miwic3R5bGUiOnsiaGVhZCI6eyJuYW1lIjoibm9uZSJ9fX1dXQ==
    \[\begin{tikzcd}
        {\Hom_{E}(N,N')} & {\Hom_{E}(\bigoplus_i(E\otimes S^{a_i}),N')} & {\Hom_{E}(N,N')} \\
        {\Hom_{\pi_*(E)}(\pi_*(N),\pi_*(N'))} & {\Hom_{\pi_*(E)}(\pi_*(\bigoplus_i(E\otimes S^{a_i})),\pi_*(N'))} & {\Hom_{\pi_*(E)}(\pi_*(N),\pi_*(N'))}
        \arrow["{r^*}", from=1-1, to=1-2]
        \arrow["{\iota^*}", from=1-2, to=1-3]
        \arrow[curve={height=-24pt}, Rightarrow, no head, from=1-1, to=1-3]
        \arrow["{\pi_*}", from=1-2, to=2-2]
        \arrow["{\pi_*}"', from=1-1, to=2-1]
        \arrow["{(\pi_*(r))^*}", from=2-1, to=2-2]
        \arrow["{(\pi_*(\iota))^*}", from=2-2, to=2-3]
        \arrow["{\pi_*}"', from=1-3, to=2-3]
        \arrow[curve={height=24pt}, Rightarrow, no head, from=2-1, to=2-3]
    \end{tikzcd}\]
	Each square commutes by functoriality of $\pi_*$. We have shown the middle vertical arrow is an isomorphism. Thus the outside arrows are isomorphisms as well, as a retract of an isomorphism is an isomorphism.
%
%	Now, we further claim that 
%	\[\pi_*:\Hom_E(\Sigma^aN,N')\to\Hom_{\pi_*(E)}(\pi_*(\Sigma^aN),\pi_*(N'))\]
%	is an isomorphism when $N$ is a retract of $\bigoplus_i(E\otimes S^{a_i})$. By the above results, it suffices to show that $\Sigma^aN$ is a retract in $E\text-\Mod$ of $\bigoplus_i(E\otimes S^{a+a_i})$. It further suffices to show that $\bigoplus_i(E\otimes S^{a+a_i})$ is isomorphic in $E\text-\Mod$ to $\Sigma^a\bigoplus_i(E\otimes S^{a_i})$. To that end, note we have isomorphisms in $\cSH$:
%	\[\Sigma^a\bigoplus_i(E\otimes S^{a_i})\cong\bigoplus_i(\Sigma^a\otimes E\otimes S^{a_i})\xr{\cong}\bigoplus_i(E\otimes \Sigma^aS^{a_i})\xr{\bigoplus_i(E\otimes\phi_{a,a_i}^{-1})}\bigoplus_i(E\otimes S^{a+a_i}),\]
%	where the first isomorphism is that given in \autoref{susp_of_coprod_is_iso_in_E-Mod_to_coprod_of_sus}, the second isomorphism is \autoref{free_susp_is_susp_of_free}, and the last arrow is an $E$-module homomorphism, as it is a coproduct of homomorphisms of free $E$-modules (\autoref{free_forgetful_E-Mod}). Thus, we've shown that $\Sigma^aN$ is a retract of $\bigoplus_i(E\otimes S^{a+a_i})\cong\Sigma^a\bigoplus_i(E\otimes S^{a_i})$ in $E\text-\Mod$, so that by the above argument for all $a\in A$ we have isomorphisms
%	\[\pi_*:\Hom_E(\Sigma^aN,N')\to\Hom_{\pi_*(E)}(\pi_*(\Sigma^aN),\pi_*(N')).\]
%    Then finally, we get the desired $A$-graded isomorphisms
%    \[\pi_*:\Hom_E^*(N,N')\to\Hom_{\pi_*(E)}^*(\pi_*(N),\pi_*(N'))\]
%    via the compositions
%    \[\Hom_E^a(N,N')\xrightarrow{\pi_*}\Hom_{\pi_*(E)}(\pi_*(\Sigma^aN),\pi_*(N'))\xrightarrow{{(q^a_N)}^*}\Hom_{\pi_*(E)}^a(\pi_*(N),\pi_*(N')),\]
%    where here $q^a_N:\pi_{*-a}(N)\xr\cong\pi_*(\Sigma^aN)$ is the $A$-graded isomorphism of left $\pi_*(E)$-modules defined in \autoref{q^a_N_isos}.
\end{proof}

\begin{proposition}\label{UCT_for_retract}
	Let $(E,\mu,e)$ be a monoid object and $X$ an object in $\cSH$. If there is a collection of $a_i\in A$ indexed by some set $I$ such that $E\otimes X$ is a retract of $\bigoplus_i(E\otimes S^{a_i})$ in $E\text-\Mod$,\footnote{Here $\bigoplus_i(E\otimes S^{a_i})$ is a coproduct (\autoref{coproduct_of_E_modules_is_coproduct_in_E_mod}) of a bunch of free $E$-module objects (\autoref{free_forgetful_E-Mod}), so it is itself a $E$-module object.} then for all $E$-module objects $(N,\kappa)$, the assignment
	\[[X,N]\to\Hom_{\pi_*(E)}(E_*(X),\pi_*(N)),\qquad [X\xr fN]\mapsto[\pi_*(\kappa)\circ \pi_*(E\otimes f)]\]
	is an isomorphism, and further extends to an $A$-graded isomorphism of $A$-graded abelian groups
	\[[X,N]_*\to\Hom_{\pi_*(E)}^*(E_*(X),\pi_*(N)).\]
\end{proposition}
\begin{proof}
	For each $a\in A$, define 
	\[U_a:[X,N]_a\to\Hom_{\pi_*(E)}^a(E_{*}(X),\pi_*(N))\]
	to be the composition
	% https://q.uiver.app/#q=WzAsNixbMSwwLCJbXFxTaWdtYV5hWCxOXSJdLFsxLDEsIlxcSG9tX3tFXFx0ZXh0LVxcTW9kfShFXFxvdGltZXMgXFxTaWdtYV5hWCxOKSJdLFsxLDMsIlxcSG9tX3tcXHBpXyooRSl9KEVfeyotYX0oWCksXFxwaV8qKE4pKSJdLFswLDAsIltYLE5dX2EiXSxbMiwzLCJcXEhvbV97XFxwaV8qKEUpfV5hKEVfKihYKSxcXHBpXyooTikpIl0sWzEsMiwiXFxIb21fe1xccGlfKihFKX0oRV8qKFxcU2lnbWFeYVgpLFxccGlfKihOKSJdLFswLDEsIlxcdGV4dHthZGp9Il0sWzMsMCwiIiwwLHsibGV2ZWwiOjIsInN0eWxlIjp7ImhlYWQiOnsibmFtZSI6Im5vbmUifX19XSxbMiw0LCIiLDAseyJsZXZlbCI6Miwic3R5bGUiOnsiaGVhZCI6eyJuYW1lIjoibm9uZSJ9fX1dLFsxLDUsIlxccGlfKigtKSJdLFs1LDIsInsoeyh0X1heYSl9XnstMX0pfV4qIl1d
	\[\begin{tikzcd}
		{[X,N]_a} & {[\Sigma^aX,N]} \\
		& {\Hom_{E\text-\Mod}(E\otimes \Sigma^aX,N)} \\
		& {\Hom_{\pi_*(E)}(E_*(\Sigma^aX),\pi_*(N)} \\
		& {\Hom_{\pi_*(E)}(E_{*-a}(X),\pi_*(N))} & {\Hom_{\pi_*(E)}^a(E_*(X),\pi_*(N))}
		\arrow["{\text{adj}}", from=1-2, to=2-2]
		\arrow[Rightarrow, no head, from=1-1, to=1-2]
		\arrow[Rightarrow, no head, from=4-2, to=4-3]
		\arrow["{\pi_*(-)}", from=2-2, to=3-2]
		\arrow["{{({(t_X^a)}^{-1})}^*}", from=3-2, to=4-2]
	\end{tikzcd}\]
	where the first isomorphism is the free-forgetful adjunction for $E$-modules (\autoref{free_forgetful_E-Mod}), the second map is that induced by the functor $\pi_*$ constructed in \autoref{E-module_N_implies_pi*N_is_pi*E_module}, and the third map is induced by the $A$-graded isomorphism of left $\pi_*(E)$-modules $(t_X^a)^{-1}:E_{*-a}(X)\to E_*(\Sigma^aX)$ from \autoref{E_homology_suspension_iso_t^a's_appendix}. By unravelling definitions, is straightforward to see that under the identification $[X,N]\cong [X,N]_0$, the map $U_0:[X,N]_0\to\Hom^0_{\pi_*(E)}(E_*(X),\pi_*(N))$ coincides with the assignment
	\[[X,N]\to\Hom_{\pi_*(E)}(E_*(X),\pi_*(N))\qquad[X\xr fN]\mapsto[\pi_*(\kappa)\circ\pi_*(E\otimes f)].\]
	Furthermore, note we have isomorphisms in $E\text-\Mod$
	\[E\otimes\Sigma^aX=E\otimes S^a\otimes X\cong S^a\otimes E\otimes X\]
	(by \autoref{free_susp_is_susp_of_free}) and
	\[S^a\otimes\bigoplus_i(E\otimes S^{a_i})\cong\bigoplus_i(S^a\otimes E\otimes S^{a_i})\cong \bigoplus_i (E\otimes S^{a}\otimes S^{a_i})\cong \bigoplus_i (E\otimes S^{a+a_i}),\]
	where the first isomorphism is in $E\text-\Mod$ by \autoref{susp_of_coprod_is_iso_in_E-Mod_to_coprod_of_sus}, the second is in $E\text-\Mod$ by \autoref{free_susp_is_susp_of_free}, and the last is a coproduct of homomorphisms of free $E$-modules (\autoref{free_forgetful_E-Mod}), so it is also an $E$-module homomorphism.
	Hence we have that $E\otimes\Sigma^aX\cong S^a\otimes E\otimes X$ is a retract of $\bigoplus_i(E\otimes S^{a+a_i})\cong S^a\otimes\bigoplus_i(E\otimes S^{a_i})$ in $E\text-\Mod$, as $E\otimes X$ is a retract of $\bigoplus_i(E\otimes S^{a_i})$ in $E\text-\Mod$, so that by \autoref{pi_*_iso_when_N_retract_of_wedge_of_susps}, the map
	\[\pi_*:\Hom_{E\text-\Mod}(E\otimes\Sigma^aX,N)\to\Hom_{\pi_*(E)}(E_*(\Sigma^aX),\pi_*(N))\]
	is an isomorphism. Thus, we have constructed a bunch of isomorphisms 
	\[U_a:[X,N]_a\to\Hom^a_{\pi_*(E)}(E_*(X),\pi_*(N)),\]
	so that by the universal property of the coproduct of abelian groups, there is a unique $A$-graded isomorphism
	\[[X,N]_*\to\Hom^*_{\pi_*(E)}(E_*(X),\pi_*(N))\]
	extending these maps, as desired.
\end{proof}

\begin{proposition}\label{if_pi_*N_graded_proj_then_retract_of_wedge_of_susps}
	Let $(E,\mu,e)$ be a monoid object and $(N,\kappa)$ an $E$-module object in $\cSH$. Further suppose that $E$ and $N$ are cellular and that $\pi_*(N)$ is a \emph{graded projective} (\autoref{graded_projective_module}) left $\pi_*(E)$-module (via \autoref{E-module_N_implies_pi*N_is_pi*E_module}). Then given some homogeneous generating set $\{x_i\}_{i\in I}\sseq \pi_*(N)$, $N$ is a retract of $\bigoplus_i(E\otimes S^{|x_i|})$ in $E\text-\Mod$.\footnote{Here $\bigoplus_i(E\otimes S^{a_i})$ is a coproduct (\autoref{coproduct_of_E_modules_is_coproduct_in_E_mod}) of a bunch of free $E$-module objects (\autoref{free_forgetful_E-Mod}), so it is itself an $E$-module object.}
\end{proposition}
\begin{proof}
	Let $M:=\bigoplus_i(E\otimes S^{|x_i|})$. We have a map
	\[r:M\to N\]
	induced by the maps
	\[r_i:E\otimes S^{|x_i|}\xrightarrow{E\otimes x_i}E\otimes N\xr{\kappa} N.\]
	This is a homomorphism of $E$-module objects:
	% https://q.uiver.app/#q=WzAsOCxbMCwwLCJFXFxvdGltZXNcXGJpZ29wbHVzX2koRVxcb3RpbWVzIFNee3x4X2l8fSkiXSxbMiwwLCJFXFxvdGltZXMgTiJdLFsyLDQsIk4iXSxbMCwyLCJcXGJpZ29wbHVzX2koRVxcb3RpbWVzIEVcXG90aW1lcyBTXnt8eF9pfH0pIl0sWzAsNCwiXFxiaWdvcGx1c19pKEVcXG90aW1lcyBTXnt8eF9pfH0pIl0sWzEsMiwiXFxiaWdvcGx1c19pKEVcXG90aW1lcyBOKSJdLFsxLDMsIlxcYmlnb3BsdXNfaSBOIl0sWzEsMSwiRVxcb3RpbWVzXFxiaWdvcGx1c19pTiJdLFswLDEsIkVcXG90aW1lcyByIl0sWzEsMiwiXFxrYXBwYSJdLFswLDMsIlxcY29uZyIsMl0sWzMsNCwiXFxiaWdvcGx1c19pKFxcbXVcXG90aW1lcyBTXnt8eF9pfH0pIiwyXSxbNCwyLCJyIl0sWzMsNSwiXFxiaWdvcGx1c19pKEVcXG90aW1lcyByX2kpIl0sWzUsNiwiXFxiaWdvcGx1c19pXFxrYXBwYSJdLFs2LDIsIlxcbmFibGEiXSxbNCw2LCJcXGJpZ29wbHVzX2lyX2kiXSxbMCw3LCJFXFxvdGltZXNcXGJpZ29wbHVzX2lyX2kiLDFdLFs3LDUsIlxcY29uZyIsMl0sWzcsMSwiRVxcb3RpbWVzXFxuYWJsYSIsMV0sWzUsMSwiXFxuYWJsYSIsMl1d
	\[\begin{tikzcd}
		{E\otimes\bigoplus_i(E\otimes S^{|x_i|})} && {E\otimes N} \\
		& {E\otimes\bigoplus_iN} \\
		{\bigoplus_i(E\otimes E\otimes S^{|x_i|})} & {\bigoplus_i(E\otimes N)} \\
		& {\bigoplus_i N} \\
		{\bigoplus_i(E\otimes S^{|x_i|})} && N
		\arrow["{E\otimes r}", from=1-1, to=1-3]
		\arrow["\kappa", from=1-3, to=5-3]
		\arrow["\cong"', from=1-1, to=3-1]
		\arrow["{\bigoplus_i(\mu\otimes S^{|x_i|})}"', from=3-1, to=5-1]
		\arrow["r", from=5-1, to=5-3]
		\arrow["{\bigoplus_i(E\otimes r_i)}", from=3-1, to=3-2]
		\arrow["{\bigoplus_i\kappa}", from=3-2, to=4-2]
		\arrow["\nabla", from=4-2, to=5-3]
		\arrow["{\bigoplus_ir_i}", from=5-1, to=4-2]
		\arrow["{E\otimes\bigoplus_ir_i}"{description}, from=1-1, to=2-2]
		\arrow["\cong"', from=2-2, to=3-2]
		\arrow["E\otimes\nabla"{description}, from=2-2, to=1-3]
		\arrow["\nabla"', from=3-2, to=1-3]
	\end{tikzcd}\]
	The right trapezoid commutes by naturality of $\nabla$. The bottom triangle commutes by the fact that $\nabla\circ\bigoplus_ir_i$ and $r$ satisfy the same universal property for the coproduct. Every other region commutes by additivity of $E\otimes-$, except the left trapezoid: Note that by expanding out how $r_i$ is defined, it becomes
	% https://q.uiver.app/#q=WzAsNixbMCwwLCJcXGJpZ29wbHVzX2koRVxcb3RpbWVzIEVcXG90aW1lcyBTXnt8eF9pfH0pIl0sWzQsMCwiXFxiaWdvcGx1c19pKEVcXG90aW1lcyBFXFxvdGltZXMgWCkiXSxbMiwwLCJcXGJpZ29wbHVzX2koRVxcb3RpbWVzIEVcXG90aW1lcyBOKSJdLFs0LDEsIlxcYmlnb3BsdXNfaShFXFxvdGltZXMgWCkiXSxbMCwxLCJcXGJpZ29wbHVzX2koRVxcb3RpbWVzIFNee3x4X2l8fSkiXSxbMiwxLCJcXGJpZ29wbHVzX2koRVxcb3RpbWVzIE4pIl0sWzAsMiwiXFxiaWdvcGx1c19pKEVcXG90aW1lcyBFXFxvdGltZXMgeF9pKSJdLFsyLDEsIlxcYmlnb3BsdXNfaShFXFxvdGltZXNcXGthcHBhKSJdLFsxLDMsIlxcYmlnb3BsdXNfaVxca2FwcGEiXSxbMCw0LCJcXGJpZ29wbHVzX2koXFxtdVxcb3RpbWVzIFNee3x4X2l8fSkiLDJdLFs0LDUsIlxcYmlnb3BsdXNfaShFXFxvdGltZXMgeF9pKSIsMl0sWzUsMywiXFxiaWdvcGx1c19pXFxrYXBwYSIsMl0sWzIsNSwiXFxiaWdvcGx1c19pKFxcbXVcXG90aW1lcyBYKSJdXQ==
	\[\begin{tikzcd}
		{\bigoplus_i(E\otimes E\otimes S^{|x_i|})} && {\bigoplus_i(E\otimes E\otimes N)} && {\bigoplus_i(E\otimes E\otimes X)} \\
		{\bigoplus_i(E\otimes S^{|x_i|})} && {\bigoplus_i(E\otimes N)} && {\bigoplus_i(E\otimes X)}
		\arrow["{\bigoplus_i(E\otimes E\otimes x_i)}", from=1-1, to=1-3]
		\arrow["{\bigoplus_i(E\otimes\kappa)}", from=1-3, to=1-5]
		\arrow["{\bigoplus_i\kappa}", from=1-5, to=2-5]
		\arrow["{\bigoplus_i(\mu\otimes S^{|x_i|})}"', from=1-1, to=2-1]
		\arrow["{\bigoplus_i(E\otimes x_i)}"', from=2-1, to=2-3]
		\arrow["{\bigoplus_i\kappa}"', from=2-3, to=2-5]
		\arrow["{\bigoplus_i(\mu\otimes X)}", from=1-3, to=2-3]
	\end{tikzcd}\]
	The left square commutes by functoriality of $-\otimes-$, and the right square commutes by coherence for $\kappa$. Hence, we've shown that $r$ is a homomorphism of $E$-modules, as desired. Thus, $r$ induces a homomorphism of left $\pi_*(E)$-modules $\pi_*(r)\in\Hom_{\pi_*(E)}(\pi_*(M),\pi_*(N))$. Further note that for all $i\in I$, $x_i$ is in the image of $\pi_*(r)$, as by definition $\pi_*(r)$ sends the class 
	\[S^{|x_i|}\xr{e\otimes S^{|x_i|}}E\otimes S^{|x_i|}\into M\]
	in $\pi_{|x_i|}(M)$ to the composition
	\[S^{|x_i|}\xr{e\otimes S^{|x_i|}}E\otimes S^{|x_i|}\xr{E\otimes x_i}E\otimes N\xr\kappa N,\]
	and by unitality of $\kappa$ this composition is simply $x_i:S^{|x_i|}\to N$. Thus, we have constructed a surjective $A$-graded homomorphism $\pi_*(r):\pi_*(M)\to \pi_*(N)$ of left $\pi_*(E)$-modules, so that since $\pi_*(N)$ is projective graded module there exists an $A$-graded left $\pi_*(E)$-module homomorphism $\iota:\pi_*(N)\to\pi_*(M)$ which makes the following diagram commute:
	% https://q.uiver.app/#q=WzAsMyxbMCwxLCJcXHBpXyooTikiXSxbMSwxLCJcXHBpXyooTikiXSxbMSwwLCJcXHBpXyooTSkiXSxbMCwxLCIiLDAseyJsZXZlbCI6Miwic3R5bGUiOnsiaGVhZCI6eyJuYW1lIjoibm9uZSJ9fX1dLFsyLDEsIlxccGlfKihyKSJdLFswLDIsIlxcaW90YSJdXQ==
	\[\begin{tikzcd}
		& {\pi_*(M)} \\
		{\pi_*(N)} & {\pi_*(N)}
		\arrow[Rightarrow, no head, from=2-1, to=2-2]
		\arrow["{\pi_*(r)}", from=1-2, to=2-2]
		\arrow["\iota", from=2-1, to=1-2]
	\end{tikzcd}\]
	Thus we have an idempotent of left $A$-graded $\pi_*(E)$-modules:
	% https://q.uiver.app/#q=WzAsMyxbMCwwLCJcXHBpXyooTSkiXSxbMSwwLCJcXHBpXyooTikiXSxbMiwwLCJcXHBpXyooTSkiXSxbMCwxLCJcXHBpXyoocikiXSxbMSwyLCJcXGlvdGEiXV0=
	\[\begin{tikzcd}
		{\pi_*(M)} & {\pi_*(N)} & {\pi_*(M)}
		\arrow["{\pi_*(r)}", from=1-1, to=1-2]
		\arrow["\iota", from=1-2, to=1-3]
	\end{tikzcd}\]
	Now, by \autoref{pi_*_iso_when_N_retract_of_wedge_of_susps}, since $M=\bigoplus_i(E\otimes S^{|x_i|})$, we have that the map
	\[\pi_*:\Hom_{E\text-\Mod}(M,M)\to\Hom_{\pi_*(E)\text-\Mod}(\pi_*(M),\pi_*(M))\]
	is an isomorphism of abelian groups, so that the above idempotent is induced by some endomorphism $\ell:M\to M$ of $E$-module objects. Further note that by functoriality of $\pi_*$,
	\[\pi_*(\ell\circ\ell)=\pi_*(\ell)\circ\pi_*(\ell)=\pi_*(\ell),\]
	and again since $\pi_*$ is an isomorphism here, we have that $\ell\circ\ell=\ell$, so that $\ell$ is an idempotent in $\cSH$. By \autoref{idempotent_splits_in_tri_cat_with_countable_coproducts}, every idempotent in $\cSH$ splits, meaning $\ell$ factors in $\cSH$ as
	\[\ell:M\xr{r'}X\xr{\iota'}M\]
	with $r'\circ \iota'=\id_X$. Since $X$ is a retract of an $E$-module object, and the corresponding idempotent is an $E$-module homomorphism, it follows purely formally that $X$ may be canonically viewed as an $E$-module object, and that $r':M\to X$ and $\iota':X\to M$ are homomorphisms of $E$-module objects (see \autoref{retract_of_module_whose_idempotent_is_module_homomorphism_is_module} for details). Note that since $E$ and each $S^{|x_i|}$ are cellular, $E\otimes S^{|x_i|}$ is cellular for all $i\in I$ (by \autoref{cellular_closed_under_tensor}), so that $M=\bigoplus_i(E\otimes S^{|x_i|})$ is cellular, as by definition an arbitrary coproduct of cellular objects is cellular. Thus by \autoref{cellular_idempotent_splits_cellularly}, $X$ is cellular as well. Now consider the following commutative diagram
	% https://q.uiver.app/#q=WzAsOSxbMywxLCJcXHBpXyooTSkiXSxbMiwyLCJcXHBpXyooWCkiXSxbNCwwLCJcXHBpXyooTikiXSxbNCwyLCJcXHBpXyooWCkiXSxbNSwxLCJcXHBpXyooTSkiXSxbNiwxLCJcXHBpXyooWCkiXSxbMCwxLCJcXHBpXyooTikiXSxbMSwxLCJcXHBpXyooTSkiXSxbMiwwLCJcXHBpXyooTikiXSxbMSwwLCJcXHBpXyooXFxpb3RhJykiLDJdLFswLDIsIlxccGlfKihyKSJdLFswLDMsIlxccGlfKihyJykiLDJdLFszLDQsIlxccGlfKihcXGlvdGEnKSIsMl0sWzIsNCwiXFxpb3RhIl0sWzQsNSwiXFxwaV8qKHInKSJdLFszLDUsIiIsMSx7ImN1cnZlIjozLCJsZXZlbCI6Miwic3R5bGUiOnsiaGVhZCI6eyJuYW1lIjoibm9uZSJ9fX1dLFs2LDcsIlxcaW90YSIsMl0sWzcsMSwiXFxwaV8qKHInKSIsMl0sWzcsOCwiXFxwaV8qKHIpIl0sWzgsMCwiXFxpb3RhIl0sWzYsOCwiIiwwLHsiY3VydmUiOi0zLCJsZXZlbCI6Miwic3R5bGUiOnsiaGVhZCI6eyJuYW1lIjoibm9uZSJ9fX1dLFsxLDMsIiIsMCx7ImxldmVsIjoyLCJzdHlsZSI6eyJoZWFkIjp7Im5hbWUiOiJub25lIn19fV0sWzgsMiwiIiwwLHsibGV2ZWwiOjIsInN0eWxlIjp7ImhlYWQiOnsibmFtZSI6Im5vbmUifX19XSxbNywwLCJcXHBpXyooXFxlbGwpIl0sWzAsNCwiXFxwaV8qKFxcZWxsKSJdXQ==
	\[\begin{tikzcd}
		&& {\pi_*(N)} && {\pi_*(N)} \\
		{\pi_*(N)} & {\pi_*(M)} && {\pi_*(M)} && {\pi_*(M)} & {\pi_*(X)} \\
		&& {\pi_*(X)} && {\pi_*(X)}
		\arrow["{\pi_*(\iota')}"', from=3-3, to=2-4]
		\arrow["{\pi_*(r)}", from=2-4, to=1-5]
		\arrow["{\pi_*(r')}"', from=2-4, to=3-5]
		\arrow["{\pi_*(\iota')}"', from=3-5, to=2-6]
		\arrow["\iota", from=1-5, to=2-6]
		\arrow["{\pi_*(r')}", from=2-6, to=2-7]
		\arrow[curve={height=18pt}, Rightarrow, no head, from=3-5, to=2-7]
		\arrow["\iota"', from=2-1, to=2-2]
		\arrow["{\pi_*(r')}"', from=2-2, to=3-3]
		\arrow["{\pi_*(r)}", from=2-2, to=1-3]
		\arrow["\iota", from=1-3, to=2-4]
		\arrow[curve={height=-18pt}, Rightarrow, no head, from=2-1, to=1-3]
		\arrow[Rightarrow, no head, from=3-3, to=3-5]
		\arrow[Rightarrow, no head, from=1-3, to=1-5]
		\arrow["{\pi_*(\ell)}", from=2-2, to=2-4]
		\arrow["{\pi_*(\ell)}", from=2-4, to=2-6]
	\end{tikzcd}\]
	From this diagram we read off that the middle diagonal composition
	\[\pi_*(X)\xr{\pi_*(\iota')}\pi_*(M)\xr{\pi_*(r)}\pi_*(N)\]
	is an isomorphism with inverse $\pi_*(r')\circ\iota$. Now, since $X$ and $N$ are cellular, and $\pi_*(r\circ\iota')$ is an isomorphism, by \autoref{cellular_closed_under_iso} we have that $r\circ\iota'$ is an isomorphism, say with inverse $p$. Thus we have a commuting diagram
	% https://q.uiver.app/#q=WzAsNCxbMiwwLCJNIl0sWzQsMCwiTiJdLFswLDAsIk4iXSxbMSwxLCJYIl0sWzAsMSwiciIsMl0sWzIsMCwiXFxpb3RhJ1xcY2lyYyBwIiwyXSxbMiwzLCJwIiwyXSxbMywwLCJcXGlvdGEnIiwyXSxbMiwxLCIiLDAseyJjdXJ2ZSI6LTQsImxldmVsIjoyLCJzdHlsZSI6eyJoZWFkIjp7Im5hbWUiOiJub25lIn19fV1d
	\[\begin{tikzcd}
		N && M && N \\
		& X
		\arrow["r"', from=1-3, to=1-5]
		\arrow["{\iota'\circ p}"', from=1-1, to=1-3]
		\arrow["p"', from=1-1, to=2-2]
		\arrow["{\iota'}"', from=2-2, to=1-3]
		\arrow[curve={height=-24pt}, Rightarrow, no head, from=1-1, to=1-5]
	\end{tikzcd}\]
	and the middle row exhibits $N$ as a retract of $M=\bigoplus_i(E\otimes S^{|x_i|})$, as desired. It remains to show this is a retract in $E\text-\Mod$, i.e., that $r$ and $\iota'\circ p$ are homomorphisms of $E$-module objects. Above we constructed $r$ to be a homomorphism of $E$-modules. We also know that $X$ is an $E$-module object and that $\iota'$ is an $E$-module homomorphism. Thus, it remains to show that $p:N\to X$ is an $E$-module homomorphism. But we know that $p$ is the inverse of $r\circ\iota'$ in $\cSH$, and we know $r$ and $\iota'$ are morphisms in $E\text-\Mod$, so that $p$ is the inverse of $r\circ\iota'$ in $E\text-\Mod$, meaning $p$ is indeed an $E$-module homomorphism as desired.
\end{proof}

\end{document}
