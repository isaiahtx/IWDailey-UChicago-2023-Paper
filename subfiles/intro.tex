\documentclass[../main.tex]{subfiles}
\begin{document}

Homotopy theory is the study of spaces, where we consider two spaces to be ``the same'' if one can be stretched or deformed into the other. More specifically, given two continuous based maps of based spaces $f,g:X\to Y$, we say these maps are \emph{homotopic} if there exists a family of based maps $h_t:X\to Y$ indexed by $t\in[0,1]$ such that $h_0=f$, $h_1=g$, and the assignment
\[X\times I\to Y,\qquad (x,t)\mapsto h_t(x)\]
is continuous. Two spaces $X$ and $Y$ are \emph{homotopy equivalent} if there exists maps $f:X\to Y$ and $g:Y\to X$ such that $f\circ g$ is homotopic to $\id_Y$ and $g\circ f$ is homotopic to $\id_X$. Intuitively, two based spaces are homotopy equivalent if we can contiously squish and deform one into the other without moving the basepoint. Thus, one of the most important problems in algebraic topology is computing the homotopy classes of based maps $[X,Y]$ between two based spaces. Sadly, this is an extremely difficult problem, and in general there is no way to characterize these sets. Yet, there is some hope, if we focus our attention to nicer spaces. For one, topologists are mainly interested in \emph{CW complexes}, which roughly are spaces that can be built inductively by gluing spheres and disks of increasingly higher dimension together. Pretty much any ``space'' one might conjure in their mind is homeomorphic to a CW complex, e.g., spheres, cubes, tori, projective space, $\bR^n$, etc., and all the ways they can be glued together. Thus, the hope is that if we can understand the sets $\pi_n(S^m):=[S^n,S^m]$ of based homotopy classes of maps of spheres, then we can understand all the ways to build CW complexes and construct (nice) maps between them. But how would we even begin to understand these sets? 

Thankfully, the situation is not entirely hopeless: There are some facts we can state about them. First of all, for $n>0$, the sets $[S^n,X]$ become groups for any based space $X$, and for $n>1$ they are in fact abelian groups. There are also some computations that can be made. For example, for $n>0$, it is known that $[S^n,S^n]\cong\bZ$. Furthermore, if $n<m$, then $[S^n,S^m]=0$. We also have the following famous theorem of Freudenthal:

\begin{theorem}
    Given $k\geq0$, the homotopy groups $\pi_{n+k}(S^n)=[S^{n+k},S^n]$ are independent of $n$ for $n>k+1$.
\end{theorem}

In other words, in order to compute the groups $[S^n,S^m]$, if $n$ and $m$ are large enough, it suffices only to know the difference between $n$ and $m$. Thus, we may consider the \emph{stable homotopy groups of spheres} $\pi^S_k$, which, up to isomorphism, are the abelian groups $[S^{n+k},S^k]$ for $n>k+1$. This theorem is actually a corollary of a more general result. To state it, we need to set up some additional machinery: Given two based spaces $X$ and $Y$, their \emph{smash product} $X\wedge Y$ is the space $X\times Y/X\vee Y$. In particular, we denote the space $S^1\wedge X$ by $\Sigma X$, which we call the \emph{suspension} of $X$. If $X$ is a CW complex, then taking its suspension has the effect of shifting all the cells of $X$ up a dimension. In particular, for all $n\geq0$, there are homeomorphisms $\Sigma S^n\cong S^{n+1}$. Then the fact that $[S^n,X]$ is an abelian group when $n>1$ is a consequence of the more general fact that $[\Sigma^2X,Y]$ is an abelian group for all based spaces $X$ and $Y$. Then we can state the Freudenthal suspension theorem in its full generality:

\begin{theorem}
    For $X$ a based CW complex with no cells of dimension $\geq 2n$ and $Y$ a based CW complex with $[S^k,Y]=*$ for $k\leq n$, there are isomorphisms
    \[[X,Y]\xr\cong[\Sigma Y,\Sigma X]\xr\cong[\Sigma^2Y,\Sigma^2X],\]
    so that $[X,Y]$ is canonically an abelian group.
\end{theorem}

\todo{Outline: justify construction of stable homotopy category, talk about how goal of ASS is to compute $[X,Y]$ via $E$-homology of $X$ and $Y$. ``The majority of this paper is dedicated to indentifying as much structure as possible on the groups $E_*(X)$ for $E$ a multiplicative (co)homology theory, so that we can successively approximate $[X,Y]$ by means of sequences of maps from $E_*(X)$ to $E_*(Y)$, via the ASS''}

\subsection{Goals \& Outline}

This paper strives to achieve several goals, in particular, we aim to:
\begin{enumerate}
    \item Provide an axiomatic generalization of classical, motivic, and equivariant stable homotopy categories.
    \item Provide a reference for the full and explicit details of the construction of the classical, motivic, and equivariant $E$-Adams spectral sequences, the characterization of their $E_2$ pages, and some basic facts about their convergence.
\end{enumerate}
This project originally aimed to achieve only (2) above, specifically for the motivic Adams spectral sequence. Along the way, the idea for the generalization came to the author after reading the pair of papers \cite{Dugger_2014} and \cite{DDIO}, which discuss graded commutativity properties of symmetric monoidal categories which are in some sense ``graded'' by (a group homomorphism into) their Picard group of isomorphism classes of invertible objects.

We warn the reader that, as a result of its original goals, this document is still primarily expository in nature, and aims to be a mostly self-contained reference. Indeed, a large portion of the results contained here constitute only slight generalizations of results already found elsewhere in the literature. Nevertheless, we believe the approach outlined here is valuable even beyond serving as a self-contained reference, as we do make several original innovations:
\begin{enumerate}
    \item We provide a general construction of the Adams spectral sequence which equally applies to the classical, motivic, and equivariant stable homotopy categories. This is quite flexible, for example, in the $G$-equivariant case: we can construct a version of the spectral sequence which intrisically keeps track of the $RO(G)$ grading, or, alternatively, could be constructed to be graded by the entirety of the Picard group of the equivariant stable homotopy category.
    \item We develop the notion of a ``tensor-triangulated category with sub-Picard grading,'' which roughly is a category which is graded by some abelian group, symmetric monoidal, and triangulated, all in a compatible way. Along with a few extra categorical conditions, such categories provide a surprisingly powerful axiomatization of the (classical, motivic, equivariant) stable homotopy category, and a shockingly large amount of the theory therewithin can be carried out entirely in this framework.
    \item We provide an encompassing notion of ``cellularity'' in a tensor triangulated category with sub-Picard grading, which parallels the same notion in the motivic stable homotopy category.
    \item We work out some of the graded-commutativity properties of $\pi_*(E)$ for a commutative monoid object $(E,\mu,e)$ in a tensor triangulated category with sub-Picard grading. In particular, we provide a complete picture of the preliminary analysis given in \cite[Remark 7.2]{Dugger_2014}.
    \item We develop much of the theory of $A$-graded abelian groups, rings, modules, submodules, quotient modules, tensor products, homomorpisms, etc.
    \item We suggest a definition for the correct notion of an ``anticommutative $A$-graded ring'' for a general abelian group $A$. In particular, we suggest a new candidate for the category in which the motivic Steenrod algebra is a Hopf algebroid/co-groupoid object.
\end{enumerate}

We give an outline of the structure of the paper. In \Cref{section:prelims}, we will develop the notion of tensor triangulated categories with sub-Picard grading, which will be defined in \autoref{sub_Picard_grading_defn}. We will then fix such a category $\cSH$ (with a few extra categorical conditions), which acts as an axiomatic model for the classical, motivic, and equivariant stable homotopy categories. In this category, we will be able to develop much of the theory of stable homotopy theory, in particular, we will be able to formulate the notion of $A$-graded stable homotopy groups $\pi_*(X)$ of objects $X$ in $\cSH$, as well as homology, and cohomology represented by objects in this category. We will show that (co)fiber sequences (i.e., distinguished triangles) in $\cSH$ give rise to long exact sequences of homotopy groups, and that $\cSH$ is equipped with an $A$-indexed family of suspension and loop autoequivalences.

After just this first section, we will actually have all the data needed to construct the Adams spectral sequence, yet we will not actually do so until the very end in \Cref{section:ASS}. The goal of this spectral sequence will be to compute the $A$-graded abelian groups of stable homotopy classes of maps ${[X,Y]}_*$ between objects $X$ and $Y$ in $\cSH$, by means of algebraic information about the $E$-homology of $X$ and $Y$. Yet, looking at just the definition of the $E$-Adams spectral sequence, it will not be immediately clear how exactly it achieves this goal in any sense. Thus, before constructing the sequence, Sections 2--6 will be devoted to formulating suitable conditions on $E$, $X$, and $Y$ under which enough structure may be captured on the $E$-homology groups $E_*(X)$ and $E_*(Y)$ that algebraic information about homomorphisms between them gives suitable information about the groups ${[X,Y]}_*$.

In \Cref{section:cellular}, we will formulate the notion of \emph{cellular} objects in $\cSH$. Intuitively, these are the objects in $\cSH$ which may be constructed by gluing together copies of spheres. In the case $\cSH$ is the motivic stable homotopy category, these objects will correspond to the standard notion of cellular motivic spaces. In the case $\cSH$ is the classical stable homotopy category, every object will turn out to be cellular, as a consequence of the fact that every space is weakly equivalent to a generalized cell complex. The class of cellular objects in $\cSH$ will satisfy many very important properties, for example, given cellular objects $X$ and $Y$ in $\cSH$, a map $f:X\to Y$ will be an isomorphism if and only if it induces an isomorphism on stable homotopy groups $\pi_*(f):\pi_*(X)\to\pi_*(Y)$. Many of the important theorems and propositions presented in this paper will require some sort of cellularity condition.

In \Cref{section:monoid_in_SH}, we will discuss the theory of monoid objects in $\cSH$, which correspond to ring spectra in stable homotopy theory. We will show that given a monoid object $E$ in $\cSH$, its stable homotopy groups $\pi_*(E)$ naturally form an $A$-graded ring, and furthermore, $E$-homology $E_*(-)$ will yield a functor from $\cSH$ to the category of $A$-graded left modules over $\pi_*(E)$. Here a great deal of effort will be put into formulating the exact sense in which the rings $\pi_*(E)$ are \emph{$A$-graded anticommutative} when $E$ is a commutative monoid object in $\cSH$. In particular, here we will develop the notion of \emph{$A$-graded anticommutative rings}, and we will show that $\pi_*(E)$ is an $A$-graded anticommutative algebra over the $A$-graded anticommutative stable homotopy ring $\pi_*(S)$ (where $S$ is the monoidal unit in $\cSH$), in a suitable sense.

In \Cref{section:important}, we will prove analogues of important theorems for homology in $\cSH$. First of all, we will prove that for $E$ a commutative monoid object and objects $X$ and $Y$ in $\cSH$, under suitable conditions we have a \emph{K\"unneth isomorphism}
\[Z_*(E)\otimes_{\pi_*(E)}E_*(W)\to\pi_*(Z\otimes E\otimes W)\]
relating the $Z$-homology of $E$ and the $E$-homology of $W$ to the stable homotopy groups of $Z\otimes E\otimes W$. We will then take a bit to develop the theory of module objects over monoid objects in $\cSH$, with which we will prove a generalization of the universal coefficient theorem, which will tell us that under suitable conditions, for a monoid object $E$ in $\cSH$ and an object $X$, the cohomology $E^*(X)$ of $X$ is the dual of the homology $E_*(X)$ as a $\pi_*(E)$-module. These two theorems will be very important for our later work.

In \Cref{section:dual_E-Steenrod_algebra}, we will show that for nice enough commutative monoid objects $E$ in $\cSH$, that the $E$-self homology $E_*(E)$, along with the ring $\pi_*(E)$, forms an \emph{$A$-graded anticommutative Hopf algebroid}, which we define to be a co-groupoid object in the category $\pi_*(S)\text-\GCA^A$ of $A$-graded anticommutative $\pi_*(S)$-algebras. This pair $(E_*(E),\pi_*(E))$ with its additional structure as a Hopf algebroid is called the \emph{dual $E$-Steenrod algebra}, over which the $A$-graded $E$-homology group $E_*(X)$ of $X$ is canonically an $A$-graded left comodule for each $X$ in $\cSH$. This will be the culmination of our efforts to place additional structure on the $E$-homology groups $E_*(X)$, and in the ensuing section we will show that this algebraic structure can be used to recover a surprising amount of information about hom-groups in $\cSH$.

In \Cref{section:ASS}, we will finally construct the $\bZ\times A$-graded spectral sequence $(E_r^{s,a}(X,Y),d_r)$ called the \emph{$E$-Adams spectral sequence for the computation of $X$ and $Y$}, and we will show that under suitable conditions, its $E_2$ page may be characterized in terms of a graded isomorphism
\[E_2^{*,*}(X,Y)\cong\Ext^{*,*}(E_{*}(X),E_*(Y)).\]
Furthermore, we will briefly discuss that the natural target group of this spectral sequence is the object ${[X,Y_E^\wedge]}_*$, where $Y_E^\wedge$ is the ``$E$-nilpotent completion'' of $Y$. Furthermore,  Thus, we will have developed a tool to compute information about the groups ${[X,Y]}_*$ from data about the $E$-homology groups of $X$ and $Y$.

Finally, in Sections \ref{section:CASS} and \ref{section:MASS}, we will provide a brief review of how our results apply to the classical and motivic stable homotopy categories, respectively.

\end{document}
