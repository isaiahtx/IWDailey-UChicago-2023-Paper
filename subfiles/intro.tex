\documentclass[../main.tex]{subfiles}
\begin{document}

Homotopy theory is the study of spaces, where we consider two spaces to be ``the same'' if one can be stretched or deformed into the other. More specifically, given two continuous based maps of based spaces $f,g:X\to Y$, we say these maps are \emph{homotopic} if there exists a family of based maps $h_t:X\to Y$ indexed by $t\in[0,1]$ such that $h_0=f$, $h_1=g$, and the assignment
\[X\times I\to Y,\qquad (x,t)\mapsto h_t(x)\]
is continuous. Two spaces $X$ and $Y$ are \emph{homotopy equivalent} if there exists maps $f:X\to Y$ and $g:Y\to X$ such that $f\circ g$ is homotopic to $\id_Y$ and $g\circ f$ is homotopic to $\id_X$. Intuitively, two based spaces are homotopy equivalent if we can contiously squish and deform one into the other without moving the basepoint. Thus, one of the most important problems in algebraic topology is computing the homotopy classes of based maps $[X,Y]$ between two based spaces. Sadly, this is an extremely difficult problem, and in general there is no way to characterize these sets. Yet, there is some hope, if we focus our attention to nicer spaces. For one, topologists are mainly interested in \emph{CW complexes}, which roughly are spaces that can be built inductively by gluing spheres and disks of increasingly higher dimension together. Thus, the hope is that if we can understand the sets $\pi_n(S^m):=[S^n,S^m]$ of based homotopy classes of maps of spheres, then we can understand all the ways to build CW complexes and construct (nice) maps between them. But how would we even begin to understand these sets? First of all, for $n>0$, these sets become groups, and for $n>1$ they are in fact abelian groups. There are also some computations that can be made. For $n>0$, it is known that $[S^n,S^n]\cong\bZ$. Furthermore, if $n<m$, then $[S^n,S^m]=0$. We also have the following famous theorem of Freudenthal:

\begin{theorem}
    Given $k\geq0$, the homotopy groups $\pi_{n+k}(S^n)=[S^{n+k},S^n]$ are independent of $n$ for $n>k+1$.
\end{theorem}

In other words, in order to compute the groups $[S^n,S^m]$, if $n$ and $m$ are large enough, it suffices only to know the difference between $n$ and $m$. Thus, we may consider the \emph{stable homotopy groups of spheres} $\pi^S_k$, which, up to isomorphism, are the abelian groups $[S^{n+k},S^k]$ for $n>k+1$. This theorem is actually a corollary of a more general result. Given a based space $X$, we may take its \emph{suspension} $\Sigma X$. If $X$ is a CW complex, taking suspension has the effect of shifting all the cells of $X$ up a dimension. In particular, for all $n\geq0$, there are homeomorphisms $\Sigma S^n\cong S^{n+1}$.

\begin{theorem}
    For $X$ a based CW complex with no cells of dimension $\geq 2n$ and $Y$ a based CW complex with $[S^k,Y]=*$ for $k\leq n$, there are isomorphisms
    \[[X,Y]\xr\cong[\Sigma Y,\Sigma X]\xr\cong[\Sigma^2Y,\Sigma^2X],\]
    and $[X,Y]$ is canonically an abelian group.
\end{theorem}

\todo{Outline: justify construction of stable homotopy category, talk about how goal of ASS is to compute $[X,Y]$ via $E$-homology of $X$ and $Y$. ``The majority of this paper is dedicated to indentifying as much structure as possible on the groups $E_*(X)$ for $E$ a multiplicative (co)homology theory, so that we can successively approximate $[X,Y]$ by means of sequences of maps from $E_*(X)$ to $E_*(Y)$, via the ASS''}

This paper strives to achieve several goals, in particular, we aim to:
\begin{enumerate}
    \item Provide an axiomatic generalization of the Adams spectral sequence which applies equally to the classical, motivic, and equivariant stable homotopy categories.
    \item Provide a reference for the full and explicit details of the construction of the classical, motivic, and equivariant $E$-Adams spectral sequences, the characterization of their $E_2$ pages, and the proofs of their convergence.
\end{enumerate}
This project originally aimed to achieve the second goal, specifically for the motivic Adams spectral sequence. Along the way, the idea for the generalization came to the author after reading the pair of papers \cite{Dugger_2014} and \cite{DDIO}. We provide several innovations:
\begin{enumerate}
    \item We provide a general construction of the Adams spectral sequence which equally applies to the classical, motivic, and equivariant stable homotopy categories. In particular, in the equivariant case, we construct a spectral sequence which intrisically keeps track of the $RO(G)$ grading, or, if one likes, could be constructed to be graded by the entirety of the Picard group of the equivariant stable homotopy category.
    \item We develop the notion of a ``tensor-triangulated category with sub-Picard grading,'' which roughly is a category which is graded by some abelian group, symmetric monoidal, and triangulated, all in a compatible way. This provides a surprisingly powerful axiomatization of the (classical, motivic, equivariant) stable homotopy category, and a shockingly large amount of the theory therewithin can be carried out entirely in this framework.
    \item We provide an encompassing notion of ``cellularity'' in a tensor triangulated category with sub-Picard grading, which parallels the same notion in the motivic stable homotopy category.
    \item We work out some of the graded-commutativity properties of $\pi_*(E)$ for a commutative monoid object $(E,\mu,e)$ in a tensor triangulated category with sub-Picard grading.
    \item We develop much of the theory of $A$-graded abelian groups, rings, modules, submodules, quotient modules, tensor products, homomorpisms, etc.
    \item We suggest a definition for the correct notion of an ``anticommutative $A$-graded ring'' for a general abelian group $A$. In particular, we suggest a new candidate for the category in which the motivic Steenrod algebra is a Hopf algebroid.
    \item\todo{add more}
\end{enumerate}

\end{document}
