\documentclass[../main.tex]{subfiles}
\begin{document}

Arguably the most famous problem in modern homotopy theory is computing the stable homotopy groups of spheres. This is an extremely difficult problem. State of the art computations by Isaksen, Wang, and Xu (\cite{isaksen2023stable}) have only computed these groups up to dimension $90$. The key tool used for these computations is the \emph{Adams spectral sequence}, of which there are many different flavors. Given a flat, cellular commutative ring spectrum $E$, the $E$-Adams spectral spectral sequence is $\bZ^2$-graded and has signature
\[E_2^{s,t}(X,Y)=\Ext_{E_*(E)}^{s}(E_*(X),E_{*+t}(Y))\Longrightarrow {[X,Y_E^\wedge]}_*.\]
Here $\Ext$ is taken in the category of $\bZ$-graded comodules over the dual $E$-Steenrod algebra. The spectral sequence was originally constructed in 1958 by Frank Adams in the case $E=H\bF_p$ and $X=Y=S$ (\cite{Adams1958}). While a great deal may be ascertained about the stable homotopy groups of spheres with just this spectral sequence, more is needed in order to compute beyond the first twenty or so stable stems.

In 1998, Voevodsky introduced concepts from homotopy theory into algebraic geometry, creating a new field called \emph{motivic homotopy theory} (\cite{voevodsky1998a1}), also called $\bA^1$-homotopy theory. Rather than working with topological spaces, in motivic homotopy theory, the fundamental objects are varieties over some base scheme $\scS$. The theory goes quite far, and one may construct a symmetric monoidal stable model category of motivic spectra over $\scS$, whose homotopy category is the motivic stable homotopy category over $\scS$.\footnote{For a review of these constructions, we refer the reader to Section 2 of \cite{WilOst}.} Later in 2009, Dugger and Isaksen constructed a motivic version of the Adams spectral sequence in the motivic stable homotopy category (\cite{DugIsak}). Given a flat, cellular motivic commutative ring spectrum $E$, the motivic $E$-Adams spectral spectral sequence is $\bZ^3$-graded and has signature
\[E_2^{s,t,u}(X,Y)=\Ext^s_{E_{*,*}(E)}(E_{*,*}(X),E_{*+t+s,*+u}(Y))\Longrightarrow{[X,Y^\wedge_E]}_{*,*}\]
(where here $\Ext$ is taken in the category of $\bZ^2$-graded left comodules over the motivic dual $E$-Steenrod algebra). Dugger and Isaksen construct the spectral sequence only for the case $E=H\bF_2$ and $X=S$. Furthermore, they leave out many details of the construction, leaving them for the reader to fill in. The work we present here was originally conceived with the aim of filling in these details and constructing the spectral sequence in the more general form given above. More general results were achieved upon pursing this aim, and the scope of our results have changed significantly.

\subsection{Goals \& Outline}

The two main goals of this paper are as follows:
\begin{enumerate}
    \item Provide an axiomatic generalization of classical, motivic, and equivariant stable homotopy categories.
    \item Provide a reference for the full and explicit details of the construction of the classical, motivic, and equivariant $E$-Adams spectral sequences, the characterization of their $E_2$ pages, and some basic facts about their convergence.
\end{enumerate}
The idea for the generalization came to the author after reading the pair of papers \cite{Dugger_2014} and \cite{DDIO}, which roughly discuss graded commutativity properties of what one might call ``sub-Picard graded symmetric monoidal categories''.

We warn the reader that, as a result of goal (2), this document is primarily expository in nature. Furthermore, it aims to be a mostly self-contained reference, which accounts for its significant length. Indeed, a large portion of the results contained here constitute only slight generalizations of results already found elsewhere in the literature. Nevertheless, we believe the approach outlined here is valuable even beyond serving as a self-contained reference, as we do make several original innovations:
\begin{enumerate}
    \item We provide a general construction of the Adams spectral sequence which equally applies to the classical, motivic, and equivariant stable homotopy categories. This is quite flexible, for example, in the $G$-equivariant case: we can construct a version of the spectral sequence which intrisically keeps track of the $RO(G)$ grading, or, alternatively, could be constructed to be graded by the entirety of the Picard group of the equivariant stable homotopy category. In particular, we give a more general version of the motivic Adams spectral sequence than that found in the literature.
    \item We develop the notion of a ``tensor-triangulated category with sub-Picard grading,'' which roughly is a category which is graded by some abelian group, symmetric monoidal, and triangulated, all in a compatible way. Along with a few extra categorical conditions, such categories provide a surprisingly powerful axiomatization of the (classical, motivic, equivariant) stable homotopy category, and a shockingly large amount of the theory therewithin can be carried out entirely in this framework.
    \item We provide an encompassing notion of ``cellularity'' in a tensor triangulated category with sub-Picard grading, which parallels the same notion in the motivic stable homotopy category.
    \item We work out some of the graded-commutativity properties of $\pi_*(E)$ for a commutative monoid object $(E,\mu,e)$ in a tensor triangulated category with sub-Picard grading. In particular, we provide a complete picture of the preliminary analysis given in \cite[Remark 7.2]{Dugger_2014}.
    \item We suggest a definition for the correct notion of an ``anticommutative $A$-graded ring'' for a general abelian group $A$. In particular, we suggest a new candidate for the category in which the motivic Steenrod algebra is a Hopf algebroid/co-groupoid object.
\end{enumerate}

This paper should be viewed as a natural successor to the nLab page on the Adams spectral sequence (\cite{nlab:introduction_to_the_adams_spectral_sequence}) written by Urs Schreiber.  Indeed, this paper tells mostly the same story told there, albeit in a more general setting. Along the way, we fill in many of the details not contained there. Furthermore, we are of the opinion that the more general and categorical approach can serve to clarify and even ``trivialize'' many of the proofs and ideas involved here. It is the hope of the author that this document can serve as an equally valuable resource for those first learning classical, motivic, and equivariant stable homotopy theory.

We give an outline of the structure of the paper. In \Cref{section:prelims}, we start by giving the required background for the paper, and we give a brief review of coherence for symmetric monoidal categories. We then develop the notion of tensor triangulated categories with sub-Picard grading, which will be defined in \autoref{sub_Picard_grading_defn}. We will discuss Dugger's paper \cite{Dugger_2014} which concern ``sub-Picard graded symmetric monoidal categories'', and we will apply some of the results from therewithin to our situation. Then we will fix such a category $\cSH$ (with a few extra categorical conditions), which acts as an axiomatic model for the classical, motivic, and equivariant stable homotopy categories. In this category, we will be able to develop much of the theory of stable homotopy theory, in particular, we will be able to formulate the notion of $A$-graded stable homotopy groups $\pi_*(X)$ of objects $X$ in $\cSH$, as well as homology, and cohomology represented by objects in this category. We will show that (co)fiber sequences (i.e., distinguished triangles) in $\cSH$ give rise to long exact sequences of homotopy groups, and that $\cSH$ is equipped with an $A$-indexed family of ``suspension'' and ``loop'' autoequivalences.

After just this first section, we will actually have all the data needed to construct the Adams spectral sequence, yet we will not actually do so until the very end in \Cref{section:ASS}. The goal of this spectral sequence will be to compute the $A$-graded abelian groups of stable homotopy classes of maps ${[X,Y]}_*$ between objects $X$ and $Y$ in $\cSH$, by means of algebraic information about the $E$-homology of $X$ and $Y$. This is useful because in practice, for a suitable homology theory $E$, it is often  easier to compute $E$-homology than it is to compute general hom-groups. To achieve this goal, Sections 2--6 will be devoted to formulating suitable conditions on $E$, $X$, and $Y$ under which enough structure may be captured on the $E$-homology groups $E_*(X)$ and $E_*(Y)$ that algebraic information about homomorphisms between them gives suitable information about the groups ${[X,Y]}_*$.

In \Cref{section:cellular}, we will formulate the notion of \emph{cellular} objects in $\cSH$. Intuitively, these are the objects in $\cSH$ which may be constructed by gluing together copies of spheres. In the case $\cSH$ is the motivic stable homotopy category, these objects will correspond to the standard notion of cellular motivic spaces. In the case $\cSH$ is the classical stable homotopy category, every object will turn out to be cellular, as a consequence of the fact that every space is weakly equivalent to a generalized cell complex. The class of cellular objects in $\cSH$ will satisfy many very important properties, for example, given cellular objects $X$ and $Y$ in $\cSH$, a map $f:X\to Y$ will be an isomorphism if and only if it induces an isomorphism on stable homotopy groups $\pi_*(f):\pi_*(X)\to\pi_*(Y)$. Many of the important theorems and propositions presented in this paper will require some sort of cellularity condition.

In \Cref{section:monoid_in_SH}, we will discuss the theory of monoid objects in $\cSH$, which correspond to ring spectra in stable homotopy theory. We will show that given a monoid object $E$ in $\cSH$, its stable homotopy groups $\pi_*(E)$ naturally form an $A$-graded ring, and furthermore, $E$-homology $E_*(-)$ will yield a functor from $\cSH$ to the category of $A$-graded left modules over $\pi_*(E)$. Here a great deal of effort will be put into formulating the exact sense in which the rings $\pi_*(E)$ are \emph{$A$-graded anticommutative} when $E$ is a commutative monoid object in $\cSH$. In particular, here we will develop the notion of \emph{$A$-graded anticommutative rings}, and we will show that $\pi_*(E)$ is an $A$-graded anticommutative algebra over the $A$-graded anticommutative stable homotopy ring $\pi_*(S)$ (where $S$ is the monoidal unit in $\cSH$), in a suitable sense. We will also briefly discuss some of the consequences of these results in the classical, motivic, and equivariant stable homotopy categories.

In \Cref{section:important}, we will prove analogues of important theorems for homology in $\cSH$. First of all, we will prove that for $E$ a commutative monoid object and objects $X$ and $Y$ in $\cSH$, under suitable conditions we have a \emph{K\"unneth isomorphism}
\[Z_*(E)\otimes_{\pi_*(E)}E_*(W)\to\pi_*(Z\otimes E\otimes W)\]
relating the $Z$-homology of $E$ and the $E$-homology of $W$ to the stable homotopy groups of $Z\otimes E\otimes W$. We will then take a bit to develop the theory of module objects over monoid objects in $\cSH$, with which we will prove a generalization of the universal coefficient theorem, which will tell us that under suitable conditions, for a monoid object $E$ in $\cSH$ and an object $X$, the cohomology $E^*(X)$ of $X$ is the dual of the homology $E_*(X)$ as a $\pi_*(E)$-module. These two theorems will be very important for our later work.

In \Cref{section:dual_E-Steenrod_algebra}, we will show that for nice enough commutative monoid objects $E$ in $\cSH$, that the $E$-self homology $E_*(E)$, along with the ring $\pi_*(E)$, forms an \emph{$A$-graded anticommutative Hopf algebroid}, which we define to be a co-groupoid object in the category $\pi_*(S)\text-\GCA^A$ of $A$-graded anticommutative $\pi_*(S)$-algebras. This pair $(E_*(E),\pi_*(E))$ with its additional structure as a Hopf algebroid is called the \emph{dual $E$-Steenrod algebra}, over which the $A$-graded $E$-homology group $E_*(X)$ of $X$ is canonically an $A$-graded left comodule for each $X$ in $\cSH$. This will be the culmination of our efforts to place additional structure on the $E$-homology groups $E_*(X)$, and we will finish the section by constructing an isomorphism
\[{[X,E\otimes Y]}_*\cong\Hom^*_{E_*(E)}(E_*(X),E_*(E\otimes Y))\]
for a suitable commutative monoid object $E$ and objects $X$ and $Y$ in $\cSH$.

In \Cref{section:ASS}, we will finally construct the $\bZ\times A$-graded spectral sequence $(E_r^{s,a}(X,Y),d_r)$ called the \emph{$E$-Adams spectral sequence for the computation of $X$ and $Y$}, and we will show that under suitable conditions, its $E_2$ page may be characterized in terms of a graded isomorphism
\[E_2^{*,*}(X,Y)\cong\Ext_{E_*(E)}^{*,*}(E_{*}(X),E_*(Y)).\]
Furthermore, we will briefly discuss that the natural target group of this spectral sequence is the object ${[X,Y_E^\wedge]}_*$, where $Y_E^\wedge$ is the ``$E$-nilpotent completion'' of $Y$. Furthermore, we will briefly discuss some conditions under which the spectral sequence strongly converges to this target group. We can summarize all of the results in the following theorem:
\begin{theorem}
    Let $(E,\mu,e)$ be a commutative monoid object in $\cSH$, and let $X$ and $Y$ be objects. Further suppose that:\begin{itemize}
        \item $E$ is cellular and flat,
        \item $X$ is cellular and $E_*(X)$ is a graded projective left $\pi_*(E)$-module, and
        \item $Y$ is cellular.
    \end{itemize}
    Then there exists an object $Y^\wedge_E$ in $\cSH$ called the \emph{$E$-nilpotent completion of $Y$} and a $\bZ\times A$-graded spectral sequence called the \emph{$E$-Adams spectral sequence for the computation of ${[X,Y]}_*$} with signature
    \[E_2^{s,a}(X,Y)=\Ext^{s,a+\mbf s}_{E_*(E)}(E_*(X),E_*(Y))\Longrightarrow{[X,Y^\wedge_E]}_*.\]
    Furthermore, if the derived $E_\infty$ term $RE_\infty$ of the sequence vanishes, then this spectral sequence converges strongly to the indicated target group.
\end{theorem}
We will give all of the relevant definitions along the way.

Finally, in \Cref{section:future}, we will suggest some further directions in which one can develop the theory we have set up.

We also include five appendices on the theory of (tensor) triangulated categories, $A$-graded abelian groups, rings, and modules, monoid objects in symmetric monoidal categories, homological (co)algebra and derived functors, and Hopf algebroids.

\subsection{Acknowledgements}

I am grateful to Peter May and Noah Ankey for mentoring me during this project. I also would like to thank Peter May for hosting the UChicago REU --- the experience was truly invaluable. Finally, I am indebted to my peers for the continual support and encouragement they have provided during the program and the months following.

\end{document}
